\documentclass{book}

\title{Mathematics}
\author{Robin Adams}

\usepackage{amsmath}
\usepackage{amssymb}
\usepackage{amsthm}
\let\proof\relax
\let\endproof\relax
\let\qed\relax
\usepackage{pf2}
\usepackage{hyperref}
\usepackage{mathabx}
\usepackage[all]{xy}

\newtheorem{ax}{Axiom}[chapter]
\newtheorem{axs}[ax]{Axiom Schema}
\newtheorem{prop}[ax]{Proposition}
\newtheorem{cor}{Corollary}[ax]
\newtheorem{thm}[ax]{Theorem}
\newtheorem{lm}[ax]{Lemma}
\theoremstyle{definition}
\newtheorem{df}[ax]{Definition}
\newtheorem{ex}[ax]{Example}

\begin{document}

\maketitle
\tableofcontents

\chapter{Primitive Terms and Axioms}

Let there be \emph{sets}. We write $A : Set$ for: $A$ is a set.

For any set $A$, let there be \emph{elements} of $A$. We write $a : El(A)$ for: $a$ is an element of $A$.

For any set $A$, let there be \emph{subsets} of $A$. We write $S : Sub(A)$ for: $S$ is a subset of $A$.

Given a set $A : Set$, an element $a : El(A)$ and a subset $S : Sub(A)$, let there be a proposition $a \in S$ or $S \ni a$. When this holds, we say $a$ is a \emph{member} or \emph{element} of $S$. We write $a \notin S$ when it does not hold.

For any sets $A, B : Set$, let there be a set $A \times B : Set$, the \emph{Cartesian product} of $A$ and $B$.


\end{document}