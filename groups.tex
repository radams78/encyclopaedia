\chapter{Semigroups}

\begin{df}[Semigroup]
A \emph{semigroup} consists of a set $S$ and an associative binary operation $\cdot$ on $S$.
\end{df}

\chapter{Monoids}

\begin{df}[Monoid]
A \emph{monoid} consists of a semigroup $M$ such that there exists $e \in M$, the \emph{unit}, such that, for all $x \in M$, we have $x e = ex = x$.

We identify a monoid $M$ with the category with one object whose morphisms are the elements of $M$, with composition given by $\cdot$.
\end{df}

\begin{prop}
    The identity in a group is unique.
\end{prop}

\begin{proof}
    \pf\ Proposition \ref{prop:id-morph-unique}.
\end{proof}

\chapter{Groups}

\begin{df}[Group]
Let $\mathcal{C}$ be a category with finite products. A \emph{group (object)} in $\mathcal{C}$ consists of an object $G \in \mathcal{C}$ and morphisms
\[ m : G^2 \rightarrow G, e : 1 \rightarrow G, i : G \rightarrow G \]
such that the following diagrams commute.
\[ \begin{tikzcd}
G^3 \arrow[r,"m \times \id{G}"] \arrow[d,"\id{G} \times m"] & G^2 \arrow[d,"m"] \\
G^2 \arrow[r,"m"] & G
\end{tikzcd} \]
\[ \begin{tikzcd}
1 \times G \arrow[r,"e \times \id{G}"] \arrow[dr,"\cong"] & G^2 \arrow[d,"m"] \\
& G
\end{tikzcd}
\qquad
\begin{tikzcd}
G \times 1 \arrow[r,"\id{G} \times e"] \arrow[dr,"\cong"] & G^2 \arrow[d,"m"] \\
& G
\end{tikzcd} \]
\[ \begin{tikzcd}
G \arrow[r,"\Delta"] \arrow[d] & G^2 \arrow[r,"\id{G} \times i"] & G^2 \arrow[d,"m"] \\
1 \arrow[rr,"e"] & & G
\end{tikzcd}
\qquad
\begin{tikzcd}
G \arrow[r,"\Delta"] \arrow[d] & G^2 \arrow[r,"i \times \id{G}"] & G^2 \arrow[d,"m"] \\
1 \arrow[rr,"e"] & & G
\end{tikzcd} \]
\end{df}

\begin{df}[Group]
We write just 'group' for 'group in $\Set$. Thus, a \emph{group} $G$ consists of a set $G$ and a binary operation $\cdot : G^2 \rightarrow G$ such that $\cdot$ is associative, and there exists $e \in G$, the \emph{identity} element of the group, such that:
    \begin{itemize}
        \item For all $x \in G$ we have $xe = ex = x$
        \item For all $x \in G$, there exists $\inv{x} \in G$, the \emph{inverse} of $x$,
              such that $x \inv{x} = \inv{x} x = e$.
    \end{itemize}

    The \emph{order} of a group $G$, denoted $|G|$, is the number of elements in
    $G$ if $G$ is finite; otherwise we write $|G| = \infty$.
\end{df}

\begin{prop}
    \label{prop:grp-inv-unique}
    The inverse of an element is unique.
\end{prop}

\begin{proof}
    \pf\ If $i$ and $j$ are inverses of $x$ then $i = ixj = j$. \qed
\end{proof}

\begin{ex}
    \begin{itemize}
        \item
              The \emph{trivial} group is $\{e\}$ under $ee = e$.
        \item $\mathbb{Z}$ is a group under addition %TODO Define this
        \item $\mathbb{Q}$ is a group under addition %TODO Define this
        \item $\mathbb{Q} - \{0\}$ is a group under multiplication
        \item $\mathbb{R}$ is a group under addition %TODO Define this
        \item $\mathbb{R} - \{0\}$ is a group under multiplication
        \item $\mathbb{C}$ is a group under addition %TODO Define this
        \item $\mathbb{C} - \{0\}$ is a group under multiplication
        \item $\{-1,1\}$ is a group under multiplication
        \item For any category $\mathcal{C}$ and object $A \in \mathcal{C}$, we have
              $\Aut{\mathcal{C}}{A}$ is a group under $gf = f \circ g$.

              For $A$ a set, we call $S_A = \Aut{\Set}{A}$ the \emph{symmetric group} or
              \emph{group of permutations} of $A$.

        \item For $n \geq 3$, the \emph{dihedral group} $D_{2n}$ consists of the set of rigid
              motions that map the regular $n$-gon onto itself under composition.
              \item Let $SL_2(\mathbb{Z}) = \left\{ \left( \begin{array}{cc}
              a & b \\
              c & d \end{array} \right)
              : a,b,c,d \in \mathbb{Z}, ad - bc = 1 \right\}$ under matrix multiplication.
\item The quaternionic group $Q_8$ is the group
\[ \{ 1, -1, i, -i, j, -j, k, -k \} \]
with multiplication table
\[ \begin{array}{cccccccc}
1 & -1 & i & -i & j & -j & k & -k \\
-1 & 1 & -i & i & -j & j & -k & k \\
i & -i & -1 & 1 & k & -k & -j & j \\
-i & i & 1 & -1 & -k & k & j & -j \\
j & -j & -k & k & -1 & 1 & i & -i \\
-j & j & k & -k & 1 & -1 & -i & i \\
k & -k & j & -j & -i & i & -1 & 1 \\
-k & k & -j & j & i & -i & 1 & -1
\end{array} \]
    \end{itemize}
\end{ex}

\begin{ex}
    \begin{itemize}
        \item The only group of order 1 is the trivial group.
        \item The only group of order 2 is $\mathbb{Z}_2$.
        \item The only group of order 3 is $\mathbb{Z}_3$.
        \item There are exactly two groups of order 4: $\mathbb{Z}_4$ and $\mathbb{Z}_2
                  \times \mathbb{Z}_2$ under $(a,b) (c,d) = (ac,bd)$.
    \end{itemize}
\end{ex}

\begin{prop}[Cancellation]
    Let $G$ be a group. Let $a,g,h \in G$. If $ag = ah$ or $ga = ha$ then $g = h$.
\end{prop}

\begin{proof}
    \pf\ If $ag = ah$ then $g = \inv{a} a g = \inv{a} a h = h$. Similarly if $ga = ha$. \qed
\end{proof}

\begin{prop}
    Let $G$ be a group and $g,h \in G$. Then $(gh)^{-1} = h^{-1}g^{-1}$.
\end{prop}

\begin{proof}
    \pf\ Since $ghh^{-1}g^{-1} = e$. \qed
\end{proof}

\begin{df}
\label{df:power-in-group}
    Let $G$ be a group. Let $g \in G$. We define $g^n \in G$ for all $n \in \mathbb{Z}$ as follows:
    \begin{align*}
        g^0     & = e                       \\
        g^{n+1} & = g^n g      & (n \geq 0) \\
        g^{-n}  & = (g^{-1})^n & (n > 0)
    \end{align*}
\end{df}

\begin{prop}
    \label{prop:power-add}
    Let $G$ be a group. Let $g \in G$ and $m,n \in \mathbb{Z}$. Then
    \[ g^{m+n} = g^m g^n \enspace . \]
\end{prop}

\begin{proof}
    \pf
    \step{1}{For all $k \in \mathbb{Z}$ we have $g^{k+1} = g^k g$}
    \begin{proof}
        \step{a}{For all $k \geq 0$ we have $g^{k+1} = g^k g$}
        \begin{proof}
            \pf\ Immediate from definition.
        \end{proof}
        \step{b}{$g^{-1+1} = g^{-1} g$}
        \begin{proof}
            \pf\ Both are equal to $e$.
        \end{proof}
        \step{c}{For all $k > 1$ we have $g^{-k+1} = g^{-k} g$}
        \begin{proof}
            \pf
            \begin{align*}
                g^{-k+1} & = (g^{-1})^{k-1}          \\
                         & = (g^{-1})^{k-1} g^{-1} g \\
                         & = (g^{-1})^k g            \\
                         & = g^{-k} g
            \end{align*}
        \end{proof}
    \end{proof}
    \step{2}{For all $k \in \mathbb{Z}$ we have $g^{k-1} = g^k g^{-1}$}
    \begin{proof}
        \pf\ Substitute $k = k-1$ above and multiply by $g^{-1}$.
    \end{proof}
    \step{3}{$g^{m+0} = g^m g^0$}
    \begin{proof}
        \pf\ Since $g^m g^0 = g^m e = g^m$.
    \end{proof}
    \step{4}{If $g^{m+n} = g^m g^n$ then $g^{m+n+1} = g^m g^{n+1}$}
    \begin{proof}
        \pf
        \begin{align*}
            g^{m+n+1} & = g^{m+n} g   & (\text{\stepref{1}}) \\
                      & = g^m g^n g                          \\
                      & = g^m g^{n+1} & (\text{\stepref{1}})
        \end{align*}
    \end{proof}
    \step{5}{If $g^{m+n} = g^m g^n$ then $g^{m+n-1} = g^m g^{n-1}$}
    \begin{proof}
        \pf
        \begin{align*}
            g^{m+n-1} g          & = g^{m+n}        & (\text{\stepref{1}}) \\
                                 & = g^m g^n                               \\
            \therefore g^{m+n-1} & = g^m g^n g^{-1}                        \\
                                 & = g^m g^{n-1}    & (\text{\stepref{2}})
        \end{align*}
    \end{proof}
    \qed
\end{proof}

\begin{prop}
    \label{prop:power-mult}
    Let $G$ be a group. Let $g \in G$ and $m,n \in \mathbb{Z}$. Then
    \[ (g^m)^n = g^{mn} \enspace . \]
\end{prop}

\begin{proof}
    \pf
    \step{1}{$(g^m)^0 = g^0$}
    \begin{proof}
        \pf\ Both sides are equal to $e$.
    \end{proof}
    \step{2}{If $(g^m)^n = g^{mn}$ then $(g^m)^{n+1} = g^{m(n+1)}$.}
    \begin{proof}
        \pf
        \begin{align*}
            (g^m)^{n+1} & = (g^m)^n g^m & (\text{Proposition \ref{prop:power-add}}) \\
                        & = g^{mn} g^m                                              \\
                        & = g^{mn + m}  & (\text{Proposition \ref{prop:power-add}})
        \end{align*}
    \end{proof}
    \step{3}{If $(g^m)^n = g^{mn}$ then $(g^m)^{n-1} = g^{m(n-1)}$.}
    \begin{proof}
        \pf
        \begin{align*}
            (g^m)^n                    & = g^{mn}                                                   \\
            \therefore (g^m)^{n-1} g^m & = g^{mn-m} g^m & (\text{Proposition \ref{prop:power-add}}) \\
            \therefore (g^m)^{n-1}     & = g^{mn-m}     & (\text{Cancellation})
        \end{align*}
    \end{proof}
    \qed
\end{proof}

\begin{df}[Commute]
    Let $G$ be a group and $g,h \in G$. We say $g$ and $h$ \emph{commute} iff $gh = hg$.
\end{df}

\begin{df}
Let $G$ be a group. Given $g \in G$ and $A \subseteq G$, we define
\[ gA = \{ ga : a \in A \}, \qquad Ag = \{ ag : a \in A \} \enspace . \]
Given sets $A,B \subseteq G$, we define
\[ AB = \{ ab : a \in A, b \in B \} \enspace . \]
\end{df}

\section{Symmetric Groups}

\begin{df}
Let $n$ be a natural number and $a_1, \ldots, a_r \in \{ 1, \ldots, n \}$ be distinct. The \emph{cycle} or \emph{$r$-cycle}
\[ (a_1 \ a_2 \ \cdots\ a_r) \in S_n \]
is the permutation that sends $a_i$ to $a_{i+1}$ ($1 \leq i < r$) and $a_r$ to $a_1$.

We call $r$ the \emph{length} of the cycle.

A \emph{transposition} is a 2-cycle.
\end{df}

\begin{prop}
Disjoint cycles commute.
\end{prop}

\begin{proof}
\pf\ Easy. \qed
\end{proof}

\begin{prop}
For any cycle $(a_1\ a_2\ \cdots\ a_r)$ in $S_n$ and $\tau \in S_n$ we have
\[ \tau (a_1\ a_2\ \cdots\ a_n) \inv{\tau} = (\tau(a_1)\ \tau(a_2)\ \cdots\ \tau(a_n)) \enspace . \]
\end{prop}

\begin{proof}
\pf\ Easy. \qed
\end{proof}

\section{Order of an Element}

\begin{df}[Order]
    Let $G$ be a group. Let $g \in G$. Then $g$ has \emph{finite order} iff there exists a positive integer $n$ such that $g^n = e$. In this case, the \emph{order} of $g$, denoted $|g|$, is the least positive integer $n$ such that $g^n = e$.

    If $g$ does not have finite order, we write $|g| = \infty$.
\end{df}

\begin{prop}
    Let $G$ be a group. Let $g \in G$ and $n$ be a positive integer. If $g^n = e$ then $|g| \mid n$.
\end{prop}

\begin{proof}
    \pf
    \step{1}{\pflet{$n = q |g| + d$ where $0 \leq d < |g|$}}
    \begin{proof}
        \pf\ Division Algorithm.
    \end{proof}
    \step{2}{$g^d = e$}
    \begin{proof}
        \pf
        \begin{align*}
            e & = g^n                                                                                 \\
              & = g^{q|g| + d}                                                                        \\
              & = (g^{|g|})^q g^d & (\text{Propositions \ref{prop:power-add}, \ref{prop:power-mult}}) \\
              & = e^q g^d                                                                             \\
              & = g^d
        \end{align*}
    \end{proof}
    \step{3}{$d = 0$}
    \begin{proof}
        \pf\ By minimality of $|g|$.
    \end{proof}
    \step{4}{$n = q|g|$}
    \qed
\end{proof}

\begin{cor}
    \label{cor:order-divides}
    Let $G$ be a group. Let $g \in G$ have finite order and $n \in \mathbb{Z}$. Then $g^n = e$ if and only if $|g| \mid n$.
\end{cor}

%TODO: In fact |g| \mid |G|
\begin{prop}
    Let $G$ be a group and $g \in G$. Then $|g| \leq |G|$.
\end{prop}

\begin{proof}
    \pf
    \step{1}{\assume{w.l.o.g. $G$ is finite.}}
    \step{2}{\pick\ $i$, $j$ with $0 \leq i < j \leq |G|$ such that $g^i = g^j$.}
    \begin{proof}
        \pf\ Otherwise $g^0$, $g^1$, \ldots, $g^{|G|}$ would be $|G| + 1$ distinct elements of $G$.
    \end{proof}
    \step{3}{$g^{j-i} = e$}
    \step{4}{$g$ has finite order and $|g| \leq |G|$}
    \begin{proof}
        \pf\ Since $|g| \leq j - i \leq j \leq |G|$.
    \end{proof}
    \qed
\end{proof}

\begin{prop}
\label{prop:order-of-g-to-the-m}
    Let $G$ be a group. Let $g \in G$ have finite order. Let $m \in \mathbb{N}$. Then
    \[ |g^m| = \frac{\lcm(m,|g|)}{m} = \frac{|g|}{\gcd(m,|g|)}\]
\end{prop}

\begin{proof}
    \pf\ Since for any integer $d$ we have
    \begin{align*}
        g^{md} = e & \Leftrightarrow |g| \mid md                  & (\text{Corollary \ref{cor:order-divides}}) \\
                   & \Leftrightarrow \lcm(m,|g|) \mid md                                                       \\
                   & \Leftrightarrow \frac{\lcm(m,|g|)}{m} \mid d & \qed
    \end{align*}
    and so $|g^m| = \frac{\lcm(m,|g|)}{m}$ by Corollary \ref{cor:order-divides}. \qed
\end{proof}

\begin{cor}
    If $g$ has odd order then $|g^2| = |g|$.
\end{cor}

\begin{prop}
    \label{prop:order-gh}
    Let $G$ be a group. Let $g,h \in G$ have finite order. Assume $gh = hg$. Then $|gh|$ has finite order and
    \[ |gh| \mid \lcm(|g|,|h|)\]
\end{prop}

\begin{proof}
    \pf\ Since $(gh)^{\lcm(|g|,|h|)} = g^{\lcm(|g|,|h|)}h^{\lcm(|g|,|h|)} = e$. \qed
\end{proof}

\begin{ex}
    This example shows that we cannot remove the hypothesis that $gh = hg$.

    In $\mathrm{GL}_2(\mathbb{R})$, take
    \[ g = \left( \begin{array}{cc} 0 & -1 \\ 1 & 0 \end{array} \right), \qquad
        h = \left( \begin{array}{cc} 0 & 1 \\ -1 & -1 \end{array} \right) \enspace . \]
    Then $|g| = 4$, $|h| = 3$ and $|gh| = \infty$.
\end{ex}

\begin{prop}
    \label{prop:order-gh-if-gcd-one}
    Let $G$ be a group and $g,h \in G$ have finite order. If $gh=hg$ and $\gcd(|g|,|h|) = 1$ then $|gh| = |g||h|$.
\end{prop}

\begin{proof}
    \pf
    \step{1}{\pflet{$N = |gh|$}}
    \step{2}{$g^N = (\inv{h})^N$}
    \step{3}{$g^{N|g|} = e$}
    \step{4}{$|g^N| \mid |g|$}
    \step{5}{$h^{-N|h|} = e$}
    \step{5}{$|g^N| \mid |h|$}
    \step{6}{$|g^N| = 1$}
    \begin{proof}
        \pf\ Since $\gcd(|g|,|h|) = 1$.
    \end{proof}
    \step{7}{$g^N = e$}
    \step{8}{$|g| \mid N$}
    \step{9}{$h^{-N} = e$}
    \step{10}{$|h| \mid N$}
    \step{11}{$N = |g||h|$}
    \begin{proof}
        \pf\ Using Proposition \ref{prop:order-gh}.
    \end{proof}
    \qed
\end{proof}

\begin{prop}
    \label{prop:product-of-all-elements}
    Let $G$ be a finite group. Assume there is exactly one element $f \in G$ of order 2. Then the product of all the elements of $G$ is $f$.
\end{prop}

\begin{proof}
    \pf\ Let the elements of $G$ be $g_1$, $g_2$, \ldots, $g_n$. Apart from $e$ and $f$, every element and its inverse are distinct elements of the list. Hence the product of the list is $ef = f$. \qed
\end{proof}

\begin{prop}
    Let $G$ be a finite group of order $n$. Let $m$ be the number of elements of $G$ of order 2. Then $n-m$ is odd.
\end{prop}

\begin{proof}
    \pf\ In the list of all elements that are not of order 2, every element and its inverse are distinct except for $e$. Hence the list has odd length. \qed
\end{proof}

\begin{cor}
    If a finite group has even order, then it contains an element of order 2.
\end{cor}

\begin{prop}
    Let $G$ be a group and $a,g \in G$. Then $|ag\inv{a}| = |g|$.
\end{prop}

\begin{proof}
    \pf\ Since
    \begin{align*}
        (ag\inv{a})^n = e & \Leftrightarrow a g^n \inv{a} = e        \\
                          & \Leftrightarrow g^n = e           & \qed
    \end{align*}
\end{proof}

\begin{prop}
    Let $G$ be a group and $g,h \in G$. Then $|gh| = |hg|$.
\end{prop}

\begin{proof}
    \pf\ Since $|gh| = |ghg\inv{g}| = |hg|$. \qed
\end{proof}

\begin{prop}
Let $G$ be a group of order $n$. Let $k$ be relatively prime to $n$. Then every element in $G$ has the form $x^k$ for some $x$.
\end{prop}

\begin{proof}
\step{1}{\pick\ integers $a$ and $b$ such that $an + bk = 1$.}
\step{2}{\pflet{$g \in G$}}
\step{3}{$g = (g^b)^k$}
\begin{proof}
\pf
\begin{align*}
g & = g. (g^n)^{-a} & (g^n = e) \\
& = g^{1-an} \\
& = g^{bk}
\end{align*}
\end{proof}
\qed
\end{proof}

\section{Generators}

\begin{df}[Generator]
    Let $G$ be a group and $a \in G$. We say $a$ \emph{generates} the group iff, for all $x \in G$, there exists an integer $n$ such that $x^n = a$.
\end{df}

\begin{ex}
\label{ex:SL2Z}
$\mathrm{SL}_2(\mathbb{Z})$ is generated by
\[ s = \left( \begin{array}{cc}
0 & -1 \\ 1 & 0
\end{array} \right), \qquad
t = \left( \begin{array}{cc}
1 & 1 \\
0 & 1
\end{array} \right) \]
\end{ex}

\begin{proof}
\pf
\step{1}{\pflet{$H = \langle s, t \rangle$}}
\step{2}{For all $q \in \mathbb{Z}$ we have $\left( \begin{array}{cc} 1 & q \\ 0 & 1 \end{array} \right) \in H$.}
\begin{proof}
	\pf\ It is $t^q$.
\end{proof}
\step{3}{For all $q \in \mathbb{Z}$ we have $\left( \begin{array}{cc} 1 & 0 \\ q & 1 \end{array} \right) \in H$.}
\begin{proof}
\pf
\begin{align*}
st^{-q}\inv{s} & = \left( \begin{array}{cc}
0 & -1 \\
1 & 0
\end{array} \right)
\left( \begin{array}{cc}
1 & -q \\ 0 & 1
\end{array} \right)
\left( \begin{array}{cc}
0 & 1 \\
-1 & 0
\end{array} \right) \\
& = \left( \begin{array}{cc}
0 & -1 \\
1 & -q
\end{array} \right)
\left( \begin{array}{cc}
0 & 1 \\
-1 & 0
\end{array} \right) \\
& = \left( \begin{array}{cc}
1 & 0 \\
q & 1
\end{array} \right)
\end{align*}
\end{proof}
\step{4}{\[ \left( \begin{array}{cc} a & b \\ c & d \end{array} \right) \left( \begin{array}{cc} 1 & q \\ 0 & 1 \end{array} \right) = \left( \begin{array}{cc} a & qa + b \\ c & qc + d \end{array} \right) \]}
\step{5}{\[ \left( \begin{array}{cc} a & b \\ c & d \end{array} \right) \left( \begin{array}{cc} 1 & 0 \\ q & 1 \end{array} \right) = \left( \begin{array}{cc} a+qb & b \\ c +qd & d \end{array} \right) \]}
\step{6}{For any $M = \left( \begin{array}{cc} a & b \\ c & d \end{array} \right) \in \mathrm{SL}_2(\mathbb{Z})$, if $c$ and $d$ are both nonzero, then there exists $N \in H$ such that the bottom row of $MN$ has one entry the same as $M$ and one entry with smaller absolute value.}
\begin{proof}
\pf\ From \stepref{4} and \stepref{5} taking $q = -1$.
\end{proof}
\step{7}{For any $M \in \mathrm{SL}_2(\mathbb{Z})$, there exists $N \in H$ such that $MN$ has a zero on the bottom row.}
\begin{proof}
\pf\ Apply \stepref{6} repeatedly.
\end{proof}
\step{8}{Any matrix in $\SL{2}{\mathbb{Z}}$ with a zero on the bottom row is in $H$.}
\begin{proof}
\step{a}{$\left( \begin{array}{cc}
1 & b \\
0 & 1
\end{array} \right) \in H$}
\begin{proof}
\pf\ \stepref{2}
\end{proof}
\step{b}{$\left( \begin{array}{cc}
-1 & b \\
0 & -1
\end{array} \right) \in H$}
\begin{proof}
\pf\ It is
$s^2 \left( \begin{array}{cc}
1 & b \\
0 & 1
\end{array} \right)$ since $s^2 = -I$.
\end{proof}
\step{c}{$\left( \begin{array}{cc}
a & 1 \\
-1 & 0
\end{array} \right) \in H$}
\begin{proof}
\pf\ It is $\left( \begin{array}{cc}
1 & -a \\ 0 & 1 \end{array} \right) s$.
\end{proof}
\step{d}{$\left( \begin{array}{cc}
a & -1 \\
1 & 0
\end{array} \right) \in H$}
\begin{proof}
\pf\ It is $s^2 \left( \begin{array}{cc}
1 & a \\ 0 & 1 \end{array} \right) s$.
\end{proof}
\end{proof}
\step{9}{Every matrix in $\SL{2}{\mathbb{Z}}$ is in $H$.}
\qed
\end{proof}

\section{$p$-groups}

\begin{df}[$p$-group]
Let $p$ be a prime. A \emph{$p$-group} is a finite group whose order is a power of $p$.
\end{df}

\chapter{Group Homomorphisms}

\begin{df}[Homomorphism]
    Let $G$ and $H$ be groups. A \emph{(group) homomorphism} $\phi : G \rightarrow H$ is a function such that, for all $x,y \in G$,
    \[ \phi(xy) = \phi(x) \phi(y) \enspace . \]
\end{df}

\begin{prop}
    Let $G$ and $H$ be groups with identities $e_G$ and $e_H$.
    Let $\phi : G \rightarrow H$ be a group homomorphism. Then $\phi(e_G) = e_H$.
\end{prop}

\begin{proof}
    \pf\ Since $\phi(e_G) = \phi(e_G e_G) = \phi(e_G) \phi(e_G)$ and so $\phi(e_G) = e_H$ by Cancellation. \qed
\end{proof}

\begin{prop}
    Let $\phi : G \rightarrow H$ be a group homomorphism. For all $x \in G$ we have $\phi(x^{-1}) = \phi(x)^{-1}$.
\end{prop}

\begin{proof}
    \pf\ Since $\phi(x) \phi(x^{-1}) = \phi(xx^{-1}) = \phi(e_G) = e_H$. \qed
\end{proof}

\begin{prop}
    Let $G$, $H$ and $K$ be groups. If $\phi : G \rightarrow H$ and $\psi : H \rightarrow K$ are homomorphisms then $\psi \circ \phi : G \rightarrow K$ is a homomorphism.
\end{prop}

\begin{proof}
    \pf\ For $x,y \in G$ we have
    \[ \psi(\phi(xy)) = \psi(\phi(x) \phi(y)) = \psi(\phi(x)) \psi(\phi(y)) \enspace . \]
\end{proof}

\begin{prop}
    Let $G$ be a group. Then $\id{G} : G \rightarrow G$ is a group homomorphism.
\end{prop}

\begin{proof}
    \pf\ For $x,y \in G$ we have $\id{G}(xy) = xy = \id{G}(x) \id{G}(y)$. \qed
\end{proof}

\begin{prop}
    Let $\phi : G \rightarrow H$ be a group homomorphism. Let $g \in G$ have finite order. Then $|\phi(g)|$ divides $|g|$.
\end{prop}

\begin{proof}
    \pf\ Since $\phi(g)^{|g|} = \phi(g^{|g|}) = e$. \qed
\end{proof}

\begin{df}[Category of Groups]
    Let $\mathbf{Grp}$ be the category of groups and group homomorphisms.
\end{df}

\begin{ex}
There are 49487365402 groups of order 1024 up to isomorphism. %TODO
\end{ex}

\begin{prop}
    A group homomorphism $\phi : G \rightarrow H$ is an isomorphism in $\mathbf{Grp}$ if and only if it is bijective.
\end{prop}

\begin{proof}
    \pf
    \step{1}{\assume{$\phi$ is bijective.} \prove{$\inv{\phi}$ is a group homomorphism.}}
    \step{2}{\pflet{$h,h' \in H$}}
    \step{3}{$\phi(\inv{\phi}(hh')) = \phi(\inv{\phi}(h)\inv{\phi}(h'))$}
    \begin{proof}
        \pf\ Both are equal to $hh'$.
    \end{proof}
    \step{4}{$\inv{\phi}(hh') = \inv{\phi}(h) \inv{\phi}(h')$}
    \qed
\end{proof}

\begin{cor}
    \[ D_6 \cong C_3 \]
\end{cor}

\begin{proof}
    \pf\ The canonical homomorphism $D_6 \rightarrow C_3$ is bijective. \qed
\end{proof}

\begin{cor}
    \[ (\mathbb{R}, +) \cong (\{ x \in \mathbb{R} : x > 0 \}, \cdot) \]
\end{cor}

\begin{proof}
    \pf\ The function that maps $x$ to $e^x$ is a bijective homomorphism. \qed
\end{proof}

\begin{prop}
    The trivial group is the zero object in $\mathbf{Grp}$.
\end{prop}

\begin{proof}
    \pf\ For any group $G$, the unique function $G \rightarrow \{e\}$ is a group homomorphism, and the only group homomorphism $\{e\} \rightarrow G$ maps $e$ to $e_G$. \qed
\end{proof}

\begin{prop}
    For any groups $G$ and $H$, the set $G \times H$ under $(g,h)(g',h') = (gg',hh')$ is the product of $G$ and $H$ in $\mathbf{Grp}$.
\end{prop}

\begin{proof}
    \pf
    \step{1}{$G \times H$ is a group.}
    \begin{proof}
        \step{a}{The multiplication is associative.}
        \begin{proof}
            \pf\ Since $(g_1,h_1)((g_2,h_2)(g_3,h_3)) = ((g_1,h_1)(g_2,h_2))(g_3,h_3) = (g_1 g_2 g_3, h_1 h_2 h_3)$.
        \end{proof}
        \step{b}{$(e_G,e_H)$ is the identity.}
        \begin{proof}
            \pf\ Since $(g,h)(e_G,e_H) = (e_G,e_H)(g,h) = (g,h)$.
        \end{proof}
        \step{c}{The inverse of $(g,h)$ is $(\inv{g},\inv{h})$.}
        \begin{proof}
            \pf\ Since $(g,h)(\inv{g},\inv{h}) = (\inv{g},\inv{h})(g,h) = (e_G,e_H)$.
        \end{proof}
    \end{proof}
    \step{2}{$\pi_1 : G \times H \rightarrow G$ is a group homomorphism.}
    \begin{proof}
        \pf\ Immediate from definitions.
    \end{proof}
    \step{3}{$\pi_2 : G \times H \rightarrow H$ is a group homomorphism.}
    \begin{proof}
        \pf\ Immediate from definitions.
    \end{proof}
    \step{4}{For any group homomorphism $\phi : K \rightarrow G$ and $\psi : K \rightarrow H$, the function $\langle \phi, \psi \rangle : K \rightarrow G \times H$ where $\langle \phi, \psi \rangle(k) = (\phi(k), \psi(k))$ is a group homomorphism.}
    \begin{proof}
        \pf
        \begin{align*}
            \langle \phi, \psi \rangle (kk') & = (\phi(kk'), \psi(kk'))                                     \\
                                             & = (\phi(k)\phi(k'), \psi(k)\psi(k'))                         \\
                                             & = (\phi(k),\psi(k)) (\phi(k'),\psi(k'))                      \\
                                             & = \langle \phi,\psi \rangle(k) \langle \phi,\psi \rangle(k')
        \end{align*}
    \end{proof}
    \qed
\end{proof}

\section{Subgroups}

\begin{df}[Subgroup]
Let $(G,\cdot)$ and $(H,*)$ be groups such that $H$ is a subset of $G$. Then $H$ is a \emph{subgroup} of $G$ iff the inclusion $i : H \hookrightarrow G$ is a group homomorphism.
\end{df}

\begin{prop}
If $(H,*)$ is a subgroup of $(G,\cdot)$ then $*$ is the restriction of $\cdot$ to $H$.
\end{prop}

\begin{proof}
\pf\ Given $x,y \in H$ we have
\[ x * y = i(x * y) = i(x) \cdot i(y) = x \cdot y \enspace .  \qquad \qed \]
\end{proof}

\begin{ex}
For any group $G$ we have $\{e\}$ is a subgroup of $G$.
\end{ex}

\begin{prop}
Let $G$ be a group. Let $H$ be a subset of $G$. Then $H$ is a subgroup of $G$ iff $H$ is nonempty and, for all $x,y \in H$, we have $x \inv{y} \in H$.
\end{prop}

\begin{proof}
\pf
\step{1}{If $H$ is a subgroup of $G$ then $H$ is nonempty.}
\begin{proof}
\pf\ Since every group has an identity element and so is nonempty.
\end{proof}
\step{2}{If $H$ is a subgroup of $G$ then, for all $x,y \in H$, we have $x \inv{y} \in H$.}
\begin{proof}
\pf\ Easy.
\end{proof}
\step{3}{If $H$ is nonempty and, for all $x,y \in H$, we have $x \inv{y} \in H$, then $H$ is a subgroup of $G$.}
\begin{proof}
	\step{a}{\assume{$H$ is nonempty.}}
	\step{b}{\assume{$\forall x,y \in H. x \inv{y} \in H$}}
	\step{c}{$e \in H$}
	\begin{proof}
		\pf\ Pick $x \in H$. We have $e = x \inv{x} \in H$.
	\end{proof}
	\step{d}{$\forall x \in H. \inv{x} \in H$}
	\begin{proof}
		\pf\ Given $x \in H$ we have $\inv{x} = e \inv{x} \in H$.
	\end{proof}
	\step{c}{$H$ is closed under the restriction of $\cdot$}
	\begin{proof}
		\pf\ Given $x,y \in H$ we have $xy = x \inv{(\inv{y})} \in H$.
	\end{proof}
	\step{d}{$H$ is a group under the restriction of $\cdot$}
	\begin{proof}
		\pf\ Associativity is inherited from $G$ and the existence of an identity element and inverses follows from \stepref{c} and \stepref{d}.
	\end{proof}
	\step{e}{The inclusion $H \hookrightarrow G$ is a group homomorphism.}
	\begin{proof}
		\pf\ For $x,y \in H$ we have $i(xy) = i(x)i(y) = xy$.
	\end{proof}
\end{proof}
\qed
\end{proof}

\begin{cor}
The intersection of a set of subgroups of $G$ is a subgroup of $G$.
\end{cor}

\begin{cor}
\label{cor:inverse-image-subgroup}
Let $\phi : G \rightarrow H$ be a group homomorphism. Let $K$ be a subgroup of $H$. Then $\inv{\phi}(K)$ is a subgroup of $G$.
\end{cor}

\begin{proof}
\pf
\step{0}{$\inv{\phi}(K)$ is nonempty.}
\begin{proof}
\pf\ Since $e \in \inv{\phi}(K)$.
\end{proof}
\step{1}{\pflet{$x,y \in \inv{\phi}(K)$}}
\step{2}{$\phi(x),\phi(y) \in K$}
\step{3}{$\phi(x)\inv{\phi(y)} \in K$}
\step{4}{$\phi(x\inv{y}) \in K$}
\step{5}{$x\inv{y} \in \inv{\phi}(K)$}
\qed
\end{proof}

\begin{cor}
Let $\phi : G \rightarrow H$ be a group homomorphism. Let $K$ be a subgroup of $G$. Then $\phi(K)$ is a subgroup of $H$.
\end{cor}

\begin{proof}
\pf
\step{1}{\pflet{$x,y \in \phi(K)$}}
\step{2}{\pick\ $a,b \in K$ such that $x = \phi(a)$ and $y = \phi(b)$}
\step{3}{$x\inv{y} = \phi(a\inv{b})$}
\step{4}{$x\inv{y} \in \phi(K)$}
\qed
\end{proof}

\begin{prop}
\label{prop:subgroups-of-Z}
Let $G$ be a subgroup of $\mathbb{Z}$. Then there exists $d \geq 0$ such that $G = d \mathbb{Z}$.
\end{prop}

\begin{proof}
\pf
\step{1}{\assume{w.l.o.g. $G \neq \{0\}$}}
\begin{proof}
\pf\ Since $\{0\} = 0 \mathbb{Z}$.
\end{proof}
\step{2}{\pflet{$d$ be the least positive element of $G$.} \prove{$G = d \mathbb{Z}$}}
\begin{proof}
\pf\ If $n \in G$ then $-n \in G$ so $G$ must contain a positive element.
\end{proof}
\step{3}{$G \subseteq d \mathbb{Z}$}
\begin{proof}
	\step{a}{\pflet{$n \in G$}}
	\step{b}{\pflet{$q$ and $r$ be the integers such that $n = qd + r$ and $0 \leq r < d$.}}
	\step{c}{$r \in G$}
	\begin{proof}
		\pf\ Since $r = n - qd$.
	\end{proof}
	\step{d}{$r = 0$}
	\begin{proof}
		\pf\ By minimality of $d$.
	\end{proof}
	\step{e}{$n = qd \in d \mathbb{Z}$}
\end{proof}
\step{4}{$d \mathbb{Z} \subseteq G$}
\qed
\end{proof}

\section{Kernel}

\begin{df}[Kernel]
Let $\phi : G \rightarrow H$ be a group homomorphism. The \emph{kernel} of $\phi$ is
\[ \ker \phi = \{ g \in G : \phi(g) = e \} \enspace . \]
\end{df}

\begin{prop}
Let $\phi : G \rightarrow H$ be a group homomorphism. Then $\ker \phi$ is a subgroup of $G$.
\end{prop}

\begin{proof}
\pf\ Corollary \ref{cor:inverse-image-subgroup}. \qed
\end{proof}

\begin{prop}
Let $\phi : G \rightarrow H$ be a group homomorphism. Then the inclusion $i : \ker \phi \hookrightarrow G$ is terminal in the category of pairs $(K,\alpha : K \rightarrow G)$ such that $\phi \circ \alpha = 0$.
\end{prop}

\begin{proof}
\pf
\step{1}{$\phi \circ i = 0$}
\step{2}{For any group $K$ and homomorphism $\alpha : K \rightarrow G$ such that $\phi \circ \alpha = 0$, there exists a unique homomorphism $\beta : K \rightarrow \ker \phi$ such that $i \circ \beta = \alpha$.}
\qed
\end{proof}

\begin{prop}
\label{prop:ker-zero}
Let $\phi : G \rightarrow H$ be a group homomorphism. Then the following are equivalent:
\begin{enumerate}
\item $\phi$ is monic.
\item $\ker \phi = \{e\}$
\item $\phi$ is injective.
\end{enumerate}
\end{prop}

\begin{proof}
\pf
\step{1}{$1 \Rightarrow 2$}
\begin{proof}
	\step{a}{\assume{$\phi$ is monic.}}
	\step{b}{\pflet{$i : \ker \phi \hookrightarrow G$, $j : \{e\} \hookrightarrow \ker \phi \hookrightarrow G$ be the inclusions.}}
	\step{c}{$\phi \circ i = \phi \circ j$}
	\step{d}{$i = j$}
\end{proof}
\step{2}{$2 \Rightarrow 3$}
\begin{proof}
	\step{a}{\assume{$\ker \phi = \{e\}$}}
	\step{b}{\pflet{$x,y \in G$}}
	\step{c}{\assume{$\phi(x) = \phi(y)$}}
	\step{d}{$\phi(x\inv{y}) = e$}
	\step{e}{$x \inv{y} \in \ker \phi$}
	\step{f}{$x \inv{y} = e$}
	\step{g}{$x = y$}
\end{proof}
\step{3}{$3 \Rightarrow 1$}
\begin{proof}
	\pf\ Easy.
\end{proof}
\qed
\end{proof}

\begin{prop}
A group homomorphism is an epimorphism if and only if it is surjective.
\end{prop}

%TODO

\section{Inner Automorphisms}

\begin{prop}
    Let $G$ be a group and $g \in G$. The function $\gamma_g : G \rightarrow G$ defined by $\gamma_g(a) = ga\inv{g}$ is an automorphism on $G$.
\end{prop}

\begin{proof}
    \pf
    \step{1}{$\gamma_g$ is a homomorphism.}
    \begin{proof}
        \pf
        \begin{align*}
            \gamma_g(ab) & = gab\inv{g}              \\
                         & = ga\inv{g} gb\inv{g}     \\
                         & = \gamma_g(a) \gamma_g(b)
        \end{align*}
    \end{proof}
    \step{2}{$\gamma_g$ is injective.}
    \begin{proof}
        \pf\ By Cancellation.
    \end{proof}
    \step{3}{$\gamma_g$ is surjective.}
    \begin{proof}
        \pf\ Given $b \in G$, we have $\gamma_g(\inv{g}bg) = b$.
    \end{proof}
    \qed
\end{proof}

\begin{df}[Inner Automorphism]
    Let $G$ be a group. An \emph{inner automorphism} on $G$ is a function of the form $\gamma_g(a) = ga\inv{g}$ for some $g \in G$.
    
    We write $\mathrm{Inn}(G)$ for the set of inner automorphisms of $G$.
\end{df}

\begin{prop}
    Let $G$ be a group.
    The function $\gamma : G \rightarrow \Aut{\Grp}{G}$ that maps $g$ to $\gamma_g$ is a group homomorphism.
\end{prop}

\begin{proof}
    \pf\ Since $\gamma_{gh}(a) = gha\inv{h}\inv{g} = \gamma_g(\gamma_h(a))$. \qed
\end{proof}

\begin{cor}
$\mathrm{Inn}(G)$ is a subgroup of $\Aut{\Grp}{G}$.
\end{cor}

\section{Direct Products}

\begin{df}[Direct Product]
    The \emph{direct product} of groups $G$ and $H$ is their product in $\mathbf{Grp}$.
\end{df}

\section{Free Groups}

\begin{prop}
    Let $A$ be a set. Let $\mathcal{F}^A$ be the category whose objects are pairs $(G,j)$ where $G$ is a group and $j$ is a function $A \rightarrow G$, with morphisms $f : (G,j) \rightarrow (H,k)$ the group homomorphisms $f : G \rightarrow H$ such that $f \circ j = k$. Then $\mathcal{F}^A$ has an initial object.
\end{prop}

\begin{proof}
    \pf
    \step{1}{\pflet{$W(A)$ be the set of words in the alphabet whose elements are the elements of $A$ together with $\{ \inv{a} : a \in A \}$.}}
    \step{2}{\pflet{$r : W(A) \rightarrow W(A)$ be the function that, given a word $w$, removes the first pair of letters of the form $a \inv{a}$ or $\inv{a}a$; if there is no such pair, then $r(w) = w$.}}
    \step{3}{Let us say that a word $w$ is a \emph{reduced word} iff $r(w) = w$.}
    \step{4}{For any word $w$ of length $n$, we have $r^{\ulcorner \frac{n}{2} \urcorner}(w)$ is a reduced word.}
    \begin{proof}
        \pf\ Since we cannot remove more than $n/2$ pairs of letters from $w$.
    \end{proof}
    \step{5}{\pflet{$R : W(A) \rightarrow W(A)$ be the function $R(w) = r^{\ulcorner \frac{n}{2} \urcorner}(w)$, where $n$ is the length of $w$.}}
    \step{5}{\pflet{$F(A)$ be the set of reduced words.}}
    \step{6}{Define $\cdot : F(A)^2 \rightarrow F(A)$ by $w \cdot w' = R(ww')$}
    \step{7}{$\cdot$ is associative.}
    \begin{proof}
        \pf\ Both $w_1 \cdot (w_2 \cdot w_3)$ and $(w_1 \cdot w_2) \cdot w_3$ are equal to $R(w_1 w_2 w_3)$.
    \end{proof}
    \step{8}{The empty word is the identity element in $F(A)$}
    \step{9}{The inverse of $a_1^{\pm 1} a_2^{\pm 1} \cdots a_n^{\pm 1}$ is $a_n^{\mp 1} \cdots a_2^{\mp 1} a_1^{\mp 1}$.}
    \step{10}{\pflet{$j : A \rightarrow F(A)$ be the function that maps $a$ to the word $a$ of length .}}
    \step{11}{\pflet{$G$ be any group and $k : A \rightarrow G$ any function.}}
    \step{12}{The only morphism $f : (F(A),j) \rightarrow (G,k)$ in $\mathcal{F}^A$ is $f(a_1^{\pm 1} a_2^{\pm 1} \cdots a_n^{\pm 1}) = k(a_1)^{\pm 1} k(a_2)^{\pm 1} \cdots k(a_n)^{\pm 1}$.}
    \qed
\end{proof}

\begin{df}[Free Group]
    For any set $A$, the \emph{free group} on $A$ is the initial object $(F(A),i)$ in $\mathcal{F}^A$.
\end{df}

\begin{prop}
    $i : A \rightarrow F(A)$ is injective.
\end{prop}

\begin{proof}
    \pf
    \step{1}{\pflet{$x,y \in A$}}
    \step{2}{\assume{$x \neq y$} \prove{$i(x) \neq i(y)$}}
    \step{3}{\pflet{$f : A \rightarrow C_2$ be the function that maps $x$ to $0$ and all other elements of $A$ to 1.}}
    \step{4}{\pflet{$\phi : F(A) \rightarrow C_2$ be the group homomorphism such that $f = \phi\circ i$.}}
    \step{5}{$f(x) \neq f(y)$}
    \step{6}{$\phi(i(x)) \neq \phi(i(y))$}
    \step{7}{$i(x) \neq i(y)$}
    \qed
\end{proof}

\begin{prop}
    \[ F(0) \cong \{e\} \]
\end{prop}

\begin{proof}
    \pf\ For any set $A$, the unique group homomorphism $\{e\} \rightarrow A$ makes the following diagram commute.
    \[ \begin{tikzcd}
            \{e\} \arrow[r] & A \\
            \emptyset \arrow[u] \arrow[ur]
        \end{tikzcd} \]
\end{proof}

\begin{prop}
    The free group on 1 is $\mathbb{Z}$ with the injection mapping 0 to $1$.
\end{prop}

\begin{proof}
    \pf\ Given any group $G$ and function $a : 1 \rightarrow G$, the required unique homomorphism $\phi : \mathbb{Z} \rightarrow G$ is defined by $\phi(n) = a(0)^n$. \qed
\end{proof}

\begin{prop}
    For any sets $A$ and $B$, we have that $F(A+B)$ is the coproduct of $F(A)$ and $F(B)$ in $\Grp$.
\end{prop}

\[ \begin{tikzcd}
        & G & \\
        F(A) \arrow[r,"\kappa_1"] \arrow[ur,"f"] & F(A+B) \arrow[u,"k"] & \arrow[l,"\kappa_2"] F(B) \arrow[ul,"g"] \\
        A \arrow[r,"k_1"] \arrow[u,"i_A"] & A+B \arrow[u,"j"] & \arrow[l,"k_2"] \arrow[u,"i_B"] B
    \end{tikzcd} \]

\begin{proof}
    \pf
    \step{1}{\pflet{$i_A : A \rightarrow F(A)$, $i_B : B \rightarrow F(B)$, $j : A + B \rightarrow F(A+B)$ be the canonical injections.}}
    \step{2}{\pflet{$\kappa_1$, $\kappa_2$ be the unique group homomorphisms that make the diagram above commute.}}
    \step{3}{\pflet{$G$ be any group and $f : F(A) \rightarrow G$, $g : F(B) \rightarrow G$ any group homomorphisms.}}
    \step{4}{\pflet{$h : A + B \rightarrow G$ be the unique function such that $h \circ k_1 = f \circ i_A$ and $h \circ k_2 = g \circ i_B$.}}
    \step{5}{\pflet{$k : F(A+B) \rightarrow G$ be the unique group homomorphism such that $k \circ j = h$.}}
    \step{6}{$k$ is the unique group homomorphism such that $k \circ \kappa_1 \circ i_A = f \circ i_A$ and $k \circ \kappa_2 \circ i_B = g \circ i_B$.}
    \step{7}{$k$ is the unique group homomorphism such that $k \circ \kappa_1 = f$ and $k \circ \kappa_2 = g$.}
    \qed
\end{proof}

\begin{df}[Subgroup Generated by a Group]
Let $G$ be a group and $A$ a subset of $G$. Let $\phi : F(A) \rightarrow G$ be the unique group homomorphism such that $\phi(a) = a$ for all $a \in A$. The subgroup \emph{generated} by $A$ is

\[ \langle A \rangle := \im \phi \]

\[ \begin{tikzcd}
F(A) \arrow[r,"\phi"] & G \\
A \arrow[u] \arrow[ur]
\end{tikzcd} \]
\end{df}

\begin{prop}
Let $G$ be a group and $A$ a subset of $G$. Then $\langle A \rangle$ is the set of all elements of the form $a_1^{\pm 1} a_2^{\pm 1} \cdots a_n^{\pm 1}$ (where $n \geq 0$) such that $a_1, \ldots, a_n \in A$.
\end{prop}

\begin{proof}
\pf\ Immediate from definitions. \qed
\end{proof}

\begin{cor}
Let $G$ be a group and $g \in G$. Then
\[ \langle g \rangle = \{ g^n : n \in \mathbb{Z} \} \enspace . \]
\end{cor}

\begin{prop}
Let $G$ be a group and $A$ a subset of $G$. Then $\langle A \rangle$ is the intersection of all the subgroups of $G$ that include $A$.
\end{prop}

\begin{proof}
\pf\ Easy. \qed
\end{proof}

\begin{df}[Finitely Generated]
Let $G$ be a group. Then $G$ is \emph{finitely generated} iff there exists a finite subset $A$ of $G$ such that $G = \langle A \rangle$.
\end{df}

\begin{prop}
Every subgroup of a finitely generated free group is free.
\end{prop}

\begin{proof}
\pf\ TODO.
\end{proof}

\begin{prop}
$F(2)$ includes subgroups isomorphic to the free group on arbitrarily many generators.
\end{prop}

\begin{proof}
\pf\ TODO
\end{proof}

\begin{prop}
\[ [F(2),F(2)] \cong F(\mathbb{Z}) \]
\end{prop}

\begin{proof}
\pf\ TODO
\end{proof}

\section{Normal Subgroups}

\begin{df}[Normal Subgroup]
A subgroup $N$ of $G$ is \emph{normal} iff, for all $g \in G$ and $n \in N$, we have $gn\inv{g} \in N$.
\end{df}

\begin{ex}
Every subgroup of $Q_8$ is normal.
\end{ex}

\begin{prop}
Let $G$ be a group and $N$ a subgroup of $G$. Then the following are equivalent.
\begin{enumerate}
\item $N$ is normal.
\item $\forall g \in G. g N \inv{g} \subseteq N$
\item $\forall g \in G. g N \inv{g} = N$
\item $\forall g \in G. g N \subseteq N g$
\item $\forall g \in G. gN = Ng$
\end{enumerate}
\end{prop}

\begin{proof}
\pf
\step{1}{$1 \Leftrightarrow 2$}
\begin{proof}
\pf\ Immediate from definitions.
\end{proof}
\step{2}{$2 \Rightarrow 3$}
\begin{proof}
	\pf\ If 2 holds then we have $gN\inv{g} \subseteq N$ and $\inv{g}Ng \subseteq N$ hence $N = gN\inv{g}$.
\end{proof}
\step{2}{$3 \Rightarrow 2$}
\begin{proof}
\pf\ Trivial.
\end{proof}
\step{3}{$2 \Leftrightarrow 4$}
\begin{proof}
\pf\ Easy.
\end{proof}
\step{4}{$3 \Leftrightarrow 5$}
\begin{proof}
\pf\ Easy.
\end{proof}
\qed
\end{proof}

\begin{prop}
\label{prop:kernel-normal}
Let $\phi : G \rightarrow H$ be a group homomorphism. Then $\ker \phi$ is a normal subgroup of $G$.
\end{prop}

\begin{proof}
\pf\ Given $g \in G$ and $n \in \ker \phi$ we have
\begin{align*}
\phi(gn\inv{g}) & = \phi(g) \phi(n) \inv{\phi(g)} \\
& = \phi(g) \inv{\phi(g)} \\
& = e
\end{align*}
and so $gn\inv{g} \in \ker \phi$. \qed
\end{proof}

\begin{prop}
\label{prop:HK-normal}
If $H$ and $K$ are normal subgroups of a group $G$ then $HK$ is normal in $G$.
\end{prop}

\begin{proof}
\pf\ For $g \in G$, $h \in H$ and $k \in K$ we have $ghk\inv{g} = (gh\inv{g})(gk\inv{g}) \in HK$.
\qed
\end{proof}

\section{Quotient Groups}

\begin{df}
Let $G$ be a group. Let $\sim$ be an equivalence relation on $G$. Then we say that $\sim$ is \emph{compatible} with the group operation on $G$ iff, for all $a, a', g \in G$, if $a \sim a'$ then $ga \sim ga'$ and $ag \sim a'g$.
\end{df}

\begin{prop}
Let $G$ be a group. Let $\sim$ be an equivalence relation on $G$. Then there exists an operation $\cdot : (G/\sim)^2 \rightarrow G/sim$ such that
\[ \forall a,b \in G. [a][b] = [ab] \]
iff $\sim$ is compatible with the group operation on $G$. In this case, $G/\sim$ is a group under $\cdot$ and the canonical function $\pi : G \rightarrow G / \sim$ is a group homomorphism, and is universal with respect to group homomorphisms $\phi : G \rightarrow G'$ such that if $a \sim a'$ then $\phi(a) = \phi(a')$.
\end{prop}

\begin{proof}
\pf\ Easy. \qed
\end{proof}

\begin{df}[Quotient Group]
Let $G$ be a group. Let $\sim$ be an equivalence relation on $G$ that is compatible with the group operation on $G$. Then $G/\sim$ is the \emph{quotient group} of $G$ by $\sim$ under $[a][b] = [ab]$.
\end{df}

\begin{prop}
Let $G$ be a group and $H$ a subgroup of $G$. Then $H$ is normal if and only if there exists a group $K$ and homomorphism $\phi : G \rightarrow K$ such that $H = \ker \phi$.
\end{prop}

\begin{proof}
\pf\ One direction is given by Proposition \ref{prop:kernel-normal}. For the other direction, take $K = G / H$ and $\phi$ to be the canonical map $G \rightarrow G / H$. \qed
\end{proof}

\begin{df}[Modular Group]
The \emph{modular group} $\mathrm{PSL}_2(\mathbb{Z})$ is $\mathrm{SL}_2(\mathbb{Z}) / \{ I, -I \}$.
\end{df}

\begin{prop}
$\mathrm{PSL}_2(\mathbb{Z})$ is generated by $\left( \begin{array}{cc} 0 & -1 \\ 1 & 0 \end{array} \right)$ and $\left( \begin{array}{cc} 1 & -1 \\ 1 & 0 \end{array} \right)$.
\end{prop}

\begin{proof}
\pf\ By Example \ref{ex:SL2Z}. 
\end{proof}

\begin{prop}[Roger Alperin]
$\mathrm{PSL}_2(\mathbb{Z})$ is presented by $(x,y|x^2,y^3)$.
\end{prop}

\begin{proof}
\pf
\step{1}{\pflet{$x = \left( \begin{array}{cc} 0 & -1 \\ 1 & 0 \end{array} \right)$}}
\step{2}{\pflet{$y = \left( \begin{array}{cc} 1 & -1 \\ 1 & 0 \end{array} \right)$}}
\step{3}{Define an action of $\mathrm{PSL}_2(\mathbb{Z})$ on $\mathbb{R} - \mathbb{Q}$ by
\[ \left( \begin{array}{cc}
a & b \\ c & d
\end{array} \right) r = \frac{ar+b}{cr+d} \enspace . \]}
\begin{proof}
	\step{a}{Given $\left( \begin{array}{cc}
a & b \\ c & d
\end{array} \right) \in \mathrm{PSL}_2(\mathbb{Z})$ and $r$ irrational we have $\frac{ar+b}{cr+d}$ is irrational.}
	\begin{proof}
		\step{i}{\assume{for a contradiction $\frac{ar+b}{cr+d} = \frac{p}{q}$ where $p$ and $q$ are integers with $q > 0$.}}
		\step{ii}{$aqr + bq = cpr + dp$}
		\step{iii}{$(aq-cp)r = dp-bq$}
		\step{iv}{$aq=cp = dp-bq = 0$}
		\step{v}{$adq - cdp = 0$}
		\step{vi}{$cdp - cbq = 0$}
		\step{vii}{$(ad-cb)q = 0$}
		\begin{proof}
			\pf\ Since $ad - cb = 1$.
		\end{proof}
		\step{viii}{$q = 0$}
		\qedstep
		\begin{proof}
			\pf\ This contradicts \stepref{i}.
		\end{proof}
	\end{proof}
	\step{b}{$-Ir = r$}
	\begin{proof}
		\pf\ Since $-Ir = \frac{-r}{-1} = r$.
	\end{proof}
	\step{b}{Given $A,B \in \mathrm{PSL}_2(\mathbb{Z})$ we have $A(Br) = (AB)r$.}
	\begin{proof}
		\pf
		\begin{align*}
		\left( \begin{array}{cc}
		a & b \\ c & d
		\end{array} \right)
		\left[
		\left( \begin{array}{cc}
		e & f \\ g & h
		\end{array} \right) r \right]
		& = \left( \begin{array}{cc}
		a & b \\ c & d
		\end{array} \right)
		\frac{er+f}{gr+h} \\
		& = \frac{a\frac{er+f}{gr+h}+b}{c\frac{er+f}{gr+h}+d} \\
		& = \frac{a(er+f)+b(gr+h)}{c(er+f)+d(gr+h)} \\
		& = \frac{(ae+bg)r+(af+bh)}{(ce+dg)r+(cf+dh)} \\
		& = \left( \begin{array}{cc}
		ae+bg & af+bh \\ ce+dg & cf+dh
		\end{array} \right) r \\
		& = \left[ 
		\left( 
		\begin{array}{cc}
		a & b \\ c & d
		\end{array} 
		\right)
		\left( \begin{array}{cc}
		e & f \\ g & h
		\end{array} \right) 
		\right] 
		r
		\end{align*}
	\end{proof}
\end{proof}
\step{4}{\[ yr = 1 - \frac{1}{r} \]}
\step{5}{\[ \inv{y} r = \frac{1}{1-r} \]}
\begin{proof}
	\pf\ Since $\inv{y} = \left( \begin{array}{cc}
	0 & 1 \\ -1 & 1
	\end{array} \right)$
\end{proof}
\step{6}{\[ yxr = 1+r \]}
\begin{proof}
	\pf\ Since $yx = \left( \begin{array}{cc}
	-1 & -1 \\
	0 & -1
	\end{array} \right)$.
\end{proof}
\step{7}{\[ \inv{y}xr = \frac{r}{1+r} \]}
\begin{proof}
	\pf\ Since $\inv{y}x = 
	\left( \begin{array}{cc}
	1 & 0 \\ 1 & 1
	\end{array} \right)
	$.
\end{proof}
\step{8}{If $r > -1$ is positive then $yxr$ is positive.}
\step{9}{If $r$ is positive then $\inv{y}xr$ is positive.}
\step{19}{If $r < -1$ then $\inv{y}xr$ is positive.}
\step{10}{If $r$ is negative then $yr$ is positive.}
\step{11}{If $r$ is negative then $\inv{y}r$ is positive.}
\step{3}{No product of the form
\[ (y^{\pm 1}x)(y^{\pm 1}x) \cdots (y^{\pm 1}x) \]
with one or more factors can equal the identity.}
\begin{proof}
	\pf\ If the last factor is $(yx)$, then the product maps numbers in $(-1,0)$ to positive numbers. If the last factor is $(\inv{y}x)$, then the product maps numbers $< -1$ to positive numbers.
\end{proof}
\step{4}{No product of the form
\[ (y^{\pm 1}x)(y^{\pm 1}x) \cdots (y^{\pm 1}x)y^{\pm 1} \]
with one or more factors can equal the identity.}
\begin{proof}
	\pf\ The product maps negative numbers to positive numbers.
\end{proof}
\step{5}{$\mathrm{PSL}_2(\mathbb{Z})$ is presented by $(x,y|x^2,y^3)$.}
\qed
\end{proof}

\begin{cor}
$\mathrm{PSL}_2(\mathbb{Z})$ is the coproduct of $C_2$ and $C_3$ in $\Grp$.
\end{cor}

\begin{thm}
Every group homomorphism $\phi : G \rightarrow H$ may be decomposed as
\[ \begin{tikzcd}
G \arrow[r] & G / \ker \phi \arrow[r,"\cong"] & \im \phi \arrow[r] & H 
\end{tikzcd} \]
\end{thm}

\begin{proof}
\pf\ Easy. \qed
\end{proof}

\begin{cor}[First Isomorphism Theorem]
Let $\phi : G \rightarrow H$ be a surjective group homomorphism. Then $H \cong G / \ker \phi$.
\end{cor}

\begin{prop}
Let $H_1$ be a normal subgroup of $G_1$ and $H_2$ a normal subgroup of $G_2$. Then $H_1 \times H_2$ is a normal subgroup of $G_1 \times G_2$, and
\[ \frac{G_1 \times G_2}{H_1 \times H_2} \cong \frac{G_1}{H_1} \times \frac{G_2}{H_2} \enspace . \]
\end{prop}

\begin{proof}
\pf\ $\pi \times \pi : G_1 \times G_2 \twoheadrightarrow G_1 / H_1 \times G_2 / H_2$ is a surjective homomorphism with kernel $H_1 \times H_2$. \qed
\end{proof}

\begin{ex}
\[ \mathbb{R} / \mathbb{Z} \cong S^1 \]
\end{ex}

\begin{proof}
\pf\ Map a real number $r$ to $(\cos r, \sin r)$. The result is a surjective group homomorphism with kernel $\mathbb{Z}$. \qed
\end{proof}

\begin{prop}
Let $H$ be a normal subgroup of a group $G$. For every subgroup $K$ of $G$ that includes $H$, we have $H$ is a normal subgroup of $K$, and $K/H$ is a subgroup of $G/H$. The mapping
\[ u : \{ \text{subgroups of } G \text{ including } H \} \rightarrow \{ \text{subgroups of } G / H \} \]
with $u(K) = K/H$ is a poset isomorphism.
\end{prop}

\begin{proof}
\pf
\step{1}{If $K$ is a subgroup of $G$ that includes $H$ then $H$ is normal in $K$.}
\step{2}{If $K$ is a subgroup of $G$ that includes $H$ then $K/H$ is a subgroup of $G/H$.}
\step{3}{If $H \subseteq K_1 \subseteq K_2$ then $K_1/H \subseteq K_2/H$.}
\step{4}{If $K_1/H = K_2/H$ then $K_1 = K_2$}
\begin{proof}
	\step{a}{\assume{$K_1 / H = K_2 / H$}}
	\step{b}{$K_1 \subseteq K_2$}
	\begin{proof}
		\step{i}{\pflet{$k \in K_1$}}
		\step{ii}{$kH \in K_2 / H$}
		\step{iii}{\pick\ $k' \in K_2$ such that $kH = k'H$}
		\step{iv}{$k\inv{k'} \in H$}
		\step{v}{$k \inv{k'} \in K_2$}
		\step{v}{$k \in K_2$}
	\end{proof}
	\step{c}{$K_2 \subseteq K_1$}
	\begin{proof}
		\pf\ Similar.
	\end{proof}
\end{proof}
\step{5}{For any subgroup $L$ of $G/H$, there exists a subgroup $K$ of $G$ that includes $H$ such that $L = K / H$.}
\begin{proof}
	\step{a}{\pflet{$L$ be a subgroup of $G / H$.}}
	\step{b}{\pflet{$K = \{ k \in G : kH \in L \}$}}
	\step{c}{$K$ is a subgroup of $G$.}
	\begin{proof}
		\pf\ Given $k,k' \in K$ we have $kH,k'H \in L$ hence $k\inv{k'}H \in L$ and so $k\inv{k'} \in K$.
	\end{proof}
	\step{d}{$H \subseteq K$}
	\begin{proof}
		\pf\ For all $h \in H$ we have $hH = H \in L$.
	\end{proof}
	\step{e}{$L = K / H$}
	\begin{proof}
		\pf\ By definition.
	\end{proof}
\end{proof}
\qed
\end{proof}

\begin{prop}[Third Isomorphism Theorem]
Let $H$ be a normal subgroup of a group $G$. Let $N$ be a subgroup of $G$ that includes $H$. Then $N / H$ is normal in $G/H$ if and only if $N$ is normal in $G$, in which case
\[ \frac{G/H}{N/H} \cong \frac{G}{N} \]
\end{prop}

\begin{proof}
\pf
\step{1}{If $N/H$ is normal in $G/H$ then $N$ is normal in $G$.}
\begin{proof}
	\step{a}{\assume{$N/H$ is normal in $G/H$.}}
	\step{b}{\pflet{$g \in G$ and $n \in N$.}}
	\step{c}{$gn\inv{g}H \in N/H$}
	\step{d}{\pick\ $n' \in N$ such that $gn\inv{g}H = n'H$}
	\step{e}{$gn\inv{g}\inv{n'} \in H$}
	\step{f}{$gn\inv{g}\inv{n'} \in N$}
	\step{g}{$gn\inv{g} \in N$}
\end{proof}
\step{2}{If $N$ is normal in $G$ then $N/H$ is normal in $G/H$ and $(G/H)/(N/H) \cong G/N$.}
\begin{proof}
	\step{a}{\assume{$N$ is normal in $G$.}}
	\step{b}{\pflet{$\phi : G/H \rightarrow G/N$ be the homomorphism $\phi(gH) = gN$}}
	\begin{proof}
		\step{i}{If $gH = g'H$ then $gN = g'N$}
		\begin{proof}
			\pf\ If $g \inv{g'} \in H$ then $g \inv{g'} \in N$.
		\end{proof}
		\step{ii}{$\phi((gH)(g'H)) = \phi(gH)\phi(g'H)$}
		\begin{proof}
			\pf\ Both are $gg'N$.
		\end{proof}
	\end{proof}
	\step{c}{$\phi$ is surjective.}
	\step{d}{$\ker \phi = N/H$}
	\step{e}{$(G/H)/(N/H) \cong G/N$}
	\begin{proof}
		\pf\ By the First Isomorphism Theorem.
	\end{proof}
\end{proof}
\qed
\end{proof}

\begin{prop}[Second Isomorphism Theorem]
Let $H$ and $K$ be subgroups of a group $G$. Assume that $H$ is normal in $G$. Then:
\begin{enumerate}
\item $HK$ is a subgroup of $G$, and $H$ is normal in $HK$.
\item $H \cap K$ is normal in $K$, and
\[ \frac{HK}{H} \cong \frac{K}{H \cap K} \enspace . \]
\end{enumerate}
\end{prop}

\begin{proof}
\pf
\step{1}{$HK$ is a subgroup of $G$.}
\begin{proof}
	\pf\ Since $hkh'k' = hh'(\inv{h'}kh')k' \in HK$.
\end{proof}
\step{2}{$H$ is normal in $HK$.}
\step{3}{$H \cap K$ is normal in $K$ and $HK/H \cong K/(H \cap K)$}
\begin{proof}
\pf\ The function that maps $k$ to $kH$ is a surjective homomorphism $K \twoheadrightarrow HK/H$ with kernel $H \cap K$. Surjectivity follows because $hkH = hk\inv{h}H$.
\end{proof}
\qed
\end{proof}

See also Proposition \ref{prop:order-of-HK} for a result that holds even if $H$ is not normal.

\section{Cosets}

\begin{prop}
\label{prop:sim-gives-H}
Let $G$ be a group. Let $\sim$ be an equivalence relation on $G$ such that, for all $a,b,g \in G$, if $a \sim b$ then $ga \sim gb$. Let $H = \{ h \in G : h \sim e \}$. Then $H$ is a subgroup of $G$ and, for all $a,b \in G$, we have
\[ a \sim b \Leftrightarrow \inv{a} b \in H \Leftrightarrow aH = bH \enspace . \]
\end{prop}

\begin{proof}
\pf
\step{0}{$e \in H$}
\step{1}{For all $x,y \in H$ we have $x \inv{y} \in H$.}
\begin{proof}
	\step{a}{\assume{$x \sim e$ and $y \sim e$.}}
	\step{b}{$e \sim \inv{y}$}
	\begin{proof}
		\pf\ Since $y \inv{y} \sim e \inv{y}$.
	\end{proof}
	\step{c}{$x \inv{y} \sim e$}
	\begin{proof}
		\pf\ Since $x \inv{y} \sim e \inv{y} \sim e$.
	\end{proof}
\end{proof}
\step{2}{If $a \sim b$ then $\inv{a} b \in H$.}
\begin{proof}
	\pf\ If $a \sim b$ then $\inv{a} b \sim \inv{a} a = e$.
\end{proof}
\step{3}{If $\inv{a}b \in H$ then $aH = bH$.}
\begin{proof}
	\step{a}{\assume{$\inv{a}b \in H$}}
	\step{c}{$bH \subseteq aH$}
	\begin{proof}
		\pf\ For any $h \in H$ we have $bh = a \inv{a} b h \in a H$.	
	\end{proof}
	\step{b}{$aH \subseteq bH$}
	\begin{proof}
		\pf\ Similar since $\inv{b}a \in H$.
	\end{proof}
\end{proof}
\step{4}{If $aH = bH$ then $a \sim b$.}
\begin{proof}
	\step{a}{\assume{$aH = bH$}}
	\step{b}{\pick\ $h \in H$ such that $a = bh$.}
	\step{c}{$\inv{b} a = h$}
	\step{d}{$\inv{b} a \in H$}
	\step{e}{$\inv{b} a \sim e$}
	\step{f}{$a \sim b$}
	\begin{proof}
		\pf\ $a = b \inv{b} a \sim b e = b$.
	\end{proof}
\end{proof}
\qed
\end{proof}

\begin{df}[Coset]
Let $G$ be a group and $H$ a subgroup of $G$. A \emph{left coset} of $H$ is a set of the form $aH$ for $a \in G$. A \emph{right coset} of $H$ is a set of the form $Ha$ for some $a \in G$.

We write $G / H$ for the set of all left cosets of $H$, and $G \backslash H$ for the set of all right cosets of $H$.
\end{df}

\begin{prop}
\[ G / H \cong G \backslash H \]
\end{prop}

\begin{proof}
\pf\ The function that maps $aH$ to $H \inv{a}$ is a bijection. \qed
\end{proof}

\begin{prop}
Let $G$ be a group and $H$ a subgroup of $G$. Define $\sim_H$ on $G$ by: $a\sim b$ iff $\inv{a} b \in H$. This defines a one-to-one correspondence between the subgroups of $G$ and the equivalence relations $\sim$ on $G$ such that, for all $a,b,g \in G$, if $a \sim b$, then $ga \sim gb$. The equivalence class of $a$ is $aH$.
\end{prop}

\begin{proof}
\pf
\step{1}{For any subgroup $H$, we have $\sim_H$ is an equivalence relation on $G$.}
\begin{proof}
	\step{1}{$\sim$ is reflexive.}
	\begin{proof}
		\pf\ For any $a \in G$ we have $\inv{a} a = e \in H$.
	\end{proof}
	\step{2}{$\sim$ is symmetric.}
	\begin{proof}
		\pf\ If $\inv{a} b \in H$ then $\inv{b} a \in H$.
	\end{proof}
	\step{3}{$\sim$ is transitive.}
	\begin{proof}
		\pf\ If $\inv{a} b \in H$ and $\inv{b} c \in H$ then $\inv{a} c = (\inv{a} b) (\inv{b} c) \in H$.
	\end{proof}
\end{proof}
\step{4}{If $a \sim_H b$ then $ga \sim_H gb$.}
\begin{proof}
	\pf\ If $\inv{a} b \in H$ then $\inv{(ga)}(gb) = \inv{a} \inv{g} g b = \inv{a} b \in H$.
\end{proof}
\step{5}{For any equivalence relation $\sim$ on $G$ such that, whenever $a \sim b$, then $ga \sim gb$, there exists a subgroup $H$ such that $\sim = \sim_H$.}
\begin{proof}
	\pf\ Proposition \ref{prop:sim-gives-H}.
\end{proof}
\step{6}{The $\sim_H$-equivalence class of $a$ is $aH$.}
\begin{proof}
	\pf
	\begin{align*}
		a \sim b & \Leftrightarrow \inv{a} b \in H \\
		& \Leftrightarrow \exists h \in H. \inv{a} b = h \\
		& \Leftrightarrow \exists h \in H. b = aH \\
		& \Leftrightarrow b \in aH
	\end{align*}
\end{proof}
\qed
\end{proof}

\begin{prop}
Let $G$ be a group and $H$ a subgroup of $G$. Define $\sim_H$ on $G$ by: $a\sim b$ iff $a \inv{b} \in H$. This defines a one-to-one correspondence between the subgroups of $G$ and the equivalence relations $\sim$ on $G$ such that, for all $a,b,g \in G$, if $a \sim b$, then $ag \sim bg$. The equivalence class of $a$ is $Ha$.
\end{prop}

\begin{proof}
\pf\ Similar. \qed
\end{proof}

\begin{prop}
Let $G$ be a group and $H$ be a subgroup of $G$. Define $\sim_L$ and $\sim_R$ on $G$ by:
\[ a \sim_L b \Leftrightarrow \inv{a} b \in H, \qquad a \sim_R b \Leftrightarrow a \inv{b} \in H \enspace . \]
Then $\sim_L = \sim_R$ if and only if $H$ is normal.
\end{prop}

\begin{proof}
\pf
\step{1}{If $\sim_L = \sim_R$ then $H$ is normal.}
\begin{proof}
	\step{a}{\assume{$\sim_L = \sim_R$}}
	\step{b}{\pflet{$h \in H$ and $g \in G$}}
	\step{c}{$g \sim_L g\inv{h}$}
	\step{d}{$g \sim_R g \inv{h} h$}
	\step{e}{$gh\inv{g} \in H$}
\end{proof}
\step{2}{If $H$ is normal then $\sim_L = \sim_R$.}
\begin{proof}
	\step{a}{\assume{$H$ is normal.}}
	\step{b}{If $a \sim_L b$ then $a \sim_R b$.}
	\begin{proof}
		\step{i}{\assume{$a \sim_L b$}}
		\step{ii}{$\inv{a} b \in H$}
		\step{iii}{$a \inv{a} b \inv{a} \in H$}
		\step{iv}{$b \inv{a} \in H$}
		\step{v}{$a \sim_R b$}
	\end{proof}
	\step{c}{If $a \sim_R b$ then $a \sim_L b$.}
	\begin{proof}
		\pf\ Similar.
	\end{proof}
\end{proof}
\qed
\end{proof}

\begin{cor}
Let $G$ be a group and $H$ be a normal subgroup of $G$. Define $\sim$ on $G$ by $a \sim b$ iff $\inv{a} b \in H$. Then $G / \sim$ is a group under $[a][b] = [ab]$.
\end{cor}

\begin{df}[Quotient Group]
Let $G$ be a group and $H$ be a normal subgroup of $G$. The \emph{quotient group} $G / H$ is $G / \sim$ where $a \sim b$ iff $\inv{a} b \in H$, under $[a][b] = [ab]$ or $(aH)(bH) = abH$.
\end{df}

\begin{cor}
Let $H$ be a normal subgroup of a group $G$. For every group homomorphism $\phi : G \rightarrow G'$ such that $H \subseteq \ker \phi$, there exists a unique group homomorphism $\overline{\phi} : G / H \rightarrow G'$ such that the following diagram commutes.
\[ \begin{tikzcd}
G \arrow[rr,"\phi"]  \arrow[dr,"\pi"] & & G' \\
& G/ H\arrow[ur,"\overline{\phi}"]
\end{tikzcd} \]
\end{cor}

\begin{prop}
    $\mathbb{Z} / n \mathbb{Z}$ has exactly $n$ elements.
\end{prop}

\begin{proof}
    \pf\ Every integer is congruent to one of 0, 1, \ldots, $n - 1$ by the division algorithm, and no two of them are conguent to one another, since if $0 \leq i < j < n$ then $0 < j - i < n$. \qed
\end{proof}

\begin{prop}
    \label{prop:order-of-m-in-ZnZ}
    Let $m$ and $n$ be integers with $n > 0$. The order of $m$ in $\mathbb{Z} / n \mathbb{Z}$ is $\frac{n}{\gcd(m,n)}$.
\end{prop}

\begin{proof}
    \pf\ By Proposition \ref{prop:order-of-g-to-the-m} since the order of 1 is $n$. \qed
\end{proof}

\begin{prop}
    The integer $m$ generates $\mathbb{Z} / n \mathbb{Z}$ if and only if $\gcd(m,n) = 1$.
\end{prop}

\begin{proof}
    \pf\ By Proposition \ref{prop:order-of-m-in-ZnZ}. \qed
\end{proof}

\begin{cor}
    If $p$ is prime then every non-zero element in $\mathbb{Z} / p \mathbb{Z}$ is a generator.
\end{cor}

\begin{prop}
    \[ \Aut{\Grp}{\mathbb{Z} / 2 \mathbb{Z} \times \mathbb{Z} / 2 \mathbb{Z}} \cong S_3 \]
\end{prop}

\begin{proof}
    \pf\ Every permutation of $\{ (1,0), (0,1), (1,1) \}$ gives an automorphism of $\mathbb{Z} / 2 \mathbb{Z} \times \mathbb{Z} / 2 \mathbb{Z}$. \qed
\end{proof}

\begin{ex}
Not all monomorphisms split in $\Grp$.

Define $\phi : \mathbb{Z} / 3 \mathbb{Z} \rightarrow S_3$ by
\[ \phi(0) = \id{3}, \qquad \phi(1) = (1 \ 3 \ 2), \qquad \phi(2) = (1 \ 2 \ 3) \enspace . \]
Then $\phi$ is monic but has no retraction.

For if $r : S_3 \rightarrow \mathbb{Z} / 3 \mathbb{Z}$ is a retraction, then we would have
\[ r(1\ 2) + r(2\ 3) = 1, \qquad r(2\ 3) + r(1\ 2) = 2 \]
which is impossible.
\end{ex}

\begin{prop}
Let $G$ be a group, $H$ a subgroup of $G$, and $g \in G$. The function that maps $h$ to $gh$ is a bijection $H \cong gH$.
\end{prop}

\begin{proof}
\pf\ By Cancellation. \qed
\end{proof}

\begin{prop}
Let $G$ be a group, $H$ a subgroup of $G$, and $g \in G$. The function that maps $h$ to $hg$ is a bijection $H \cong Hg$.
\end{prop}

\begin{proof}
\pf\ By Cancellation. \qed
\end{proof}

\begin{prop}
\label{prop:order-of-HK}
Let $H$ and $K$ be finite subgroups of a group $G$. Then
\[ |HK| = \frac{|H||K|}{|H \cap K|} \enspace . \]
\end{prop}

\begin{proof}
\pf
\step{1}{\pflet{$f : \{ hK : h \in H \} \rightarrow H/(H \cap K)$ be the function $f(hK) = h(H \cap K)$}}
\begin{proof}
	\pf\ This is well-defined because if $hK = h'K$ then $\inv{h} h' \in H \cap K$ so $h(H \cap K) = h'(H \cap K)$.
\end{proof}
\step{2}{$f$ is injective.}
\begin{proof}
	\pf\ If $h(H \cap K) = h'(H \cap K)$ then $hK = h'K$.
\end{proof}
\step{3}{$f$ is surjective.}
\begin{proof}
	\pf\ Clear.
\end{proof}
\step{4}{\[ \frac{|HK|}{|K|} = \frac{|H|}{|H \cap K|} \]}
\qed
\end{proof}

\section{Congruence}

\begin{df}[Congruence]
Given integers $a$, $b$, $n$ with $n$ positive, we say $a$ is \emph{congruent} to $b$ \emph{modulo} $n$, and write $a \equiv b (\mod n)$, iff $a + n \mathbb{Z} = b + n \mathbb{Z}$ in $\mathbb{Z} / n \mathbb{Z}$.
\end{df}

\begin{prop}
Given integers $a$, $b$, $n$ with $n$ positive, we have $a \equiv b (\mod n)$ iff $n \mid a - b$.
\end{prop}

\begin{proof}
\pf\ By Proposition \ref{prop:sim-gives-H}. \qed
\end{proof}

\begin{prop}
    If $a \equiv a' \mod n$ and $b \equiv b' \mod n$ then $a + b \equiv a' + b' \mod n$.
\end{prop}

\begin{proof}
    \pf\ If $n \mid a' - a$ and $n \mid b' - b$ then $n \mid (a' + b') - (a + b)$. \qed
\end{proof}

\begin{prop}
    If $a \equiv a' \mod n$ and $b \equiv b' \mod n$ then $ab \equiv a'b' \mod n$.
\end{prop}

\begin{proof}
    \pf\ If $n \mid a' - a$ and $n \mid b' - b$ then $n \mid a'b' - ab = a'(b'-b) + (a'-a)b$. \qed
\end{proof}

\section{Cyclic Groups}

\begin{df}[Cyclic Group]
    The \emph{cyclic} groups are $\mathbb{Z}$ and $\mathbb{Z} / n \mathbb{Z}$ for positive integers $n$.
\end{df}

\begin{prop}
    If $m$ and $n$ are positive integers with $\gcd(m,n) = 1$ then $C_{mn} \cong C_m \times C_n$.
\end{prop}

\begin{proof}
    \pf\ The function that maps $x$ to $(x \mod m, x \mod n)$ is an isomorphism. \qed
\end{proof}

\begin{prop}
Let $G$ be a group and $g \in G$. Then $\langle g \rangle$ is cyclic.
\end{prop}

\begin{proof}
\pf\ If $g$ has finite order then $\langle g \rangle \cong C_{|g|}$, otherwise $\langle g \rangle \cong \mathbb{Z}$. \qed
\end{proof}

\begin{prop}
Every finitely generated subgroup of $\mathbb{Q}$ is cyclic.
\end{prop}

\begin{proof}
\pf
\step{1}{\pflet{$G = \langle a_1/b, \ldots, a_n/b \rangle$ where $a_1$, \ldots, $a_n$, $b$ are integers with $b > 0$}}
\step{2}{\pflet{$a = \gcd(a_1, \ldots, a_n)$}}
\step{3}{$G = \langle a/b \rangle$}
\qed
\end{proof}

\begin{cor}
$\mathbb{Q}$ is not finitely generated.
\end{cor}

\section{Commutator Subgroup}

\begin{df}[Commutator]
Let $G$ be a group and $g,h \in G$. The \emph{commutator} of $g$ and $h$ is
\[ [g,h] = gh\inv{g} \inv{h} \enspace . \]
\end{df}

\begin{df}[Commutator Subgroup]
Let $G$ be a group. The \emph{commutator subgroup}, denoted $[G,G]$ or $G'$, is the subgroup generated by the elements of the form $ab\inv{a}\inv{b}$.
 
We write $G^{(i)}$ for the result of taking the commutator subgroup $i$ times starting with $G$.
\end{df}

\begin{lm}
\label{lm:commutator-characteristic}
Let $\phi : G_1 \rightarrow G_2$ be a group homomorphism. Then, for all $g,h \in G_1$, we have
\[ \phi([g,h]) = [\phi(g), \phi(h)] \]
and so $\phi(G_1') \subseteq G_2'$.
\end{lm}

\begin{proof}
\pf\ Easy. \qed
\end{proof}

\begin{lm}
Let $N$ and $H$ be normal subgroups of a group $G$. Then $[N,H] \subseteq N \cap H$.
\end{lm}

\begin{proof}
\pf
\step{1}{\pflet{$n \in N$ and $h \in H$} \prove{$nh\inv{n}\inv{h} \in N \cap H$}}
\step{2}{$nh\inv{n} \in H$}
\begin{proof}
	\pf\ Since $H$ is normal.
\end{proof}
\step{3}{$n h \inv{n} \inv{h} \in H$}
\step{4}{$h \inv{n} \inv{h} \in N$}
\begin{proof}
	\pf\ Since $N$ is normal.
\end{proof}
\step{5}{$nh\inv{n}\inv{h} \in N$}
\step{6}{$nh\inv{n}\inv{h} \in N \cap H$}
\qed
\end{proof}

\begin{cor}
\label{cor:N-H-commute}
Let $N$ and $H$ be normal subgroups of $G$. If $N \cap H = \{e\}$, then every element in $N$ commutes with every element in $H$.
\end{cor}

\begin{prop}
\label{prop:NH-cong-N-times-H}
Let $N$ and $H$ be normal subgroups of $G$. If $N \cap H = \{e\}$ then $NH \cong N \times H$.
\end{prop}

\begin{proof}
\pf
\step{1}{\pflet{$\phi : N \times H \rightarrow NH$ be the function $\phi(n,h) = nh$.}}
\step{2}{$\phi$ is a homomorphism.}
\begin{proof}
	\pf
	\begin{align*}
		\phi((n,h)(n',h')) & = \phi(nn',hh') \\
		& = nn'hh' \\
		& = nhn'h' & (\text{Corollary \ref{cor:N-H-commute}}) \\
		& = \phi(n,h)\phi(n',h')
	\end{align*}
\end{proof}
\step{3}{$\ker \phi = \{(e,e)\}$}
\begin{proof}
	\step{a}{\pflet{$(n,h) \in \ker \phi$}}
	\step{b}{$nh = e$}
	\step{c}{$n = \inv{h}$}
	\step{d}{$n \in N \cap H$}
	\step{e}{$n = e$}
	\step{f}{$h = e$}
	\begin{proof}
		\pf\ By \stepref{c}.
	\end{proof}
\end{proof}
\step{4}{$\phi : N \times H \cong NH$}
\qed
\end{proof}

\section{Presentations}

\begin{df}[Presentation]
A \emph{presentation} of a group $G$ is a pair $(A,R)$ where $A$ is a set and $R \subseteq F(A)$ is a set of words such that
\[ G \cong F(A) / N(R) \]
where $N(R)$ is the smallest normal subgroup of $F(A)$ that includes $R$.
\end{df}

\begin{ex}
\begin{itemize}
\item The free group on a set $A$ is presented by $(A, \emptyset)$.
\item $S_3$ is presented by $(x,y|x^2,y^3,xyxy)$.
\item 
$(a,b \mid a^2, b^2, (ab)^n)$ is a presentation of $D_{2n}$.
\item $(x,y \mid x^2 y^{-2}, y^4, xyx^{-1}y)$ is a presentation of $Q_8$.
\end{itemize}
\end{ex}

\begin{prop}[Word Problem]
Let $(A,R)$ be a presentation of the group $G$. Let $w_1, w_2 \in F(A)$ be two words. Then it is undecidable in general if $w_1N(R) = w_2N(R)$ in $G$.
\end{prop}

%TODO
\begin{df}[Finitely Presented]
A group is \emph{finitely presented} iff it has a presentation $(A,R)$ where both $A$ and $R$ are finite.
\end{df}

\begin{prop}
Let $(A|R)$ be a presentation of $G$ and $(A'|R')$ a presentation of $H$. Assume w.l.o.g. $A$ and $A'$ are disjoint. Then the group $G * G'$ presented by $(A \cup A' | R \cup R')$ is the coproduct of $G$ and $G'$ in $\mathbf{Grp}$.
\end{prop}

\[ \begin{tikzcd}
A \arrow[d] \arrow[r] & A \cup A' \arrow[d] & A' \arrow[l] \arrow[d] \\
 F(A) \arrow[d] \arrow[r] & F(A \cup A') \arrow[d] & F(A') \arrow[l] \arrow[d] \\
G \arrow[r, "\kappa_1"] & G * G' & G' \arrow[l, "\kappa_2"]
\end{tikzcd} \]

\begin{proof}
\pf
\step{1}{\pflet{$\kappa_1 : G \rightarrow G * G'$ and $\kappa_2 : G' \rightarrow G * G'$ be the unique homomorphisms that make the diagram above commute.}}
\step{2}{\pflet{$\phi : G \rightarrow H$ and $\psi : G' \rightarrow H$ be any homomorphisms.}}
\step{3}{\pflet{$[\phi, \psi] : F(A \cup A') \rightarrow H$ be the unique homomorphism such that \ldots}}
\step{4}{$R \cup R' \subseteq \ker [\phi, \psi]$}
\step{5}{$[\phi, \psi]$ factors uniquely through the morphism $F(A \cup A') \rightarrow G * G'$}
\qed
\end{proof}

\section{Index of a Subgroup}

\begin{df}[Index]
Let $G$ be a group and $H$ a subgroup of $G$. The \emph{index} of $H$ in $G$, denoted $|G:H|$, is the number of left cosets of $H$ in $G$ if this is finite, otherwise $\infty$.
\end{df}

\begin{thm}[Lagrange's Theorem]
Let $G$ be a finite group and $H$ a subgroup of $G$. Then
\[ |G| = |G : H| |H| \enspace . \]
\end{thm}

\begin{proof}
\pf\ $G/H$ is a partition of $G$ into $|G:H|$ subsets, each of size $|H|$. \qed
\end{proof}

\begin{cor}
For $p$ a prime number, the only group of order $p$ is $C_p$.
\end{cor}

\begin{proof}
\pf\ Let $G$ be a group of order $p$ and $g \in G$ with $g \neq e$. Then $|\langle g \rangle|$ divides $p$ and is not 1, hence is $p$, that is, $G = \langle g \rangle$. \qed
\end{proof}

\begin{thm}[Cauchy's Theorem]
Let $G$ be a finite group. If $p$ is prime and $p \mid |G|$ then the number of cyclic subgroups of order $p$ is congruent to 1 modulo $p$. In particular, there exists an element of order $p$.
\end{thm}

\begin{proof}
\pf
\step{1}{\pflet{$S = \{ (a_1, a_2, \ldots, a_p) \in G^p : a_1 a_2 \cdots a_p = e \}$}}
\step{2}{$|S| = |G|^{p-1}$}
\begin{proof}
	\pf\ Given any $a_1, \ldots, a_{p-1} \in G$, there exists a unique $a_p$ such that $(a_1, \ldots, a_p) \in S$, namely $a_p = (a_1 \cdots a_{p-1})^{-1}$.
\end{proof}
\step{3}{$p \mid |S|$}
\step{4}{Define an action of $\mathbb{Z} / p \mathbb{Z}$ on $S$ by
\[ m \cdot (a_1, \ldots, a_p) = (a_m, a_{m+1}, \ldots, a_p, a_1, a_2, \ldots, a_{m-1}) \enspace . \]}
\begin{proof}
	\pf\ If $(a_1, \ldots, a_p) \in S$ then $(a_2, a_3, \ldots, a_p, a_1) \in S$ since $a_1 = (a_2 \cdots a_p)^{-1}$.
\end{proof}
\step{5}{\pflet{$Z$ be the set of fixed points of this action.}}
\step{6}{$|Z| \equiv 0 (\mod p)$}
\begin{proof}
	\pf\ Corollary \ref{cor:fixed-points-of-action-of-p-group}, \stepref{3}.
\end{proof}
\step{7}{$Z = \{(a, a, \ldots, a) : a^p = e \}$}
\step{8}{$Z \neq \emptyset$}
\begin{proof}
	\pf\ Since $(e, e, \ldots, e) \in Z$.
\end{proof}
\step{10}{An element $a$ has order $p$ iff $(a, a, \ldots, a) \in Z$ and $a \neq e$.}
\step{11}{\pflet{$N$ be the number of cyclic subgroups of order $p$.}}
\step{12}{The number of elements of order $p$ is $N(p-1)$}
\step{13}{$|Z| = N(p-1) + 1$}
\step{14}{$-N+1 \equiv 0 (\mod p)$}
\begin{proof}
	\pf\ From \stepref{6}.
\end{proof}
\step{15}{$N \equiv 1 (\mod p)$}
\qed
\end{proof}

\begin{prop}
Let $G$ be a group. Let $K$ be a subgroup of $G$ and $H$ a subgroup of $K$. If $|G:H|$, $|G:K|$ and $|K:H|$ are all finite then
\[ |G:H| = |G:K| |K:H| \enspace . \]
\end{prop}

\begin{proof}
\pf
\step{1}{\pflet{$G/K = \{ g_1 K, g_2 K, \ldots, g_m K \}$}}
\step{2}{\pflet{$K / H = \{ k_1 H, k_2 H, \ldots, k_n H \}$}}
\step{3}{$G/H = \{ g_i k_j H : 1 \leq i \leq m, 1 \leq j \leq n \}$}
\begin{proof}
	\step{a}{\pflet{$g \in G$}}
	\step{b}{\pick\ $i$ such that $gK = g_i K$}
	\step{c}{$\inv{g} g_i \in K$}
	\step{d}{\pick\ $j$ such that $\inv{g} g_i H = k_j H$}
	\step{e}{$\inv{g} g_i k_j \in H$}
	\step{f}{$gH = g_i k_j H$}
\end{proof}
\step{4}{If $g_i k_j H = g_{i'} k_{j'} H$ then $i = i'$ and $j = j'$.}
\begin{proof}
	\step{a}{\assume{$g_i k_j H = g_{i'} k_{j'} H$}}
	\step{b}{$g_i K = g_{i'} K$}
	\step{c}{$i = i'$}
	\step{d}{$k_j H = k_{j'} H$}
	\step{e}{$j = j'$}
\end{proof}
\qed
\end{proof}


\section{Cokernels}

\begin{prop}
Let $\phi : G \rightarrow H$ be a homomorphism between groups. Then there exists a group $K$ and homomorphism $\pi : H \rightarrow K$ that is initial with respect to all homomorphism $\alpha : H \rightarrow L$ such that $\alpha \circ \phi = 0$.
\end{prop}

\begin{proof}
\pf
\step{0}{\pflet{$N$ be the intersection of all the normal subgroups of $H$ that include $\im \phi$.}}
\step{1}{\pflet{$K = H / N$ and $\pi$ be the canonical homomorphism.}}
\step{2}{\pflet{$\pi \circ \phi = 0$}}
\step{3}{\pflet{$\alpha : H \rightarrow L$ satisfy $\alpha \circ \phi = 0$}}
\step{4}{$\im \phi \subseteq \ker \alpha$}
\step{4a}{$N \subseteq \ker \alpha$}
\step{5}{There exists a unique $\overline{\alpha} : H / \im \phi \rightarrow L$ such that $\overline{\alpha} \circ \pi = \alpha$}
\qed
\end{proof}

\begin{df}[Cokernel]
For any homomorphism $\phi : G \rightarrow H$ in $\mathbf{Grp}$, the \emph{cokernel} of $\phi$ is the group $\coker \phi$ and homomorphism $\pi : H \rightarrow \coker \phi$ that is initial among homomorphisms $\alpha : H \rightarrow L$ such that $\alpha \circ \phi = 0$.
\end{df}

\begin{ex}
It is not true that a homomorphism with trivial cokernel is epi. The inclusion $\langle (1\ 2) \rangle \hookrightarrow S_3$ has trivial cokernel but is not epi.
\end{ex}

\section{Cayley Graphs}

\begin{df}[Cayley Graph]
Let $G$ be a finitely generated group. Let $A$ be a finite set of generators for $G$. The \emph{Cayley graph} of $G$ with respect to $A$ is the directed graph whose vertices are the elements of $G$, with an edge $g_1 \rightarrow g_2$ labelled by $a \in A$ iff $g_2 = g_1 a$.
\end{df}

\begin{prop}
$G$ is the free group on $A$ iff the Cayley graph with respect to $A$ is a tree.
\end{prop}

\begin{proof}
\pf\ Both are equivalent to saying that the product of two different strings of elements of $A$ and/or their inverses are not equal. \qed
\end{proof}

\section{Characteristic Subgroups}

\begin{df}[Characteristic Subgroup]
Let $G$ be a group. Let $H$ be a subgroup of $G$. Then $H$ is a \emph{characteristic} subgroup of $G$ iff, for every automorphism $\phi$ of $G$, we have $\phi(H) \subseteq H$.
\end{df}

\begin{prop}
Characteristic subgroups are normal.
\end{prop}

\begin{proof}
\pf\ Take $\phi$ to be conjugation with respect to an arbitrary element. \qed
\end{proof}

\begin{prop}
Let $G$ be a group. Let $K$ be a normal subgroup of $G$ and $H$ a characteristic subgroup of $K$. Then $H$ is normal in $G$.
\end{prop}

\begin{proof}
\pf\ For any $a \in G$ we have conjugation by $a$ is an automorphism on $K$, hence $H$ is closed under it. \qed
\end{proof}

\begin{prop}
\label{prop:unique-subgroup-normal}
Let $G$ be a group. Let $H$ be a subgroup of $G$. Suppose there is no other subgroup of $G$ isomorphic to $H$. Then $H$ is characteristic, hence normal.
\end{prop}

\begin{proof}
\pf\ For any automorphism $\phi$ on $G$, we have $\phi(H)$ is isomorphic to $H$, hence $\phi(H) = H$. \qed
\end{proof}

\begin{prop}
\label{prop:K-GK-relatively-prime-K-characteristic}
Let $G$ be a finite group. Let $K$ be a normal subgroup of $G$. Assume $|K|$ and $|G/K|$ are relatively prime. Then $K$ is characteristic.
\end{prop}

\begin{proof}
\pf
\step{1}{\pflet{$K'$ be a subgroup of $G$ isomorphic to $K$.} \prove{$K' = K$}}
\step{2}{$|K' / (K \cap K')|$ divides both $|K'| = |K|$ and $|G/K|$}
\step{3}{$|K'/(K \cap K')| = 1$}
\step{4}{$K' = K \cap K'$}
\step{5}{$K' = K$}
\qed
\end{proof}

\begin{prop}
The commutator subgroup of a group is characteristic.
\end{prop}

\begin{proof}
\pf\ Lemma \ref{lm:commutator-characteristic}. \qed
\end{proof}

\section{Simple Groups}

\begin{df}[Simple Group]
A group $G$ is \emph{simple} iff its only normal subgroups are $\{e\}$ and $G$.
\end{df}

\begin{prop}
\label{prop:image_of_simple_group}
Let $G$ be a group. Then $G$ is simple if and only if the only homomorphic images of $G$ are $1$ and $G$.
\end{prop}

\begin{proof}
\pf\ Both are equivalent to saying that, for any surjective homomorphism $\phi : G \rightarrow G'$, either $\phi$ has kernel $\{e\}$ (in which case it is an isomorphism) or $\phi$ has kernel $G$ (in which case $G' = 1$.) \qed
\end{proof}

\section{Sylow Subgroups}

\begin{df}[Sylow Subgroup]
Let $p$ be a prime number. Let $G$ be a finite group. A \emph{$p$-Sylow subgroup} of $G$ is a subgroup of order $p^r$, where $r$ is the largest integer such that $p^r$ divides $|G|$.
\end{df}

\begin{prop}
Let $p$ be prime. Let $G$ be a finite group. Let $P$ be a $p$-Sylow subgroup of $G$. If $P$ is normal then $P$ is characteristic.
\end{prop}

\begin{proof}
\pf\ Proposition \ref{prop:K-GK-relatively-prime-K-characteristic}. \qed
\end{proof}

\begin{cor}
\label{cor:normal-normal-normal}
Let $p$ be prime. Let $G$ be a finite group. Let $P$ be a $p$-Sylow subgroup of $G$. Let $H$ be a subgroup of $G$ that includes $P$. If $P$ is normal in $H$ and $H$ is normal in $G$ then $P$ is normal in $G$.
\end{cor}

\begin{prop}
Let $G$ be a finite group. Let $P_1$, \ldots, $P_r$ be its nontrivial Sylow subgroups. Assume all $P_i$ are normal in $G$. Then
\[ G \cong P_1 \times \cdots \times P_r \enspace . \]
\end{prop}

\begin{proof}
\pf
\step{1}{$P_1 P_2 \cdots P_r \cong P_1 \times P_2 \times \cdots \times P_r$}
\begin{proof}
	\step{a}{$P_1 \cong P_1$}
	\step{b}{For $1 \leq i < r$, if $P_1 P_2 \cdots P_i \cong P_1 \times P_2 \times \cdots \times P_i$ then $P_1 P_2 \cdots P_i P_{i+1} \cong P_1 \times P_2 \times \cdots P_i \times P_{i+1}$}
	\begin{proof}
		\step{i}{\pflet{$1 \leq i < r$}}
		\step{ii}{\assume{$P_1 P_2 \cdots P_i \cong P_1 \times P_2 \times \cdots \times P_i$}}
		\step{iii}{$P_1 P_2 \cdots P_i$ is normal in $G$.}
		\step{iv}{$P_1 P_2 \cdots P_i \cap P_{i+1} = \{e\}$}
		\begin{proof}
			\step{one}{\pflet{$|P_j| = p_j^{k_j}$ for all $j$.}}
			\step{two}{The order of any element of $P_1 P_2 \cdots P_i$ divides $p_1^{k_1} p_2^{k_2} \cdots p_i^{k_i}$}
			\step{three}{The order of any element of $P_{i+1}$ divides $p_{i+1}^{k_{i+1}}$}
			\step{four}{The $p_j$ are all distinct.}
			\begin{proof}
				\pf\ Any $p_j$-Sylow subgroup is congruent to $P_j$ hence equal to $P_j$ since $P_j$ is normal.
			\end{proof}
			\step{five}{The only element in $P_1 P_2 \cdots P_i$ and $P_{i+1}$ is $e$.}
		\end{proof}
		\step{v}{$P_1 P_2 \cdots P_i P_{i+1} \cong P_1 P_2 \cdots P_i \times P_{i+1}$}
		\begin{proof}
			\pf\ Proposition \ref{prop:NH-cong-N-times-H}.
		\end{proof}
		\step{vi}{$P_1 P_2 \cdots P_i P_{i+1} \cong P_1 \times P_2 \times \cdots \times P_i \times P_{i+1}$}
	\end{proof}
\end{proof}
\step{2}{$G = P_1 P_2 \cdots P_r$}
\begin{proof}
	\pf\ Since $|G| = p_1^{k_1} p_2^{k_2} \cdots p_r^{k_r}$.
\end{proof}
\qed
\end{proof}

\section{Series of Subgroups}

\begin{df}[Series of Subgroups]
Let $G$ be a group. A \emph{series} of subgroups of $G$ is a sequence $(G_n)$ of subgroups of $G$ such that
\[ G = G_0 \supsetneq G_1 \supsetneq G_2 \supsetneq \cdots \]
It is a \emph{normal} series iff $G_{n+1}$ is normal in $G_n$ for all $n$.
\end{df}

\begin{prop}
The maximal length of a normal series in $G$ is 0 iff $G$ is trivial.
\end{prop}

\begin{proof}
\pf\ Since $1$ is normal in $G$ for every $G$. \qed
\end{proof}

\begin{prop}
The maximal length of a normal series in $G$ is 1 iff $G$ is non-trivial and simple.
\end{prop}

\begin{proof}
\pf\ Immediate from definitions. \qed
\end{proof}

\begin{ex}
$\mathbb{Z}$ has normal series of arbitrary length.
\end{ex}

\begin{proof}
\pf\ We have $\mathbb{Z} \supsetneq 2 \mathbb{Z} \supsetneq 4 \mathbb{Z} \supsetneq \cdots$. \qed
\end{proof}

\begin{ex}
The maximal length of a normal series in $\mathbb{Z} / n \mathbb{Z}$ is the number of primes in the prime factorization of $n$.
\end{ex}

\begin{proof}
\pf\ Let $n = p_1 p_2 \cdots p_k$. A normal series of maximal length is
\[ \mathbb{Z} / p_1 p_2 \cdots p_k \mathbb{Z} \supsetneq \mathbb{Z} / p_1 p_2 \cdots p_{k-1} \mathbb{Z} \supsetneq \cdots \supsetneq \mathbb{Z} / p_1 \mathbb{Z} \supsetneq \{e\} \enspace . \qed \]
\end{proof}

\begin{df}[Equivalent Normal Series]
Let
\begin{align*}
G & = G_0 \supsetneq G_1 \supsetneq G_2 \supsetneq \cdots \supsetneq G_n = \{e\} \\
G & = G'_0 \supsetneq G'_1 \supsetneq G'_2 \supsetneq \cdots \supsetneq G'_m = \{e\}
\end{align*}
be two normal series in a group $G$. Then the two series are \emph{equivalent} iff $m = n$ and there exists a permutation $\sigma \in S_n$ such that, for all $i$, we have $G_i / G_{i+1} \cong G'_{\sigma(i)} / G'_{\sigma(i) + 1}$.
\end{df}

\begin{df}[Composition Series]
Let $G$ be a group. A \emph{composition series} for $G$ is a series of subgroups in $G$
\[ G = G_0 \supsetneq G_1 \supsetneq G_2 \supsetneq \cdots \supsetneq G_n = \{e\} \]
such that, for all $i$, we have $G_i / G_{i+1}$ is simple.
\end{df}

\begin{prop}
A normal series of maximal length in a group is a composition series.
\end{prop}

\begin{proof}
\pf\ Easy. \qed
\end{proof}

\begin{cor}
Every finite group has a composition series.
\end{cor}

\begin{cor}
If a group has a composition series then every normal subgroup has a composition series.
\end{cor}

\begin{df}[Refinement]
A series of subgroups $S_1$ is a \emph{refinement} of the series $S_2$ iff every subgroup in $S_2$ appears in $S_1$.
\end{df}

\begin{lm}
\label{lm:Schreier}
Let $G$ be a group. Let $Q$, $N$ and $L$ be subgroups of $G$. Assume $L$ is a normal subgroup of $Q$ and $qN = Nq$ for all $q \in Q$. Then
\[ \frac{QN}{LN} \cong \frac{Q}{L(Q \cap N)} \enspace . \]
\end{lm}

\begin{proof}
\pf
\step{1}{$QN$ is a subgroup of $G$.}
\begin{proof}
	\pf\ Since $QN = NQ$.
\end{proof}
\step{2}{$LN$ is a subgroup of $G$.}
\begin{proof}
	\pf\ Since $LN = NL$.
\end{proof}
\step{3}{$LN$ is normal in $QN$.}
\begin{proof}
	\step{a}{\pflet{$l \in L$, $q \in Q$, and $n,n' \in N$.} \prove{$qnln'\inv{n}\inv{q} \in LN$}}
	\step{b}{\pick\ $n_1 \in N$ such that $nl = ln_1$}
	\step{c}{\pick\ $n_2 \in N$ such that $n_1n' \inv{n} \inv{q} = \inv{q} n_2$}
	\step{d}{$qnln' \inv{n} \inv{q} = ql\inv{q}n_2 \in LN$}
	\begin{proof}
		\pf\ Since $L$ is normal in $Q$.
	\end{proof}
\end{proof}
\step{4}{The function $f : Q \rightarrow QN/LN$ that maps $q$ to $qLN$ is a surjective homomorphism.}
\step{5}{$\ker f = L(Q \cap N)$}
\begin{proof}
	\step{a}{$\ker f \subseteq L (Q \cap N)$}
	\begin{proof}
		\step{i}{\pflet{$x \in \ker f$}}
		\step{ii}{$x \in LN$}
		\step{iii}{\pick\ $l \in L$ and $n \in N$ such that $x = ln$}
		\step{iv}{$n = \inv{l} x \in Q \cap N$}
		\step{v}{$x \in L(Q \cap N)$}
	\end{proof}
	\step{b}{$L(Q \cap N) \subseteq \ker f$}
	\begin{proof}
		\pf\ Since $L(Q \cap N) \subseteq Q$ and $L(Q \cap N) \subseteq LN$.
	\end{proof}
\end{proof}
\qedstep
\begin{proof}
	\pf\ First Isomorphism Theorem.
\end{proof}
\qed
\end{proof}

\begin{thm}[Schreier]
Any two normal series in a group have equivalent refinements.
\end{thm}

\begin{proof}
\pf
\step{1}{\pflet{$G$ be a group.}}
\step{2}{\pflet{$S_1: G = G_0 \supsetneq G_1 \supsetneq G_2 \supsetneq \cdots \supsetneq G_m = \{e\}$ and $S_2: G = H_0 \supsetneq H_1 \supsetneq H_2 \supsetneq \cdots \supsetneq H_n = \{e\}$ be two normal series in $G$.}}
\step{3}{For each $i$, we have
\[ G_i = G_i \cap H_0 \supseteq G_i \cap H_1 \supseteq \cdots \supseteq G_i \cap H_n = \{e\} \]
is a series of subgroups in $G_i$.}
\step{4}{For each $i$, we have
\[ G_i = (G_i \cap H_0) G_{i+1} \supseteq (G_i \cap H_1) G_{i+1} \supseteq \cdots \supseteq (G_i \cap H_n) G_{i+1} = G_{i+1} \]
is a normal series in $G_i$.}
\begin{proof}
	\step{a}{\pflet{$0 \leq i < m$ and $0 \leq j < n$} \prove{$(G_i \cap H_{j+1})G_{i+1}$ is normal in $(G_i \cap H_j)G_{i+1}$}}
	\step{b}{\pflet{$x \in G_i \cap H_{j+1}$, $y \in G_{i+1}$, $a \in G_i \cap H_j$ and $b \in G_{i+1}$} \prove{$abxy\inv{b}\inv{a} \in (G_i \cap H_{j+1}) G_{i+1}$}}
	\step{c}{$ax\inv{a} \in G_i \cap H_{j+1}$}
	\begin{proof}
		\pf\ Since $a,x \in G_i$ and $H_{j+1}$ is normal in $H_j$.
	\end{proof}
	\step{d}{$a\inv{x}bx\inv{a} \in G_{i+1}$}
	\begin{proof}
		\pf\ Since $G_{i+1}$ is normal in $G_i$.
	\end{proof}
	\step{e}{$y\inv{b} \in G_{i+1}$}
	\step{f}{$ay\inv{b}\inv{a} \in G_{i+1}$}
	\begin{proof}
		\pf\ Since $G_{i+1}$ is normal in $G_i$.
	\end{proof}
	\step{g}{$abxy\inv{b}\inv{a} = (ax\inv{a})(a\inv{x}bx\inv{a}ay\inv{b}\inv{a}) \in (G_i \cap H_{j+1})G_{i+1}$}
\end{proof}
\step{5}{Let $S$ be the series obtained by concatenating the series \stepref{4} for $G_0$ to $G_1$, $G_1$ to $G_2$, \ldots, $G_{m-1}$ to $G_m$}
\step{6}{$S$ is a refinement of $S_1$.}
\step{7}{$S$ is normal.}
\step{8}{\pflet{$T$ be the similarly constructed normal refinement of $S_2$.}}
\step{9}{For all $i$, $j$ we have
\[ \frac{(G_i \cap H_j) G_{i+1}}{(G_i \cap H_{j+1}) G_{i+1}} \cong \frac{G_i \cap H_j}{(G_i \cap H_{j+1})(G_{i+1} \cap H_j)} \]}
\begin{proof}
	\step{a}{$G_i \cap H_{j+1}$ is normal in $G_i \cap H_j$}
	\step{b}{For all $q \in G_i \cap H_j$ we have $qG_{i+1} = G_{i+1}q$}
	\begin{proof}
		\pf\ Since for all $q \in G_i$ we have $q G_{i+1} = G_{i+1}q$.
	\end{proof}
	\qedstep
	\begin{proof}
		\pf\ Lemma \ref{lm:Schreier}
	\end{proof}
\end{proof}
\step{10}{For all $i$, $j$ we have
\[ \frac{(G_i \cap H_j) H_{j+1}}{(G_{i+1} \cap H_j) H_{j+1}} \cong \frac{G_i \cap H_j}{(G_{i+1} \cap H_j)(G_i \cap H_{j+1})} \]}
\begin{proof}
	\pf\ Lemma \ref{lm:Schreier}
\end{proof}
\step{11}{For all $i$, $j$ we have
\[ \frac{(G_i \cap H_j) G_{i+1}}{(G_i \cap H_{j+1}) G_{i+1}} \cong \frac{(G_i \cap H_j) H_{j+1}}{(G_{i+1} \cap H_j) H_{j+1}} \]}
\step{12}{$S$ and $T$ are equivalent.}
\qed
\end{proof}

\begin{cor}[Jordan-H\"{o}lder]
Any two composition series for a group are equivalent.
\end{cor}

\begin{df}[Composition Factors]
Let $G$ be a group that has a composition series. The multiset of \emph{composition factors} of $G$ is the multiset of quotients of any composition series.
\end{df}

\begin{ex}
Non-isomorphic groups can have the same composition factors. For example, $C_2 \times C_2$ and $C_4$ both have composition factors $\{| C_2 , C_2 |\}$.
\end{ex}

\begin{prop}
\label{prop:composition-series-N-GN}
Let $G$ be a group. Let $N$ be a normal subgroup of $G$. Then $G$ has a composition series if and only if $N$ and $G/N$ both have composition series, in which case the composition factors of $G$ are the union of the composition factors of $N$ and the composition factors of $G/N$.
\end{prop}

\begin{proof}
\pf
\step{1}{If $G$ has a composition series then $N$ and $G/N$ have composition series.}
\begin{proof}
	\step{a}{\pflet{$G = G_0 \supsetneq G_1 \supsetneq G_2 \supsetneq \cdots \supsetneq G_n = \{e\}$ be a composition series for $G$.}}
	\step{b}{$N$ has a composition series.}
	\begin{proof}
		\step{i}{For all $i$, we have $\frac{G_i \cap N}{G_{i+1} \cap N}$ is either trivial or isomorphic to $G_i / G_{i+1}$.}
		\begin{proof}
			\step{one}{The homomorphism $G_i \cap N \hookrightarrow G_i \twoheadrightarrow G_i / G_{i+1}$ has kernel $G_{i+1} \cap N$.}
			\step{two}{There is an injective homomorphism $(G_i \cap N) / (G_{i+1} \cap N) \rightarrow G_i / G_{i+1}$.}
			\begin{proof}
				\pf\ First Isomorphism Theorem.
			\end{proof}
			\step{three}{$(G_i \cap N) / (G_{i+1} \cap N)$ is either trivial or isomorphic to $G_i / G_{i+1}$.}
			\begin{proof}
				\pf\ Since $G_i / G_{i+1}$ is simple.
			\end{proof}
		\end{proof}
		\step{ii}{Eliminating all duplicates from the series $N = G_0 \cap N \supseteq G_1 \cap N \supseteq G_2 \cap N \supseteq \cdots \supseteq G_n \cap N = \{e\}$ gives a composition series for $N$.}	
	\end{proof}
	\step{c}{$G/N$ has a composition series.}
	\begin{proof}
		\step{i}{For all $i$ we have $\frac{(G_iN)/N}{(G_{i+1}N)/N}$ is either trivial or isomorphic to $G_i / G_{i+1}$.}
		\begin{proof}
			\step{one}{\pflet{$0 \leq i < n$}}
			\step{two}{$\frac{(G_i N)/N}{(G_{i+1}N)N} \cong G_i N / G_{i+1} N$}
			\begin{proof}
				\pf\ Third Isomorphism Theorem.
			\end{proof}
			\step{three}{There exists a surjective homomorphism
			\[ \frac{G_i}{G_{i+1}} \twoheadrightarrow \frac{G_i N}{G_{i+1}N} \enspace . \]}
			\begin{proof}
				\step{A}{\pflet{$f$ be the homomorphism $G_i \hookrightarrow G_i N \twoheadrightarrow G_i N / G_{i+1} N$}}
				\step{B}{$f$ is surjective.}
				\step{C}{$f(G_{i+1}) = \{e\}$}
				\qedstep
				\begin{proof}
					\pf\ By the universal property of quotient groups.
				\end{proof}
			\end{proof}
			\step{four}{$G_i N / G_{i+1} N$ is either trivial or isomorphic to $G_i / G_{i+1}$.}
			\begin{proof}
				\pf\ Proposition \ref{prop:image_of_simple_group}.
			\end{proof}
		\end{proof}
		\step{ii}{Eliminating all duplicates from the series $G/N = G_0 N/ N \supseteq G_1 N / N \supseteq G_2 N / N \supseteq \cdots \supseteq G_n N / N = \{e\}$ gives a composition series for $G/N$.}
	\end{proof}
\end{proof}
\step{2}{If $N$ and $G/N$ have composition series, then $G$ has a composition series, and the composition factors of $G$ are the union of the composition factors of $N$ and the composition factors of $G/N$.}
\begin{proof}
	\step{a}{\pflet{$N = N_0 \supsetneq N_1 \supsetneq N_2 \supsetneq \cdots \supsetneq N_n = \{e\}$ be a composition series for $N$.}}
	\step{b}{\pflet{$G/N = H_0 \supsetneq H_1 \supsetneq H_2 \supsetneq \cdots \supsetneq H_m = \{e\}$ be a composition series for $G/N$.}}
	\step{c}{$G = \inv{\pi}(H_0) \supsetneq \inv{\pi}(H_1) \supsetneq \cdots \inv{\pi}(H_m) = N_0 \supsetneq N_1 \supsetneq N_2 \supsetneq \cdots \supsetneq N_n$ is a composition series for $G$.}
\end{proof}
\qed
\end{proof}

\begin{prop}
Let $G_1$ and $G_2$ be groups. Then $G_1 \times G_2$ has a composition series if and only if $G_1$ and $G_2$ both have composition series.
\end{prop}

\begin{proof}
\pf
\step{1}{If $G_1 \times G_2$ has a composition series then $G_1$ has a composition series.}
\begin{proof}
	\step{a}{\pflet{$G_1 \times G_2 = A_0 \supsetneq A_1 \supsetneq \cdots \supsetneq A_n = \{e\}$ be a composition series.}}
	\step{b}{For each $i$, we have $\pi_1(A_i) / \pi_1(A_{i+1})$ is either isomorphic to $A_i / A_{i+1}$ or trivial.}
	\step{c}{Eliminating duplicates from $G_1 = \pi_1(A_0) \supseteq \pi_1(A_1) \supseteq \cdots \supseteq \pi_1(A_n) = \{e\}$ gives a composition series for $G_1$.}
\end{proof}
\step{2}{If $G_1 \times G_2$ has a composotion series then $G_2$ has a composition series.}
\begin{proof}
	\pf\ Similar.
\end{proof}
\step{3}{If $G_1$ and $G_2$ have composition series then $G_1 \times G_2$ has a composition series.}
\begin{proof}
	\step{a}{\pflet{$G_1 = H_0 \supsetneq H_1 \supsetneq \cdots \supsetneq H_m = \{e\}$ be a composition series for $G_1$.}}
	\step{b}{\pflet{$G_2 = K_0 \supsetneq K_1 \supsetneq \cdots \supsetneq K_n = \{e\}$ be a composition series for $G_2$.}}
	\step{c}{$G_1 \times G_2 = H_0 \times K_0 \supsetneq H_1 \times K_0 \supsetneq \cdots \supsetneq H_m \times K_0 \supsetneq H_m \times K_1 \supsetneq \cdots \supsetneq H_m \times K_n = \{e\}$ is a composition series for $G_1 \times G_2$.}
\end{proof}
\qed
\end{proof}

\begin{df}[Cyclic Series]
A normal series of subgroups is \emph{cyclic} iff every quotient is cyclic.
\end{df}

\chapter{Abelian Groups}

\begin{df}[Abelian Group]
    A group is \emph{Abelian} iff any two elements commute.
\end{df}

In an Abelian group $G$, we often denote the group operation by $+$, the
identity element by $0$ and the inverse of an element $g$ by $-g$. We write
$ng$ for $g^n$ ($g \in G$, $n \in \mathbb{Z}$).

\begin{ex}
    Every group of order $\leq 4$ is Abelian.
\end{ex}

\begin{ex}
    For any positive integer $n$, we have $\mathbb{Z} / n \mathbb{Z}$ is an Abelian group under addition.
\end{ex}

\begin{ex}
    $S_n$ is not Abelian for $n \geq 3$. If $x = \left( \begin{array}{cc} 1 & 2 \end{array} \right)$ and $y = \left( \begin{array}{ccc} 1 & 3 & 2 \end{array} \right)$ then $xy = \left( \begin{array}{cc} 2 & 3 \end{array} \right)$ and $yx = \left( \begin{array}{cc} 1 & 3 \end{array} \right)$.
\end{ex}

\begin{ex}
There are 42 Abelian groups of order 1024 up to isomorphism. %TODO
\end{ex}

\begin{prop}
    Let $G$ be a group. If $g^2 = e$ for all $g \in G$ then $G$ is Abelian.
\end{prop}

\begin{proof}
    \pf\ For any $g,h \in G$ we have
    \begin{align*}
        ghgh           & = e                                                   \\
        \therefore hgh & = g  & (\text{multiplying on the left by }g)          \\
        \therefore hg  & = gh & (\text{multiplying on the right by } h) & \qed
    \end{align*}
\end{proof}

\begin{prop}
    Let $G$ be a group. Then $G$ is Abelian if and only if the function that maps $g$ to $\inv{g}$ is a group homomorphism.
\end{prop}

\begin{proof}
    \pf
    \step{1}{If $G$ is Abelian then the function that maps $g$ to $\inv{g}$ is a group homomorphism.}
    \begin{proof}
        \pf\ Since $\inv{(gh)} = \inv{h} \inv{g} = \inv{g} \inv{h}$.
    \end{proof}
    \step{2}{If the function that maps $g$ to $\inv{g}$ is a group homomorphism then $G$ is Abelian.}
    \begin{proof}
        \pf\ Since $gh = \inv{(\inv{g})}\inv{(\inv{h})} = \inv{(\inv{g}\inv{h})} = hg$.
    \end{proof}
    \qed
\end{proof}

\begin{prop}
    Let $G$ be a group. Then $G$ is Abelian if and only if the function that maps $g$ to $g^2$ is a group homomorphism.
\end{prop}

\begin{proof}
    \pf
    \step{1}{If $G$ is Abelian then the function that maps $g$ to $g^2$ is a group homomorphism.}
    \begin{proof}
        \pf\ Since $(gh)^2 = g^2h^2$.
    \end{proof}
    \step{2}{If the function that maps $g$ to $g^2$ is a group homomorphism then $G$ is Abelian.}
    \begin{proof}
        \pf\ Since we have $(gh)^2 = ghgh = g^2 h^2$ and so $hg = gh$.
    \end{proof}
    \qed
\end{proof}

\begin{prop}
    Let $G$ be a group. Then $G$ is Abelian if and only if the homomorphism $\gamma : G \rightarrow \Aut{\Grp}{G}$ is the trivial homomorphism.
\end{prop}

\begin{proof}
    \pf
    \step{1}{If $G$ is Abelian then $\gamma$ is trivial.}
    \begin{proof}
        \pf\ Since $\gamma_g(a) = ga\inv{g} = a$.
    \end{proof}
    \step{2}{If $\gamma$ is trivial then $G$ is Abelian.}
    \begin{proof}
        \pf\ If $\gamma_g(a) = ga\inv{g} = a$ for all $g$ and $a$ then $ga = ag$ for all $g$, $a$.
    \end{proof}
    \qed
\end{proof}

\begin{prop}
    \label{prop:maximal-finite-order}
    Let $G$ be an Abelian group. Let $g,h \in G$. If $g$ has maximal finite order in $G$, and $h$ has finite order, then $|h| \mid |g|$.
\end{prop}

\begin{proof}
    \pf
    \step{1}{\assume{for a contradiction $|h| \nmid |g|$.}}
    \step{2}{\pick\ a prime $p$ such that $|g| = p^mr$, $|h| = p^n s$ where $p \nmid r$, $p \nmid s$ and $m < n$.}
    \step{3}{$|g^{p^m}h^s| = p^n r$}
    \begin{proof}
        \pf\ Proposition \ref{prop:order-gh-if-gcd-one}.
    \end{proof}
    \step{4}{$|g| < |g^{p^m}h^s|$}
    \qedstep
    \begin{proof}
        \pf\ This contradicts the maximality of $|g|$.
    \end{proof}
    \qed
\end{proof}

\begin{prop}
    Given a set $A$ and an Abelian group $H$, the set $H^A$ is an Abelian group under
    \[ (\phi + \psi)(a) = \phi(a) + \psi(a) \qquad (\phi,\psi \in H^A, a \in A) \enspace . \]
\end{prop}

\begin{proof}
    \pf
    \step{2}{$\phi + (\psi + \chi) = (\phi + \psi) + \chi$}
    \step{2a}{$\phi + \psi = \psi + \phi$}
    \step{3}{\pflet{$0 : A \rightarrow H$ be the function $0(a) = 0$.}}
    \step{4}{$\phi + 0 = 0 + \phi = \phi$}
    \step{5}{Given $\phi : A \rightarrow H$, define $-\phi : A \rightarrow H$ by $(-\phi)(a) = -(\phi(a))$.}
    \step{6}{$\phi + (-\phi) = (-\phi) + \phi = 0$}
    \qed
\end{proof}

\begin{prop}
    Given a group $G$ and an Abelian group $H$, the set $\mathbf{Grp}[G,H]$ is a subgroup of $H^G$.
\end{prop}

\begin{proof}
    \pf
    \step{1}{Given $\phi, \psi : G \rightarrow H$ group homomorphisms, we have $\phi - \psi$ is a group homomorphism.}
    \begin{proof}
        \pf
        \begin{align*}
            (\phi - \psi)(g + g') & = \phi(g + g') - \psi(g + g')             \\
                                  & = \phi(g) + \phi(g') - \psi(g) - \psi(g') \\
                                  & = \phi(g) - \psi(g) + \phi(g') - \psi(g') \\
                                  & = (\phi - \psi)(g) + (\phi - \psi)(g')
        \end{align*}
    \end{proof}
    \qed
\end{proof}

\begin{prop}
Let $G$ be a group. The following are equivalent.
\begin{enumerate}
\item $\mathrm{Inn}(G)$ is cyclic.
\item $\mathrm{Inn}(G)$ is trivial.
\item $G$ is Abelian.
\end{enumerate}
\end{prop}

\begin{proof}
\pf
\step{1}{$1 \Rightarrow 2$}
\begin{proof}
	\step{a}{\assume{$\mathrm{Inn}(G) = \langle \gamma_g \rangle$}}
	\step{b}{$g$ commutes with every element of $G$}
	\begin{proof}
		\step{b}{\pflet{$x \in G$}}
		\step{c}{\pick\ $n \in \mathbb{Z}$ such that $\gamma_x = \gamma_g^n$}
		\step{d}{$\forall y \in G. xy\inv{x} = g^nyg^{-n}$}
		\step{e}{$x g \inv{x} = g$}
	\end{proof}
	\step{c}{$\gamma_g = \id{G}$}
\end{proof}
\step{2}{$2 \Rightarrow 3$}
\begin{proof}
	\step{a}{\assume{$\forall g \in G. \gamma_g = \id{G}$}}
	\step{b}{\pflet{$x,y \in G$}}
	\step{c}{$\gamma_x(y) = y$}
	\step{d}{$xy\inv{x} = y$}
	\step{e}{$xy = yx$}
\end{proof}
\step{2}{$3 \Rightarrow 2$}
\begin{proof}
	\pf\ If $xy = yx$ for all $x,y$ then $\gamma_x(y) = y$ for all $x$, $y$.
\end{proof}
\step{3}{$2 \Rightarrow 1$}
\begin{proof}
	\pf\ Easy.
\end{proof}
\qed
\end{proof}

\begin{cor}
If $\Aut{\Grp}{G}$ is cyclic then $G$ is Abelian.
\end{cor}

\begin{prop}
Every subgroup of an Abelian group is normal.
\end{prop}

\begin{proof}
\pf\ Let $G$ be an Abelian group and $N$ a subgroup of $G$. Given $g \in G$ and $n \in N$ we have $gn\inv{g} = n \in N$. \qed
\end{proof}

\begin{prop}
For any group $G$, the group $G / [G,G]$ is Abelian.
\end{prop}

\begin{proof}
\pf\ For any $g,h \in G$ we have
\begin{align*}
gh\inv{(hg)} & \in [G,G] \\
\therefore gh[G,G] & = hg[G,G] & \qed
\end{align*}
\end{proof}

\begin{prop}
Let $G$ be a finite Abelian group. Let $p$ be a prime divisor of $|G|$. Then $G$ has an element of order $p$.
\end{prop}

\begin{proof}
\pf
\step{1}{\assume{as induction hypothesis the result holds for all groups smaller than $G$.}}
\step{2}{\pick\ $g \in G - \{0\}$.}
\step{3}{\pick\ an element $h \in \langle g \rangle$ with prime order $q$.}
\step{4}{\case{$q = p$}}
\begin{proof}
\pf\ $h$ is the required element.
\end{proof}
\step{5}{\case{$q \neq p$}}
\begin{proof}
\step{a}{\pick\ $r \in G$ such that $r + \langle h \rangle$ has order $p$ in $G / \langle h \rangle$.}
\begin{proof}
\pf\ By induction hypothesis since $|G / \langle h \rangle| = |G| / q$.
\end{proof}
\step{b}{$pr \in \langle h \rangle$}
\step{c}{\pick\ $k$ such that $pr = kh$}
\step{d}{$pqr = e$}
\step{e}{$qr$ has order $p$.}
\end{proof}
\qed
\end{proof}

\begin{cor}
For $n$ an odd integer, any Abelian group of order $2n$ has exactly one element of order 2.
\end{cor}

\begin{proof}
\pf\ If $x$ and $y$ are distinct elements of order 2 then $\langle x,y \rangle = \{ e, x, y, xy \}$ has size 4 and so $4 \mid 2n$ which is a contradiction. \qed
\end{proof}

\begin{ex}
It is not true that, if $G$ is a finite group and $d \mid |G|$, then $G$ has an element of order $d$. The quaternionic group has no element of order 4.
\end{ex}

\begin{prop}
If $G$ is a finite Abelian group and $d \mid |G|$ then $G$ has a subgroup of size $d$.
\end{prop}

\begin{proof}
\pf
\step{1}{\assume{as induction hypothesis the result is true for all $d' < d$.}}
\step{2}{\assume{w.l.o.g. $d \neq 1$.}}
\step{3}{\pick\ a prime $p$ such that $p \mid d$.}
\step{4}{\pick\ an element $g \in G$ of order $p$.}
\step{5}{$d/p \mid |G/\langle g \rangle|$}
\step{6}{\pick\ a subgrop $H$ of $G/\langle g \rangle$ of size $d/p$.}
\step{7}{$\inv{\pi}(H)$ is a subgroup of $G$ of size $d$.}
\qed
\end{proof}

\begin{prop}
Let $(G,\cdot)$ be a group. Let $\circ : G^2 \rightarrow G$ be a group homomorphism such that $(G,\circ)$ is a group. Then $\circ$ and $\cdot$ coincide, and $G$ is Abelian.
\end{prop}

\begin{proof}
\pf
\step{1}{For all $g_1,g_2,h_1,h_2 \in G$ we have
\[ (g_1 g_2) \circ (h_1 h_2) = (g_1 \circ h_1)(g_2 \circ h_2) \]}
\step{2}{$e \circ e = e$}
\begin{proof}
\pf
\begin{align*}
e \circ e & = (ee) \circ (ee) \\
& = (e \circ e) (e \circ e)
\end{align*}
Hence $e \circ e = e$ by Cancellation.
\end{proof}
\step{3}{$e$ is the identity of $(G,\circ)$}
\step{4}{For all $g,h \in G$ we have
\[ g \circ h = g h \]}
\begin{proof}
\pf
\begin{align*}
g \circ h & = (ge) \circ (eh) \\
& = (g \circ e)(e \circ h) \\
& = gh
\end{align*}
\end{proof}
\step{5}{For all $g,h \in G$ we have $gh = hg$.}
\begin{proof}
\pf
\begin{align*}
gh & = (e \circ g)(h \circ e) \\
& = (eh) \circ (ge) \\
& = h \circ g \\
& = hg
\end{align*}
\end{proof}
\qed
\end{proof}

\begin{cor}
\label{cor:group-in-group}
If $(G, m : G^2 \rightarrow G, e : 1 \rightarrow G, i : G \rightarrow G)$ is a group object in $\Grp$ then $m$ is the multiplication of $G$, $e(*)$ is the identity of $G$, $i(g) = \inv{g}$, and $G$ is Abelian.

Conversely, if $(G, m)$ is any Abelian group, then $(G,m,e,i)$ is a group object in $\Grp$ where $e(*) = e$ and $i(g) = \inv{g}$.
\end{cor}

\begin{prop}
\label{prop:all-orders-leq-two}
Let $G$ be a group.
If every element of $G$ has order $\leq 2$ then $G$ is Abelian.
\end{prop}

\begin{proof}
\pf
\step{1}{\pflet{$x,y \in G$} \prove{$xy = yx$}}
\step{2}{\assume{w.l.o.g. $x \neq e \neq y$.}}
\step{3}{$x^2 = e = y^2$}
\step{4}{$\inv{x} = x$ and $\inv{y} = y$.}
\step{5}{\case{$xy = e$}}
\begin{proof}
	\pf\ Then $y = \inv{x}$ and so $xy = yx = e$.
\end{proof}
\step{6}{\case{$xy \neq e$}}
\begin{proof}
	\step{a}{$(xy)^2 = e$}
	\step{b}{$xyxy = e$}
	\step{c}{$xy = \inv{y} \inv{x}$}
	\step{d}{$xy = yx$}
\end{proof}
\qed
\end{proof}

\begin{prop}
Every Abelian group is solvable.
\end{prop}

\begin{proof}
\pf\ If $G$ is Abelian then $G' = \{e\}$. \qed
\end{proof}

\begin{prop}
The only non-trivial simple finite Abelian groups are $\mathbb{Z} / p \mathbb{Z}$ for $p$ a prime.
\end{prop}

\begin{proof}
\pf
\step{1}{\pflet{$G$ be a non-trivial simple finite Abelian group.}}
\step{2}{\pick\ a prime $p$ that divides $|G|$.}
\step{3}{\pick\ an element $a \in G$ of order $p$.}
\begin{proof}
	\pf\ Cauchy's Theorem.
\end{proof}
\step{4}{$\langle a \rangle = G$}
\qed
\end{proof}

\section{The Category of Abelian Groups}

\begin{df}[Category of Abelian Groups]
    Let $\Ab$ be the full subcategory of $\mathbf{Grp}$ whose objects are the Abelian groups.
\end{df}

\begin{prop}
If $(G, m : G^2 \rightarrow G, e : 1 \rightarrow G, i : G \rightarrow G)$ is a group object in $\Ab$ then $m$ is the multiplication of $G$, $e(*)$ is the identity of $G$, $i(g) = \inv{g}$, and $G$ is Abelian.

Conversely, if $(G, m)$ is any Abelian group, then $(G,m,e,i)$ is a group object in $\Ab$ where $e(*) = e$ and $i(g) = \inv{g}$.
\end{prop}

\begin{proof}
\pf\ Immediate from Corollary \ref{cor:group-in-group}. \qed
\end{proof}

\begin{df}[Direct Sum]
    Given Abelian groups $G$ and $H$, we also call the direct product of $G$ and $H$ the \emph{direct sum} and denote it $G \oplus H$.
\end{df}

\begin{prop}
    Given Abelian groups $G$ and $H$, the direct sum $G \oplus H$ is the coproduct of $G$ and $H$ in $\Ab$.
\end{prop}

\begin{proof}
    \pf
    \step{1}{\pflet{$\kappa_1 : G \rightarrow G \oplus H$ be the group homomorphism $\kappa_1(g) = (g,e_H)$.}}
    \step{2}{\pflet{$\kappa_2 : H \rightarrow G \oplus H$ be the group homomorphism $\kappa_2(h) = (e_G,h)$.}}
    \step{3}{Given group homomorphism $\phi : G \rightarrow K$ and $\psi : H \rightarrow K$, define $[\phi, \psi] : G \oplus H \rightarrow K$ by $[\phi,\psi](g,h) = \phi(g) + \psi(h)$.}
    \step{4}{$[\phi,\psi]$ is a group homomorphism.}
    \begin{proof}
        \pf
        \begin{align*}
            [\phi,\psi]((g,h) + (g',h')) & = [\phi,\psi](g + g', h + h')             \\
                                         & = \phi(g + g') + \psi(h + h')             \\
                                         & = \phi(g) + \phi(g') + \psi(h) + \psi(h') \\
                                         & = \phi(g) + \psi(h) + \phi(g') + \psi(h') \\
                                         & = [\phi,\psi](g,h) + [\phi,\psi](g',h')
        \end{align*}
    \end{proof}
    \step{5}{$[\phi,\psi] \circ \kappa_1 = \phi$}
    \begin{proof}
        \pf
        \begin{align*}
            [\phi,\psi](\kappa_1(g)) & = [\phi,\psi](g,e_h)  \\
                                     & = \phi(g) + \psi(e_H) \\
                                     & = \phi(g) + e_K       \\
                                     & = \phi(g)
        \end{align*}
    \end{proof}
    \step{6}{$[\phi,\psi] \circ \kappa_2 = \psi$}
    \begin{proof}
        \pf\ Similar.
    \end{proof}
    \step{7}{If $f : G \oplus H \rightarrow K$ is a group homomorphism with $f \circ \kappa_1 = \phi$ and $f \circ \kappa_2 = \psi$ then $f = [\phi,\psi]$.}
    \begin{proof}
        \pf
        \begin{align*}
            f(g,h) & = f((g,e_H) + (e_G,h))            \\
                   & = f(\kappa_1(g)) + f(\kappa_2(h)) \\
                   & = \phi(g) + \psi(h)
        \end{align*}
    \end{proof}
    \qed
\end{proof}

\begin{thm}
Every finitely generated Abelian group is a direct sum of cyclic groups.
\end{thm}

\begin{proof}
\pf\ TODO \qed
\end{proof}

\section{Free Abelian Groups}

\begin{prop}
    Let $A$ be a set. Let $\mathcal{F}^A$ be the category whose objects are pairs $(G,j)$ where $G$ is an Abelian group and $j$ is a function $A \rightarrow G$, with morphisms $f : (G,j) \rightarrow (H,k)$ the group homomorphisms $f : G \rightarrow H$ such that $f \circ j = k$. Then $\mathcal{F}^A$ has an initial object.
\end{prop}

\begin{proof}
    \pf
    \step{1}{\pflet{$\mathbb{Z}^{\oplus A}$ be the subgroup of $\mathbb{Z}^A$ consisting of all functions $\alpha : A \rightarrow \mathbb{Z}$ such that $\alpha(a) = 0$ for only finitely many $a \in A$.}}
    \step{2}{\pflet{$i : A \rightarrow \mathbb{Z}^{\oplus A}$ be the function such that $i(a)(b) = 1$ if $a = b$ and 0 if $a \neq b$.}}
    \step{3}{\pflet{$G$ be any Abelian group and $j : A \rightarrow G$ any function.}}
    \step{4}{The unique homomorphism $\phi : \mathbb{Z}^{\oplus A} \rightarrow G$ required is defined by $\phi(\alpha) = \sum_{a \in A} \alpha(a) j(a)$}
    \qed
\end{proof}

\begin{df}[Free Abelian Group]
    For any set $A$, the \emph{free Abelian group} on $A$ is the initial object $(F^{ab}(A),i)$ in $\mathcal{F}^A$.
\end{df}

\begin{prop}
    For any sets $A$ and $B$, we have that $F^{ab}(A+B)$ is the coproduct of $F^{ab}(A)$ and $F^{ab}(B)$ in $\Grp$.
\end{prop}

\[ \begin{tikzcd}
        & G & \\
        F^{ab}(A) \arrow[r,"\kappa_1"] \arrow[ur,"f"] & F^{ab}(A+B) \arrow[u,"k"] & \arrow[l,"\kappa_2"] F^{ab}(B) \arrow[ul,"g"] \\
        A \arrow[r,"k_1"] \arrow[u,"i_A"] & A+B \arrow[u,"j"] & \arrow[l,"k_2"] \arrow[u,"i_B"] B
    \end{tikzcd} \]

\begin{proof}
    \pf
    \step{1}{\pflet{$i_A : A \rightarrow F^{ab}(A)$, $i_B : B \rightarrow F^{ab}(B)$, $j : A + B \rightarrow F^{ab}(A+B)$ be the canonical injections.}}
    \step{2}{\pflet{$\kappa_1$, $\kappa_2$ be the unique group homomorphisms that make the diagram above commute.}}
    \step{3}{\pflet{$G$ be any group and $f : F^{ab}(A) \rightarrow G$, $g : F^{ab}(B) \rightarrow G$ any group homomorphisms.}}
    \step{4}{\pflet{$h : A + B \rightarrow G$ be the unique function such that $h \circ k_1 = f \circ i_A$ and $h \circ k_2 = g \circ i_B$.}}
    \step{5}{\pflet{$k : F^{ab}(A+B) \rightarrow G$ be the unique group homomorphism such that $k \circ j = h$.}}
    \step{6}{$k$ is the unique group homomorphism such that $k \circ \kappa_1 \circ i_A = f \circ i_A$ and $k \circ \kappa_2 \circ i_B = g \circ i_B$.}
    \step{7}{$k$ is the unique group homomorphism such that $k \circ \kappa_1 = f$ and $k \circ \kappa_2 = g$.}
    \qed
\end{proof}

\begin{prop}
\label{prop:Fab-reflects-cong}
    For $A$ and $B$ finite sets, if $\Fab{A} \cong \Fab{B}$ then $A \cong B$.
\end{prop}

\begin{proof}
    \pf
    \step{1}{For any set $C$, define $\sim$ on $\Fab{C}$ by: $f \sim f'$ iff there exists $g \in \Fab{C}$ such that $f - f' = 2g$.}
    \step{2}{For any set $C$, $\sim$ is an equivalence relation on $\Fab{C}$.}
    \step{3}{For any set $C$, we have $\Fab{C}/\sim$ is finite if and only if $C$ is finite, in which case $|\Fab{C}/\sim| = 2^{|C|}$.}
    \begin{proof}
        \pf\ There is a bijection between $\Fab{C}/\sim$ and the finite subsets of $C$, which maps $f$ to $\{ c \in C : f(c) \text{ is odd} \}$.
    \end{proof}
    \step{4}{If $\Fab{A} \cong \Fab{B}$ then $A \cong B$.}
    \begin{proof}
        \pf\ If $|\Fab{A}/\sim| = |\Fab{B} / \sim|$ then $2^{|A|} = 2^{|B|}$ and so $|A| = |B|$.
    \end{proof}
    \qed
\end{proof}

\begin{prop}
Let $G$ be an Abelian group. Then $G$ is finitely generated if and only if there exists a surjective homomorphism $\mathbb{Z}^{\oplus n} \twoheadrightarrow G$ for some $n$.
\end{prop}

\begin{proof}
\pf
\step{1}{If $G$ is finitely generated then there exists a surjective homomorphism $\mathbb{Z}^{\oplus n} \twoheadrightarrow G$ for some $n$.}
\begin{proof}
	\pf\ Let $G = \langle a_1, \ldots, a_n \rangle$. Define $\phi : \mathbb{Z}^{\oplus n} \twoheadrightarrow G$ by $\phi(i_1, \ldots, i_n) = i_1 \cdot a_1 + \cdots + i_n \cdot a_n$.
\end{proof}
\step{2}{If there exists a surjective homomorphism $\phi : \mathbb{Z}^{\oplus n} \twoheadrightarrow G$ for some $n$ then $G$ is finitely generated.}
\begin{proof}
	\pf\ $G$ is generated by $\phi(1, 0, \ldots, 0)$, $\phi(0, 1, 0, \ldots, 0)$, \ldots, $\phi(0, \ldots, 0, 1)$.
\end{proof}
\qed
\end{proof}

\begin{prop}
Let $A$ be a set. Let $i : A \hookrightarrow F(A)$ be the free group on $A$. Then $\pi \circ i : A \rightarrow F(A) / [F(A),F(A)]$ is the free Abelian group on $A$.
\end{prop}

\[ \begin{tikzcd}
F(A)/[F(A),F(A)] \arrow[dr,"h"] \\
F(A) \arrow[r,"g"] \arrow[u,"\pi"] & G \\
A \arrow[u,"i"] \arrow[ur,"f"]
\end{tikzcd} \]

\begin{proof}
\pf
\step{1}{\pflet{$G$ be an Abelian group and $f : A \rightarrow G$ a function.}}
\step{2}{\pflet{$g : F(A) \rightarrow G$ be the unique group homomorphism such that $g \circ i = f$.}}
\step{3}{$[F(A),F(A)] \subseteq \ker g$}
\begin{proof}
	\pf\ For all $x,y \in F(A)$ we have $g(xy\inv{x}\inv{y}) = g(x) + g(y) - g(x) - g(y) = 0$.
\end{proof}
\step{4}{\pflet{$h : F(A)/[F(A),F(A)]$ be the unique group homomorphism such that $h \circ \pi = g$.}}
\step{5}{$h$ is the unique group homomorphism such that $h \circ \pi \circ i = f$.}
\qed
\end{proof}

\begin{cor}
Let $A$ and $B$ be sets. Let $F(A)$ and $F(B)$ be the free groups on $A$ and $B$ respectively. If $F(A) \cong F(B)$ then $A \cong B$.
\end{cor}

\begin{proof}
\pf\ Proposition \ref{prop:Fab-reflects-cong}. \qed
\end{proof}

\section{Cokernels}

\begin{prop}
Let $\phi : G \rightarrow H$ be a homomorphism between Abelian groups. Then there exists an Abelian group $K$ and homomorphism $\pi : H \rightarrow K$ that is initial with respect to all homomorphism $\alpha : H \rightarrow L$ such that $\alpha \circ \phi = 0$.
\end{prop}

\begin{proof}
\pf
\step{1}{\pflet{$K = H / \im \phi$ and $\pi$ be the canonical homomorphism.}}
\step{2}{\pflet{$\pi \circ \phi = 0$}}
\step{3}{\pflet{$\alpha : H \rightarrow L$ satisfy $\alpha \circ \phi = 0$}}
\step{4}{$\im \phi \subseteq \ker \alpha$}
\step{5}{There exists a unique $\overline{\alpha} : H / \im \phi \rightarrow L$ such that $\overline{\alpha} \circ \pi = \alpha$}
\qed
\end{proof}

\begin{df}[Cokernel]
For any homomorphism $\phi : G \rightarrow H$ in $\Ab$, the \emph{cokernel} of $\phi$ is the Abelian group $\coker \phi$ and homomorphism $\pi : H \rightarrow \coker \phi$ that is initial among homomorphisms $\alpha : H \rightarrow L$ such that $\alpha \circ \phi = 0$.
\end{df}

\begin{prop}
$\pi : H \rightarrow \coker \phi$ is initial among functions $f : H \rightarrow X$ such that, for all $x,y \in H$, if $x + \im \phi = y + \im \phi$ then $f(x) = f(y)$.
\end{prop}

\begin{proof}
\pf\ Easy. \qed
\end{proof}

\begin{prop}
Let $\phi : G \rightarrow H$ be a homomorphism of Abelian groups. Then the following are equivalent.
\begin{itemize}
\item $\phi$ is an epimorphism.
\item $\coker \phi$ is trivial.
\item $\phi$ is surjective.
\end{itemize}
\end{prop}

\begin{proof}
\pf
\step{1}{$1 \Rightarrow 2$}
\begin{proof}
	\step{a}{\assume{$\phi$ is epi.}}
	\step{b}{\pflet{$\pi : H \rightarrow \coker \phi$ be the canonical homomorphism.}}
	\step{c}{$\pi \circ \phi = 0 \circ \phi$}
	\step{d}{$\pi = 0$}
	\step{e}{$\coker \phi = \im \pi$ is trivial.}
\end{proof}
\step{2}{$2 \Rightarrow 3$}
\begin{proof}
	\pf\ If $\coker \phi = H / \im \phi$ is trivial then $\im \phi = H$.
\end{proof}
\step{3}{$3 \Rightarrow 1$}
\begin{proof}
	\pf\ If it is surjective then it is epi in $\mathbf{Set}$.
\end{proof}
\qed
\end{proof}

\section{Commutator Subgroups}

\begin{prop}
Let $G$ be a group. Let $G'$ be the commutator subgroup of $G$. Then $G / G'$ is Abelian.
\end{prop}

\begin{proof}
\pf\ Since $gh\inv{g}\inv{h}G' = G'$ so $ghG' = hgG'$. \qed
\end{proof}

\begin{prop}
Let $G$ be a group and $A$ an Abelian group. Let $\alpha : G \rightarrow A$ be a homomorphism. Then $G' \subseteq \ker \alpha$.
\end{prop}

\begin{proof}
\pf\ Since $\phi([g,h]) = \phi(g) \phi(h) \inv{\phi(g)} \inv{\phi(h)} = e$. \qed
\end{proof}

\begin{cor}
Let $G$ be a group.
The canonical projection $G \twoheadrightarrow G / G'$ is initial in the category of homomorphisms from $G$ to an Abelian group.
\end{cor}

\begin{df}[Abelian Series]
A normal series of subgroups is \emph{Abelian} iff every quotient is Abelian.
\end{df}

\begin{lm}
\label{lm:GH-Abelian}
Let $G$ be a group. Let $H$ be a normal subgroup of $G$. If $G/H$ is Abelian then $G' \subseteq G/H$.
\end{lm}

\begin{proof}
\pf\ Given $g,h \in G$ we have
\begin{align*}
ghH & = hgH \\
\therefore gh\inv{g}\inv{h} & \in H & \qed
\end{align*}
\end{proof}

\begin{prop}
\label{prop:solvable}
Let $G$ be a finite group. The following are equivalent.
\begin{enumerate}
\item All composition factors of $G$ are cyclic.
\item $G$ has a cyclic series of subgroups ending in $\{e\}$.
\item $G$ has an Abelian series of subgroups ending in $\{e\}$.
\item $G$ is solvable.
\end{enumerate}
\end{prop}

\begin{proof}
\pf
\step{1}{$1 \Rightarrow 2$}
\begin{proof}
	\pf\ Trivial.
\end{proof}
\step{2}{$2 \Rightarrow 3$}
\begin{proof}
	\pf\ Trivial.
\end{proof}
\step{3}{$3 \Rightarrow 4$}
\begin{proof}
	\step{a}{\pflet{$G = G_0 \supseteq G_1 \supseteq \cdots \supseteq G_n = \{e\}$ be an Abelian series of subgroups.}}
	\step{b}{For all $i$ we have $G^{(i)} \subseteq G_i$.}
	\begin{proof}
		\pf\ Lemma \ref{lm:GH-Abelian}.
	\end{proof}
	\step{c}{$G^{(n)} = \{e\}$}
\end{proof}
\step{4}{$4 \Rightarrow 1$}
\begin{proof}
	\pf\ Extend the derived series of $G$ to a composition series, using the fact that every simple Abelian group is cyclic.
\end{proof}
\qed
\end{proof}

\begin{cor}
All $p$-groups are solvable.
\end{cor}

\begin{proof}
\pf\ Their composition factors are simple $p$-groups, hence cyclic. \qed
\end{proof}

\begin{cor}
Let $G$ be a group and $N$ a normal subgroup. Then $G$ is solvable if and only if both $N$ and $G/N$ are solvable.
\end{cor}

\begin{proof}
\pf\ By Proposition \ref{prop:composition-series-N-GN}. \qed
\end{proof}

\begin{cor}
Let $G$ be a finite solvable group. Then the composition factors of $G$ are exactly $C_p$ for $p$ a prime factor of $G$ (with the same multiplicities).
\end{cor}

\begin{proof}
\pf\ Since each composition factor is simple and cyclic hence removes one prime factor in $|G|$. \qed
\end{proof}

\section{Derived Series}

\begin{df}[Derived Series]
Let $G$ be a group. The \emph{derived series} of $G$ is the series of subgroups
\[ G \supseteq G' \supseteq G'' \supseteq G''' \supseteq \cdots \]
where $G'$ is the commutator subgroup of $G$.

We write $G^{(i)}$ for the $i+1$st entry in the derived series
\end{df}

\begin{prop}
\label{prop:derived-group-characteristic}
Each $G^{(i)}$ is characteristic.
\end{prop}

\begin{proof}
\pf
\step{0}{$G$ is characteristic in $G$.}
\begin{proof}
	\pf\ Trivial.
\end{proof}
\step{1}{If $G^{(i)}$ is characteristic in $G$ then $G^{(i+1)}$ is characteristic in $G$.}
\begin{proof}
	\step{a}{\assume{$G^{(i)}$ is characteristic.}}
	\step{b}{\pflet{$\phi : G \cong G$ be an automorphism of $G$.}}
	\step{c}{For all $g,h \in G^{(i)}$ we have $\phi([g,h]) \in G^{(i+1)}$.}
	\begin{proof}
		\pf\ Since $\phi([g,h]) = [\phi(g),\phi(h)]$ and $\phi(g),\phi(h) \in G^{(i)}$.
	\end{proof}
	\step{d}{$\phi(G^{(i+1)}) \subseteq G^{(i+1)}$}
\end{proof}
\qed
\end{proof}

\section{Solvable Groups}

\begin{df}[Solvable]
A group is \emph{solvable} iff its derived series terminates in $\{e\}$.
\end{df}

\begin{thm}[Feit-Thompson]
Every finite group of odd order is solvable.
\end{thm}

%TODO

\begin{cor}
Every non-Abelian finite simple group has even order.
\end{cor}

\begin{proof}
\pf\ A non-Abelian finite simple group of odd order is solvable, hence its composition factors are all Abelian. But a simple group is its own only composition factor. \qed
\end{proof}

\begin{prop}
Let $H$ be a nontrivial normal subgroup of a solvable group $G$. Then $H$ contains a nontrivial Abelian subgroup that is normal in $G$.
\end{prop}

\begin{proof}
\pf
\step{1}{\pflet{$r$ be the largest number such that $H \cap G^{(r)}$ is non-trivial.}}
\step{2}{\pflet{$K = H \cap G^{(r)}$}}
\step{3}{$K$ is Abelian.}
\begin{proof}
	\pf\ Since $[K,K] \subseteq G^{(r+1)} = \{e\}$.
\end{proof}
\step{4}{$K$ is normal.}
\begin{proof}
	\pf\ Proposition \ref{prop:derived-group-characteristic}.
\end{proof}
\qed
\end{proof}

\begin{thm}[Burnside]
Let $p$ and $q$ be primes. Every group of order $p^a q^b$ is solvable.
\end{thm}

%TODO

\chapter{Group Actions}

\section{Group Actions}

\begin{df}[Action]
Let $G$ be a group. Let $A$ be an object of a category $\mathcal{C}$. A \emph{(left) action} of $G$ on $A$ is a group homomorphism $G \rightarrow \mathrm{Aut}_\mathcal{C}(A)$.

It is \emph{faithful} or \emph{effective} iff it is injective.
\end{df}

\begin{prop}
Let $A$ be a set. An action of the group $G$ on the set $A$ is given by a function $\cdot : G \times A \rightarrow A$ such that
\begin{itemize}
\item $\forall a \in A. ea = a$
\item $\forall g,h \in G. \forall a \in A. (gh)a = g(ha)$
\end{itemize}
\end{prop}

\begin{proof}
\pf\ Just unfolding definitions. \qed
\end{proof}

\begin{ex}
Left multiplication defines a faithful action of any group on its own underlying set.

In fact, for any subgroup $H$ of a group $G$, left multiplication defines an action of $G$ on $G/H$.
\end{ex}

\begin{cor}[Cayley's Theorem]
Every group $G$ is a subgroup of a symmetric group, namely $\Aut{\Set}{G}$.
\end{cor}

\begin{ex}
Conjugation $g * h = gh\inv{g}$ is an action of any group on its own underlying set.
\end{ex}

\begin{df}[Transitive]
An action of a group $G$ on a set $A$ is \emph{transitive} iff, for all $a,b \in A$, there exists $g \in G$ such that $ga = b$.
\end{df}

\begin{ex}
Left multiplication of a group $G$ is a transitive action of $G$ on $G$.
\end{ex}

\begin{df}[Orbit]
Given an action of a group $G$ on a set $A$ and $a \in A$, the \emph{orbit} of $a$ is
\[ \mathrm{O}_G(a) := \{ ga : g \in G \} \enspace . \]
\end{df}

\begin{prop}
Given an action of a group $G$ on a set $A$, the orbits form a partition of $A$.
\end{prop}

\begin{proof}
\pf
\step{1}{Every element of $A$ is in some orbit.}
\begin{proof}
\pf\ Since $a \in \mathrm{O}_G(a)$.
\end{proof}
\step{2}{Distinct orbits are disjoint.}
\begin{proof}
\step{a}{\pflet{$a \in \mathrm{O}_G(b) \cap \mathrm{O}_G(c)$}}
\step{b}{\pick\ $g,h \in G$ such that $a = gb = hc$.}
\step{c}{$\mathrm{O}_G(b) \subseteq \mathrm{O}_G(c)$}
\begin{proof}
\pf\ For all $k \in G$ we have $kb = k\inv{g}hc$.
\end{proof}
\step{d}{$\mathrm{O}_G(c) \subseteq \mathrm{O}_G(b)$}
\begin{proof}
\pf\ Similar.
\end{proof}
\end{proof}
\qed
\end{proof}

\begin{prop}
Given an action of a group $G$ on a set $A$ and $a \in A$, the action is transitive on $\mathrm{O}_G(a)$.
\end{prop}

\begin{proof}
\pf
\step{1}{The restriction of the action is an action on $\mathrm{O}_G(a)$.}
\begin{proof}
\pf\ Since $g(ha) = (gh)a$, the action maps $\mathrm{O}_G(a)$ to itself.
\end{proof}
\step{2}{The restricted action is transitive.}
\begin{proof}
\pf\ Given $ga, ha \in \mathrm{O}_G(a)$, we have $ha = (h\inv{g})(ga)$.
\end{proof}
\qed
\end{proof}

\begin{df}[Stabilizer Subgroup]
Given an action of a group $G$ on a set $A$ and $a \in A$, the \emph{stabilizer subgroup} of $a$ is
\[ \Stab{G}{a} := \{ g \in G : ga = a \} \enspace . \]
\end{df}

\begin{prop}
Stabilizer subgroups are subgroups.
\end{prop}

\begin{proof}
\pf\ If $g,h \in \Stab{G}{a}$ then $g \inv{h} a = a$ so $g \inv{h} \in \Stab{G}{a}$. \qed
\end{proof}

\begin{prop}
Let $G$ act on a set $A$. Let $a \in A$ and $g \in G$. Then
\[ \Stab{G}{ga} = g \Stab{G}{a} \inv{g} \enspace . \]
\end{prop}

\begin{proof}
\pf
\begin{align*}
h \in \Stab{G}{ga} & \Leftrightarrow hga = ga \\
& \Leftrightarrow \inv{g}hga = a \\
& \Leftrightarrow \inv{g}hg \in \Stab{G}{a} \\
& \Leftrightarrow h \in g \Stab{G}{a} \inv{g} & \qed
\end{align*}
\end{proof}

\begin{cor}
Let $G$ be an action on a set $A$ and $a \in A$. If $\Stab{G}{a}$ is normal in $G$, then for any $b \in \mathrm{O}_G(a)$ we have $\Stab{G}{a} = \Stab{G}{b}$.
\end{cor}

\begin{df}[Free]
An action of a group $G$ on a set $A$ is \emph{free} iff, whenever $ga = a$, then $g = e$.
\end{df}

\begin{ex}
The action of left multiplication is free.
\end{ex}

\begin{prop}
Let $G$ be a group. Let $H$ be a subgroup of $G$ of finite index $n$. Then $H$ includes a subgroup $K$ that is normal in $G$ and such that $|G:K|$ divides $\gcd(|G|,n!)$.
\end{prop}

\begin{proof}
\pf
\step{1}{\pflet{$\sigma : G \rightarrow \Aut{\Set}{G/H}$ be the action of left multiplication.}}
\step{2}{\pflet{$K = \ker \sigma$}}
\step{3}{$K \subseteq H$}
\begin{proof}
	\step{a}{\pflet{$g \in K$}}
	\step{b}{$\sigma(g)(H) = H$}
	\step{c}{$gH = H$}
	\step{d}{$g \in H$}
\end{proof}
\step{4}{$K$ is normal in $G$.}
\begin{proof}
	\pf\ Proposition \ref{prop:kernel-normal}.
\end{proof}
\step{5}{$|G : K| \mid |G|$}
\begin{proof}
	\pf\ Lagrange's Theorem.
\end{proof}
\step{6}{$|G : K| \mid n!$} 
\begin{proof}
	\pf\ Since $G / K$ is a subgroup of $\Aut{\Set}{G/H}$.
\end{proof}
\qed
\end{proof}

\begin{cor}
Let $G$ be a finite group. Let $H$ be a subgroup of $G$ of index $p$ where $p$ is the smallest prime that divides $|G|$. Then $H$ is normal in $G$.
\end{cor}

\begin{proof}
\pf
\step{1}{\pick\ a subgroup $K$ of $H$ normal in $G$ such that $|G : K|$ divides $\gcd(|G|, p!)$.}
\step{2}{$|G : K|$ divides $p$.}
\step{3}{$|G:H||H:K|$ divides $p$.}
\step{4}{$|H:K| = 1$}
\step{9}{$H = K$}
\step{10}{$H$ is normal.}
\qed
\end{proof}

\begin{cor}
\label{cor:index-two-normal}
Any subgroup of index 2 is normal.
\end{cor}

\begin{prop}
Let $G$ be a group with finite set of generators $A$. Then left multiplication defines a free action of $G$ on its Cayley graph.
\end{prop}

\begin{proof}
\pf\ Easy since if $g_2 = g_1 a$ then $h g_2 = h g_1 a$. \qed
\end{proof}

\begin{cor}
A free group acts freely on a tree.
\end{cor}

\begin{thm}
If a group $G$ acts freely on a tree then $G$ is free.
\end{thm}

%TODO

\begin{cor}
Every subgroup of the free group on a finite set is free.
\end{cor}

\begin{proof}
\pf\ If $H$ is a subgroup of $F(A)$ then left multiplication defines a free action of $H$ on the Cayley graph of $F(A)$, which is a tree. \qed
\end{proof}

\begin{prop}
\label{prop:class-formula}
Let $S$ be a finite set. Let $G$ be a group acting on $S$. Let $Z$ be the set of fixed points of the action:
\[ Z = \{a \in S : \forall g \in G. ga = a \} \enspace . \]
Let $A$ be a set of representatives for the nontrivial orbits of the action. Then
\[ |S| = |Z| + \sum_{a \in A} [G : \Stab{G}{a}] \enspace . \]
\end{prop}

\begin{proof}
\pf\ Immediate from the fact that the orbits partition $S$. \qed
\end{proof}

\begin{cor}
\label{cor:fixed-points-of-action-of-p-group}
Let $p$ be a prime.
Let $S$ be a finite set. Let $G$ be a $p$-group acting on $S$. Let $Z$ be the set of fixed points of the action. Then $|Z| \cong |S| (\mod p)$.
\end{cor}

\begin{cor}
\label{cor:p-not-divide-fixed-point}
Let $p$ be a prime. Let $S$ be a finite set. Let $G$ be a $p$-group acting on $S$. If $p$ does not divide $|S|$ then the action has a fixed point.
\end{cor}

\section{Category of $G$-Sets}

\begin{df}
Given a group $G$, let $G-\mathbf{Set}$ be the category with:
\begin{itemize}
\item objects all pairs $(A, \rho)$ such that $A$ is a set and $\rho : G \times A \rightarrow A$ is an action of $G$ on $A$;
\item morphisms $f : (A, \rho) \rightarrow (B, \sigma)$ are functions $f : A \rightarrow B$ that are \emph{($G$-)equivariant}, i.e.
\[ \forall g \in G. \forall a \in A. f(\rho(g,a)) = \sigma(g,f(a)) \enspace . \]
\end{itemize}
\end{df}

\begin{prop}
A $G$-equivariant function $f : A \rightarrow B$ is an isomorphism in $G-\mathbf{Set}$ if and only if it is bijective.
\end{prop}

\begin{proof}
\pf
\step{1}{\pflet{$f : A \rightarrow B$ be $G$-equivariant and bijective.} \prove{$\inv{f}$ is $G$-equivariant.}}
\step{2}{\pflet{$g \in G$ and $b \in B$}}
\step{3}{$\inv{f}(gb) = g \inv{f}(b)$}
\begin{proof}
\pf
\begin{align*}
f(\inv{f}(gb)) & = gb \\
& = g f(\inv{f}(b)) \\
& = f(g \inv{f}(b))
\end{align*}
\end{proof}
\qed
\end{proof}

\begin{prop}
Let $G$ be a group and $A$ a transitive $G$-set. Let $a \in A$. Then $A$ is isomorphic to $G / \Stab{G}{a}$ under left multiplication.
\end{prop}

\begin{proof}
\pf
\step{1}{\pflet{$f : G / \Stab{G}{a} \rightarrow A$ be the function $f(g \Stab{G}{a}) = ga$.}}
\begin{proof}
	\step{a}{\assume{$g \Stab{G}{a} = h \Stab{G}{a}$} \prove{$ga = ha$}}
	\step{b}{$\inv{g} h \in \Stab{G}{a}$}
	\step{c}{$\inv{g} h a = a$}
	\step{d}{$ha = ga$}
\end{proof}
\step{2}{$f$ is $G$-equivariant.}
\begin{proof}
	\pf\ Since $f(gh \Stab{G}{a}) = gha = g f(h \Stab{G}{a})$.
\end{proof}
\step{3}{$f$ is injective.}
\begin{proof}
	\pf\ If $ga = ha$ then $\inv{g}h \in \Stab{G}{a}$ so $g \Stab{G}{a} = h \Stab{G}{a}$.
\end{proof}
\step{4}{$f$ is surjective.}
\begin{proof}
	\pf\ Since for all $b \in A$ there exists $g \in G$ such that $ga = b$.
\end{proof}
\qed
\end{proof}

\begin{cor}
If $O$ is an orbit of the action of a finite group $G$ on a set $A$, then $O$ is finite and $|O|$ divides $|G|$.
\end{cor}

\begin{cor}
Let $H$ be a subgroup of $G$ and $g \in G$. Then
\[ G / H \cong G / (g H \inv{g}) \]
in $G-\mathbf{Set}$.
\end{cor}

\begin{proof}
\pf\ Taking $A = G / H$ and $a = gH$. \qed
\end{proof}

\begin{prop}
Given a family of $G$-sets $\{ A_i \}_{i \in I}$, we have $\prod_{i \in I} A_i$ is their product in $G-\mathbf{Set}$ under
\[ g \{a_i\}_{i \in I} = \{ga_i\}_{i \in I} \enspace . \]
\end{prop}

\begin{proof}
\pf\ Easy. \qed
\end{proof}

\begin{prop}
Given a family of $G$-sets $\{ A_i \}_{i \in I}$, we have $\coprod_{i \in I} A_i$ is their product in $G-\mathbf{Set}$ under
\[ g (i,a_i) = (i, ga_i) \enspace . \]
\end{prop}

\begin{proof}
\pf\ Easy. \qed
\end{proof}

\begin{prop}
Every finite $G$-set is a coproduct of $G$-sets of the form $G/H$.
\end{prop}

\begin{proof}
\pf\ If $O(a_1)$, \ldots, $O(a_n)$ are the orbits of the $G$-set $A$, then $G$ is the coproduct of $G / \Stab{G}{a_1}$, \ldots, $G / \Stab{G}{a_n}$. \qed
\end{proof}

\begin{prop}
For any group $G$ we have $G \cong \Aut{G-\mathbf{Set}}{G}$ (considering $G$ as a $G$-set under left multiplication).
\end{prop}

\begin{proof}
\pf
\step{1}{Define $\phi : G \rightarrow \Aut{G-\mathbf{Set}}{G}$ by $\phi(g)(g') = g'\inv{g}$.}
\begin{proof}
	\step{a}{\pflet{$g \in G$} \prove{$\lambda g' \in G. g'\inv{g}$ is an automorphism of $G$ in $G-\mathbf{Set}$.}}
	\step{b}{$\phi(g)$ is $G$-equivariant.}
	\begin{proof}
		\pf\ Since $\phi(g)(h_1h_2) = h_1h_2\inv{g} = h_1 \phi(g)(h_2)$.
	\end{proof}
	\step{c}{$\phi(g)$ is injective.}
	\begin{proof}
		\pf\ By Cancellation.
	\end{proof}
	\step{d}{$\phi(g)$ is surjective.}
	\begin{proof}
		\pf\ For any $h \in G$ we ahev $h = \phi(g)(hg)$.
	\end{proof}
\end{proof}
\step{2}{$\phi$ is a group homomorphism.}
\begin{proof}
	\pf\ $\phi(g_1 g_2)(h) = h\inv{g_2} \inv{g_1} = \phi(g_1)(\phi(g_2)(h))$.
\end{proof}
\step{3}{$\phi$ is injective.}
\begin{proof}
	\pf\ If $\phi(g) = \phi(g')$ then $g = \phi(g)(e) = \phi(g')(e) = g'$.
\end{proof}
\step{4}{$\phi$ is surjective.}
	\begin{proof}
		\step{i}{\pflet{$\sigma \in \Aut{G-\mathbf{Set}}{G}$}}
		\step{ii}{\pflet{$g = \sigma(e)$} \prove{$\sigma = \phi(\inv{g})$}}
		\step{iii}{$\sigma(h) = hg$}
		\begin{proof}
		\pf\ $\sigma(h) = \sigma(he) = h \sigma(e) = hg$.
		\end{proof}
	\end{proof}
\qed
\end{proof}

\section{Center}

\begin{df}[Center]
The \emph{center} of a group $G$, $Z(G)$, is the kernel of the conjugation action $\sigma : G \rightarrow S_G$.
\end{df}

\begin{prop}
The center of a group $G$ is
\[ Z(G) = \{ g \in G : \forall a \in G. ag = ga \} \enspace . \]
\end{prop}

\begin{proof}
\pf\ Immediate from definitions. \qed
\end{proof}

\begin{lm}
\label{lm:G-ZG-cyclic}
Let $G$ be a finite group. Assume $G / Z(G)$ is cyclic. Then $G$ is Abelian and so $G/Z(G)$ is trivial.
\end{lm}

\begin{proof}
\pf
\step{1}{\pick\ $g \in G$ such that $gZ(G)$ generates $G/Z(G)$.}
\step{2}{\pflet{$a,b \in G$}}
\step{3}{\pick\ $r,s \in \mathbb{Z}$ such that $aZ(G) = g^rZ(G)$ and $bZ(G) = g^sZ(G)$}
\step{4}{\pflet{$z = g^{-r}a \in Z(G)$ and $w = g^{-s}b \in Z(G)$}}
\step{5}{$a = g^rz$ and $b = g^sw$}
\step{6}{$ab = ba$}
\begin{proof}
	\pf
	\begin{align*}
		ab & = g^rzg^sw \\
		& = g^{r+s}zw \\
		& = g^swg^rz \\
		& = ba
	\end{align*}
\end{proof}
\qed
\end{proof}

\begin{prop}
\label{prop:subgroup-of-Z-normal}
Let $G$ be a group. Let $N$ be a subgroup of $Z(G)$. Then $N$ is normal in $G$.
\end{prop}

\begin{proof}
\pf\ For all $n \in N$ and $g \in G$ we have $gn\inv{g} = ng\inv{g} = n \in N$ since $n \in Z(G)$. \qed
\end{proof}

\begin{prop}
For any group $G$ we have $G / Z(G) \cong \Inn{G}$.
\end{prop}

\begin{proof}
\pf\ The homomorphism $g \mapsto \gamma_g$ is a surjective homomorphism with kernel $Z(G)$. \qed
\end{proof}

\begin{prop}
\label{prop:pq-abelian}
Let $p$ and $q$ be prime integers. Let $G$ be a group of order $pq$. Then either $G$ is Abelian or the center of $G$ is trivial.
\end{prop}

\begin{proof}
\pf\ Otherwise we would have $|Z(G)| = p$ say and so $|\Inn{G}| = q$, meaning $\Inn{G}$ is cyclic, hence trivial, which is a contradiction.
\qed
\end{proof}

\begin{thm}[First Sylow Theorem]
Let $p$ be a prime and $k \in \mathbb{N}$. Let $G$ be a finite group. If $p^k$ divides $|G|$ then $G$ has a subgroup of order $p^k$.
\end{thm}

\begin{proof}
\pf
\step{1}{\assume{as induction hypothesis the statement is true for all groups smaller than $G$.}}
\step{2}{\assume{w.l.o.g. $k \neq 0$ and $|G| \neq p$}}
\step{3}{\case{There exists a proper subgroup $H$ of $G$ such that $p$ does not divide $[G:H]$.}}
\begin{proof}
	\pf\ Then $H$ has a subgroup of order $p^k$ by induction hypothesis \stepref{1}.
\end{proof}
\step{4}{\case{For every proper subgroup $H$ of $G$ we have $p$ divides $[G:H]$.}}
\begin{proof}
	\step{a}{$p$ divides $|Z(G)|$.}
	\begin{proof}
		\pf\ By the Class Formula.
	\end{proof}
	\step{b}{\pick\ $a \in Z(G)$ that has order $p$.}
	\begin{proof}
		\pf\ Cauchy's Theorem.
	\end{proof}
	\step{c}{\pflet{$N = \langle a \rangle$}}
	\step{d}{$N$ is normal.}
	\begin{proof}
		\pf\ Proposition \ref{prop:subgroup-of-Z-normal}.
	\end{proof}
	\step{e}{$p^{k-1}$ divides $|G/N|$.}
	\step{f}{\pick\ a subgroup $Q$ of $G/N$ of order $p^{k-1}$.}
	\begin{proof}
		\pf\ Induction hypothesis \stepref{1}.
	\end{proof}
	\step{g}{\pflet{$P = \inv{\pi}(Q)$}}
	\step{h}{$|P| = p^k$}
\end{proof}
\qed
\end{proof}

\begin{thm}[Second Sylow Theorem]
Let $G$ be a finite group. Let $p$ be a prime. Let $P$ be a $p$-Sylow subgroup of $G$. Let $H$ be a subgroup of $G$ that is a $p$-group. Then $H$ is a subgroup of a conjugate of $P$.
\end{thm}

\begin{proof}
\pf
\step{1}{\pick\ a fixed point $gP$ for the action of $H$ on the set of left cosets of $P$ by left multiplication.}
\begin{proof}
	\pf\ Corollary \ref{cor:p-not-divide-fixed-point}.
\end{proof}
\step{2}{For all $h \in H$ we have $hgP = gP$}
\step{3}{$H \subseteq gP\inv{g}$}
\qed
\end{proof}

\section{Centralizer}

\begin{df}[Centralizer]
Let $G$ be a group. Let $a \in G$. The \emph{centralizer} or \emph{normalizer} of $a$, denoted $Z_G(a)$, is the stabilizer of $a$ under the action of conjugation.
\end{df}

\begin{prop}
\[ Z_G(a) = \{ g \in G : ga = ag \} \]
\end{prop}

\begin{proof}
\pf\ Immediate from definitions. \qed
\end{proof}

\section{Conjugacy Class}

\begin{df}[Conjugacy Class]
Let $G$ be a group. Let $a \in G$. The \emph{conjugacy class} of $a$, denoted $[a]$, is the orbit of $a$ under the action of conjugation.
\end{df}

\begin{prop}[Class Formula]
Let $G$ be a finite group. Let $A$ be a set of representatives of the non-trivial conjugacy classes. Then
\[ |G| = |Z(G)| + \sum_{a \in A} [G : Z(a)] \enspace . \]
\end{prop}

\begin{proof}
\pf\ Proposition \ref{prop:class-formula}. \qed
\end{proof}

\begin{cor}
Let $p$ be a prime. Let $G$ be a $p$-group and $H$ a nontrivial normal subgroup of $G$. Then $H \cap Z(G) \neq \{e\}$.
\end{cor}

\begin{proof}
\pf\ Let $A$ be a set of representatives of the non-trivial conjugacy classes. Let $A \cap H = \{a_1, \ldots, a_n\}$. Then
\[ |H| = |H \cap Z(G)| + \sum_{i=1}^n [G : Z(a_i)] \enspace . \]
Since $p \mid |H|$ and $p \mid [G : Z(a_i)]$ for all $i$, we have $p \mid |H \cap Z(G)|$. \qed
\end{proof}

\begin{cor}
\label{cor:p-group-non-trivial-center}
Let $p$ be a prime.
Every $p$-group has a non-trivial center.
\end{cor}

\begin{cor}
Let $p$ be a prime. Every group $G$ of order $p^2$ is Abelian.
\end{cor}

\begin{proof}
\pf\ By Proposition \ref{prop:pq-abelian}. \qed
\end{proof}

\begin{prop}
Let $p$ be a prime and $r$ a non-negative integer. Let $G$ be a group of order $p^r$. Then, for $k = 0, 1, \ldots, r$, we have $G$ has a normal subgroup of order $p^k$.
\end{prop}

\begin{proof}
\pf
\step{1}{\assume{as induction hypothesis the result holds for $r' < r$.}}
\step{2}{\assume{w.l.o.g. $k > 0$}}
\begin{proof}
	\pf\ Since $\{e\}$ is a normal subgroup of order $p^0$.
\end{proof}
\step{2}{\pick\ a subgroup $N$ of $Z(G)$ of order $p$.}
\begin{proof}
	\step{a}{$p \mid |Z(G)|$}
	\begin{proof}
		\pf\ From Corollary \ref{cor:p-group-non-trivial-center}.
	\end{proof}
	\step{b}{$Z(G)$ has a subgroup of order $p$.}
	\begin{proof}
		\pf\ Cauchy's Theorem.
	\end{proof}
\end{proof}
\step{3}{$N$ is normal.}
\begin{proof}
	\pf\ Proposition \ref{prop:subgroup-of-Z-normal}.
\end{proof}
\step{4}{\pick\ a normal subgroup $M$ of $G/N$ of order $p^{k-1}$.}
\begin{proof}
	\pf\ From the induction hypothesis \stepref{1}.
\end{proof}
\step{5}{$\inv{\pi}(M)$ is a normal subgroup of $G$ of order $p^k$.}
\qed
\end{proof}

\begin{ex}
The only non-Abelian group of order 6 is $S_3$.
\end{ex}

\begin{proof}
\pf
\step{1}{\pflet{$G$ be a non-Adelian group of order 6.}}
\step{2}{$Z(G) = \{e\}$}
\begin{proof}
	\pf\ Otherwise $Z(G)$ has order 2 or 3 and is cyclic, contradicting Lemma \ref{lm:G-ZG-cyclic}.
\end{proof}
\step{3}{$G$ has three conjugacy classes: $Z(G)$, a class of size 2 and a class of size 3.}
\begin{proof}
	\pf\ By the Class Formula since the only way to make 5 using non-trivial factors of 6 is $2+3$.
\end{proof}
\step{4}{\pick\ an element $y \in G$ of order 3.}
\begin{proof}
	\pf\ It cannot be that every element is of order $\leq 2$ by Proposition \ref{prop:all-orders-leq-two}.
\end{proof}
\step{5}{$\langle y \rangle$ is normal in $G$.}
\begin{proof}
	\pf\ Since it has index 2.
\end{proof}
\step{6}{The conjugacy class $y$ is $\{y,y^2\}$.}
\begin{proof}
	\pf\ Since $\langle y \rangle$ must be a union of conjugacy classes.
\end{proof}
\step{6}{The conjugacy class of size 2 is $\{y,y^2\}$.}
\begin{proof}
	\pf\ Since $y^2$ has order 3 and so its conjugacy class is of size 2 similarly, and there is only one conjugacy class of size 2.
\end{proof}
\step{7}{\pick\ $x \in G$ such that $yx = xy^2$.}
\begin{proof}
	\pf\ $y^2$ is conjugate to $y$ so there exists $x$ such that $\inv{x}yx = y^2$.
\end{proof}
\step{8}{$x$ has order 2.}
\begin{proof}
	\pf\ $x$ is not in the conjugacy class of size 2 so its order cannot be 3.
\end{proof}
\step{9}{$x$ and $y$ generate $G$.}
\begin{proof}
	\pf\ Since $e$, $y$, $y^2$, $x$, $xy$, $xy^2$ are all distinct.
\end{proof}
\step{10}{$G \cong S_3$}
\begin{proof}
	\pf\ We now know the entire multiplication table of $G$.
\end{proof}
\qed
\end{proof}

\begin{prop}
Let $G$ be a finite group. Let $H$ be a subgroup of $G$ of order 2. Let $a \in H$. Let $[a]_H$ be the conjugacy class of $a$ in $H$, and $[a]_G$ the conjugacy class of $a$ in $G$. If $Z_G(a) \subseteq H$ then $[a]_H$ is half the size of $[a]_G$; otherwise, $[a]_H = [a]_G$.
\end{prop}

\begin{proof}
\pf
\step{1}{$H$ is normal in $G$.}
\begin{proof}
	\pf\ Corollary \ref{cor:index-two-normal}.
\end{proof}
\step{2}{$HZ_G(a)$ is a subgroup of $G$.}
\step{3}{$H$ is normal in $HZ_G(a)$.}
\step{4}{$H \cap Z_G(a)$ is normal in $Z_G(a)$.}
\step{5}{\[ \frac{HZ_G(a)}{H} \cong \frac{Z_G(a)}{H \cap Z_G(a)} \]}
\step{6}{If $Z_G(a) \subseteq H$ then $|[a]_H| = |[a]_G|/2$.}
\begin{proof}
	\pf\ In this case we have $Z_H(a) = Z_G(a)$ and so $|[a]_H| = |H| / |Z_H(a)| = (|G|/2)/|Z_G(a)| = |[a]_G|/2$.
\end{proof}
\step{7}{If $Z_G(a) \nsubseteq H$ then $[a]_H = [a]_G$.}
\begin{proof}
	\pf
	\step{a}{\pick\ $b \in Z_G(a) - H$}
	\step{b}{$H\inv{b} = G - H$}
	\step{c}{$G = H Z_G(a)$}
	\begin{proof}
		\pf\ For $x \in H$ we have $x = xe$ and for $x \notin H$ we have $x \in H\inv{b}$ hence $xb \in H$ and $x = (xb)b$.
	\end{proof}
	\step{b}{$|[a]_H| = |[a]_G|$}
	\begin{proof}
	\pf
	\begin{align*}
		|[a]_H| & = \frac{|H|}{|Z_H(a)|} \\
		& = \frac{|H|}{|H \cap Z_G(a)|} \\
		& = \frac{|Z_G(a)||H|}{|Z_G(a)||H \cap Z_G(a)|} \\
		& = \frac{|HZ_G(a)|}{|Z_G(a)|} \\
		& = \frac{|G|}{|Z_G(a)|} \\
		& = |[a]_G|
	\end{align*}
	\end{proof}
\end{proof}
\qed
\end{proof}

\section{Conjugation on Sets}

\begin{df}[Conjugation]
Let $G$ be a group. Define an action of $G$ on $\mathcal{P} G$ called \emph{conjugation} that takes $g$ and $A$ to
\[ gA\inv{g} = \{ ga\inv{g} : a \in A \} \enspace . \]
\end{df}

\begin{prop}
The conjugate of a subgroup is a subgroup.
\end{prop}

\begin{proof}
\pf\ Let $H$ be a subgroup of $G$. Given $gh_1\inv{g},gh_2\inv{g} \in gH\inv{g}$, we have
\[ (gh_1\inv{g})\inv{(gh_2\inv{g})} = gh_1\inv{h_2}\inv{g} \in gH\inv{g} \enspace . \qquad \qed \]
\end{proof}

\begin{df}[Normalizer]
Let $G$ be a group and $A \subseteq G$. The \emph{normalizer} of $A$, denoted $N_G(A)$, is its stabilizer under conjugation.
\end{df}

\begin{prop}
Let $G$ be a group, $g \in G$ and $A$ a finite subset of $G$. If $gA\inv{g} \subseteq A$ then $gA\inv{g} = A$ and so $g \in N_G(A)$.
\end{prop}

\begin{proof}
\pf\ Conjugation by $g$ is an injection from $A$ into $A$, hence a bijection. \qed
\end{proof}

%TODO Example where $A$ is infinite and this is not true.

\begin{prop}
\label{prop:H-normal-in-NGH}
Let $G$ be a group and $H$ a subgroup of $G$. Then $N_G(H)$ is the largest subgroup of $G$ that includes $H$ such that $H$ is normal in $N_G(H)$.
\end{prop}

\begin{proof}
\pf
\step{0}{$N_G(H)$ is a subgroup of $G$.}
\begin{proof}
	\pf\ If $a,b \in N_G(H)$ then $a\inv{b}Hb\inv{a} = aH\inv{a} = H$ so $a\inv{b} \in N_G(H)$.
\end{proof}
\step{1}{$H \subseteq N_G(H)$}
\begin{proof}
	\pf\ Easy.
\end{proof}
\step{2}{$H$ is normal in $N_G(H)$.}
\begin{proof}
	\pf\ If $a \in N_G(H)$ then $aH\inv{a} = H$ by definition.
\end{proof}
\step{3}{For any subgroup $K$ of $G$, if $H \subseteq K$ and $H$ is normal in $K$ then $K \subseteq N_G(H)$.}
\begin{proof}
	\pf\ $H$ is normal in $K$ means that, for all $a \in K$, we have $aH\inv{a} = H$ and so $a \in N_G(H)$.
\end{proof}
\qed
\end{proof}

\begin{cor}
\label{cor:normal-NG}
Let $G$ be a group and $H$ a subgroup of $G$. Then $H$ is normal if and only if $G = N_G(H)$.
\end{cor}

\begin{prop}
\label{prop:number-of-subgroup-conjugate-to-H}
Let $G$ be a group and $H$ a subgroup of $G$. If $[G:N_G(H)]$ is finite, then it is the number of subgroups conjugate to $H$.
\end{prop}

\begin{proof}
\pf\ By the Orbit-Stabilizer Theorem. \qed
\end{proof}

\begin{cor}
\label{cor:number-of-conjugates-divides-index}
Let $G$ be a group and $H$ a subgroup of $G$. If $[G:H]$ is finite, the the number of subgroups conjugate to $H$ is finite and divides $[G:H]$.
\end{cor}

\begin{lm}
\label{lm:index-of-H-in-NGH}
Let $H$ be a $p$-group that is a subgroup of a finite group $G$. Then
\[ [N_G(H) : H] \equiv [G:H] (\mod p) \enspace . \]
\end{lm}

\begin{proof}
\pf
\step{1}{\assume{w.l.o.g. $H$ is non-trivial.}}
\step{2}{$gH$ is a fixed point of the action of $H$ on the set of left cosets of $H$ by left multiplication if and only if $g \in N_G(H)$.}
\begin{proof}
	\pf
	\begin{align*}
		gH \text{ is a fixed point}
		& \Leftrightarrow \forall h \in H. hgH = gH \\
		& \Leftrightarrow H \subseteq gH\inv{g} \\
		& \Leftrightarrow H = gH\inv{g} & (|gH\inv{g}| = |H|) \\
		& \Leftrightarrow g \in N_G(H)
	\end{align*}
\end{proof}
\step{3}{The number of fixed points in $[N_G(H) : H]$.}
\qedstep
\begin{proof}
	\pf\ Corollary \ref{cor:fixed-points-of-action-of-p-group}.
\end{proof}
\qed
\end{proof}

\begin{prop}
Let $H$ be a $p$-subgroup of a finite group $G$ that is not a $p$-Sylow subgroup. Then there exists a $p$-subgroup $H'$ of $G$ such that $H$ is a normal subgroup of $H'$ and $[H':H] = p$.
\end{prop}

\begin{proof}
\pf
\step{1}{$p$ divides $[N_G(H) : H]$.}
\begin{proof}
	\pf\ Lemma \ref{lm:index-of-H-in-NGH}.
\end{proof}
\step{2}{\pick\ $gH \in N_G(H) /H$ of order $p$.}
\begin{proof}
	\pf\ Cauchy's Theorem.
\end{proof}
\step{3}{\pflet{$H' = \inv{\pi}(\langle gH \rangle)$}}
\step{4}{$H$ is a normal subgroup of $H'$.}
\step{5}{$[H':H] = p$}
\qed
\end{proof}

\begin{cor}
\label{cor:p-squared-not-simple}
No $p$-group of order $\geq p^2$ is simple.
\end{cor}

\begin{lm}
\label{lm:third-sylow-1}
Let $p$ be a prime. Let $G$ be a finite group. Let $P$ be a $p$-Sylow subgroup of $G$. Every $p$-subgroup of $N_G(P)$ is a subgroup of $P$.
\end{lm}

\begin{proof}
\pf
\step{1}{\pflet{$H$ be a $p$-subgroup of $N_G(P)$.}}
\step{2}{$P$ is normal in $N_G(P)$.}
\begin{proof}
	\pf\ Proposition \ref{prop:H-normal-in-NGH}.
\end{proof}
\step{3}{$PH$ is a subgroup of $N_G(P)$.}
\begin{proof}
	\pf\ Second Isomorphism Theorem.
\end{proof}
\step{4}{$|PH/P| = |H/(P \cap H)|$}
\begin{proof}
	\pf\ Second Isomorphism Theorem.
\end{proof}
\step{5}{$PH$ is a $p$-group.}
\begin{proof}
	\step{a}{\assume{for a contradiction $q$ is prime, $q \mid |PH|$ and $q \neq p$}}
	\step{b}{$q \mid |PH/P|$}
	\step{c}{$q \mid |H/(P \cap H)|$}
	\step{d}{$q \mid |H|$}
	\qedstep
	\begin{proof}
		\pf\ This contradicts the fact that $H$ is a $p$-group, \stepref{1}.
	\end{proof}
\end{proof}
\step{6}{$PH=P$}
\begin{proof}
	\pf\ By maximality of $P$.
\end{proof}
\step{7}{$H \subseteq P$}
\qed
\end{proof}

\begin{lm}
\label{lm:third-sylow-2}
Let $p$ be a prime. Let $G$ be a finite group. Let $P$ be a $p$-Sylow subgroup of $G$. Let $P$ act by conjugation on the set of $p$-Sylow subgroups of $G$. Then $P$ is the unique fixed point of this action.
\end{lm}

\begin{proof}
\pf
\step{1}{$P$ is a fixed point of this action.}
\begin{proof}
	\pf\ For any $x \in P$ we have $xP\inv{x} = P$.
\end{proof}
\step{2}{If $Q$ is any fixed point of the action then $Q = P$.}
\begin{proof}
	\step{a}{\pflet{$Q$ be a fixed point of the action.}}
	\step{b}{For all $x \in P$ we have $xQ\inv{x} = Q$.}
	\step{c}{$P \subseteq N_G(Q)$}
	\step{d}{$P \subseteq Q$}
	\begin{proof}
		\pf\ Lemma \ref{lm:third-sylow-1}.
	\end{proof}
	\step{e}{$P = Q$}
	\begin{proof}
		\pf\ Since $|P| = |Q|$.
	\end{proof}
\end{proof}
\qed
\end{proof}

\begin{thm}[Third Sylow Theorem]
Let $p$ be a prime. Let $G$ be a finite group of order $p^rm$ where $p$ does not divide $m$. Then the number of $p$-Sylow subgroups of $G$ divides $m$ and is congruent to 1 modulo $p$.
\end{thm}

\begin{proof}
\pf
\step{1}{\pflet{$N_p$ be the number of $p$-Sylow subgroups of $G$.}}
\step{2}{\pick\ a $p$-Sylow subgroup $P$.}
\begin{proof}
	\pf\ One exists by the First Sylow Theorem.
\end{proof}
\step{3}{The $p$-Sylow subgroups of $G$ are exactly the conjugates of $P$.}
\begin{proof}
	\pf\ Second Sylow Theorem
\end{proof}
\step{3a}{$m = N_p [N_G(P) : P]$}
\begin{proof}
	\pf\ Since $N_p = [G : N_G(P)]$ by Proposition \ref{prop:number-of-subgroup-conjugate-to-H}.
\end{proof}
\step{4}{$N_p$ divides $m$.}
\step{5}{$mN_p \equiv m (\mod p)$}
\begin{proof}
	\step{a}{$m \equiv [N_G(P) : P] (\mod p)$}
	\begin{proof}
		\pf\ Lemma \ref{lm:index-of-H-in-NGH}.
	\end{proof}
	\step{b}{$mN_p \equiv m (\mod p)$}
	\begin{proof}
		\pf\ By \stepref{3a}.
	\end{proof}
\end{proof}
\step{6}{$N_p \equiv 1 (\mod p)$}
\qed
\end{proof}

\begin{proof}
\pf
\step{1}{\pflet{$N_p$ be the number of $p$-Sylow subgroups of $G$.}}
\step{2}{\pick\ a $p$-Sylow subgroup $P$ of $G$.}
\begin{proof}
	\pf\ First Sylow Theorem.
\end{proof}
\step{3}{$N_p$ is the number of conjugates of $P$.}
\begin{proof}
	\pf\ Second Sylow Theorem.
\end{proof}
\step{4}{$N_p \mid m$}
\begin{proof}
	\pf\ Corollary \ref{cor:number-of-conjugates-divides-index}.
\end{proof}
\step{5}{$P$ acts on the set of conjugates of $P$ with one fixed point.}
\begin{proof}
	\pf\ Lemma \ref{lm:third-sylow-2}.
\end{proof}
\step{6}{$N_p \equiv 1 (\mod p)$}
\begin{proof}
	\pf\ Corollary \ref{cor:fixed-points-of-action-of-p-group}.
\end{proof}
\qed
\end{proof}

\begin{cor}
\label{cor:mpr-not-simple1}
Let $G$ be a finite group. Let $p$ be a prime number. If $|G| = mp^r$ and the only divisor $d$ of $m$ such that $d \equiv 1 (\mod p)$ is $d = 1$, then $G$ is not simple.
\end{cor}

\begin{proof}
\pf\ There must be 1 $p$-Sylow subgroup, which has order $p^r$ and is normal.
\qed
\end{proof}

\begin{cor}
\label{cor:mpr-not-simple}
Let $G$ be a finite group. Let $p$ be a prime number. If $|G| = mp^r$ where $1 < m < p$ then $G$ is not simple.
\end{cor}

\begin{prop}
Let $p$ and $q$ be prime numbers with $p < q$. Let $G$ be a group of order $pq$ with a normal subgroup $H$ of order $p$. Then $G$ is cyclic.
\end{prop}

\begin{proof}
\pf
\step{1}{\pflet{$\gamma : G \rightarrow \Aut{\Grp}{H}$ be the action of conjugation.}}
\step{2}{$H$ is cyclic of order $p$.}
\step{3}{$|\Aut{\Grp}{H}| = p-1$}
\step{4}{$|\im \gamma| \mid pq$}
\begin{proof}
	\pf\ Since $\im \gamma$ is a quotient group of $G$.
\end{proof}
\step{5}{$|\im \gamma| \mid p-1$}
\step{5a}{$|\im \gamma| = 1$}
\step{6}{$\gamma = 0$}
\step{7}{$H \subseteq Z(G)$}
\step{8}{$G$ is Abelian.}
\begin{proof}
	\pf\ Lemma \ref{lm:G-ZG-cyclic}.
\end{proof}
\step{9}{\pick\ an element $g$ of order $p$.}
\begin{proof}
	\pf\ Cauchy's Theorem.
\end{proof}
\step{10}{\pick\ an element $h$ of order $q$.}
\begin{proof}
	\pf\ Cauchy's Theorem.
\end{proof}
\step{11}{$|gh| = pq$}
\begin{proof}
	\pf\ Proposition \ref{prop:order-gh-if-gcd-one}.
\end{proof}
\qed
\end{proof}

\begin{proof}
\pf
\step{1}{\assume{w.l.o.g. $q \not\equiv 1 (\mod p)$}}
\begin{proof}
	\pf\ Since the only non-cyclic group of order 6 is $S_3$ which does not have a normal subgroup of order 2.
\end{proof}
\step{2}{\pick\ a subgroup $K$ of order $q$.}
\step{4}{$K$ is normal.}
\begin{proof}
	\pf\ Since $K$ is the unique $q$-Sylow subgroup by the Third Sylow Theorem.
\end{proof}
\step{5}{$H \cap K = \{e\}$}
\step{6}{$HK \cong H \times K$}
\begin{proof}
	\pf\ Proposition \ref{HK-cong-H-times-K}.
\end{proof}
\step{7}{$|HK| = pq$}
\step{8}{$HK = G$}
\step{11}{$G \cong \mathbb{Z} / pq \mathbb{Z}$}
\begin{proof}
	\pf
	\begin{align*}
		G & \cong H \times K \\
		& \cong \mathbb{Z} / p \mathbb{Z} \times \mathbb{Z} / q \mathbb{Z} \\
		& \cong \mathbb{Z} / pq \mathbb{Z}
	\end{align*}
\end{proof}
\qed
\end{proof}

\begin{cor}
Let $p$ and $q$ be prime numbers with $p < q$ and $q \not\equiv 1 (\mod p)$. Then the only group of order $pq$ is the cyclic group.
\end{cor}

\begin{proof}
\pf\ By the Third Sylow Theorem, such a group must have exactly one $p$-Sylow subgroup, which is therefore normal. \qed
\end{proof}

\begin{prop}
\label{prop:NG-NG}
Let $p$ be prime. Let $G$ be a finite group. Let $P$ be a $p$-Sylow subgroup of $G$. Then
\[ N_G(N_G(P)) = N_G(P) \enspace . \]
\end{prop}

\begin{proof}
\pf
\step{1}{$P$ is normal in $N_G(P)$.}
\begin{proof}
	\pf\ Proposition \ref{prop:H-normal-in-NGH}.
\end{proof}
\step{2}{$N_G(P)$ is normal in $N_G(N_G(P))$.}
\begin{proof}
	\pf\ Proposition \ref{prop:H-normal-in-NGH}.
\end{proof}
\step{3}{$P$ is normal in $N_G(N_G(P))$.}
\begin{proof}
	\pf\ Corollary \ref{cor:normal-normal-normal}.
\end{proof}
\step{5}{$N_G(N_G(P)) \subseteq N_G(P)$}
\begin{proof}
	\pf\ Proposition \ref{prop:H-normal-in-NGH}.
\end{proof}
\step{6}{$N_G(N_G(P)) = N_G(P)$}
\qed
\end{proof}

\begin{prop}
\label{prop:pqr-not-simple}
Let $p$, $q$ and $r$ be three distinct prime numbers.	Then there is no simple group of order $pqr$.
\end{prop}

\begin{proof}
\pf
\step{1}{\pflet{$G$ be a group of order $pqr$.}}
\step{2}{\assume{w.l.o.g. $p < q < r$}}
\step{3}{\assume{for a contradiction $G$ is simple.}}
\step{4}{The number of subgroups of order $p$ is at least $p+1$.}
\begin{proof}
	\pf\ Third Sylow Theorem
\end{proof}
\step{5}{The number of subgroups of order $q$ is at least $q+1$.}
\begin{proof}
	\pf\ Third Sylow Theorem
\end{proof}
\step{6}{The number of subgroups of order $r$ is $pq$.}
\begin{proof}
	\pf\ By the Third Sylow Theorem, the number divides $pq$, and it cannot be 1 (lest that subgroup be normal) or $p$ or $q$ (as these are less than $r$ hence not congruent to 1 modulo $r$).
\end{proof}
\step{7}{There are at least $p^2-1$ elements of order $p$.}
\step{8}{There are at least $q^2 - 1$ elements of order $q$.}
\step{9}{There are at least $pqr-pq$ elements of order $r$.}
\qedstep
\begin{proof}
	\pf\ This is a contradiction as the total number of elements of order 1, $p$, $q$ and $r$ is
	\begin{align*}
		1 + (p^2 - 1) + (q^2 - 1) + (pqr - pq)
		& = p^2 + q^2 + pqr - pq - 1 \\
		& > pqr + p^2 - 1 \\
		& > pqr
	\end{align*}
\end{proof}
\qed
\end{proof}

\begin{prop}
Let $G$ be a finite simple group. Let $H$ be a subgroup of $G$ of index $N > 1$. Then $|G|$ divides $N!$.
\end{prop}

\begin{proof}
\pf
\step{1}{\pick\ a subgroup $K$ of $H$ that is normal in $G$ such that $[G:K]$ divides $\gcd(|G|,N!)$.}
\step{2}{$K = \{e\}$}
\step{3}{$[G:K] = |G|$}
\step{4}{$|G|$ divides $N!$}
\qed
\end{proof}

\begin{cor}
\label{cor:G-divides-Npfac}
Let $G$ be a finite simple group. Let $p$ be a prime factor of $|G|$.
Let $N_p$ be the number of $p$-Sylow subgroups of $G$. Then $|G|$ divides $N_p!$.
\end{cor}

\begin{proof}
\pf\ Since $N_p = [G : N_G(P)]$ and $N_p > 1$ since $G$ is simple. \qed
\end{proof}

\begin{df}[Centralizer]
Let $G$ be a group and $A \subseteq G$. The \emph{centralizer} of $A$ is
\[ Z_G(A) := \{ g \in G : \forall a \in A. ga\inv{g} = a \} \enspace . \]
\end{df}

%TODO Example where $Z_G(A) \neq N_G(A)$.

\begin{prop}
Let $H$ and $K$ be subgroups of $G$ with $H \subseteq N_G(K)$. Then the function $\gamma : H \rightarrow \Aut{\Grp}{K}$ defined by conjugation
\[ \gamma_h(k) = hk\inv{h} \]
is a homomorphism of groups with $\ker \gamma = H \cap Z_G(K)$.
\end{prop}

\begin{proof}
\pf
\step{1}{For all $g,h \in H$ we have $\gamma_{gh} = \gamma_g \circ \gamma_h$.}
\begin{proof}
	\pf\ Since $\gamma_{gh}(k) = \gamma_g(\gamma_h(k)) = ghk\inv{h}\inv{g}$.
\end{proof}
\step{2}{For all $h \in H$ we have $\gamma_h = \id{K}$ iff $h \in Z_G(K)$.}
\begin{proof}
	\pf\ Both are equivalent to $\forall k \in K. hk\inv{h} = k$, i.e. $\forall k \in K. hk = kh$.
\end{proof}
\qed
\end{proof}

\section{Nilpotent Groups}

\begin{df}[Nilpotent]
Let $G$ be a group. Define inductively a sequence $(Z_n)$ of subgroups of $G$ by $Z_0 = \{e\}$, and $Z_{i+1}$ is the inverse image under $\pi$ of the center of $G / Z_i$.

Then $G$ is \emph{nilpotent} iff $Z_n = G$ for some $n$.

We prove this is well-defined by proving that, for all $i$, we have $Z_i$ is normal in $G$.
\end{df}

\begin{proof}
\pf
\step{1}{\assume{as induction hypothesis $Z_i$ is normal in $G$.} \prove{$Z_{i+1}$ is normal in $G$.}}
\step{2}{\pflet{$x \in Z_{i+1}$ and $g \in G$} \prove{$gx\inv{g} \in Z_{i+1}$} \prove{For all $h \in G$ we have $gx\inv{g}hZ_i = hgx\inv{g}Z_i$}}
\step{3}{\pflet{$h \in G$}}
\step{4}{$gx\inv{g}hZ_i = hgx\inv{g}Z_i$}
\begin{proof}
	\pf
	\begin{align*}
		gx\inv{g}hZ_i & = g\inv{g}hxZ_i \\
		& = hxZ_i \\
		& = hg\inv{g}xZ_i \\
		& = hgx\inv{g}Z_i
	\end{align*}
\end{proof}
\qed
\end{proof}

\begin{prop}
Every Abelian group is nilpotent.
\end{prop}

\begin{proof}
\pf\ Let $G$ be an Abelian group. The center of $G/Z_0$ is $G/Z_0$, hence $Z_1 = G$. \qed
\end{proof}

\begin{prop}
Let $G$ be a group. Then $G$ is nilpotent if and only if $G/Z(G)$ is nilpotent.
\end{prop}

\begin{proof}
\pf
\step{1}{\pflet{$(Z_n)$ be the sequence of subgroups of $G$ where $Z_0 = \{e\}$ and $Z_{n+1}$ is the inverse image of the center of $G / Z_n$.}}
\step{2}{$G / Z_0 \cong G$}
\step{3}{$Z_1 = Z(G)$}
\step{4}{The corresponding sequence of subgroups for $G/ Z(G)$ is $G/Z(G)$, $Z_2 / Z(G)$, $Z_3 / Z(G)$, \ldots}
\step{5}{$G$ is nilpotent iff $G/Z(G)$ is nilpotent.}
\begin{proof}
	\pf\ Both are equivalent to $\exists n. Z_n = g$ and to $\exists n. Z_n/Z(G) = G/Z(G)$.
\end{proof}
\qed
\end{proof}

\begin{prop}
Every $p$-group is nilpotent.
\end{prop}

\begin{proof}
\pf\ Each $Z_n$ is a $p$-group and so has non-trivial center, hence each $Z_{n+1}$ is larger than $Z_n$ and so the sequence must terminate. \qed
\end{proof}

\begin{prop}
Every nilpotent group is solvable.
\end{prop}

\begin{proof}
\pf\ Let $(Z_n)$ be the defining sequence of subgroups. Then $Z_{n+1}/Z_n = Z(G/Z_n)$ is Abelian for all $n$, hence the group is solvable by Proposition \ref{prop:solvable}. \qed
\end{proof}

\begin{ex}
The converse is not true --- $S_3$ is solvable but not nilpotent.
\end{ex}

\begin{prop}
Let $G$ be a nilpotent group. Then every nontrivial normal subgroup of $G$ intersects $Z(G)$ non-trivially.
\end{prop}

\begin{proof}
\pf
\step{1}{\pflet{$H$ be a nontrivial normal subgroup of $G$.}}
\step{2}{\pflet{$(Z_n)$ be the sequence of subgroups with $Z_0 = \{e\}$ and $Z_{n+1}$ the inverse image of $Z(G/Z_n)$.}}
\step{3}{\pflet{$r$ be least such that $H \cap Z_r \neq \{e\}$.}}
\step{4}{\pick\ $h \in H \cap Z_r$ with $h \neq e$.}
\step{5}{$hZ_{r-1} \in Z(G/Z_{r-1})$}
\step{6}{For all $g \in G$ we have $ghZ_{r-1} = hgZ_{r-1}$}
\step{7}{For all $g \in G$ we have $gh\inv{g}\inv{h} \in Z_{r-1}$}
\step{8}{For all $g \in G$ we have $gh\inv{g}\inv{h} = e$}
\begin{proof}
	\pf\ Since $gh\inv{g}\inv{h} \in H$ and $H \cap Z_{r-1} = \{e\}$.
\end{proof}
\step{9}{For all $g \in G$ we have $gh = hg$}
\step{10}{$h \in H \cap Z(G)$}
\qed
\end{proof}

\begin{ex}
We cannot weaken the hypothesis to $G$ being solvable.
$S_3$ is solvable and $\mathbb{Z} / 2 \mathbb{Z}$ is a nontrivial normal subgroup but its intersection with $Z(S_3)$ is just $\{e\}$.
\end{ex}

\begin{prop}
\label{prop:H-sub-NGH}
Let $G$ be a finite nilpotent group. Let $H$ be a proper subgroup of $G$. Then $H \subsetneq N_G(H)$.
\end{prop}

\begin{proof}
\pf
\step{0}{\assume{as induction hypothesis the theorem holds for all groups smaller than $G$.}}
\step{1}{$Z(G)$ is non-trivial.}
\step{2}{\case{$Z(G) \nsubseteq H$}}
\begin{proof}
	\step{a}{\pick\ $g \in Z(G) - H$}
	\step{b}{$g \in N_G(H) - H$}
\end{proof}
\step{3}{\case{$Z(G) \subseteq H$}}
\begin{proof}
	\step{a}{$H / Z(G) \subsetneq N_{G/Z(G)}(H/Z(G))$}
	\begin{proof}
		\pf\ By induction hypothesis \stepref{0}.
	\end{proof}
	\step{b}{\pick\ $g$ such that $gZ(G) \in N_{G/Z(G)}(H/Z(G)) - H / Z(G)$}
	\step{c}{$g \in N_G(H)$}
	\begin{proof}
		\step{i}{\pflet{$h \in H$} \prove{$gh\inv{g} \in H$}}
		\step{ii}{$gh\inv{g}Z(G) \in H / Z(G)$}
		\step{iii}{\pick\ $h_1 \in H$ such that $gh\inv{g}Z(G) = h_1Z(G)$}
		\step{iv}{$gh\inv{g}\inv{h_1} \in Z(G)$}
		\step{v}{$gh\inv{g}\inv{h_1} \in H$}
		\begin{proof}
			\pf\ \stepref{3}
		\end{proof}
		\step{vi}{$gh\inv{g} \in H$}
	\end{proof}
	\step{d}{$g \notin H$}
\end{proof}
\qed
\end{proof}

\begin{cor}
Every Sylow subgroup of a finite nilpotent group is normal.
\end{cor}

\begin{proof}
\pf
\step{1}{\pflet{$G$ be a finite nilpotent group.}}
\step{2}{\pflet{$P$ be Sylow subgroup of $G$}}
\step{3}{$N_G(P) = N_G(N_G(P))$}
\begin{proof}
	\pf\ Proposition \ref{prop:NG-NG}.
\end{proof}
\step{4}{$N_G(P) = G$}
\begin{proof}
	\pf\ Proposition \ref{prop:H-sub-NGH}.
\end{proof}
\step{5}{$P$ is normal.}
\qed
\end{proof}


%TODO The converse holds

\section{Symmetric Groups}

\begin{prop}
Every permutation in $S_n$ is the product of a unique set of disjoint cycles.
\end{prop}

\begin{proof}
\pf\ Since any permutation acts as a cycle on any of its orbits. \qed
\end{proof}

\begin{cor}
The transpositions generate $S_n$.
\end{cor}

\begin{proof}
\pf\ Since any cycle is a product of transpositions:
\[ (a_1\ a_2\ \cdots\ a_n) = (a_1\ a_n) \circ \cdots \circ (a_1\ a_3) \circ (a_1\ a_2) \enspace . \qed \]
\end{proof}

\begin{cor}
$S_n$ is generated by $(1\ 2)$ and $(1\ 2\ 3\ \cdots\ n)$.
\end{cor}

\begin{proof}
\pf
\step{1}{Any transposition of the form $(i\ i+1)$ is in the subgroup generated by these two permutations.}
\begin{proof}
	\pf\ It is $(1\ 2\ \cdots\ n)^i (1\ 2) (1\ 2\ \cdots\ n)^{-i}$.
\end{proof}
\step{2}{Any transposition of the form $(1\ i)$ is in the subgroup generated by these two permutations.}
\begin{proof}
	\pf\ It is $(i-1\ i)\cdots (3\ 4)(2\ 3)(1\ 2)(2\ 3) \cdots (i-1\ i)$.
\end{proof}
\step{3}{Any transposition is in the subgroup generated by these two permutations.}
\begin{proof}
	\pf\ Since $(i\ j) = (1\ i)(1\ j)(1\ i)$
\end{proof}
\step{4}{These two permutations generate $S_n$.}
\begin{proof}
	\pf\ By the previous Corollary.
\end{proof}
\qed	
\end{proof}

\begin{df}[Type]
For any $\sigma \in S_n$, the \emph{type} of $\sigma$ is the partition of $n$ consisting of the sizes of the orbits of $\sigma$.
\end{df}

\begin{prop}
Two permutations in $S_n$ are conjugate if and only if they have the same type.
\end{prop}

\begin{proof}
\pf
\step{1}{Two permutations that are conjugate have the same type.}
\begin{proof}
	\pf\ Since
	\[ \tau (a_1\ a_2\ \cdots\ a_r) (b_1\ b_2\ \cdots\ b_s) \cdots (c_1\ c_2\ \cdots\ c_t) \inv{tau} = (\tau(a_1)\ \tau(a_2)\ \cdots\ \tau(a_r)) (\tau(b_1)\ \tau(b_2)\ \cdots\ \tau(b_s)) \cdots (\tau(c_1)\ \tau(c_2)\ \cdots\ \tau(c_t)) \]
\end{proof}
\step{2}{Two permutaitons with the same type are conjugate.}
\begin{proof}
	\step{a}{\pflet{$\rho = (a_1\ a_2\ \cdots\ a_r) (b_1\ b_2\ \cdots\ b_s) \cdots (c_1\ c_2\ \cdots\ c_t)$ and $\sigma = (a_1'\ a_2'\ \cdots\ a_r') (b_1'\ b_2'\ \cdots\ b_s') \cdots (c_1'\ c_2'\ \cdots\ c_t')$}}
	\step{b}{\pflet{$\tau$ be the permutation $\tau(a_i) = a_i'$, $\tau(b_i) = b_i'$, \ldots, $\tau(c_i) = c_i'$}}
	\step{c}{$\sigma = \tau \rho \inv{\tau}$}
\end{proof}
\qed
\end{proof}

\begin{cor}
The number of conjugacy classes in $S_n$ equals the number of permutations of $n$.
\end{cor}

\begin{df}[Sign]
Define $\Delta_n \in \mathbb{Z}[x_1, \ldots, x_n]$ by
\[ \Delta_n = \prod_{1 \leq i < j \leq n} (x_i - x_j) \]
Define an action of $S_n$ on $\mathbb{Z}[x_1, \ldots, x_n]$ by
\[ \sigma p(x_1, \ldots, x_n) = p(x_{\sigma(1)}, \ldots, x_{\sigma(n)}) \enspace . \]
The \emph{sign} of a permutation $\sigma \in S_n$ is the number $\epsilon(\sigma) \in \{ 1, -1 \}$ such that
\[ \sigma \Delta_n = \epsilon(\sigma) \Delta_n \enspace. \]

We say $\sigma$ is \emph{even} if $\epsilon(\sigma) = 1$ and \emph{odd} if $\epsilon(\sigma) = -1$.
\end{df}

\begin{prop}
$\epsilon$ is a group homomorphism $S_n \rightarrow \mathbb{Z}^*$.
\end{prop}

\begin{proof}
\pf
\step{1}{\pflet{$\rho, \sigma \in S_n$}}
\step{2}{$(\rho \circ \sigma) \Delta_n = \rho (\sigma \Delta_n)$}
\step{3}{$\epsilon(\rho \circ \sigma) \Delta_n = \epsilon(\rho) \epsilon(\sigma) \Delta_n$}
\step{4}{$\epsilon(\rho \circ \sigma) = \epsilon(\rho) \epsilon(\sigma)$}
\qed
\end{proof}

\begin{prop}
Let $\sigma = \tau_1 \cdots \tau_r$ where each $\tau_i$ is a transposition. Then $\sigma$ is even if and only if $r$ is even.
\end{prop}

\begin{proof}
\pf\ Since every transposition is odd and $\epsilon$ is a homomorphism, we have $\epsilon(\tau_1 \cdots \tau_r) = (-1)^r$. \qed
\end{proof}

\begin{cor}
A cycle is even if and only if its length is odd.
\end{cor}

\subsection{Transitive Subgroups}

\begin{df}[Transitive]
A subgroup of $S_n$ is \emph{transitive} iff its action on $\{1, \ldots, n\}$ is transitive.
\end{df}

\begin{prop}
If $G$ is a transitive subgroup of $S_n$ then $n \mid |G|$.
\end{prop}

\begin{proof}
\pf\ By Proposition \ref{prop:class-formula} we have
\[ n = [G : \Stab{G}{1}] \]
and so $n \mid |G|$. \qed
\end{proof}

\section{Alternating Groups}

\begin{df}
Let $n \in \mathbb{N}$. The \emph{alternating group} $A_n$ is the subgroup of $S_n$ consisting of the even permutations.

We call $A_5$ the \emph{icosahedral (rotating) group}.
\end{df}

\begin{prop}
For $n \geq 2$ we have
$A_n$ is normal in $S_n$ and
\[ [S_n : A_n] = 2 \enspace . \]
\end{prop}

\begin{proof}
\pf\ Since $\epsilon : S_n \rightarrow \{ 1, -1 \}$ is a homomorphism with kernel $A_n$. \qed
\end{proof}

\begin{prop}
\label{prop:size-of-sigma-An}
Let $n \geq 2$ and $\sigma \in A_n$. Let $[\sigma]_{A_n}$ be the conjugacy class of $\sigma$ in $A_n$, and $[\sigma]_{S_n}$ the conjugacy class of $\sigma$ is $S_n$. Then:
\begin{enumerate}
\item If $Z_{S_n}(\sigma) \subseteq A_n$ then $|[\sigma]_{S_n}| = 2 |[\sigma]_{A_n}|$.
\item If not then $[\sigma]_{S_n} = [\sigma]_{A_n}$.
\end{enumerate}
\end{prop}

\begin{proof}
\pf
\step{1}{$Z_{A_n}(\sigma) = A_n \cap Z_{S_n}(\sigma)$}
\step{2}{$|[\sigma]_{S_n}| = [S_n : Z_{S_n}(\sigma)]$}
\begin{proof}
	\pf\ Orbit-Stabilizer Theorem.
\end{proof}
\step{3}{$|[\sigma]_{A_n}| = [A_n : Z_{A_n}(\sigma)]$}
\begin{proof}
	\pf\ Orbit-Stabilizer Theorem.
\end{proof}
\step{4}{If $Z_{S_n}(\sigma) \subseteq A_n$ then $|[\sigma]_{S_n}| = 2|[\sigma]_{A_n}|$.}
\begin{proof}
	\pf
	\begin{align*}
		|[\sigma]_{S_n}| & = [S_n : Z_{S_n}(\sigma)] \\
		& = [S_n : A_n][A_n : Z_{S_n}(\sigma)] \\
		& = 2|[\sigma]_{A_n}|
	\end{align*}
\end{proof}
\step{5}{If $Z_{S_n}(\sigma) \nsubseteq A_n$ then $[\sigma]_{S_n} = [\sigma]_{A_n}$.}
\begin{proof}
	\step{a}{\assume{$Z_{S_n}(\sigma) \nsubseteq A_n$}}
	\step{b}{$A_n Z_{S_n}(\sigma) = S_n$}
	\begin{proof}
		\pf\ Since $A_n \subseteq A_n Z_{S_n}(\sigma)$ and $[S_n : A_n] = 2$.
	\end{proof}
	\step{c}{$|[\sigma]_{S_n}| = |[\sigma]_{A_n}|$}
	\begin{proof}
		\pf
		\begin{align*}
			|[\sigma]_{S_n}| & = [S_n : Z_{S_n}(\sigma)] \\
			& = [A_n Z_{S_n}(\sigma) : Z_{S_n}(\sigma)] \\
			& = [A_n : A_n \cap Z_{S_n}(\sigma)] & (\text{Second Isomorphism Theorem}) \\
			& = [A_n : Z_{A_n}(\sigma)] \\
			& = |[\sigma]_{A_n}|
		\end{align*}
	\end{proof}
\end{proof}
\qed
\end{proof}

\begin{prop}
\label{prop:conjugacy-split}
Let $n \geq 2$. Let $\sigma \in A_n$. Then $|[\sigma]_{S_n}| = 2|[\sigma]_{A_n}|$ if and only if the type of $\sigma$ consists of distinct odd numbers.
\end{prop}

\begin{proof}
\pf
\step{1}{If $|[\sigma]_{S_n}| = 2 |[\sigma]_{A_n}|$ then the type of $\sigma$ consists of distinct odd numbers.}
\begin{proof}
	\step{a}{If the type of $\sigma$ has an even number then $Z_{S_n}(\sigma) \nsubseteq A_n$.}
	\begin{proof}
		\pf\ If $(a_1\ a_2\ \cdots\ a_n)$ is an even cycle that is a factor of $\sigma$ then $(1\ 2\ \cdots\ n)$ is an odd permutation in $Z_{S_n}(\sigma)$.
	\end{proof}
	\step{b}{If the type of $\sigma$ has an odd number repeated then $Z_{S_n}(\sigma) \nsubseteq A_n$.}
	\begin{proof}
		\pf\ If $(a_1\ a_2\ \cdots\ a_n)$ and $(b_1\ b_2\ \cdots\ b_n)$ are two distinct odd factors of $\sigma$ then $(a_1\ b_1)(a_2\ b_2) \cdots (a_n\ b_n)$ is an odd permutation in $Z_{S_n}(\sigma)$.
	\end{proof}
	\qedstep
	\begin{proof}
		\pf\ Proposition \ref{prop:size-of-sigma-An}
	\end{proof}
\end{proof}
\step{2}{If the type of $\sigma$ consists of distinct odd numbers then $|[\sigma]_{S_n}| = 2 |[\sigma]_{A_n}|$.}
\begin{proof}
	\step{a}{\pflet{$\sigma = (a_{11}\ \cdots\ a_{1\lambda_1})(b_{21}\ \cdots\ b_{2\lambda_2}) \cdots (c_{n1}\ \cdots\ c_{n\lambda_n})$ where the $\lambda_i$ are all odd and distinct.}}
	\step{b}{\pflet{$\tau \in Z_{S_n}(\sigma)$} \prove{$\tau$ is even.}}
	\step{c}{$(\tau(a_{i1})\ \cdots\ \tau(a_{i\lambda_i})) = (\tau_{i1}\ \cdots\ \tau_{i\lambda_i})$}
	\step{d}{The action of $\tau$ on $\{a_{i1}, \ldots, a_{i\lambda_i}\}$ is $(a_{i1}\ \cdots\ a_{i\lambda_i})^{r_i}$ for some $r_i$}
	\step{e}{$\tau = \prod_{i=1}^n (a_{i1}\ \cdots\ a_{i \lambda_i})^{r_i}$}
	\step{f}{$\tau$ is even.}
\end{proof}
\qed
\end{proof}

\begin{cor}
\label{cor:A5-simple}
$A_5$ is simple.
\end{cor}

\begin{proof}
\pf
\step{1}{\assume{for a contradiction $G$ is a non-trivial proper normal subgroup of $A_5$.}}
\step{2}{$|G|$ is one of 2, 3, 4, 5, 6, 10, 12, 15, 20 or 30.}
\step{3}{There are conjugacy classes in $A_5$ whose sizes total to 1, 2, 3, 4, 5, 9, 11, 14, 19 or 29.}
\step{4}{The types of the even permutations in $S_5$ are $[1,1,1,1,1]$, $[2,2,1]$, $[3,1,1]$ and $[5]$.}
\step{5}{The size of the conjugacy class of type $[2,2,1]$ in $S_5$ is 15.}
\begin{proof}
	\pf\ There are 5 ways to choose the element not in the 2-cycles, and then 3 ways to arrange the other 4 elements into two 2-cycles.
\end{proof}
\step{6}{The size of the conjugacy class of type $[2,2,1]$ in $A_5$ is 15.}
\begin{proof}
	\pf\ Proposition \ref{prop:conjugacy-split}.
\end{proof}
\step{6}{The size of the conjugacy class of type $[3,1,1]$ in $S_5$ is 20.}
\begin{proof}
	\pf\ There are 10 ways to choose the three elements in the 3-cycle, and then two 3-cycles that they can form.
\end{proof}
\step{7}{The size of the conjugacy class of type $[3,1,1]$ in $A_5$ is 20.}
\begin{proof}
	\pf\ Proposition \ref{prop:conjugacy-split}.
\end{proof}
\step{8}{The size of the conjugacy class of type $[5]$ in $S_5$ is 24.}
\begin{proof}
	\pf\ There are four choices for the value at 1, then three choices for its value, then two choices for its value, then one choice for its value.
\end{proof}
\step{9}{The size of the conjugacy class of type $[5]$ in $S_5$ is 12.}
\begin{proof}
	\pf\ Proposition \ref{prop:conjugacy-split}.
\end{proof}
\qedstep
\begin{proof}
	\pf\ This contradicts \stepref{3}.
\end{proof}
\qed
\end{proof}

\begin{prop}
\label{prop:A6-simple}
$A_6$ is simple.
\end{prop}

\begin{proof}
\pf
\step{1}{\assume{for a contradiction $G$ is a non-trivial proper normal subgroup of $A_6$.}}
\step{2}{$|G|$ is one of 2, 3, 4, 5, 6, 8, 9, 10, 12, 15, 18, 20, 24, 30, 36, 40, 45, 60, 72, 90, 120, 180.}
\step{3}{There are conjugacy classes in $A_6$ whose sizes total to 1, 2, 3, 4, 5, 7, 8, 9, 11, 14, 17, 19, 23, 29, 35, 39, 44, 59, 71, 89, 119 or 179.}
\step{4}{The types of the even permutations in $S_6$ are $[1,1,1,1,1,1]$, $[2,2,1,1]$, $[3,1,1,1]$, $[3,3]$, $[4,2]$, $[5,1]$.}
\step{5}{The size of the conjugacy class of type $[2,2,1,1]$ in $S_6$ is 45.}
\step{6}{The size of the conjugacy class of type $[2,2,1,1]$ in $A_6$ is 45.}
\step{7}{The size of the conjugacy class of type $[3,1,1,1]$ in $S_6$ is 40.}
\step{8}{The size of the conjugacy class of type $[3,1,1,1]$ in $A_6$ is 40.}
\step{9}{The size of the conjugacy class of type $[3,3]$ in $S_6$ is 80.}
\step{10}{The size of the conjugacy class of type $[3,3]$ in $A_6$ is 80.}
\step{11}{The size of the conjugacy class of type $[4,2]$ in $S_6$ is 90.}
\step{12}{The size of the conjugacy class of type $[4,2]$ in $A_6$ is 90.}
\step{13}{The size of the conjugacy class of type $[5,1]$ in $S_6$ is 144.}
\step{14}{The size of the conjugacy class of type $[5,1]$ in $A_6$ is 72.}
\step{15}{The size of the conjugacy class of type $[6]$ in $S_6$ is 120.}
\step{16}{The size of the conjugacy class of type $[6]$ in $A_6$ is 120.}
\qedstep
\begin{proof}
	\pf\ This is a contradiction.
\end{proof}
\qed
\end{proof}

\begin{prop}
The icosahedral group $A_5$ is the group of symmetries of an icosahedron obtained through rigid motions.
\end{prop}

\begin{proof}
\pf\ Routine. \qed
\end{proof}

\begin{prop}
\label{prop:An-generated-by-3-cycles}
The alternating group $A_n$ is generated by 3-cycles.
\end{prop}

\begin{proof}
\pf
\step{1}{The product of two transpositions is generated by 3-cycles.}
\begin{proof}
	\step{a}{$(ab)(ab) = e$}
	\step{b}{$(ab)(ac) = (acb)$ for $b \neq c$}
	\step{c}{$(ab)(cd) = (adc)(abc)$ for $c \neq d$ and $c,d \notin \{a,b\}$}
\end{proof}
\qed
\end{proof}

\begin{prop}
\label{prop:all-3-cycles}
Let $n \geq 5$. If a normal subgroup of $A_n$ contains a 3-cycle, then it contains all 3-cycles.
\end{prop}

\begin{proof}
\pf
\step{1}{\pflet{$N$ be a normal subgroup of $A_n$.}}
\step{2}{\pflet{$(abc) \in N$}}
\step{3}{$N$ contains the conjugacy class of $(abc)$.}
\step{4}{The conjugacy class of $(abc)$ in $N$ is the same as its conjugacy class in $S_n$.}
\begin{proof}
	\pf\ Proposition \ref{prop:conjugacy-split} since the type of $(abc)$ is $[3,1,1,\ldots]$.
\end{proof}
\step{5}{$N$ contains all 3-cycles.}
\qed
\end{proof}

\begin{prop}
For $n \geq 4$, the center of $A_n$ is trivial.
\end{prop}

%TODO

\begin{thm}
For $n \geq 5$, the alternating group $A_n$ is simple.
\end{thm}

\begin{proof}
\pf
\step{0}{$A_5$ is simple.}
\begin{proof}
	\pf\ Corollary \ref{cor:A5-simple}.
\end{proof}
\step{00}{For $n \geq 6$ we have $A_n$ is simple.}
\begin{proof}
\step{000}{\pflet{$n \geq 6$}}
\step{1}{\pflet{$N$ be a nontrivial normal subgroup of $A_n$.}}
\step{2}{$N$ contains a 3-cycle.}
\begin{proof}
	\step{a}{\pick\ $\tau \in N$ such that $\tau \neq \id{}$ and $\tau$ acts on at most 6 elements.}
	\step{b}{\pick\ $T \subseteq \{1, \ldots, n\}$ with $|T| = 6$ such that $\tau$ acts on $T$.}
	\step{c}{Consider $A_6$ as a subgroup of $A_n$ by letting it act on $T$.}
	\step{d}{$N \cap A_6$ is normal.}
	\step{e}{$N \cap A_6$ is nontrivial.}
	\step{f}{$N \cap A_6 = A_6$}
	\begin{proof}
		\pf\ Proposition \ref{prop:A6-simple}.
	\end{proof}
	\step{g}{$N$ contains a 3-cycle.}
\end{proof}
\step{3}{$N$ contains all 3-cycles.}
\begin{proof}
	\pf\ Proposition \ref{prop:all-3-cycles}.
\end{proof}
\step{4}{$N = A_n$}
\begin{proof}
	\pf\ Proposition \ref{prop:An-generated-by-3-cycles}.
\end{proof}
\end{proof}
\qed
\end{proof}

\begin{cor}
For $n \geq 5$, we have $S_n$ is unsolvable.
\end{cor}

\begin{proof}
\pf\ Since the composition factors of $S_n$ are $C_2$ and $A_n$. \qed
\end{proof}

\chapter{Extensions}

\begin{df}[Extension]
Let $G$, $N$ and $H$ be groups. Then $G$ is an \emph{extension} of $H$ by $N$ iff there exist homomorphisms $N \rightarrow G$ and $G \rightarrow H$ such that
\[ 1 \rightarrow N \rightarrow G \rightarrow H \rightarrow 1 \]
is an exact sequence.
\end{df}

\begin{df}[Split Extension]
An exact sequence of groups
\[ 1 \rightarrow N \rightarrow G \rightarrow H \rightarrow 1 \]
\emph{splits} iff $H$ is a subgroup of $G$ and $N \cap H = \{e\}$.
\end{df}

\chapter{Classification of Groups}

\begin{ex}
\begin{itemize}
\item The only group of order 1 is the trivial group.
\item The only group of order 2 is $C_2$.
\item The only group of order 3 is $C_3$.
\item There are two groups of order 4: $C_4$ and $C_2 \times C_2$.
\item The only group of order 5 is $C_5$.
\item There are two groups of order 6: $C_6$ and $S_3$.
\item The only group of order 7 is $C_7$.
\item There are two groups of order 9: $C_9$ and $C_3 \times C_3$.
\item There are two groups of order 10: $C_{10}$ and $D_{10}$.
\item The only group of order 11 is $C_{11}$.
\item The only group of order 13 is $C_{13}$.
\item There are two groups of order 14: $C_{14}$ and $D_{14}$.
\item The only group of order 15 is $C_{15}$.
\end{itemize}
\end{ex}

\begin{prop}
\label{prop:order-eight}
The only non-Abelian groups of order 8 are $D_8$ and $Q_8$.
\end{prop}

\begin{proof}
\pf
\step{1}{\pflet{$G$ be a non-Abelian group of order 8.}}
\step{2}{$G$ has no element of order 8.}
\begin{proof}
	\pf\ If it does then it is $C_8$ and hence Abelian.
\end{proof}
\step{3}{\pick\ an element $y$ of order 4.}
\begin{proof}
	\step{a}{\pick\ an element $a$ of order 2.}
	\step{b}{$G/\langle a \rangle$ is isomorphic to $C_4$ or $C_2 \times C_2$.}
	\step{c}{\pick\ an element $y \langle a \rangle$ of order 2 in $G / \langle a \rangle$}
	\step{d}{$y^2 \in \langle a \rangle$}
	\step{e}{\case\ $y^2 = a$}
	\begin{proof}
		\pf\ In this case $y$ is of order 4.
	\end{proof}
	\step{f}{\case\ $y^2 = e$}
	\begin{proof}
		\pf\ In this case $G \cong C_2^3$ which is Abelian.
	\end{proof}
\end{proof}
\step{4}{\pick\ $x \notin \langle y \rangle$ such that $x^2 = e$ or $x^2 = y^2$}
\begin{proof}
	\step{a}{$G / \langle y \rangle \cong C_2$}
	\step{b}{\pick\ $x \langle y \rangle \in G / \langle y \rangle$ of order 2.}
	\step{c}{$x^2 \in \langle y \rangle$}
	\step{d}{$x^2 \neq y$ and $x^2 \neq y^3$}
	\step{e}{$x^2 = e$ or $x^2 = y^2$}
\end{proof}
\step{5}{$xy = y^3 x$}
\begin{proof}
	\step{a}{$xy \neq e$}
	\begin{proof}
		\pf\ Since $\inv{y} = y^3 \neq x$.
	\end{proof}
	\step{b}{$xy \neq y$}
	\begin{proof}
		\pf\ $xy = y$ implies $x=e$.
	\end{proof}
	\step{c}{$xy \neq y^2$}
	\begin{proof}
		\pf\ $xy = y^2$ implies $x = y$.
	\end{proof}
	\step{d}{$xy \neq y^3$}
	\begin{proof}
		\pf\ $xy = y^3$ implies $x = y^2$.
	\end{proof}
	\step{e}{$xy \neq x$}
	\begin{proof}
		\pf\ $xy = x$ implies $y = e$.
	\end{proof}
	\step{f}{$xy \neq yx$}
	\begin{proof}
		\pf\ $xy = yx$ implies $G$ is Abelian.
	\end{proof}
	\step{g}{$xy \neq y^2 x$}
	\begin{proof}
		\step{i}{\assume{for a contradiction $xy = y^2 x$}}
		\step{ii}{$xy^2 = x$}
		\begin{proof}
			\pf
			\begin{align*}
				xy^2 & = y^2xy \\
				& = y^4x \\
				& = x
			\end{align*}
		\end{proof}
		\step{iii}{$y^2 = e$}
	\end{proof}
\end{proof}
\step{4a}{The multiplication table of $G$ is one of the following.
\[ \begin{array}{cccccccc}
e & y & y^2 & y^3 & x & yx & y^2 x & y^3 x \\
y & y^2 & y^3 & e & yx & y^2 x & y^3 x & x \\
y^2 & y^3 & e & y & y^2 x & y^3 x & x & yx \\
y^3 & e & y & y^2 & y^3x & x & yx & y^2x \\
x & y^3x & y^2x & yx & e & y^3 & y^2 & y \\
yx & x & y^3 x & y^2 x & y & e & y^3 & y^2 \\
y^2 x & yx & x & y^3 x & y^2 & y & e & y^3 \\
y^3 x & y^2 x & yx & x & y^3 & y^2 & y & e
\end{array} \]
\vspace{0.5cm}
\[ \begin{array}{cccccccc}
e & y & y^2 & y^3 & x & yx & y^2 x & y^3 x \\
y & y^2 & y^3 & e & yx & y^2 x & y^3 x & x \\
y^2 & y^3 & e & y & y^2 x & y^3 x & x & yx \\
y^3 & e & y & y^2 & y^3x & x & yx & y^2x \\
x & y^3x & y^2x & yx & y^2 & y & e & y^3 \\
yx & x & y^3 x & y^2 x & y^3 & y^2 & y & e \\
y^2 x & yx & x & y^3 x & e & y^3 & y^2 & y \\
y^3 x & y^2 x & yx & x & y & e & y^3 & y^2
\end{array} \]}
\step{6}{$G \cong D_8$ or $G \cong Q_8$.}
\qed
\end{proof}

\begin{prop}
Let $q$ be an odd prime. Then $D_{2q}$ is the only non-Abelian group of order $2q$.
\end{prop}

\begin{proof}
\pf
\step{1}{\pflet{$G$ be a non-Abelian group of order $2q$.}}
\step{2}{\pick\ $y \in G$ of order $q$.}
\begin{proof}
	\pf\ Cauchy's Theorem
\end{proof}
\step{2.5}{$\langle y \rangle$ is the only subgroup of order $q$.}
\begin{proof}
	\pf\ Third Sylow Theorem
\end{proof}
\step{3}{$\langle y \rangle$ is normal.}
\step{4}{\pick\ $x \in G - \langle y \rangle - \{e\}$}
\step{4.5}{$|x| = 2$}
\begin{proof}
	\pf\ We cannot have $|x| = 2q$ since $G$ is not cyclic, and $|x| \neq q$ since $\langle x \rangle$ is not the subgroup of order $q$.
\end{proof}
\step{5}{$xy\inv{x} \in \langle y \rangle$}
\begin{proof}
	\pf\ Since $x \langle y \rangle \inv{x} = \langle y \rangle$ by \stepref{2.5}.
\end{proof}
\step{6}{\pick\ $r$ such that $0 \leq r < q$ and $xy\inv{x} = y^r$.}
\step{7}{$y^{r^2} = y$}
\begin{proof}
	\pf
	\begin{align*}
		y^{r^2} & = (xy \inv{x})^r & (\text{\stepref{6}}) \\
		& = x y^r \inv{x} \\
		& = x^2 y x^{-2} & (\text{\stepref{6}}) \\
		& = y & (\text{\stepref{4.5}})
	\end{align*}
\end{proof}
\step{8}{$q \mid (r-1)(r+1)$}
\begin{proof}
	\pf\ Since $y^{(r-1)(r+1)} = e$ and $|y| = q$ by \stepref{2}.
\end{proof}
\step{9}{$r = 1$ or $r = q-1$}
\begin{proof}
	\pf\ Since $0 \leq r < q$ by \stepref{6}.
\end{proof}
\step{10}{$r \neq 1$}
\begin{proof}
	\step{a}{\assume{for a contradiction $r = 1$.}}
	\step{b}{$xy = yx$}
	\begin{proof}
		\pf\ \stepref{6}
	\end{proof}
	\step{c}{$|xy| = 2q$}
	\begin{proof}
		\pf\ Proposition \ref{prop:order-gh-if-gcd-one}
	\end{proof}
	\step{d}{$G$ is cyclic.}
	\qedstep
	\begin{proof}
		\pf\ This contradicts \stepref{1}.
	\end{proof}
\end{proof}
\step{11}{$x^2 = e$ and $y^q = e$ and $yx = xy^{q-1}$}
\step{12}{$G \cong D_{2q}$}
\qed
\end{proof}

\begin{cor}
For $q$ an odd prime, the only groups of order $2q$ are $C_{2q}$ and $D_{2q}$.
\end{cor}

\begin{prop}
There is no non-Abelian simple group of order less than 60.
\end{prop}

\begin{proof}
\pf
We rule out the other sizes as follows:
\begin{itemize}
\item 1 --- Only group is the trivial group.
\item 2 --- Prime therefore cyclic
\item 3 --- Prime therefore cyclic
\item 4 --- Corollary \ref{cor:p-squared-not-simple}
\item 5 --- Prime therefore cyclic
\item 6 --- Corollary \ref{cor:mpr-not-simple}
\item 7 --- Prime therefore cyclic
\item 8 --- Corollary \ref{cor:p-squared-not-simple}
\item 9 --- Corollary \ref{cor:p-squared-not-simple}
\item 10 --- Corollary \ref{cor:mpr-not-simple}
\item 11 --- Prime therefore cyclic
\item 12 ---
\step{12}{There is no simple non-Abelian group of order 12.}
\begin{proof}
	\step{a}{\assume{for a contradiction $G$ is a simple non-Abelian group of order 12.}}
	\step{b}{$G$ has 4 3-Sylow subgroups.}
	\step{c}{$G$ has 8 elements of order 3.}
	\step{d}{$G$ has 3 elements of order 2 or 4.}
	\step{e}{$G$ has one 2-Sylow subgroup.}
	\step{f}{The 2-Sylow subgroup of $G$ is normal.}
	\qedstep
	\begin{proof}
		\pf\ This contradicts \stepref{a}.
	\end{proof}
\end{proof}
\item 13 --- Prime therefore cyclic
\item 14 --- Corollary \ref{cor:mpr-not-simple}
\item 15 --- Corollary \ref{cor:mpr-not-simple}
\item 16 --- Corollary \ref{cor:p-squared-not-simple}
\item 17 --- Prime therefore cyclic
\item 18 --- Corollary \ref{cor:mpr-not-simple}
\item 19 --- Prime therefore cyclic
\item 20 --- Corollary \ref{cor:mpr-not-simple}
\item 21 --- Corollary \ref{cor:mpr-not-simple}
\item 22 --- Corollary \ref{cor:mpr-not-simple}
\item 23 --- Prime therefore cyclic
\item 24 ---
\step{24}{There is no simple non-Abelian group of order 24.}
\begin{proof}
	\step{a}{\assume{for a contradiction $G$ is a simple non-Abelian group of order 24.}}
	\step{b}{$G$ has 3 2-Sylow subgroups.}
	\step{c}{\pflet{$\gamma : G \rightarrow S_3$ be the action of conjugation of $G$ on the set of 2-Sylow subgroups.}}
	\step{d}{$\ker \gamma \neq \{e\}$}
	\begin{proof}
		\pf\ $\gamma$ cannot be injective since $|G| > |S_3|$.
	\end{proof}
	\step{e}{$\ker \gamma \neq G$}
	\step{f}{$\ker \gamma$ is a proper non-trivial normal subgroup of $G$.}
	\qedstep
	\begin{proof}
		\pf\ This contradicts \stepref{a}.
	\end{proof}
\end{proof}
\item 25 --- Corollary \ref{cor:p-squared-not-simple}
\item 26 --- Corollary \ref{cor:mpr-not-simple}
\item 27 --- Corollary \ref{cor:p-squared-not-simple}
\item 28 --- Corollary \ref{cor:mpr-not-simple}
\item 29 --- Prime therefore cyclic
\item 30 --- Proposition \ref{prop:pqr-not-simple} 
\item 31 --- Prime therefore cyclic
\item 32 --- Corollary \ref{cor:p-squared-not-simple}
\item 33 --- Corollary \ref{cor:mpr-not-simple}
\item 34 --- Corollary \ref{cor:mpr-not-simple}
\item 35 --- Corollary \ref{cor:mpr-not-simple}
\item 36 ---
\step{36}{There is no simple non-Abelian group of order 36.}
\begin{proof}
	\step{a}{\assume{for a contradiction $G$ is a simple non-Abelian group of order 36.}}
	\step{b}{$G$ has 4 3-Sylow subgroups.}
	\step{c}{\pflet{$\gamma : G \rightarrow S_4$ be the action of conjugation of $G$ on the set of 2-Sylow subgroups.}}
	\step{d}{$\ker \gamma \neq \{e\}$}
	\begin{proof}
		\pf\ $\gamma$ cannot be injective since $|G| > |S_4|$.
	\end{proof}
	\step{e}{$\ker \gamma \neq G$}
	\step{f}{$\ker \gamma$ is a proper non-trivial normal subgroup of $G$.}
	\qedstep
	\begin{proof}
		\pf\ This contradicts \stepref{a}.
	\end{proof}
\end{proof}
\item 37 --- Prime therefore cyclic
\item 38 --- Corollary \ref{cor:mpr-not-simple}
\item 39 --- Corollary \ref{cor:mpr-not-simple}
\item 40 --- There can be only 1 5-Sylow subgroup.
\item 41 --- Prime therefore cyclic
\item 42 --- Proposition \ref{prop:pqr-not-simple} 
\item 43 --- Prime therefore cyclic
\item 44 --- Corollary \ref{cor:mpr-not-simple}
\item 45 --- There can be only 1 5-Sylow subgroup.
\item 46 --- Corollary \ref{cor:mpr-not-simple}
\item 47 --- Prime therefore cyclic
\item 48 ---
\step{48}{There is no simple non-Abelian group of order 48.}
\begin{proof}
	\step{a}{\assume{for a contradiction $G$ is a simple non-Abelian group of order 48.}}
	\step{b}{$G$ has 3 2-Sylow subgroups.}
	\step{c}{\pflet{$\gamma : G \rightarrow S_3$ be the action of conjugation of $G$ on the set of 2-Sylow subgroups.}}
	\step{d}{$\ker \gamma \neq \{e\}$}
	\begin{proof}
		\pf\ $\gamma$ cannot be injective since $|G| > |S_3|$.
	\end{proof}
	\step{e}{$\ker \gamma \neq G$}
	\step{f}{$\ker \gamma$ is a proper non-trivial normal subgroup of $G$.}
	\qedstep
	\begin{proof}
		\pf\ This contradicts \stepref{a}.
	\end{proof}
\end{proof}
\item 49 --- Corollary \ref{cor:p-squared-not-simple}
\item 50 --- Corollary \ref{cor:mpr-not-simple}
\item 51 --- Corollary \ref{cor:mpr-not-simple}
\item 52 --- Corollary \ref{cor:mpr-not-simple}
\item 53 --- Prime therefore cyclic
\item 54 --- Corollary \ref{cor:mpr-not-simple}
\item 55 --- Corollary \ref{cor:mpr-not-simple}
\item 56 --- Corollary \ref{cor:mpr-not-simple}
\item 57 --- Corollary \ref{cor:mpr-not-simple}
\item 58 --- Corollary \ref{cor:mpr-not-simple}
\item 59 --- Prime therefore cyclic
\end{itemize}
\end{proof}

\begin{prop}
Every simple group of order 60 has a subgroup of index 5.
\end{prop}

\begin{proof}
\pf
\step{1}{\pflet{$G$ be a simple group of order 60.}}
\step{2}{The number of 2-Sylow subgroups of $G$ is either 5 or 15.}
\begin{proof}
	\step{a}{\pflet{$n$ be the number of 2-Sylow subgroups.}}
	\step{b}{$60 | n!$}
	\begin{proof}
		\pf\ Corollary \ref{cor:G-divides-Npfac}.
	\end{proof}
	\step{c}{$n \geq 5$}
	\step{e}{$n \mid 15$}
	\begin{proof}
		\pf\ Third Sylow Theorem
	\end{proof}
	\step{f}{$n = 5$ or $n = 15$}
\end{proof}
\step{3}{\assume{w.l.o.g.\ $G$ has 15 2-Sylow subgroups.}}
\step{4}{$G$ has 4 or 10 3-Sylow subgroups.}
\step{4a}{$G$ has 10 3-Sylow subgroups.}
\begin{proof}
	\pf\ Corollary \ref{cor:G-divides-Npfac}.
\end{proof}
\step{5}{$G$ has exactly 6 5-Sylow subgroups.}
\step{6}{The number of elements of order 3 is 20.}
\step{7}{The number of elements of order 5 is 24.}
\step{8}{The number of elements of order 2 or 4 is 15.}
\step{9}{\pick\ two 2-Sylow subgroups $H_1$ and $H_2$ with non-trivial intersection.}
\step{10}{\pflet{$g \in G$ be such that $H_1 \cap H_2 = \{e, g\}$.}}
\step{11}{\pflet{$K = Z_G(H_1 \cap H_2)$}}
\step{12}{$|K| = 12$ or $|K| = 20$}
\begin{proof}
	\pf\ We have $4 \mid |K|$ since $H_1 \leq K$, and $|K| \geq 6$ since $H_1 \cup H_2 \subseteq K$. We also have $|K| \mid 60$.
\end{proof}
\step{13}{$[G:K] \neq 3$}
\begin{proof}
	\pf\ There cannot be an embedding of $G$ in $S_3$.
\end{proof}
\step{14}{$[G:K] = 5$}
\qed 
\end{proof}

\begin{thm}
$A_5$ is the only simple group of order 60.
\end{thm}

\begin{proof}
\pf
\step{1}{\pflet{$G$ be a simple group of order 60.}}
\step{2}{\pick\ a subgroup $K$ of $G$ of index 5.}
\step{3}{\pflet{$\phi : G \rightarrow S_5$ be the action of $G$ on $G / K$ of left multiplication.}}
\step{4}{$\phi$ is injective.}
\begin{proof}
	\pf\ Since $\ker \phi$ is a proper normal subgroup of $G$ hence $\ker \phi = \{e\}$.
\end{proof}
\step{5}{$\phi(G)$ is a subgroup of $S_5$ of index 2.}
\step{6}{$\phi(G)$ is normal in $S_5$.}
\step{7}{$\phi(G) \cap A_5$ is a normal subgroup of $A_5$}
\step{8}{$\phi(G) \cap A_5 = \{e\}$ or $\phi(G) \cap A_5 = A_5$}
\begin{proof}
	\pf\ Corollary \ref{cor:A5-simple}.
\end{proof}
\step{9}{$\phi(G) \cap A_5 = A_5$}
\begin{proof}
	\pf\ We cannot have $\phi(G) \cap A_5 = \{e\}$ lest
	\[ |\phi(G) A_5| = |\phi(G)||A_5| / |\phi(G) \cap A_5| = 3600 \]
	by the Second Isomorphism Theorem.
\end{proof}
\step{10}{$\phi(G) = A_5$}
\step{11}{$\phi : G \cong A_5$}
\qed
\end{proof}

\begin{prop}
There is no non-Abelian simple group of order between 60 and 168.
\end{prop}

\begin{proof}
\pf\ We rule out the other sizes as follows:
\begin{itemize}
\item 61 --- prime therefore cyclic
\item 62 --- Corollary \ref{cor:mpr-not-simple}
\item 63 --- Corollary \ref{cor:mpr-not-simple1}
\item 64 --- Corollary \ref{cor:p-squared-not-simple}
\item 65 --- Corollary \ref{cor:mpr-not-simple}
\item 66 --- Corollary \ref{cor:mpr-not-simple}
\item 67 --- prime therefore cyclic
\item 68 --- Corollary \ref{cor:mpr-not-simple}
\item 69 --- Corollary \ref{cor:mpr-not-simple}
\item 70 --- Proposition \ref{prop:pqr-not-simple}
\item 71 --- prime therefore cyclic
\item 72
\step{72}{There is no simple non-Abelian group of order 72}
\begin{proof}
	\pf
	\step{a}{\assume{for a contradiction $G$ is a simple non-Abelian group of order 72.}}
	\step{c}{$G$ has 4 3-Sylow subgroups.}
	\step{d}{\pflet{$\gamma : G \rightarrow S_4$ be the action of conjugation on the set of 3-Sylow subgroups.}}
	\step{e}{$\ker \gamma \neq 1$}
	\begin{proof}
		\pf\ Since $|G| > |S_4|$.
	\end{proof}
	\step{f}{$\ker \gamma$ is a non-trivial proper subgroup of $G$.}
	\qedstep
	\begin{proof}
		\pf\ This is a contradiction.
	\end{proof}
\end{proof}
\item 73 --- prime therefore cyclic
\item 74 --- Corollary \ref{cor:mpr-not-simple}
\item 75 --- Corollary \ref{cor:mpr-not-simple}
\item 76 --- Corollary \ref{cor:mpr-not-simple}
\item 77 --- Corollary \ref{cor:mpr-not-simple}
\item 78 --- Corollary \ref{cor:mpr-not-simple}
\item 79 --- prime therefore cyclic
\item 80
\step{80}{There is no simple non-Abelian group of order 80.}
\begin{proof}
	\pf
	\step{a}{\assume{for a contradiction $G$ is a simple non-Abelian group of order 80.}}
	\step{c}{$G$ has 5 2-Sylow subgroups.}
	\step{d}{\pflet{$\gamma : G \rightarrow S_5$ be the action of conjugation on the set of 2-Sylow subgroups.}}
	\step{e}{$\ker \gamma \neq 1$}
	\begin{proof}
		\pf\ Otherwise $\im \gamma$ would be a subgroup of $S_5$ of order 80, contradicting Lagrange's Theorem.
	\end{proof}
	\step{f}{$\ker \gamma$ is a non-trivial normal subgroup of $G$.}
	\qedstep
	\begin{proof}
		\pf\ This is a contradiction.
	\end{proof}
\end{proof}

\item 81 --- Corollary \ref{cor:p-squared-not-simple}
\item 82 --- Corollary \ref{cor:mpr-not-simple}
\item 83 --- prime therefore cyclic
\item 84 --- Corollary \ref{cor:mpr-not-simple1}
\item 85 --- Corollary \ref{cor:mpr-not-simple}
\item 86 --- Corollary \ref{cor:mpr-not-simple}
\item 87 --- Corollary \ref{cor:mpr-not-simple}
\item 88 --- Corollary \ref{cor:mpr-not-simple}
\item 89 --- prime therefore cyclic
\item 90 --- Corollary \ref{cor:mpr-not-simple1}
\item 91 --- Corollary \ref{cor:mpr-not-simple}
\item 92 --- Corollary \ref{cor:mpr-not-simple}
\item 93 --- Corollary \ref{cor:mpr-not-simple}
\item 94 --- Corollary \ref{cor:mpr-not-simple}
\item 95 --- Corollary \ref{cor:mpr-not-simple}
\item 96 --- There are 3 2-Sylow subgroups. The kernel of the action of conjugation $G \rightarrow S_3$ is a non-trivial normal subgroup of $G$.
\item 97 --- prime therefore cyclic
\item 98 --- Corollary \ref{cor:mpr-not-simple}
\item 99 --- Corollary \ref{cor:mpr-not-simple}
\item 100 --- Corollary \ref{cor:mpr-not-simple}
\item 101 --- prime therefore cyclic
\item 102 --- Proposition \ref{prop:pqr-not-simple}
\item 103 --- prime therefore cyclic
\item 104 --- Corollary \ref{cor:mpr-not-simple}
\item 105 --- Proposition \ref{prop:pqr-not-simple}
\item 106 --- Corollary \ref{cor:mpr-not-simple}
\item 107 --- prime therefore cyclic
\item 108 --- There are 4 3-Sylow subgroups. The kernel of the action of conjugation $G \rightarrow S_4$ is a non-trivial normal subgroup of $G$.
\item 109 --- prime therefore cyclic
\item 110 --- Proposition \ref{prop:pqr-not-simple}
\item 111 --- Corollary \ref{cor:mpr-not-simple}
\item 112
\step{112}{There is no simple non-Abelian group of order 112.}
\begin{proof}
	\step{a}{\assume{for a contradiction $G$ is a simple non-Abelian group of order 112.}}
	\step{b}{$G$ has exactly 7 2-Sylow subgroups.}
	\step{c}{\pflet{$\gamma : G \rightarrow A_7$ be the action of conjugation of $G$ on the set of 2-Sylow subgroups.}}
	\begin{proof}
		\pf\ $\gamma(g)$ is always an even permutation since $G$ has no subgroup of index 2.
	\end{proof}
	\step{d}{$\ker \gamma \neq 1$}
	\begin{proof}
		\pf\ Since $|G|$ does not divide $|A_7| = 7!/2$.
	\end{proof}
	\step{e}{$\ker \gamma$ is a non-trivial normal subgroup of $G$.}
	\qedstep
\end{proof}
\item 113 --- prime therefore cyclic
\item 114 --- Proposition \ref{prop:pqr-not-simple}
\item 115 --- Corollary \ref{cor:mpr-not-simple}
\item 116 --- Corollary \ref{cor:mpr-not-simple}
\item 117 --- Corollary \ref{cor:mpr-not-simple}
\item 118 --- Corollary \ref{cor:mpr-not-simple}
\item 119 --- Corollary \ref{cor:mpr-not-simple}
\item 120
\step{120}{There is no simple non-Abelian group of order 120.}
\begin{proof}
	\pf
	\step{a}{\assume{for a contradiction $G$ is a simple non-Abelian group of order 120.}}
	\step{b}{There are exactly 6 5-Sylow subgroups.}
	\step{c}{\pflet{$\gamma : G \rightarrow A_6$ be the action of conjugation on the set of 5-Sylow subgroups.}}
	\step{d}{$\im \gamma$ is a subgroup of $A_6$ of order 120.}
	\qedstep
	\begin{proof}
		\pf\ This is a contradiction by inspection of the list of subgroups of $A_6$.
	\end{proof}
\end{proof}
\item 121 --- Corollary \ref{cor:p-squared-not-simple}
\item 122 --- Corollary \ref{cor:mpr-not-simple}
\item 123 --- Corollary \ref{cor:mpr-not-simple}
\item 124 --- Corollary \ref{cor:mpr-not-simple}
\item 125 --- Corollary \ref{cor:p-squared-not-simple}
\item 126 --- Corollary \ref{cor:mpr-not-simple1}
\item 127 --- prime therefore cyclic
\item 128 --- Corollary \ref{cor:p-squared-not-simple}
\item 129 --- Corollary \ref{cor:mpr-not-simple}
\item 130 --- Proposition \ref{prop:pqr-not-simple}
\item 131 --- prime therefore cyclic
\item 132
\step{132}{There is no simple non-Abelian group of order 132.}
\begin{proof}
\step{a}{\assume{for a contradiction $G$ is a simple non-Abelian group of order 132.}}
\step{b}{There are at least 4 3-Sylow subgroups.}
\step{c}{There are at least 8 elements of order 3.}
\step{d}{There are exactly 12 11-Sylow subgroups.}
\step{e}{There are exactly 120 elements of order 11.}
\step{f}{There are exactly 3 elements of order 2.}
\step{g}{There is a unique 2-Sylow subgroups.}
\qedstep
\begin{proof}
	\pf\ This is a contradiction.
\end{proof}
\end{proof}
\item 133 --- Corollary \ref{cor:mpr-not-simple}
\item 134 --- Corollary \ref{cor:mpr-not-simple}
\item 135 --- Corollary \ref{cor:mpr-not-simple1}
\item 136 --- Corollary \ref{cor:mpr-not-simple}
\item 137 --- prime therefore cyclic
\item 138 --- Proposition \ref{prop:pqr-not-simple}
\item 139 --- prime therefore cyclic
\item 140 --- Corollary \ref{cor:mpr-not-simple1}
\item 141 --- Corollary \ref{cor:mpr-not-simple}
\item 142 --- Corollary \ref{cor:mpr-not-simple}
\item 143 --- Corollary \ref{cor:mpr-not-simple}
\item 144 --- Burnside's Theorem
\item 145 --- Burnside's Theorem
\item 146 --- Burnside's Theorem
\item 147 --- Burnside's Theorem
\item 148 --- Burnside's Theorem
\item 149 --- prime therefore cyclic
\item 150 --- There are exactly 6 5-Sylow subgroups. The kernel of the action of conjugation $G \rightarrow A_5$ is a non-trivial normal subgroup since 150 does not divide $|A_5| = 60$.
\item 151 --- prime therefore cyclic
\item 152 --- Burnside's Theorem
\item 153 --- Burnside's Theorem
\item 154 --- Proposition \ref{prop:pqr-not-simple}
\item 155 --- Burnside's Theorem
\item 156 --- Corollary \ref{cor:mpr-not-simple}
\item 157 --- prime therefore cyclic
\item 158 --- Burnside's Theorem
\item 159 --- Burnside's Theorem
\item 160 --- Burnside's Theorem
\item 161 --- Burnside's Theorem
\item 162 --- Burnside's Theorem
\item 163 --- prime therefore cyclic
\item 164 --- Burnside's Theorem
\item 165 --- Proposition \ref{prop:pqr-not-simple}
\item 166 --- Burnside's Theorem
\item 167 --- prime therefore cyclic
\end{itemize}
\end{proof}

\begin{prop}
Every group of order $< 120$ and $\neq 60$ is solvable.
\end{prop}

\begin{proof}
\pf
\step{1}{\pflet{$G$ be a group of order $n$ where $n < 120$ and $n \neq 60$.}}
\step{2}{If $n$ is odd then $G$ is solvable.}
\begin{proof}
	\pf\ Feit-Thompson Theorem
\end{proof}
\step{3}{If $n$ has at most two prime factors then $G$ is solvable.}
\begin{proof}
	\pf\ Burnside's Theorem
\end{proof}
\step{4}{\case{$n = pqr$ for some primes $p$, $q$, $r$}}
\begin{proof}
	\pf\ Its composition factors must be $C_p$, $C_q$ and $C_r$.
\end{proof}
\step{8}{\case{$n = 84$}}
\begin{proof}
	\pf\ By the Third Sylow Theorem, the 7-Sylow subgroup is normal. Since every group of order 12 is solvable, so is every group of order 84.
\end{proof}
\qed
\end{proof}