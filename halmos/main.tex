\documentclass{report}
\title{Summary of Halmos' Naive Set Theory}
\author{Robin Adams}

\usepackage{amsmath}
\usepackage{amssymb}
\usepackage{amsthm}
\let\proof\relax
\let\endproof\relax
\let\qed\relax
\usepackage{pf2}
\usepackage{hyperref}

\newtheorem{ax}{Axiom}[chapter]
\newtheorem{prop}[ax]{Proposition}
\newtheorem{cor}{Corollary}[ax]
\newtheorem{thm}[ax]{Theorem}
\newtheorem{lm}[ax]{Lemma}
\theoremstyle{definition}
\newtheorem{df}[ax]{Definition}

\begin{document}
\maketitle
\tableofcontents

\chapter{Primitive Terms and Axioms}

Let there be \emph{sets}. We assume that everything is a set.

Let there be a binary relation of \emph{membership}, $\in$. If $x \in A$ we say that $x$ \emph{belongs} to $A$, $x$ is an \emph{element} of $A$, or $x$ is \emph{contained} in $A$. If this does not hold we write $x \notin A$.

\begin{ax}[Axiom of Extensionality]
Two sets are equal if and only if they have the same elements.
\end{ax}

\begin{ax}[Axiom of Comprehension, Aussonderungsaxiom]
To every set $A$ and to every condition $S(x)$ there corresponds a set $B$ whose elements are exactly those elements $x$ of $A$ for which $S(x)$ holds.
\end{ax}

\begin{ax}
\label{ax:set_exists}
A set exists.
\end{ax}

\begin{ax}[Axiom of Pairing]
For any two sets, there exists a set that they both belong to.
\end{ax}

\begin{ax}[Union Axiom]
For every set $A$, there exists a set that contains all the elements that belong to at least one element of $A$.
\end{ax}

\chapter{The Subset Relation}

\begin{df}[Subset]
Let $A$ and $B$ be sets. We say that $A$ is a \emph{subset} of $B$, or $B$ \emph{includes} $A$, and write $A \subseteq B$ or $B \supseteq A$, iff every element of $A$ is an element of $B$.
\end{df}

\begin{thm}
For any set $A$, we have $A \subseteq A$.
\end{thm}

\begin{proof}
\pf\ Every element of $A$ is an element of $A$. \qed
\end{proof}

\begin{thm}
For any sets $A$, $B$ and $C$, if $A \subseteq B$ and $B \subseteq C$ then $A \subseteq C$.
\end{thm}

\begin{proof}
\pf\ If every element of $A$ is an element of $B$, and every element of $B$ is an element of $C$, then every element of $A$ is an element of $C$. \qed
\end{proof}

\begin{thm}
For any sets $A$ and $B$, if $A \subseteq B$ and $B \subseteq A$ then $A = B$.
\end{thm}

\begin{proof}
\pf\ If every element of $A$ is an element of $B$, and every element of $B$ is an element of $A$, then $A$ and $B$ have the same elements, and therefore are equal by the Axiom of Extensionality. \qed
\end{proof}

\begin{df}[Proper Subset]
Let $A$ and $B$ be sets. We say that $A$ is a \emph{proper} subset of $B$, or $B$ \emph{properly} includes $A$, and write $A \subsetneq B$ or $B \supsetneq A$, iff $A \subseteq B$ and $A \neq B$.
\end{df}

\chapter{Comprehension Notation}

\begin{df}
Given a set $A$ and a condition $S(x)$, we write $\{ x \in A : S(x) \}$ for the set whose elements are exactly those elements $x$ of $A$ for which $S(x)$ holds.
\end{df}

\begin{proof}
\pf\ This exists by the Axiom of Comprehension and is unique by the Axiom of Extensionality. \qed
\end{proof}

\begin{thm}
There is no set that contains every set.
\end{thm}

\begin{proof}
\pf
\step{1}{\pflet{$A$ be a set.} \prove{There exists a set $B$ such that $B \notin A$.}}
\step{2}{\pflet{$B = \{ x \in A : x \notin x \}$}}
\step{3}{If $B \in A$ then we have $B \in B$ if and only if $B \notin B$.}
\step{4}{$B \notin A$}
\qed
\end{proof}

\chapter{Unordered Pairs}

\begin{thm}
There exists a set with no elements.
\end{thm}

\begin{proof}
\pf\ Pick a set $A$ by Axiom \ref{ax:set_exists}. Then the set $\{ x \in A : x \neq x \}$ has no elements. \qed
\end{proof}

\begin{df}[Empty Set]
The \emph{empty set} $\emptyset$ is the set with no elements.
\end{df}

\begin{thm}
For any set $A$ we have $\emptyset \subset A$.
\end{thm}

\begin{proof}
\pf\ Vacuous. \qed
\end{proof}

\begin{df}[(Unordered) Pair]
For any sets $a$ and $b$, the \emph{(unordered) pair} $\{a,b\}$ is the set whose elements are just $a$ and $b$.
\end{df}

\begin{proof}
\pf\ This exists by the Axioms of Pairing and Comprehension, and is unique by the Axiom of Extensionality. \qed
\end{proof}

\begin{df}[Singleton]
For any set $a$, the \emph{singleton} $\{a\}$ is defined to be $\{a,a\}$.
\end{df}

\chapter{Unions and Intersections}

\begin{df}[Union]
For any set $\mathcal{C}$, the \emph{union} of $\mathcal{C}$, $\bigcup \mathcal{C}$, is the set whose elements are the elements of the elements of $\mathcal{C}$.
\end{df}

\begin{proof}
\pf\ This exists by the Union Axiom and Comprehension Axiom, and is unique by the Axiom of Extensionality. \qed
\end{proof}

\begin{prop}
\[ \bigcup \emptyset = \emptyset \]
\end{prop}

\begin{proof}
\pf\ There is no set that is an element of an element of $\emptyset$. \qed
\end{proof}

\begin{prop}
For any set $A$, we have $\bigcup \{A\} = A$.
\end{prop}

\begin{proof}
\pf\ For any $x$, we have $x$ is an element of an element of $\{A\}$ if and only if $x$ is an element of $A$. \qed
\end{proof}

\begin{df}
We write $A \cup B$ for $\bigcup \{A,B\}$.
\end{df}

\begin{prop}
For any set $A$, we have $A \cup \emptyset = A$.
\end{prop}

\begin{proof}
\pf\ $x \in A \cup \emptyset$ iff $x \in A$ or $x \in \emptyset$, iff $x \in A$. \qed
\end{proof}

\begin{prop}[Commutativity]
For any sets $A$ and $B$, we have $A \cup B = B \cup A$.
\end{prop}

\begin{proof}
\pf\ $x \in A \cup B$ iff $x \in A$ or $x \in B$, iff $x \in B$ or $x \in A$, iff $x \in B \cup A$. \qed
\end{proof}

\begin{prop}[Associativity]
For any sets $A$, $B$ and $C$, we have $A \cup (B \cup C) = (A \cup B) \cup C$.
\end{prop}

\begin{proof}
\pf\ Each is the set of all $x$ such that $x \in A$ or $x \in B$ or $x \in C$. \qed
\end{proof}

\begin{prop}[Idempotence]
For any set $A$, we have $A \cup A = A$.
\end{prop}

\begin{proof}
\pf\ $x \in A$ or $x \in A$ is equivalent to $x \in A$. \qed
\end{proof}

\begin{prop}
For any sets $A$ and $B$, we have $A \subseteq B$ if and only if $A \cup B = B$.
\end{prop}

\begin{proof}
\pf\ For any $x$, the statement "if $x \in A$ then $x \in B$" is equivalent to "$x \in A$ or $x \in B$ if and only if $x \in B$". \qed
\end{proof}

\begin{prop}
For any sets $a$ and $b$, we have $\{a\} \cup \{b\} = \{a,b\}$.
\end{prop}

\begin{proof}
\pf\ Immediate from definitions. \qed
\end{proof}

\begin{df}
Given sets $a$, $b$ and $c$, let
\[ \{a,b,c\} := \{a\} \cup \{b\} \cup \{c\} \enspace . \]
\end{df}

\end{document}