\documentclass{book}
\title{Mathematics}
\author{Robin Adams}

\usepackage{amssymb}
\usepackage{amsthm}
\let\proof\relax
\let\endproof\relax
\let\qed\relax
\usepackage{pf2}

\newtheorem{ax}{Axiom}
\newtheorem{prop}{Proposition}[chapter]
\newtheorem{thm}[prop]{Theorem}
\newtheorem{cor}{Corollary}[prop]
\theoremstyle{definition}
\newtheorem{df}[prop]{Definition}

\begin{document}

\maketitle
\tableofcontents

\part{Set Theory}

\chapter{Axioms}

\section{Classes}

We speak informally about \emph{classes}. A \emph{class} is determined by a unary predicate. We write $\{ x \mid P(x) \}$ for the class determined by $P$.

We say an object $a$ is a \emph{member} or \emph{element} of the class $\mathbf{A} = \{ x \mid P(x) \}$, or $\mathbf{A}$ \emph{contains} $a$, and write $a \in \mathbf{A}$ or $\mathbf{A} \ni a$, iff $P(a)$ is true. 

We say two classes are \emph{equal} iff they have exactly the same elements.

We write $\{ x \in \mathbf{A} \mid P(x) \}$ for $\{ x \mid x \in \mathbf{A} \wedge P(x) \}$. We write $\{ t[x_1, \ldots, x_n] \mid P(x_1, \ldots, x_n) \}$ for $\{ y \mid \exists x_1 \cdots \exists x_n (P(x_1, \ldots, x_n) \wedge y = t[x_1, \ldots, x_n])\}$.

\begin{df}[Disjoint]
    Two classes are \emph{disjoint} iff they have no common element.
\end{df}

\begin{df}[Subclass]
    Given classes $\mathbf{A}$ and $\mathbf{B}$, we say $\mathbf{A}$ is a \emph{subclass} of $\mathbf{B}$, $\mathbf{B}$ is a \emph{superclass} of $\mathbf{A}$, or $\mathbf{B}$ \emph{includes} $\mathbf{A}$, and write $\mathbf{A} \subseteq \mathbf{B}$ or $\mathbf{B} \supseteq \mathbf{A}$, iff every element of $\mathbf{A}$ is an element of $\mathbf{B}$.

    If, in addition, $\mathbf{A} \neq \mathbf{B}$, then we say $\mathbf{A}$ is a \emph{proper} subclass of $\mathbf{B}$, $\mathbf{B}$ is a \emph{proper} superclass of $\mathbf{A}$, or $\mathbf{B}$ \emph{properly} includes $\mathbf{A}$, and write $\mathbf{A} \subsetneq \mathbf{B}$ or $\mathbf{B} \supsetneq \mathbf{A}$.
\end{df}

\begin{prop}
    Every class is a subclass of itself.
\end{prop}

\begin{proof}
    \pf\ For any class $\mathbf{A}$, we have that every element of $\mathbf{A}$ is an element of $\mathbf{A}$. \qed
\end{proof}

\begin{df}[Empty Class]
    The \emph{empty class} $\emptyset$ is $\{ x \mid \bot\}$. All other classes are \emph{nonempty}.
\end{df}

\begin{prop}
    The empty class is a subclass of every class.
\end{prop}

\begin{proof}
    \pf\ For any class $\mathbf{A}$, vacuously every element of $\emptyset$ is an element of $\mathbf{A}$. \qed
\end{proof}

\begin{df}[Universal Class]
    The \emph{universal class} $\mathbf{V}$ is $\{ x \mid \top \}$.
\end{df}

\begin{df}
    Given objects $a_1$, \ldots, $a_n$, we write $\{ a_1, \ldots, a_n \}$ for the class $\{ x \mid x = a_1 \vee \cdots \vee x = a_n \}$.

    A class of the form $\{ a \}$ is called a \emph{singleton}.
\end{df}

\begin{df}[Union]
    The \emph{union} of classes $\mathbf{A}$ and $\mathbf{B}$ is the class $\mathbf{A} \cup \mathbf{B} = \{ x \mid x \in \mathbf{A} \vee x \in \mathbf{B} \}$.
\end{df}

\begin{df}[Intersection]
    The \emph{intersection} of classes $\mathbf{A}$ and $\mathbf{B}$ is the class $\mathbf{A} \cap \mathbf{B} = \{ x \mid x \in \mathbf{A} \wedge x \in \mathbf{B} \}$.
\end{df}

\begin{df}[Relative Complement]
    Let $\mathbf{A}$ and $\mathbf{B}$ be classes. The \emph{relative complement} of $\mathbf{B}$ in $\mathbf{A}$ is the class $\mathbf{A} - \mathbf{B} = \{ x \in \mathbf{A} \mid x \notin \mathbf{B} \}$.
\end{df}

\section{Primitive Notions}

Let there be \emph{sets}.

Let there be a binary relation $\in$ between sets, called \emph{membership}. When $a \in b$ holds, we say $a$ is a \emph{member} or \emph{element} of $b$, or $a$ is \emph{in} $b$, or $b$ \emph{contains} $a$, and we also write $b \ni a$. When this does not hold, we write $a \notin b$ or $b \not\ni a$.

\begin{df}[Pairwise Disjoint]
    Let $A$ be a set. We say the elements of $A$ are \emph{pairwise disjoint} iff, for all $x,y \in A$, if there exists $z$ such that $z \in x$ and $z \in y$, then $x = y$.
\end{df}

\section{Axioms}

\begin{ax}[Extensionality]
    Two sets with exactly the same elements are equal.
\end{ax}

Thanks to this axiom, we may identify a set $A$ with the class $\{ x \mid x \in A \}$. Our usage of the symbols $\in$ and $=$ is consistent.

\begin{df}
    We say that a class $\mathbf{A}$ \emph{is a set} iff there exists a set $A$ such that $A = \mathbf{A}$. That is, $\{ x \mid P(x) \}$ is a class iff there exists a set $A$ such that $\forall x (x \in A \Leftrightarrow P(x))$. Otherwise, $\mathbf{A}$ is a \emph{proper class}.
\end{df}

\begin{df}[Subset]
    A (proper) \emph{subset} of a class is a (proper) subclass that is a set.

    A (proper) \emph{superset} of a class is a (proper) superclass that is a set.
\end{df}

\begin{df}[Union]
    For any class $\mathbf{A}$, the \emph{union} of $\mathbf{A}$ is the class $\{ x \mid \exists A \in \mathbf{A}. x \in A \}$.
\end{df}

\begin{ax}[Regularity]
    For any nonempty set $A$, there exists a set $m \in A$ such that $m$ and $A$ are disjoint.
\end{ax}

\begin{ax}[Union]
    The union of a set is a set.
\end{ax}

\begin{ax}[Replacement]
    For any property $P(x,y)$, the following is an axiom:

    Let $A$ be a set. Assume that, for any $x \in A$, there exists at most one $y$ such that $P(x,y)$. Then $\{ y \mid \exists x \in A. P(x,y) \}$ is a set.
\end{ax}

\begin{ax}[Infinity]
    There exists a set $I$ such that:
    \begin{itemize}
        \item $I$ has an element that is empty.
        \item For all $x \in I$, there exists $y \in I$ such that the elements of $y$ are exactly $x$ and the members of $x$.
    \end{itemize}
\end{ax}

\begin{ax}[Power Set]
    For any set $A$, the class of all subsets of $A$ is a set.
\end{ax}

\begin{ax}[Choice]
    Let $A$ be a set whose elements are nonempty and pairwise disjoint. Then there exists a set $B$ that has exactly one member in common with each member of $A$.
\end{ax}

\section{Basic Constructions on Sets}

\begin{prop}
    The empty class $\emptyset$ is a set.
\end{prop}

\begin{proof}
    Immediate from the Axiom of Infinity. \qed
\end{proof}

\begin{df}[Power Set]
    For any set $A$, the \emph{power set} of $A$, denoted $\mathcal{P} A$, is the set of all subsets of $A$.

    (This is a set by the Power Set Axiom.)
\end{df}

\begin{thm}
    For any sets $a$ and $b$, the class $\{a,b\}$ is a set.
\end{thm}

\begin{proof}
    \pf
    \step{1}{\pflet{$a$ and $b$ be sets.}}
    \step{2}{\pflet{$P(x,y)$ be the property $(x = \emptyset \wedge y = a) \vee (x = \mathcal{P} \emptyset \wedge y = b)$.}}
    \step{3}{For any $x \in \mathcal{P} \mathcal{P} \emptyset$, there exists at most one $y$ such that $P(x,y)$.}
    \begin{proof}
        \step{a}{\pflet{$x \in \mathcal{P} \mathcal{P} \emptyset$}}
        \step{b}{\assume{$P(x,y)$ and $P(x,z)$} \prove{$y = z$}}
        \step{c}{\case{$x = \emptyset$, $y = a$, $x = \emptyset$ and $z = a$}}
        \begin{proof}
            \pf\ Then $y = z$.
        \end{proof}
        \step{d}{\case{$x = \emptyset$, $y = a$, $x = \mathcal{P} \emptyset$ and $z = b$}}
        \begin{proof}
            \pf\ This case is impossible since we have $\emptyset \in \mathcal{P} \emptyset$ but $\emptyset \notin \emptyset$.
        \end{proof}
        \step{e}{\case{$x = \mathcal{P} \emptyset$, $y = b$, $x = \emptyset$ and $z = a$}}
        \begin{proof}
            \pf\ This case is impossible since we have $\emptyset \in \mathcal{P} \emptyset$ but $\emptyset \notin \emptyset$.
        \end{proof}
        \step{f}{\case{$x = \mathcal{P} \emptyset$, $y = b$, $x = \mathcal{P} \emptyset$ and $z = b$}}
        \begin{proof}
            \pf\ Then $y = z$.
        \end{proof}
    \end{proof}
    \step{4}{\pflet{$A = \{ y \mid \exists x \in \mathcal{P} \mathcal{P} \emptyset. P(x,y) \}$}}
    \begin{proof}
        \pf\ By \stepref{3} and the Axiom of Replacement.
    \end{proof}
    \step{5}{$A = \{a,b\}$}
    \begin{proof}
        \step{a}{$a \in A$}
        \begin{proof}
            \pf\ Since $\emptyset \in \mathcal{P} \mathcal{P} \emptyset$ and $P(\emptyset, a)$.
        \end{proof}
        \step{b}{$b \in A$}
        \begin{proof}
            \pf\ Since $\mathcal{P} \emptyset \in \mathcal{P} \mathcal{P} \emptyset$ and $P(\mathcal{P} \emptyset, b)$.
        \end{proof}
        \step{c}{For all $y \in A$ we have $y = a$ or $y = b$.}
        \begin{proof}
            \pf\ Immediate from \stepref{4}.
        \end{proof}
    \end{proof}
    \qed
\end{proof}

\begin{cor}
    For any set $a$, the class $\{a\}$ is a set.
\end{cor}
    
\begin{prop}
    The union of two sets is a set.
\end{prop}

\begin{proof}
    \pf\ Since for sets $A$ and $B$ we have $A \cup B = \bigcup \{A,B\}$. \qed
\end{proof}

\begin{prop}
    For any sets $a_1$, \ldots, $a_n$, the class $\{ a_1, \ldots, a_n \}$ is a set.
\end{prop}

\begin{proof}
    \pf\ It is $\{ a_1 \} \cup \cdots \cup \{ a_n \}$. \qed
\end{proof}

\begin{thm}[Comprehension]
    Every subclass of a set is a set.
\end{thm}

\begin{proof}
    \pf
    \step{1}{\pflet{$A$ be a set, $\mathbf{B}$ a class with $\mathbf{B} \subseteq A$.}}
    \step{2}{\pflet{$P(x,y)$ be the property $x \in \mathbf{B} \wedge y = x$}}
    \step{3}{For any $x \in A$ there exists at most one $y$ such that $P(x,y)$.}
    \step{4}{$\mathbf{B} = \{ y \mid \exists x \in A. P(x,y) \}$.}
    \qedstep
    \begin{proof}
        \pf\ Hence $\mathbf{B}$ is a set by the Axiom of Replacement.
    \end{proof}
    \qed
\end{proof}

\begin{cor}
    For any set $A$ and class $\mathbf{B}$, the intersection $A \cap \mathbf{B}$ is a set.
\end{cor}

\begin{cor}
    For any set $A$ and class $\mathbf{B}$, the relative complement $A - \mathbf{B}$ is a set.
\end{cor}

\begin{thm}[Russell's Paradox]
    The universal class $\mathbf{V}$ is a proper class.
\end{thm}

\begin{proof}
    \pf
    \step{1}{\pflet{$\mathbf{R} = \{ x \mid x \notin x \}$}}
    \step{2}{$\mathbf{R}$ is not a set.}
    \begin{proof}
        \pf\ If it were, we would have $\mathbf{R} \in \mathbf{R}$ if and only if $\mathbf{R} \notin \mathbf{R}$.
    \end{proof}
    \step{3}{$\mathbf{V}$ is not a set.}
    \begin{proof}
        \pf\ By Comprehension.
    \end{proof}
    \qed
\end{proof}

\begin{df}[Intersection]
    The \emph{intersection} of a class $\mathbf{A}$ is the class
    \[ \bigcap \mathbf{A} = \{ x \mid \forall A \in \mathbf{A}. x \in A\} \enspace . \]
\end{df}

\begin{prop}
    The intersection of a nonempty class is a set.
\end{prop}

\begin{proof}
    \pf
    \step{1}{\pflet{$\mathbf{A}$ be a nonempty class.}}
    \step{2}{\pick\ $A \in \mathbf{A}$}
    \step{3}{$\bigcap \mathbf{A} \subseteq A$}
    \step{4}{$\bigcap \mathbf{A}$ is a set.}
    \begin{proof}
        \pf\ By Comprehension.
    \end{proof}
    \qed
\end{proof}

\chapter{Ordered Pairs and Relations}

\begin{df}[Ordered Pair]
    For any sets $a$ and $b$, the \emph{ordered pair} $(a,b)$ is defined to be $\{ \{ a \}, \{ a,b\}\}$.
\end{df}

\begin{prop}
    For any sets $a$, $b$, $c$ and $d$, if $(a,b) = (c,d)$ then $a = c$ and $b = d$.
\end{prop}

\begin{proof}
    \pf
    \step{1}{\pflet{$a$, $b$, $c$, $d$ be sets.}}
    \step{2}{\assume{$(a,b) = (c,d)$}}
    \step{3}{$a = c$}
    \begin{proof}
        \pf\ Since $\{a\} = \bigcap (a,b) = \bigcap (c,d) = \{c\}$.
    \end{proof}
    \step{4}{$\{a,b\} = \{c,d\}$}
    \begin{proof}
        \pf\ Since $\{a,b\} = \bigcup (a,b) = \bigcup (c,d) = \{c,d\}$.
    \end{proof}
    \step{4}{$b = d$}
    \begin{proof}
        \step{a}{\case{$a = b$}}
        \begin{proof}
            \pf\ Then the set $\{a,b\} = \{c,d\}$ is a singleton, and so $a = b = c = d$.
        \end{proof}
        \step{b}{\case{$a \neq b$}}
        \begin{proof}
            \pf\ Then we have $\{b\} = \{a,b\} - \{a\}= \{c,d\} - \{c\}$ and so $b = d$.
        \end{proof}
    \end{proof}
    \qed
\end{proof}

\begin{df}[Cartesian Product]
\end{df}

\end{document}