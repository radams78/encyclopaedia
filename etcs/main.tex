\documentclass{book}

\title{Mathematics}
\author{Robin Adams}

\usepackage{amsmath}
\usepackage{amssymb}
\usepackage{amsthm}
\let\proof\relax
\let\endproof\relax
\let\qed\relax
\usepackage{pf2}
\usepackage{hyperref}
\usepackage{mathabx}
\usepackage[all]{xy}

\newtheorem{ax}{Axiom}[chapter]
\newtheorem{prop}[ax]{Proposition}
\newtheorem{cor}{Corollary}[ax]
\newtheorem{thm}[ax]{Theorem}
\newtheorem{lm}[ax]{Lemma}
\theoremstyle{definition}
\newtheorem{df}[ax]{Definition}
\newtheorem{ex}[ax]{Example}

\newcommand{\id}[1]{\ensuremath{\mathrm{id}_{#1}}}
\newcommand{\inv}[1]{\ensuremath{{#1}^{-1}}}

\begin{document}

\maketitle
\tableofcontents

\chapter{Sets and Functions}

\section{Primitive Terms}

Let there be \emph{sets}.

Given sets $A$ and $B$, let there be \emph{functions} from $A$ to $B$. We write $f : A \rightarrow B$ iff $f$ is a function from $A$ to $B$, and call $A$ the \emph{domain} of $f$ and $B$ the \emph{codomain}.

Given functions $f : A \rightarrow B$ and $g : B \rightarrow C$, let there be a function $g \circ f : A \rightarrow C$, the \emph{composite} of $f$ and $g$.

\section{The Axioms}

\begin{ax}[Associativity]
Given $f : A \rightarrow B$, $g : B \rightarrow C$ and $h : C \rightarrow D$, we have
\[ h (gf) = (hg) f \enspace . \]
\end{ax}

\begin{ax}[Identity]
For any set $A$, there exists a function $i : A \rightarrow A$ such that:
\begin{itemize}
\item for any set $B$ and function $f : A \rightarrow B$, we have $fi = f$
\item for any set $B$ and function $f : B \rightarrow A$, we have $if = f$.
\end{itemize}
\end{ax}

\begin{prop}
For any set $A$, there exists a unique function $i : A \rightarrow A$ such that:
\begin{itemize}
\item for any set $B$ and function $f : A \rightarrow B$, we have $fi = f$
\item for any set $B$ and function $f : B \rightarrow A$, we have $if = f$.
\end{itemize}
\end{prop}

\begin{proof}
\pf\ If $i$ and $j$ both satisfy these conditions then $i = ij = j$. \qed
\end{proof}

\begin{df}[Identity Function]
For any set $A$, the \emph{identity function} on $A$, $\mathrm{id}_A$, is the unique function $A \rightarrow A$ such that:
\begin{itemize}
\item for any set $B$ and function $f : A \rightarrow B$, we have $f \mathrm{id}_A = f$
\item for any set $B$ and function $f : B \rightarrow A$, we have $\mathrm{id}_B f = f$.
\end{itemize}
\end{df}

\begin{df}[Isomorphism]
A function $f : A \rightarrow B$ is an \emph{isomorphism}, $f : A \cong B$, iff there exists a function $g : B \rightarrow A$ such that $fg = \mathrm{id}_B$ and $gf = \mathrm{id}_A$.
\end{df}

\begin{ax}[Terminal Set]
There exists an \emph{empty} set $\emptyset$ such that, for any set $A$, there exists exists exactly one function $\emptyset \rightarrow A$.
\end{ax}

\begin{prop}
If $S$ and $T$ are empty sets then there exists a unique isomorphism $S \cong T$.
\end{prop}

\begin{proof}
\pf
\step{1}{\pflet{$f$ be the unique function $S \rightarrow T$}}
\step{2}{\pflet{$f^{-1}$ be the unique function $T \rightarrow S$}}
\step{3}{$f f^{-1} = \mathrm{id}_T$}
\begin{proof}
	\pf\ Each is the unique function $T \rightarrow T$.
\end{proof}
\step{4}{$f^{-1} f = \mathrm{id}_S$}
\begin{proof}
	\pf\ Each is the unique function $S \rightarrow S$.
\end{proof}
\qed
\end{proof}

\begin{df}[Empty Set]
Let $\emptyset$ be the set such that, for any set $A$, there exists exactly one function $\emptyset \rightarrow A$.
\end{df}

\begin{ax}[Terminal Set]
There exists a \emph{terminal} set $1$ such that, for any set $A$, there exists exists exactly one function $A \rightarrow 1$.1
\end{ax}

\begin{prop}
If $S$ and $T$ are terminal sets then there exists a unique isomorphism $S \cong T$.
\end{prop}

\begin{proof}
\pf
\step{1}{\pflet{$f$ be the unique function $S \rightarrow T$}}
\step{2}{\pflet{$f^{-1}$ be the unique function $T \rightarrow S$}}
\step{3}{$f f^{-1} = \mathrm{id}_T$}
\begin{proof}
	\pf\ Each is the unique function $T \rightarrow T$.
\end{proof}
\step{4}{$f^{-1} f = \mathrm{id}_S$}
\begin{proof}
	\pf\ Each is the unique function $S \rightarrow S$.
\end{proof}
\qed
\end{proof}

\begin{df}[Terminal Set]
Let 1 be the set such that, for any set $A$, there exists exactly one function $!_A : A \rightarrow 1$.
\end{df}

\begin{df}[Element]
An \emph{element} of a set $A$ is a function $1 \rightarrow A$. We write $a \in A$ for $a : 1 \rightarrow A$.

Given $f : A \rightarrow B$ and $a \in A$, we write $f(a)$ for $fa$.
\end{df}

\begin{ax}[Extensionality]
Let $A$ and $B$ be sets. Let $f,g : A \rightarrow B$. If $\forall x \in A. f(x) = g(x)$ then $f = g$.
\end{ax}

\begin{ax}[Non-degeneracy]
The empty set $\emptyset$ has no elements.
\end{ax}

\begin{ax}[Disjoint Unions]
For any sets $A$ and $B$, there exists a set $A + B$, the \emph{disjoint union} of $A$ and $B$, and functions $\kappa_1 : A \rightarrow A + B$, $\kappa_2 : B \rightarrow A + B$, the \emph{injections}, such that, for any set $X$ and functions $f : A \rightarrow X$ and $g : B \rightarrow X$, there exists a unique function $[f,g] : A + B \rightarrow X$ such that
\[ [f,g] \kappa_1 = f, \qquad [f,g] \kappa_2 = g \enspace . \]
\end{ax}

\begin{df}[Surjective]
A function $f : A \rightarrow B$ is \emph{surjective}, $f : A \twoheadrightarrow B$, iff, for all $b \in B$, there exists $a \in A$ such that $f(a) = b$.
\end{df}

\begin{prop}
If $f : A \twoheadrightarrow B$ and $g : B \twoheadrightarrow C$ are surjective then $gf : A \twoheadrightarrow C$ is surjective.
\end{prop}

\begin{proof}
\pf
\step{1}{\pflet{$c \in C$}}
\step{2}{\pick\ $b \in B$ such that $g(b) = c$.}
\step{3}{\pick\ $a \in A$ such that $f(a) = b$.}
\step{4}{$gf(a) = c$}
\qed
\end{proof}

\begin{df}[Injective]
A function $f : A \rightarrow B$ is \emph{injective},
$f : A \rightarrowtail B$, iff, for all $x, x' \in A$, if $f(x) = f(x')$ then $x = x'$.
\end{df}

\begin{prop}
If $f : A \rightarrowtail B$ and $g : B \rightarrowtail C$ are injective then $gf : A \rightarrowtail C$ is injective.
\end{prop}

\begin{proof}
\pf\ If $g(f(x)) = g(f(x'))$ then $f(x) = f(x')$ since $g$ is injective, hence $x = x'$ since $f$ is injective. \qed
\end{proof}

\begin{prop}
Let $f : A \rightarrow B$ and $g : B \rightarrow C$. If $gf$ is injective then $f$ is injective.
\end{prop}

\begin{proof}
\pf
\step{1}{\pflet{$x, x' \in A$}}
\step{2}{\assume{$f(x) = f(x')$}}
\step{3}{$g(f(x)) = g(f(x'))$}
\step{4}{$x = x'$}
\qed
\end{proof}

\begin{prop}
Let $f : A \rightarrow B$ be injective. For any set $X$ and functions $x,y : X \rightarrow A$, if $fx = fy$ then $x = y$.
\end{prop}

\begin{proof}
\pf
\step{1}{\pflet{$f : A \rightarrow B$}}
\step{2}{\assume{$f$ is injective.}}
\step{3}{\pflet{$X$ be a set.}}
\step{4}{\pflet{$x, y : X \rightarrow A$}}
\step{5}{\assume{$fx = fy$}}
\step{6}{\pflet{$t \in X$} \prove{$x(t) = y(t)$}}
\step{7}{$f(x(t)) = f(y(t))$}
\begin{proof}
	\pf\ \stepref{5}
\end{proof}
\step{8}{$x(t) = y(t)$}
\begin{proof}
	\pf\ \stepref{2}
\end{proof}
\qed
\end{proof}

\begin{prop}
Any function $f : 1 \rightarrow A$ is injective.
\end{prop}

\begin{proof}
\pf\ For any $x,y \in 1$, if $f(x) = f(y)$ then $x = y$ since $1$ has only one element. \qed
\end{proof}

\begin{prop}
For any sets $A$ and $B$, the injections $\kappa_1 : A \rightarrow A + B$ and $\kappa_2 : B \rightarrow A + B$ are injective.
\end{prop}

\begin{proof}
\pf
\step{1}{$\kappa_1$ is injective.}
\begin{proof}
	\step{a}{\pflet{$x,y \in A$}}
	\step{b}{\assume{$\kappa_1(x) = \kappa_1(y)$}}
	\step{c}{\pflet{$f : A + B \rightarrow A$ be the function $f = [\mathrm{id}_A, x \circ !_B]$}}
	\step{d}{$x = y$}
	\begin{proof}
		\pf\ $x = f(\kappa_1(x)) = f(\kappa_1(y)) = y$.
	\end{proof}
\end{proof}
\step{2}{$\kappa_2$ is injective.}
\begin{proof}
	\pf\ Similar.
\end{proof}
\qed
\end{proof}

\begin{df}[Bijective]
A function is \emph{bijective} iff it is injective and surjective.
\end{df}

\begin{df}[Constant]
A function $f : A \rightarrow B$ is \emph{constant} iff there exists $b \in B$ such that $f = b \circ !_A$.
\end{df}

\section{Subsets}

\begin{df}[Subset]
A \emph{subset} of a set $A$ is a pair $(B,i)$ such that $B$ is a set and $i : B \rightarrowtail A$ is an injective function.
\end{df}

\begin{df}[Equality of Subsets]
Given subsets $(U,i)$ and $(V,j)$ of a set $A$, we say $(U,i)$ and $(V,j)$ are \emph{equal}, $(U,i) = (V,j)$, iff there exists an isomorphism $h : U \cong V$ such that $jh = i$.
\end{df}

\begin{df}[Inclusion]
Let $(B,i)$ and $(C,j)$ be subsets of $A$. Then $(B,i)$ is \emph{included} in $(C,j)$, $(B,i) \subseteq (C,j)$, iff there exists $k : B \rightarrow C$ such that $jk=i$.
\end{df}

\begin{prop}
Let $(U,i)$ and $(V,j)$ be subsets of a set $A$. Then we have $(U,i) = (V,j)$ iff $(U,i) \subseteq (V,j)$ and $(V,j) \subseteq (U,i)$.
\end{prop}

\begin{proof}
\pf
\step{1}{\pflet{$(U,i)$ and $(V,j)$ be subsets of a set $A$.}}
\step{2}{If $(U,i) = (V,j)$ then $(U,i) \subseteq (V,j)$ and $(V,j) \subseteq (U,i)$}
\begin{proof}
	\step{a}{\assume{$(U,i) = (V,j)$}}
	\step{b}{\pick\ an isomorphism $h : U \cong V$ such that $jh = i$.}
	\step{c}{$(U,i) \subseteq (V,j)$}
	\begin{proof}
		\pf\ Since $jh = i$.
	\end{proof}
	\step{d}{$(V,j) \subseteq (U,i)$}
	\begin{proof}
		\pf\ Since $i \inv{h} = j$.
	\end{proof}
\end{proof}
\step{3}{If $(U,i) \subseteq (V,j)$ and $(V,j) \subseteq (U,i)$ then $(U,i) = (V,j)$.}
\begin{proof}
	\step{a}{\assume{$(U,i) \subseteq (V,j)$ and $(V,j) \subseteq (U,i)$}}
	\step{b}{\pick\ $h : U \rightarrow V$ such that $jh = i$.}
	\step{c}{\pick\ $\inv{h} : V \rightarrow U$ such that $i \inv{h} = j$.}
	\step{d}{$h \inv{h} = \id{V}$}
	\begin{proof}
		\pf\ $j h \inv{h} = i \inv{h} = j$
	\end{proof}
	\step{e}{$\inv{h} h = \id{U}$}
	\begin{proof}
		\pf\ $i \inv{h} h = j h = i$
	\end{proof}
\end{proof}
\qed
\end{proof}

\begin{df}[Membership]
Let $(B,i)$ be a subset of $A$ and $a \in A$. Then $a$ is a \emph{member} of $(B,i)$, $a \in (B,i)$, iff there exists $b \in B$ such that $i(b) = a$.
\end{df}

\begin{prop}
Let $A$ be a set. Let $a \in A$, and let $S$ and $T$ be subsets of $A$. If $a \in S$ and $S \subseteq T$ then $a \in T$.
\end{prop}

\begin{proof}
\pf
\step{1}{\pflet{$S = (B,i)$ and $T = (C,j)$}}
\step{2}{\pick\ $k : B \rightarrow C$ such that $jk = i$}
\step{3}{\pick\ $b \in B$ such that $i(b) = a$}
\step{4}{$j(k(b)) = a$}
\step{5}{$a \in T$}
\qed
\end{proof}

\begin{cor}
If $a \in S$ and $S = T$ then $a \in T$.
\end{cor}

%TODO If $\forall x \in S. x \in T$ then $S \subseteq T$

\section{The Subset Classifier}

\begin{df}
\[ 2 = 1 + 1 \]
Let $\top = \kappa_1 \in 2$.
\end{df}

\begin{df}[Characteristic Function]
Let $i : X \rightarrowtail A$ be an injective function. Then a function $\chi_{(X,i)} : A \rightarrow 2$ is the \emph{characteristic function} of $(X,i)$ if and only if:
\begin{itemize}
\item $\chi i = \top !_X$
\item for every set $T$ and function $a : T \rightarrow A$ such that $\chi_{(X,i)} a = \top !_T$, there exists a unique $\overline{a} : T \rightarrow X$ such that $i \overline{a} = a$.
\end{itemize}
\end{df}

\[ \xymatrix{
T \ar[r]^{\overline{a}} \ar[dr]_a & X \ar[d]^i \ar[r]^{!} & 1 \ar[d]^{\top} \\
& A \ar[r]_{\chi_{(X,i)}} & 2
} \]

\begin{ax}[Subset Classifier]$ $
\begin{enumerate}
\item For any set $A$ and function $\phi : A \rightarrow 2$, there exists a set $X$ and monomorphism $i : X \rightarrowtail A$ such that $\phi$ is the characteristic function of $(X,i)$.
\item For any set $A$, every part of $A$ has a characteristic function.
\end{enumerate}
\end{ax}

\begin{prop}
$\id{2}$ is the characteristic function of $(1,\top)$
\end{prop}

\begin{proof}
For any set $T$ and function $a : T \rightarrow 2$ such that $\id{2} a = \top !_T$, then we have $\top !_T = a$. \qed
\end{proof}

\begin{prop}
Let $A$ be a set. Let $(U,i)$ and $(V,j)$ be subsets of $A$. Then $(U,i) = (V,j)$ if and only if they have the same characteristic function.
\end{prop}

\begin{proof}
\pf
\step{1}{If $(U,i) = (V,j)$ then $(U,i)$ and $(V,j)$ have the same characteristic function.}
\begin{proof}
	\step{a}{\assume{$(U,i) = (V,j)$}}
	\step{b}{\pflet{$h : U \cong V$ be the isomorphism such that $jh = i$.}}
	\step{c}{\pflet{$\chi : A \rightarrow 2$ be the characteristic function of $(U,i)$.} \prove{$\chi$ is the characteristic function of $(V,j)$.}}
	\step{cc}{$\chi j = \top !_V$}
	\begin{proof}
		\step{i}{$\chi i = \top !_U$}
		\step{ii}{$\chi j h = \top !_U$}
		\step{iii}{$\chi j = \top !_U \inv{h}$}
		\step{iv}{$\chi j = \top !_V$}
	\end{proof}
	\step{d}{\pflet{$T$ be a set and $a : T \rightarrow A$ satisfy $\chi a = \top !_T$}}
	\step{e}{\pflet{$\overline{a} : T \rightarrow U$ be the unique function such that $i \overline{a} = a$.}}
	\step{f}{$h \overline{a}$ is the unique function $T \rightarrow V$ such that $j h \overline{a} = a$.}
\end{proof}
\step{2}{If $(U,i)$ and $(V,j)$ have the same characteristic function then $(U,i) = (V,j)$.}
\begin{proof}
	\step{a}{\assume{$\chi : A \rightarrow 2$ is the characteristic function of $(U,i)$ and of $(V,j)$}}
	\step{b}{$\chi i = \top !_U$}
	\step{c}{There exists $h : U \rightarrow V$ such that $jh = i$.}
	\step{d}{$(U,i) \subseteq (V,j)$}
	\step{d}{$\chi j = \top !_V$}
	\step{e}{There exists $k : V \rightarrow U$ such that $ik = j$.}
	\step{f}{$(V,j) \subseteq (U,i)$}
	\step{g}{$(U,i) = (V,j)$}
\end{proof}
\qed
\end{proof}

%TODO 2 is a coseparator
\end{document}