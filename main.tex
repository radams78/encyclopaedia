\documentclass{book}

\title{Encyclopaedia of Mathematics and Physics}
\author{Robin Adams}
\date{}

\usepackage{amsthm}
\let\proof\relax
\let\endproof\relax
\let\qed\relax
\usepackage{pf2}

\newtheorem{prop}{Proposition}[chapter]
\newtheorem{cor}{Corollary}[prop]
\theoremstyle{definition}
\newtheorem{df}[prop]{Definition}

\begin{document}

\maketitle
\tableofcontents

\chapter{Relations}

\begin{df}[Antisymmetric]
A relation $R$ on a set $A$ is \emph{antisymmetric} iff, whenever $xRy$ and $yRx$, then $x = y$.
\end{df}

\begin{df}[Transitive]
A relation $R$ on a type $A$ is \emph{transitive} iff, whenever $xRy$ and $yRz$, then $xRz$.
\end{df}

\chapter{Order Theory}

\begin{df}[Linear Order]
A \emph{linear order} on a set $A$ is a binary relation $\leq$ on $A$ that is transitive, antisymmetric and:
\[ \forall x,y \in A. x \leq y \vee y \leq x \enspace . \]

A \emph{linearly ordered set} is a pair $(A, \leq)$ where $A$ is a set and $\leq$ is a binary relation on $A$.

We write $x < y$ for $x \leq y$ and $x \neq y$.
\end{df}

\begin{df}[Upper Bound]
Let $S$ be a linearly ordered set, $u \in S$ and $E \subseteq S$. Then $u$ is an \emph{upper bound} in $E$ iff $\forall x \in E. x \leq u$. We say $E$ is \emph{bounded above} iff it has an upper bound.

The \emph{up-set} of $E$, denoted $E \uparrow$, is the set of upper bounds of $E$.
\end{df}

\begin{df}[Lower Bound]
Let $S$ be a linearly ordered set, $l \in S$ and $E \subseteq S$. Then $u$ is an \emph{lower bound} in $E$ iff $\forall x \in E. l \leq x$. We say $E$ is \emph{bounded below} iff it has a lower bound.

The \emph{down-set} of $E$, denoted $E \downarrow$, is the set of lower bounds of $E$.
\end{df}

\begin{df}[Supremum]
Let $S$ be a linearly ordered set, $u \in S$ and $E \subseteq S$. Then $u$ is the \emph{least upper bound} or \emph{supremum} of $E$ iff $u$ is an upper bound for $E$ and, for any upper bound $u'$ for $E$, we have $u \leq u'$.
\end{df}

\begin{df}[Infimum]
Let $S$ be a linearly ordered set, $l \in S$ and $E \subseteq S$. Then $l$ is the \emph{greatest lower bound} or \emph{infimum} of $E$ iff $l$ is a lower bound for $E$ and, for any lower bound $l'$ for $E$, we have $l' \leq l$.
\end{df}

\begin{df}[Least Upper Bound Property]
A linearly ordered set $S$ has the \emph{least upper bound property} iff every nonempty subset of $S$ that is bounded above has a least upper bound.
\end{df}

\begin{prop}
Let $S$ be a linearly ordered set and $E \subseteq S$.
\begin{enumerate}
\item If $E \downarrow$ has a supremum $l$, then $l$ is the infimum of $E$.
\item If $E \uparrow$ has an infimum $u$, then $U$ is the supremum of $E$.
\end{enumerate}
\end{prop}

\begin{proof}
\pf
\step{1}{If $E \downarrow$ has a supremum $l$, then $l$ is the infimum of $E$.}
\begin{proof}
	\step{1}{$l$ is a lower bound for $E$.}
	\begin{proof}
		\step{a}{\pflet{$x \in E$}}
		\step{b}{$x$ is an upper bound for $E \downarrow$.}
		\begin{proof}
			\pf\ For all $y \in E \downarrow$ we have $y \leq x$.
		\end{proof}
		\step{c}{$l \leq x$}
	\end{proof}
	\step{2}{For any lower bound $l'$ for $E$, we have $l' \leq l$.}
	\begin{proof}
		\pf\ Since $l$ is an upper bound for $E \downarrow$.
	\end{proof}
\end{proof}
\step{2}{If $E \uparrow$ has an infimum $u$, then $u$ is the supremum of $E$.}
\begin{proof}
	\pf\ Dual.
\end{proof}
\qed
\end{proof}

\begin{cor}
A linearly ordered set has the least upper bound property if and only if every nonempty set bounded below has an infimum.
\end{cor}

\chapter{Real Analysis}

\begin{prop}
There is no rational $p$ such that $p^2 = 2$.
\end{prop}

\begin{proof}
\pf
\step{1}{\assume{for a contradiction $p^2 = 2$.}}
\step{2}{\pick\ integers $m$, $n$ not both even such that $p = m / n$.} %TODO
\step{3}{$m^2 = 2n^2$}
\step{4}{$m$ is even.}
\step{5}{\pick\ an integer $k$ such that $m = 2k$.}
\step{6}{$4k^2 = 2n^2$}
\step{7}{$2k^2 = n^2$}
\step{8}{$n$ is even.}
\qedstep
\begin{proof}
	\pf\ \stepref{2}, \stepref{4} and \stepref{8} form a contradiction.
\end{proof}
\qed
\end{proof}
\end{document}