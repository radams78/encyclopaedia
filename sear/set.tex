\part{Set Theory}
\chapter{Primitive Terms and Axioms}

\section{Primitive Terms} % CHECKED FORMALIZED

Let there be \emph{sets}.

For any set $A$, let there be \emph{elements} of $A$. We write $a \in A$ for: $a$ is an element of $A$.

For any sets $A$ and $B$, let there be a set $B^A$, whose elements are called \emph{functions} from $A$ to $B$. We write $f : A \rightarrow B$ for $f \in B^A$.

For any function $f : A \rightarrow B$ and element $a \in A$, let there be an element $f(a) \in B$, the \emph{value} of the function $f$ at the \emph{argument} $a$.

\section{Injections, Surjections and Bijections} % CHECKED FORMALIZED

\begin{df}[Injective]
A function $f : A \rightarrow B$ is \emph{injective} or an \emph{injection} iff, for all $x,y \in A$, if $f(x) = f(y)$ then $x = y$.
\end{df}

\begin{df}[Surjective]
A function $f : A \rightarrow B$ is \emph{surjective} or a \emph{surjection} iff, for all $y \in B$, there exists $x \in A$ such that $f(x) = y$.
\end{df}

\begin{df}[Bijective]
A function $f : A \rightarrow B$ is \emph{bijective} or a \emph{bijection} iff it is injective and surjective.

Sets $A$ and $B$ are \emph{equinumerous}, $A \approx B$, iff there exists a bijection between them.
\end{df}

If we prove there exists a set $X$ such that $P(X)$, and that any two sets that satisfy $P$ are bijective, then we may introduce a constant $C$ and define "Let $C$ be the set such that $P(C)$".

\section{Axioms} % CHECKED FORMALIZED

\begin{axs}[Choice]
Let $P[X,Y,x,y]$ be a formula where $X$ and $Y$ are set variables, $x \in X$ and $y \in Y$. Then the following is an axiom.

Let $A$ and $B$ be sets. Assume that, for all $a \in A$, there exists $b \in B$ such that $P[A,B,a,b]$. Then there exists a function $f : A \rightarrow B$ such that $\forall a \in A. P[A,B,a,f(a)]$.
\end{axs}

\begin{ax}[Extensionality]
Let $f, g : A \rightarrow B$. If, for all $x \in A$, we have $f(x) = g(x)$, then $f = g$.
\end{ax}

\begin{df}[Composition]
Let $f : A \rightarrow B$ and $g : B \rightarrow C$. The \emph{composite} $g \circ f : A \rightarrow C$ is the function such that, for all $a \in A$, we have
\[ (g \circ f)(a) = g(f(a)) \enspace . \]
\end{df}

\begin{ax}[Pairing]
For any sets $A$ and $B$, there exists a set $A \times B$, the \emph{Cartesian product} of $A$ and $B$, and functions $\pi_1 : A \times B \rightarrow A$ and $\pi_2 : A \times B \rightarrow B$ such that, for all $a \in A$ and $b \in B$, there exists a unique $(a,b) \in A \times B$ such that $\pi_1(a,b) = a$ and $\pi_2(a,b) = b$.
\end{ax}

\begin{axs}[Separation]
For every property $P[X,x]$ where $X$ is a set variable and $x \in X$, the following is an axiom:

For every set $A$, there exists a set $S = \{ x \in A : P[A,x] \}$ and an injection $i : S \rightarrow A$ such that, for all $x \in A$, we have
\[ (\exists y \in S. i(y) = x) \Leftrightarrow P[A,x] \enspace . \]
\end{axs}

\begin{ax}[Infinity]
There exists a set $\mathbb{N}$, an element $0 \in \mathbb{N}$, and a function $s : \mathbb{N} \rightarrow \mathbb{N}$ such that:
\begin{itemize}
\item $\forall n \in \mathbb{N}. s(n) \neq 0$
\item $\forall m,n \in \mathbb{N}. s(m) = s(n) \Rightarrow m = n$.
\end{itemize}
\end{ax}

\begin{axs}[Collection]
Let $P[X,Y,x]$ be a formula with set variables $X$ and $Y$ and an element variable $x \in X$. Then the following is an axiom.

For any set $A$, there exist sets $B$ and $Y$ and functions $p : B \rightarrow A$, and $m : B \times Y \Rightarrow \mathbb{N}$ such that:
\begin{itemize}
\item $m$ is injective.
\item $\forall b \in B. P[A, \{ y \in Y : m(b,y) = 0 \}, p(b)]$
\item For all $a \in A$, if $\exists Y. P[A,Y,a]$, then there exists $b \in B$ such that $a = p(b)$.
\end{itemize}
\end{axs}

\begin{ax}[Universe]
There exists a set $E$, a set $U$ and a function $el : E \rightarrow U$ such that the following holds.

Let us say that a set $A$ is \emph{small} iff there exists $u \in U$ such that $A \approx \{ e \in E : el(e) = u \}$.

\begin{itemize}
\item $\mathbb{N}$ is small.
\item For any $U$-small sets $A$ and $B$, the set $B^A$ is small.
\item For any $U$-small sets $A$ and $B$, the set $A \times B$ is small.
\item Let $f : A \rightarrow B$ be a function. If $B$ is small and $\{ a \in A : f(a) = b \}$ is small for all $b \in B$, then $A$ is small.
\item If $p : B \twoheadrightarrow A$ is a surjective function such that $A$ is small, then there exists a $U$-small set $C$, a surjection $q : C \twoheadrightarrow A$, and a function $f : C \rightarrow B$ such that $q = p \circ f$.
\end{itemize}
\end{ax}

\chapter{Sets and Functions}

\section{Composition} % CHECKED

\begin{prop}
Given functions $f : A \rightarrow B$, $g : B \rightarrow C$ and $h : C \rightarrow D$, we have
\[ h \circ (g \circ f) = (h \circ g) \circ f \enspace . \]
\end{prop}

\begin{proof}
\pf
\step{1}{For all $x \in A$ we have $(h \circ (g \circ f))(x) = ((h \circ g) \circ f)(x)$.}
\begin{proof}
	\step{a}{\pflet{$x \in A$}}
	\step{b}{$(h \circ (g \circ f))(x) = ((h \circ g) \circ f)(x)$}
	\begin{proof}
		\pf
		\begin{align*}
			(h \circ (g \circ f))(x)
			& = h ((g \circ f)(x)) & (\text{Definition of composition}) \\
			& = h(g(f(x))) & (\text{Definition of composition}) \\
			& = (h \circ g)(f(x)) & (\text{Definition of composition}) \\
			& = ((h \circ g) \circ f)(x) & (\text{Definition of composition})
		\end{align*}
	\end{proof}
\end{proof}
\qedstep
\begin{proof}
	\pf\ By the Axiom of Extensionality.
\end{proof}
\qed
\end{proof}

\subsection{Injections} % CHECKED FORMALIZED

\begin{prop}
\label{prop:comp_inj}
The composite of injective functions is injective.
\end{prop}

\begin{proof}
\pf
\step{1}{\pflet{$A$, $B$ and $C$ be sets.}}
\step{2}{\pflet{$f : A \rightarrow B$}}
\step{3}{\pflet{$g : B \rightarrow C$}}
\step{4}{\assume{$g$ is injective.}}
\step{5}{\assume{$f$ is injective.}}
\step{6}{\pflet{$x,y \in A$}}
\step{7}{\assume{$(g \circ f)(x) = (g \circ f)(y)$} \prove{$x = y$}}
\step{8}{$g(f(x)) = g(f(y))$}
\begin{proof}
	\pf
	\begin{align*}
		g(f(x)) & = (g \circ f)(x) & (\text{definition of composition}) \\
		& = (g \circ f)(y) & (\text{\stepref{7}}) \\
		& = g(f(y))& (\text{definition of composition})
	\end{align*}
\end{proof}
\step{9}{$f(x) = f(y)$}
\begin{proof}
	\pf\ \stepref{4}, \stepref{8}
\end{proof}
\step{10}{$x = y$}
\begin{proof}
	\pf\ \stepref{5}, \stepref{9}
\end{proof}
\qed
\end{proof}

\begin{prop}
For functions $f : A \rightarrow B$ and $g : B \rightarrow C$, if $g \circ f$ is injective then $f$ is injective.
\end{prop}

\begin{proof}
\pf
\step{1}{\pflet{$A$, $B$ and $C$ be sets.}}
\step{2}{\pflet{$f : A \rightarrow B$}}
\step{3}{\pflet{$g : B \rightarrow C$}}
\step{4}{\assume{$g \circ f$ is injective.}}
\step{5}{\pflet{$x,y \in A$}}
\step{6}{\assume{$f(x) = f(y)$}}
\step{7}{$(g \circ f)(x) = (g \circ f)(y)$}
\begin{proof}
	\pf
	\begin{align*}
		(g \circ f)(x) & = g(f(x)) & (\text{definition of composition}) \\
		& = g(f(y)) & (\text{\stepref{6}}) \\
		& = (g \circ f)(y) & (\text{definition of composition})
	\end{align*}
\end{proof}
\step{8}{$x = y$}
\begin{proof}
	\pf\ \stepref{4}, \stepref{7}
\end{proof}
\qed
\end{proof}

\begin{prop}
\label{prop:injective}
Let $f : A \rightarrow B$ be injective. For every set $X$ and functions $x,y : X \rightarrow A$, if $f \circ x = f \circ y$ then $x = y$.
\end{prop}

\begin{proof}
\pf
	\step{a}{\assume{$f$ is injective.}}
	\step{b}{\pflet{$X$ be a set.}}
	\step{c}{\pflet{$x,y : X \rightarrow A$}}
	\step{d}{\assume{$f \circ x = f \circ y$}}
	\step{e}{$\forall t \in X. x(t) = y(t)$}
	\begin{proof}
		\step{i}{\pflet{$t \in X$}}
		\step{ii}{$f(x(t)) = f(y(t))$}
		\begin{proof}
			\pf
			\begin{align*}
				f(x(t)) & = (f \circ x)(t) & (\text{definition of composition}) \\
				& = (f \circ y)(t) & (\text{\stepref{d}}) \\
				& = f(y(t)) & (\text{definition of composition})
			\end{align*}
		\end{proof}
		\step{iii}{$x(t) = y(t)$}
		\begin{proof}
			\pf\ \stepref{a}, \stepref{ii}
		\end{proof}
	\end{proof}
	\step{f}{$x = y$}
	\begin{proof}
		\pf\ Axiom of Extensionality, \stepref{e}
	\end{proof}
\qed
\end{proof}

We will prove the converse as Proposition \ref{prop:injection2}.

\subsection{Surjections} % CHECKED FORMALIZED

\begin{prop}
\label{prop:comp_surj}
The composite of surjective functions is surjective.
\end{prop}

\begin{proof}
\pf
\step{1}{\pflet{$A$, $B$ and $C$ be sets.}}
\step{2}{\pflet{$f : A \rightarrow B$ and $g : B \rightarrow C$}}
\step{3}{\assume{$g$ is surjective.}}
\step{4}{\assume{$f$ is surjective.}}
\step{5}{\pflet{$c \in C$}}
\step{6}{\pick\ $b \in B$ such that $g(b) = c$.}
\begin{proof}
	\pf\ \stepref{3}
\end{proof}
\step{7}{\pick\ $a \in A$ such that $f(a) = b$.}
\begin{proof}
	\pf\ \stepref{4}
\end{proof}
\step{8}{$(g \circ f)(a) = c$}
\begin{proof}
	\pf
	\begin{align*}
		(g \circ f)(a) & = g(f(a)) & (\text{definition of composition}) \\
		& = g(b) & (\text{\stepref{7}}) \\
		& = c & (\text{\stepref{6}})
	\end{align*}
\end{proof}
\qed
\end{proof}

\begin{prop}
Let $f : A \rightarrow B$ and $g : B \rightarrow C$. If $g \circ f$ is surjective then $g$ is surjective.
\end{prop}

\begin{proof}
\pf
\step{1}{\pflet{$A$, $B$ and $C$ be sets.}}
\step{2}{\pflet{$f : A \rightarrow B$ and $g : B \rightarrow C$.}}
\step{3}{\assume{$g \circ f$ is surjective.}}
\step{4}{\pflet{$c \in C$}}
\step{5}{\pick\ $a \in A$ such that $(g \circ f)(a) = c$}
\begin{proof}
	\pf\ \stepref{3}
\end{proof}
\step{6}{$g(f(a)) = c$}
\begin{proof}
	\pf\ From \stepref{5} and the definition of composition.
\end{proof}
\qedstep
\begin{proof}
	\pf\ There exists $b \in B$ such that $g(b) = c$, namely $b = f(a)$.
\end{proof}
\qed
\end{proof}

\subsection{Bijections} % CHECKED

\begin{prop}
The composite of bijections is a bijection.
\end{prop}

\begin{proof}
\pf\ Propositions \ref{prop:comp_inj} and \ref{prop:comp_surj}. \qed
\end{proof}

\begin{thm}[Schroeder-Bernstein]
Let $A$ and $B$ be sets. If there exist injections $A \rightarrow B$ and $B \rightarrow A$, then $A \approx B$.
\end{thm}

\begin{proof}
\pf
\step{1}{\pflet{$f : A \rightarrowtail B$ and $g : B \rightarrowtail A$ be injections.}}
\step{2}{Define the subsets $A_n$ of $A$ by
\begin{align*}
A_0 & := A - g(B) \\
A_{n+1} & := g(f(A_n))
\end{align*}}
\step{3}{Define $h : A \rightarrow B$ by
\[ h(x) = \begin{cases}
f(x) & \text{if } \exists n. x \in A_n \\
\inv{g}(x) & \text{otherwise}
\end{cases} \]}
\step{4}{$h$ is injective.}
\begin{proof}
	\step{a}{\pflet{$x,y \in A$}}
	\step{b}{\assume{$h(x) = h(y)$}}
	\step{c}{\case{$x \in A_m$ and $y \in A_n$.}}
	\begin{proof}
		\pf\ Then $f(x) = f(y)$ so $x = y$ since $f$ is injective.
	\end{proof}
	\step{d}{\case{$x \in A_m$ and there is no $y$ such that $y \in A_n$.}}
	\begin{proof}
		\step{i}{$f(x) = \inv{g}(y)$}
		\step{ii}{$y = g(f(x))$}
		\step{iii}{$y \in A_{m+1}$}
		\qedstep
		\begin{proof}
			\pf\ This is a contradiction.
		\end{proof}
	\end{proof}
	\step{e}{\case{$y \in A_n$ and there is no $m$ such that $x \in A_m$.}}
	\begin{proof}
		\pf\ Similar.
	\end{proof}
	\step{f}{\case{There is no $m$ such that $x \in A_m$ and there is no $n$ such that $y \in A_n$.}}
	\begin{proof}
		\pf\ Then $\inv{g}(x) = \inv{g}(y)$ and so $x = y$.
	\end{proof}
\end{proof}
\step{5}{$h$ is surjective.}
\begin{proof}
	\step{a}{\pflet{$y \in B$}}
	\step{b}{\case{$g(y) \in A_n$}}
	\begin{proof}
		\step{i}{$n \neq 0$}
		\step{ii}{\pick\ $x \in A_{n-1}$ such that $g(y) = g(f(x))$}
		\step{iii}{$y = f(x)$}
		\step{iv}{$y = h(x)$}
	\end{proof}
	\step{c}{\case{There is no $n$ such that $g(y) \in A_n$.}}
	\begin{proof}
		\pf\ Then $h(g(y)) = y$.
	\end{proof}
\end{proof}
\qed
\end{proof}

\begin{prop}
\[ (A \times B)^C \approx A^C \times B^C \]
\end{prop}

\begin{proof}
\pf\ The function that maps $f$ to $(\pi_1 \circ f, \pi_2 \circ f)$ is a bijection. \qed
\end{proof}

\begin{prop}
\[ A^{B \times C} \approx (A^B)^C \]
\end{prop}

\begin{proof}
\pf\ The function $\Phi$ such that $\Phi(f)(c)(b) = f(b,c)$ is a bijection. \qed
\end{proof}

\section{Identity Function} % CHECKED

\begin{df}[Identity]
For any set $A$, the \emph{identity} function $\id{A} : A \rightarrow A$ is the function defined by $\id{A}(a) = a$.
\end{df}

\subsection{Injections, Surjections, Bijections} % CHECKED

\begin{prop}
For any set $A$, the identity function $\id{A}$ is a bijection.
\end{prop}

\begin{proof}
\pf
\step{1}{\pflet{$A$ be a set.}}
\step{2}{$\id{A}$ is injective.}
\begin{proof}
	\pf\ If $\id{A}(x) = \id{A}(y)$ then $x = y$.
\end{proof}
\step{3}{$\id{A}$ is surjective.}
\begin{proof}
	\pf\ For any $y \in A$, there exists $x \in A$ such that $\id{A}(x) = y$, namely $x = y$.
\end{proof}
\qed
\end{proof}

\subsection{Composition} % CHECKED

\begin{prop}
Let $f : A \rightarrow B$. Then $\id{B} \circ f = f = f \circ \id{A}$.
\end{prop}

\begin{proof}
\pf\ Each is the function that maps $a$ to $f(a)$. \qed
\end{proof}

\begin{prop}
Let $f : A \rightarrow B$.
\begin{enumerate}
\item If there exists $g : B \rightarrow A$ such that $g \circ f = \id{A}$ then $f$ is injective.
\item If $f$ is injective and $A$ is nonempty, then there exists $g : B \rightarrow A$ such that $g \circ f = \id{A}$.
\end{enumerate}
\end{prop}

\begin{proof}
\pf
\step{1}{If there exists $g : B \rightarrow A$ such that $g \circ f = \id{A}$ then $f$ is injective.}
\begin{proof}
	\pf\ If $f(x) = f(y)$ then $x = g(f(x)) = g(f(y)) = y$.
\end{proof}
\step{2}{If $f$ is injective and $A$ is nonempty, then there exists $g : B \rightarrow A$ such that $g \circ f = \id{A}$.}
\begin{proof}
	\step{a}{\assume{$f$ is injective and $A$ is nonempty.}}
	\step{b}{\pick\ $a \in A$}
	\step{c}{Choose a function $g : B \rightarrow A$ such that $f(g(x)) = x$ if there exists $y \in A$ such that $f(y) = x$, otherwise $g(x) = a$.}
	\step{d}{\pflet{$x \in A$} \prove{$g(f(x)) = x$}}
	\step{e}{$f(g(f(x))) = f(x)$}
	\step{f}{$g(f(x)) = x$}
\end{proof}
\qed
\end{proof}

\begin{prop}
Let $f : A \rightarrow B$. Then $f$ is surjective if and only if there exists $g : B \rightarrow A$ such that $f \circ g = \id{B}$.
\end{prop}

\begin{proof}
\pf
\step{2}{If $f$ is surjective then there exists $g : B \rightarrow A$ such that $f \circ g = \id{B}$.}
\begin{proof}
	\step{a}{\assume{$f$ is surjective.}}
	\step{b}{\pick\ $g : B \rightarrow A$ such that, for all $b \in B$, we have $f(g(b)) = b$.}
	\begin{proof}
		\pf\ Axiom of Choice.
	\end{proof}
	\step{c}{$f \circ g = \id{B}$.}
\end{proof}
\step{3}{If there exists $g : B \rightarrow A$ such that $f \circ g = \id{B}$ then $f$ is surjective.}
\begin{proof}
	\step{a}{\pflet{$g : B \rightarrow A$ such that $f \circ g = \id{B}$}}
	\step{b}{\pflet{$X$ be a set.}}
	\step{c}{\pflet{$h,k : B \rightarrow X$}}
	\step{d}{\assume{$h \circ f = k \circ f$}}
	\step{e}{$h = k$}
	\begin{proof}
		\pf\ $h = h \circ f \circ g = k \circ f \circ g = k$
	\end{proof}
\end{proof}
\qed
\end{proof}

\begin{cor}
Let $A$ and $B$ be sets.
\begin{enumerate}
\item If there exists a surjective function $A \rightarrow B$ then there exists an injective function $B \rightarrow A$.
\item If there exists an injective function $A \rightarrow B$ and $A$ is nonempty then there exists a surjective function $B \rightarrow A$.
\end{enumerate}
\end{cor}

\begin{prop}
Let $f : A \rightarrow B$. Then $f$ is bijective if and only if there exists a function $\inv{f} : B \rightarrow A$, the \emph{inverse} of $f$, such that $f \circ \inv{f} = \id{B}$ and $\inv{f} \circ f = \id{A}$, in which case the inverse is unique.
\end{prop}

\begin{proof}
\pf
\step{1}{If $f$ is bijective then there exists $\inv{f} : B \rightarrow A$ such that $f \circ \inv{f} = \id{B}$ and $\inv{f} \circ f = \id{A}$.}
\begin{proof}
	\step{a}{\assume{$f$ is bijective.}}
	\step{b}{\pick\ $g : B \rightarrow A$ such that $f \circ g = \id{B}$}
	\begin{proof}
		\pf\ Proposition \ref{prop:surjective}.
	\end{proof}
	\step{c}{$f \circ g \circ f = f$}
	\step{d}{$g \circ f = \id{A}$}
	\begin{proof}
		\pf\ Proposition \ref{prop:injective}.
	\end{proof}
\end{proof}
\step{2}{If there exists $\inv{f} : B \rightarrow A$ such that $f \circ \inv{f} = \id{B}$ and $\inv{f} \circ f = \id{A}$, then $f$ is bijective.}
\begin{proof}
	\step{a}{\pflet{$\inv{f} : B \rightarrow A$ satisfy $f \circ \inv{f} = \id{B}$ and $\inv{f} \circ f = \id{A}$}}
	\step{b}{$f$ is injective.}
	\begin{proof}
		\pf\ If $f(x) = f(y)$ then $x = \inv{f}(f(x)) = \inv{f}(f(y)) = y$.
	\end{proof}
	\step{c}{$f$ is surjective.}
	\begin{proof}
		\pf\ Proposition \ref{prop:surjective}.
	\end{proof}
\end{proof}
\step{3}{If $g, h : B \rightarrow A$ satisfy $f \circ g = \id{B}$ and $g \circ f = \id{A}$ and $f \circ h = \id{B}$ and $h \circ f = \id{A}$ then $g = h$.}
\begin{proof}
	\pf\ We have $g = g \circ f \circ h = h$.
\end{proof}
\qed
\end{proof}

\section{The Empty Set} % CHECKED

\begin{thm}
There exists a set which has no elements.
\end{thm}

\begin{proof}
\pf\ Take $\{ x \in \mathbb{N} : \bot \}$. \qed
\end{proof}

\begin{thm}
If $E$ and $E'$ have no elements then $E \approx E'$.
\end{thm}

\begin{proof}
\pf
\step{1}{\pflet{$E$ and $E'$ have no elements.}}
\step{2}{\pick\ a function $F : E \rightarrow E'$.}
\begin{proof}
	\pf\ Axiom of Choice since vacuously $\forall x \in E. \exists y \in E'. \top$.
\end{proof}
\step{4}{$F$ is injective.}
\begin{proof}
	\pf\ Vacuously, for all $x,y \in E$, if $F(x) = F(y)$ then $x = y$.
\end{proof}
\step{5}{$F$ is surjective.}
\begin{proof}
	\pf\ Vacuously, for all $y \in E$, there exists $x \in E$ such that $F(x) = y$.
\end{proof}
\qed
\end{proof}

\begin{df}[Empty Set]
The \emph{empty set} $\emptyset$ is the set with no elements.
\end{df}

\section{The Singleton} % CHECKED

\begin{thm}
There exists a set that has exactly one element.
\end{thm}

\begin{proof}
\pf\ The set $\{ x \in \mathbb{N} : x = 0 \}$ has exactly one element. \qed
\end{proof}

\begin{thm}
If $A$ and $B$ both have exactly one element then $A \approx B$.
\end{thm}

\begin{proof}
\pf
\step{1}{\pflet{$A$ and $B$ both have exactly one element $a$ and $b$ respectively.}}
\step{2}{\pflet{$F : A \rightarrow B$ be the function such that, for all $x \in A$, we have $(x = a \wedge F(x) = b)$}}
\step{3}{$F$ is a bijection.}
\qed
\end{proof}

\begin{df}[Singleton]
Let 1 be the set that has exactly one element. Let $*$ be its element.
\end{df}

\subsection{Injections}

\begin{prop}
\label{prop:injection2}
Let $f : A \rightarrow B$. Assume that, for every set $X$ and functions $x,y : X \rightarrow A$, if $f \circ x = f \circ y$ then $x = y$. Then $f$ is injective.
\end{prop}

\begin{proof}
\pf\ Take $X = 1$. \qed
\end{proof}

\section{The Set Two}

\begin{prop}
\label{prop:surjective}
Let $f : A \rightarrow B$. Then $f$ is surjective if and only if, for any set $X$ and functions $g,h : B \rightarrow X$, if $g \circ f = h \circ f$ then $g = h$.
\end{prop}

\begin{proof}
\pf
\step{1}{If $f$ is surjective then, for any set $X$ and functions $g,h : B \rightarrow X$, if $g \circ f = h \circ f$ then $g = h$.}
\begin{proof}
	\step{a}{\assume{$f$ is surjective.}}
	\step{b}{\pflet{$X$ be a set.}}
	\step{c}{\pflet{$g, h : B \rightarrow X$}}
	\step{d}{\assume{$g \circ f = h \circ f$}}
	\step{e}{\pflet{$b \in B$} \prove{$g(b) = h(b)$}}
	\step{f}{\pick\ $a \in A$ such that $f(a) = b$}
	\step{g}{$g(b) = h(b)$}
	\begin{proof}
		\pf\ $g(b) = g(f(a)) = h(f(a)) = h(b)$
	\end{proof}
\end{proof}
\step{4}{If, for any set $X$ and functions $g, h : B \rightarrow x$, if $g \circ f = h \circ f$ then $g = h$, then $f$ is surjective.}
\begin{proof}
	\step{a}{\assume{For any set $X$ and functions $g,h : B \rightarrow X$, if $g \circ f = h \circ f$ then $g = h$.}}
	\step{b}{\pflet{$b \in B$}}
	\step{c}{\pflet{$h : B \rightarrow 2$ be the function that maps everything to 1.}}
	\step{d}{\pflet{$k : B \rightarrow 2$ be the function that maps $b$ to 0 and everything else to 1.}}
	\step{e}{$h \neq k$}
	\step{f}{$h \circ f \neq k \circ f$}
	\step{g}{\pick\ $a \in A$ such that $h(f(a)) \neq k(f(a))$}
	\step{h}{$f(a) = b$}
\end{proof}
\qed
\end{proof}

\section{Subsets} % CHECKED

\begin{df}[Subset]
A \emph{subset} of a set $A$ consists of a set $S$ and an injection $i : S \rightarrowtail A$. We write $(S,i) \subseteq A$.

We say two subsets $(S,i)$ and $(T,j)$ are \emph{equal}, $(S,i) = (T,j)$, iff there exists a bijection $\phi : S \approx T$ such that $j \circ \phi = i$.
\end{df}

\begin{prop}
For any subset $(S,i)$ of $A$ we have $(S,i) = (S,i)$.
\end{prop}

\begin{proof}
\pf\ We have $\id{S} : S \approx S$ and $i \circ \id{S} = i$.
\end{proof}

\begin{prop}
If $(S,i) = (T,j)$ then $(T,j) = (S,i)$.
\end{prop}

\begin{proof}
\pf\ If $\phi : S \approx T$ and $j \circ \phi = i$ then $\inv{\phi} : T \approx S$ and $i \circ \inv{\phi} = j$. \qed
\end{proof}

\begin{prop}
If $(R,i) = (S,j)$ and $(S,j) = (T,k)$ then $(R,i) = (T,k)$.
\end{prop}

\begin{proof}
\pf\ If $\phi : R \approx S$ and $j \circ \phi = i$, and $\psi : S \approx T$ and $k \circ \psi = j$, then $\psi \circ \phi : R \approx T$ and $k \circ \psi \circ \phi = i$. \qed
\end{proof}

\begin{df}[Membership]
Given $(S,i) \subseteq A$ and $a \in A$, we write $a \in (S,i)$ for $\exists s \in S. i(s) = a$.
\end{df}

\begin{prop}
If $a \in (S,i)$ and $(S,i) = (T,j)$ then $a \in (T,j)$.
\end{prop}

\begin{proof}
\pf\ If $i(s) = a$ then $j(\phi(s)) = a$. \qed
\end{proof}

\begin{df}[Union]
Given subsets $S$ and $T$ of $A$, the \emph{union} is the subset $\{ x \in A : x \in S \vee x \in T \}$.
\end{df}

\begin{df}[Intersection]
Given subsets $S$ and $T$ of $A$, the \emph{intersection} is the subset $\{ x \in A : x \in S \wedge x \in T \}$.
\end{df}

\begin{prop}[Distributive Law]
\[ R \cap (S \cup T) = (R \cap S) \cup (R \cap T) \]
\end{prop}

\begin{prop}[Distributive Law]
\[ R \cup (S \cap T) = (R \cup S) \cap (R \cup T) \]
\end{prop}

\begin{df}
Given a set $A$, we write $\emptyset$ for the subset $(\emptyset, !)$ where $!$ is the unique function $\emptyset \rightarrow A$.
\end{df}

\begin{prop}
\[ S \cup \emptyset = S \]
\end{prop}

\begin{prop}
\[ S \cap \emptyset = S \]
\end{prop}

\begin{df}[Inclusion]
Given subsets $(S,i)$ and $(T,j)$ of a set $A$, we write $(S,i) \subseteq (T,j)$ iff there exists $f : S \rightarrow T$ such that $j \circ f = i$.
\end{df}

\begin{prop}
\[ \emptyset \subseteq S \]
\end{prop}

\begin{df}[Disjoint]
Subsets $S$ and $T$ of $A$ are \emph{disjoint} iff $S \cap T = \emptyset$.
\end{df}

\begin{df}[Difference]
Given subsets $S$ and $T$ of $A$, the \emph{difference} of $S$ and $T$ is $S - T = \{ x \in A : x \in S \wedge x \notin T \}$.
\end{df}

\begin{prop}[De Morgan's Law]
\[ R - (S \cup T) = (R - S) \cap (R - T) \]
\end{prop}

\begin{prop}[De Morgan's Law]
\[ R - (S \cap T) = (R - S) \cup (R - T) \]
\end{prop}

\section{Saturated Set}

\begin{df}[Saturated]
Let $A$ and $B$ be sets. Let $f : A \rightarrow B$ be surjective. Let $C \subseteq A$. Then $C$ is \emph{saturated} with respect to $f$ iff, for all $x \in C$ and $y \in A$, if $f(x) = f(y)$ then $y \in C$.
\end{df}


\section{Union} % CHECKED

\begin{df}[Union]
Given $\mathcal{A} \in \mathcal{P} \mathcal{P} X$, its \emph{union} is
\[ \bigcup \mathcal{A} := \{ x \in X : \exists S \in \mathcal{A}. x \in S \} \in \mathcal{P} X \enspace . \]
\end{df}

\subsection{Intersection} % CHECKED

\begin{df}[Intersection]
Given $\mathcal{A} \in \mathcal{P} \mathcal{P} X$, its \emph{intersection} is
\[ \bigcap \mathcal{A} := \{ x \in X : \forall S \in \mathcal{A}. x \in S \} \in \mathcal{P} X \enspace . \]
\end{df}

\subsection{Direct Image} % CHECKED

\begin{df}[Direct Image]
Let $f : A \rightarrow B$. Let $S$ be a subset of $A$. The \emph{(direct) image} of $S$ under $f$ is the subset of $B$ given by
\[ f(S) := \{ f(a) : a \in S \} \enspace . \]
\end{df}

\begin{prop}
$ $
\begin{enumerate}
\item If $S \subseteq T$ then $f(S) \subseteq f(T)$
\item $f(\bigcup \mathcal{S}) = \bigcup_{S \in \mathcal{S}} f(S)$
\end{enumerate}
\end{prop}

\begin{ex}
It is not true in general that $f(\bigcap \mathcal{S}) = \bigcap_{S \in \mathcal{S}} f(S)$. Take $f$ to be the only function $\{0,1\} \rightarrow \{0\}$, and $\mathcal{S} = \{\{0\},\{1\}\}$. Then $f(\bigcap \mathcal{S}) = \emptyset$ but $\bigcap_{S \in \mathcal{S}} f(S) = \{0\}$.
\end{ex}

\begin{ex}
It is not true in general that $f(S - T) = f(S) - f(T)$. Take $f$ to be the only function $\{0,1\} \rightarrow \{0\}$, $S = \{0\}$ and $T = \{1\}$. Then $f(S-T) = \{0\}$ but $f(S) - f(T) = \emptyset$.
\end{ex}

\section{Inverse Image} % CHECKED

\begin{df}[Inverse Image]
Let $f : A \rightarrow B$. Let $S$ be a subset of $B$. The \emph{inverse image} or \emph{preimage} of $S$ under $f$ is the subset of $A$ given by
\[ \inv{f}(S) := \{ x \in A : f(x) \in S \} \enspace . \]
\end{df}

\begin{prop}
\begin{enumerate}
\item If $S \subseteq T$ then $\inv{f}(S) \subseteq \inv{f}(T)$
\item $\inv{f}(\bigcup \mathcal{S}) = \bigcup_{S \in \mathcal{S}} \inv{f}(S)$
\item $\inv{f}(\bigcap \mathcal{S}) = \bigcap_{S \in \mathcal{S}} \inv{f}(S)$
\item $\inv{f}(S - T) = \inv{f}(S) - \inv{f}(T)$
\item $S \subseteq \inv{f}(f(S))$. Equality holds if $f$ is injective.
\item $f(\inv{f}(T)) \subseteq T$. Equality holds if $f$ is surjective.
\item $\inv{(g \circ f)}(S) = \inv{f}(\inv{g}(S))$
\end{enumerate}
\end{prop}

\subsection{Saturated Sets}

\begin{prop}
Let $A$ and $B$ be sets. Let $f : A \rightarrow B$ be surjective. Let $C \subseteq A$. Then $C$ is saturated if and only if there exists $D \subseteq B$ such that $C = \inv{f}(D)$.
\end{prop}

\begin{proof}
\pf
\step{1}{If $C$ is saturated then there exists $D \subseteq B$ such that $C = \inv{f}(D)$.}
\begin{proof}
	\step{a}{\assume{$C$ is saturated.}}
	\step{b}{\pflet{$D = f(C)$}}
	\step{c}{$C \subseteq \inv{f}(D)$}
	\begin{proof}
		\step{i}{\pflet{$x \in C$}}
		\step{ii}{$f(x) \in D$}
		\begin{proof}
			\pf\ \stepref{b}
		\end{proof}
		\step{iii}{$x \in \inv{f}(D)$}
	\end{proof}
	\step{d}{$\inv{f}(D) \subseteq C$}
	\begin{proof}
		\step{i}{\pflet{$x \in \inv{f}(D)$}}
		\step{ii}{$f(x) \in D$}
		\step{iii}{\pick\ $y \in C$ such that $f(x) = f(y)$}
		\begin{proof}
			\pf\ \stepref{b}
		\end{proof}
		\step{iv}{$x \in C$}
		\begin{proof}
			\pf\ \stepref{a}
		\end{proof}
	\end{proof}
\end{proof}
\step{2}{If there exists $D \subseteq B$ such that $C = \inv{f}(D)$ then $C$ is saturated.}
\begin{proof}
	\step{a}{\pflet{$D \subseteq B$ be such that $C = \inv{f}(D)$.}}
	\step{b}{\pflet{$x \in C$ and $y \in A$}}
	\step{c}{\assume{$f(x) = f(y)$}}
	\step{d}{$f(x) \in D$}
	\step{e}{$f(y) \in D$}
	\step{f}{$y \in C$}
\end{proof}
\qed
\end{proof}

\section{Relations} % CHECKED

\begin{df}[Relation]
Let $A$ and $B$ be sets. A \emph{relation} $R$ between $A$ and $B$, $R : A \looparrowright B$, is a subset of $A \times B$.

Given $a \in A$ and $b \in B$, we write $aRb$ for $(a,b) \in R$.

A relation \emph{on} a set $A$ is a relation between $A$ and $A$.
\end{df}

\begin{df}[Reflexive]
A relation $R$ on a set $A$ is \emph{reflexive} iff $\forall a \in A. aRa$.
\end{df}

\begin{df}[Symmetric]
A relation $R$ on a set $A$ is \emph{symmetric} iff, whenever $xRy$, then $yRx$.
\end{df}

\begin{df}[Transitive]
A relation $R$ on a set $A$ is \emph{transitive} iff, whenever $xRy$ and $yRz$, then $xRz$.
\end{df}

\subsection{Equivalence Relations}

\begin{df}[Equivalence Relation]
A relation $R$ on a set $A$ is an \emph{equivalence relation} iff it is reflexive, symmetric and transitive.
\end{df}

\begin{df}[Equivalence Class]
Let $R$ be an equivalence relation on a set $A$ and $a \in A$. The \emph{equivalence class} of $a$ with respect to $R$ is
\[ \{ x \in A : xRa \} \enspace . \]
\end{df}

\begin{prop}
Two equivalence classes are either disjoint or equal.
\end{prop}


\section{Power Set}

\begin{df}[Power Set]
The \emph{power set} of a set $A$ is $\mathcal{P} A := 2^A$.

Given $S \in \mathcal{P} A$ and $a \in A$, we write $a \in A$ for $S(a) = 1$.
\end{df}

\begin{df}[Pairwise Disjoint]
Let $P \subseteq \mathcal{P} A$. We say the members of $P$ are \emph{pairwise disjoint} iff, for all $S,T \in P$, if $S \neq T$ then $S \cap T = \emptyset$.
\end{df}

\subsection{Partitions}

\begin{df}[Partition]
Let $A$ be a set. A \emph{partition} of $A$ is a set $P \in \mathcal{P} \mathcal{P} A$ such that:
\begin{itemize}
\item $\bigcup P = A$
\item Every member of $P$ is nonempty.
\item The members of $P$ are pairwise disjoint.
\end{itemize}
\end{df}

\section{Cartesian Product}

\begin{df}[Cartesian Product]
Let $A$ and $B$ be sets. The \emph{Cartesian product} of $A$ and $B$, $A \times B$, is the tabulation of the relation $A \looparrowright B$ that holds for all $a \in A$ and $b \in B$. The associated functions $\pi_1 : A \times B \rightarrow A$ and $\pi_2 : A \times B \rightarrow B$ are called the \emph{projections}.

Given $a \in A$ and $b \in B$, we write $(a,b)$ for the unique element of $A \times B$ such that $\pi_1(a,b) = a$ and $\pi_2(a,b) = b$.
\end{df}

\section{Quotient Sets}

\begin{prop}
Let $\sim$ be an equivalence relation on $X$. Then there exists a set $X/\sim$, the \emph{quotient set} of $X$ with respect to $\sim$, and a surjective function $\pi : X \twoheadrightarrow X / \sim$, the \emph{canonical projection}, such that, for all $x,y \in X$, we have $x \sim y$ if and only if $\pi(x) = \pi(y)$.

Further, if $p : X \twoheadrightarrow Q$ is another quotient with respect to $\sim$, then there exists a unique bijection $\phi : X / \sim \approx Q$ such that $\phi \circ \pi = p$.
\end{prop}

\section{Partitions}

\begin{df}[Partition]
A \emph{partition} of a set $X$ is a set of pairwise disjoint subsets of $X$ whose union is $X$.
\end{df}

\section{Disjoint Union}

\begin{thm}
For any sets $A$ and $B$, there exists a set $A + B$, the \emph{disjoint union} of $A$ and $B$, and functions $\kappa_1 : A \rightarrow A + B$ and $\kappa_2 : B \rightarrow A + B$, the \emph{injections}, such that, for every set $X$ and functions $f : A \rightarrow X$ and $g : B \rightarrow X$, there exists a unique function $[f,g] : A + B \rightarrow X$ such that $[f,g] \circ \kappa_1 = f$ and $[f,g] \circ \kappa_2 = g$.
\end{thm}

\begin{proof}
\pf
\step{1}{\pflet{$A + B := \{ p \in \mathcal{P} A \times \mathcal{P} B : \exists a \in A. p = (\{a\},\emptyset) \vee \exists b \in B. p = (\emptyset, \{b\}) \}$}}
% TODO
\end{proof}

\begin{df}[Restriction]
Let $f : A \rightarrow B$ and let $(S,i)$ be a subset of $A$. The \emph{restriction} of $f$ to $S$ is the function $f \restriction S : S \rightarrow B$ defined by $f \restriction S = f \circ i$.
\end{df}

\section{Natural Numbers}

\begin{thm}[Principle of Recursive Definition]
Let $A$ be a set. Let $F$ be the set of all functions $\{ m \in \mathbb{N} : m < n \} \rightarrow A$ for some $n$. Let $\rho : F \rightarrow A$. Then there exists a unique $g : \mathbb{N} \rightarrow A$ such that, for all $n \in \mathbb{N}$, we have
\[ g(n) = \rho(g \restriction \{ m \in \mathbb{N} : m < n \}) \enspace . \]
\end{thm}

\begin{proof}
\pf
\step{1}{Given a subset $B \subseteq \mathbb{N}$, let us say that a function $g : B \rightarrow A$ is \emph{acceptable} iff, for all $n \in B$, we have
\[ \forall m < n. m \in B \]
and
\[ g(n) = \rho(g \restriction \{ m \in \mathbb{N} : m < n \}) \enspace . \]}
\step{2}{For all $n \in \mathbb{N}$, there exists an acceptable function $\{ m \in \mathbb{N} : m < n \} \rightarrow A$.}
\begin{proof}
	\step{a}{\pflet{$P[n]$ be the property: There exists an acceptable function $\{ m \in \mathbb{N} : m < n \} \rightarrow A$.}}
	\step{b}{$P[0]$}
	\begin{proof}
		\pf\ The unique function $\emptyset \rightarrow A$ is acceptable.
	\end{proof}
	\step{c}{For any natural number $n$, if $P[n]$ then $P[n+1]$.}
	\begin{proof}
		\step{i}{\assume{$P[n]$}}
		\step{ii}{\pick\ an acceptable $f : \{ m \in \mathbb{N} : m < n \} \rightarrow A$.}
		\step{iii}{\pflet{$g : \{ m \in \mathbb{N} : m < n + 1 \} \rightarrow A$ be the function
		\[ g(m) = \begin{cases}
		f(m) & \text{if } m < n \\
		\rho(f) & \text{if } m = n
		\end{cases} \]}}
		\step{iv}{$g$ is acceptable.}
	\end{proof}
\end{proof}
\step{3}{If $g : B \rightarrow A$ and $h : C \rightarrow A$ are acceptable, then $g$ and $h$ agree on $B \cap C$.}
\step{4}{Define $g : \mathbb{N} \rightarrow A$ by: $g(n) = a$ iff there exists an acceptable $h : \{ m \in \mathbb{N} : m < n + 1 \}$ such that $h(n) = a$.}
\step{5}{$g$ is acceptable.}
\step{6}{If $g' : \mathbb{N} \rightarrow A$ is acceptable then $g' = g$.}
\qed
\end{proof}

\section{Finite and Infinite Sets}

\begin{df}[Finite]
A set $A$ is \emph{finite} iff there exists $n \in \mathbb{N}$ such that $A \approx \{ m \in \mathbb{N} : m < n \}$. In this case, we say $A$ has \emph{cardinality} $n$.
\end{df}

\begin{prop}
Let $n \in \mathbb{N}$. Let $A$ be a set. Let $a_0 \in A$. Then $A \approx \{ m \in \mathbb{N} : m < n + 1 \}$ if and only if $A - \{a_0\} \approx \{ m \in \mathbb{N} : m < n \}$.
\end{prop}

\begin{thm}
Let $A$ be a set. Suppose that $A \approx \{ m \in \mathbb{N} : m < n \}$. Let $B$ be a proper subset of $A$. Then $B \not\approx \{ m \in \mathbb{N} : m < n \}$ but there exists $m < n$ such that $B \approx \{ k \in \mathbb{N} : k < m \}$.
\end{thm}

\begin{proof}
\pf
\step{1}{\pflet{$P[n]$ be the property: for every set $A$, if $A approx \{ m \in \mathbb{N} : m < n \}$, then for every proper subset $B$ of $A$, we have $B \not\approx \{ m \in \mathbb{N} : m < n \}$ but there exists $m < n$ such that $B \approx \{ k \in \mathbb{N} : k < m \}$.}}
\step{2}{$P[0]$}
\begin{proof}
	\pf\ If $A \approx \{ m \in \mathbb{N} : m < 0 \}$ then $A$ is empty and so has no proper subset.
\end{proof}
\step{3}{For every natural number $n$, if $P[n]$ then $P[n+1]$.}
\begin{proof}
	\step{a}{\pflet{$n$ be a natural number.}}
	\step{b}{\assume{$P[n]$}}
	\step{c}{\pflet{$A$ be a set.}}
	\step{d}{\assume{$A \approx \{ m \in \mathbb{N} : m < n + 1 \}$}}
	\step{e}{\pflet{$B$ be a proper subset of $A$.}}
	\step{f}{\case{$B = \emptyset$}}
	\begin{proof}
		\pf\ Then $B \not\approx \{ m \in \mathbb{N} : m < n + 1\}$ but $B \approx \{ k \in \mathbb{N} : k < 0 \}$.
	\end{proof}
	\step{g}{\case{$B \neq \emptyset$}}
	\begin{proof}
		\step{i}{\pick\ $b_0 \in B$}
		\step{ii}{$A - \{b_0\} \approx \{ m \in \mathbb{N} : m < n \}$}
		\step{iii}{$B - \{b_0\}$ is a proper subset of $A - \{b_0\}$}
		\step{iv}{$B - \{b_0\} \not\approx \{m \in \mathbb{N} : m < n \}$}
		\step{v}{$B \approx \{m \in \mathbb{N} : m < n + 1\}$}
		\step{vi}{\pick\ $m < n$ such that $B - \{b_0\} \approx \{ k \in \mathbb{N} : k < m \}$}
		\step{vii}{$m + 1 < n + 1$}
		\step{viii}{$B \approx \{ k \in \mathbb{N} : k < m + 1 \}$}
	\end{proof}
\end{proof}
\qed
\end{proof}

\begin{cor}
\label{cor:Dedekind_infinite}
If $A$ is finite then there is no bijection between $A$ and a proper subset of $A$.
\end{cor}

\begin{cor}
$\mathbb{N}$ is infinite.
\end{cor}

\begin{cor}
The cardinality of a finite set is unique.
\end{cor}

\begin{cor}
A subset of a finite set is finite.
\end{cor}

\begin{cor}
If $A$ is finite and $B$ is a proper subset of $A$ then $|B| < |A|$.
\end{cor}

\begin{cor}
Let $A$ be a set. Then the following are equivalent:
\begin{enumerate}
\item $A$ is finite.
\item There exists a surjection from an initial segment of $\mathbb{N}$ onto $A$.
\item There exists an injection from $A$ to an initial segment of $\mathbb{N}$.
\end{enumerate}
\end{cor}

\begin{cor}
A finite union of finite sets is finite.
\end{cor}

\begin{cor}
A finite Cartesian product of finite sets is finite.
\end{cor}

\begin{thm}
Let $A$ be a set. The following are equivalent:
\begin{enumerate}
\item There exists an injective function $\mathbb{N} \rightarrowtail A$.
\item There exists a bijection between $A$ and a proper subset of $A$.
\item $A$ is infinite.
\end{enumerate}
\end{thm}

\begin{proof}
\pf
\step{1}{$1 \Rightarrow 2$}
\begin{proof}
	\step{a}{\pflet{$f : \mathbb{N} \rightarrowtail A$ be injective.}}
	\step{b}{\pflet{$s : \mathbb{N} \approx \mathbb{N} - \{0\}$ be the function $s(n) = n + 1$.}}
	\step{c}{$f \circ s \circ \inv{f} : A \approx A - \{f(0)\}$}
\end{proof}
\step{2}{$2 \Rightarrow 3$}
\begin{proof}
	\pf\ Corollary \ref{cor:Dedekind_infinite}.
\end{proof}
\step{3}{$3 \Rightarrow 1$}
\begin{proof}
	\pf\ Choose a function $f : \mathbb{N} \rightarrow A$ such that $f(n) \in A - \{ f(m) : m < n \}$ for all $n$.
\end{proof}
\qed
\end{proof}

\section{Countable Sets}

\begin{df}[Countable]
A set $A$ is \emph{countably infinite} iff $A \approx \mathbb{N}$.
\end{df}

\begin{prop}
$\mathbb{N} \times \mathbb{N}$ is countably infinite.
\end{prop}

\begin{proof}
\pf\ Define $f : \mathbb{N} \times \mathbb{N} \approx \{ (x,y) \in \mathbb{N} \times \mathbb{N} : y \leq x \}$ by
\[ f(x,y) = (x+y,y) \]
Define $g : \{ (x,y) \in \mathbb{N} \times \mathbb{N} : y \leq x \} \approx \mathbb{N}$ by
\[ g(x,y) = x(x-1)/2 + y \enspace . \qed \]
\end{proof}

\begin{prop}
Every infinite subset of $\mathbb{N}$ is countably infinite.
\end{prop}

\begin{proof}
\pf
\step{1}{\pflet{$C$ be an infinite subset of $\mathbb{N}$}}
\step{2}{Define $h : \mathbb{Z} \rightarrow C$ by recursion thus: $h(n)$ is the smallest element of $C - \{ h(m) : m < n \}$.}
\step{3}{$h$ is injective.}
\begin{proof}
	\pf\ If $m < n$ then $h(m) \neq h(n)$ because $h(n) \in C - \{ h(m) : m < n \}$.
\end{proof}
\step{4}{$h$ is surjective.}
\begin{proof}
	\step{a}{For all $n \in \mathbb{N}$ we have $n \leq h(n)$.}
	\step{b}{\pflet{$c \in C$}}
	\step{c}{$c \leq h(c)$}
	\step{d}{\pflet{$n$ be least such that $c \leq h(n)$}}
	\step{e}{$c \in C - \{ h(m) : m < n \}$}
	\step{f}{$h(n) \leq c$}
	\step{g}{$h(n) = c$}
\end{proof}
\qed
\end{proof}

\begin{df}[Countable]
A set is \emph{countable} iff it is either finite or countably infinite; otherwise it is \emph{uncountable}.
\end{df}

\begin{prop}
Let $B$ be a nonempty set. Then the following are equivalent.
\begin{enumerate}
\item $B$ is countable.
\item There exists a surjection $\mathbb{N} \twoheadrightarrow B$.
\item There exists an injection $B \rightarrowtail \mathbb{N}$.
\end{enumerate}
\end{prop}

\begin{proof}
\pf
\step{1}{$1 \Rightarrow 2$}
\begin{proof}
	\step{a}{\assume{$B$ is countable.}}
	\step{b}{\case{$B$ is finite.}}
	\begin{proof}
		\step{i}{\pick\ a natural number $n$ and bijection $f : \{ m \in \mathbb{N} : m < n \} \approx B$}
		\step{ii}{\pick\ $b \in B$}
		\step{iii}{Extend $f$ to a surjection $g : \mathbb{N} \twoheadrightarrow B$ by setting $g(m) = b$ for $m \geq n$.}
	\end{proof}
	\step{c}{\case{$B$ is countably infinite.}}
	\begin{proof}
		\pf\ Then there exists a bijection $\mathbb{N} \approx B$.
	\end{proof}
\end{proof}
\step{2}{$2 \Rightarrow 3$}
\begin{proof}
	\pf\ Given a surjection $f : \mathbb{N} \twoheadrightarrow B$, define $g : B \rightarrowtail \mathbb{N}$ by $g(b)$ is the smallest number such that $f(g(b)) = b$.
\end{proof}
\step{3}{$3 \Rightarrow 1$}
\begin{proof}
	\step{a}{\pflet{$f : B \rightarrowtail \mathbb{N}$ be injective.}}
	\step{b}{$f(B)$ is countable.}
	\step{c}{$B \approx f(B)$}
	\step{d}{$B$ is countable.}
\end{proof}
\qed
\end{proof}

\begin{cor}
A subset of a countable set is countable.
\end{cor}

\begin{cor}
$\mathbb{N} \times \mathbb{N}$ is countably infinite.
\end{cor}

\begin{proof}
\pf\ The function that maps $(m,n)$ to $2^m 3^n$ is injective. \qed
\end{proof}

\begin{cor}
The Cartesian product of two countable sets is countable.
\end{cor}

\begin{thm}
A countable union of countable sets is countable.
\end{thm}

\begin{proof}
\pf
\step{1}{\pflet{$A$ be a set.}}
\step{2}{\pflet{$\mathcal{B} \subseteq \mathcal{P} A$ be a countable set of countable sets such that $\bigcup \mathcal{B} = A$}}
\step{3}{\pick\ a surjection $B : \mathbb{N} \twoheadrightarrow \mathcal{B}$}
\step{4}{\assume{w.l.o.g. each $B(n)$ is nonempty.}}
\step{5}{For $n \in \mathbb{N}$, \pick\ a surjective function $g_n : \mathbb{N} \twoheadrightarrow B(n)$}
\step{6}{\pflet{$h : \mathbb{N} \times \mathbb{N} \rightarrow A$ be the function $h(m,n) = g_m(n)$}}
\step{7}{$h$ is surjective.}
\qed
\end{proof}

\begin{thm}
$2^\mathbb{N}$ is uncountable.
\end{thm}

\begin{proof}
\pf
\step{1}{\pflet{$f : \mathbb{N} \rightarrow 2^\mathbb{N}$} \prove{$f$ is not surjective.}}
\step{2}{Define $g : \mathbb{N} \rightarrow 2$ by $g(n) = 1 - f(n)(n)$.}
\step{3}{For all $n \in \mathbb{N}$ we have $g(n) \neq f(n)(n)$.}
\step{4}{For all $n \in \mathbb{N}$ we have $g \neq f(n)$.}
\qed
\end{proof}

\begin{thm}
For any set $A$, there is no surjective function $A \rightarrow \mathcal{P} A$.
\end{thm}

\begin{proof}
\pf
\step{1}{\pflet{$f : A \rightarrow \mathcal{P} A$}}
\step{2}{\pflet{$S = \{ x \in A : x \notin f(x) \}$}}
\step{3}{For all $a \in A$ we have $S \neq f(a)$}
\begin{proof}
	\pf\ We have $a \in S$ if and only if $a \notin f(a)$.
\end{proof}
\qed
\end{proof}

\begin{cor}
For any set $A$, there is no injective function $\mathcal{P} A \rightarrow A$.
\end{cor}

\chapter{Order Theory}
