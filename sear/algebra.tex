\chapter{Monoid Theory}

\begin{df}[Monoid]
A \emph{monoid} is a category with one object.
\end{df}

\begin{df}
Let $\mathcal{C}$ be a category and $X \in \mathcal{C}$. The monoid $\mathrm{End}_\mathcal{C}(X)$ is the set of all morphisms $X \rightarrow X$ under composition.
\end{df}

\begin{prop}
For any functor $F : \mathcal{C} \rightarrow \mathcal{D}$ and $X \in \mathcal{C}$, we have that $F : \mathrm{End}_\mathcal{C}(X) \rightarrow \mathrm{End}_\mathcal{D}(FX)$ is a monoid homomorphism.
\end{prop}

\begin{proof}
\pf\ Since $F \id{X} = \id{FX}$ and $F(g \circ f) = Fg \circ Ff$. \qed
\end{proof}

\chapter{Group Theory}

\section{Category of Small Groups}

\begin{df}
Let $\mathbf{Grp}$ be the category of small groups and group homomorphisms.
\end{df}

\begin{df}
We identify any group $G$ with the category with one object whose morphisms are the elements of $G$ with composition given by the multiplication in $G$.
\end{df}

\begin{prop}
The trivial group is a zero object in $\mathbf{Grp}$.
\end{prop}

\begin{proof}
\pf\ Easy. \qed
\end{proof}

The zero morphism $G \rightarrow H$ maps every element in $G$ to $e$.

\begin{df}
Let $\mathcal{C}$ be a category and $X \in \mathcal{C}$. We write $\mathrm{Aut}_\mathcal{C}(X)$ for the set of all isomorphisms $X \cong X$ under composition.
\end{df}

\begin{prop}
Let $F : \mathcal{C} \rightarrow \mathcal{D}$ be a functor and $X \in \mathcal{C}$. Then $F : \mathrm{Aut}_\mathcal{C}(X) \rightarrow \mathrm{Aut}_\mathcal{D}(FX)$ is a group homomorphism.
\end{prop}

\begin{proof}
\pf\ Since $F \id{X} = \id{FX}$, $F(g \circ f) = Fg \circ Ff$, and $F \inv{f} = \inv{(Ff)}$. \qed
\end{proof}

\begin{prop}
$\mathbf{Grp}$ has products.
\end{prop}

\begin{df}[Free Product]
The product of a family of groups in $\mathbf{Grp}$ is called the \emph{free product}.
\end{df}

\begin{prop}
$\mathbf{Ab}$ has products given by direct sums.
\end{prop}

\begin{df}[Left Coset]
Let $G$ be a group and $H$ a subgroup of $G$. The \emph{left cosets} of $H$ are the sets of the form
\[ xH := \{ xh : h \in H \} \]
We write $G/H$ for the set of left cosets of $H$ in $G$.
\end{df}

\begin{prop}
Let $G$ be a group and $H$ a subgroup of $G$. Then $G/H$ is a partition of $G$.
\end{prop}

\begin{proof}
\pf
\step{1}{$\bigcup (G / H) = G$}
\begin{proof}
	\pf\ Since $x = xe$ and so $x \in xH$.
\end{proof}
\step{2}{Any two distinct left cosets of $H$ are disjoint.}
\begin{proof}
	\pf\ Since if $z \in xH$ and $z \in yH$ then $xH = yH = zH$.
\end{proof}
\qed
\end{proof}

\begin{df}
Let $G$ be a group. Let $A$ and $B$ be subsets of $G$. Then
\[ AB := \{ ab : a \in A, b \in B \} \enspace . \]
\end{df}

\begin{df}
Let $G$ be a group. Let $A$ be a subset of $G$. Then
\[ \inv{A} := \{ \inv{a} : a \in A \} \enspace . \]
\end{df}

\chapter{Ring Theory}

\begin{df}
Let $\mathbf{Ring}$ be the concrete category of rings and ring homomorphisms.
\end{df}

\begin{df}[Spectrum]
Let $R$ be a commutative ring. The \emph{spectrum} of $R$, $\spec R$, is the set of all prime ideals of $R$.
\end{df}

\begin{df}[Zariski Topology]
Let $R$ be a commutative ring. The \emph{Zariski topology} on $\spec R$ is the topology where the closed sets are the sets of the form
\[ VE := \{ p \in \spec R : E \subseteq p \} \]
for any $E \in \mathcal{P} R$.

We prove this is a topology.
\end{df}

\begin{proof}
\pf
\step{1}{\pflet{$\mathcal{C} = \{VE : E \in \mathcal{P} R\}$}}
\step{2}{For all $\mathcal{A} \subseteq \mathcal{C}$ we have $\bigcap \mathcal{A} \in \mathcal{C}$}
\begin{proof}
	\step{a}{\pflet{$\mathcal{A} \subseteq \mathcal{C}$}}
	\step{b}{\pflet{$E = \bigcup \{ E' \in \mathcal{P} R : VE' \in \mathcal{A} \}$} \prove{$VE = \bigcap \mathcal{A}$}}
	\step{c}{For all $p \in \spec R$, if $E \subseteq p$ then $p \in \bigcap \mathcal{A}$}
	\begin{proof}
		\step{i}{\pflet{$p \in \spec R$}}
		\step{ii}{\assume{$E \subseteq p$}}
		\step{iii}{\pflet{$E' \in \mathcal{P} R$ with $VE' \in \mathcal{A}$}}
		\step{iv}{$E' \subseteq E$}
		\step{v}{$E' \subseteq p$}
		\step{vi}{$p \in VE'$}
	\end{proof}
	\step{d}{For all $p \in \spec R$, if $p \in \bigcap \mathcal{A}$ then $E \subseteq p$}
	\begin{proof}
		\step{i}{\pflet{$p \in \bigcap \mathcal{A}$}}
		\step{ii}{For all $E' \in \mathcal{P} R$ with $VE' \in \mathcal{A}$ we have $E' \subseteq p$}
		\step{iii}{$E \subseteq p$}
	\end{proof}
\end{proof}
\step{3}{For all $C,D \in \mathcal{C}$ we have $C \cup D \in \mathcal{C}$.}
\begin{proof}
	\pf\ Since $VE \cup VE' = V(E \cap E')$
\end{proof}
\step{4}{$\emptyset \in \mathcal{C}$}
\begin{proof}
	\step{a}{$VR = \emptyset$}
	\begin{proof}
		\pf\ If $p \in VR$ then $R \subseteq p$ contradicting the fact that $p$ is a prime ideal.
	\end{proof}
\end{proof}
\qed
\end{proof}

\begin{df}
For any ring $R$, let $R-\mathbf{Mod}$ be the category of small $R$-modules and $R$-module homomorphisms.
\end{df}

\begin{prop}
$R-\mathbf{Mod}$ has products and coproducts.
\end{prop}

\chapter{Field Theory}

\begin{prop}
$\mathbf{Field}$ does not have binary products.
\end{prop}

\begin{proof}
\pf\ There cannot be a field $K$ with field homomorphisms $K \rightarrow \mathbb{Z}_2$ and $K \rightarrow \mathbb{Z}_3$, because its characteristic would be both 2 and 3. \qed
\end{proof}

\chapter{Linear Algebra}

\begin{df}[Span]
Let $V$ be a vector space and $A \subseteq V$. The \emph{span} of $A$ is the set of all linear combinations of elements of $A$.
\end{df}

\begin{df}[Independent]
Let $V$ be a vector space and $A \subseteq V$. Then $A$ is \emph{linearly independent} iff, whenever
\[ \alpha_1 v_1 + \cdots + \alpha_n v_n = 0 \]
where $v_1, \ldots, v_n \in A$, then
\[ \alpha_1 = \cdots = \alpha_n = 0 \enspace . \]
\end{df}

\begin{prop}
\label{prop:extend_linearly_independent}
Let $V$ be a vector space, $A \subseteq V$ and $v \in V$. If $A$ is linearly independent and $v \notin \spn A$, then $A \cup \{ v \}$ is independent.
\end{prop}

\begin{proof}
\pf
\step{1}{\pflet{$\alpha_1 v_1 + \cdots + \alpha_n v_n + \beta v = 0$ where $v_1, \ldots, v_n \in A$}}
\step{2}{$\beta = 0$}
\begin{proof}
	\pf\ Otherwise $v = (\alpha_1 / \beta) v_1 + \cdots + (\alpha_n / \beta) v_n \in \spn A$.
\end{proof}
\step{3}{$\alpha_1 = \cdots = \alpha_n = 0$}
\begin{proof}
	\pf\ Since $A$ is linearly independent.
\end{proof}
\qed
\end{proof}

\begin{thm}
Every vector space has a basis.
\end{thm}

\begin{proof}
\pf
\step{1}{\pflet{$V$ be a vector space.}}
\step{2}{\pick\ a maximal linearly independent set $\mathcal{B}$.}
\begin{proof}
	\pf\ By Tukey's Lemma.
\end{proof}
\step{3}{$\spn \mathcal{B} = V$}
\begin{proof}
	\pf\ Proposition \ref{prop:extend_linearly_independent}.
\end{proof}
\qed
\end{proof}

\begin{df}
For any field $K$, we write $\mathbf{Vect}_K$ for $K-\mathbf{Mod}$.
\end{df}

Dual space functor $\mathbf{Vect}_K^{\mathrm{op}} \rightarrow \mathbf{Vect}_K$.
