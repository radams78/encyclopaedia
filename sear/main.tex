\documentclass{book}

\title{Mathematics}
\author{Robin Adams}

\usepackage{amsmath}
\usepackage{amssymb}
\usepackage{amsthm}
\let\proof\relax
\let\endproof\relax
\let\qed\relax
\usepackage{pf2}
\usepackage{hyperref}
\usepackage{mathabx}
\usepackage[all]{xy}

\newtheorem{ax}{Axiom}[chapter]
\newtheorem{axs}[ax]{Axiom Schema}
\newtheorem{prop}[ax]{Proposition}
\newtheorem{cor}{Corollary}[ax]
\newtheorem{thm}[ax]{Theorem}
\newtheorem{thms}[ax]{Theorem Schema}
\newtheorem{lm}[ax]{Lemma}
\theoremstyle{definition}
\newtheorem{df}[ax]{Definition}
\newtheorem{ex}[ax]{Example}

\newcommand{\El}[1]{\ensuremath{\mathrm{El} \left( {#1} \right)}}
\newcommand{\id}[1]{\ensuremath{\mathrm{id}_{#1}}}
\newcommand{\inv}[1]{\ensuremath{{#1}^{-1}}}

\begin{document}

\maketitle
\tableofcontents

\chapter{Primitive Terms and Axioms}

\section{Primitive Terms} %CHECKED FORMALIZED

Let there be \emph{sets}. We write $A : \mathrm{Set}$ for: $A$ is a set.

For any set $A$, let there be \emph{elements} of $A$. We write $a: \El{A}$ for: $a$ is an element of $A$.

For any sets $A$ and $B$, let there be \emph{relations} between $A$ and $B$. We write $R : A \looparrowright B$ for: $R$ is a relation between $A$ and $B$.

For any set $A$ and elements $a,b : \El{A}$, let there be a proposition that $a$ and $b$ are \emph{equal}, $a = b$.

For any relation $R : A \looparrowright B$ and elements $a : \El{A}$, $b : \El{B}$, let there be a proposition $aRb$, that $R$ \emph{holds} between $a$ and $b$.

\section{Axioms} %CHECKED

\begin{df}[Function]
Let $A$ and $B$ be sets and $F : A \looparrowright B$. Then $F$ is a \emph{function} from $A$ to $B$, $F : A \rightarrow B$, if and only if, for all $x \in A$, there exists a unique $y \in B$ such that $xFy$.
We denote this unique $y$ by $F(x)$.
\end{df}

\begin{axs}[Comprehension]
For any formula $\phi[X,Y,x,y]$ where $X$ and $Y$ are set variables and $x : \El{X}$ and $y : \El{Y}$, the following is an axiom:

For any sets $A$ and $B$, there exists a relation $R : A \looparrowright B$ such that, for all $a : \El{A}$ and $b : \El{B}$, we have $aRb$ if and only if $\phi[A,B,a,b]$.
\end{axs}

\begin{ax}[Tabulations]
For any sets $A$ and $B$ and relation $R : A \looparrowright B$, there exists a set $|R|$, a \emph{tabulation} of $R$, and functions $p : |R| \rightarrow A$ and $q : |R| \rightarrow B$ such that:
\begin{itemize}
\item For all $x: \El{A}$ and $y : \El{B}$, we have $xRy$ if and only if there exists $r : \El{|R|}$ such that $p(r) = x$ and $q(r) = y$
\item For all $r,s : \El{|R|}$, if $p(r) = p(s)$ and $q(r) = q(s)$ then $r = s$.
\end{itemize}
\end{ax}

\begin{ax}[Infinity]
There exists a set $\mathbb{N}$, an element $0 : \El{\mathbb{N}}$, and a function $s : \mathbb{N} \rightarrow \mathbb{N}$ such that:
\begin{itemize}
\item $\forall n : \El{\mathbb{N}}. s(n) \neq 0$
\item $\forall m,n : \El{\mathbb{N}}. s(m) = s(n) \Rightarrow m = n$.
\end{itemize}
\end{ax}

\begin{ax}[Choice]
Let $R : A \looparrowright B$ be a relation such that $\forall a : \El{A}. \exists b : \El{B}. aRb$. Then there exists a function $f : A \rightarrow B$ such that $\forall a : \El{A}. a R f(a)$.
\end{ax}

\section{Consequences of the Axioms}

\subsection{Definitions Used in the Axioms} %CHECKED

\begin{df}[Injective]
A function $f : A \rightarrow B$ is \emph{injective} iff, for all $x,y : \El{A}$, if $f(x) = f(y)$ then $x = y$.
\end{df}

\begin{df}[Surjective]
A function $f : A \rightarrow B$ is \emph{surjective} iff, for all $y : \El{B}$, there exists $x : \El{A}$ such that $f(x) = y$.
\end{df}

\begin{df}[Bijective]
A function $f : A \rightarrow B$ is \emph{bijective} or a \emph{bijection} iff it is injective and surjective.

Sets $A$ and $B$ are \emph{equinumerous}, $A \approx B$, iff there exists a bijection between them.
\end{df}

If we prove there exists a set $X$ such that $P(X)$, and that any two sets that satisfy $P$ are bijective, then we may introduce a constant $C$ and define "Let $C$ be the set such that $P(C)$".

\subsection{Tabulations}

\begin{thm}
Let $R : A \looparrowright B$. Let $p : T \rightarrow A$ and $q : T \rightarrow B$ form a tabulation of $R$. Let $p' : T' \rightarrow A$ and $q' : T' \rightarrow B$ form a tabulation of $R$. Then there exists a unique bijection $f : T \approx T'$ such that $\forall t : \El{T}. p'(f(t)) = p(t)$ and $\forall t : \El{T}. q'(f(t)) = q(t)$.
\end{thm}

\begin{proof}
\pf
\step{1}{\pflet{$f : T \looparrowright T'$ be the relation such that $tft'$ iff $p(t) = p'(t')$ and $q(t) = q'(t')$}}
\begin{proof}
	\pf\ Axiom of Comprehension
\end{proof}
\step{2}{$f$ is a function.}
\begin{proof}
	\step{a}{\pflet{$x : \El{T}$}}
	\step{b}{$p(x)Rq(x)$}
	\begin{proof}
		\pf\ Since $T$ is a tabulation of $R$.
	\end{proof}
	\step{b}{There exists a unique $y : \El{T'}$ such that $p'(y) = p(x)$ and $q'(y) = q(x)$.}
	\begin{proof}
		\pf\ Since $T'$ is a tabulation of $R$.
	\end{proof}
\end{proof}
\step{3}{$f$ is injective.}
\begin{proof}
	\step{a}{\pflet{$x,y : \El{T}$}}
	\step{b}{\assume{$f(x) = f(y)$}}
	\step{c}{$p'(f(x)) = p'(f(y))$ and $q'(f(x)) = q'(f(y))$}
	\step{d}{$p(x) = p(y)$ and $q(x) = q(y)$}
	\step{e}{$x = y$}
	\begin{proof}
		\pf\ Since $T$ is a tabulation of $R$.
	\end{proof}
\end{proof}
\step{4}{$f$ is surjective.}
\begin{proof}
	\step{a}{\pflet{$y : \El{T'}$}}
	\step{b}{$p'(y)Rq'(y)$}
	\begin{proof}
		\pf\ Since $T'$ is a tabulation of $R$.
	\end{proof}
	\step{c}{There exists $x : \El{T}$ such that $p(x) = p'(y)$ and $q(x) = q'(y)$.}
	\begin{proof}
		\pf\ Since $T$ is a tabulation of $R$.
	\end{proof}
\end{proof}
\step{5}{If $g : T \approx T'$ satisfies $\forall t : \El{T}. p'(g(t)) = p(t)$ and $\forall t : \El{T}. q'(g(t)) = q(t)$.}
\begin{proof}
	\step{a}{\pflet{$g : T \approx T'$ satisfy $\forall t : \El{T}. p'(g(t)) = p(t)$ and $\forall t : \El{T}. q'(g(t)) = q(t)$.}}
	\step{b}{For all $t : \El{T}$ we have $p'(f(t)) = p'(g(t))$ and $q'(f(t)) = q'(g(t))$.}
	\step{c}{For all $t : \El{T}$ we have $f(t) = g(t)$.}
\end{proof}
\qed
\end{proof}

\subsection{The Empty Set}

\begin{thm}
There exists a set which has no elements.
\end{thm}

\begin{proof}
\pf
\step{1}{\pick\ a set $A$}
\begin{proof}
	\pf\ By the Axiom of Infinity, a set exists.
\end{proof}
\step{2}{\pflet{$R : A \looparrowright A$ be the relation such that, for all $x,y \in A$, we have $\neg (xRy)$}}
\begin{proof}
	\pf\ By the Axiom of Comprehension.
\end{proof}
\step{3}{\pflet{$|R|$ be the tabulation of $R$ with projections $p,q : |R| \rightarrow A$.} \prove{$|R|$ has no elements.}}
\begin{proof}
	\pf\ By the Axiom of Tabulations.
\end{proof}
\step{4}{\assume{for a contradiction $r : \El{|R|}$}}
\step{5}{$p(r) R q(r)$}
\qedstep
\begin{proof}
	\pf\ This contradicts \stepref{2}.
\end{proof}
\qed
\end{proof}

\begin{thm}
If $E$ and $E'$ have no elements then $E \approx E'$.
\end{thm}

\begin{proof}
\pf
\step{1}{\pflet{$E$ and $E'$ have no elements.}}
\step{2}{\pflet{$F : E \looparrowright E'$ be the relation such that, for all $x : \El{E}$ and $y : \El{E'}$, we have $xFy$.}}
\begin{proof}
	\pf\ Axiom of Comprehension.
\end{proof}
\step{3}{$F$ is a function.}
\begin{proof}
	\pf\ Vacuously, for all $x : \El{E}$, there exists a unique $y : \El{E'}$ such that $xFy$.
\end{proof}
\step{4}{$F$ is injective.}
\begin{proof}
	\pf\ Vacuously, for all $x,y : \El{E}$, if $F(x) = F(y)$ then $x = y$.
\end{proof}
\step{5}{$F$ is surjective.}
\begin{proof}
	\pf\ Vacuously, for all $y : \El{E}$, there exists $x : \El{E}$ such that $F(x) = y$.
\end{proof}
\qed
\end{proof}

\begin{df}[Empty Set]
The \emph{empty set} $\emptyset$ is the set with no elements.
\end{df}

\subsection{The Singleton}

\begin{thm}
There exists a set that has exactly one element.
\end{thm}

\begin{proof}
\pf
\step{1}{\pick\ a set $A$ that has an element.}
\begin{proof}
	\pf\ By the Axiom of Infinity, there exists a set that has an element.
\end{proof}
\step{2}{\pick\ $a : \El{A}$}
\step{3}{\pflet{$R : A \looparrowright A$ be the relation such that, for all $x,y : \El{A}$, we have $xRy$ if and only if $x = y = a$.}}
\begin{proof}
	\pf\ By the Axiom of Comprehension.
\end{proof}
\step{4}{\pflet{$|R|$ be the tabulation of $R$ with projections $p,q : |R| \rightarrow A$.} \prove{$|R|$ has exactly one element.}}
\begin{proof}
	\pf\ By the Axiom of Tabulations.
\end{proof}
\step{5}{\pflet{$r : \El{|R|}$ be the element such that $p(r) = q(r) = a$}}
\begin{proof}
	\pf\ Since $aRa$ by \stepref{3}.
\end{proof}
\step{6}{\pflet{$s : \El{|R|}$} \prove{$s = r$}}
\step{7}{$p(s) R q(s)$}
\begin{proof}
	\pf\ By the Axiom of Tabulations.
\end{proof}
\step{8}{$p(s) = q(s) = a$}
\begin{proof}
	\pf\ By \stepref{3}.
\end{proof}
\step{9}{$p(s) = p(r)$ and $q(s) = q(r)$}
\begin{proof}
	\pf\ By \stepref{5}.
\end{proof}
\step{10}{$s = r$}
\begin{proof}
	\pf\ By the Axiom of Tabulations.
\end{proof}
\qed
\end{proof}

\begin{thm}
If $A$ and $B$ both have exactly one element then $A \approx B$.
\end{thm}

\begin{proof}
\pf
\step{1}{\pflet{$A$ and $B$ both have exactly one element.}}
\step{2}{\pflet{$F : A \looparrowright B$ be the relation such that, for all $x : \El{A}$ and $y : \El{B}$, we have $xFy$.}}
\step{3}{$F$ is a function.}
\begin{proof}
	\pf\ If $xFy$ and $xFy'$ then $y = y'$ because $B$ has only one element.
\end{proof}
\step{4}{$F$ is injective.}
\begin{proof}
	\pf\ If $F(x) = F(x')$ then $x = x'$ because $A$ has only one element.
\end{proof}
\step{5}{$F$ is surjective.}
\begin{proof}
	\step{a}{\pflet{$y : \El{B}$}}
	\step{b}{\pflet{$x$ be the element of $A$.}}
	\step{c}{$F(x) = y$}
\end{proof}
\qed
\end{proof}

\begin{df}[Singleton]
Let 1 be the set that has exactly one element. Let $*$ be its element.
\end{df}

\subsection{Subsets}

\begin{df}[Subset]
A \emph{subset} of a set $A$ is a relation $1 \looparrowright S$.

Given $S : 1 \looparrowright S$ and $a : \El{A}$, we write $a \in S$ for $*Sa$.
\end{df}

\begin{thms}
For any property $P[X,x]$ where $X$ is a set variable and $x : \El{X}$, the following is a theorem:

For any set $A$, there exists a set $B$ and injection $i : B \rightarrow A$ such that, for all $x : \El{A}$, we have $P[A,x]$ if and only if there exists $b : \El{B}$ such that $i(b) = x$.
\end{thms}

\begin{proof}
\pf
\step{1}{\pflet{$S : 1 \looparrowright A$ be the relation such that, for all $e : \El{1}$ and $a : \El{A}$, we have $eSa$ if and only if $P[A,a]$.}}
\begin{proof}
	\pf\ Axiom of Comprehension.
\end{proof}
\step{2}{\pflet{$B$ be the tabulation of $S$ with projections $p : B \rightarrow 1$ and $i : B \rightarrow A$.}}
\begin{proof}
	\pf\ Axiom of Tabulations.
\end{proof}
\step{3}{$i$ is injective.}
\begin{proof}
	\step{a}{\pflet{$r,s : \El{B}$}}
	\step{b}{\assume{$i(r) = i(s)$}}
	\step{c}{$p(r) = p(s)$}
	\begin{proof}
		\pf\ Since 1 has only one element.
	\end{proof}
	\step{d}{$r = s$}
	\begin{proof}
		\pf\ Axiom of Tabulations.
	\end{proof}
\end{proof}
\step{4}{For all $x : \El{A}$, we have $P[A,x]$ if and only if there exists $b : \El{B}$ such that $i(b) = x$.}
\begin{proof}
	\step{a}{\pflet{$x : \El{A}$}}
	\step{b}{If $P[A,x]$ then there exists $b : \El{B}$ such that $i(b) = x$}
	\begin{proof}
		\step{i}{\assume{$P[A,x]$}}
		\step{ii}{$*Sx$}
		\begin{proof}
			\pf\ \stepref{1}
		\end{proof}
		\step{iii}{There exists $b : \El{B}$ such that $p(b) = *$ and $i(b) = x$}
		\begin{proof}
			\pf\ Axiom of Tabulations.
		\end{proof}
	\end{proof}
	\step{c}{For all $b : \El{B}$ we have $P[A,i(b)]$}
	\begin{proof}
		\step{i}{\pflet{$b : \El{B}$}}
		\step{ii}{$p(b)Si(b)$}
		\begin{proof}
			\pf\ Axiom of Tabulations.
		\end{proof}
		\step{iii}{$P[A,i(b)]$}
		\begin{proof}
			\pf\ \stepref{1}
		\end{proof}
	\end{proof}
\end{proof}
\qed
\end{proof}

\section{Composition}

\begin{df}[Composite]
Let $\phi : A \looparrowright B$ and $\psi : B \looparrowright C$. The \emph{composite} $\psi \circ \phi : A \looparrowright C$ is the relation such that $a (\psi \circ \phi) c$ iff there exists $b$ such that $a \phi b$ and $b \psi c$.
\end{df}

\begin{df}[Identity]
For any set $A$, the \emph{identity} function $\id{A} : A \rightarrow A$ is the function defined by $\id{A}(a) = a$.
\end{df}

\begin{thm}
Composition of relations is associative, and the identity function is an identity for composition. The composite of functions is a function. The composite of injective functions is injective. The composite of surjective functions is surjective. The composite of bijections is a bijection. A function $f : A \rightarrow B$ is a bijection iff there exists a function $\inv{f} : B \rightarrow A$ such that $\inv{f} f = \id{A}$ and $f \inv{f} = \id{B}$, in which case $\inv{f}$ is unique.
\end{thm}

\section{Axioms Part Two}

\begin{ax}[Power Set]
For any set $A$, there exists a set $\mathcal{P} A$, the \emph{power set} of $A$, and a relation $\in : A \looparrowright \mathcal{P} A$, called \emph{membership}, such that, for any subset $S$ of $A$, there exists a unique $\overline{S} \in \mathcal{P} A$ such that, for all $x \in A$, we have $x \in \overline{S}$ if and only if $x \in S$.

We usually write just $S$ for $\overline{S}$.
\end{ax}

\begin{axs}[Collection]
Let $P[X,Y,x]$ be a formula with set variables $X$ and $Y$ and an element variable $x \in X$. Then the following is an axiom.

For any set $A$, there exists a set $B$, a function $p : B \rightarrow A$, a set $Y$ and a relation $M : B \looparrowright Y$ such that:
\begin{itemize}
\item $\forall b \in B. P[A, \{ y \in Y : bMy \}, p(b)]$
\item For all $a \in A$, if $\exists Y. P[A,Y,a]$, then there exists $b \in B$ such that $a = p(b)$.
\end{itemize}
\end{axs}

\begin{df}[Universe]
Let $E : U \looparrowright X$ be a relation. Let us say that a set $A$ is \emph{small} iff there exists $u \in U$ such that $A \approx \{ x \in X : uEx \}$.

Then $(U,X,E)$ form a \emph{universe} if and only if:
\begin{itemize}
\item $\mathbb{N}$ is $U$-small.
\item For any $U$-small sets $A$ and $B$ and relation $R : A \looparrowright B$, the tabulation of $R$ is $U$-small.
\item If $A$ is $U$-small then so is $\mathcal{P} A$
\item Let $f : A \rightarrow B$ be a function. If $B$ is $U$-small and $f^{-1}(b)$ is $U$-small for all $b \in B$, then $A$ is $U$-small.
\item If $p : B \twoheadrightarrow A$ is a surjective function such that $A$ is $U$-small, then there exists a $U$-small set $C$, a surjection $q : C \twoheadrightarrow A$, and a function $f : C \rightarrow B$ such that $q = pf$.
\end{itemize}
\end{df}

\begin{ax}[Universe]
There exists a universe.
\end{ax}

Let $E : U \looparrowright X$ be a universe. We shall say a set is \emph{small} iff it is $U$-small, and \emph{large} otherwise.

\section{Cartesian Product}

\begin{df}[Cartesian Product]
Let $A$ and $B$ be sets. The \emph{Cartesian product} of $A$ and $B$, $A \times B$, is the tabulation of the relation $A \looparrowright B$ that holds for all $a \in A$ and $b \in B$. The associated functions $\pi_1 : A \times B \rightarrow A$ and $\pi_2 : A \times B \rightarrow B$ are called the \emph{projections}.

Given $a \in A$ and $b \in B$, we write $(a,b)$ for the unique element of $A \times B$ such that $\pi_1(a,b) = a$ and $\pi_2(a,b) = b$.
\end{df}

\chapter{Topology}

\section{Topological Spaces}

\begin{df}[Topological Space]
Let $X$ be a set and $\mathcal{O} \subseteq \mathcal{P} X$. Then we say $(X, \mathcal{O})$ is a \emph{topological space} iff:
\begin{itemize}
\item For any $\mathcal{U} \subseteq \mathcal{O}$ we have $\bigcup \mathcal{U} \in \mathcal{O}$.
\item For any $U, V \in \mathcal{O}$ we have $U \cap V \in \mathcal{O}$.
\item $X \in \mathcal{O}$
\end{itemize}
We call $\mathcal{O}$ the \emph{topology} of the toplogical space, and call its elements \emph{open} sets. We shall often write $X$ for the topological space $(X, \mathcal{O})$.
\end{df}

\begin{df}[Closed Set]
Let $X$ be a topological space and $A \subseteq X$. Then $A$ is \emph{closed} iff $X - A$ is open.
\end{df}

\begin{prop}
A set $B$ is open if and only if $X - B$ is closed.
\end{prop}

\begin{prop}
Let $X$ be a set and $\mathcal{C} \subseteq \mathcal{P} X$. Then there exists a topology $\mathcal{O}$ on $X$ such that $\mathcal{C}$ is the set of closed sets if and only if:
\begin{itemize}
\item For any $\mathcal{D} \subseteq \mathcal{C}$ we have $\bigcap \mathcal{D} \in \mathcal{C}$
\item For any $C, D \in \mathcal{C}$ we have $C \cup D \in \mathcal{C}$.
\item $\emptyset \in \mathcal{C}$
\end{itemize}
In this case, $\mathcal{O}$ is unique and is given by $\mathcal{O} = \{ X - C : C \in \mathcal{C} \}$.
\end{prop}

\begin{df}[Neighbourhood]
Let $X$ be a topological space, $Sx \in X$ and $U \subseteq X$. Then $U$ is a \emph{neighbourhood} of $x$, and $x$ is an \emph{interior} point of $U$, iff there exists an open set $V$ such that $x \in V \subseteq U$.
\end{df}

\begin{prop}
A set $B$ is open if and only if it is a neighbourhood of each of its points.
\end{prop}

\begin{prop}
Let $X$ be a set and $\mathcal{N} : X \rightarrow \mathcal{P} X$. Then there exists a topology $\mathcal{O}$ on $X$ such that, for all $x \in X$, we have $\mathcal{N}_x$ is the set of neighbourhoods of $x$, if and only if:
\begin{itemize}
\item For all $x \in X$ and $N \in \mathcal{N}_x$ we have $x \in N$
\item For all $x \in X$ we have $X \in \mathcal{N}_x$
\item For all $x \in X$, $N \in \mathcal{N}_x$ and $V \subseteq \mathcal{P} X$, if $N \subseteq V$ then $V \in \mathcal{N}_x$
\item For all $x \in X$ and $M, N \in \mathcal{N}_x$ we have $M \cap N \in \mathcal{N}_x$
\item For all $x \in X$ and $N \in \mathcal{N}_x$, there exists $M \in \mathcal{N}_x$ such that $M \subseteq N$ and $\forall y \in M. M \in \mathcal{N}_y$.
\end{itemize}
In this case, $\mathcal{O}$ is unique and is given by $\mathcal{O} = \{ U : \forall x \in U. U \in \mathcal{N}_x \}$.
\end{prop}

\begin{df}[Exterior Point]
Let $X$ be a topological space, $x \in X$ and $B \subseteq X$. Then $x$ is an \emph{exterior point} of $B$ iff $B - X$ is a neighbourhood of $x$.
\end{df}

\begin{df}[Boundary Point]
Let $X$ be a topological space, $x \in X$ and $B \subseteq X$. Then $x$ is a \emph{boundary point} of $B$ iff it is neither an interior point nor an exterior point of $B$.
\end{df}

\begin{df}[Interior]
Let $X$ be a topological space and $B \subseteq X$. The \emph{interior} of $B$, $B^\circ$, is the set of all interior points of $B$.
\end{df}

\begin{prop}
The interior of $B$ is the union of all the open sets included in $B$.
\end{prop}

\begin{df}[Closure]
Let $X$ be a topological space and $B \subseteq X$. The \emph{closure} of $B$, $\overline{B}$, is the set of all points that are not exterior points of $B$.
\end{df}

\begin{prop}
The closure of $B$ is the intersection of all the closed sets that include $B$.
\end{prop}

\begin{prop}
A set $B$ is open iff $X - B = \overline{X - B}$.
\end{prop}

\begin{prop}[Kuratowski Closure Axioms]
Let $X$ be a set and $\overline{\ } : \mathcal{P} X \rightarrow \mathcal{P} X$. Then there exists a topology $\mathcal{O}$ such that, for all $B \subseteq X$, $\overline{B}$ is the closure of $B$, if and only if:
\begin{itemize}
\item $\overline{\emptyset} = \emptyset$
\item For all $A \subseteq X$ we have $A \subseteq \overline{A}$
\item For all $A \subseteq X$ we have $\overline{\overline{A}} = \overline{A}$
\item For all $A, B \subseteq X$ we have $\overline{A \cup B} = \overline{A} \cup \overline{B}$
\end{itemize}
In this case, $\mathcal{O}$ is unique and is defined by $\mathcal{O} = \{ U : X - U = \overline{X - U} \}$.
\end{prop}

\subsection{Subspaces}

\begin{df}[Subspace]
Let $X$ be a topological space and $X_0 \subseteq X$. The \emph{subspace topology} on $X_0$ is $\{ U \cap X_0 : U \text{ is open in } X \}$.
\end{df}

\subsection{Topological Disjoint Union}

\begin{df}
Let $X$ and $Y$ be topological spaces. The \emph{disjoint union} is $X + Y$ where $U \subseteq X + Y$ is open if and only if $\inv{\kappa_1}(U)$ is open in $X$ and $\inv{\kappa_2}(U)$ is open in $Y$.
\end{df}

\subsection{Product Topology}

\begin{df}
Let $X$ and $Y$ be topological spaces. The \emph{product topology} on $X \times Y$ is the set of all subsets $W \subseteq X \times Y$ such that, for all $(x,y) \in W$, there exist neighbourhoods $U$ of $x$ in $X$ and $V$ of $y$ in $Y$ such that $U \times V \subseteq W$.
\end{df}

\subsection{Bases}

\begin{df}[Basis]
Let $X$ be a topological space. A \emph{basis} for the topology on $X$ is a set of open sets $\mathcal{B}$ such that every open set is the union of a subset of $\mathcal{B}$.
\end{df}

\subsection{Subbases}

\begin{df}[Subbasis]
Let $X$ be a topological space. A \emph{subbasis} for the topology on $X$ is a subset $\mathcal{S} \subseteq \mathcal{P} X$ such that every open set is a union of finite intersections of $\mathcal{S}$.
\end{df}

\section{Continuous Functions}

\begin{df}[Continuous]
Let $X$ and $Y$ be topological spaces. A function $f : X \rightarrow Y$ is \emph{continuous} iff, for every open set $V$ in $Y$, the inverse image $\inv{f}(V)$ is open in $X$.
\end{df}

\begin{prop}
\begin{enumerate}
\item $\id{X}$ is continuous
\item The composite of two continuous functions is continuous.
\item If $f : X \rightarrow Y$ is continuous and $X_0 \subseteq X$ then $f \restriction X_0 : X_0 \rightarrow Y$ is continuous.
\item If $f : X + Y \rightarrow Z$, then $f$ is continuous iff $f \circ \kappa_1 : X \rightarrow Z$ and $f \circ \kappa_2 : Y \rightarrow Z$ are continuous.
\item If $f : Z \rightarrow X \times Y$, then $f$ is continuous iff $\pi_1 \circ f$ and $\pi_2 \circ f$ are continuous.
\end{enumerate}
\end{prop}

\begin{df}[Homeomorphism]
Let $X$ and $Y$ be topological spaces. A \emph{homeomorphism} between $X$ and $Y$ is a bijection $f : X \approx Y$ such that $f$ and $\inv{f}$ are continuous.
\end{df}

\section{Convergence}

\begin{df}[Convergence]
Let $X$ be a topological space. Let $(x_n)$ be a sequence in $X$. A point $a : \El{X}$ is a \emph{limit} of the sequence iff, for every neighbourhood $U$ of $a$, there exists $n_0$ such that $\forall n \geq n_0. x_n \in U$.
\end{df}

\section{Connected Spaces}

\begin{df}[Connected]
A topological space is \emph{connected} iff it is not the union of two nonempty open disjoint subsets.
\end{df}

\begin{prop}
The continuous image of a connected space is connected.
\end{prop}

\begin{prop}
Let $X$ be a topological space and $A,B \subseteq X$. If $X = A \cup B$, $A \cap B \neq \emptyset$, and $A$ and $B$ are connected, then $X$ is connected.
\end{prop}

\begin{prop}
If $X$ and $Y$ are nonempty topological spaces, then
$X \times Y$ is connected if and only if $X$ and $Y$ are connected.
\end{prop}

\begin{df}[Path-connected]
A topological space $X$ is \emph{path-connected} iff, for any points $a,b \in X$, there exists a continuous function $\alpha : [0,1] \rightarrow X$, called a \emph{path}, such that $\alpha(0) = a$ and $\alpha(1) = b$.
\end{df}

\begin{prop}
The continuous image of a path connected space is path connected.
\end{prop}

\begin{prop}
Let $X$ be a topological space and $A,B \subseteq X$. If $X = A \cup B$, $A \cap B \neq \emptyset$, and $A$ and $B$ are path connected, then $X$ is path connected.
\end{prop}

\begin{prop}
If $X$ and $Y$ are nonempty topological spaces, then
$X \times Y$ is path connected if and only if $X$ and $Y$ are path connected.
\end{prop}

\section{Hausdorff Spaces}

\begin{df}[Hausdorff]
A topological space is a \emph{Hausdorff} space or a \emph{$T_2$} space iff any two distinct points have disjoint neighbourhoods.
\end{df}

\begin{prop}
In a Hausdorff space, a sequence has at most one limit.
\end{prop}

\begin{prop}
\begin{enumerate}
\item Every subspace of a Hausdorff space is Hausdorff.
\item The disjoint union of two Hausdorff spaces is Hausdorff.
\item The product of two Hausdorff spaces is Hausdorff.
\end{enumerate}
\end{prop}

\section{Compactness}

\begin{df}[Compact]
A topological space is \emph{compact} iff every open cover has a finite subcover.
\end{df}

\begin{prop}
Let $X$ be a compact topological space. Let $P$ be a set of open sets such that, for all $U,V \in P$, we have $U \cup V \in P$. Assume that every point has an open neighbourhood in $P$. Then $X \in P$.
\end{prop}

\begin{proof}
\pf
\step{1}{$P$ is an open cover of $X$}
\step{2}{\pick\ a finite subcover $U_1, \ldots, U_n \in P$}
\step{3}{$X = U_1 \cup \cdots \cup U_n \in P$}
\qed
\end{proof}

\begin{cor}
Let $f$ be a compact space and $f : X \rightarrow \mathbb{R}$ be locally bounded. Then $f$ is bounded.
\end{cor}

\begin{proof}
\pf\ Take $P = \{ U \text{ open in } X : f \text{ is bounded on } U \}$. \qed
\end{proof}

\begin{prop}
The continuous image of a compact space is compact.
\end{prop}

\begin{prop}
A closed subspace of a compact space is compact.
\end{prop}

\begin{prop}
Let $X$ and $Y$ be nonempty spaces. Then the following are equivalent.
\begin{enumerate}
\item $X$ and $Y$ are compact.
\item $X + Y$ is compact.
\item $X \times Y$ is compact.
\end{enumerate}
\end{prop}

\begin{prop}
A compact subspace of a Hausdorff space is closed.
\end{prop}

\begin{prop}
A continuous bijection from a compact space to a Hausdorff space is a homeomorphism.
\end{prop}

\section{Metric Spaces}

%TODO Define real numbers
\begin{df}[Metric Space]
Let $X$ be a set and $d : X^2 \rightarrow \mathbb{R}$. We say $(X,d)$ is a \emph{metric space} iff:
\begin{itemize}
\item For all $x,y \in X$ we have $d(x,y) \geq 0$
\item For all $x,y \in X$ we have $d(x,y) = 0$ iff $x = y$
\item For all $x,y \in X$ we have $d(x,y) = d(y,x)$
\item (\emph{Triangle Inequality}) For all $x,y,z \in X$ we have $d(x,z) \leq d(x,y) + d(y,z)$
\end{itemize}
We call $d$ the \emph{metric} of the metric space $(X,d)$. We often write $X$ for the metric space $(X,d)$.
\end{df}

\begin{df}[Topology of a Metric Space]
Let $(X,d)$ be a metric space. The topology \emph{induced} by the metric $d$ is defined by: for $V \subseteq X$, we have $V$ is open if and only if, for all $x \in V$, there exists $\epsilon > 0$ such that $\{ y \in X : d(x,y) < \epsilon \} \subseteq V$.
\end{df}

\begin{df}[Metrizable]
A topological space is \emph{metrizable} iff there exists a metric that induces its topology.
\end{df}

\begin{prop}
Every metrizable space is Hausdorff.
\end{prop}

\end{document}