% READ
% CHECK
% FORMALIZE
% READ
% READ
% CHECK
% READ

\documentclass{book}

\title{Mathematics}
\author{Robin Adams}

\usepackage{amsmath}
\usepackage{amssymb}
\usepackage{amsthm}
\let\proof\relax
\let\endproof\relax
\let\qed\relax
\usepackage{pf2}
\usepackage{hyperref}
\usepackage{mathabx}
\usepackage[all]{xy}

\newtheorem{ax}{Axiom}[chapter]
\newtheorem{axs}[ax]{Axiom Schema}
\newtheorem{prop}[ax]{Proposition}
\newtheorem{cor}{Corollary}[ax]
\newtheorem{thm}[ax]{Theorem}
\newtheorem{thms}[ax]{Theorem Schema}
\newtheorem{lm}[ax]{Lemma}
\theoremstyle{definition}
\newtheorem{df}[ax]{Definition}
\newtheorem{ex}[ax]{Example}

\newcommand{\El}[1]{\ensuremath{\mathrm{El} \left( {#1} \right)}}
\newcommand{\id}[1]{\ensuremath{\mathrm{id}_{#1}}}
\newcommand{\inv}[1]{\ensuremath{{#1}^{-1}}}

\begin{document}

\maketitle
\tableofcontents

\chapter{Primitive Terms and Axioms}

\section{Primitive Terms} % CHECKED

Let there be \emph{sets}. We write $A : \mathrm{Set}$ for: $A$ is a set.

For any set $A$, let there be \emph{elements} of $A$. We write $a: \El{A}$ for: $a$ is an element of $A$.

For any sets $A$ and $B$, let there be \emph{functions} from $A$ to $B$. We write $f : A \rightarrow B$ iff $f$ is a function from $A$ to $B$.

For any function $f : A \rightarrow B$ and element $a : \El{A}$, let there be an element $f(a) : \El{B}$, the \emph{value} of the function $f$ at the \emph{argument} $a$.

\section{Axioms} %CHECKED FORMALIZED

\begin{axs}[Choice]
Let $P[X,Y,x,y]$ be a formula where $X$ and $Y$ are set variables, $x : \El{X}$ and $y : \El{Y}$. Then the following is an axiom.

Let $A$ and $B$ be sets. Assume that, for all $a : \El{A}$, there exists $b : \El{B}$ such that $P[A,B,a,b]$. Then there exists a function $f : A \rightarrow B$ such that $\forall a : \El{A}. P[A,B,a,f(a)]$.
\end{axs}

\begin{ax}[Pairing]
For any sets $A$ and $B$, there exists a set $A \times B$, the \emph{Cartesian product} of $A$ and $B$, and functions $\pi_1 : A \times B \rightarrow A$ and $\pi_2 : A \times B \rightarrow B$ such that, for all $a : \El{A}$ and $b : \El{B}$, there exists a unique $(a,b) : \El{A \times B}$ such that $\pi_1(a,b) = a$ and $\pi_2(a,b) = b$.
\end{ax}

\begin{df}[Injective]
A function $f : A \rightarrow B$ is \emph{injective} or an \emph{injection} iff, for all $x,y : \El{A}$, if $f(x) = f(y)$ then $x = y$.
\end{df}

\begin{axs}[Separation]
For every property $P[X,x]$ where $X$ is a set variable and $x : \El{X}$, the following is an axiom:

For every set $A$, there exists a set $S = \{ x : \El{A} \mid P[A,x] \}$ and an injection $i : S \rightarrow A$ such that, for all $x : \El{A}$, we have
\[ (\exists y : S. i(y) = x) \Leftrightarrow P[A,x] \enspace . \]
\end{axs}

\begin{ax}[Infinity]
There exists a set $\mathbb{N}$, an element $0 : \El{\mathbb{N}}$, and a function $s : \mathbb{N} \rightarrow \mathbb{N}$ such that:
\begin{itemize}
\item $\forall n : \El{\mathbb{N}}. s(n) \neq 0$
\item $\forall m,n : \El{\mathbb{N}}. s(m) = s(n) \Rightarrow m = n$.
\end{itemize}
\end{ax}

\section{Consequences of the Axioms}

\subsection{Definitions}

\begin{df}
Let $f,g : A \rightarrow B$. We say $f$ and $g$ are \emph{equal}, $f = g$, iff $\forall x : \El{A}. f(x) = g(x)$.
\end{df}

\begin{df}[Surjective]
A function $f : A \rightarrow B$ is \emph{surjective} iff, for all $y : \El{B}$, there exists $x : \El{A}$ such that $f(x) = y$.
\end{df}

\begin{df}[Bijective]
A function $f : A \rightarrow B$ is \emph{bijective} or a \emph{bijection} iff it is injective and surjective.

Sets $A$ and $B$ are \emph{equinumerous}, $A \approx B$, iff there exists a bijection between them.
\end{df}

If we prove there exists a set $X$ such that $P(X)$, and that any two sets that satisfy $P$ are bijective, then we may introduce a constant $C$ and define "Let $C$ be the set such that $P(C)$".

\subsection{The Empty Set}

\begin{thm}
There exists a set which has no elements.
\end{thm}

\begin{proof}
\pf
\step{1}{\pick\ a set $A$}
\begin{proof}
	\pf\ By the Axiom of Infinity, a set exists.
\end{proof}
\step{2}{\pflet{$S = \{ x : \El{A} \mid \bot \}$ with injection $i : S \rightarrow A$}}
\begin{proof}
	\pf\ Axiom of Separation.
\end{proof}
\step{3}{$S$ has no elements.}
\qed
\end{proof}

\begin{thm}
If $E$ and $E'$ have no elements then $E \approx E'$.
\end{thm}

\begin{proof}
\pf
\step{1}{\pflet{$E$ and $E'$ have no elements.}}
\step{2}{\pick\ a function $F : E \rightarrow E'$.}
\begin{proof}
	\pf\ Axiom of Choice since vacuously $\forall x : \El{E}. \exists y : \El{E'}. \top$.
\end{proof}
\step{4}{$F$ is injective.}
\begin{proof}
	\pf\ Vacuously, for all $x,y : \El{E}$, if $F(x) = F(y)$ then $x = y$.
\end{proof}
\step{5}{$F$ is surjective.}
\begin{proof}
	\pf\ Vacuously, for all $y : \El{E}$, there exists $x : \El{E}$ such that $F(x) = y$.
\end{proof}
\qed
\end{proof}

\begin{df}[Empty Set]
The \emph{empty set} $\emptyset$ is the set with no elements.
\end{df}

\subsection{The Singleton}

\begin{thm}
There exists a set that has exactly one element.
\end{thm}

\begin{proof}
\pf
\step{1}{\pick\ a set $A$ that has an element.}
\begin{proof}
	\pf\ By the Axiom of Infinity, there exists a set that has an element.
\end{proof}
\step{2}{\pick\ $a : \El{A}$}
\step{3}{\pflet{$R : A \looparrowright A$ be the relation such that, for all $x,y : \El{A}$, we have $xRy$ if and only if $x = y = a$.}}
\begin{proof}
	\pf\ By the Axiom of Comprehension.
\end{proof}
\step{4}{\pflet{$|R|$ be the tabulation of $R$ with projections $p,q : |R| \rightarrow A$.} \prove{$|R|$ has exactly one element.}}
\begin{proof}
	\pf\ By the Axiom of Tabulations.
\end{proof}
\step{5}{\pflet{$r : \El{|R|}$ be the element such that $p(r) = q(r) = a$}}
\begin{proof}
	\pf\ Since $aRa$ by \stepref{3}.
\end{proof}
\step{6}{\pflet{$s : \El{|R|}$} \prove{$s = r$}}
\step{7}{$p(s) R q(s)$}
\begin{proof}
	\pf\ By the Axiom of Tabulations.
\end{proof}
\step{8}{$p(s) = q(s) = a$}
\begin{proof}
	\pf\ By \stepref{3}.
\end{proof}
\step{9}{$p(s) = p(r)$ and $q(s) = q(r)$}
\begin{proof}
	\pf\ By \stepref{5}.
\end{proof}
\step{10}{$s = r$}
\begin{proof}
	\pf\ By the Axiom of Tabulations.
\end{proof}
\qed
\end{proof}

\begin{thm}
If $A$ and $B$ both have exactly one element then $A \approx B$.
\end{thm}

\begin{proof}
\pf
\step{1}{\pflet{$A$ and $B$ both have exactly one element.}}
\step{2}{\pflet{$F : A \looparrowright B$ be the relation such that, for all $x : \El{A}$ and $y : \El{B}$, we have $xFy$.}}
\step{3}{$F$ is a function.}
\begin{proof}
	\pf\ If $xFy$ and $xFy'$ then $y = y'$ because $B$ has only one element.
\end{proof}
\step{4}{$F$ is injective.}
\begin{proof}
	\pf\ If $F(x) = F(x')$ then $x = x'$ because $A$ has only one element.
\end{proof}
\step{5}{$F$ is surjective.}
\begin{proof}
	\step{a}{\pflet{$y : \El{B}$}}
	\step{b}{\pflet{$x$ be the element of $A$.}}
	\step{c}{$F(x) = y$}
\end{proof}
\qed
\end{proof}

\begin{df}[Singleton]
Let 1 be the set that has exactly one element. Let $*$ be its element.
\end{df}

\subsection{Subsets}

\begin{df}[Subset]
A \emph{subset} of a set $A$ is a relation $1 \looparrowright S$.

Given $S : 1 \looparrowright S$ and $a : \El{A}$, we write $a \in S$ for $*Sa$.
\end{df}

\begin{thms}
For any property $P[X,x]$ where $X$ is a set variable and $x : \El{X}$, the following is a theorem:

For any set $A$, there exists a set $B$ and injection $i : B \rightarrow A$ such that, for all $x : \El{A}$, we have $P[A,x]$ if and only if there exists $b : \El{B}$ such that $i(b) = x$.
\end{thms}

\begin{proof}
\pf
\step{1}{\pflet{$S : 1 \looparrowright A$ be the relation such that, for all $e : \El{1}$ and $a : \El{A}$, we have $eSa$ if and only if $P[A,a]$.}}
\begin{proof}
	\pf\ Axiom of Comprehension.
\end{proof}
\step{2}{\pflet{$B$ be the tabulation of $S$ with projections $p : B \rightarrow 1$ and $i : B \rightarrow A$.}}
\begin{proof}
	\pf\ Axiom of Tabulations.
\end{proof}
\step{3}{$i$ is injective.}
\begin{proof}
	\step{a}{\pflet{$r,s : \El{B}$}}
	\step{b}{\assume{$i(r) = i(s)$}}
	\step{c}{$p(r) = p(s)$}
	\begin{proof}
		\pf\ Since 1 has only one element.
	\end{proof}
	\step{d}{$r = s$}
	\begin{proof}
		\pf\ Axiom of Tabulations.
	\end{proof}
\end{proof}
\step{4}{For all $x : \El{A}$, we have $P[A,x]$ if and only if there exists $b : \El{B}$ such that $i(b) = x$.}
\begin{proof}
	\step{a}{\pflet{$x : \El{A}$}}
	\step{b}{If $P[A,x]$ then there exists $b : \El{B}$ such that $i(b) = x$}
	\begin{proof}
		\step{i}{\assume{$P[A,x]$}}
		\step{ii}{$*Sx$}
		\begin{proof}
			\pf\ \stepref{1}
		\end{proof}
		\step{iii}{There exists $b : \El{B}$ such that $p(b) = *$ and $i(b) = x$}
		\begin{proof}
			\pf\ Axiom of Tabulations.
		\end{proof}
	\end{proof}
	\step{c}{For all $b : \El{B}$ we have $P[A,i(b)]$}
	\begin{proof}
		\step{i}{\pflet{$b : \El{B}$}}
		\step{ii}{$p(b)Si(b)$}
		\begin{proof}
			\pf\ Axiom of Tabulations.
		\end{proof}
		\step{iii}{$P[A,i(b)]$}
		\begin{proof}
			\pf\ \stepref{1}
		\end{proof}
	\end{proof}
\end{proof}
\qed
\end{proof}

\section{Composition}

\begin{df}[Composite]
Let $\phi : A \looparrowright B$ and $\psi : B \looparrowright C$. The \emph{composite} $\psi \circ \phi : A \looparrowright C$ is the relation such that $a (\psi \circ \phi) c$ iff there exists $b$ such that $a \phi b$ and $b \psi c$.
\end{df}

\begin{df}[Identity]
For any set $A$, the \emph{identity} function $\id{A} : A \rightarrow A$ is the function defined by $\id{A}(a) = a$.
\end{df}

\begin{thm}
Composition of relations is associative, and the identity function is an identity for composition. The composite of functions is a function. The composite of injective functions is injective. The composite of surjective functions is surjective. The composite of bijections is a bijection. A function $f : A \rightarrow B$ is a bijection iff there exists a function $\inv{f} : B \rightarrow A$ such that $\inv{f} f = \id{A}$ and $f \inv{f} = \id{B}$, in which case $\inv{f}$ is unique.
\end{thm}

\section{Axioms Part Two}

\begin{ax}[Power Set]
For any set $A$, there exists a set $\mathcal{P} A$, the \emph{power set} of $A$, and a relation $\in : A \looparrowright \mathcal{P} A$, called \emph{membership}, such that, for any subset $S$ of $A$, there exists a unique $\overline{S} \in \mathcal{P} A$ such that, for all $x \in A$, we have $x \in \overline{S}$ if and only if $x \in S$.

We usually write just $S$ for $\overline{S}$.
\end{ax}

\begin{axs}[Collection]
Let $P[X,Y,x]$ be a formula with set variables $X$ and $Y$ and an element variable $x \in X$. Then the following is an axiom.

For any set $A$, there exists a set $B$, a function $p : B \rightarrow A$, a set $Y$ and a relation $M : B \looparrowright Y$ such that:
\begin{itemize}
\item $\forall b \in B. P[A, \{ y \in Y : bMy \}, p(b)]$
\item For all $a \in A$, if $\exists Y. P[A,Y,a]$, then there exists $b \in B$ such that $a = p(b)$.
\end{itemize}
\end{axs}

\begin{df}[Universe]
Let $E : U \looparrowright X$ be a relation. Let us say that a set $A$ is \emph{small} iff there exists $u \in U$ such that $A \approx \{ x \in X : uEx \}$.

Then $(U,X,E)$ form a \emph{universe} if and only if:
\begin{itemize}
\item $\mathbb{N}$ is $U$-small.
\item For any $U$-small sets $A$ and $B$ and relation $R : A \looparrowright B$, the tabulation of $R$ is $U$-small.
\item If $A$ is $U$-small then so is $\mathcal{P} A$
\item Let $f : A \rightarrow B$ be a function. If $B$ is $U$-small and $f^{-1}(b)$ is $U$-small for all $b \in B$, then $A$ is $U$-small.
\item If $p : B \twoheadrightarrow A$ is a surjective function such that $A$ is $U$-small, then there exists a $U$-small set $C$, a surjection $q : C \twoheadrightarrow A$, and a function $f : C \rightarrow B$ such that $q = pf$.
\end{itemize}
\end{df}

\begin{ax}[Universe]
There exists a universe.
\end{ax}

Let $E : U \looparrowright X$ be a universe. We shall say a set is \emph{small} iff it is $U$-small, and \emph{large} otherwise.

\section{Cartesian Product}

\begin{df}[Cartesian Product]
Let $A$ and $B$ be sets. The \emph{Cartesian product} of $A$ and $B$, $A \times B$, is the tabulation of the relation $A \looparrowright B$ that holds for all $a \in A$ and $b \in B$. The associated functions $\pi_1 : A \times B \rightarrow A$ and $\pi_2 : A \times B \rightarrow B$ are called the \emph{projections}.

Given $a \in A$ and $b \in B$, we write $(a,b)$ for the unique element of $A \times B$ such that $\pi_1(a,b) = a$ and $\pi_2(a,b) = b$.
\end{df}

\section{Quotient Sets}

\begin{prop}
Let $\sim$ be an equivalence relation on $X$. Then there exists a set $X/\sim$, the \emph{quotient set} of $X$ with respect to $\sim$, and a surjective function $\pi : X \twoheadrightarrow X / \sim$, the \emph{canonical projection}, such that, for all $x,y : \El{X}$, we have $x \sim y$ if and only if $\pi(x) = \pi(y)$.

Further, if $p : X \twoheadrightarrow Q$ is another quotient with respect to $\sim$, then there exists a unique bijection $\phi : X / \sim \approx Q$ such that $\phi \circ \pi = p$.
\end{prop}

\chapter{Topology}

\section{Topological Spaces}

\begin{df}[Topological Space]
Let $X$ be a set and $\mathcal{O} \subseteq \mathcal{P} X$. Then we say $(X, \mathcal{O})$ is a \emph{topological space} iff:
\begin{itemize}
\item For any $\mathcal{U} \subseteq \mathcal{O}$ we have $\bigcup \mathcal{U} \in \mathcal{O}$.
\item For any $U, V \in \mathcal{O}$ we have $U \cap V \in \mathcal{O}$.
\item $X \in \mathcal{O}$
\end{itemize}
We call $\mathcal{O}$ the \emph{topology} of the toplogical space, and call its elements \emph{open} sets. We shall often write $X$ for the topological space $(X, \mathcal{O})$.
\end{df}

\begin{df}[Closed Set]
Let $X$ be a topological space and $A \subseteq X$. Then $A$ is \emph{closed} iff $X - A$ is open.
\end{df}

\begin{prop}
A set $B$ is open if and only if $X - B$ is closed.
\end{prop}

\begin{prop}
Let $X$ be a set and $\mathcal{C} \subseteq \mathcal{P} X$. Then there exists a topology $\mathcal{O}$ on $X$ such that $\mathcal{C}$ is the set of closed sets if and only if:
\begin{itemize}
\item For any $\mathcal{D} \subseteq \mathcal{C}$ we have $\bigcap \mathcal{D} \in \mathcal{C}$
\item For any $C, D \in \mathcal{C}$ we have $C \cup D \in \mathcal{C}$.
\item $\emptyset \in \mathcal{C}$
\end{itemize}
In this case, $\mathcal{O}$ is unique and is given by $\mathcal{O} = \{ X - C : C \in \mathcal{C} \}$.
\end{prop}

\begin{df}[Neighbourhood]
Let $X$ be a topological space, $Sx \in X$ and $U \subseteq X$. Then $U$ is a \emph{neighbourhood} of $x$, and $x$ is an \emph{interior} point of $U$, iff there exists an open set $V$ such that $x \in V \subseteq U$.
\end{df}

\begin{prop}
A set $B$ is open if and only if it is a neighbourhood of each of its points.
\end{prop}

\begin{prop}
Let $X$ be a set and $\mathcal{N} : X \rightarrow \mathcal{P} X$. Then there exists a topology $\mathcal{O}$ on $X$ such that, for all $x \in X$, we have $\mathcal{N}_x$ is the set of neighbourhoods of $x$, if and only if:
\begin{itemize}
\item For all $x \in X$ and $N \in \mathcal{N}_x$ we have $x \in N$
\item For all $x \in X$ we have $X \in \mathcal{N}_x$
\item For all $x \in X$, $N \in \mathcal{N}_x$ and $V \subseteq \mathcal{P} X$, if $N \subseteq V$ then $V \in \mathcal{N}_x$
\item For all $x \in X$ and $M, N \in \mathcal{N}_x$ we have $M \cap N \in \mathcal{N}_x$
\item For all $x \in X$ and $N \in \mathcal{N}_x$, there exists $M \in \mathcal{N}_x$ such that $M \subseteq N$ and $\forall y \in M. M \in \mathcal{N}_y$.
\end{itemize}
In this case, $\mathcal{O}$ is unique and is given by $\mathcal{O} = \{ U : \forall x \in U. U \in \mathcal{N}_x \}$.
\end{prop}

\begin{df}[Exterior Point]
Let $X$ be a topological space, $x \in X$ and $B \subseteq X$. Then $x$ is an \emph{exterior point} of $B$ iff $B - X$ is a neighbourhood of $x$.
\end{df}

\begin{df}[Boundary Point]
Let $X$ be a topological space, $x \in X$ and $B \subseteq X$. Then $x$ is a \emph{boundary point} of $B$ iff it is neither an interior point nor an exterior point of $B$.
\end{df}

\begin{df}[Interior]
Let $X$ be a topological space and $B \subseteq X$. The \emph{interior} of $B$, $B^\circ$, is the set of all interior points of $B$.
\end{df}

\begin{prop}
The interior of $B$ is the union of all the open sets included in $B$.
\end{prop}

\begin{df}[Closure]
Let $X$ be a topological space and $B \subseteq X$. The \emph{closure} of $B$, $\overline{B}$, is the set of all points that are not exterior points of $B$.
\end{df}

\begin{prop}
The closure of $B$ is the intersection of all the closed sets that include $B$.
\end{prop}

\begin{prop}
A set $B$ is open iff $X - B = \overline{X - B}$.
\end{prop}

\begin{prop}[Kuratowski Closure Axioms]
Let $X$ be a set and $\overline{\ } : \mathcal{P} X \rightarrow \mathcal{P} X$. Then there exists a topology $\mathcal{O}$ such that, for all $B \subseteq X$, $\overline{B}$ is the closure of $B$, if and only if:
\begin{itemize}
\item $\overline{\emptyset} = \emptyset$
\item For all $A \subseteq X$ we have $A \subseteq \overline{A}$
\item For all $A \subseteq X$ we have $\overline{\overline{A}} = \overline{A}$
\item For all $A, B \subseteq X$ we have $\overline{A \cup B} = \overline{A} \cup \overline{B}$
\end{itemize}
In this case, $\mathcal{O}$ is unique and is defined by $\mathcal{O} = \{ U : X - U = \overline{X - U} \}$.
\end{prop}

\subsection{Subspaces}

\begin{df}[Subspace]
Let $X$ be a topological space and $X_0 \subseteq X$. The \emph{subspace topology} on $X_0$ is $\{ U \cap X_0 : U \text{ is open in } X \}$.
\end{df}

\begin{ex}
The \emph{unit sphere} $S^2$ is $\{ x \in \mathbb{R}^3 : \| x \| = 1 \}$ as a subspace of $\mathbb{R}^3$.
\end{ex}

\subsection{Topological Disjoint Union}

\begin{df}
Let $X$ and $Y$ be topological spaces. The \emph{disjoint union} is $X + Y$ where $U \subseteq X + Y$ is open if and only if $\inv{\kappa_1}(U)$ is open in $X$ and $\inv{\kappa_2}(U)$ is open in $Y$.
\end{df}

\subsection{Product Topology}

\begin{df}
Let $X$ and $Y$ be topological spaces. The \emph{product topology} on $X \times Y$ is the set of all subsets $W \subseteq X \times Y$ such that, for all $(x,y) \in W$, there exist neighbourhoods $U$ of $x$ in $X$ and $V$ of $y$ in $Y$ such that $U \times V \subseteq W$.
\end{df}

\subsection{Bases}

\begin{df}[Basis]
Let $X$ be a topological space. A \emph{basis} for the topology on $X$ is a set of open sets $\mathcal{B}$ such that every open set is the union of a subset of $\mathcal{B}$.
\end{df}

\subsection{Subbases}

\begin{df}[Subbasis]
Let $X$ be a topological space. A \emph{subbasis} for the topology on $X$ is a subset $\mathcal{S} \subseteq \mathcal{P} X$ such that every open set is a union of finite intersections of $\mathcal{S}$.
\end{df}

\section{Continuous Functions}

\begin{df}[Continuous]
Let $X$ and $Y$ be topological spaces. A function $f : X \rightarrow Y$ is \emph{continuous} iff, for every open set $V$ in $Y$, the inverse image $\inv{f}(V)$ is open in $X$.
\end{df}

\begin{prop}
\begin{enumerate}
\item $\id{X}$ is continuous
\item The composite of two continuous functions is continuous.
\item If $f : X \rightarrow Y$ is continuous and $X_0 \subseteq X$ then $f \restriction X_0 : X_0 \rightarrow Y$ is continuous.
\item If $f : X + Y \rightarrow Z$, then $f$ is continuous iff $f \circ \kappa_1 : X \rightarrow Z$ and $f \circ \kappa_2 : Y \rightarrow Z$ are continuous.
\item If $f : Z \rightarrow X \times Y$, then $f$ is continuous iff $\pi_1 \circ f$ and $\pi_2 \circ f$ are continuous.
\end{enumerate}
\end{prop}

\begin{df}[Homeomorphism]
Let $X$ and $Y$ be topological spaces. A \emph{homeomorphism} between $X$ and $Y$ is a bijection $f : X \approx Y$ such that $f$ and $\inv{f}$ are continuous.
\end{df}

\begin{df}[Retraction]
Let $X$ be a topological space and $A$ a subspace of $X$. A continuous function $\rho : X \rightarrow A$ is a \emph{retraction} iff $\rho \restriction A = \id{A}$. We say $A$ is a \emph{retract} of $X$ iff there exists a retraction.
\end{df}

\section{Convergence}

\begin{df}[Convergence]
Let $X$ be a topological space. Let $(x_n)$ be a sequence in $X$. A point $a : \El{X}$ is a \emph{limit} of the sequence iff, for every neighbourhood $U$ of $a$, there exists $n_0$ such that $\forall n \geq n_0. x_n \in U$.
\end{df}

\section{Connected Spaces}

\begin{df}[Connected]
A topological space is \emph{connected} iff it is not the union of two nonempty open disjoint subsets.
\end{df}

\begin{prop}
The continuous image of a connected space is connected.
\end{prop}

\begin{prop}
Let $X$ be a topological space and $A,B \subseteq X$. If $X = A \cup B$, $A \cap B \neq \emptyset$, and $A$ and $B$ are connected, then $X$ is connected.
\end{prop}

\begin{prop}
If $X$ and $Y$ are nonempty topological spaces, then
$X \times Y$ is connected if and only if $X$ and $Y$ are connected.
\end{prop}

\begin{df}[Path-connected]
A topological space $X$ is \emph{path-connected} iff, for any points $a,b \in X$, there exists a continuous function $\alpha : [0,1] \rightarrow X$, called a \emph{path}, such that $\alpha(0) = a$ and $\alpha(1) = b$.
\end{df}

\begin{prop}
The continuous image of a path connected space is path connected.
\end{prop}

\begin{prop}
Let $X$ be a topological space and $A,B \subseteq X$. If $X = A \cup B$, $A \cap B \neq \emptyset$, and $A$ and $B$ are path connected, then $X$ is path connected.
\end{prop}

\begin{prop}
If $X$ and $Y$ are nonempty topological spaces, then
$X \times Y$ is path connected if and only if $X$ and $Y$ are path connected.
\end{prop}

\section{Hausdorff Spaces}

\begin{df}[Hausdorff]
A topological space is a \emph{Hausdorff} space or a \emph{$T_2$} space iff any two distinct points have disjoint neighbourhoods.
\end{df}

\begin{prop}
In a Hausdorff space, a sequence has at most one limit.
\end{prop}

\begin{prop}
\begin{enumerate}
\item Every subspace of a Hausdorff space is Hausdorff.
\item The disjoint union of two Hausdorff spaces is Hausdorff.
\item The product of two Hausdorff spaces is Hausdorff.
\end{enumerate}
\end{prop}

\begin{prop}
Let $A$ be a topological space and $B$ a Hausdorff space. Let $f,g : A \rightarrow B$ be continuous. Let $X \subseteq A$ be dense. If $f$ and $g$ agree on $X$, then $f = g$.
\end{prop}

\begin{proof}
\pf
\step{1}{\assume{for a contradiction $a \in A$ and $f(a) \neq g(a)$.}}
\step{2}{\pick\ disjoint neighbourhoods $U$ and $V$ of $f(a)$ and $g(a)$ respectively.}
\step{3}{\pick\ $x \in \inv{f}(U) \cap \inv{g}(V)$}
\step{4}{$f(x) = g(x) \in U \cap V$}
\qedstep
\begin{proof}
\pf\ This is a contradiction.
\end{proof}
\qed
\end{proof}

\begin{prop}
Let $X$ and $Y$ be metric spaces. Let $f : X \rightarrow Y$ be uniformly continuous. Let $\hat{X}$ and $\hat{Y}$ be the completions of $X$ and $Y$. Then $f$ extends uniquely to a continuous map $\hat{X} \rightarrow \hat{Y}$.
\end{prop}

\begin{proof}
\pf\ The extension maps $\lim_{n \rightarrow \infty} x_n$ to $\lim_{n \rightarrow \infty} f(x_n)$. \qed
\end{proof}

\section{Compactness}

\begin{df}[Compact]
A topological space is \emph{compact} iff every open cover has a finite subcover.
\end{df}

\begin{prop}
Let $X$ be a compact topological space. Let $P$ be a set of open sets such that, for all $U,V \in P$, we have $U \cup V \in P$. Assume that every point has an open neighbourhood in $P$. Then $X \in P$.
\end{prop}

\begin{proof}
\pf
\step{1}{$P$ is an open cover of $X$}
\step{2}{\pick\ a finite subcover $U_1, \ldots, U_n \in P$}
\step{3}{$X = U_1 \cup \cdots \cup U_n \in P$}
\qed
\end{proof}

\begin{cor}
Let $f$ be a compact space and $f : X \rightarrow \mathbb{R}$ be locally bounded. Then $f$ is bounded.
\end{cor}

\begin{proof}
\pf\ Take $P = \{ U \text{ open in } X : f \text{ is bounded on } U \}$. \qed
\end{proof}

\begin{prop}
The continuous image of a compact space is compact.
\end{prop}

\begin{prop}
A closed subspace of a compact space is compact.
\end{prop}

\begin{prop}
Let $X$ and $Y$ be nonempty spaces. Then the following are equivalent.
\begin{enumerate}
\item $X$ and $Y$ are compact.
\item $X + Y$ is compact.
\item $X \times Y$ is compact.
\end{enumerate}
\end{prop}

\begin{prop}
A compact subspace of a Hausdorff space is closed.
\end{prop}

\begin{prop}
A continuous bijection from a compact space to a Hausdorff space is a homeomorphism.
\end{prop}

\section{Quotient Spaces}

\begin{df}[Quotient Space]
Let $X$ be a topological space and $\sim$ an equivalence relation on $X$. The \emph{quotient topology} on $X / \sim$ is defined by: $U : \El{\mathcal{P} X}$ is open in $X / \sim$ if and only if $\inv{\pi}(U)$ is open in $X$.
\end{df}

\begin{prop}
\label{prop:map_from_quotient_continuous}
Let $X$ and $Y$ be topological spaces. Let $\sim$ be an equivalence relation on $X$. Let $f : X / \sim \rightarrow Y$. Then $f$ is continuous if and only if $f \circ \pi$ is continuous.
\end{prop}

\begin{prop}
Let $X$ and $Y$ be topological spaces. Let $\sim$ be an equivalence relation on $X$. Let $\phi : Y \rightarrow X / \sim$.

Assume that, for all $y \in Y$, there exists a neighbourhood $U$ of $y$ and a continuous function $\Phi : U \rightarrow X$ such that $\pi \circ \Phi = \phi \restriction U$. Then $\phi$ is continuous.
\end{prop}

\begin{prop}
A quotient of a connected space is connected.
\end{prop}

\begin{prop}
A quotient of a path connected space is path connected.
\end{prop}

\begin{prop}
Let $X$ be a topological space and $\sim$ an equivalence relation on $X$. If $X / \sim$ is Hausdorff then every equivalence class of $\sim$ is closed in $X$.
\end{prop}

\begin{df}
Let $X$ be a topological space and $A_1, \ldots, A_r \subseteq X$. Then $X / A_1, \ldots, A_r$ is the quotient space of $X$ with respect to $\sim$ where $x \sim y$ iff $x = y$ or $\exists i (x \in A_i \wedge y \in A_i)$.
\end{df}

\begin{df}[Cone]
Let $X$ be a topological space. The \emph{cone over $X$} is the space $(X \times [0,1]) / (X \times \{1\})$.
\end{df}

\begin{df}[Suspension]
Let $X$ be a topological space. The \emph{suspension} of $X$ is the space
\[ \Sigma X := (X \times [-1,1]) / (X \times \{-1\}),(X \times \{1\}) \]
\end{df}

\begin{df}[Wedge Product]
Let $x_0 \in X$ and $y_0 \in Y$. The \emph{wedge product} $X \vee Y$ is $(X \times \{y_0\}) \cup (\{x_0\} \times Y)$ as a subspace of $X \times Y$.
\end{df}

\begin{df}[Smash Product]
Let $x_0 \in X$ and $y_0 \in Y$. The \emph{smash product} $X \wedge Y$ is $(X \times Y) / (X \vee Y)$.
\end{df}

\begin{ex}
$D^n / S^{n-1} \cong S^n$
\end{ex}

\begin{proof}
\pf
\step{1}{\pflet{$\phi : D^n / S^{n-1} \rightarrow S^n$ be the function induced by the map $D^n \rightarrow S^n$ that maps the radii of $D^n$ onto the meridians of $S^n$ from the north to the south pole.}}
\step{2}{$\phi$ is a bijection.}
\step{3}{$\phi$ is a homeomorphism.}
\begin{proof}
	\pf\ Since $D^n / S^{n-1}$ is compact and $S^n$ is Hausdorff.
\end{proof}
\qed
\end{proof}

\section{Gluing}

\begin{df}[Gluing]
Let $X$ and $Y$ be topological spaces, $X_0 \subseteq X$ and $\phi : X_0 \rightarrow Y$ a continuous map. Then $Y \cup_\phi X$ is the quotient space $(X + Y)/ \sim$, where $\sim$ is the equivalence relation generated by $x \sim \phi(x)$ for all $x : \El{X}$.
\end{df}

\begin{prop}
$Y$ is a subspace of $Y \cup_\phi X$.
\end{prop}

\begin{df}
Let $X$ be a topological space and $\alpha : X \cong X$ a homeomorphism. Then $(X \times [0,1]) / \alpha$ is the quotient space of $X \times [0,1]$ by the equivalence relation generated by $(x,0) \sim (\alpha(x),1)$ for all $x : \El{X}$.
\end{df}

\begin{df}[M\"{o}bius Strip]
The \emph{M\"{o}bius strip} is $([-1,1] \times [0,1])/ \alpha$ where $\alpha(x) = -x$.
\end{df}

\begin{df}[Klein Bottle]
The \emph{Klein bottle} is $(S^1 \times [0,1]) / \alpha$ where $\alpha(z) = \overline{z}$.
\end{df}

\begin{prop}
Let $M$ be the M\"{o}bius strip and $K$ the Klein bottle. Then $M \cup_{\id{\partial M}} M \cong K$.
\end{prop}

\begin{proof}
\pf
\step{1}{\pflet{$f : ([-1,1] \times [0,1]) + ([-1,1] \times [0,1]) \rightarrow S^1 \times [0,1]$ be the function that maps $\kappa_1(\theta,t)$ to $(e^{\pi i \theta / 2}, t)$ and $\kappa_2(\theta,t)$ to $(-e^{- \pi i \theta / 2}, t)$.}}
\step{2}{$f$ induces a bijection $M \cup_{\id{\partial M}} M \approx K$}
\step{3}{$f$ is a homeomorphism.}
\qed
\end{proof}

\section{Metric Spaces}

%TODO Define real numbers
\begin{df}[Metric Space]
Let $X$ be a set and $d : X^2 \rightarrow \mathbb{R}$. We say $(X,d)$ is a \emph{metric space} iff:
\begin{itemize}
\item For all $x,y \in X$ we have $d(x,y) \geq 0$
\item For all $x,y \in X$ we have $d(x,y) = 0$ iff $x = y$
\item For all $x,y \in X$ we have $d(x,y) = d(y,x)$
\item (\emph{Triangle Inequality}) For all $x,y,z \in X$ we have $d(x,z) \leq d(x,y) + d(y,z)$
\end{itemize}
We call $d$ the \emph{metric} of the metric space $(X,d)$. We often write $X$ for the metric space $(X,d)$.
\end{df}

\begin{df}[Topology of a Metric Space]
Let $(X,d)$ be a metric space. The topology \emph{induced} by the metric $d$ is defined by: for $V \subseteq X$, we have $V$ is open if and only if, for all $x \in V$, there exists $\epsilon > 0$ such that $\{ y \in X : d(x,y) < \epsilon \} \subseteq V$.
\end{df}

\begin{df}[Metrizable]
A topological space is \emph{metrizable} iff there exists a metric that induces its topology.
\end{df}

\begin{prop}
Every metrizable space is Hausdorff.
\end{prop}

\section{Complete Metric Spaces}

\begin{df}[Complete]
A metric space is \emph{complete} iff every Cauchy sequence converges.
\end{df}

\begin{ex}
$\mathbb{R}$ is complete.
\end{ex}

\begin{prop}
The product of two complete metric spaces is complete.
\end{prop}

\begin{prop}
Every compact metric space is complete.
\end{prop}

\begin{prop}
Let $X$ be a complete metric space and $A \subseteq X$. Then $A$ is complete if and only if $A$ is closed.
\end{prop}

\begin{df}[Completion]
Let $X$ be a metric space. A \emph{completion} of $X$ is a complete metric space $\hat{X}$ and injection $i : X \rightarrowtail \hat{X}$ such that:
\begin{itemize}
\item The metric on $X$ is the restriction of the metric on $\hat{X}$
\item $X$ is dense in $\hat{X}$.
\end{itemize}
\end{df}

\begin{prop}
Let $i_1 : X \rightarrow Y_1$ and $i_2 : X \rightarrow Y_2$ be completions of $X$. Then there exists a unique isometry $\phi : Y_1 \cong Y_2$ such that $\phi \circ i_1 = i_2$.
\end{prop}

\begin{proof}
\pf\ Define $\phi(\lim_{n \rightarrow \infty} i_1(x_n)) = \lim_{n \rightarrow \infty} i_2(x_n)$. \qed
\end{proof}

\begin{thm}
Every metric space has a completion.
\end{thm}

\begin{proof}
\pf\ Let $\hat{X}$ be the set of Cauchy sequences in $X$ quotiented by $\sim$ where $(x_n) \sim (y_n)$ if and only if $d(x_n, y_n) \rightarrow 0$. \qed
\end{proof}

\chapter{Homotopy Theory}

\section{Homotopies}

\begin{df}[Homotopy]
Let $X$ and $Y$ be topological spaces. Let $f,g : X \rightarrow Y$ be continuous. A \emph{homotopy} between $f$ and $g$ is a continuous function $h : X \times [0,1] \rightarrow Y$ such that
\begin{itemize}
\item $\forall x : \El{X}. h(x,0) = f(x)$
\item $\forall x : \El{X}. h(x,1) = g(x)$
\end{itemize}
We say $f$ and $g$ are \emph{homotopic}, $f \simeq g$, iff there exists a homotopy between them.

Let $[X,Y]$ be the set of all homotopy classes of functions $X \rightarrow Y$.
\end{df}

\begin{prop}
Let $f,f' : X \rightarrow Y$ and $g,g' : Y \rightarrow Z$ be continuous. If $f \simeq f'$ and $g \simeq g'$ then $g \circ f \simeq g' \circ f'$.
\end{prop}

\section{Homotopy Equivalence}

\begin{df}[Homotopy Equivalence]
Let $X$ and $Y$ be topological spaces. A \emph{homotopy equivalence} between $X$ and $Y$, $f : X \simeq Y$, is a continuous function $f : X \rightarrow Y$ such that there exists a continuous function $g : Y \rightarrow X$, the \emph{homotopy inverse} to $f$, such that $g \circ f \simeq \id{X}$ and $f \circ g \simeq \id{Y}$.
\end{df}

\begin{df}[Contractible]
A topological space $X$ is \emph{contractible} iff $X \simeq 1$.
\end{df}

\begin{ex}
$\mathbb{R}^n$ is contractible.
\end{ex}

\begin{df}[Deformation Retract]
Let $X$ be a topological space and $A$ a subspace of $X$. A retraction $\rho : X \rightarrow A$ is a \emph{deformation retraction} iff $i \circ \rho \simeq \id{X}$, where $i$ is the inclusion $A \rightarrowtail X$. We say $A$ is a \emph{deformation retract} of $X$ iff there exists a deformation retraction.
\end{df}
\chapter{Topological Groups}

\begin{df}[Topological Group]
A \emph{topological group} is a group $G$ with a topology such that the function $G^2 \rightarrow G$ that maps $(x,y)$ to $x\inv{y}$ is continuous.
\end{df}

\begin{ex}
$GL(n,\mathbb{R})$ and $GL(n,\mathbb{C})$ are topological groups.
\end{ex}

\begin{prop}
Any subgroup of a topological group is a topological group under the subspace topology.
\end{prop}

\begin{df}[Homogeneous Space]
A \emph{homogeneous space} is a topological space of the form $G/H$, where $G$ is a topological group and $H$ is a normal subgroup of $G$, under the quotient topology.
\end{df}

\begin{prop}
Let $G$ be a topological group and $H$ a normal subgroup of $G$. Then $G/H$ is Hausdorff if and only if $H$ is closed.
\end{prop}

\begin{proof}
\pf\ See Bourbaki, N., General Topology. III.12 \qed
\end{proof}

\section{Continuous Actions}

\begin{df}[Continuous Action]
Let $G$ be a topological group and $X$ a topological space. A \emph{continuous action} of $G$ on $X$ is a continuous function $\cdot : G \times X \rightarrow X$ such that:
\begin{itemize}
\item $\forall x : \El{X}. ex = x$
\item $\forall g,h : \El{G}. \forall x : \El{X}. g(hx) = (gh)x$
\end{itemize}

A \emph{$G$-space} consists of a topological space $X$ and a continuous action of $G$ on $X$.
\end{df}

\begin{df}[Orbit]
Let $X$ be a $G$-space and $x \in X$. The \emph{orbit} of $x$ is $\{ gx : g \in G \}$.

The \emph{orbit space} $X / G$ is the set of all orbits under the quotient topology.
\end{df}

\begin{prop}
Define an action of $SO(2)$ on $S^2$ by $g(x_1, x_2, x_3) = (g(x_1, x_2), x_3)$. Then $S^2 / SO(2) \cong [-1,1]$.
\end{prop}

\begin{proof}
\pf
\step{1}{\pflet{$f_3 : S^2 / SO(2) \rightarrow [-1,1]$ be the function induced by $\pi_3 : S^2 \rightarrow [-1,1]$}}
\step{2}{$f_3$ is bijective.}
\step{3}{$S^2 / SO(2)$ is compact.}
\begin{proof}
	\pf\ It is the continuous image of $S^2$ which is compact.
\end{proof}
\step{4}{$[-1,1]$ is Hausdorff.}
\step{5}{$f_3$ is a homeomorphism.}
\qed
\end{proof}

\begin{df}[Stabilizer]
Let $X$ be a $G$-space and $x \in X$. The \emph{stabilizer} of $x$ is $G_x := \{ g : \El{G} \mid gx = x \}$.
\end{df}

\begin{prop}
The function that maps $gG_x$ to $gx$ is a continuous bijection from $G / G_x$ to $Gx$.
\end{prop}

\begin{proof}
\pf
\step{1}{If $gG_x = hG_x$ then $gx = hx$.}
\begin{proof}
	\step{a}{\assume{$gG_x = hG_x$}}
	\step{b}{$\inv{g}h \in G_x$}
	\step{c}{$\inv{g}h x = x$}
	\step{d}{$gx = hx$}
\end{proof}
\step{2}{If $gx = hx$ then $gG_x = hG_x$.}
\begin{proof}
	\pf\ Similar.
\end{proof}
\step{3}{The function is continuous.}
\begin{proof}
	\pf\ Proposition \ref{prop:map_from_quotient_continuous}.
\end{proof}
\qed
\end{proof}

\chapter{Topological Vector Spaces}

\begin{df}[Topological Vector Space]
Let $K$ be either $\mathbb{R}$ or $\mathbb{C}$. A \emph{topological vector space} over $K$ consists of a 	vector space $E$ over $K$ and a topology on $E$ such that:
\begin{itemize}
\item Substraction is a continuous function $E^2 \rightarrow E$
\item Multiplication is a continuous function $K \times E \rightarrow E$
\end{itemize}
\end{df}

\begin{prop}
Every topological vector space is a topological group under addition.
\end{prop}

\begin{proof}
\pf\ Immediate from the definition. \qed
\end{proof}

\begin{thm}
The usual topology on a finite dimensional vector space over $K$ is the only one that makes it into a Hausdorff topological vector space.
\end{thm}

\begin{proof}
\pf\ See Bourbaki. Elements de Mathematique, Livre V: Espaces Vectoriels Topologiques, Th. 2, p. 18 \qed
\end{proof}

\begin{prop}
Let $E$ be a topological vector space and $E_0$ a subspace of $E$. Then $\overline{E_0}$ is a subspace of $E$.
\end{prop}

\begin{df}
Let $E$ be a topological vector space. The topological space \emph{associated} with $E$ is $E / \overline{\{0\}}$.
\end{df}

\section{Cauchy Sequences}

\begin{df}[Cauchy Sequence]
Let $E$ be a topological vector space. A sequence $(x_n)$ in $E$ is a \emph{Cauchy sequence} iff, for every neighbourhood $U$ of 0, there exists $n_0$ such that $\forall m,n \geq n_0. x_n - x_m \in U$.
\end{df}

\begin{df}[Complete Topological Vector Space]
A topological vector space is \emph{complete} iff every Cauchy sequence converges.
\end{df}

\section{Seminorms}

\begin{df}[Seminorm]
Let $E$ be a vector space over $K$. A \emph{seminorm} on $E$ is a function $\|\ \| : E \rightarrow \mathbb{R}$ such that:
\begin{enumerate}
\item $\forall x : \El{E}. \| x \| \geq 0$
\item $\forall \alpha : \El{K}. \forall x : \El{E}. \| \alpha x \| = |\alpha| \|x\|$
\item \emph{Triangle Inequality} $\forall x,y : \El{E}. \| x + y \| \leq \| x \| + \| y \|$
\end{enumerate}
\end{df}

\begin{ex}
The function that maps $(x_1, \ldots, x_n)$ to $|x_i|$ is a seminorm on $\mathbb{R}^n$.
\end{ex}

\begin{df}
Let $E$ be a vector space over $K$.
Let $\Lambda$ be a set of seminorms on $E$. The topology \emph{generated} by $\Lambda$ is the topology generated by the subbasis consisting of all sets of the form $B_\epsilon^\lambda(x) = \{ y \in E : \lambda(y-x) < \epsilon \}$ for $\epsilon > 0$, $\lambda \in \Lambda$ and $x : \El{E}$.
\end{df}

\begin{prop}
$E$ is a topological vector space under this topology. It is Hausdorff iff, for all $x : \El{E}$, if $\forall \lambda \in \Lambda. \lambda(x) = 0$ then $x = 0$.
\end{prop}

\section{Fr\'{e}chet Spaces}

\begin{df}[Pre-Fr\'{e}chet Space]
A \emph{pre-Fr\'{e}chet space} is a Hausdorff topological vector space whose topology is generated by a countable set of seminorms.
\end{df}

\begin{prop}
Let $E$ be a pre-Fr\'{e}chet space whose topology is generated by the family of seminorms $\{ \|\ \|_n : n \in \mathbb{Z}^+ \}$. Then
\[ d(x,y) = \sum_{n=1}^\infty \frac{1}{2^n} \frac{\|x-y\|_n}{1 + \|x-y\|_n} \]
is a metric that induces the same topology. The two definitions of Cauchy sequence agree.
\end{prop}

\begin{df}[Fr\'{e}chet Space]
A \emph{Fr\'{e}chet space} is a complete pre-Fr\'{e}chet space.
\end{df}

\section{Normed Spaces}

\begin{df}[Normed Space]
Let $E$ be a vector space over $K$. A \emph{norm} on $E$ is a function $\|\ \| : E \rightarrow \mathbb{R}$ is a seminorm such that, $\forall x \in E. \| x \| = 0 \Leftrightarrow x = 0$.

A \emph{normed space} consists of a vector space with a norm.
\end{df}

\begin{prop}
If $E$ is a normed space then $d(x,y) = \| x - y \|$ is a metric on $E$ that makes $E$ into a topological vector space. The two definitions of Cauchy sequence agree on $E$.
\end{prop}

\begin{prop}
Let $\|\ \|$ be a seminorm on the vector space $E$. Then $\|\ \|$ defines a norm on $E / \overline{\{0\}}$.
\end{prop}

\begin{prop}
Let $E$ and $F$ be normed spaces. Any continuous linear map $E \rightarrow F$ is uniformly continuous.
\end{prop}

\begin{df}
For $p \geq 1$. let $\mathcal{L}^p(\mathbb{R}^n)$ be the vector space of all Lebesgue-measurable functions $f : \mathbb{R}^n \rightarrow \mathbb{R}$ such that $|f|^p$ is Lebesgue-integrable. Then
\[ \| f \|_p := \sqrt{p}{\int_{\mathbb{R}^n} |f(x)|^p dx} \]
defines a seminorm on $\mathcal{L}^p(\mathbb{R}^n)$. Let
\[ L^p(\mathbb{R}^n) := \mathcal{L}^p(\mathbb{R}^n) / \overline{\{0\}} \enspace . \]
\end{df}

\section{Inner Product Spaces}

\begin{prop}
If $E$ is an inner product space then $\| x \| = \sqrt{\langle x,x \rangle}$ is a norm on $E$.
\end{prop}

\section{Banach Spaces}

\begin{df}[Banach Space]
A \emph{Banach space} is a complete normed space.
\end{df}

\begin{ex}
For any topological space $X$, the set $C(X)$ of bounded continuous functions $X \rightarrow \mathbb{R}$ is a Banach space under $\| f \| = \sup_{x \in X} |f(x)|$.
\end{ex}

\begin{prop}
The completion of a normed space is a Banach space.
\end{prop}

\begin{prop}
Let $E$ and $F$ be normed spaces. Let $f : E \rightarrow F$ be a continuous linear map. Then the extension to the completions $\hat{E} \rightarrow \hat{F}$ is linear.
\end{prop}

\begin{prop}
$L^p(\mathbb{R}^n)$ is a Banach space.
\end{prop}

\section{Hilbert Spaces}

\begin{df}[Hilbert Space]
A \emph{Hilbert space} is a complete inner product space.
\end{df}

\begin{ex}
The set of \emph{square-integrable functions} is the set of Lebesgue integrable functions $[-\pi,\pi] \rightarrow \mathbb{R}$ quotiented by: $f \sim g$ iff $\{ x \in [-\pi,\pi] : f(x) \neq g(x) \}$ has measure 0. This is a Hilbert space under
\[ \langle f,g \rangle = \frac{1}{\pi} \int_{- \pi}{\pi} f(x) g(x) dx \enspace . \]
\end{ex}

\begin{prop}
The completion of an inner product space is a Hilbert space.
\end{prop}

\section{Locally Convex Spaces}

\begin{df}[Locally Convex Space]
A topological vector space is \emph{locally convex} iff every neighbourhood of 0 includes a convex neighbourhood of 0.
\end{df}

\begin{prop}
A topological vector space is locally convex if and only if its topology is generated by a set of seminorms.
\end{prop}

\begin{proof}
\pf\ See K\"{o}the, G. Topological Vector Spaces 1. Section 18. \qed
\end{proof}

\begin{prop}
A locally convex topological vector space is a pre-Fr\'{e}chet space if and only if it is metrizable.
\end{prop}

\begin{proof}
\pf\ See K\"{o}the, G. Topological Vector Spaces 1. Section 18. \qed
\end{proof}

\begin{ex}
Let $E$ be an infinite dimensional Hilbert space. Let $E'$ be the same vector space under the \emph{weak topology}, the coarsest topology such that every continuous linear map $E \rightarrow \mathbb{R}$ is continuous as a map $E' \rightarrow \mathbb{R}$. Then $E$ is locally convex Hausdorff but not metrizable.

Proof: See Dieudonne, J. A., Treatise on Analysis, Vol. II, New York and London: Academic Press, 1970, p. 76.
\end{ex}

\begin{df}[Thom Space]
Let $E$ be a vector bundle with a Riemannian metric, $DE = \{ x : \El{E} \mid \| x \| \leq 1 \}$ its disc bundle and $SE := \{ v : \El{E} \mid \| v \| = 1 \}$ its sphere bundle. The \emph{Thom space} of $E$ is the quotient space $DE / SE$.
\end{df}

\end{document}