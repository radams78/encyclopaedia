% READ
% READ
% CHECK
% FORMALIZE
% READ
% READ
% CHECK
 
\documentclass{book}

\title{Mathematics}
\author{Robin Adams}

\usepackage{amsmath}
\usepackage{amssymb}
\usepackage{amsthm}
\let\proof\relax
\let\endproof\relax
\let\qed\relax
\usepackage{pf2}
\usepackage{hyperref}
\usepackage{mathabx}
\usepackage[all]{xy}

\newtheorem{ax}{Axiom}[section]
\newtheorem{axs}[ax]{Axiom Schema}
\newtheorem{prop}[ax]{Proposition}
\newtheorem{cor}{Corollary}[ax]
\newtheorem{thm}[ax]{Theorem}
\newtheorem{thms}[ax]{Theorem Schema}
\newtheorem{lm}[ax]{Lemma}
\theoremstyle{definition}
\newtheorem{df}[ax]{Definition}
\newtheorem{ex}[ax]{Example}

\newcommand{\id}[1]{\ensuremath{\mathrm{id}_{#1}}}
\newcommand{\inv}[1]{\ensuremath{{#1}^{-1}}}
\newcommand{\Ob}[1]{\ensuremath{\mathrm{Ob} \left( {#1} \right)}}
\newcommand{\Set}{\ensuremath{\mathbf{Set}}}
\newcommand{\op}[1]{\ensuremath{{#1}^{\mathrm{op}}}}
\newcommand{\spec}{\ensuremath{\operatorname{spec}}}
\newcommand{\Top}{\ensuremath{\mathbf{Top}}}
\newcommand{\dom}{\ensuremath{\operatorname{dom}}}
\newcommand{\spn}{\ensuremath{\operatorname{span}}}
\newcommand{\diam}{\ensuremath{\operatorname{diam}}}

\begin{document}

\maketitle
\tableofcontents

\chapter{Primitive Terms and Axioms}

\section{Primitive Terms} % CHECKED FORMALIZED

Let there be \emph{sets}.

For any set $A$, let there be \emph{elements} of $A$. We write $a \in A$ for: $a$ is an element of $A$.

For any sets $A$ and $B$, let there be a set $B^A$, whose elements are called \emph{functions} from $A$ to $B$. We write $f : A \rightarrow B$ for $f \in B^A$.

For any function $f : A \rightarrow B$ and element $a \in A$, let there be an element $f(a) \in B$, the \emph{value} of the function $f$ at the \emph{argument} $a$.

\section{Injections, Surjections and Bijections} % CHECKED FORMALIZED

\begin{df}[Injective]
A function $f : A \rightarrow B$ is \emph{injective} or an \emph{injection} iff, for all $x,y \in A$, if $f(x) = f(y)$ then $x = y$.
\end{df}

\begin{df}[Surjective]
A function $f : A \rightarrow B$ is \emph{surjective} or a \emph{surjection} iff, for all $y \in B$, there exists $x \in A$ such that $f(x) = y$.
\end{df}

\begin{df}[Bijective]
A function $f : A \rightarrow B$ is \emph{bijective} or a \emph{bijection} iff it is injective and surjective.

Sets $A$ and $B$ are \emph{equinumerous}, $A \approx B$, iff there exists a bijection between them.
\end{df}

If we prove there exists a set $X$ such that $P(X)$, and that any two sets that satisfy $P$ are bijective, then we may introduce a constant $C$ and define "Let $C$ be the set such that $P(C)$".

\section{Axioms} % CHECKED FORMALIZED

\begin{axs}[Choice]
Let $P[X,Y,x,y]$ be a formula where $X$ and $Y$ are set variables, $x \in X$ and $y \in Y$. Then the following is an axiom.

Let $A$ and $B$ be sets. Assume that, for all $a \in A$, there exists $b \in B$ such that $P[A,B,a,b]$. Then there exists a function $f : A \rightarrow B$ such that $\forall a \in A. P[A,B,a,f(a)]$.
\end{axs}

\begin{ax}[Extensionality]
Let $f, g : A \rightarrow B$. If, for all $x \in A$, we have $f(x) = g(x)$, then $f = g$.
\end{ax}

\begin{df}[Composition]
Let $f : A \rightarrow B$ and $g : B \rightarrow C$. The \emph{composite} $g \circ f : A \rightarrow C$ is the function such that, for all $a \in A$, we have
\[ (g \circ f)(a) = g(f(a)) \enspace . \]
\end{df}

\begin{ax}[Pairing]
For any sets $A$ and $B$, there exists a set $A \times B$, the \emph{Cartesian product} of $A$ and $B$, and functions $\pi_1 : A \times B \rightarrow A$ and $\pi_2 : A \times B \rightarrow B$ such that, for all $a \in A$ and $b \in B$, there exists a unique $(a,b) \in A \times B$ such that $\pi_1(a,b) = a$ and $\pi_2(a,b) = b$.
\end{ax}

\begin{axs}[Separation]
For every property $P[X,x]$ where $X$ is a set variable and $x \in X$, the following is an axiom:

For every set $A$, there exists a set $S = \{ x \in A : P[A,x] \}$ and an injection $i : S \rightarrow A$ such that, for all $x \in A$, we have
\[ (\exists y \in S. i(y) = x) \Leftrightarrow P[A,x] \enspace . \]
\end{axs}

\begin{ax}[Infinity]
There exists a set $\mathbb{N}$, an element $0 \in \mathbb{N}$, and a function $s : \mathbb{N} \rightarrow \mathbb{N}$ such that:
\begin{itemize}
\item $\forall n \in \mathbb{N}. s(n) \neq 0$
\item $\forall m,n \in \mathbb{N}. s(m) = s(n) \Rightarrow m = n$.
\end{itemize}
\end{ax}

\begin{axs}[Collection]
Let $P[X,Y,x]$ be a formula with set variables $X$ and $Y$ and an element variable $x \in X$. Then the following is an axiom.

For any set $A$, there exist sets $B$ and $Y$ and functions $p : B \rightarrow A$, and $m : B \times Y \Rightarrow \mathbb{N}$ such that:
\begin{itemize}
\item $m$ is injective.
\item $\forall b \in B. P[A, \{ y \in Y : m(b,y) = 0 \}, p(b)]$
\item For all $a \in A$, if $\exists Y. P[A,Y,a]$, then there exists $b \in B$ such that $a = p(b)$.
\end{itemize}
\end{axs}

\begin{ax}[Universe]
There exists a set $E$, a set $U$ and a function $el : E \rightarrow U$ such that the following holds.

Let us say that a set $A$ is \emph{small} iff there exists $u \in U$ such that $A \approx \{ e \in E : el(e) = u \}$.

\begin{itemize}
\item $\mathbb{N}$ is small.
\item For any $U$-small sets $A$ and $B$, the set $B^A$ is small.
\item For any $U$-small sets $A$ and $B$, the set $A \times B$ is small.
\item Let $f : A \rightarrow B$ be a function. If $B$ is small and $\{ a \in A : f(a) = b \}$ is small for all $b \in B$, then $A$ is small.
\item If $p : B \twoheadrightarrow A$ is a surjective function such that $A$ is small, then there exists a $U$-small set $C$, a surjection $q : C \twoheadrightarrow A$, and a function $f : C \rightarrow B$ such that $q = p \circ f$.
\end{itemize}
\end{ax}

\chapter{Sets and Functions}

\section{Composition} % CHECKED

\begin{prop}
Given functions $f : A \rightarrow B$, $g : B \rightarrow C$ and $h : C \rightarrow D$, we have
\[ h \circ (g \circ f) = (h \circ g) \circ f \enspace . \]
\end{prop}

\begin{proof}
\pf
\step{1}{For all $x \in A$ we have $(h \circ (g \circ f))(x) = ((h \circ g) \circ f)(x)$.}
\begin{proof}
	\step{a}{\pflet{$x \in A$}}
	\step{b}{$(h \circ (g \circ f))(x) = ((h \circ g) \circ f)(x)$}
	\begin{proof}
		\pf
		\begin{align*}
			(h \circ (g \circ f))(x)
			& = h ((g \circ f)(x)) & (\text{Definition of composition}) \\
			& = h(g(f(x))) & (\text{Definition of composition}) \\
			& = (h \circ g)(f(x)) & (\text{Definition of composition}) \\
			& = ((h \circ g) \circ f)(x) & (\text{Definition of composition})
		\end{align*}
	\end{proof}
\end{proof}
\qedstep
\begin{proof}
	\pf\ By the Axiom of Extensionality.
\end{proof}
\qed
\end{proof}

\subsection{Injections} % CHECKED FORMALIZED

\begin{prop}
\label{prop:inj_comp}
The composite of injective functions is injective.
\end{prop}

\begin{proof}
\pf
\step{1}{\pflet{$A$, $B$ and $C$ be sets.}}
\step{2}{\pflet{$f : A \rightarrow B$}}
\step{3}{\pflet{$g : B \rightarrow C$}}
\step{4}{\assume{$g$ is injective.}}
\step{5}{\assume{$f$ is injective.}}
\step{6}{\pflet{$x,y \in A$}}
\step{7}{\assume{$(g \circ f)(x) = (g \circ f)(y)$} \prove{$x = y$}}
\step{8}{$g(f(x)) = g(f(y))$}
\begin{proof}
	\pf
	\begin{align*}
		g(f(x)) & = (g \circ f)(x) & (\text{definition of composition}) \\
		& = (g \circ f)(y) & (\text{\stepref{7}}) \\
		& = g(f(y))& (\text{definition of composition})
	\end{align*}
\end{proof}
\step{9}{$f(x) = f(y)$}
\begin{proof}
	\pf\ \stepref{4}, \stepref{8}
\end{proof}
\step{10}{$x = y$}
\begin{proof}
	\pf\ \stepref{5}, \stepref{9}
\end{proof}
\qed
\end{proof}

\begin{prop}
For functions $f : A \rightarrow B$ and $g : B \rightarrow C$, if $g \circ f$ is injective then $f$ is injective.
\end{prop}

\begin{proof}
\pf
\step{1}{\pflet{$A$, $B$ and $C$ be sets.}}
\step{2}{\pflet{$f : A \rightarrow B$}}
\step{3}{\pflet{$g : B \rightarrow C$}}
\step{4}{\assume{$g \circ f$ is injective.}}
\step{5}{\pflet{$x,y \in A$}}
\step{6}{\assume{$f(x) = f(y)$}}
\step{7}{$(g \circ f)(x) = (g \circ f)(y)$}
\begin{proof}
	\pf
	\begin{align*}
		(g \circ f)(x) & = g(f(x)) & (\text{definition of composition}) \\
		& = g(f(y)) & (\text{\stepref{6}}) \\
		& = (g \circ f)(y) & (\text{definition of composition})
	\end{align*}
\end{proof}
\step{8}{$x = y$}
\begin{proof}
	\pf\ \stepref{4}, \stepref{7}
\end{proof}
\qed
\end{proof}

\begin{prop}
\label{prop:injective}
Let $f : A \rightarrow B$ be injective. For every set $X$ and functions $x,y : X \rightarrow A$, if $f \circ x = f \circ y$ then $x = y$.
\end{prop}

\begin{proof}
\pf
	\step{a}{\assume{$f$ is injective.}}
	\step{b}{\pflet{$X$ be a set.}}
	\step{c}{\pflet{$x,y : X \rightarrow A$}}
	\step{d}{\assume{$f \circ x = f \circ y$}}
	\step{e}{$\forall t \in X. x(t) = y(t)$}
	\begin{proof}
		\step{i}{\pflet{$t \in X$}}
		\step{ii}{$f(x(t)) = f(y(t))$}
		\begin{proof}
			\pf
			\begin{align*}
				f(x(t)) & = (f \circ x)(t) & (\text{definition of composition}) \\
				& = (f \circ y)(t) & (\text{\stepref{d}}) \\
				& = f(y(t)) & (\text{definition of composition})
			\end{align*}
		\end{proof}
		\step{iii}{$x(t) = y(t)$}
		\begin{proof}
			\pf\ \stepref{a}, \stepref{ii}
		\end{proof}
	\end{proof}
	\step{f}{$x = y$}
	\begin{proof}
		\pf\ Axiom of Extensionality, \stepref{e}
	\end{proof}
\qed
\end{proof}

We will prove the converse as Proposition \ref{prop:injection2}.

\subsection{Surjections} % CHECKED

\begin{prop}
\label{prop:surj_comp}
The composite of surjective functions is surjective.
\end{prop}

\begin{proof}
\pf
\step{1}{\pflet{$A$, $B$ and $C$ be sets.}}
\step{2}{\pflet{$f : A \rightarrow B$ and $g : B \rightarrow C$}}
\step{3}{\assume{$g$ is surjective.}}
\step{4}{\assume{$f$ is surjective.}}
\step{5}{\pflet{$c \in C$}}
\step{6}{\pick\ $b \in B$ such that $g(b) = c$.}
\begin{proof}
	\pf\ \stepref{3}
\end{proof}
\step{7}{\pick\ $a \in A$ such that $f(a) = b$.}
\begin{proof}
	\pf\ \stepref{4}
\end{proof}
\step{8}{$(g \circ f)(a) = c$}
\begin{proof}
	\pf
	\begin{align*}
		(g \circ f)(a) & = g(f(a)) & (\text{definition of composition}) \\
		& = g(b) & (\text{\stepref{7}}) \\
		& = c & (\text{\stepref{6}})
	\end{align*}
\end{proof}
\qed
\end{proof}

\begin{prop}
Let $f : A \rightarrow B$ and $g : B \rightarrow C$. If $g \circ f$ is surjective then $g$ is surjective.
\end{prop}

\begin{proof}
\pf
\step{1}{\pflet{$A$, $B$ and $C$ be sets.}}
\step{2}{\pflet{$f : A \rightarrow B$ and $g : B \rightarrow C$.}}
\step{3}{\assume{$g \circ f$ is surjective.}}
\step{4}{\pflet{$c \in C$}}
\step{5}{\pick\ $a \in A$ such that $(g \circ f)(a) = c$}
\begin{proof}
	\pf\ \stepref{3}
\end{proof}
\step{6}{$g(f(a)) = c$}
\begin{proof}
	\pf\ From \stepref{5} and the definition of composition.
\end{proof}
\qedstep
\begin{proof}
	\pf\ There exists $b \in B$ such that $g(b) = c$, namely $b = f(a)$.
\end{proof}
\qed
\end{proof}

\begin{prop}
Let $f : A \rightarrow B$ be a surjection. For any set $X$ and functions $x,y : B \rightarrow X$, if $x \circ f = y \circ f$ then $x = y$.
\end{prop}

\begin{proof}
\pf
\step{1}{\pflet{$b \in B$}}
\step{2}{\pick\ $a \in A$ such that $f(a) = b$}
\step{3}{$x(f(a)) = y(f(a))$}
\step{4}{$x(b) = y(b)$}
\qedstep
\begin{proof}
	\pf\ Axiom of Extensionality.
\end{proof}
\qed
\end{proof}

We will prove the converse as Proposition \ref{prop:surjective}.

\subsection{Bijections}

\begin{prop}
The composite of bijections is a bijection.
\end{prop}

\begin{proof}
\pf
\step{1}{\pflet{$A$, $B$ and $C$ be sets.}}
\step{2}{\pflet{$f : A \rightarrow B$ and $g : B \rightarrow C$}}
\step{3}{\assume{$g$ is bijective.}}
\step{4}{\assume{$f$ is bijective.}}
\step{5}{$g$ is injective.}
\begin{proof}
	\pf\ From \stepref{3}.
\end{proof}
\step{6}{$g$ is surjective.}
\begin{proof}
	\pf\ From \stepref{3}.
\end{proof}
\step{7}{$f$ is injective.}
\begin{proof}
	\pf\ From \stepref{4}.
\end{proof}
\step{8}{$f$ is surjective.}
\begin{proof}
	\pf\ From \stepref{4}.
\end{proof}
\step{9}{$g \circ f$ is injective.}
\begin{proof}
	\pf\ Proposition \ref{prop:inj_comp}, \stepref{5}, \stepref{7}.
\end{proof}
\step{10}{$g \circ f$ is surjective.}
\begin{proof}
	\pf\ Proposition \ref{prop:surj_comp}, \stepref{6}, \stepref{8}.
\end{proof}
\step{11}{$g \circ f$ is bijective.}
\begin{proof}
	\pf\ \stepref{9}, \stepref{10}
\end{proof}
\qed
\end{proof}

\subsection{Equinumerosity}

\begin{prop}
\[ (A \times B)^C \approx A^C \times B^C \]
\end{prop}

\begin{proof}
\pf\ The function that maps $f$ to $(\pi_1 \circ f, \pi_2 \circ f)$ is a bijection. \qed
\end{proof}

\begin{prop}
\[ A^{B \times C} \approx (A^B)^C \]
\end{prop}

\begin{proof}
\pf\ The function $\Phi$ such that $\Phi(f)(c)(b) = f(b,c)$ is a bijection. \qed
\end{proof}

\section{Domination}

\begin{df}[Dominate]
Let $A$ and $B$ be sets. We say that $B$ \emph{dominates} $A$, and write $A \preccurlyeq B$, iff there exists an injective function $A \rightarrow B$.
\end{df}

\begin{thm}[Schroeder-Bernstein]
Let $A$ and $B$ be sets. If $A \preccurlyeq B$ and $B \preccurlyeq A$ then $A \approx B$.
\end{thm}

\begin{proof}
\pf
\step{1}{\pflet{$f : A \rightarrowtail B$ and $g : B \rightarrowtail A$ be injections.}}
\step{2}{Define the subsets $A_n$ of $A$ by
\begin{align*}
A_0 & := A - g(B) \\
A_{n+1} & := g(f(A_n))
\end{align*}}
\step{3}{Define $h : A \rightarrow B$ by
\[ h(x) = \begin{cases}
f(x) & \text{if } \exists n. x \in A_n \\
\inv{g}(x) & \text{otherwise}
\end{cases} \]}
\step{4}{$h$ is injective.}
\begin{proof}
	\step{a}{\pflet{$x,y \in A$}}
	\step{b}{\assume{$h(x) = h(y)$}}
	\step{c}{\case{$x \in A_m$ and $y \in A_n$.}}
	\begin{proof}
		\pf\ Then $f(x) = f(y)$ so $x = y$ since $f$ is injective.
	\end{proof}
	\step{d}{\case{$x \in A_m$ and there is no $y$ such that $y \in A_n$.}}
	\begin{proof}
		\step{i}{$f(x) = \inv{g}(y)$}
		\step{ii}{$y = g(f(x))$}
		\step{iii}{$y \in A_{m+1}$}
		\qedstep
		\begin{proof}
			\pf\ This is a contradiction.
		\end{proof}
	\end{proof}
	\step{e}{\case{$y \in A_n$ and there is no $m$ such that $x \in A_m$.}}
	\begin{proof}
		\pf\ Similar.
	\end{proof}
	\step{f}{\case{There is no $m$ such that $x \in A_m$ and there is no $n$ such that $y \in A_n$.}}
	\begin{proof}
		\pf\ Then $\inv{g}(x) = \inv{g}(y)$ and so $x = y$.
	\end{proof}
\end{proof}
\step{5}{$h$ is surjective.}
\begin{proof}
	\step{a}{\pflet{$y \in B$}}
	\step{b}{\case{$g(y) \in A_n$}}
	\begin{proof}
		\step{i}{$n \neq 0$}
		\step{ii}{\pick\ $x \in A_{n-1}$ such that $g(y) = g(f(x))$}
		\step{iii}{$y = f(x)$}
		\step{iv}{$y = h(x)$}
	\end{proof}
	\step{c}{\case{There is no $n$ such that $g(y) \in A_n$.}}
	\begin{proof}
		\pf\ Then $h(g(y)) = y$.
	\end{proof}
\end{proof}
\qed
\end{proof}

\section{Identity Function} % CHECKED

\begin{df}[Identity]
For any set $A$, the \emph{identity} function $\id{A} : A \rightarrow A$ is the function defined by $\id{A}(a) = a$.
\end{df}

\subsection{Injections, Surjections, Bijections} % CHECKED

\begin{prop}
For any set $A$, the identity function $\id{A}$ is a bijection.
\end{prop}

\begin{proof}
\pf
\step{1}{\pflet{$A$ be a set.}}
\step{2}{$\id{A}$ is injective.}
\begin{proof}
	\pf\ If $\id{A}(x) = \id{A}(y)$ then $x = y$.
\end{proof}
\step{3}{$\id{A}$ is surjective.}
\begin{proof}
	\pf\ For any $y \in A$, there exists $x \in A$ such that $\id{A}(x) = y$, namely $x = y$.
\end{proof}
\qed
\end{proof}

\subsection{Composition} % CHECKED

\begin{prop}
Let $f : A \rightarrow B$. Then $\id{B} \circ f = f = f \circ \id{A}$.
\end{prop}

\begin{proof}
\pf\ Each is the function that maps $a$ to $f(a)$. \qed
\end{proof}

\begin{prop}
Let $f : A \rightarrow B$.
\begin{enumerate}
\item If there exists $g : B \rightarrow A$ such that $g \circ f = \id{A}$ then $f$ is injective.
\item If $f$ is injective and $A$ is nonempty, then there exists $g : B \rightarrow A$ such that $g \circ f = \id{A}$.
\end{enumerate}
\end{prop}

\begin{proof}
\pf
\step{1}{If there exists $g : B \rightarrow A$ such that $g \circ f = \id{A}$ then $f$ is injective.}
\begin{proof}
	\pf\ If $f(x) = f(y)$ then $x = g(f(x)) = g(f(y)) = y$.
\end{proof}
\step{2}{If $f$ is injective and $A$ is nonempty, then there exists $g : B \rightarrow A$ such that $g \circ f = \id{A}$.}
\begin{proof}
	\step{a}{\assume{$f$ is injective and $A$ is nonempty.}}
	\step{b}{\pick\ $a \in A$}
	\step{c}{Choose a function $g : B \rightarrow A$ such that $f(g(x)) = x$ if there exists $y \in A$ such that $f(y) = x$, otherwise $g(x) = a$.}
	\step{d}{\pflet{$x \in A$} \prove{$g(f(x)) = x$}}
	\step{e}{$f(g(f(x))) = f(x)$}
	\step{f}{$g(f(x)) = x$}
\end{proof}
\qed
\end{proof}

\begin{prop}
Let $f : A \rightarrow B$. Then $f$ is surjective if and only if there exists $g : B \rightarrow A$ such that $f \circ g = \id{B}$.
\end{prop}

\begin{proof}
\pf
\step{2}{If $f$ is surjective then there exists $g : B \rightarrow A$ such that $f \circ g = \id{B}$.}
\begin{proof}
	\step{a}{\assume{$f$ is surjective.}}
	\step{b}{\pick\ $g : B \rightarrow A$ such that, for all $b \in B$, we have $f(g(b)) = b$.}
	\begin{proof}
		\pf\ Axiom of Choice.
	\end{proof}
	\step{c}{$f \circ g = \id{B}$.}
\end{proof}
\step{3}{If there exists $g : B \rightarrow A$ such that $f \circ g = \id{B}$ then $f$ is surjective.}
\begin{proof}
	\step{a}{\pflet{$g : B \rightarrow A$ such that $f \circ g = \id{B}$}}
	\step{b}{\pflet{$X$ be a set.}}
	\step{c}{\pflet{$h,k : B \rightarrow X$}}
	\step{d}{\assume{$h \circ f = k \circ f$}}
	\step{e}{$h = k$}
	\begin{proof}
		\pf\ $h = h \circ f \circ g = k \circ f \circ g = k$
	\end{proof}
\end{proof}
\qed
\end{proof}

\begin{cor}
Let $A$ and $B$ be sets.
\begin{enumerate}
\item If there exists a surjective function $A \rightarrow B$ then there exists an injective function $B \rightarrow A$.
\item If there exists an injective function $A \rightarrow B$ and $A$ is nonempty then there exists a surjective function $B \rightarrow A$.
\end{enumerate}
\end{cor}

\begin{prop}
Let $f : A \rightarrow B$. Then $f$ is bijective if and only if there exists a function $\inv{f} : B \rightarrow A$, the \emph{inverse} of $f$, such that $f \circ \inv{f} = \id{B}$ and $\inv{f} \circ f = \id{A}$, in which case the inverse is unique.
\end{prop}

\begin{proof}
\pf
\step{1}{If $f$ is bijective then there exists $\inv{f} : B \rightarrow A$ such that $f \circ \inv{f} = \id{B}$ and $\inv{f} \circ f = \id{A}$.}
\begin{proof}
	\step{a}{\assume{$f$ is bijective.}}
	\step{b}{\pick\ $g : B \rightarrow A$ such that $f \circ g = \id{B}$}
	\begin{proof}
		\pf\ Proposition \ref{prop:surjective}.
	\end{proof}
	\step{c}{$f \circ g \circ f = f$}
	\step{d}{$g \circ f = \id{A}$}
	\begin{proof}
		\pf\ Proposition \ref{prop:injective}.
	\end{proof}
\end{proof}
\step{2}{If there exists $\inv{f} : B \rightarrow A$ such that $f \circ \inv{f} = \id{B}$ and $\inv{f} \circ f = \id{A}$, then $f$ is bijective.}
\begin{proof}
	\step{a}{\pflet{$\inv{f} : B \rightarrow A$ satisfy $f \circ \inv{f} = \id{B}$ and $\inv{f} \circ f = \id{A}$}}
	\step{b}{$f$ is injective.}
	\begin{proof}
		\pf\ If $f(x) = f(y)$ then $x = \inv{f}(f(x)) = \inv{f}(f(y)) = y$.
	\end{proof}
	\step{c}{$f$ is surjective.}
	\begin{proof}
		\pf\ Proposition \ref{prop:surjective}.
	\end{proof}
\end{proof}
\step{3}{If $g, h : B \rightarrow A$ satisfy $f \circ g = \id{B}$ and $g \circ f = \id{A}$ and $f \circ h = \id{B}$ and $h \circ f = \id{A}$ then $g = h$.}
\begin{proof}
	\pf\ We have $g = g \circ f \circ h = h$.
\end{proof}
\qed
\end{proof}

\section{The Empty Set} % CHECKED

\begin{thm}
There exists a set which has no elements.
\end{thm}

\begin{proof}
\pf\ Take $\{ x \in \mathbb{N} : \bot \}$. \qed
\end{proof}

\begin{thm}
If $E$ and $E'$ have no elements then $E \approx E'$.
\end{thm}

\begin{proof}
\pf
\step{1}{\pflet{$E$ and $E'$ have no elements.}}
\step{2}{\pick\ a function $F : E \rightarrow E'$.}
\begin{proof}
	\pf\ Axiom of Choice since vacuously $\forall x \in E. \exists y \in E'. \top$.
\end{proof}
\step{4}{$F$ is injective.}
\begin{proof}
	\pf\ Vacuously, for all $x,y \in E$, if $F(x) = F(y)$ then $x = y$.
\end{proof}
\step{5}{$F$ is surjective.}
\begin{proof}
	\pf\ Vacuously, for all $y \in E$, there exists $x \in E$ such that $F(x) = y$.
\end{proof}
\qed
\end{proof}

\begin{df}[Empty Set]
The \emph{empty set} $\emptyset$ is the set with no elements.
\end{df}

\section{The Singleton} % CHECKED

\begin{thm}
There exists a set that has exactly one element.
\end{thm}

\begin{proof}
\pf\ The set $\{ x \in \mathbb{N} : x = 0 \}$ has exactly one element. \qed
\end{proof}

\begin{thm}
If $A$ and $B$ both have exactly one element then $A \approx B$.
\end{thm}

\begin{proof}
\pf
\step{1}{\pflet{$A$ and $B$ both have exactly one element $a$ and $b$ respectively.}}
\step{2}{\pflet{$F : A \rightarrow B$ be the function such that, for all $x \in A$, we have $(x = a \wedge F(x) = b)$}}
\step{3}{$F$ is a bijection.}
\qed
\end{proof}

\begin{df}[Singleton]
Let 1 be the set that has exactly one element. Let $*$ be its element.
\end{df}

\subsection{Injections}

\begin{prop}
\label{prop:injection2}
Let $f : A \rightarrow B$. Assume that, for every set $X$ and functions $x,y : X \rightarrow A$, if $f \circ x = f \circ y$ then $x = y$. Then $f$ is injective.
\end{prop}

\begin{proof}
\pf\ Take $X = 1$. \qed
\end{proof}

\section{The Set Two} % CHECKED

\begin{df}[The Set Two]
Let $2 = \{ x \in \mathbb{N} : x = 0 \vee x = 1 \}$.
\end{df}

\begin{prop}
\label{prop:surjective}
Let $f : A \rightarrow B$. Assume that, for any set $X$ and functions $g,h : B \rightarrow X$, if $g \circ f = h \circ f$ then $g = h$. Then $f$ is surjective.
\end{prop}

\begin{proof}
\pf
	\step{a}{\assume{For any set $X$ and functions $g,h : B \rightarrow X$, if $g \circ f = h \circ f$ then $g = h$.}}
	\step{b}{\pflet{$b \in B$}}
	\step{c}{\pflet{$h : B \rightarrow 2$ be the function that maps everything to 1.}}
	\step{d}{\pflet{$k : B \rightarrow 2$ be the function that maps $b$ to 0 and everything else to 1.}}
	\step{e}{$h \neq k$}
	\step{f}{$h \circ f \neq k \circ f$}
	\step{g}{\pick\ $a \in A$ such that $h(f(a)) \neq k(f(a))$}
	\step{h}{$f(a) = b$}
\qed
\end{proof}

\section{Subsets} % CHECKED

\begin{df}[Subset]
A \emph{subset} of a set $A$ consists of a set $S$ and an injection $i : S \rightarrowtail A$. We write $(S,i) \subseteq A$.

We say two subsets $(S,i)$ and $(T,j)$ are \emph{equal}, $(S,i) = (T,j)$, iff there exists a bijection $\phi : S \approx T$ such that $j \circ \phi = i$.
\end{df}

\begin{prop}
For any subset $(S,i)$ of $A$ we have $(S,i) = (S,i)$.
\end{prop}

\begin{proof}
\pf\ We have $\id{S} : S \approx S$ and $i \circ \id{S} = i$.
\end{proof}

\begin{prop}
If $(S,i) = (T,j)$ then $(T,j) = (S,i)$.
\end{prop}

\begin{proof}
\pf\ If $\phi : S \approx T$ and $j \circ \phi = i$ then $\inv{\phi} : T \approx S$ and $i \circ \inv{\phi} = j$. \qed
\end{proof}

\begin{prop}
If $(R,i) = (S,j)$ and $(S,j) = (T,k)$ then $(R,i) = (T,k)$.
\end{prop}

\begin{proof}
\pf\ If $\phi : R \approx S$ and $j \circ \phi = i$, and $\psi : S \approx T$ and $k \circ \psi = j$, then $\psi \circ \phi : R \approx T$ and $k \circ \psi \circ \phi = i$. \qed
\end{proof}

\begin{df}[Membership]
Given $(S,i) \subseteq A$ and $a \in A$, we write $a \in (S,i)$ for $\exists s \in S. i(s) = a$.
\end{df}

\begin{prop}
If $a \in (S,i)$ and $(S,i) = (T,j)$ then $a \in (T,j)$.
\end{prop}

\begin{proof}
\pf\ If $i(s) = a$ then $j(\phi(s)) = a$. \qed
\end{proof}

\begin{df}[Union]
Given subsets $S$ and $T$ of $A$, the \emph{union} is the subset $\{ x \in A : x \in S \vee x \in T \}$.
\end{df}

\begin{df}[Intersection]
Given subsets $S$ and $T$ of $A$, the \emph{intersection} is the subset $\{ x \in A : x \in S \wedge x \in T \}$.
\end{df}

\begin{prop}[Distributive Law]
\[ R \cap (S \cup T) = (R \cap S) \cup (R \cap T) \]
\end{prop}

\begin{prop}[Distributive Law]
\[ R \cup (S \cap T) = (R \cup S) \cap (R \cup T) \]
\end{prop}

\begin{df}
Given a set $A$, we write $\emptyset$ for the subset $(\emptyset, !)$ where $!$ is the unique function $\emptyset \rightarrow A$.
\end{df}

\begin{prop}
\[ S \cup \emptyset = S \]
\end{prop}

\begin{prop}
\[ S \cap \emptyset = S \]
\end{prop}

\begin{df}[Inclusion]
Given subsets $(S,i)$ and $(T,j)$ of a set $A$, we write $(S,i) \subseteq (T,j)$ iff there exists $f : S \rightarrow T$ such that $j \circ f = i$.
\end{df}

\begin{prop}
\[ \emptyset \subseteq S \]
\end{prop}

\begin{df}[Disjoint]
Subsets $S$ and $T$ of $A$ are \emph{disjoint} iff $S \cap T = \emptyset$.
\end{df}

\begin{df}[Difference]
Given subsets $S$ and $T$ of $A$, the \emph{difference} of $S$ and $T$ is $S - T = \{ x \in A : x \in S \wedge x \notin T \}$.
\end{df}

\begin{prop}[De Morgan's Law]
\[ R - (S \cup T) = (R - S) \cap (R - T) \]
\end{prop}

\begin{prop}[De Morgan's Law]
\[ R - (S \cap T) = (R - S) \cup (R - T) \]
\end{prop}

\section{Power Set}

\begin{df}[Power Set]
The \emph{power set} of a set $A$ is
\[ \mathcal{P} A := 2^A \]
\end{df}

\begin{df}[Cover]
Let $X$ be a set and $\mathcal{A} \subseteq \mathcal{P} X$. Then $\mathcal{A}$ is a \emph{cover} of $X$, or \emph{covers} $X$, iff $\bigcup \mathcal{A} = X$.

Given a subset $Y$ of $X$ and $\mathcal{A} \subseteq \mathcal{P} X$, we say $\mathcal{A}$ \emph{covers} $Y$ iff $Y \subseteq \bigcup \mathcal{A}$.
\end{df}

\section{Saturated Set}

\begin{df}[Saturated]
Let $A$ and $B$ be sets. Let $f : A \rightarrow B$ be surjective. Let $C \subseteq A$. Then $C$ is \emph{saturated} with respect to $f$ iff, for all $x \in C$ and $y \in A$, if $f(x) = f(y)$ then $y \in C$.
\end{df}


\section{Union} % CHECKED

\begin{df}[Union]
Given $\mathcal{A} \in \mathcal{P} \mathcal{P} X$, its \emph{union} is
\[ \bigcup \mathcal{A} := \{ x \in X : \exists S \in \mathcal{A}. x \in S \} \in \mathcal{P} X \enspace . \]
\end{df}

\subsection{Intersection} % CHECKED

\begin{df}[Intersection]
Given $\mathcal{A} \in \mathcal{P} \mathcal{P} X$, its \emph{intersection} is
\[ \bigcap \mathcal{A} := \{ x \in X : \forall S \in \mathcal{A}. x \in S \} \in \mathcal{P} X \enspace . \]
\end{df}

\subsection{Direct Image} % CHECKED

\begin{df}[Direct Image]
Let $f : A \rightarrow B$. Let $S$ be a subset of $A$. The \emph{(direct) image} of $S$ under $f$ is the subset of $B$ given by
\[ f(S) := \{ f(a) : a \in S \} \enspace . \]
\end{df}

\begin{prop}
$ $
\begin{enumerate}
\item If $S \subseteq T$ then $f(S) \subseteq f(T)$
\item $f(\bigcup \mathcal{S}) = \bigcup_{S \in \mathcal{S}} f(S)$
\end{enumerate}
\end{prop}

\begin{ex}
It is not true in general that $f(\bigcap \mathcal{S}) = \bigcap_{S \in \mathcal{S}} f(S)$. Take $f$ to be the only function $\{0,1\} \rightarrow \{0\}$, and $\mathcal{S} = \{\{0\},\{1\}\}$. Then $f(\bigcap \mathcal{S}) = \emptyset$ but $\bigcap_{S \in \mathcal{S}} f(S) = \{0\}$.
\end{ex}

\begin{ex}
It is not true in general that $f(S - T) = f(S) - f(T)$. Take $f$ to be the only function $\{0,1\} \rightarrow \{0\}$, $S = \{0\}$ and $T = \{1\}$. Then $f(S-T) = \{0\}$ but $f(S) - f(T) = \emptyset$.
\end{ex}

\section{Inverse Image} % CHECKED

\begin{df}[Inverse Image]
Let $f : A \rightarrow B$. Let $S$ be a subset of $B$. The \emph{inverse image} or \emph{preimage} of $S$ under $f$ is the subset of $A$ given by
\[ \inv{f}(S) := \{ x \in A : f(x) \in S \} \enspace . \]
\end{df}

\begin{prop}
\begin{enumerate}
\item If $S \subseteq T$ then $\inv{f}(S) \subseteq \inv{f}(T)$
\item $\inv{f}(\bigcup \mathcal{S}) = \bigcup_{S \in \mathcal{S}} \inv{f}(S)$
\item $\inv{f}(\bigcap \mathcal{S}) = \bigcap_{S \in \mathcal{S}} \inv{f}(S)$
\item $\inv{f}(S - T) = \inv{f}(S) - \inv{f}(T)$
\item $S \subseteq \inv{f}(f(S))$. Equality holds if $f$ is injective.
\item $f(\inv{f}(T)) \subseteq T$. Equality holds if $f$ is surjective.
\item $\inv{(g \circ f)}(S) = \inv{f}(\inv{g}(S))$
\end{enumerate}
\end{prop}

\subsection{Saturated Sets}

\begin{prop}
Let $A$ and $B$ be sets. Let $f : A \rightarrow B$ be surjective. Let $C \subseteq A$. Then $C$ is saturated if and only if there exists $D \subseteq B$ such that $C = \inv{f}(D)$.
\end{prop}

\begin{proof}
\pf
\step{1}{If $C$ is saturated then there exists $D \subseteq B$ such that $C = \inv{f}(D)$.}
\begin{proof}
	\step{a}{\assume{$C$ is saturated.}}
	\step{b}{\pflet{$D = f(C)$}}
	\step{c}{$C \subseteq \inv{f}(D)$}
	\begin{proof}
		\step{i}{\pflet{$x \in C$}}
		\step{ii}{$f(x) \in D$}
		\begin{proof}
			\pf\ \stepref{b}
		\end{proof}
		\step{iii}{$x \in \inv{f}(D)$}
	\end{proof}
	\step{d}{$\inv{f}(D) \subseteq C$}
	\begin{proof}
		\step{i}{\pflet{$x \in \inv{f}(D)$}}
		\step{ii}{$f(x) \in D$}
		\step{iii}{\pick\ $y \in C$ such that $f(x) = f(y)$}
		\begin{proof}
			\pf\ \stepref{b}
		\end{proof}
		\step{iv}{$x \in C$}
		\begin{proof}
			\pf\ \stepref{a}
		\end{proof}
	\end{proof}
\end{proof}
\step{2}{If there exists $D \subseteq B$ such that $C = \inv{f}(D)$ then $C$ is saturated.}
\begin{proof}
	\step{a}{\pflet{$D \subseteq B$ be such that $C = \inv{f}(D)$.}}
	\step{b}{\pflet{$x \in C$ and $y \in A$}}
	\step{c}{\assume{$f(x) = f(y)$}}
	\step{d}{$f(x) \in D$}
	\step{e}{$f(y) \in D$}
	\step{f}{$y \in C$}
\end{proof}
\qed
\end{proof}

\section{Relations} % CHECKED

\begin{df}[Relation]
Let $A$ and $B$ be sets. A \emph{relation} $R$ between $A$ and $B$, $R : A \looparrowright B$, is a subset of $A \times B$.

Given $a \in A$ and $b \in B$, we write $aRb$ for $(a,b) \in R$.

A relation \emph{on} a set $A$ is a relation between $A$ and $A$.
\end{df}

\begin{df}[Reflexive]
A relation $R$ on a set $A$ is \emph{reflexive} iff $\forall a \in A. aRa$.
\end{df}

\begin{df}[Symmetric]
A relation $R$ on a set $A$ is \emph{symmetric} iff, whenever $xRy$, then $yRx$.
\end{df}

\begin{df}[Transitive]
A relation $R$ on a set $A$ is \emph{transitive} iff, whenever $xRy$ and $yRz$, then $xRz$.
\end{df}

\subsection{Equivalence Relations}

\begin{df}[Equivalence Relation]
A relation $R$ on a set $A$ is an \emph{equivalence relation} iff it is reflexive, symmetric and transitive.
\end{df}

\begin{df}[Equivalence Class]
Let $R$ be an equivalence relation on a set $A$ and $a \in A$. The \emph{equivalence class} of $a$ with respect to $R$ is
\[ \{ x \in A : xRa \} \enspace . \]
\end{df}

\begin{prop}
Two equivalence classes are either disjoint or equal.
\end{prop}


\section{Power Set}

\begin{df}[Power Set]
The \emph{power set} of a set $A$ is $\mathcal{P} A := 2^A$.

Given $S \in \mathcal{P} A$ and $a \in A$, we write $a \in A$ for $S(a) = 1$.
\end{df}

\begin{df}[Pairwise Disjoint]
Let $P \subseteq \mathcal{P} A$. We say the members of $P$ are \emph{pairwise disjoint} iff, for all $S,T \in P$, if $S \neq T$ then $S \cap T = \emptyset$.
\end{df}

\subsection{Partitions}

\begin{df}[Partition]
Let $A$ be a set. A \emph{partition} of $A$ is a set $P \in \mathcal{P} \mathcal{P} A$ such that:
\begin{itemize}
\item $\bigcup P = A$
\item Every member of $P$ is nonempty.
\item The members of $P$ are pairwise disjoint.
\end{itemize}
\end{df}

\section{Cartesian Product}

\begin{df}[Cartesian Product]
Let $A$ and $B$ be sets. The \emph{Cartesian product} of $A$ and $B$, $A \times B$, is the tabulation of the relation $A \looparrowright B$ that holds for all $a \in A$ and $b \in B$. The associated functions $\pi_1 : A \times B \rightarrow A$ and $\pi_2 : A \times B \rightarrow B$ are called the \emph{projections}.

Given $a \in A$ and $b \in B$, we write $(a,b)$ for the unique element of $A \times B$ such that $\pi_1(a,b) = a$ and $\pi_2(a,b) = b$.
\end{df}

\section{Quotient Sets}

\begin{prop}
Let $\sim$ be an equivalence relation on $X$. Then there exists a set $X/\sim$, the \emph{quotient set} of $X$ with respect to $\sim$, and a surjective function $\pi : X \twoheadrightarrow X / \sim$, the \emph{canonical projection}, such that, for all $x,y \in X$, we have $x \sim y$ if and only if $\pi(x) = \pi(y)$.

Further, if $p : X \twoheadrightarrow Q$ is another quotient with respect to $\sim$, then there exists a unique bijection $\phi : X / \sim \approx Q$ such that $\phi \circ \pi = p$.
\end{prop}

\section{Partitions}

\begin{df}[Partition]
A \emph{partition} of a set $X$ is a set of pairwise disjoint subsets of $X$ whose union is $X$.
\end{df}

\section{Disjoint Union}

\begin{thm}
For any sets $A$ and $B$, there exists a set $A + B$, the \emph{disjoint union} of $A$ and $B$, and functions $\kappa_1 : A \rightarrow A + B$ and $\kappa_2 : B \rightarrow A + B$, the \emph{injections}, such that, for every set $X$ and functions $f : A \rightarrow X$ and $g : B \rightarrow X$, there exists a unique function $[f,g] : A + B \rightarrow X$ such that $[f,g] \circ \kappa_1 = f$ and $[f,g] \circ \kappa_2 = g$.
\end{thm}

\begin{proof}
\pf
\step{1}{\pflet{$A + B := \{ p \in \mathcal{P} A \times \mathcal{P} B : \exists a \in A. p = (\{a\},\emptyset) \vee \exists b \in B. p = (\emptyset, \{b\}) \}$}}
% TODO
\end{proof}

\begin{df}[Restriction]
Let $f : A \rightarrow B$ and let $(S,i)$ be a subset of $A$. The \emph{restriction} of $f$ to $S$ is the function $f \restriction S : S \rightarrow B$ defined by $f \restriction S = f \circ i$.
\end{df}

\section{Natural Numbers}

\begin{thm}[Principle of Recursive Definition]
Let $A$ be a set. Let $F$ be the set of all functions $\{ m \in \mathbb{N} : m < n \} \rightarrow A$ for some $n$. Let $\rho : F \rightarrow A$. Then there exists a unique $g : \mathbb{N} \rightarrow A$ such that, for all $n \in \mathbb{N}$, we have
\[ g(n) = \rho(g \restriction \{ m \in \mathbb{N} : m < n \}) \enspace . \]
\end{thm}

\begin{proof}
\pf
\step{1}{Given a subset $B \subseteq \mathbb{N}$, let us say that a function $g : B \rightarrow A$ is \emph{acceptable} iff, for all $n \in B$, we have
\[ \forall m < n. m \in B \]
and
\[ g(n) = \rho(g \restriction \{ m \in \mathbb{N} : m < n \}) \enspace . \]}
\step{2}{For all $n \in \mathbb{N}$, there exists an acceptable function $\{ m \in \mathbb{N} : m < n \} \rightarrow A$.}
\begin{proof}
	\step{a}{\pflet{$P[n]$ be the property: There exists an acceptable function $\{ m \in \mathbb{N} : m < n \} \rightarrow A$.}}
	\step{b}{$P[0]$}
	\begin{proof}
		\pf\ The unique function $\emptyset \rightarrow A$ is acceptable.
	\end{proof}
	\step{c}{For any natural number $n$, if $P[n]$ then $P[n+1]$.}
	\begin{proof}
		\step{i}{\assume{$P[n]$}}
		\step{ii}{\pick\ an acceptable $f : \{ m \in \mathbb{N} : m < n \} \rightarrow A$.}
		\step{iii}{\pflet{$g : \{ m \in \mathbb{N} : m < n + 1 \} \rightarrow A$ be the function
		\[ g(m) = \begin{cases}
		f(m) & \text{if } m < n \\
		\rho(f) & \text{if } m = n
		\end{cases} \]}}
		\step{iv}{$g$ is acceptable.}
	\end{proof}
\end{proof}
\step{3}{If $g : B \rightarrow A$ and $h : C \rightarrow A$ are acceptable, then $g$ and $h$ agree on $B \cap C$.}
\step{4}{Define $g : \mathbb{N} \rightarrow A$ by: $g(n) = a$ iff there exists an acceptable $h : \{ m \in \mathbb{N} : m < n + 1 \}$ such that $h(n) = a$.}
\step{5}{$g$ is acceptable.}
\step{6}{If $g' : \mathbb{N} \rightarrow A$ is acceptable then $g' = g$.}
\qed
\end{proof}

\section{Finite and Infinite Sets}

\begin{df}[Finite]
A set $A$ is \emph{finite} iff there exists $n \in \mathbb{N}$ such that $A \approx \{ m \in \mathbb{N} : m < n \}$. In this case, we say $A$ has \emph{cardinality} $n$.
\end{df}

\begin{prop}
Let $n \in \mathbb{N}$. Let $A$ be a set. Let $a_0 \in A$. Then $A \approx \{ m \in \mathbb{N} : m < n + 1 \}$ if and only if $A - \{a_0\} \approx \{ m \in \mathbb{N} : m < n \}$.
\end{prop}

\begin{thm}
Let $A$ be a set. Suppose that $A \approx \{ m \in \mathbb{N} : m < n \}$. Let $B$ be a proper subset of $A$. Then $B \not\approx \{ m \in \mathbb{N} : m < n \}$ but there exists $m < n$ such that $B \approx \{ k \in \mathbb{N} : k < m \}$.
\end{thm}

\begin{proof}
\pf
\step{1}{\pflet{$P[n]$ be the property: for every set $A$, if $A approx \{ m \in \mathbb{N} : m < n \}$, then for every proper subset $B$ of $A$, we have $B \not\approx \{ m \in \mathbb{N} : m < n \}$ but there exists $m < n$ such that $B \approx \{ k \in \mathbb{N} : k < m \}$.}}
\step{2}{$P[0]$}
\begin{proof}
	\pf\ If $A \approx \{ m \in \mathbb{N} : m < 0 \}$ then $A$ is empty and so has no proper subset.
\end{proof}
\step{3}{For every natural number $n$, if $P[n]$ then $P[n+1]$.}
\begin{proof}
	\step{a}{\pflet{$n$ be a natural number.}}
	\step{b}{\assume{$P[n]$}}
	\step{c}{\pflet{$A$ be a set.}}
	\step{d}{\assume{$A \approx \{ m \in \mathbb{N} : m < n + 1 \}$}}
	\step{e}{\pflet{$B$ be a proper subset of $A$.}}
	\step{f}{\case{$B = \emptyset$}}
	\begin{proof}
		\pf\ Then $B \not\approx \{ m \in \mathbb{N} : m < n + 1\}$ but $B \approx \{ k \in \mathbb{N} : k < 0 \}$.
	\end{proof}
	\step{g}{\case{$B \neq \emptyset$}}
	\begin{proof}
		\step{i}{\pick\ $b_0 \in B$}
		\step{ii}{$A - \{b_0\} \approx \{ m \in \mathbb{N} : m < n \}$}
		\step{iii}{$B - \{b_0\}$ is a proper subset of $A - \{b_0\}$}
		\step{iv}{$B - \{b_0\} \not\approx \{m \in \mathbb{N} : m < n \}$}
		\step{v}{$B \approx \{m \in \mathbb{N} : m < n + 1\}$}
		\step{vi}{\pick\ $m < n$ such that $B - \{b_0\} \approx \{ k \in \mathbb{N} : k < m \}$}
		\step{vii}{$m + 1 < n + 1$}
		\step{viii}{$B \approx \{ k \in \mathbb{N} : k < m + 1 \}$}
	\end{proof}
\end{proof}
\qed
\end{proof}

\begin{cor}
\label{cor:Dedekind_infinite}
If $A$ is finite then there is no bijection between $A$ and a proper subset of $A$.
\end{cor}

\begin{cor}
$\mathbb{N}$ is infinite.
\end{cor}

\begin{cor}
The cardinality of a finite set is unique.
\end{cor}

\begin{cor}
A subset of a finite set is finite.
\end{cor}

\begin{cor}
If $A$ is finite and $B$ is a proper subset of $A$ then $|B| < |A|$.
\end{cor}

\begin{cor}
Let $A$ be a set. Then the following are equivalent:
\begin{enumerate}
\item $A$ is finite.
\item There exists a surjection from an initial segment of $\mathbb{N}$ onto $A$.
\item There exists an injection from $A$ to an initial segment of $\mathbb{N}$.
\end{enumerate}
\end{cor}

\begin{cor}
A finite union of finite sets is finite.
\end{cor}

\begin{cor}
A finite Cartesian product of finite sets is finite.
\end{cor}

\begin{thm}
Let $A$ be a set. The following are equivalent:
\begin{enumerate}
\item There exists an injective function $\mathbb{N} \rightarrowtail A$.
\item There exists a bijection between $A$ and a proper subset of $A$.
\item $A$ is infinite.
\end{enumerate}
\end{thm}

\begin{proof}
\pf
\step{1}{$1 \Rightarrow 2$}
\begin{proof}
	\step{a}{\pflet{$f : \mathbb{N} \rightarrowtail A$ be injective.}}
	\step{b}{\pflet{$s : \mathbb{N} \approx \mathbb{N} - \{0\}$ be the function $s(n) = n + 1$.}}
	\step{c}{$f \circ s \circ \inv{f} : A \approx A - \{f(0)\}$}
\end{proof}
\step{2}{$2 \Rightarrow 3$}
\begin{proof}
	\pf\ Corollary \ref{cor:Dedekind_infinite}.
\end{proof}
\step{3}{$3 \Rightarrow 1$}
\begin{proof}
	\pf\ Choose a function $f : \mathbb{N} \rightarrow A$ such that $f(n) \in A - \{ f(m) : m < n \}$ for all $n$.
\end{proof}
\qed
\end{proof}

\section{Countable Sets}

\begin{df}[Countable]
A set $A$ is \emph{countably infinite} iff $A \approx \mathbb{N}$.
\end{df}

\begin{prop}
$\mathbb{N} \times \mathbb{N}$ is countably infinite.
\end{prop}

\begin{proof}
\pf\ Define $f : \mathbb{N} \times \mathbb{N} \approx \{ (x,y) \in \mathbb{N} \times \mathbb{N} : y \leq x \}$ by
\[ f(x,y) = (x+y,y) \]
Define $g : \{ (x,y) \in \mathbb{N} \times \mathbb{N} : y \leq x \} \approx \mathbb{N}$ by
\[ g(x,y) = x(x-1)/2 + y \enspace . \qed \]
\end{proof}

\begin{prop}
Every infinite subset of $\mathbb{N}$ is countably infinite.
\end{prop}

\begin{proof}
\pf
\step{1}{\pflet{$C$ be an infinite subset of $\mathbb{N}$}}
\step{2}{Define $h : \mathbb{Z} \rightarrow C$ by recursion thus: $h(n)$ is the smallest element of $C - \{ h(m) : m < n \}$.}
\step{3}{$h$ is injective.}
\begin{proof}
	\pf\ If $m < n$ then $h(m) \neq h(n)$ because $h(n) \in C - \{ h(m) : m < n \}$.
\end{proof}
\step{4}{$h$ is surjective.}
\begin{proof}
	\step{a}{For all $n \in \mathbb{N}$ we have $n \leq h(n)$.}
	\step{b}{\pflet{$c \in C$}}
	\step{c}{$c \leq h(c)$}
	\step{d}{\pflet{$n$ be least such that $c \leq h(n)$}}
	\step{e}{$c \in C - \{ h(m) : m < n \}$}
	\step{f}{$h(n) \leq c$}
	\step{g}{$h(n) = c$}
\end{proof}
\qed
\end{proof}

\begin{df}[Countable]
A set is \emph{countable} iff it is either finite or countably infinite; otherwise it is \emph{uncountable}.
\end{df}

\begin{prop}
Let $B$ be a nonempty set. Then the following are equivalent.
\begin{enumerate}
\item $B$ is countable.
\item There exists a surjection $\mathbb{N} \twoheadrightarrow B$.
\item There exists an injection $B \rightarrowtail \mathbb{N}$.
\end{enumerate}
\end{prop}

\begin{proof}
\pf
\step{1}{$1 \Rightarrow 2$}
\begin{proof}
	\step{a}{\assume{$B$ is countable.}}
	\step{b}{\case{$B$ is finite.}}
	\begin{proof}
		\step{i}{\pick\ a natural number $n$ and bijection $f : \{ m \in \mathbb{N} : m < n \} \approx B$}
		\step{ii}{\pick\ $b \in B$}
		\step{iii}{Extend $f$ to a surjection $g : \mathbb{N} \twoheadrightarrow B$ by setting $g(m) = b$ for $m \geq n$.}
	\end{proof}
	\step{c}{\case{$B$ is countably infinite.}}
	\begin{proof}
		\pf\ Then there exists a bijection $\mathbb{N} \approx B$.
	\end{proof}
\end{proof}
\step{2}{$2 \Rightarrow 3$}
\begin{proof}
	\pf\ Given a surjection $f : \mathbb{N} \twoheadrightarrow B$, define $g : B \rightarrowtail \mathbb{N}$ by $g(b)$ is the smallest number such that $f(g(b)) = b$.
\end{proof}
\step{3}{$3 \Rightarrow 1$}
\begin{proof}
	\step{a}{\pflet{$f : B \rightarrowtail \mathbb{N}$ be injective.}}
	\step{b}{$f(B)$ is countable.}
	\step{c}{$B \approx f(B)$}
	\step{d}{$B$ is countable.}
\end{proof}
\qed
\end{proof}

\begin{cor}
A subset of a countable set is countable.
\end{cor}

\begin{cor}
$\mathbb{N} \times \mathbb{N}$ is countably infinite.
\end{cor}

\begin{proof}
\pf\ The function that maps $(m,n)$ to $2^m 3^n$ is injective. \qed
\end{proof}

\begin{cor}
The Cartesian product of two countable sets is countable.
\end{cor}

\begin{thm}
A countable union of countable sets is countable.
\end{thm}

\begin{proof}
\pf
\step{1}{\pflet{$A$ be a set.}}
\step{2}{\pflet{$\mathcal{B} \subseteq \mathcal{P} A$ be a countable set of countable sets such that $\bigcup \mathcal{B} = A$}}
\step{3}{\pick\ a surjection $B : \mathbb{N} \twoheadrightarrow \mathcal{B}$}
\step{4}{\assume{w.l.o.g. each $B(n)$ is nonempty.}}
\step{5}{For $n \in \mathbb{N}$, \pick\ a surjective function $g_n : \mathbb{N} \twoheadrightarrow B(n)$}
\step{6}{\pflet{$h : \mathbb{N} \times \mathbb{N} \rightarrow A$ be the function $h(m,n) = g_m(n)$}}
\step{7}{$h$ is surjective.}
\qed
\end{proof}

\begin{thm}
$2^\mathbb{N}$ is uncountable.
\end{thm}

\begin{proof}
\pf
\step{1}{\pflet{$f : \mathbb{N} \rightarrow 2^\mathbb{N}$} \prove{$f$ is not surjective.}}
\step{2}{Define $g : \mathbb{N} \rightarrow 2$ by $g(n) = 1 - f(n)(n)$.}
\step{3}{For all $n \in \mathbb{N}$ we have $g(n) \neq f(n)(n)$.}
\step{4}{For all $n \in \mathbb{N}$ we have $g \neq f(n)$.}
\qed
\end{proof}

\begin{thm}
For any set $A$, there is no surjective function $A \rightarrow \mathcal{P} A$.
\end{thm}

\begin{proof}
\pf
\step{1}{\pflet{$f : A \rightarrow \mathcal{P} A$}}
\step{2}{\pflet{$S = \{ x \in A : x \notin f(x) \}$}}
\step{3}{For all $a \in A$ we have $S \neq f(a)$}
\begin{proof}
	\pf\ We have $a \in S$ if and only if $a \notin f(a)$.
\end{proof}
\qed
\end{proof}

\begin{cor}
For any set $A$, there is no injective function $\mathcal{P} A \rightarrow A$.
\end{cor}

\section{Fixed Points}

\begin{df}[Fixed Point]
Let $A$ be a set and $f : A \rightarrow A$.
A \emph{fixed point} of $f$ is an element $a \in A$ such that $f(a) = a$.
\end{df}

\section{Finite Intersection Property}

\begin{df}[Finite Intersection Property]
Let $X$ be a set. Let $\mathcal{C} \subseteq \mathcal{P} X$. Then $\mathcal{C}$ has the \emph{finite intersection property} iff every finite nonempty subset of $\mathcal{C}$ has nonempty intersection.
\end{df}

\chapter{Relations}

\begin{df}[Reflexive]
A relation $R \subseteq A \times A$ is \emph{reflexive} iff, for all $a \in A$, we have $(a,a) \in R$.
\end{df}

\begin{df}[Antisymmetric]
A relation $R \subseteq A \times A$ is \emph{antisymmetric} iff, for all $a,b \in A$, if $(a,b) \in R$ and $(b,a) \in R$ then $a = b$.
\end{df}

\begin{df}[Transitive]
A relation $R \subseteq A \times A$ is \emph{transitive} iff, for all $a,b,c \in A$, if $(a,b) \in R$ and $(b,c) \in R$ then $(a,c) \in R$.
\end{df}

\begin{df}[Partial Order]
A \emph{partial order} on a set $A$ is a relation on $A$ that is reflexive, antisymmetric and transitive.

We say $(A, \leq)$ is a \emph{partially ordered set} or \emph{poset} iff $\leq$ is a partial order on $A$.
\end{df}

\begin{df}[Greatest]
Let $A$ be a poset and $a \in A$. Then $a$ is the \emph{greatest} element iff $\forall x \in A. x \leq a$.
\end{df}

\begin{df}[Least]
Let $A$ be a poset and $a \in A$. Then $a$ is the \emph{least} element iff $\forall x \in A. a \leq x$.
\end{df}

\begin{df}[Upper Bound]
Let $A$ be a poset, $S \subseteq A$, and $u \in A$. Then $u$ is an \emph{upper bound} for $S$ iff $\forall x \in S. x \leq u$. We say $S$ is \emph{bounded above} iff it has an upper bound.
\end{df}

\begin{df}[Lower Bound]
Let $A$ be a poset, $S \subseteq A$, and $l \in A$. Then $l$ is a \emph{lower bound} for $S$ iff $\forall x \in S. l \leq x$. We say $S$ is \emph{bounded below} iff it has a lower bound.
\end{df}

\begin{df}[Supremum]
Let $A$ be a poset, $S \subseteq A$ and $s \in A$. Then $s$ is the \emph{supremum} or \emph{least upper bound} for $S$ iff $s$ is the least element in the sub-poset of upper bounds for $A$.
\end{df}

\begin{df}[Supremum]
Let $A$ be a poset, $S \subseteq A$ and $i \in A$. Then $i$ is the \emph{infimum} or \emph{greatest lower bound} for $S$ iff $i$ is the greatest element in the sub-poset of lower bounds for $A$.
\end{df}

\begin{df}[Least Upper Bound Property]
A poset $A$ has the \emph{least upper bound property} iff every nonempty subset of $A$ that is bounded above has a least upper bound.
\end{df}

\begin{prop}
Let $A$ be a poset. Then $A$ has the least upper bound property if and only if every nonempty subset of $A$ that is bounded below has a greatest lower bound.
\end{prop}

\begin{proof}
\pf
\step{1}{If $A$ has the least upper bound property then every subset of $A$ that is bounded below has a greatest lower bound.}
\begin{proof}
	\step{a}{\assume{$A$ has the least upper bound property.}}
	\step{b}{\pflet{$S \subseteq A$ be nonempty and bounded below.}}
	\step{c}{\pflet{$L$ be the set of lower bounds of $S$.}}
	\step{d}{$L$ is nonempty.}
	\begin{proof}
		\pf\ Because $S$ is bounded below.
	\end{proof}
	\step{e}{$L$ is bounded above.}
	\begin{proof}
		\pf\ Pick an element $s \in S$. Then $s$ is an upper bound for $L$.
	\end{proof}
	\step{f}{\pflet{$s$ be the supremum of $L$.}}
	\step{g}{$s$ is the greatest lower bound of $S$.}
	\begin{proof}
		\step{i}{$s$ is a lower bound of $S$.}
		\begin{proof}
			\step{one}{\pflet{$x \in S$}}
			\step{two}{$x$ is an upper bound for $L$.}
			\step{three}{$s \leq x$}
		\end{proof}
		\step{ii}{For any lower bound $l$ of $S$ we have $l \leq s$.}
		\begin{proof}
			\pf\ Immediate from \stepref{f}.
		\end{proof}
	\end{proof}
\end{proof}
\step{2}{If every subset of $A$ that is bounded below has a greatest lower bound, then $A$ has the least upper bound property.}
\begin{proof}
	\pf\ Dual.
\end{proof}
\qed
\end{proof}

\chapter{Order Theory}

\section{Strict Partial Orders}

\begin{df}[Strict Partial Order]
A \emph{strict partial order} on a set $A$ is a relation on $A$ that is irreflexive and transitive.
\end{df}

\begin{prop}
\begin{enumerate}
\item If $\leq$ is a partial order on $A$ then $<$ is a strict partial order on $A$, where $x < y$ iff $x \leq y \wedge x \neq y$.
\item If $<$ is a strict partial order on $A$ then $\leq$ is a partial order on $A$, where $x \leq y$ iff $x < y \vee x = y$.
\item These two relations are inverses of one another.
\end{enumerate}
\end{prop}

\subsection{Linear Orders}

\begin{df}[Linear Order]
A \emph{linear order} on a set $A$ is a partial order $\leq$ on $A$ such that, for all $x,y \in A$, we have $x \leq y$ or $y \leq x$.

A \emph{linearly ordered set} is a pair $(X, \leq)$ such that $X$ is a set and $\leq$ is a linear order on $X$.
\end{df}

\begin{df}[Open Interval]
Let $X$ be a linearly ordered set and $a,b \in X$. The \emph{open interval} $(a,b)$ is the set
\[ \{ x \in X : a < x < b \} \enspace . \]
\end{df}

\begin{df}[Immediate Predecessor, Immediate Successor]
Let $X$ be a linearly ordered set and $a,b \in X$. Then $b$ is the \emph{(immediate) successor} of $a$, and $a$ is the \emph{(immediate) predecessor} of $b$, iff $a < B$ and there is no $x$ such that $a < x < b$.
\end{df}

\begin{df}[Dictionary Order]
Let $A$ and $B$ be linearly ordered sets. The \emph{dictionary order} on $A \times B$ is the order defined by
\[ (a,b) < (a',b') \Leftrightarrow a < a' \vee (a = a' \wedge b < b') \enspace . \]
\end{df}

\begin{thm}[Maximum Principle]
Every poset has a maximal linearly ordered subset.
\end{thm}

\begin{proof}
\pf
\step{1}{\pflet{$(A, \leq)$ be a poset.}}
\step{2}{\pick\ a well ordering $\preccurlyeq$ of $A$.}
\begin{proof}
	\pf\ Well Ordering Theorem.
\end{proof}
\step{3}{\pflet{$h : A \rightarrow 2$ be the function defined by $\preccurlyeq$-recursion thus:
\[ h(a) =
\begin{cases}
1 & \text{if $a$ is $\leq$-comparable with every $b \prec a$ such that $h(b) = 1$} \\
0 & \text{otherwise}
\end{cases} \]}}
\step{4}{\pflet{$B = \{x \in A : h(x) = 1\}$} \prove{$B$ is a maximal subset linearly ordered by $\leq$.}}
\step{5}{$B$ is linearly ordered by $\leq$.}
\begin{proof}
	\step{a}{\pflet{$x,y \in B$}}
	\step{b}{\assume{w.l.o.g. $x \preccurlyeq y$}}
	\step{c}{$y$ is $\leq$-comparable with $x$}
\end{proof}
\step{6}{For any subset $C \subseteq A$ linearly ordered by $\leq$, if $B \subseteq C$ then $B = C$.}
\begin{proof}
	\step{a}{\pflet{$x \in C$}}
	\step{b}{$x$ is comparable with every $y \preccurlyeq x$ such that $h(x) = 1$}
	\step{c}{$x \in B$}
\end{proof}
\qed
\end{proof}

\begin{thm}[Zorn's Lemma]
Let $A$ be a poset.
If every linearly ordered subset of $A$ is bounded above, then $A$ has a maximal element.
\end{thm}

\begin{proof}
\pf
\step{1}{\pick\ a maximal linearly ordered subset $B$ of $A$.}
\begin{proof}
	\pf\ Maximal Principle
\end{proof}
\step{2}{\pick\ an upper bound $c$ for $B$. \prove{$c$ is maximal.}}
\step{3}{\pflet{$x \in A$}}
\step{4}{\assume{$c \leq x$} \prove{$x = c$}}
\step{5}{$x$ is an upper bound for $B$.}
\step{6}{$x \in B$}
\begin{proof}
	\pf\ By the maximality of $B$, since $B \cup \{x\}$ is linearly ordered.
\end{proof}
\step{7}{$x \leq c$}
\begin{proof}
	\pf\ \stepref{2}
\end{proof}
\step{8}{$x = c$}
\qed
\end{proof}

\begin{cor}[Kuratowski's Lemma]
Let $\mathcal{A} \subseteq \mathcal{P} X$. Suppose that, for every subset $\mathcal{B} \subseteq \mathcal{A}$ that is linearly ordered by inclusion, we have $\bigcup \mathcal{B} \in \mathcal{A}$. Then $\mathcal{A}$ has a maximal element.
\end{cor}

\begin{df}[Closed Interval]
Let $X$ be a linearly ordered set. Let $a,b \in X$ with $a < b$. The \emph{closed interval} $[a,b]$ is
\[ [a,b] := \{ x \in X : a \leq x \leq b \} \enspace . \]
\end{df}

\begin{df}[Half-Open Interval]
Let $X$ be a linearly ordered set. Let $a,b \in X$ with $a < b$. The \emph{half-open intervals} $(a,b]$ and $[a,b)$ are defined by
\begin{align*}
(a,b] & := \{ x \in X : a < x \leq b \} \\
[a,b) & := \{ x \in X : a \leq x < b \}
\end{align*}
\end{df}

\begin{df}[Open Ray]
Let $X$ be a linearly ordered set and $a \in X$. The \emph{open rays} $(a, +\infty)$ and $(-\infty, a)$ are defined by:
\begin{align*}
(a, +\infty) & := \{ x \in X : a < x \} \\
(-\infty, a) & := \{ x \in X : x < a \}
\end{align*}
\end{df}

\begin{df}[Closed Ray]
Let $X$ be a linearly ordered set and $a \in X$. The \emph{closed rays} $[a, +\infty)$ and $(-\infty, a]$ are defined by:
\begin{align*}
[a, +\infty) & := \{ x \in X : a \leq x \} \\
(-\infty, a] & := \{ x \in X : x \leq a \}
\end{align*}
\end{df}

\begin{df}[Convex]
Let $X$ be a linearly ordered set and $Y \subseteq X$. Then $Y$ is \emph{convex} iff, for all $a,b \in Y$ and $c \in X$, if $a < c < b$ then $c \in Y$.
\end{df}

\subsection{Sets of Finite Type}

\begin{df}[Finite Type]
Let $X$ be a set. Let $\mathcal{A} \subseteq \mathcal{P} X$. Then $\mathcal{A}$ is of \emph{finite type} if and only if, for any $B \subseteq X$, we have $B \in \mathcal{A}$ if and only if every finite subset of $B$ is in $\mathcal{A}$.
\end{df}

\begin{prop}[Tukey's Lemma]
Let $X$ be a set. Let $\mathcal{A} \subseteq \mathcal{P} X$. If $\mathcal{A}$ is of finite type, then $\mathcal{A}$ has a maximal element.
\end{prop}

\begin{proof}
\pf
\step{1}{For every subset $\mathcal{B} \subseteq \mathcal{A}$ that is linearly ordered by inclusion, we have $\bigcup \mathcal{B} \in \mathcal{A}$.}
\begin{proof}
	\step{a}{\pflet{$\mathcal{B} \subseteq \mathcal{A}$}}
	\step{b}{\assume{$\mathcal{B}$ is linearly ordered by inclusion.}}
	\step{c}{Every finite subset of $\bigcup \mathcal{B}$ is in $\mathcal{A}$}
	\step{d}{$\bigcup \mathcal{B} \in \mathcal{A}$}
\end{proof}
\qedstep
\begin{proof}
	\pf\ Kuratowski's Lemma.
\end{proof}
\qed
\end{proof}

\section{Linear Continuua}

\begin{df}[Linear Continuum]
A \emph{linear continuum} is a linearly ordered set with more than one element that is dense and has the least upper bound property.
\end{df}

\begin{prop}
Every convex subset of a linear continuum with more than one element is a linear continuum.
\end{prop}

\begin{proof}
\pf\ Easy. \qed
\end{proof}

\begin{cor}
Every interval and ray in a linear continuum is a linear continuum.
\end{cor}

\section{Well Orders}

\begin{df}[Well Ordered Set]
A \emph{well ordered set} is a linearly ordered set such that every nonempty subset has a least element.
\end{df}

\begin{prop}
Any subset of a well ordered set is well ordered.
\end{prop}

\begin{prop}
The product of two well ordered sets is well ordered under the dictionary order.
\end{prop}

\begin{thm}[Well Ordering Theorem]
Every set has a well ordering.
\end{thm}

\begin{proof}
\pf
\step{1}{\pflet{$X$ be a set.}}
\step{2}{\pick\ a choice function $c : \mathcal{P} X - \{ \emptyset \} \rightarrow X$}
\step{3}{Define a \emph{tower} to be a pair $(T,<)$ where $T \subseteq X$, $<$ is a well ordering of $T$, and
\[ \forall x \in T. x = c(X - \{ y \in T : y < x \}) \enspace . \]}
\step{4}{Given two towers, either they are equal or one is a section of the other.}
\begin{proof}
	\step{a}{\pflet{$(T_1,<_1)$ and $(T_2,<_2)$ be towers.}}
	\step{b}{\assume{w.l.o.g. there exists a strictly monotone function $h : T_1 \rightarrow T_2$}}
	\step{c}{$h(T_1)$ is either $T_2$ or a section of $T_2$}
	\begin{proof}
		\pf\ Proposition \ref{prop:order_preserving_maps}.
	\end{proof}
	\step{d}{$\forall x \in T_1. h(x) = x$}
	\begin{proof}
		\step{i}{\pflet{$x \in T_1$}}
		\step{ii}{\assume{as transfinite induction hypothesis $\forall y < x. h(y) = y$}}
		\step{iii}{$h(x)$ is the least element of $T_2 - \{ h(y) \in T_1 : y < x \}$}
		\step{iv}{$h(x)$ is the least element of $T_2 - \{ y \in T_1 : y < x \}$}
		\begin{proof}
			\pf\ \stepref{ii}
		\end{proof}
		\step{v}{$h(x) = x$}
		\begin{proof}
			\pf
			\begin{align*}
				h(x) & = c(X - \{y \in T_2 : y < h(x)\}) & (\text{\stepref{3}}) \\
				& = c(X - \{y \in T_2 : y < x \}) & (\text{\stepref{iv}}) \\
				& = c(X - \{ y \in T_1 : y < x \}) & (\text{\stepref{ii}}) \\
				& = x & (\text{\stepref{3}})
			\end{align*}
		\end{proof}
	\end{proof}
\end{proof}
\step{5}{If $(T,<)$ is a tower and $T \neq X$, then there exists a tower of which $(T,<)$ is a section.}
\begin{proof}
	\pf\ Let $T_1 = T \cup \{c(T)\}$ and $<_1$ be the extension of $<$ such that $x < c(T)$ for all $x \in T$.
\end{proof}
\step{6}{\pflet{$\mathbf{T} = \bigcup \{ T : \exists R. (T,R) \text{ is a tower} \}$ and $\mathbf{R} = \bigcup \{ R : \exists T. (T,R) \text{ is a tower} \}$}}
\step{7}{$(\mathbf{T},\mathbf{R})$ is a tower.}
\begin{proof}
	\step{a}{$\mathbf{R}$ is irreflexive.}
	\begin{proof}
		\pf\ Since for every tower $(T,<)$ we have $<$ is irreflexive.
	\end{proof}
	\step{b}{$\mathbf{R}$ is transitive.}
	\begin{proof}
		\step{i}{\assume{$x \mathbf{R} y$ and $y \mathbf{R} z$}}
		\step{ii}{\pick\ towers $(T_1,<_1)$ and $(T_2,<_2)$ such that $x <_1 y$ and $y <_2 z$}
		\step{iii}{\assume{w.l.o.g. $(T_1,<_1)$ is either $(T_2,<_2)$ or a section of $(T_2,<_2)$}}
		\step{iv}{$x <_2 y <_2 z$}
		\step{v}{$x <_2 z$}
		\step{vi}{$x \mathbf{R} z$}
	\end{proof}
	\step{c}{For all $x,y \in \mathbf{T}$, either $x\mathbf{R}y$ or $x= y$ or $y\mathbf{R}x$}
	\begin{proof}
		\pf\ There exists a tower that has both $x$ and $y$.
	\end{proof}
	\step{d}{Every nonempty subset of $\mathbf{T}$ has an $\mathbf{R}$-least element.}
	\begin{proof}
		\step{i}{\pflet{$A \subseteq \mathbf{T}$ be nonempty.}}
		\step{ii}{\pick\ $a \in A$}
		\step{iii}{\pick\ a tower $(T,<)$ such that $a \in T$.}
		\step{iv}{\pflet{$b$ be the $<$-least element of $A \cap T$} \prove{$b$ is $\mathbf{R}$-least in $A$.}}
		\step{vi}{\pflet{$x \in A$}}
		\step{vii}{Etc.}
	\end{proof}
	\step{e}{$\forall x \in \mathbf{T}. x = c(X - \{y \in \mathbf{T} : y\mathbf{R}x \})$}
\end{proof}
\step{8}{$\mathbf{T} = X$}
\step{9}{$\mathbf{R}$ is a well ordering of $X$.}
\qed
\end{proof}

\begin{prop}
There exists a well-ordered set with a largest element $\Omega$ such that $(-\infty, \Omega)$ is uncountable but, for all $\alpha < \Omega$, we have $(-\infty, \alpha)$ is countable.
\end{prop}

\begin{proof}
\pf
\step{1}{\pick\ an uncountable well ordered set $B$.}
\step{2}{\pflet{$C = 2 \times B$ under the dictionary order.}}
\step{3}{\pflet{$\Omega$ be the least element of $C$ such that $(-\infty, \Omega)$ is uncountable.}}
\step{4}{\pflet{$A = (-\infty, \Omega]$}}
\step{5}{$A$ is a well ordered set with largest element $\Omega$ such that $(-\infty, \Omega)$ is uncountable but, for all $\alpha < \Omega$, we have $(-\infty, \alpha)$ is countable.}
\qed
\end{proof}

\begin{prop}
Every well ordered set has the least upper bound property.
\end{prop}

\begin{proof}
\pf\ For any subset that is bounded above, the set of upper bounds is nonempty, hence has a least element. \qed
\end{proof}

\begin{prop}
In a well ordered set, every element that is not greatest has a successor.
\end{prop}

\begin{proof}
\pf\ If $a$ is not greatest, then $\{ x : x > a \}$ is nonempty, hence has a least element. \qed
\end{proof}

\begin{thm}[Transfinite Induction]
Let $J$ be a well ordered set. Let $S \subseteq J$. Assume that, for every $\alpha \in J$, if $\forall x < \alpha. x \in S$ then $\alpha in S$. Then $S = J$.
\end{thm}

\begin{proof}
\pf\ Otherwise $J - S$ would be a nonempty subset of $J$ with no least element. \qed
\end{proof}

\begin{prop}
Let $I$ be a well ordered set. Let $\{A_i\}_{i \in I}$ be a family of well ordered sets. Define $<$ on $\coprod_{i \in I} A_i$ by: $\kappa_i(a) < \kappa_j(b)$ iff either $i < j$, or $i = j$ and $a < b$ in $A_i$. Then $<$ well orders $\coprod_{i \in I} A_i$.
\end{prop}

\begin{proof}
\pf\ Easy. \qed
\end{proof}

\begin{thm}[Principle of Transfinite Recursion]
Let $J$ be a well ordered set. Let $C$ be a set. Let $\mathcal{F}$ be the set of all functions from a section of $J$ into $C$. Let $\rho : \mathcal{F} \rightarrow C$. Then there exists a unique function $h : J \rightarrow C$ such that, for all $\alpha \in J$, we have
\[ h(\alpha) = \rho(h \restriction (-\infty, \alpha)) \enspace . \]
\end{thm}

\begin{proof}
\pf
\step{1}{For a function $h$ mapping either a section of $J$ or all of $J$ into $C$, let us say $h$ is \emph{acceptable} iff, for all $x \in \dom h$, we have $(-\infty, x) \subseteq \dom h$ and $h(x) = \rho(h \restriction (-\infty, x))$.}
\step{1}{If $h$ and $k$ are acceptable functions then $h(x) = k(x)$ for all $x$ in both domains.}
\begin{proof}
	\step{a}{\pflet{$x \in J$}}
	\step{b}{\assume{as transfinite induction hypothesis that, for all $y < x$ and any acceptable functions $h$ and $k$ with $y \in \dom h \cap \dom k$, we have $h(y) = k(y)$}}
	\step{c}{\pflet{$h$ and $k$ be acceptable functions with $x \in \dom h \cap \dom k$}}
	\step{d}{$h \restriction (-\infty, x) = k \restriction (-\infty, x)$}
	\begin{proof}
		\pf\ By \stepref{b}.
	\end{proof}
	\step{d}{$h(x) = k(x)$}
	\begin{proof}
		\pf\ By \stepref{c}, each is the least element of the set in \stepref{d}.
	\end{proof}
\end{proof}
\step{2}{For $\alpha \in J$, if there exists an acceptable function $(- \infty, \alpha) \rightarrow C$, then there exists an acceptable function $(-\infty, \alpha] \rightarrow C$.}
\begin{proof}
	\step{a}{\pflet{$\alpha \in J$}}
	\step{b}{\pflet{$f : (-\infty, \alpha) \rightarrow C$ be acceptable.}}
	\step{c}{\pflet{$g : (-\infty, \alpha] \rightarrow C$ be the function given by
	\[ g(x) = 
	\begin{cases}
	f(x) & \text{if } x < \alpha \\
	\rho(f) & \text{if } x = \alpha
	\end{cases} \]}}
	\step{d}{$g$ is acceptable.}
\end{proof}
\step{3}{Let $K \subseteq J$. Assume that, for all $\alpha \in K$, there exists an acceptable function $(-\infty, \alpha) \rightarrow C$. Then there exists an acceptable function $\bigcup_{\alpha \in K} (-\infty, \alpha) \rightarrow C$.}
\begin{proof}
	\step{a}{Define $f : \bigcup_{\alpha \in K} (-\infty, \alpha) \rightarrow C$ by: $f(x) = y$ iff there exists $\alpha \in K$ and $g : (-\infty, \alpha) \rightarrow C$ acceptable such that $g(x) = y$.}
\end{proof}
\step{4}{For every $\beta \in J$, there exists an acceptable function $(-\infty, \beta) \rightarrow C$}
\begin{proof}
	\step{a}{\pflet{$\beta \in J$}}
	\step{b}{\assume{as transfinite induction hypothesis that, for all $\alpha < \beta$, there exists an acceptable function $(-\infty, \alpha) \rightarrow C$}}
	\step{c}{\case{$\beta$ has a predecessor}}
	\begin{proof}
		\step{i}{\pflet{$\alpha$ be the predecessor of $\beta$.}}
		\step{ii}{There exists an acceptable function $(-\infty, \alpha) \rightarrow C$.}
		\step{iii}{There exists an acceptable function $(-\infty, \beta) \rightarrow C$.}
		\begin{proof}
			\pf\ By \stepref{2} since $(-\infty, \beta) = (-\infty, \alpha]$.
		\end{proof}
	\end{proof}
	\step{d}{\case{$\beta$ has no predecessor.}}
	\begin{proof}
		\pf\ The result follows by \stepref{3} since $(-\infty, \beta) = \bigcup_{\alpha < \beta} (-\infty, \alpha)$.
	\end{proof}
\end{proof}
\step{5}{There exists an acceptable function $J \rightarrow C$.}
\begin{proof}
	\step{a}{\case{$J$ has a greatest element.}}
	\begin{proof}
		\step{i}{\pflet{$g$ be greatest.}}
		\step{ii}{There exists an acceptable function $(-\infty, g) \rightarrow C$.}
		\begin{proof}
			\pf\ \stepref{4}
		\end{proof}
		\step{iii}{There exists an acceptable function $J \rightarrow C$.}
		\begin{proof}
			\pf\ By \stepref{2} since $J = (-\infty, g]$.
		\end{proof}
	\end{proof}
	\step{b}{\case{$J$ has no greatest element.}}
	\begin{proof}
		\pf\ By \stepref{3} since $J = \bigcup_{\alpha \in J} (-\infty, \alpha)$.
	\end{proof}
\end{proof}
\qed
\end{proof}

\begin{cor}[Cardinal Comparability]
Let $A$ and $B$ be sets. Then either $A \preccurlyeq B$ or $B \preccurlyeq A$.
\end{cor}

\begin{proof}
\pf\ Choose well orderings of $A$ and $B$. Then either there exists a surjection $A \twoheadrightarrow B$, or there exists an injective function $h : A \rightarrowtail B$ defined by transfinite recursion by $h(x)$ is the least element of $B - h((-\infty, x))$. \qed
\end{proof}

\begin{prop}
\label{prop:order_preserving_maps}
Let $J$ and $E$ be well ordered sets. Let $h : J \rightarrow E$. Then the following are equivalent.
\begin{enumerate}
\item $h$ is strictly monotone and $h(J)$ is either $E$ or a section of $E$.
\item For all $\alpha \in J$, we have $h(\alpha)$ is the least element of $E - h((-\infty, \alpha))$.
\end{enumerate}
\end{prop}

\begin{proof}
\pf
\step{1}{$1 \Rightarrow 2$}
\begin{proof}
	\step{a}{\assume{1}}
	\step{b}{$h(J)$ is closed downwards.}
	\step{c}{\pflet{$\alpha \in J$}}
	\step{d}{$h(\alpha) \in E - h((-\infty, \alpha))$}
	\begin{proof}
		\pf\ If $\beta < \alpha$ then $h(\beta) < h(\alpha)$.
	\end{proof}
	\step{e}{For all $y \in E - h((-\infty, \alpha))$ we have $h(\alpha) \leq y$}
	\begin{proof}
		\step{i}{\assume{for a contradiction $y < h(\alpha)$}}
		\step{ii}{$y \in h(J)$}
		\step{iii}{\pick\ $\beta \in J$ such that $h(\beta) = y$}
		\step{iv}{$h(\beta) < h(\alpha)$}
		\step{v}{$\beta < \alpha$}
		\qedstep
		\begin{proof}
			\pf\ This contradicts the fact that $y \notin h((-\infty, \alpha))$.
		\end{proof}
	\end{proof}
\end{proof}
\step{2}{$2 \Rightarrow 1$}
\begin{proof}
	\step{a}{\assume{2}}
	\step{b}{$h$ is strictly monotone.}
	\begin{proof}
		\step{i}{\pflet{$\alpha, \beta \in J$ with $\alpha < \beta$}}
		\step{ii}{$h(\alpha) \neq h(\beta)$}
		\begin{proof}
			\pf\ Because $h(\beta) \in E - h((-\infty, \beta))$.
		\end{proof}
		\step{iii}{$h(\alpha) \leq h(\beta)$}
		\begin{proof}
			\pf Because $h(\alpha)$ is least in $E - h((-\infty, \alpha))$.
		\end{proof}
		\step{iv}{$h(\alpha) < h(\beta)$}
	\end{proof}
	\step{c}{$h(J)$ is either $E$ or a section of $E$.}
	\begin{proof}
		\step{i}{\assume{$h(J) \neq E$}}
		\step{ii}{\pflet{$e$ be least in $E - h(J)$} \prove{$h(J) = (-\infty, e)$}}
		\step{iii}{$h(J) \subseteq (-\infty, e)$}
		\begin{proof}
			\step{one}{\pflet{$\alpha \in J$}}
			\step{two}{$h(\alpha) \neq e$}
			\begin{proof}
				\pf\ $e \notin h(J)$
			\end{proof}
			\step{three}{$h(\alpha) \leq e$}
			\begin{proof}
				\pf\ Since $h(\alpha)$ is least in $E - h((-\infty, \alpha))$.
			\end{proof}
			\step{four}{$h(\alpha) < e$}
		\end{proof}
		\step{iv}{$(-\infty, e) \subseteq h(J)$}
		\begin{proof}
			\pf\ If $e' < e$ then $e' \in h(J)$ by leastness of $e$.
		\end{proof}
	\end{proof}
\end{proof}
\qed
\end{proof}


\part{Category Theory}

\section{Relations}

\begin{df}[Reflexive]
A relation $R \subseteq A \times A$ is \emph{reflexive} iff, for all $a \in A$, we have $(a,a) \in R$.
\end{df}

\begin{df}[Antisymmetric]
A relation $R \subseteq A \times A$ is \emph{antisymmetric} iff, for all $a,b \in A$, if $(a,b) \in R$ and $(b,a) \in R$ then $a = b$.
\end{df}

\begin{df}[Transitive]
A relation $R \subseteq A \times A$ is \emph{transitive} iff, for all $a,b,c \in A$, if $(a,b) \in R$ and $(b,c) \in R$ then $(a,c) \in R$.
\end{df}

\begin{df}[Partial Order]
A \emph{partial order} on a set $A$ is a relation on $A$ that is reflexive, antisymmetric and transitive.

We say $(A, \leq)$ is a \emph{partially ordered set} or \emph{poset} iff $\leq$ is a partial order on $A$.
\end{df}

\begin{df}[Greatest]
Let $A$ be a poset and $a \in A$. Then $a$ is the \emph{greatest} element iff $\forall x \in A. x \leq a$.
\end{df}

\begin{df}[Least]
Let $A$ be a poset and $a \in A$. Then $a$ is the \emph{least} element iff $\forall x \in A. a \leq x$.
\end{df}

\begin{df}[Upper Bound]
Let $A$ be a poset, $S \subseteq A$, and $u \in A$. Then $u$ is an \emph{upper bound} for $S$ iff $\forall x \in S. x \leq u$. We say $S$ is \emph{bounded above} iff it has an upper bound.
\end{df}

\begin{df}[Lower Bound]
Let $A$ be a poset, $S \subseteq A$, and $l \in A$. Then $l$ is a \emph{lower bound} for $S$ iff $\forall x \in S. l \leq x$. We say $S$ is \emph{bounded below} iff it has a lower bound.
\end{df}

\begin{df}[Supremum]
Let $A$ be a poset, $S \subseteq A$ and $s \in A$. Then $s$ is the \emph{supremum} or \emph{least upper bound} for $S$ iff $s$ is the least element in the sub-poset of upper bounds for $A$.
\end{df}

\begin{df}[Supremum]
Let $A$ be a poset, $S \subseteq A$ and $i \in A$. Then $i$ is the \emph{infimum} or \emph{greatest lower bound} for $S$ iff $i$ is the greatest element in the sub-poset of lower bounds for $A$.
\end{df}

\begin{df}[Least Upper Bound Property]
A poset $A$ has the \emph{least upper bound property} iff every nonempty subset of $A$ that is bounded above has a least upper bound.
\end{df}

\begin{prop}
Let $A$ be a poset. Then $A$ has the least upper bound property if and only if every nonempty subset of $A$ that is bounded below has a greatest lower bound.
\end{prop}

\begin{proof}
\pf
\step{1}{If $A$ has the least upper bound property then every subset of $A$ that is bounded below has a greatest lower bound.}
\begin{proof}
	\step{a}{\assume{$A$ has the least upper bound property.}}
	\step{b}{\pflet{$S \subseteq A$ be nonempty and bounded below.}}
	\step{c}{\pflet{$L$ be the set of lower bounds of $S$.}}
	\step{d}{$L$ is nonempty.}
	\begin{proof}
		\pf\ Because $S$ is bounded below.
	\end{proof}
	\step{e}{$L$ is bounded above.}
	\begin{proof}
		\pf\ Pick an element $s \in S$. Then $s$ is an upper bound for $L$.
	\end{proof}
	\step{f}{\pflet{$s$ be the supremum of $L$.}}
	\step{g}{$s$ is the greatest lower bound of $S$.}
	\begin{proof}
		\step{i}{$s$ is a lower bound of $S$.}
		\begin{proof}
			\step{one}{\pflet{$x \in S$}}
			\step{two}{$x$ is an upper bound for $L$.}
			\step{three}{$s \leq x$}
		\end{proof}
		\step{ii}{For any lower bound $l$ of $S$ we have $l \leq s$.}
		\begin{proof}
			\pf\ Immediate from \stepref{f}.
		\end{proof}
	\end{proof}
\end{proof}
\step{2}{If every subset of $A$ that is bounded below has a greatest lower bound, then $A$ has the least upper bound property.}
\begin{proof}
	\pf\ Dual.
\end{proof}
\qed
\end{proof}

\subsection{Strict Partial Orders}

\begin{df}[Strict Partial Order]
A \emph{strict partial order} on a set $A$ is a relation on $A$ that is irreflexive and transitive.
\end{df}

\begin{prop}
\begin{enumerate}
\item If $\leq$ is a partial order on $A$ then $<$ is a strict partial order on $A$, where $x < y$ iff $x \leq y \wedge x \neq y$.
\item If $<$ is a strict partial order on $A$ then $\leq$ is a partial order on $A$, where $x \leq y$ iff $x < y \vee x = y$.
\item These two relations are inverses of one another.
\end{enumerate}
\end{prop}

\subsection{Linear Orders}

\begin{df}[Linear Order]
A \emph{linear order} on a set $A$ is a partial order $\leq$ on $A$ such that, for all $x,y \in A$, we have $x \leq y$ or $y \leq x$.

A \emph{linearly ordered set} is a pair $(X, \leq)$ such that $X$ is a set and $\leq$ is a linear order on $X$.
\end{df}

\begin{df}[Open Interval]
Let $X$ be a linearly ordered set and $a,b \in X$. The \emph{open interval} $(a,b)$ is the set
\[ \{ x \in X : a < x < b \} \enspace . \]
\end{df}

\begin{df}[Immediate Predecessor, Immediate Successor]
Let $X$ be a linearly ordered set and $a,b \in X$. Then $b$ is the \emph{(immediate) successor} of $a$, and $a$ is the \emph{(immediate) predecessor} of $b$, iff $a < B$ and there is no $x$ such that $a < x < b$.
\end{df}

\begin{df}[Dictionary Order]
Let $A$ and $B$ be linearly ordered sets. The \emph{dictionary order} on $A \times B$ is the order defined by
\[ (a,b) < (a',b') \Leftrightarrow a < a' \vee (a = a' \wedge b < b') \enspace . \]
\end{df}

\begin{thm}[Maximum Principle]
Every poset has a maximal linearly ordered subset.
\end{thm}

\begin{proof}
\pf
\step{1}{\pflet{$(A, \leq)$ be a poset.}}
\step{2}{\pick\ a well ordering $\preccurlyeq$ of $A$.}
\begin{proof}
	\pf\ Well Ordering Theorem.
\end{proof}
\step{3}{\pflet{$h : A \rightarrow 2$ be the function defined by $\preccurlyeq$-recursion thus:
\[ h(a) =
\begin{cases}
1 & \text{if $a$ is $\leq$-comparable with every $b \prec a$ such that $h(b) = 1$} \\
0 & \text{otherwise}
\end{cases} \]}}
\step{4}{\pflet{$B = \{x \in A : h(x) = 1\}$} \prove{$B$ is a maximal subset linearly ordered by $\leq$.}}
\step{5}{$B$ is linearly ordered by $\leq$.}
\begin{proof}
	\step{a}{\pflet{$x,y \in B$}}
	\step{b}{\assume{w.l.o.g. $x \preccurlyeq y$}}
	\step{c}{$y$ is $\leq$-comparable with $x$}
\end{proof}
\step{6}{For any subset $C \subseteq A$ linearly ordered by $\leq$, if $B \subseteq C$ then $B = C$.}
\begin{proof}
	\step{a}{\pflet{$x \in C$}}
	\step{b}{$x$ is comparable with every $y \preccurlyeq x$ such that $h(x) = 1$}
	\step{c}{$x \in B$}
\end{proof}
\qed
\end{proof}

\begin{thm}[Zorn's Lemma]
Let $A$ be a poset.
If every linearly ordered subset of $A$ is bounded above, then $A$ has a maximal element.
\end{thm}

\begin{proof}
\pf
\step{1}{\pick\ a maximal linearly ordered subset $B$ of $A$.}
\begin{proof}
	\pf\ Maximal Principle
\end{proof}
\step{2}{\pick\ an upper bound $c$ for $B$. \prove{$c$ is maximal.}}
\step{3}{\pflet{$x \in A$}}
\step{4}{\assume{$c \leq x$} \prove{$x = c$}}
\step{5}{$x$ is an upper bound for $B$.}
\step{6}{$x \in B$}
\begin{proof}
	\pf\ By the maximality of $B$, since $B \cup \{x\}$ is linearly ordered.
\end{proof}
\step{7}{$x \leq c$}
\begin{proof}
	\pf\ \stepref{2}
\end{proof}
\step{8}{$x = c$}
\qed
\end{proof}

\begin{cor}[Kuratowski's Lemma]
Let $\mathcal{A} \subseteq \mathcal{P} X$. Suppose that, for every subset $\mathcal{B} \subseteq \mathcal{A}$ that is linearly ordered by inclusion, we have $\bigcup \mathcal{B} \in \mathcal{A}$. Then $\mathcal{A}$ has a maximal element.
\end{cor}

\begin{df}[Closed Interval]
Let $X$ be a linearly ordered set. Let $a,b \in X$ with $a < b$. The \emph{closed interval} $[a,b]$ is
\[ [a,b] := \{ x \in X : a \leq x \leq b \} \enspace . \]
\end{df}

\begin{df}[Half-Open Interval]
Let $X$ be a linearly ordered set. Let $a,b \in X$ with $a < b$. The \emph{half-open intervals} $(a,b]$ and $[a,b)$ are defined by
\begin{align*}
(a,b] & := \{ x \in X : a < x \leq b \} \\
[a,b) & := \{ x \in X : a \leq x < b \}
\end{align*}
\end{df}

\begin{df}[Open Ray]
Let $X$ be a linearly ordered set and $a \in X$. The \emph{open rays} $(a, +\infty)$ and $(-\infty, a)$ are defined by:
\begin{align*}
(a, +\infty) & := \{ x \in X : a < x \} \\
(-\infty, a) & := \{ x \in X : x < a \}
\end{align*}
\end{df}

\begin{df}[Closed Ray]
Let $X$ be a linearly ordered set and $a \in X$. The \emph{closed rays} $[a, +\infty)$ and $(-\infty, a]$ are defined by:
\begin{align*}
[a, +\infty) & := \{ x \in X : a \leq x \} \\
(-\infty, a] & := \{ x \in X : x \leq a \}
\end{align*}
\end{df}

\begin{df}[Convex]
Let $X$ be a linearly ordered set and $Y \subseteq X$. Then $Y$ is \emph{convex} iff, for all $a,b \in Y$ and $c \in X$, if $a < c < b$ then $c \in Y$.
\end{df}

\subsection{Sets of Finite Type}

\begin{df}[Finite Type]
Let $X$ be a set. Let $\mathcal{A} \subseteq \mathcal{P} X$. Then $\mathcal{A}$ is of \emph{finite type} if and only if, for any $B \subseteq X$, we have $B \in \mathcal{A}$ if and only if every finite subset of $B$ is in $\mathcal{A}$.
\end{df}

\begin{prop}[Tukey's Lemma]
Let $X$ be a set. Let $\mathcal{A} \subseteq \mathcal{P} X$. If $\mathcal{A}$ is of finite type, then $\mathcal{A}$ has a maximal element.
\end{prop}

\begin{proof}
\pf
\step{1}{For every subset $\mathcal{B} \subseteq \mathcal{A}$ that is linearly ordered by inclusion, we have $\bigcup \mathcal{B} \in \mathcal{A}$.}
\begin{proof}
	\step{a}{\pflet{$\mathcal{B} \subseteq \mathcal{A}$}}
	\step{b}{\assume{$\mathcal{B}$ is linearly ordered by inclusion.}}
	\step{c}{Every finite subset of $\bigcup \mathcal{B}$ is in $\mathcal{A}$}
	\step{d}{$\bigcup \mathcal{B} \in \mathcal{A}$}
\end{proof}
\qedstep
\begin{proof}
	\pf\ Kuratowski's Lemma.
\end{proof}
\qed
\end{proof}

\section{Well Orders}

\begin{df}[Well Ordered Set]
A \emph{well ordered set} is a linearly ordered set such that every nonempty subset has a least element.
\end{df}

\begin{prop}
Any subset of a well ordered set is well ordered.
\end{prop}

\begin{prop}
The product of two well ordered sets is well ordered under the dictionary order.
\end{prop}

\begin{thm}[Well Ordering Theorem]
Every set has a well ordering.
\end{thm}

\begin{proof}
\pf
\step{1}{\pflet{$X$ be a set.}}
\step{2}{\pick\ a choice function $c : \mathcal{P} X - \{ \emptyset \} \rightarrow X$}
\step{3}{Define a \emph{tower} to be a pair $(T,<)$ where $T \subseteq X$, $<$ is a well ordering of $T$, and
\[ \forall x \in T. x = c(X - \{ y \in T : y < x \}) \enspace . \]}
\step{4}{Given two towers, either they are equal or one is a section of the other.}
\begin{proof}
	\step{a}{\pflet{$(T_1,<_1)$ and $(T_2,<_2)$ be towers.}}
	\step{b}{\assume{w.l.o.g. there exists a strictly monotone function $h : T_1 \rightarrow T_2$}}
	\step{c}{$h(T_1)$ is either $T_2$ or a section of $T_2$}
	\begin{proof}
		\pf\ Proposition \ref{prop:order_preserving_maps}.
	\end{proof}
	\step{d}{$\forall x \in T_1. h(x) = x$}
	\begin{proof}
		\step{i}{\pflet{$x \in T_1$}}
		\step{ii}{\assume{as transfinite induction hypothesis $\forall y < x. h(y) = y$}}
		\step{iii}{$h(x)$ is the least element of $T_2 - \{ h(y) \in T_1 : y < x \}$}
		\step{iv}{$h(x)$ is the least element of $T_2 - \{ y \in T_1 : y < x \}$}
		\begin{proof}
			\pf\ \stepref{ii}
		\end{proof}
		\step{v}{$h(x) = x$}
		\begin{proof}
			\pf
			\begin{align*}
				h(x) & = c(X - \{y \in T_2 : y < h(x)\}) & (\text{\stepref{3}}) \\
				& = c(X - \{y \in T_2 : y < x \}) & (\text{\stepref{iv}}) \\
				& = c(X - \{ y \in T_1 : y < x \}) & (\text{\stepref{ii}}) \\
				& = x & (\text{\stepref{3}})
			\end{align*}
		\end{proof}
	\end{proof}
\end{proof}
\step{5}{If $(T,<)$ is a tower and $T \neq X$, then there exists a tower of which $(T,<)$ is a section.}
\begin{proof}
	\pf\ Let $T_1 = T \cup \{c(T)\}$ and $<_1$ be the extension of $<$ such that $x < c(T)$ for all $x \in T$.
\end{proof}
\step{6}{\pflet{$\mathbf{T} = \bigcup \{ T : \exists R. (T,R) \text{ is a tower} \}$ and $\mathbf{R} = \bigcup \{ R : \exists T. (T,R) \text{ is a tower} \}$}}
\step{7}{$(\mathbf{T},\mathbf{R})$ is a tower.}
\begin{proof}
	\step{a}{$\mathbf{R}$ is irreflexive.}
	\begin{proof}
		\pf\ Since for every tower $(T,<)$ we have $<$ is irreflexive.
	\end{proof}
	\step{b}{$\mathbf{R}$ is transitive.}
	\begin{proof}
		\step{i}{\assume{$x \mathbf{R} y$ and $y \mathbf{R} z$}}
		\step{ii}{\pick\ towers $(T_1,<_1)$ and $(T_2,<_2)$ such that $x <_1 y$ and $y <_2 z$}
		\step{iii}{\assume{w.l.o.g. $(T_1,<_1)$ is either $(T_2,<_2)$ or a section of $(T_2,<_2)$}}
		\step{iv}{$x <_2 y <_2 z$}
		\step{v}{$x <_2 z$}
		\step{vi}{$x \mathbf{R} z$}
	\end{proof}
	\step{c}{For all $x,y \in \mathbf{T}$, either $x\mathbf{R}y$ or $x= y$ or $y\mathbf{R}x$}
	\begin{proof}
		\pf\ There exists a tower that has both $x$ and $y$.
	\end{proof}
	\step{d}{Every nonempty subset of $\mathbf{T}$ has an $\mathbf{R}$-least element.}
	\begin{proof}
		\step{i}{\pflet{$A \subseteq \mathbf{T}$ be nonempty.}}
		\step{ii}{\pick\ $a \in A$}
		\step{iii}{\pick\ a tower $(T,<)$ such that $a \in T$.}
		\step{iv}{\pflet{$b$ be the $<$-least element of $A \cap T$} \prove{$b$ is $\mathbf{R}$-least in $A$.}}
		\step{vi}{\pflet{$x \in A$}}
		\step{vii}{Etc.}
	\end{proof}
	\step{e}{$\forall x \in \mathbf{T}. x = c(X - \{y \in \mathbf{T} : y\mathbf{R}x \})$}
\end{proof}
\step{8}{$\mathbf{T} = X$}
\step{9}{$\mathbf{R}$ is a well ordering of $X$.}
\qed
\end{proof}

\begin{prop}
There exists a well-ordered set with a largest element $\Omega$ such that $(-\infty, \Omega)$ is uncountable but, for all $\alpha < \Omega$, we have $(-\infty, \alpha)$ is countable.
\end{prop}

\begin{proof}
\pf
\step{1}{\pick\ an uncountable well ordered set $B$.}
\step{2}{\pflet{$C = 2 \times B$ under the dictionary order.}}
\step{3}{\pflet{$\Omega$ be the least element of $C$ such that $(-\infty, \Omega)$ is uncountable.}}
\step{4}{\pflet{$A = (-\infty, \Omega]$}}
\step{5}{$A$ is a well ordered set with largest element $\Omega$ such that $(-\infty, \Omega)$ is uncountable but, for all $\alpha < \Omega$, we have $(-\infty, \alpha)$ is countable.}
\qed
\end{proof}

\begin{prop}
Every well ordered set has the least upper bound property.
\end{prop}

\begin{proof}
\pf\ For any subset that is bounded above, the set of upper bounds is nonempty, hence has a least element. \qed
\end{proof}

\begin{prop}
In a well ordered set, every element that is not greatest has a successor.
\end{prop}

\begin{proof}
\pf\ If $a$ is not greatest, then $\{ x : x > a \}$ is nonempty, hence has a least element. \qed
\end{proof}

\begin{thm}[Transfinite Induction]
Let $J$ be a well ordered set. Let $S \subseteq J$. Assume that, for every $\alpha \in J$, if $\forall x < \alpha. x \in S$ then $\alpha in S$. Then $S = J$.
\end{thm}

\begin{proof}
\pf\ Otherwise $J - S$ would be a nonempty subset of $J$ with no least element. \qed
\end{proof}

\begin{prop}
Let $I$ be a well ordered set. Let $\{A_i\}_{i \in I}$ be a family of well ordered sets. Define $<$ on $\coprod_{i \in I} A_i$ by: $\kappa_i(a) < \kappa_j(b)$ iff either $i < j$, or $i = j$ and $a < b$ in $A_i$. Then $<$ well orders $\coprod_{i \in I} A_i$.
\end{prop}

\begin{proof}
\pf\ Easy. \qed
\end{proof}

\begin{thm}[Principle of Transfinite Recursion]
Let $J$ be a well ordered set. Let $C$ be a set. Let $\mathcal{F}$ be the set of all functions from a section of $J$ into $C$. Let $\rho : \mathcal{F} \rightarrow C$. Then there exists a unique function $h : J \rightarrow C$ such that, for all $\alpha \in J$, we have
\[ h(\alpha) = \rho(h \restriction (-\infty, \alpha)) \enspace . \]
\end{thm}

\begin{proof}
\pf
\step{1}{For a function $h$ mapping either a section of $J$ or all of $J$ into $C$, let us say $h$ is \emph{acceptable} iff, for all $x \in \dom h$, we have $(-\infty, x) \subseteq \dom h$ and $h(x) = \rho(h \restriction (-\infty, x))$.}
\step{1}{If $h$ and $k$ are acceptable functions then $h(x) = k(x)$ for all $x$ in both domains.}
\begin{proof}
	\step{a}{\pflet{$x \in J$}}
	\step{b}{\assume{as transfinite induction hypothesis that, for all $y < x$ and any acceptable functions $h$ and $k$ with $y \in \dom h \cap \dom k$, we have $h(y) = k(y)$}}
	\step{c}{\pflet{$h$ and $k$ be acceptable functions with $x \in \dom h \cap \dom k$}}
	\step{d}{$h \restriction (-\infty, x) = k \restriction (-\infty, x)$}
	\begin{proof}
		\pf\ By \stepref{b}.
	\end{proof}
	\step{d}{$h(x) = k(x)$}
	\begin{proof}
		\pf\ By \stepref{c}, each is the least element of the set in \stepref{d}.
	\end{proof}
\end{proof}
\step{2}{For $\alpha \in J$, if there exists an acceptable function $(- \infty, \alpha) \rightarrow C$, then there exists an acceptable function $(-\infty, \alpha] \rightarrow C$.}
\begin{proof}
	\step{a}{\pflet{$\alpha \in J$}}
	\step{b}{\pflet{$f : (-\infty, \alpha) \rightarrow C$ be acceptable.}}
	\step{c}{\pflet{$g : (-\infty, \alpha] \rightarrow C$ be the function given by
	\[ g(x) = 
	\begin{cases}
	f(x) & \text{if } x < \alpha \\
	\rho(f) & \text{if } x = \alpha
	\end{cases} \]}}
	\step{d}{$g$ is acceptable.}
\end{proof}
\step{3}{Let $K \subseteq J$. Assume that, for all $\alpha \in K$, there exists an acceptable function $(-\infty, \alpha) \rightarrow C$. Then there exists an acceptable function $\bigcup_{\alpha \in K} (-\infty, \alpha) \rightarrow C$.}
\begin{proof}
	\step{a}{Define $f : \bigcup_{\alpha \in K} (-\infty, \alpha) \rightarrow C$ by: $f(x) = y$ iff there exists $\alpha \in K$ and $g : (-\infty, \alpha) \rightarrow C$ acceptable such that $g(x) = y$.}
\end{proof}
\step{4}{For every $\beta \in J$, there exists an acceptable function $(-\infty, \beta) \rightarrow C$}
\begin{proof}
	\step{a}{\pflet{$\beta \in J$}}
	\step{b}{\assume{as transfinite induction hypothesis that, for all $\alpha < \beta$, there exists an acceptable function $(-\infty, \alpha) \rightarrow C$}}
	\step{c}{\case{$\beta$ has a predecessor}}
	\begin{proof}
		\step{i}{\pflet{$\alpha$ be the predecessor of $\beta$.}}
		\step{ii}{There exists an acceptable function $(-\infty, \alpha) \rightarrow C$.}
		\step{iii}{There exists an acceptable function $(-\infty, \beta) \rightarrow C$.}
		\begin{proof}
			\pf\ By \stepref{2} since $(-\infty, \beta) = (-\infty, \alpha]$.
		\end{proof}
	\end{proof}
	\step{d}{\case{$\beta$ has no predecessor.}}
	\begin{proof}
		\pf\ The result follows by \stepref{3} since $(-\infty, \beta) = \bigcup_{\alpha < \beta} (-\infty, \alpha)$.
	\end{proof}
\end{proof}
\step{5}{There exists an acceptable function $J \rightarrow C$.}
\begin{proof}
	\step{a}{\case{$J$ has a greatest element.}}
	\begin{proof}
		\step{i}{\pflet{$g$ be greatest.}}
		\step{ii}{There exists an acceptable function $(-\infty, g) \rightarrow C$.}
		\begin{proof}
			\pf\ \stepref{4}
		\end{proof}
		\step{iii}{There exists an acceptable function $J \rightarrow C$.}
		\begin{proof}
			\pf\ By \stepref{2} since $J = (-\infty, g]$.
		\end{proof}
	\end{proof}
	\step{b}{\case{$J$ has no greatest element.}}
	\begin{proof}
		\pf\ By \stepref{3} since $J = \bigcup_{\alpha \in J} (-\infty, \alpha)$.
	\end{proof}
\end{proof}
\qed
\end{proof}

\begin{cor}[Cardinal Comparability]
Let $A$ and $B$ be sets. Then either $A \preccurlyeq B$ or $B \preccurlyeq A$.
\end{cor}

\begin{proof}
\pf\ Choose well orderings of $A$ and $B$. Then either there exists a surjection $A \twoheadrightarrow B$, or there exists an injective function $h : A \rightarrowtail B$ defined by transfinite recursion by $h(x)$ is the least element of $B - h((-\infty, x))$. \qed
\end{proof}

\begin{prop}
\label{prop:order_preserving_maps}
Let $J$ and $E$ be well ordered sets. Let $h : J \rightarrow E$. Then the following are equivalent.
\begin{enumerate}
\item $h$ is strictly monotone and $h(J)$ is either $E$ or a section of $E$.
\item For all $\alpha \in J$, we have $h(\alpha)$ is the least element of $E - h((-\infty, \alpha))$.
\end{enumerate}
\end{prop}

\begin{proof}
\pf
\step{1}{$1 \Rightarrow 2$}
\begin{proof}
	\step{a}{\assume{1}}
	\step{b}{$h(J)$ is closed downwards.}
	\step{c}{\pflet{$\alpha \in J$}}
	\step{d}{$h(\alpha) \in E - h((-\infty, \alpha))$}
	\begin{proof}
		\pf\ If $\beta < \alpha$ then $h(\beta) < h(\alpha)$.
	\end{proof}
	\step{e}{For all $y \in E - h((-\infty, \alpha))$ we have $h(\alpha) \leq y$}
	\begin{proof}
		\step{i}{\assume{for a contradiction $y < h(\alpha)$}}
		\step{ii}{$y \in h(J)$}
		\step{iii}{\pick\ $\beta \in J$ such that $h(\beta) = y$}
		\step{iv}{$h(\beta) < h(\alpha)$}
		\step{v}{$\beta < \alpha$}
		\qedstep
		\begin{proof}
			\pf\ This contradicts the fact that $y \notin h((-\infty, \alpha))$.
		\end{proof}
	\end{proof}
\end{proof}
\step{2}{$2 \Rightarrow 1$}
\begin{proof}
	\step{a}{\assume{2}}
	\step{b}{$h$ is strictly monotone.}
	\begin{proof}
		\step{i}{\pflet{$\alpha, \beta \in J$ with $\alpha < \beta$}}
		\step{ii}{$h(\alpha) \neq h(\beta)$}
		\begin{proof}
			\pf\ Because $h(\beta) \in E - h((-\infty, \beta))$.
		\end{proof}
		\step{iii}{$h(\alpha) \leq h(\beta)$}
		\begin{proof}
			\pf Because $h(\alpha)$ is least in $E - h((-\infty, \alpha))$.
		\end{proof}
		\step{iv}{$h(\alpha) < h(\beta)$}
	\end{proof}
	\step{c}{$h(J)$ is either $E$ or a section of $E$.}
	\begin{proof}
		\step{i}{\assume{$h(J) \neq E$}}
		\step{ii}{\pflet{$e$ be least in $E - h(J)$} \prove{$h(J) = (-\infty, e)$}}
		\step{iii}{$h(J) \subseteq (-\infty, e)$}
		\begin{proof}
			\step{one}{\pflet{$\alpha \in J$}}
			\step{two}{$h(\alpha) \neq e$}
			\begin{proof}
				\pf\ $e \notin h(J)$
			\end{proof}
			\step{three}{$h(\alpha) \leq e$}
			\begin{proof}
				\pf\ Since $h(\alpha)$ is least in $E - h((-\infty, \alpha))$.
			\end{proof}
			\step{four}{$h(\alpha) < e$}
		\end{proof}
		\step{iv}{$(-\infty, e) \subseteq h(J)$}
		\begin{proof}
			\pf\ If $e' < e$ then $e' \in h(J)$ by leastness of $e$.
		\end{proof}
	\end{proof}
\end{proof}
\qed
\end{proof}

\chapter{Category Theory}

\section{Categories}

\begin{df}
A \emph{category} $\mathcal{C}$ consists of:
\begin{itemize}
\item a set $\Ob{\mathcal{C}}$ of \emph{objects}. We write $A \in \mathcal{C}$ for $A \in \Ob{\mathcal{C}}$.
\item for any objects $X$ and $Y$, a set $\mathcal{C}[X,Y]$ of \emph{morphisms} from $X$ to $Y$. We write $f : X \rightarrow Y$ for $f \in \mathcal{C}[X,Y]$.
\item for any objects $X$, $Y$ and $Z$, a function $\circ : \mathcal{C}[Y,Z] \times \mathcal{C}[X,Y] \rightarrow \mathcal{C}[X,Z]$, called \emph{composition}.
\end{itemize}
such that:
\begin{itemize}
\item Given $f : X \rightarrow Y$, $g : Y \rightarrow Z$ and $h : Z \rightarrow W$, we have $h \circ (g \circ f) = (h \circ g) \circ f$
\item For any object $X$, there exists a morphism $\id{X} : X \rightarrow X$, the \emph{identity morphism} on $X$, such that:
\begin{itemize}
\item for any object $Y$ and morphism $f : Y \rightarrow X$ we have $\id{X} \circ f = f$
\item for any object $Y$ and morphism $f : X \rightarrow Y$ we have $f \circ \id{X} = f$
\end{itemize} 
\end{itemize}
\end{df}

We write the composite of morphism $f_1$, \ldots, $f_n$ as $f_n \circ \cdots \circ f_1$. This is unambiguous thanks to Associativity.

\begin{df}
Let $\Set$ be the category of small sets and functions.
\end{df}

\begin{df}
Let $\mathbf{LPos}$ be the category of linearly ordered sets and monotone functions.
\end{df}

\begin{prop}
Any finite linearly ordered set is isomorphic to $\{ m \in \mathbb{N} : m < n \}$ for some $n$.
\end{prop}

\begin{proof}
\pf
\step{1}{Every finite nonempty linearly ordered set has a greatest element.}
\begin{proof}
	\step{a}{\pflet{$P[n]$ be the property: for any linearly ordered set $A$, if there exists a bijection $A \approx \{ m \in \mathbb{N} : m < n \}$ and $A$ is nonempty then $A$ has a greatest element.}}
	\step{b}{$P[0]$}
	\begin{proof}
		\pf\ Vacuous.
	\end{proof}
	\step{c}{$\forall n \in \mathbb{N}. P[n] \Rightarrow P[n+1]$}
	\begin{proof}
		\step{i}{\pflet{$n \in \mathbb{N}$}}
		\step{ii}{\assume{$P[n]$}}
		\step{iii}{\pflet{$A$ be a nonempty linearly ordered set.}}
		\step{iv}{\pflet{$f : A \approx \{ m \in \mathbb{N} : m < n+1\}$}}
		\step{v}{\pflet{$a = \inv{f}(n)$}}
		\step{vi}{$f \restriction (A - \{a\}) : A - \{a\} \approx \{ m \in \mathbb{N} : m < n \}$}
		\step{vii}{\assume{w.l.o.g. $a$ is not greatest in $A$.}}
		\step{viii}{\pflet{$b$ be greatest in $A - \{a\}$}}
		\begin{proof}
			\pf\ \stepref{ii}
		\end{proof}
		\step{ix}{$b$ is greatest in $A$.}
	\end{proof}
\end{proof}
\step{2}{\pflet{$P[n]$ be the property: for any linearly ordered set $A$, if there exists a bijection $A \approx \{ m \in \mathbb{N} : m < n \}$ then there exists an isomorphism in $\mathbf{LPos}$ $A \cong \{ m \in \mathbb{N} : m < n \}$.}}
\step{3}{$P[0]$}
\begin{proof}
	\pf\ If there exists a bijection $A \approx \emptyset$ then $A$ is empty and so the unique function $A \rightarrow \emptyset$ is an order isomorphism.
\end{proof}
\step{4}{For every natural number $n$, if $P[n]$ then $P[n+1]$.}
\begin{proof}
	\step{a}{\pflet{$n$ be a natural number.}}
	\step{b}{\assume{$P[n]$}}
	\step{c}{\pflet{$A$ be a linearly ordered set.}}
	\step{d}{\assume{$A$ has $n+1$ elements.}}
	\step{e}{\pflet{$a$ be the greatest element in $A$.}}
	\step{f}{\pflet{$f : A - \{a\} \cong \{m \in \mathbb{N} : m < n \}$ be an order isomorphism.}}
	\begin{proof}
		\pf\ \stepref{b}
	\end{proof}
	\step{g}{Define $g : A \rightarrow \{ m \in \mathbb{N} : m < n + 1 \}$ by
	\[ g(x) = \begin{cases}
	f(x) & \text{if } x \neq a \\
	n & \text{if } x = a
	\end{cases} \]}
	\step{h}{$g$ is an order isomorphism.}
\end{proof}
\step{5}{$\forall n \in \mathbb{N}. P[n]$}
\qed
\end{proof}

\begin{cor}
Any finite linearly ordered set is well ordered.
\end{cor}

\begin{prop}
Let $J$ and $E$ be well ordered sets. Suppose there is a strictly monotone map $J \rightarrow E$. Then $J$ is isomorphic either to $E$ or a section of $E$.
\end{prop}

\begin{proof}
\pf
\step{1}{\pflet{$k : J \rightarrow E$ be strictly monotone.}}
\step{2}{\assume{w.l.o.g. $E$ is nonempty.}}
\step{3}{\pick\ $e_0 \in E$}
\step{4}{\pflet{$h : J \rightarrow E$ be the function defined by transfinite recursion thus:
\[ h(\alpha) = \begin{cases}
\text{the least element in } E - h((-\infty, \alpha)) & \text{if } h((-\infty, \alpha)) \neq E \\
e_0 & \text{if } h((-\infty, \alpha)) = E
\end{cases} \]}}
\step{5}{$\forall \alpha \in J. h(\alpha) \leq k(\alpha)$}
\begin{proof}
	\step{a}{\pflet{$\alpha \in J$}}
	\step{b}{\assume{as transfinite induction hypothesis $\forall \beta < \alpha. h(\beta) \leq k(\beta)$.}}
	\step{c}{$\forall \beta < \alpha. h(\beta) < k(\alpha)$}
	\step{d}{$h((-\infty, \alpha)) \neq E$}
	\step{e}{$h(\alpha)$ is the least element in $E - h((-\infty, \alpha))$.}
	\step{f}{$k(\alpha) \in E - h((-\infty, \alpha))$}
	\step{g}{$h(\alpha) \leq k(\alpha)$}
\end{proof}
\step{6}{$\forall \alpha \in J. h((-\infty, \alpha)) \neq E$}
\begin{proof}
	\pf\ For $\beta < \alpha$ we have $h(\beta) \leq k(\beta) < k(\alpha)$ so $k(\alpha) \notin h((-\infty, \alpha))$.
\end{proof}
\step{7}{For all $\alpha \in J$, we have $h(\alpha)$ is the least element of $E - h((-\infty, \alpha))$.}
\step{8}{$h$ is strictly monotone and $h(J)$ is either $E$ or a section of $E$.}
\begin{proof}
	\pf\ Proposition \ref{prop:order_preserving_maps}.
\end{proof}
\qed
\end{proof}

\begin{prop}
If $A$ and $B$ are well ordered sets, then exactly one of the following conditions hold: $A \cong B$, or $A$ is isomorphic to a section of $B$, or $B$ is isomorphic to a section of $A$.
\end{prop}

\begin{proof}
\pf
\step{1}{At least one of the conditions holds.}
\begin{proof}
	\step{a}{$B$ is isomorphic to either $A + B$ or a section of $A + B$.}
	\step{b}{\case{$B \cong A + B$}}
	\begin{proof}
		\step{i}{\pflet{$\phi$ be the isomorphism $B \cong A + B$}}
		\step{ii}{\pflet{$b_0$ be the least element in $B$.}}
		\step{iii}{$A$ is isomorphic to the section $(-\infty, \inv{\phi}(\kappa_2(b_0)))$ of $B$.}
	\end{proof}
	\step{c}{\case{$a \in A$ and $B \cong (-\infty, \kappa_1(a))$}}
	\begin{proof}
		\pf\ Then $B$ is isomorphic to the section $(-\infty, a)$ of $A$.
	\end{proof}
	\step{d}{\case{$b \in B$ and $\phi : B \cong (-\infty, \kappa_2(b))$}}
	\begin{proof}
		\step{i}{\case{$b$ is least in $B$.}}
		\begin{proof}
			\pf\ Then $A \cong B$.
		\end{proof}
		\step{ii}{\case{$b$ is not least in $B$.}}
		\begin{proof}
			\step{one}{\pflet{$b_0$ be least in $B$.}}
			\step{two}{$A$ is isomorphic to the section $(-\infty, \inv{\phi}(\kappa_2(b_0)))$ of $B$.}
		\end{proof}
	\end{proof}
\end{proof}
\step{2}{At most one of the conditions holds.}
\begin{proof}
	\pf\ Since a well ordered set cannot be isomorphic to a section of itself.
\end{proof}
\qed
\end{proof}

\begin{thm}
There exists a well ordered set, unique up to order isomorphism, that is uncountable but such that every section is countable.
\end{thm}

\begin{proof}
\pf
\step{1}{There exists a well ordered set that is uncountable but such that every section is countable.}
\begin{proof}
	\step{a}{\pick\ a well ordered set $A$ with an element $\Omega \in A$ such that $(-\infty, \Omega)$ is uncountable but $\forall \alpha < \Omega. (-\infty, \alpha)$ is countable.}
	\step{b}{\pflet{$(-\infty, Omega)$ is uncountable but every section is countable.}}
\end{proof}
\step{2}{If $A$ and $B$ are uncountable well ordered sets such that every section is countable, then $A \cong B$.}
\begin{proof}
	\pf\ Since it cannot be that one of $A$ and $B$ is isomorphic to a section of the other.
\end{proof}
\qed
\end{proof}

\begin{df}[Minimal Uncountable Well Ordered Set]
The \emph{minimal uncountable well ordered set} $\Omega$ is the well ordered set that is uncountable but such that every section is countable.

We write $\overline{\Omega}$ for the well ordered set $\Omega \cup \{ \Omega \}$ where $\Omega$ is greatest.
\end{df}

\begin{prop}
Every countable subset of $\Omega$ is bounded above.
\end{prop}

\begin{proof}
\pf
\step{1}{\pflet{$A$ be a countable subset of $\Omega$.}}
\step{2}{For all $a \in A$ we have $(-\infty, a)$ is countable.}
\step{3}{$\bigcup_{a \in A} (- \infty, a)$ is countable.}
\step{4}{$\bigcup_{a \in A} (-\infty, a) \neq \Omega$}
\step{5}{\pick\ $x \in \Omega - \bigcup_{a \in A} (-\infty, a)$}
\step{6}{$x$ is an upper bound for $A$.}
\qed
\end{proof}

\begin{prop}
$\Omega$ has no greatest element.
\end{prop}

\begin{proof}
\pf\ For any $\alpha \in \Omega$ we have $(- \infty, \alpha]$ is countable and hence not the whole of $\Omega$. \qed
\end{proof}

\begin{prop}
There are uncountably many elements of $\Omega$ that have no predecessor.
\end{prop}

\begin{proof}
\pf
\step{1}{\pflet{$A$ be the set of all elements of $\Omega$ that have no predecessor.}}
\step{2}{\pflet{$f : A \times \mathbb{N} \rightarrow \Omega$ be the function that maps $(a,n)$ to the $n$th successor of $a$.}}
\step{3}{$f$ is surjective.}
\begin{proof}
	\step{a}{\assume{for a contradiction $x \in \Omega$ and there is no element $a \in A$ and $n \in \mathbb{N}$ such that $x$ is the $n$th successor of $a$.}}
	\step{b}{\pflet{$x_n$ be the $n$th predecessor of $x$ for $n \in \mathbb{N}$.}}
	\step{c}{$\{x_n : n \in \mathbb{N} \}$ is a nonempty subset of $\Omega$ with no least element.}
\end{proof}
\step{4}{$A \times \mathbb{N}$ is uncountable.}
\step{5}{$A$ is uncountable.}
\qed
\end{proof}

\begin{df}
We identify a poset $(A, \leq)$ with the category with:
\begin{itemize}
\item set of objects $A$
\item for $a,b \in A$, the set of homomorphisms is $\{ x \in 1 : a \leq b \}$
\end{itemize}
\end{df}

\begin{prop}
A category is a poset iff, for any two objects, there exists at most one morphism between them.
\end{prop}

\begin{prop}
The identity morphism on an object is unique.
\end{prop}

\begin{proof}
\pf
\step{1}{\pflet{$\mathcal{C}$ be a category.}}
\step{2}{\pflet{$A \in \mathcal{C}$}}
\step{3}{\pflet{$i,j : A \rightarrow A$ be identity morphisms on $A$.}}
\step{4}{$i = j$}
\begin{proof}
	\pf
	\begin{align*}
		i & = i \circ j & (j \text{ is an identity on } A) \\
		& = j & (i \text{ is an identity on } A)
	\end{align*}
\end{proof}
\qed
\end{proof}

\begin{prop}
Let $A$ be a linearly ordered set. Then $A$ is well ordered if and only if it does not contain a subset of order type $\mathbb{N}^\mathrm{op}$.
\end{prop}

\begin{proof}
\pf
\step{1}{If $A$ is well ordered then it does not contain a subset of order type $\mathbb{N}^\mathrm{op}$.}
\begin{proof}
	\pf\ A subset of order type $\mathbb{N}^\mathrm{op}$ would be a subset with no least element.
\end{proof}
\step{2}{If $A$ is not well ordered then it contains a subset of order type $\mathbb{N}^\mathrm{op}$.}
\begin{proof}
	\step{a}{\assume{$A$ is not well ordered.}}
	\step{b}{\pick\ a nonempty subset $S$ with no least element.}
	\step{c}{\pick\ $a_0 \in S$}
	\step{d}{Extend to a sequence $(a_n)$ in $S$ such that $a_{n+1} < a_n$ for all $n$.}
	\step{e}{$\{ a_n : n \in \mathbb{N} \}$ has order type $\mathbb{N}^\mathrm{op}$.}
\end{proof}
\qed
\end{proof}

\begin{cor}
Let $A$ be a linearly ordered set. If every countable subset of $A$ is well ordered, then $A$ is well ordered.
\end{cor}

\begin{df}
Given $f : A \rightarrow B$ and an object $C$, define the function $f^* : \mathcal{C}[B,C] \rightarrow \mathcal{C}[A,C]$ by $f^*(g) = g \circ f$.
\end{df}

\begin{df}
Given $f : A \rightarrow B$ and an object $C$, define the function $f_* : \mathcal{C}[C,A] \rightarrow \mathcal{C}[C,B]$ by $f_*(g) = f \circ g$.
\end{df}

\subsection{Monomorphisms}

\begin{df}[Monomorphism]
Let $f : A \rightarrow B$. Then $f$ is \emph{monic} or a \emph{monomorphism}, $f : A \rightarrowtail B$, iff, for any object $X$ and functions $x,y : X \rightarrow A$, if $f \circ x = f \circ y$ then $x = y$.
\end{df}

\subsection{Epimorphisms}

\begin{df}[Epimorphism]
Let $f : A \rightarrow B$. Then $f$ is \emph{epic} or an \emph{epimorphism}, $f : A \twoheadrightarrow B$, iff, for any object $X$ and functions $x,y : B \rightarrow X$, if $x \circ f = y \circ f$ then $x = y$.
\end{df}

\subsection{Sections and Retractions}

\begin{df}[Section, Retraction]
Let $r : A \rightarrow B$ and $s : B \rightarrow A$. Then $r$ is a \emph{retraction} of $s$, and $s$ is a \emph{section} of $r$, iff $rs = \id{B}$.
\end{df}

\begin{prop}
\label{prop:section_is_retraction}
Let $f : A \rightarrow B$ and $r,s : B \rightarrow A$. If $r$ is a retraction of $f$ and $s$ is a section of $f$ then $r = s$.
\end{prop}

\begin{proof}
\pf
\begin{align*}
r & = r \id{B} & (\text{Unit Law}) \\
& = r f s & (\text{$s$ is a section of $f$}) \\
& = \id{A} s & (\text{$r$ is a retraction of $f$}) \\
& = s & (\text{Unit Law}) & \qed
\end{align*}
\end{proof}

\begin{prop}
Every section is monic.
\end{prop}

\begin{proof}
\pf
\step{1}{\pflet{$s : B \rightarrow A$ be a section of $r : A \rightarrow B$.}}
\step{2}{\pflet{$X$ be an object and $x,y : X \rightarrow B$}}
\step{3}{\assume{$s \circ x = s \circ y$}}
\step{4}{$x = y$}
\begin{proof}
	\pf\ $x = r \circ s \circ x = r \circ s \circ y = y$.
\end{proof}
\qed
\end{proof}

\begin{prop}
Every retraction is epic.
\end{prop}

\begin{proof}
\pf\ Dual. \qed
\end{proof}

\subsection{Isomorphisms}

\begin{df}[Isomorphism]
A morphism $f : A \rightarrow B$ is an \emph{isomorphism}, $f : A \cong B$, iff there exists a morphism $\inv{f} : B \rightarrow A$ that is both a retraction and section of $f$.

Objects $A$ and $B$ are \emph{isomorphic}, $A \cong B$, iff there exists an isomorphism between them.
\end{df}

\begin{prop}
The inverse of an isomorphism is unique.
\end{prop}

\begin{proof}
\pf\ From Proposition \ref{prop:section_is_retraction}. \qed
\end{proof}

\begin{prop}
If $f : A \cong B$ then $\inv{f} : B \cong A$ and $\inv{(\inv{f})} = f$.
\end{prop}

\begin{proof}
\pf\ Since $f\inv{f} = \id{B}$ and $\inv{f}f = \id{A}$. \qed
\end{proof}

Isomorphism.

Define the opposite category.

Slice categories

\begin{df}
Let $\mathcal{C}$ be a category and $B \in \mathcal{C}$. The category $\mathcal{C}_B^B$ of objects \emph{over and under} $B$ is the category with:
\begin{itemize}
\item objects all triples $(X,u,p)$ such that $u : B \rightarrow X$ and $p : X \rightarrow B$
\item morphisms $f : (X,u,p) \rightarrow (Y,u',p')$ all morphisms $f : X \rightarrow Y$ such that $fu=u'$ and $p'f=p$.
\end{itemize}
\end{df}

\begin{prop}
\[ \mathcal{C}_B^B \cong (\mathcal{C} / B) \backslash \id{B} \cong (\mathcal{C} \backslash B) / \id{B} \]
\end{prop}

$(B, \id{B}, \id{B})$ is the zero object in $\mathcal{C}_B^B$.

\subsection{Initial Objects}

\begin{df}[Initial Object]
An object $I$ is \emph{initial} iff, for any object $X$, there exists exactly one morphism $I \rightarrow X$.
\end{df}

\begin{prop}
The empty set is initial in $\Set$.
\end{prop}

\begin{proof}
\pf\ For any set $A$, the nowhere-defined function is the unique function $\emptyset \rightarrow A$. \qed
\end{proof}

\begin{prop}
\label{prop:initial_unique}
If $I$ and $I'$ are initial objects, then there exists a unique isomorphism $I \cong I'$.
\end{prop}

\begin{proof}
\pf
\step{1}{\pflet{$i : I \rightarrow I'$ be the unique morphism $I \rightarrow I'$.}}
\step{2}{\pflet{$\inv{i} : I' \rightarrow I$ be the unique morphism $I' \rightarrow I$.}}
\step{3}{$i \inv{i} = \id{I'}$}
\begin{proof}
	\pf\ There is only one morphism $I' \rightarrow I'$.
\end{proof}
\step{4}{$\inv{i} i = \id{I}$}
\begin{proof}
	\pf\ There is only one morphism $I \rightarrow I$.
\end{proof}
\qed
\end{proof}

\subsection{Terminal Objects}

\begin{df}[Terminal Object]
An object $T$ is \emph{terminal} iff, for any object $X$, there exists exactly one morphism $X \rightarrow T$.
\end{df}

\begin{prop}
1 is terminal in $\Set$.
\end{prop}

\begin{proof}
\pf\ For any set $A$, the constant function to $*$ is the only function $A \rightarrow 1$. \qed
\end{proof}

\begin{prop}
If $T$ and $T'$ are terminal objects, then there exists a unique isomorphism $T \cong T'$.
\end{prop}

\begin{proof}
\pf\ Dual to Proposition \ref{prop:initial_unique}. \qed
\end{proof}

\subsection{Zero Objects}

\begin{df}[Zero Object]
An object $Z$ is a \emph{zero object} iff it is an initial object and a terminal object.
\end{df}

\begin{df}[Zero Morphism]
Let $\mathcal{C}$ be a category with a zero object $Z$. Let $A,B \in \mathcal{C}$. The \emph{zero morphism} $A \rightarrow B$ is the unique morphism $A \rightarrow Z \rightarrow B$.
\end{df}

\begin{prop}
There is no zero object in $\Set$.
\end{prop}

\begin{proof}
\pf\ Since $\emptyset \not\approx 1$. \qed
\end{proof}

\subsection{Triads}

\begin{df}[Triad]
Let $\mathcal{C}$ be a category. A \emph{triad} consists of objects $X$, $Y$, $M$ and morphisms $\alpha : X \rightarrow M$, $\beta : Y \rightarrow M$. We call $M$ the \emph{codomain} of the triad.
\end{df}

\subsection{Cotriads}

\begin{df}[Cotriad]
Let $\mathcal{C}$ be a category. A \emph{cotriad} consists of objects $X$, $Y$, $W$ and morphisms $\xi : W \rightarrow X$, $\eta : W \rightarrow Y$. We call $W$ the \emph{domain} of the triad.
\end{df}

\subsection{Pullbacks}

\begin{df}[Pullback]
A diagram
\[ \xymatrix{
W \ar[r]^\xi \ar[d]_\eta & X \ar[d]^\alpha \\
Y \ar[r]_\beta & M
} \]
is a \emph{pullback} iff $\alpha \xi = \beta \eta$ and, for every object $Z$ and morphism $f : Z \rightarrow X$ and $g : Z \rightarrow Y$ such that $\alpha f = \beta g$, there exists a unique $h : Z \rightarrow W$ such that $\xi h = f$ and $\eta h = g$.

In this case we also say that $\eta$ is the \emph{pullback} of $\beta$ along $\alpha$.
\end{df}

\begin{prop}
\label{prop:pullback_unique}
If $\xi : W \rightarrow X$ and $\eta : W \rightarrow Y$ form a pullback of $\alpha : X \rightarrow M$ and $\beta : Y \rightarrow M$, and $\xi' : W' \rightarrow X$ and $\eta' : W' \rightarrow Y$ also form the pullback of $\alpha$ and $\beta$, then there exists a unique isomorphism $\phi : W \cong W'$ such that $\eta' \phi = \eta$ and $\xi' \phi = \xi$.
\end{prop}

\begin{proof}
\pf
\step{1}{\pflet{$\phi : W \rightarrow W'$ be the unique morphism such that $\eta' \phi = \eta$ and $\xi' \phi = \xi$.}}
\step{2}{\pflet{$\inv{\phi} : W' \rightarrow W$ be the unique morphism such that $\eta \inv{\phi} = \eta'$ and $\xi \inv{\phi} = \xi'$.}}
\step{3}{$\phi \inv{\phi} = \id{W'}$}
\begin{proof}
	\pf\ Each is the unique $x : W' \rightarrow W'$ such that $\eta' x = \eta'$ and $\xi' x = \xi'$.
\end{proof}
\step{4}{$\inv{\phi} \phi = \id{W}$}
\begin{proof}
	\pf\ Each is the unique $x : W \rightarrow W$ such that $\eta x = \eta$ and $\xi x = \xi$.
\end{proof}
\qed
\end{proof}

\begin{prop}
\label{prop:pullback_id}
For any morphism $h : A \rightarrow B$, the following diagram is a pullback diagram.
\[ \xymatrix{
A \ar[r]^h \ar@{=}[d] & B \ar@{=}[d] \\
A \ar[r]_h & B
} \]
\end{prop}

\begin{proof}
\pf
\step{1}{\pflet{$Z$ be an object.}}
\step{2}{\pflet{$f : Z \rightarrow B$ and $g : Z \rightarrow A$ satisfy $\id{B} f = hg$}}
\step{3}{$g : Z \rightarrow B$ is the unique morphism such that $\id{A} g = g$ and $hg = f$.}
\qed
\end{proof}

\begin{prop}
\label{prop:pullback_iso}
The pullback of an isomorphism is an isomorphism.
\end{prop}

\begin{proof}
\pf
\step{1}{\pflet{
\[ \xymatrix{
W \ar[r]^\xi \ar[d]_\eta & X \ar[d]^\alpha \\
Y \ar[r]_\beta & M
} \]
be a pullback diagram.}}
\step{2}{\assume{$\beta$ is an isomorphism.}}
\step{3}{\pflet{$\inv{\xi}$ be the unique morphism $X \rightarrow W$ such that $\xi \inv{\xi} = \id{X}$ and $\eta \inv{\xi} = \inv{\beta} \alpha$.}}
\begin{proof}
	\pf\ This exists since $\alpha \id{X} = \beta \inv{\beta} \alpha = \alpha$.
\end{proof}
\step{4}{$\inv{\xi} \xi = \id{W}$}
\begin{proof}
	\pf\ Each is the unique $x : W \rightarrow W$ such that $\xi x = \xi$ and $\eta x = \eta$.
\end{proof}
\qed
\end{proof}

\begin{prop}
\label{prop:pullback_slice_under}
Let $\beta : (Y,y) \rightarrow (M,m)$ and $\alpha : (X,x) \rightarrow (M,m)$ in $\mathcal{C} \backslash A$. Let
\[ \xymatrix{
W \ar[r]^\xi \ar[d]_\eta & X \ar[d]^\alpha \\
Y \ar[r]_\beta & M
} \]
be a pullback in $\mathcal{C}$. Let $w : A \rightarrow W$ be the unique morphism such that $\xi w = x$ and $\eta w = y$. Then $\xi : (W,w) \rightarrow (X,x)$ and $\eta : (W,w) \rightarrow (Y,y)$ is the pullback of $\beta$ and $\alpha$ in $\mathcal{C} \backslash A$.
\end{prop}

\begin{proof}
\pf
\step{1}{\pflet{$(Z,z) \in \mathcal{C} \backslash A$}}
\step{2}{\pflet{$f : (Z,z) \rightarrow (X,x)$ and $g : (Z,z) \rightarrow (Y,y)$ satisfy $\alpha f = \beta g$.}}
\step{3}{\pflet{$h : Z \rightarrow W$ be the unique morphism such that $\xi h = f$ and $\eta h = g$.}}
\step{4}{$hz=w$}
\begin{proof}
	\step{a}{$\xi h z = \xi w$}
	\begin{proof}
		\pf
		\begin{align*}
			\xi h z & = f z & (\text{\stepref{3}}) \\
			& = x & (\text{\stepref{2}}) \\
			& = \xi w
		\end{align*}
	\end{proof}
	\step{b}{$\eta h z = \eta w$}
	\begin{proof}
		\pf\ Similar.
	\end{proof}
\end{proof}
\step{4}{$h : (Z,z) \rightarrow (W,w)$}
\qed
\end{proof}

\begin{prop}
\label{prop:pullback_slice_over}
Let $\beta : (Y,y) \rightarrow (M,m)$ and $\alpha : (X,x) \rightarrow (M,m)$ in $\mathcal{C} / A$. Let
\[ \xymatrix{
W \ar[r]^\xi \ar[d]_\eta & X \ar[d]^\alpha \\
Y \ar[r]_\beta & M
} \]
be a pullback in $\mathcal{C}$. Let $w = x\xi : W \rightarrow A$. Then $\xi : (W,w) \rightarrow (X,x)$ and $\eta : (W,w) \rightarrow (Y,y)$ form a pullback of $\alpha$ and $\beta$ in $\mathcal{C} / A$.
\end{prop}

\begin{proof}
\pf
\step{1}{$\eta : (W,w) \rightarrow (Y,y)$}
\begin{proof}
	\pf
	\begin{align*}
		y \eta & = m \beta \eta \\
		& = m \alpha \xi \\
		& = x \xi \\
		& = w
	\end{align*}
\end{proof}
\step{2}{\pflet{$(Z,z) \in \mathcal{C} / A$}}
\step{3}{\pflet{$f : (Z,z) \rightarrow (X,x)$ and $g : (Z,z) \rightarrow (Y,y)$ satisfy $\alpha f = \beta g$.}}
\step{4}{\pflet{$h : Z \rightarrow W$ be the unique morphism such that $\xi h = f$ and $\eta h = g$.}}
\step{5}{$h : (Z,z) \rightarrow (W,w)$}
\begin{proof}
	\pf
	\begin{align*}
		w h & = x \xi h \\
		& = xf & (\text{\stepref{4}}) \\
		& = z & (\text{\stepref{3}})
	\end{align*}
\end{proof}
\qed
\end{proof}

\begin{prop}
In $\Set$, let $\alpha : X \rightarrow M$ and $\beta : Y \rightarrow M$. Let $W = \{ (x,y) \in X \times Y : \alpha(x) = \beta(y) \}$ with inclusion $i : W \rightarrow X \times Y$. Let $\xi = \pi_1 i : W \rightarrow X$ and $\eta : \pi_2 i : W \rightarrow Y$. Then $\xi$ and $\eta$ form the pullback of $\alpha$ and $\beta$.
\end{prop}

\begin{proof}
\pf
\step{1}{$\alpha \xi = \beta \eta$}
\begin{proof}
	\pf\ For $w \in W$, if $i(w) = (x,y)$ then then $\alpha(\xi(w)) = \alpha(x) = \beta(y) = \beta(\eta(w))$.
\end{proof}
\step{2}{For every set $Z$ and functions $f : Z \rightarrow X$, $g : Z \rightarrow Y$ such that $\alpha f = \beta g$, there exists a unique $h : Z \rightarrow W$ such that $\xi h = f$ and $\eta h = g$}
\begin{proof}
	\pf\ For $z \in Z$, let $h(z)$ be the unique element of $W$ such that $i(h(z)) = (f(z),g(z))$.
\end{proof}
\qed
\end{proof}

Pullback lemma

\subsection{Pushouts}

\begin{df}[Pushout]
A diagram
\begin{equation}
\label{eq:pushout}
\xymatrix{
W \ar[r]^\xi \ar[d]_\eta & X \ar[d]^\alpha \\
Y \ar[r]_\beta & M
}
\end{equation}
is a \emph{pushout} iff $\alpha \xi = \beta \eta$ and, for every object $Z$ and morphism $f : X \rightarrow Z$ and $g : Y \rightarrow Z$ such that $f \xi = g \eta$, there exists a unique $h : M \rightarrow Z$ such that $h \alpha = f$ and $h \beta = g$.

We also say that $\beta$ is the \emph{pushout} of $\xi$ along $\eta$.
\end{df}

\begin{prop}
If $\alpha : X \rightarrow M$ and $\beta : Y \rightarrow M$ form a pushout of $\xi : W \rightarrow X$ and $\eta : W \rightarrow Y$, and $\alpha' : X \rightarrow M'$ and $\beta' : Y \rightarrow M'$ also form a pushout of $\xi$ and $\eta$, then there exists a unique isomorphism $\phi : M \cong M'$ such that $\phi \alpha = \alpha'$ and $\phi \beta = \beta'$.
\end{prop}

\begin{proof}
\pf\ Dual to Proposition \ref{prop:pullback_unique}. \qed
\end{proof}

\begin{prop}
For any morphism $h : A \rightarrow B$, the following diagram is a pushout diagram.
\[ \xymatrix{
A \ar[r]^h \ar@{=}[d] & B \ar@{=}[d] \\
A \ar[r]_h & B
} \]
\end{prop}

\begin{proof}
\pf\ Dual to Proposition \ref{prop:pullback_id}.
\end{proof}

\begin{prop}
The diagram (\ref{eq:pushout}) is a pushout in $\mathcal{C}$ iff it is a pullback in $\mathcal{C}^\mathrm{op}$.
\end{prop}

\begin{proof}
\pf\ Immediate from definitions. \qed
\end{proof}

\begin{prop}
The pushout of an isomorphism is an isomorphism.
\end{prop}

\begin{proof}
\pf\ Dual to Proposition \ref{prop:pullback_iso}. \qed
\end{proof}

\begin{prop}
Let $\xi : (W,w) \rightarrow (X,x)$ and $\eta : (W,w) \rightarrow (Y,y)$ in $\mathcal{C} \backslash A$. Let
\[ \xymatrix{
W \ar[r]^\xi \ar[d]_\eta & X \ar[d]^\alpha \\
Y \ar[r]_\beta & M
} \]
be a pushout in $\mathcal{C}$. Let $m : = \alpha x : A \rightarrow M$. Then $\alpha : (X,x) \rightarrow (M,m)$ and $\beta : (Y,y) \rightarrow (M,m)$ is the pushout of $\xi$ and $\eta$ in $\mathcal{C} \backslash A$.
\end{prop}

\begin{proof}
\pf\ Dual to Proposition \ref{prop:pullback_slice_over}. \qed
\end{proof}

\begin{prop}
Let $\xi : (W,w) \rightarrow (X,x)$ and $\eta : (W,w) \rightarrow (Y,y)$ in $\mathcal{C} / A$. Let
\[ \xymatrix{
W \ar[r]^\xi \ar[d]_\eta & X \ar[d]^\alpha \\
Y \ar[r]_\beta & M
} \]
be a pushout in $\mathcal{C}$. Let $m : M \rightarrow A$ be the unique morphism such that $m \alpha = x$ and $m \beta = y$. Then $\alpha : (X,x) \rightarrow (M,m)$ and $\beta : (Y,y) \rightarrow (M,m)$ is the pushout of $\xi$ and $\eta$ in $\mathcal{C} \backslash A$.
\end{prop}

\begin{proof}
\pf\ Dual to Proposition \ref{prop:pullback_slice_under}. \qed
\end{proof}

\begin{prop}
$\Set$ has pushouts.
\end{prop}

\begin{proof}
\pf
\step{1}{\pflet{$\xi : W \rightarrow X$ and $\eta : W \rightarrow Y$.}}
\step{2}{\pflet{$\sim$ be the equivalence relation on $X + Y$ generated by $\xi(w) \sim \eta(w)$ for all $w \in W$}}
\step{3}{\pflet{$M = (X + Y) / \sim$ with canonical projection $\pi : X + Y \twoheadrightarrow M$.}}
\step{33}{\pflet{$\alpha = \pi \circ \kappa_1 : X \rightarrow M$}}
\step{34}{\pflet{$\beta = \pi \circ \kappa_2 : Y \rightarrow M$}}
\step{4}{\pflet{$Z$ be any set, $f : X \rightarrow Z$ and $g : Y \rightarrow Z$.}}
\step{5}{\assume{$f \xi = g \eta$}}
\step{6}{\pflet{$h : X + Y \rightarrow Z$ be the function defined by $h(x) = f(x)$ and $h(y) = g(y)$ for $x \in X$ and $y \in Y$}}
\step{7}{$h$ respects $\sim$}
\begin{proof}
	\pf\ For $w \in W$ we have
	\begin{align*}
		h(\xi(w)) & = f(\xi(w)) & (\text{\stepref{6}}) \\
		& = g(\eta(w)) & (\text{\stepref{5}}) \\
		& = h(\eta(w)) & (\text{\stepref{6}})
	\end{align*}
\end{proof}
\step{8}{\pflet{$\overline{h} : M \rightarrow Z$ be the induced function.}}
\step{9}{$\overline{h} \alpha = f$}
\begin{proof}
	\pf
	\begin{align*}
		\overline{h}(\alpha(x)) & = \overline{h}(\pi(\kappa_1(x))) \\
		& = h(\kappa_1(x)) \\
		& = f(x)
	\end{align*}
\end{proof}
\step{10}{$\overline{h} \beta = g$}
\begin{proof}
	\pf\ Similar.
\end{proof}
\step{11}{For all $k : M \rightarrow Z$, if $k \alpha = f$ and $k \beta = g$ then $k = \overline{h}$.}
\begin{proof}
	\pf
	\begin{align*}
		k(\pi(\kappa_1(x))) & = k(\alpha(x)) \\
		& = f(x) \\
		k(\pi(\kappa_2(y))) & = k(\beta(y)) \\
		& = g(y) \\
		\therefore k \circ \pi & = h \\
		\therefore k & = \overline{h}
	\end{align*}
\end{proof}
\qed
\end{proof}

\begin{df}
Let $u : A \rightarrowtail X$ be an injection. The \emph{pointed set obtained from $X$ by collapsing $(A,u)$}, denoted $X / (A,u)$, is the pushout
\[ \xymatrix{
A \ar[r] \ar[d]^u & 1 \ar[d]_{*} \\
X \ar[r] & X / (A,u)
} \]
\end{df}

\begin{prop}
In $\mathbf{Set}_*$, any two morphisms $1 \rightarrow X$ and $1 \rightarrow Y$ have a pushout.
\end{prop}

\begin{proof}
\pf\ The pushout of $a : (1,*) \rightarrow (X,x)$ and $b : (1,*) \rightarrow (Y,y)$ is $(X+Y/\sim, x)$ where $\sim$ is the equivalence relation generated by $x \sim y$. \qed
\end{proof}

\begin{df}[Wedge]
The \emph{wedge} of pointed sets $X$ and $Y$, $X \vee Y$, is the pushout of the unique morphism $1 \rightarrow X$ and $1 \rightarrow Y$.
\end{df}

\begin{df}[Smash]
Let $X$ and $Y$ be pointed sets. Let $\xi : X \vee Y \rightarrow X$ be the unique morphism such that the following diagram commutes.
\[ \xymatrix{
1 \ar[r] \ar[d] & X \ar[d] \ar@{=}@/^/[ddr] \\
Y \ar[r] \ar@/_/[drr]_0 & X \vee Y \ar[dr]^\xi \\
& & X
} \]
Let $\eta : X \vee Y \rightarrow Y$ be the unique morphism such that the following diagram commutes.
\[ \xymatrix{
1 \ar[r] \ar[d] & X \ar[d] \ar@/^/[ddr]^0 \\
Y \ar[r] \ar@{=}@/_/[drr] & X \vee Y \ar[dr]^\eta \\
& & Y
} \]
Let $\zeta = \langle \xi, \eta \rangle : X \vee Y \rightarrow X \times Y$. The \emph{smash} of $X$ and $Y$, $X \wedge Y$, is the result of collapsing $X \times Y$ with respect to $\zeta$.
\end{df}

Pushout lemma

\subsection{Subcategories}

\begin{df}[Subcategory]
A \emph{subcategory} $\mathcal{C}'$ of a category $\mathcal{C}$ consists of:
\begin{itemize}
\item a subset $\Ob{\mathcal{C}'}$ of $\mathcal{C}$
\item for all $A,B \in \Ob{\mathcal{C}'}$, a subset $\mathcal{C}'[A,B] \subseteq \mathcal{C}[A,B]$
\end{itemize}
such that:
\begin{itemize}
\item for all $A \in \Ob{\mathcal{C}'}$, we have $\id{A} \in \mathcal{C}'[A,A]$
\item for all $f \in \mathcal{C}'[A,B]$ and $g \in \mathcal{C}'[B,C]$, we have $g \circ f \in \mathcal{C}'[A,C]$.
\end{itemize}

It is a \emph{full} subcategory iff, for all $A,B \in \Ob{\mathcal{C}'}$, we have $\mathcal{C}'[A,B] = \mathcal{C}[A,B]$.
\end{df}

\subsection{Opposite Category}

\begin{df}[Opposite Category]
For any category $\mathcal{C}$, the \emph{opposite} category $\op{\mathcal{C}}$ is the category with
\begin{itemize}
\item $\Ob{\op{\mathcal{C}}} = \Ob{\mathcal{C}}$
\item $\op{\mathcal{C}}[A,B] = \mathcal{C}[B,A]$
\item Given $f \in \op{\mathcal{C}}[A,B]$ and $g \in \op{\mathcal{C}}[B,C]$, their composite in $\op{\mathcal{C}}$ is $f \circ g$, where $\circ$ is composition in $\mathcal{C}$.
\end{itemize}
\end{df}

\begin{prop}
An object is initial in $\mathcal{C}$ iff it is terminal in $\op{\mathcal{C}}$.
\end{prop}

\begin{proof}
\pf\ Immediate from definitions. \qed
\end{proof}

\begin{prop}
An object is terminal in $\mathcal{C}$ iff it is initial in $\op{\mathcal{C}}$.
\end{prop}

\begin{proof}
\pf\ Immediate from definitions. \qed
\end{proof}

\begin{cor}
If $T$ and $T'$ are terminal objects in $\mathcal{C}$ then there exists a unique isomorphism $T \cong T'$.
\end{cor}

\subsection{Groupoids}

\begin{df}[Groupoid]
A \emph{groupoid} is a category in which every morphism is an isomorphism.
\end{df}

\subsection{Concrete Categories}

\begin{df}[Concrete Category]
A \emph{concrete category} $\mathcal{C}$ consists of:
\begin{itemize}
\item a set $\Ob{\mathcal{C}}$ of \emph{objects}
\item for any object $A \in \Ob{\mathcal{C}}$, a set $|A|$
\item for any objects $A,B \in \Ob{\mathcal{C}}$, a set of functions $\mathcal{C}[A,B] \subseteq |B|^{|A|}$
\end{itemize}
such that:
\begin{itemize}
\item for any $f \in \mathcal{C}[A,B]$ and $g \in \mathcal{C}[B,C]$, we have $g \circ f \in \mathcal{C}[A,C]$
\item for any object $A$ we have $\id{|A|} \in \mathcal{C}[A,A]$.
\end{itemize}
\end{df}

\subsection{Power of Categories}

\begin{df}
Let $\mathcal{C}$ be a category and $J$ a set. The category $\mathcal{C}^J$ is the category with:
\begin{itemize}
\item objects all $J$-indexed families of objects of $\mathcal{C}$
\item morphisms $\{X_j\}_{j \in J} \rightarrow \{Y_j\}_{j \in J}$ all families $\{f_j\}_{j \in J}$ where $f_j : X_j \rightarrow Y_j$
\end{itemize}
\end{df}

\subsection{Arrow Category}

\begin{df}[Arrow Category]
Let $\mathcal{C}$ be a category. The \emph{arrow category} $\mathcal{C}^\rightarrow$ is the category with:
\begin{itemize}
\item objects all triples $(A,B,f)$ where $f : A \rightarrow B$ in $\mathcal{C}$
\item morphisms $(A,B,f) \rightarrow (C,D,g)$ all pairs $(u : A \rightarrow C, v : B \rightarrow D)$ such that $vf=gu$.
\end{itemize}
\end{df}

\subsection{Slice Category}

\begin{df}[Slice Category]
Let $\mathcal{C}$ be a category and $A \in \mathcal{C}$. The \emph{slice category under $A$}, $\mathcal{C} \backslash A$, is the category with:
\begin{itemize}
\item objects all pairs $(B,f)$ where $B \in \mathcal{C}$ and $f : A \rightarrow B$
\item morphisms $(B,f) \rightarrow (C,g)$ are morphisms $u : B \rightarrow C$ such that $uf=g$.
\end{itemize}

We identify this with the subcategory of $\mathcal{C}^\rightarrow$ formed by mapping $(B,f)$ to $(A,B,f)$ and $u$ to $(\id{A},u)$.
\end{df}

\begin{prop}
\label{prop:retraction_in_slice}
If $s : (B,f) \rightarrow (C,g)$ in $\mathcal{C} \backslash A$, then any retraction of $s$ in $\mathcal{C}$ is a retraction of $s$ in $\mathcal{C} \backslash A$.
\end{prop}

\begin{proof}
\pf
\step{1}{\pflet{$r : C \rightarrow B$ be a retraction of $s$ in $\mathcal{C}$.}}
\step{2}{$rg = f$}
\begin{proof}
	\pf\ $rg = rsf = f$.
\end{proof}
\step{3}{$r : (C,g) \rightarrow (B,f)$ in $\mathcal{C} \backslash A$}
\step{4}{$rs = \id{(B,f)}$}
\begin{proof}
	\pf\ Because composition is inherited from $\mathcal{C}$.
\end{proof}
\qed
\end{proof}

\begin{prop}
\label{prop:initial_in_slice}
$\id{A}$ is the initial object in $\mathcal{C} \backslash A$.
\end{prop}

\begin{proof}
\pf\ For any $(B,f) \in \mathcal{C} \backslash A$, we have $f$ is the only morphism $A \rightarrow B$ such that $f \id{A} = f$. \qed
\end{proof}

\begin{prop}
\label{prop:zero_in_slice}
If $A$ is terminal in $\mathcal{C}$ then $\id{A}$ is the zero object in $\mathcal{C} \backslash A$.
\end{prop}

\begin{proof}
\pf\ For any $(B,f) \in \mathcal{C} \backslash A$, the unique morphism $! : B \rightarrow A$ is the unique morphism such that $!f = \id{A}$. \qed
\end{proof}

\begin{df}[Pointed Sets]
The \emph{category of pointed sets} is $\Set \backslash 1$.
\end{df}

\begin{df}
Let $\mathcal{C}$ be a category and $A \in \mathcal{C}$. The \emph{slice category over $A$}, $\mathcal{C} / A$, is the category with:
\begin{itemize}
\item objects all pairs $(B,f)$ with $f : B \rightarrow A$
\item morphisms $u : (B,f) \rightarrow (C,g)$ all morphisms $u : B \rightarrow C$ such that $gu=f$.
\end{itemize}
\end{df}

\begin{prop}
Let $u : (B,f) \rightarrow (C,g) : \mathcal{C} / A$. Any section of $u$ in $\mathcal{C}$ is a section of $u$ in $\mathcal{C} / A$.
\end{prop}

\begin{proof}
\pf\ Dual to Proposition \ref{prop:retraction_in_slice}. \qed
\end{proof}

\begin{prop}
$\id{A}$ is terminal in $\mathcal{C} / A$.
\end{prop}

\begin{proof}
\pf\ Dual to Proposition \ref{prop:initial_in_slice}. \qed
\end{proof}

\begin{prop}
If $A$ is initial in $\mathcal{C}$ then $\id{A}$ is the zero object in $\mathcal{C} / A$.
\end{prop}

\begin{proof}
\pf\ Dual to Proposition \ref{prop:zero_in_slice}. \qed
\end{proof}

\begin{df}
Let $A \in \mathcal{C}$.
The category of objects \emph{over and under} $A$, written $\mathcal{C}_A^A$, is the category with:
\begin{itemize}
\item objects all triples $(X,u,p)$ where $u : A \rightarrow X$, $p : X \rightarrow A$ and $pu = \id{A}$
\item morphism $f : (X,u,p) \rightarrow (Y,v,q)$ all morphisms $f : X \rightarrow Y$ such that $fu = v$ and $qf = p$
\end{itemize}
\end{df}

\begin{prop}
$(A, \id{A}, \id{A})$ is the zero object in $\mathcal{C}_A^A$.
\end{prop}

\begin{proof}
\pf\ For any object $(X,u,p)$, we have $p$ is the unique morphism $(X,u,p) \rightarrow (A, \id{A}, \id{A})$, and $u$ is the unique morphism $(A, \id{A}, \id{A}) \rightarrow (X,u,p)$. \qed
\end{proof}

\begin{df}[Fibre Collapsing]
Let $B$ be a set. Let $u : (A,a) \rightarrow (X,x)$ in $\Set / B$. Form the pushout
\[ \xymatrix{
A \ar[r]^a \ar[d]^u & B \ar[d]_j \\
X \ar[r]_i & C
} \]
Let $c : C \rightarrow B$ be the unique morphism such that $cj = \id{B}$ and $ci = x$. Then $(C,j,c) \in \Set_B^B$ is called the set over and under $B$ obtained from $X$ by \emph{fibre collapsing} with respect to $u$. If $(A,u)$ is a subset of $X$, we denote this set over and under $B$ by $X /_B (A,u)$.
\end{df}

\begin{df}[Fibre Wedge]
Let $B$ be a small set. Let $(X,u_X,p_X),(Y, u_Y, p_Y) \in \Set_B^B$. The \emph{fibre wedge} of $X$ and $Y$ is the pushout of $u_X$ and $u_Y$:
\[ \xymatrix{
B \ar[r]^{u_X} \ar[d]^{u_Y} & X \ar[d] \\
Y \ar[r] & X \vee_B Y
} \]
\end{df}

\begin{df}[Fibre Smash]
Let $X, Y \in \Set_B^B$. Let $\xi : X \vee_B Y \rightarrow X$ be the unique morphism such that the following diagram commutes.
\[ \xymatrix{
1 \ar[r] \ar[d] & X \ar[d] \ar@{=}@/^/[ddr] \\
Y \ar[r] \ar@/_/[drr]_0 & X \vee_B Y \ar[dr]^\xi \\
& & X
} \]
Let $\eta : X \vee_B Y \rightarrow Y$ be the unique morphism such that the following diagram commutes.
\[ \xymatrix{
1 \ar[r] \ar[d] & X \ar[d] \ar@/^/[ddr]^0 \\
Y \ar[r] \ar@{=}@/_/[drr] & X \vee_B Y \ar[dr]^\eta \\
& & Y
} \]
Let $\zeta = \langle \xi, \eta \rangle : X \vee_B Y \rightarrow X \times Y$. The \emph{fibre smash} of $X$ and $Y$, $X \wedge_B Y$, is the result of collapsing $X \times Y$ with respect to $\zeta$.
\end{df}

\begin{prop}
$\mathbf{Set}$ has products and coproducts.
\end{prop}

\begin{prop}
Let $\mathcal{C}$ be a category. Let $\{X_\alpha\}_{\alpha \in I}$ be a family of objects in $\mathcal{C}$ and $Z \in \mathcal{C}$. Let $\coprod_{\alpha \in I} X_\alpha$ be the coproduct of $\{ X_\alpha \}_{\alpha \in I}$. Then
\[ \mathcal{C}[\coprod_{\alpha \in I} X_\alpha, Z] \approx \prod_{\alpha \in I} \mathcal{C}[X_\alpha, Z] \enspace . \]
\end{prop}

\begin{prop}
Let $\mathcal{C}$ be a category. Let $\{X_\alpha\}_{\alpha \in I}$ be a family of objects in $\mathcal{C}$ and $Z \in \mathcal{C}$. Let $\prod_{\alpha \in I} X_\alpha$ be the product of $\{ X_\alpha \}_{\alpha \in I}$. Then
\[ \mathcal{C}[Z, \prod_{\alpha \in I} X_\alpha] \approx \prod_{\alpha \in I} \mathcal{C}[Z, X_\alpha] \enspace . \]
\end{prop}

\begin{prop}
A product in $\mathcal{C}$ constitutes a product in $\mathcal{C} \backslash A$.
\end{prop}

\begin{prop}
A coproduct in $\mathcal{C}$ constitutes a product in $\mathcal{C} / A$.
\end{prop}

\section{Functors}

\begin{df}[Functor]
Let $\mathcal{C}$ and $\mathcal{D}$ be categories. A \emph{functor} $F : \mathcal{C} \rightarrow \mathcal{D}$ consists of:
\begin{itemize}
\item a function $F : \mathrm{Ob}(\mathcal{C}) \rightarrow \mathrm{Ob}(\mathcal{D})$
\item for every morphism $f : A \rightarrow B$ in $\mathcal{C}$, a morphism $Ff : FA \rightarrow FB$ in $\mathcal{D}$
\end{itemize}
such that:
\begin{itemize}
\item for all $A \in \mathrm{Ob}(C)$ we have $F \id{A} = \id{FA}$
\item for any morphism $f : A \rightarrow B$ and $g : B \rightarrow C$ in $\mathcal{C}$, we have $F(g \circ f) = Fg \circ Ff$
\end{itemize}
\end{df}

\begin{prop}
Functors preserve isomorphisms.
\end{prop}

\begin{proof}
\pf
\step{1}{\pflet{$F : \mathcal{C} \rightarrow \mathcal{D}$ be a functor.}}
\step{2}{\pflet{$f : A \cong B$ in $\mathcal{C}$}}
\step{3}{$F \inv{f} \circ Ff = \id{FA}$}
\begin{proof}
	\pf
	\begin{align*}
		F \inv{f} \circ Ff & = F(\inv{f} \circ f) \\
		& = F \id{A} \\
		& = \id{FA}
	\end{align*}
\end{proof}
\step{4}{$Ff \circ F \inv{f} = \id{FB}$}
\begin{proof}
	\pf
	\begin{align*}
		Ff \circ F \inv{f} & = F(f \circ \inv{f}) \\
		& = F \id{B} \\
		& = \id{FB}
	\end{align*}
\end{proof}
\qed
\end{proof}

\begin{df}[Identity Functor]
For any category $\mathcal{C}$, the \emph{identity} functor on $\mathcal{C}$ is the functor $I_\mathcal{C} : \mathcal{C} \rightarrow \mathcal{C}$ defined by
\begin{align*}
I_\mathcal{C} A & := A & (A \in \mathcal{C}) \\
I_\mathcal{C} f & := f & (f : A \rightarrow B \text{ in } \mathcal{C})
\end{align*}
\end{df}

\begin{prop}
Let $F : \mathcal{C} \rightarrow \mathcal{D}$. If $r : A \rightarrow B$ is a retraction of $s : B \rightarrow A$ in $\mathcal{C}$ then $Fr$ is a retraction of $Fs$.
\end{prop}

\begin{proof}
\pf
\begin{align*}
Fr \circ Fs & = F(r \circ s) \\
& = F \id{B} \\
& = \id{FB} & \qed
\end{align*}
\end{proof}

\begin{cor}
Let $F : \mathcal{C} \rightarrow \mathcal{D}$. If $\phi : A \cong B$ is an isomorphism in $\mathcal{C}$ then $F \phi : FA \cong FB$ is an isomorphism in $\mathcal{D}$ with $\inv{(F \phi)} = F \inv{\phi}$.
\end{cor}

\begin{df}[Composition of Functors]
Given functors $F : \mathcal{C} \rightarrow \mathcal{D}$ and $G : \mathcal{D} \rightarrow \mathcal{E}$, the \emph{composite} functor $GF : \mathcal{C} \rightarrow \mathcal{E}$ is defined by
\begin{align*}
(GF)A & = G(FA) & (A \in \mathcal{C}) \\
(GF)f & = G(Ff) & (f : A \rightarrow B : \mathcal{C})
\end{align*}
\end{df}

\begin{df}[Category of Categories]
Let $\mathbf{Cat}$ be the category of small categories and functors.
\end{df}

\begin{df}[Isomorphism of Categories]
Let $F : \mathcal{C} \rightarrow \mathcal{D}$ be a functor. Then $F$ is an \emph{isomorphism of categories} iff there exists a functor $\inv{F} : \mathcal{D} \rightarrow \mathcal{C}$, the \emph{inverse} of $F$, such that $F \inv{F} = I_{\mathcal{D}}$ and $\inv{F} F = I_{\mathcal{C}}$.

Categories $\mathcal{C}$ and $\mathcal{D}$ are \emph{isomorphic}, $\mathcal{C} \cong \mathcal{D}$, iff there exists an isomorphism between them.
\end{df}

\begin{prop}
If $A$ is initial in $\mathcal{C}$ then $\mathcal{C} \backslash A \cong \mathcal{C}$.
\end{prop}

\begin{proof}
\pf
\step{1}{Define $F : \mathcal{C} \backslash A \rightarrow \mathcal{C}$ by
\begin{align*}
F (B,f) & = B \\
F (u : (B,f) \rightarrow (C,g)) & = u
\end{align*}}
\step{2}{Define $G : \mathcal{C} \rightarrow \mathcal{C} \backslash A$ by
\begin{align*}
G B & = (B, !_B) & \text{where $!_B$ is the unique morphism $A \rightarrow B$} \\
G (u : B \rightarrow C) & = u : (B, !_B) \rightarrow (C, !_C)
\end{align*}}
\step{3}{$FG = \id{\mathcal{C}}$}
\step{4}{$GF = \id{\mathcal{C} \backslash A}$}
\begin{proof}
	\pf\ Since $GF(B,f) = (B, !_B) = (B,f)$ because the morphism $A \rightarrow B$ is unique.
\end{proof}
\qed
\end{proof}

\begin{prop}
If $A$ is terminal in $\mathcal{C}$ then $\mathcal{C} / A \cong \mathcal{C}$.
\end{prop}

\begin{proof}
\pf\ Dual. \qed
\end{proof}

\begin{prop}
\[ \mathcal{C}_A^A \cong (\mathcal{C} / A) \backslash (A, \id{A}) \cong (\mathcal{C} \backslash A) / (A, \id{A}) \]
\end{prop}

\begin{proof}
\pf
\step{1}{Define a functor $F : \mathcal{C}_A^A \rightarrow (\mathcal{C} / A) \backslash (A, \id{A})$.}
\begin{proof}
	\step{a}{Given $A \stackrel{u}{\rightarrow} X \stackrel{p}{\rightarrow} A$ in $\mathcal{C}_A^A$, let $F(X,u,p) = ((X,p),u)$}
	\step{b}{Given $f : (A \stackrel{u}{\rightarrow} X \stackrel{p}{\rightarrow} A) \rightarrow (A \stackrel{v}{\rightarrow} Y \stackrel{q}{\rightarrow} A)$, let $Ff = f$.}
\end{proof}
\step{2}{Define a functor $G : (\mathcal{C} / A) \backslash (A, \id{A}) \rightarrow \mathcal{C}_A^A$.}
\step{3}{Define a functor $H : \mathcal{C}_A^A \rightarrow (\mathcal{C} \backslash A) / (A, \id{A})$.}
\step{4}{Define a functor $K : (\mathcal{C} \backslash A) / (A, \id{A}) \rightarrow \mathcal{C}_A^A$.}
\qed
\end{proof}

\begin{df}[Forgetful Functor]
For any concrete category $\mathcal{C}$, define the \emph{forgetful} functor $U : \mathcal{C} \rightarrow \Set$ by:
\begin{align*}
U A & = |A| \\
U f & = f
\end{align*}
\end{df}

\begin{df}[Switching Functor]
For any category $\mathcal{C}$, define the \emph{switching functor} $T : \mathcal{C} \times \mathcal{C} \rightarrow \mathcal{C} \times \mathcal{C}$ by
\begin{align*}
T(A,B) & = (B,A) \\
T(f,g) & = (g,f)
\end{align*}
\end{df}

\begin{df}[Reduction]
Let $\Phi : \Set \rightarrow \Set$ be a functor. The \emph{reduction} of $\Phi$ is the functor $\Phi^* : \Set_* \rightarrow \Set_*$ defined by: $\Phi^*(X,a)$ is the collapse of $\Phi(X)$ with respect to $\Phi(a) : \Phi(1) \rightarrowtail \Phi(X)$.
\end{df}

\begin{df}
Extend the wedge $\vee$ to a functor $\Set_* \times \Set_* \rightarrow \Set_*$ by defining, given $f : X \rightarrow X'$ and $g : Y \rightarrow Y'$, thene $f \vee g$ is the unique morphism that makes the following diagram commute.
\[ \xymatrix{
1 \ar[r] \ar[d] & X \ar[d] \ar[dr]^f \\
Y \ar[r] \ar[dr]_g & X \vee Y \ar[dr]^{f \vee g} & X' \ar[d] \\
& Y' \ar[r] & X' \vee Y'
} \]
\end{df}

\begin{df}
Extend smash to a functor $\wedge : \Set_* \times \Set_* \rightarrow \Set_*$ as follows. Given $f : X \rightarrow X'$ and $g : Y \rightarrow Y'$, let $f \wedge g : X \wedge Y \rightarrow X' \wedge Y'$ be the unique morphism such that the following diagram commutes.
\[ \xymatrix{
X \vee Y \ar[r] \ar[d] \ar[ddr] & 1 \ar[d] \ar@{=}@/^/[ddr] \\
X \times Y \ar[r] \ar[ddr]_{f \times g} & X \wedge Y \ar[ddr] \\
& X' \vee Y' \ar[r] \ar[d] & 1 \ar[d]\\\
& X' \times Y' \ar[r] & X' \wedge Y'
} \]
\end{df}

\begin{df}[Reduction]
Let $B$ be a small set.
Let $\Phi_B : \Set / B \rightarrow \Set / B$ be a functor. The \emph{reduction} of $\Phi_B$ is the functor $\Phi_B^B : \Set_B^B \rightarrow \Set_B^B$ defined as follows.

For $(X, u : B \rightarrow X, p : X \rightarrow B) \in \Set_B^B$, let $\Phi_B^B(X)$ be the set over and under $B$ obtained from $\Phi_B(X)$ by collapsing with respect to $\Phi_B(u) : \Phi_B(B) \rightarrow \Phi_B(X)$.
\end{df}

\begin{df}
Extend $\vee_B$ to a functor $\Set_B^B \times \Set_B^B \rightarrow \Set_B^B$.
\end{df}

\begin{df}
Extend $\wedge_B$ to a functor $\Set_B^B \times \Set_B^B \rightarrow \Set_B^B$.
\end{df}

\begin{df}[Faithful]
A functor $F : \mathcal{C} \rightarrow \mathcal{D}$ is \emph{faithful} iff, for any objects $A,B \in \mathcal{C}$ and morphisms $f,g : A \rightarrow B : \mathcal{C}$, if $Ff = Fg$ then $f = g$.
\end{df}

\begin{df}[Full]
A functor $F : \mathcal{C} \rightarrow \mathcal{D}$ is \emph{full} iff, for any objects $A, B \in \mathcal{C}$ and morphism $g : FA \rightarrow FB : \mathcal{D}$, there exists $f : A \rightarrow B : \mathcal{C}$ such that $Ff = g$.
\end{df}

\begin{df}[Fully Faithful]
A functor $F : \mathcal{C} \rightarrow \mathcal{D}$ is \emph{fully faithful} iff it is full and faithful.
\end{df}

\begin{df}[Full Embedding]
A functor $F : \mathcal{C} \rightarrow \mathcal{D}$ is a \emph{full embedding} iff it is fully faithful and injective on objects.
\end{df}

\section{Natural Transformations}

\begin{df}[Natural Transformation]
Let $F,G : \mathcal{C} \rightarrow \mathcal{D}$. A \emph{natural transformation} $\tau : F \Rightarrow G$ is a family of morphisms $\{ \tau_X : F X \rightarrow G X \}_{X \in \mathcal{C}}$ such that, for every morphism $f : X \rightarrow Y : \mathcal{C}$, we have $Gf \circ \tau_X = \tau_Y \circ Ff$.
\end{df}

\[ \xymatrix{
FX \ar[r]^{Ff} \ar[d]_{\tau_X} & FY \ar[d]^{\tau_Y} \\
GX \ar[r]_{Gf} & GY
} \]

\begin{df}[Natural Isomorphism]
A natural transformation $\tau : F \Rightarrow G : \mathcal{C} \rightarrow \mathcal{D}$ is a \emph{natural isomorphism}, $\tau : F \cong G$, iff for all $X \in \mathcal{C}$, $\tau_X$ is an isomorphism $F X \cong G X$.

Functors $F$ and $G$ are \emph{naturally isomorphic}, $F \cong G$, iff there exists a natural isomorphism between them.
\end{df}

\begin{df}[Inverse]
Let $\tau : F \cong G$. The \emph{inverse} natural isomorphism $\inv{\tau} : G \cong F$ is defined by $(\inv{\tau})_X = \inv{\tau_X}$.
\end{df}

\section{Bifunctors}

\begin{df}[Commutative]
A bifunctor $\Box : \mathcal{C}^2 \rightarrow \mathcal{C}$ is \emph{commutative} iff $\Box \cong \Box \circ T$, where $T : \mathcal{C}^2 \rightarrow \mathcal{C}^2$ is the swap functor.
\end{df}

\begin{prop}
$\vee : \Set_* \times \Set_* \rightarrow \Set_*$ is commutative.
\end{prop}

\begin{proof}
\pf\ Since the pushout of $f$ and $g$ is the pushout of $g$ and $f$. \qed
\end{proof}

\begin{prop}
$\wedge : \Set_* \times \Set_* \rightarrow \Set_*$ is commutative.
\end{prop}

\begin{proof}
\pf\ In the diagram defining $X \wedge Y$, construct the isomorphism between the version with $X$ and $Y$ and the version with $Y$ with $X$ for every object. \qed
\end{proof}

\begin{prop}
$\vee_B : \Set_B^B \times \Set_B^B \rightarrow \Set_B^B$ is commutative.
\end{prop}

\begin{prop}
$\wedge_B : \Set_B^B \times \Set_B^B \rightarrow \Set_B^B$ is commutative.
\end{prop}

\begin{df}[Associative]
A bifunctor $\Box$ is \emph{associative} iff $\Box \circ (\Box \times \id{}) \cong \Box \circ (\id{} \times \Box)$.
\end{df}

\begin{prop}
$\vee : \Set_* \times \Set_* \rightarrow \Set_*$ is associative.
\end{prop}

\begin{proof}
\pf\ Since $X \vee (Y \vee Z)$ and $(X \vee Y) \vee Z$ are both the pushout of the unique morphisms $1 \rightarrow X$, $1 \rightarrow Y$ and $1 \rightarrow Z$. \qed
\end{proof}

\begin{prop}
$\wedge : \Set_* \times \Set_* \rightarrow \Set_*$ is associative.
\end{prop}

\begin{proof}
\pf\ Draw isomorphisms between the diagrams for $X \wedge (Y \wedge Z)$ and $(X \wedge Y) \wedge Z$. \qed
\end{proof}

Product and coproduct are commutative and associative.

\begin{prop}
$\vee_B : \Set_B^B \times \Set_B^B \rightarrow \Set_B^B$ is associative.
\end{prop}

\begin{prop}
$\wedge_B : \Set_B^B \times \Set_B^B \rightarrow \Set_B^B$ is associative.
\end{prop}

\begin{prop}
Let $\mathcal{C}$ be a category with binary coproducts. Let $\Box : \mathcal{C} \times \mathcal{C} \rightarrow \mathcal{C}$ be a bifunctor. Then $\Box$ \emph{distributes} over $+$ iff the canonical morphism
\[ (X \Box Z) + (Y \Box Z) \rightarrow (X + Y) \Box Z \]
is an isomorphism for all $X$, $Y$, $Z$.
\end{prop}

\begin{prop}
In a category with binary products and binary coproducts, then $\times$ distributes over $+$.
\end{prop}

\begin{prop}
In $\Set / *$, we have $\times$ does not distribute over $\vee$.
\end{prop}

\begin{prop}
In $\Set / *$, we have $\wedge$ distributes over $\vee$.
\end{prop}

\begin{prop}
In $\Set/ B$, we have $\times_B$ distributes over $+_B$.
\end{prop}

\begin{prop}
In $\Set / B^B$, we have $\wedge_B$ distributes over $\vee_B$.
\end{prop}

\section{Functor Categories}

\begin{df}[Functor Category]
Given categories $\mathcal{C}$ and $\mathcal{D}$, define the \emph{functor category} $\mathcal{C}^\mathcal{D}$ to be the category with objects the functors from $\mathcal{D}$ to $\mathcal{C}$ and morphisms the natural transformations.
\end{df}

\begin{df}[Yoneda Embedding]
Let $\mathcal{C}$ be a category. The \emph{Yoneda embedding} $Y : \mathcal{C} \rightarrow \Set^{\op{\mathcal{C}}}$ is the functor that maps an object $A$ to $\mathcal{C}[-,A]$ and morphisms similarly.
\end{df}

\begin{thm}[Yoneda Lemma]
Let $\mathcal{C}$ be a category. There exists a natural isomorphism
\[ \phi_{XF} : \Set^{\op{\mathcal{C}}}[\mathcal{C}[-,X],F] \cong FX \]
that maps $\tau : \mathcal{C}[-,X] \Rightarrow F$ to $\tau_X(\id{X})$.
\end{thm}

\begin{proof}
\pf
\step{1}{$\phi$ is natural in $X$.}
\begin{proof}
	\pf
	\step{a}{\pflet{$f : X \rightarrow Y : \mathcal{C}$}}
	\step{b}{\pflet{$\tau : \mathcal{C}[-,X] \Rightarrow F$}}
	\step{c}{$Ff(\phi(\tau)) = \phi(\tau \circ \mathcal{C}[-,f])$}
	\begin{proof}
		\pf
		\begin{align*}
			\phi(\tau \circ \mathcal{C}[-,f])
			& = \tau_Y(\id{Y} \circ f) \\
			& = \tau_Y(f) \\
			& = \tau_Y(f \circ \id{X}) \\
			& = Ff(\tau_X(\id{X})) & (\text{$\tau$ natural}) \\
			& = Ff(\phi(\tau))
		\end{align*}
	\end{proof}
\end{proof}
\step{2}{$\phi$ is natural in $F$.}
\begin{proof}
	\step{a}{\pflet{$\alpha : F \Rightarrow G : \op{\mathcal{C}} \rightarrow \Set$}}
	\step{b}{\pflet{$\tau : \mathcal{C}[-,X] \Rightarrow F$}}
	\step{c}{$\alpha_X(\phi(\tau)) = \phi(\alpha \bullet \tau)$}
	\begin{proof}
		\pf\ $\phi(\alpha \bullet \tau) = \alpha_X(\tau_X(\id{X})) = \alpha_X(\phi(\tau))$
	\end{proof}
\end{proof}
\step{3}{Each $\phi_{XF}$ is injective.}
\begin{proof}
	\step{a}{\pflet{$\sigma, \tau : \mathcal{C}[-,X] \Rightarrow F$}}
	\step{b}{\assume{$\phi(\sigma) = \phi(\tau)$}}
	\step{c}{\pflet{$f : Y \rightarrow X$}}
	\step{d}{$\sigma_Y(f) = \tau_Y(f)$}
	\begin{proof}
		\pf
		\begin{align*}
			\sigma_Y(f) & = \sigma_Y(\id{X} \circ f) \\
			& = Ff(\sigma_X(\id{X})) & (\text{$\sigma$ is natural}) \\
			& = Ff(\tau_X(\id{X})) & (\text{\stepref{b}}) \\
			& = \tau_Y(\id{X} \circ f) & (\text{$\tau$ is natural}) \\
			& = \tau_Y(f)
		\end{align*}
	\end{proof}
\end{proof}
\step{4}{Each $\phi_{XF}$ is surjective.}
\begin{proof}
	\step{a}{\pflet{$X \in \mathcal{C}$ and $F : \mathcal{C} \rightarrow \mathcal{D}$}}
	\step{b}{\pflet{$a \in FX$}}
	\step{c}{\pflet{$\tau : \mathcal{C}[-,X] \Rightarrow F$ be given by $\tau_Y(g) = Fg(a)$ for $g : Y \rightarrow X$}}
	\step{d}{$\tau$ is natural.}
	\begin{proof}
		\step{i}{\pflet{$h : Y \rightarrow Z : \mathcal{C}$} \prove{$Fh \circ \tau_Z = \tau_Y \circ \mathcal{C}[h, \id{X}]$}}
		\step{ii}{\pflet{$g : Z \rightarrow X$}}
		\step{iii}{$Fh(\tau_Z(g)) = \tau_Y(g \circ h)$}
		\begin{proof}
			\pf
			\begin{align*}
				\tau_Y(g \circ h) & = F(g \circ h)(a) \\
				& = Fh(Fg(a)) \\
				& = Fh(\tau_Z(g))
			\end{align*}
		\end{proof}
	\end{proof}
	\step{e}{$\phi(\tau) = a$}
	\begin{proof}
		\pf
		\begin{align*}
			\phi_X(\tau) & = \tau_X(\id{X}) \\
			& = F \id{X}(a) \\
			& = a
		\end{align*}
	\end{proof}
\end{proof}
\qed
\end{proof}

\begin{cor}
The Yoneda embedding is fully faithful.
\end{cor}

\begin{cor}
Given objects $A$ and $B$ in $\mathcal{C}$, we have $A \cong B$ if and only if $\mathcal{C}[-,A] \cong \mathcal{C}[-,B]$.
\end{cor}

\chapter{The Real Numbers}

% TODO Define the real numbers

\begin{thm}
The following hold in the real numbers:
\begin{enumerate}
\item $x + (y + z) = (x + y) + z$
\item $x(yz) = (xy)z$
\item $x + y = y + x$
\item $xy = yx$
\item $x + 0 = x$
\item $x1 = x$
\item $x + (-x) = 0$
\item If $x \neq 0$ then $x \cdot (1/x) = 1$
\item $x(y+z) = xy+xz$
\item If $x > y$ then $x + z > y + z$.
\item If $x > y$ and $z > 0$ then $xz > yz$.
\item $\mathbb{R}$ has the least upper bound property.
\item If $x < y$ then there exists $z$ such that $x < z < y$.
\end{enumerate}
\end{thm}

%TODO Prove these

\begin{df}[Subtraction]
We write $x - y$ for $x + (-y)$.
\end{df}

\begin{df}
Given real numbers $x$ and $y$ with $y \neq 0$, we write $x/y$ for $x \inv{y}$.
\end{df}
\begin{thm}
\label{thm:y_equals_zero}
For any real numbers $x$ and $y$, if $x + y = x$ then $y = 0$.
\end{thm}

\begin{proof}
\pf
\step{1}{\pflet{$x,y \in \mathbb{R}$}}
\step{2}{\assume{$x + y = x$}}
\step{3}{$y = 0$}
\begin{proof}
	\pf
	\begin{align*}
		y & = y + 0 & (\text{Definition of zero}) \\
		& = y + (x + (-x)) & (\text{Definition of } -x) \\
		& = (y + x) + (-x) & (\text{Associativity of Addition}) \\
		& = (x + y) + (-x) & (\text{Commutativity of Addition}) \\
		& = x + (-x) & (\text{\stepref{2}}) \\
		& = 0 & (\text{Definition of } -x)
	\end{align*}
\end{proof}
\qed
\end{proof}

\begin{thm}
\label{thm:multiply_by_zero}
\[ \forall x \in \mathbb{R}. 0x = 0 \]
\end{thm}

\begin{proof}
\pf
\step{1}{\pflet{$x \in \mathbb{R}$}}
\step{2}{$xx + 0x = xx$}
\begin{proof}
	\pf
	\begin{align*}
		xx + 0x & = (x + 0) x & (\text{Distributive Law}) \\
		& = xx & (\text{Definition of } 0)
	\end{align*}
\end{proof}
\step{3}{$0x = 0$}
\begin{proof}
	\pf\ Theorem \ref{thm:y_equals_zero}, \stepref{2}.
\end{proof}
\qed
\end{proof}

\begin{thm}
\[ -0 = 0 \]
\end{thm}

\begin{proof}
\pf\ Since $0 + 0 = 0$. \qed
\end{proof}

\begin{thm}
\label{thm:minus_minus}
\[ \forall x \in \mathbb{R}. -(-x) = x \]
\end{thm}

\begin{proof}
\pf\ Since $-x + x = 0$. \qed
\end{proof}

\begin{thm}
\label{thm:multiply_by_negative}
\[ \forall x,y \in \mathbb{R}. x(-y) = -(xy) \]
\end{thm}

\begin{proof}
\pf
\begin{align*}
x(-y) + xy & = x((-y)+y) & (\text{Distributive Law}) \\
& = x0 & (\text{Definition of } -y) \\
& = 0 & (\text{Theorem \ref{thm:multiply_by_zero}}) \enspace \qed
\end{align*}
\end{proof}

\begin{thm}
\label{thm:multiply_by_minus_one}
\[ \forall x \in \mathbb{R}. (-1)x = -x \]
\end{thm}

\begin{proof}
\pf
\begin{align*}
	(-1)x & = -(1\cdot x) & (\text{Theorem \ref{thm:multiply_by_negative}}) \\
	& = -x & (\text{Definition of } 1) \enspace \qed
\end{align*}
\end{proof}

\subsection{Subtraction}

\begin{thm}
\[ \forall x,y,z \in \mathbb{R}. x(y-z) = xy-xz \]
\end{thm}

\begin{proof}
\pf
\begin{align*}
x (y-z) & = x(y + (-z)) & (\text{Definition of subtraction}) \\
& = xy + x(-z) & (\text{Distributive Law}) \\
& = xy + (-(xz)) & (\text{Theorem \ref{thm:multiply_by_negative}}) \\
& = xy - xz & (\text{Definition of subtraction}) \enspace \qed
\end{align*}
\end{proof}

\begin{thm}
\label{thm:negate_sum}
\[ \forall x,y \in \mathbb{R}. -(x+y) = -x-y \]
\end{thm}

\begin{proof}
\pf
\begin{align*}
-(x+y) & = (-1)(x+y) & (\text{Theorem \ref{thm:multiply_by_minus_one}}) \\
& = (-1)x + (-1)y & (\text{Distributive Law}) \\
& = -x + (-y) & (\text{Theorem \ref{thm:multiply_by_minus_one}}) \\
& = -x-y & (\text{Definition of subtraction}) \enspace \qed
\end{align*}
\end{proof}

\begin{thm}
\[ \forall x,y \in \mathbb{R}. -(x-y) = -x+y \]
\end{thm}

\begin{proof}
\pf
\begin{align*}
-(x-y) & = -(x + (-y)) & (\text{Definition of subtraction}) \\
& = -x-(-y) & (\text{Theorem \ref{thm:negate_sum}}) \\
& = -x + (-(-y)) & (\text{Definition of subtraction}) \\
& = -x + y & (\text{Theorem \ref{thm:minus_minus}}) \enspace \qed
\end{align*}
\end{proof}

\begin{df}[Reciprocal]
Given $x \in \mathbb{R}$ with $x \neq 0$, the \emph{reciprocal} of $x$, $1/x$, is the unique real number such that $x \cdot 1/x = 1$.
\end{df}

\begin{thm}
For any real numbers $x$ and $y$, if $x \neq 0$ and $xy = x$ then $y = 1$.
\end{thm}

\begin{proof}
\pf
\step{1}{\pflet{$x,y \in \mathbb{R}$}}
\step{2}{\assume{$x \neq 0$}}
\step{3}{\assume{$xy = x$}}
\step{4}{$y = 1$}
\begin{proof}
	\pf
	\begin{align*}
		y & = y1 & (\text{Definition of } 1) \\
		& = y (x \cdot 1/x) & (\text{Definition of } 1/x, \text{\stepref{2}}) \\
		& = (yx) 1/x & (\text{Associativity of Multiplication}) \\
		& = (xy) 1/x & (\text{Commutativity of Multiplication}) \\
		& = x \cdot 1/x & (\text{\stepref{3}}) \\
		& = 1 & (\text{Definition of } 1/x, \text{\stepref{2}})
	\end{align*}
\end{proof}
\qed
\end{proof}

\begin{df}[Quotient]
Given real numbers $x$ and $y$ with $y \neq 0$, the \emph{quotient} $x/y$ is defined by
\[ x/y = x \cdot 1/y \enspace . \]
\end{df}

\begin{thm}
For any real number $x$, if $x \neq 0$ then $x/x = 1$.
\end{thm}

\begin{proof}
\pf\ Immediate from definitions. \qed
\end{proof}

\begin{thm}
\[ \forall x \in \mathbb{R}. x/1 = x \]
\end{thm}

\begin{proof}
\pf
\step{1}{\pflet{$x \in \mathbb{R}$}}
\step{2}{$1/1 = 1$}
\begin{proof}
	\pf\ Since $1 \cdot 1 = 1$.
\end{proof}
\step{3}{$x/1 = x$}
\begin{proof}
	\pf\ Since $x/1 = x \cdot 1/1 = x \cdot 1 = x$.
\end{proof}
\qed
\end{proof}

\begin{thm}
For any real numbers $x$ and $y$, if $x \neq 0$ and $y \neq 0$ then $xy \neq 0$.
\end{thm}

\begin{proof}
\pf
\step{1}{\pflet{$x,y \in \mathbb{R}$}}
\step{2}{\assume{$xy = 0$ and $x \neq 0$} \prove{$y = 0$}}
\step{3}{$y = 0$}
\begin{proof}
	\pf
	\begin{align*}
		y & = 1y & (\text{Definition of } 1) \\
		& = (1/x)xy & (\text{Definition of } 1/x, \text{\stepref{2}}) \\
		& = (1/x)0 & (\text{\stepref{2}}) \\
		& = 0 & (\text{Theorem \ref{thm:multiply_by_zero}})
	\end{align*}
\end{proof}
\qed
\end{proof}

\begin{thm}
For any real numbers $y$ and $z$, if $y \neq 0$ and $z \neq 0$ then $(1/y)(1/z) = 1/(yz)$.
\end{thm}

\begin{proof}
\pf\ Since $yz(1/y)(1/z) = 1 \cdot 1 = 1$. \qed
\end{proof}

\begin{cor}
For any real numbers $x$, $y$, $z$, $w$ with $y \neq 0 \neq w$, we have $(x/y)(z/w) = (xz)/(yw)$.
\end{cor}

\begin{thm}
For any real numbers $x$, $y$, $z$, $w$ with $y \neq 0 \neq w$, we have
\[ \frac{x}{y} + \frac{z}{w} = \frac{xw + yz}{yw} \]
\end{thm}

\begin{proof}
\pf
\begin{align*}
yw \left( \frac{x}{y} + \frac{z}{w} \right)
& = y w \frac{x}{y} + yw \frac{z}{w} \\
& = wx + yz & \qed
\end{align*}
\end{proof}

\begin{thm}
For any real number $x$, if $x \neq 0$ then $1/x \neq 0$.
\end{thm}

\begin{proof}
\pf\ Since $x \cdot 1/x = 1 \neq 0$. \qed
\end{proof}

\begin{thm}
For any real numbers $w$, $z$, if $w \neq 0 \neq z$ then $1 / (w / z) = z/w$.
\end{thm}

\begin{proof}
\pf\ Since $(z/w)(w/z) = (wz)/(wz) = 1$. \qed
\end{proof}

\begin{thm}
\label{thm:multiply_quotient}
For any real numbers $a$, $x$ and $y$, if $y \neq 0$ then $(ax)/y = a(x/y)$
\end{thm}

\begin{proof}
\pf\ Since $ya(x/y) = ax$. \qed
\end{proof}

\begin{thm}
For any real numbers $x$ and $y$, if $y \neq 0$ then $(-x)/y = x/(-y) = -(x/y)$.
\end{thm}

\begin{proof}
\pf
\step{1}{$(-x)/y = -(x/y)$}
\begin{proof}
	\pf\ Take $a = -1$ in Theorem \ref{thm:multiply_quotient}.
\end{proof}
\step{2}{$x/(-y) = -(x/y)$}
\begin{proof}
	\pf\ Since $(-y)(-(x/y)) = y(x/y) = x$.
\end{proof}
\qed
\end{proof}

\begin{thm}
For any real numbers $x$, $y$, $z$ and $w$, if $x > y$ and $w > z$ then $x + w > y + z$.
\end{thm}

\begin{proof}
\pf\ We have $y + z < x + z < x + w$ by Monotonicity of Addition twice. \qed
\end{proof}

\begin{cor}
For any real numbers $x$ and $y$, if $x > 0$ and $y > 0$ then $x + y > 0$.
\end{cor}

\begin{thm}
For any real numbers $x$ and $y$, if $x > 0$ and $y > 0$ then $xy > 0$.
\end{thm}

\begin{proof}
\pf
\begin{align*}
	xy & > 0y & (\text{Monotonicity of Multiplication}) \\
	& = 0 & (\text{Theorem \ref{thm:multiply_by_zero}}) \enspace \qed
\end{align*}
\end{proof}

\begin{thm}
\label{thm:negation_positive}
For any real number $x$, we have $x > 0$ iff $-x < 0$.
\end{thm}

\begin{proof}
\pf
\step{1}{If $0 < x$ then $-x < 0$}
\begin{proof}
	\pf\ By Monotonicity of Addition adding $-x$ to both sides.
\end{proof}
\step{2}{If $-x < 0$ then $0 < x$}
\begin{proof}
	\pf\ By Monotonicity of Addition adding $x$ to both sides.
\end{proof}
\qed
\end{proof}

\begin{thm}
For any real numbers $x$ and $y$, we have $x > y$ iff $-x < -y$.
\end{thm}

\begin{proof}
\pf
\step{1}{If $y < x$ then $-x < -y$.}
\begin{proof}
	\pf\ By Monotonicity of Addition adding $-x-y$ to both sides.
\end{proof}
\step{2}{If $-x < -y$ then $y < x$.}
\begin{proof}
	\pf\ By Monotonicity of Addition adding $x + y$ to both sides.
\end{proof}
\qed
\end{proof}

\begin{thm}
\label{thm:multiplication_antitone}
For any real numbers $x$, $y$ and $z$, if $x > y$ and $z < 0$ then $xz < yz$.
\end{thm}

\begin{proof}
\pf
\step{1}{\pflet{$x$, $y$ and $z$ be real numbers.}}
\step{2}{\assume{$x > y$}}
\step{3}{\assume{$z < 0$}}
\step{4}{$-z > 0$}
\begin{proof}
	\pf\ Theorem \ref{thm:negation_positive}, \stepref{3}.
\end{proof}
\step{5}{$x(-z) > y(-z)$}
\begin{proof}
	\pf\ \stepref{2}, \stepref{4}, Monotonicity of Multiplication.
\end{proof}
\step{6}{$-(xz) > -(yz)$}
\begin{proof}
	\pf\ Theorem \ref{thm:multiply_by_negative}, \stepref{5}.
\end{proof}
\step{7}{$xz < yz$}
\begin{proof}
	\pf\ Theorem \ref{thm:negation_positive}, \stepref{6}.
\end{proof}
\qed
\end{proof}

\begin{thm}
\label{thm:square_positive}
For any real number $x$, if $x \neq 0$ then $xx > 0$.
\end{thm}

\begin{proof}
\pf
\step{1}{If $x > 0$ then $xx > 0$}
\begin{proof}
	\pf\ By Monotonicity of Multiplication.
\end{proof}
\step{2}{If $x < 0$ then $xx > 0$}
\begin{proof}
	\pf\ Theorem \ref{thm:multiplication_antitone}.
\end{proof}
\qed
\end{proof}

\begin{thm}
\[ 0 < 1 \]
\end{thm}

\begin{proof}
\pf\ By Theorem \ref{thm:square_positive} since $1 = 1 \cdot 1$. \qed
\end{proof}

\begin{df}[Positive]
A real number $x$ is \emph{positive} iff $x > 0$.

We write $\mathbb{R}_+$ for the set of positive reals.
\end{df}

\begin{thm}
For any real numbers $x$ and $y$, we have $xy$ is positive if and only if $x$ and $y$ are both positive or both negative.
\end{thm}

\begin{proof}
\pf\ By the Monotonicity of Multiplication and Theorem \ref{thm:multiplication_antitone}. \qed
\end{proof}

\begin{cor}
For any real number $x$, if $x > 0$ then $1/x > 0$.
\end{cor}

\begin{proof}
\pf\ Since $x \cdot 1/x = 1$ is positive. \qed
\end{proof}

\begin{thm}
For any real numbers $x$ and $y$, if $x > y > 0$ then $1/x < 1/y$.
\end{thm}

\begin{proof}
\pf\ If $1/y \leq 1/x$ then $1 < 1$ by Monotonicity of Multiplication. \qed
\end{proof}

\begin{thm}
For any real numbers $x$ and $y$, if $x < y$ then $x < (x+y)/2 < y$.
\end{thm}

\begin{proof}
\pf\ We have $2x < x+y$ and $x + y < 2y$ by Monotonicity of Addition, hence $x < (x+y)/2 < y$ by Monotonicity of Multiplication since $1/2 > 0$. \qed
\end{proof}

\begin{cor}
$\mathbb{R}$ is a linear continuum.
\end{cor}

\begin{df}[Negative]
A real number $x$ is \emph{negative} iff $x < 0$.

We write $\overline{\mathbb{R}_+}$ for the set of nonnegative reals.
\end{df}

\begin{thm}
For every positive real number $a$, there exists a unique positive real $\sqrt{a}$ such that $\sqrt{a}^2 = a$.
\end{thm}

\begin{proof}
\pf
\step{0}{\pflet{$a$ be a positive real.}}
\step{1}{For any real numbers $x$ and $h$, if $0 \leq h < 1$, then
\[ (x + h)^2 < x^2 + h(2x+1) \enspace . \]}
\begin{proof}
	\step{a}{\pflet{$x$ and $h$ be real numbers.}}
	\step{c}{\assume{$0 \leq h < 1$}}
	\step{d}{$(x+h)^2 < x^2 + h(2x+1)$}
	\begin{proof}
		\pf
		\begin{align*}
			(x+h)^2 & = x^2 + 2hx + h^2 \\
			& < x^2 + 2hx + h & (\text{\stepref{c}}) \\
			& = x^2 + h(2x+1) 
		\end{align*}
	\end{proof}
\end{proof}
\step{2}{For any real numbers $x$ and $h$, if $h > 0$ then
\[ (x-h)^2 > x^2 - 2hx \enspace . \]}
\begin{proof}
	\step{a}{\pflet{$x$ and $h$ be real numbers.}}
	\step{c}{\assume{$h > 0$}}
	\step{d}{$(x-h)^2 > x^2 - 2hx$}
	\begin{proof}
		\pf
		\begin{align*}
			(x-h)^2 & = x^2 - 2hx + h^2 \\
			& > x^2 - 2hx & (\text{\stepref{c}})
		\end{align*}
	\end{proof}
\end{proof}
\step{3}{For any positive real $x$, if $x^2 < a$ then there exists $h > 0$ such that $(x+h)^2 < a$.}
\begin{proof}
	\step{a}{\pflet{$x$ be a positive real.}}
	\step{b}{\assume{$x^2 < a$}}
	\step{c}{\pflet{$h = \min((a-x^2)/(2x+1),1/2)$}}
	\step{d}{$0 < h < 1$}
	\step{e}{$(x+h)^2 < a$}
	\begin{proof}
		\pf
		\begin{align*}
			(x+h)^2 & < x^2 + h(2x+1) & (\text{\stepref{1}}) \\
			& \leq a
		\end{align*}
	\end{proof}
\end{proof}
\step{4}{For any positive real $x$, if $x^2 > a$ then there exists $h > 0$ such that $(x-h)^2 > a$.}
\begin{proof}
	\step{a}{\pflet{$x$ be a positive real.}}
	\step{b}{\assume{$x^2 > a$}}
	\step{c}{\pflet{$h = (x^2 - a)/2x$}}
	\step{d}{$h > 0$}
	\step{e}{$(x-h)^2 > a$}
	\begin{proof}
		\pf
		\begin{align*}
			(x-h)^2 & > x^2 - 2hx \\
			& = a & (\text{\stepref{c}})
		\end{align*}
	\end{proof}
\end{proof}
\step{5}{\pflet{$B = \{ x \in \mathbb{R} : x^2 < a \}$}}
\step{6}{$B$ is bounded above.}
\begin{proof}
	\pf\ If $a \geq 1$ then $a$ is an upper bound. If $a < 1$ then 1 is an upper bound.
\end{proof}
\step{7}{$B$ contains at least one positive real.}
\begin{proof}
	\pf\ If $a \geq 1$ then $1 \in B$. If $a < 1$ then $a \in B$.
\end{proof}
\step{8}{\pflet{$b = \sup B$}}
\step{9}{$b^2 = a$}
\begin{proof}
	\step{a}{$b^2 \geq a$}
	\begin{proof}
		\step{i}{\assume{for a contradiction $b^2 < a$}}
		\step{ii}{\pick\ $h > 0$ such that $(b+h)^2 < a$}
		\begin{proof}
			\pf\ \stepref{3}
		\end{proof}
		\step{iii}{$b + h \in B$}
		\qedstep
		\begin{proof}
			\pf\ This contradicts \stepref{8}.
		\end{proof}
	\end{proof}
	\step{b}{$b^2 \leq a$}
	\begin{proof}
		\step{i}{\assume{for a contradiction $b^2 > a$}}
		\step{ii}{\pick\ $h > 0$ such that $(b-h)^2 > a$}
		\begin{proof}
			\pf\ \stepref{4}
		\end{proof}
		\step{iii}{\pick\ $x \in B$ such that $b - h < x$}
		\begin{proof}
			\pf\ \stepref{8}
		\end{proof}
		\step{iv}{$(b-h)^2 < x^2 < a$}
		\qedstep
		\begin{proof}
			\pf\ This contradicts \stepref{ii}
		\end{proof}
	\end{proof}
\end{proof}
\step{10}{For any positive reals $b$ and $c$, if $b^2 = c^2$ then $b = c$.}
\begin{proof}
	\step{a}{\pflet{$b$ and $c$ be positive reals.}}
	\step{b}{\assume{$b^2 = c^2$}}
	\step{c}{$b^2 - c^2 = 0$}
	\step{d}{$(b-c)(b+c) = 0$}
	\step{e}{$b - c = 0$ or $b+c = 0$}
	\step{f}{$b+c \neq 0$}
	\begin{proof}
		\pf\ Since $b + c > 0$
	\end{proof}
	\step{g}{$b - c = 0$}
	\step{h}{$b = c$}
\end{proof}
\qed
\end{proof}

%TODO
\begin{thm}
The set of real numbers is uncountable.
\end{thm}

\begin{df}
We write $\mathbb{R}^\infty$ for the set of sequences in $\mathbb{R}^\omega$ that are eventually zero.
\end{df}

\begin{df}[Hilbert Cube]
The \emph{Hilbert cube} is $\prod_{n=0}^\infty [0,1/(n+1)]$.
\end{df}

\begin{df}[Unit Ball]
The \emph{unit ball} is $B^2 := \{ (x,y) \in \mathbb{R}^2 : x^2 + y^2 \leq 1 \}$.
\end{df}

\chapter{Integers and Rationals}

\section{Positive Integers}

\begin{df}[Inductive]
A set of real numbers $A$ is \emph{inductive} iff $1 \in A$ and $\forall x \in A. x + 1 \in A$.
\end{df}

\begin{df}[Positive Integer]
The set $\mathbb{Z}_+$ of \emph{positive integers} is the intersection of the set of inductive sets.
\end{df}

\begin{prop}
Every positive integer is positive.
\end{prop}

\begin{proof}
\pf\ The set of positive reals is inductive. \qed
\end{proof}

\begin{prop}
1 is the least element of $\mathbb{Z}_+$.
\end{prop}

\begin{proof}
\pf\ Since $\{ x \in \mathbb{R} : x \geq 1 \}$ is inductive. \qed
\end{proof}

\begin{prop}
$\mathbb{Z}_+$ is inductive.
\end{prop}

\begin{proof}
\pf\ 1 is an element of every inductive set, and for all $x \in \mathbb{R}$, if $x$ is an element of every inductive set then so is $x + 1$. \qed
\end{proof}

\begin{thm}[Principle of Induction]
If $A$ is an inductive set of positive integers then $A = \mathbb{Z}_+$.
\end{thm}

\begin{proof}
\pf\ Immediate from definitions. \qed
\end{proof}

\begin{thm}[Well-Ordering Property]
$\mathbb{Z}_+$ is well ordered.
\end{thm}

\begin{proof}
\pf\ Construct the obvious order isomorphism $\omega \cong \mathbb{Z}_+$. \qed
\end{proof}

\begin{thm}[Archimedean Ordering Property]
The set $\mathbb{Z}_+$ is unbounded above.
\end{thm}

\begin{proof}
\pf
\step{1}{\assume{for a contradiction $\mathbb{Z}_+$ is bounded above.}}
\step{2}{\pflet\ $s = \sup \mathbb{Z}_+$}
\step{3}{\pick\ $n \in \mathbb{Z}_+$ such that $s-1 < n$}
\step{4}{$s < n+1$}
\qedstep
\begin{proof}
	\pf\ \stepref{2} and \stepref{4} form a contradiction.
\end{proof}
\qed
\end{proof}

\subsection{Exponentiation}

\begin{df}
For $a$ a real number and $n$ a positive integer, define the real number $a^n$ recursively as follows:
\begin{align*}
a^1 & = a \\
a^{n+1} & = a^n a
\end{align*}
\end{df}

\begin{thm}
\label{thm:law_of_exponents_1}
For all $a \in \mathbb{R}$ and $m,n \in mathbb{Z}_+$, we have
\[ a^n a^m = a^{n+m} \]
\end{thm}

\begin{proof}
\pf
\step{1}{\pflet{$P(m)$ be the property $\forall a \in \mathbb{R}. \forall n \in \mathbb{Z}_+. a^n a^m = a^{n+m}$}}
\step{2}{$P(1)$}
\begin{proof}
	\pf\ $a^n a^1 = a^n a = a^{n+1}$.
\end{proof}
\step{3}{$\forall m \in \mathbb{Z}_+. P(m) \Rightarrow P(m+1)$}
\begin{proof}
	\step{a}{\pflet{$m$ be a positive integer.}}
	\step{b}{\assume{$P(m)$}}
	\step{c}{\pflet{$a \in \mathbb{R}$}}
	\step{d}{\pflet{$n \in \mathbb{Z}_+$}}
	\step{e}{$a^n a^{m+1} = a^{n+m+1}$}
	\begin{proof}
		\pf
		\begin{align*}
			a^n a^{m+1} & = a^n a^m a \\
			& = a^{n+m} a & (\text{\stepref{b}}) \\
			& = a^{n+m+1}
		\end{align*}
	\end{proof}
\end{proof}
\qedstep
\begin{proof}
	\pf\ By induction.
\end{proof}
\qed
\end{proof}

\begin{thm}
For all $a \in \mathbb{R}$ and $m,n \in \mathbb{Z}_+$,
\[ (a^n)^m = a^{nm} \enspace . \]
\end{thm}

\begin{proof}
\pf
\step{1}{\pflet{$P(m)$ be the property $\forall a \in \mathbb{R}. \forall n \in \mathbb{Z}_+. (a^n)^m = a^{nm}$.}}
\step{2}{$P(1)$}
\begin{proof}
	\pf\ $(a^n)^1 = a^n = a^{n \cdot 1}$
\end{proof}
\step{3}{$\forall m \in \mathbb{Z}_+. P(m) \Rightarrow P(m+1)$}
\begin{proof}
	\pf
	\begin{align*}
		(a^n)^{m+1} & = (a^n)^m a^n \\
		& = a^{nm} a^n \\
		& = a^{nm+n} & (\text{Theorem \ref{thm:law_of_exponents_1}}) \\
		& = a^{n(m+1)}
	\end{align*}
\end{proof}
\qed
\end{proof}

\begin{thm}
For any real numbers $a$ and $b$ and positive integer $m$,
\[ a^m b^m = (ab)^m \enspace . \]
\end{thm}

\begin{proof}
\pf\ Induction on $m$. \qed
\end{proof}

\section{Integers}

\begin{df}[Integer]
The set $\mathbb{Z}$ of \emph{integers} is
\[ \mathbb{Z} = \mathbb{Z}_+ \cup \{ 0 \} \cup \{ -x : x \in \mathbb{Z}_+\} \enspace . \]
\end{df}

\begin{prop}
The sum, difference and product of two integers is an integer.
\end{prop}

\begin{proof}
\pf\ Easy. \qed
\end{proof}

\begin{ex}
$1/2$ is not an integer.
\end{ex}

\begin{prop}
For any integer $n$, there is no integer $a$ such that $n < a < n+1$.
\end{prop}

\begin{proof}
\pf
\step{1}{For any positive integer $n$, there is no integer $a$ such that $n < a < n+1$.}
\begin{proof}
	\step{a}{There is no integer $a$ such that $1 < a < 2$.}
	\begin{proof}
		\step{i}{There is no positive integer $a$ such that $1 < a < 2$.}
		\begin{proof}
			\step{one}{We do not have $1 < 1 < 2$.}
			\step{two}{For any positive integer $n$, we do not have $1 < n +1 < 2$.}
			\begin{proof}
				\pf\ Since $n \geq 1$ so $n+1 \geq 2$.
			\end{proof}
		\end{proof}
		\step{ii}{We do not have $1 < 0 < 2$.}
		\step{iii}{For any positive integer $a$, we do not have $1 < -a < 2$.}
		\begin{proof}
			\pf\ Since $-a < 0 < 1$.
		\end{proof}
	\end{proof}
	\step{b}{For any positive integer $n$, if there is no integer $a$ such that $n < a < n+1$, then there is no integer $a$ such that $n+1 < a < n+2$.}
	\begin{proof}
		\pf\ If $n+1 < a < n+2$ then $n < a-1 < n+1$.
	\end{proof}
\end{proof}
\step{2}{There is no integer $a$ such that $0 < a < 1$.}
\begin{proof}
	\pf\ If $0 < a < 1$ then $1 < a+1 < 2$.
\end{proof}
\step{3}{For any positive integer $n$, there is no integer $a$ such that $-n < a < -n+1$.}
\begin{proof}
	\pf\ If $-n < a < -n+1$ then $n-1 < -a < n$.
\end{proof}
\qed
\end{proof}

\begin{thm}
Every nonempty subset of $\mathbb{Z}$ bounded above has a largest element.
\end{thm}

\begin{proof}
\pf
\step{1}{\pflet{$S$ be a nonempty subset of $\mathbb{Z}$ bounded above.}}
\step{2}{\pflet{$u$ be an upper bound for $S$.}}
\step{3}{\pick\ an integer $n > u$}
\begin{proof}
	\pf\ Archimedean property.
\end{proof}
\step{4}{\pflet{$k$ be the least positive integer such that $n - k \in S$.}}
\begin{proof}
	\step{a}{\pick\ $m \in S$}
	\step{b}{$n - m$ is a positive integer.}
	\step{c}{There exists a positive integer $k$ such that $n - k \in S$.}
\end{proof}
\step{5}{$n - k$ is the greatest element in $S$.}
\begin{proof}
	\step{a}{\pflet{$m \in S$}}
	\step{b}{$n - m \geq k$}
	\step{c}{$m \leq n - k$}
\end{proof}
\qed
\end{proof}

\begin{thm}
\label{thm:sandwich}
For any real number $x$, if $x$ is not an integer then there exists a unique integer $n$ such that $n < x < n + 1$.
\end{thm}

\begin{proof}
\pf
\step{1}{$\{ n \in \mathbb{Z} : n < x \}$ is a nonempty set of integers bounded above.}
\begin{proof}
	\step{a}{\pick\ $m > -x$}
	\begin{proof}
		\pf\ Archimedean property.
	\end{proof}
	\step{b}{$-m < x$}
	\step{c}{$\{ n \in \mathbb{Z} : n < x \}$ is nonempty.}
\end{proof}
\step{2}{\pflet{$n$ be the greatest integer such that $n < x$}}
\step{3}{$x < n + 1$}
\step{4}{If $n'$ is an integer with $n' < x < n' + 1$ then $n' = n$.}
\begin{proof}
	\pf\ We have $n' < n+1$ so $n' \leq n$, and $n < n' + 1$ so $n \leq n'$.
\end{proof}
\qed
\end{proof}

\begin{df}[Even]
An integer $n$ is \emph{even} iff $n/2$ is an integer; otherwise, $n$ is \emph{odd}.
\end{df}

\begin{thm}
If the integer $m$ is odd then there exists an integer $n$ such that $m = 2n+1$.
\end{thm}

\begin{proof}
\pf
\step{1}{\pflet{$n$ be the integer such that $n < m/2 < n+1$}}
\begin{proof}
	\pf\ Theorem \ref{thm:sandwich}.
\end{proof}
\step{2}{$2n < m < 2n+2$}
\step{3}{$m = 2n+1$}
\qed
\end{proof}

\begin{thm}
\label{thm:product_odd}
The product of two odd integers is odd.
\end{thm}

\begin{proof}
\pf\ $(2m+1)(2n+1) = 2(2mn + m + n) + 1$. \qed
\end{proof}

\begin{cor}
If $p$ is an odd integer and $n$ is a positive integer then $p^n$ is an odd integer.
\end{cor}

\begin{df}[Exponentiation]
Extend the definition of exponentiation so $a^n$ is defined for:
\begin{itemize}
\item all real numbers $a$ and non-negative integers $n$
\item all non-zero real numbers $a$ and integers $n$
\end{itemize}
as follows:
\begin{align*}
a^0 & = 1 \\
a^{-n} & = 1/a^n & (n \text{ a positive integer})
\end{align*}
\end{df}

\begin{thm}[Laws of Exponents]
For all non-zero reals $a$ and $b$ and integers $m$ and $n$,
\begin{align*}
a^n a^m & = a^{n+m} \\
(a^n)^m & = a^{nm} \\
a^m b^m & = (ab)^m
\end{align*}
\end{thm}

\begin{proof}
\pf\ Easy. \qed
\end{proof}

\begin{thm}
$\mathbb{Z}$ is countable.
\end{thm}

\begin{proof}
\pf\ The function that maps an integer $n$ to $2n$ if $n \geq 0$ and $-1-2n$ if $n < 0$ is a bijection $\mathbb{Z} \approx \mathbb{N}$. \qed
\end{proof}

\section{Rational Numbers}

\begin{df}[Rational Number]
The set $\mathbb{Q}$ of \emph{rational numbers} is the set of all real numbers that are the quotient of two integers. A real that is not rational is \emph{irrational}.
\end{df}

\begin{thm}
$\sqrt{2}$ is irrational.
\end{thm}

\begin{proof}
\pf
\step{1}{For any positive rational $a$, there exist positive integers $m$ and $n$ not both even such that $a = m / n$.}
\begin{proof}
	\step{a}{\pflet{$a$ be a positive rational.}}
	\step{b}{\pflet{$n$ be the least positive integer such that $na$ is a positive integer.}}
	\step{c}{\pflet{$m = na$}}
	\step{d}{\assume{for a contradiction $m$ and $n$ are both even.}}
	\step{e}{$m/2 = (n/2)a$}
	\qedstep
	\begin{proof}
		\pf\ This contradicts the leastness of $n$ (\stepref{b}).
	\end{proof}
\end{proof}
\step{2}{\assume{for a contradiction $\sqrt{2}$ is rational.}}
\step{3}{\pick\ positive integers $m$ and $n$ not both even such that $\sqrt{2} = m / n$.}
\step{4}{$m^2 = 2n^2$}
\step{5}{$m^2$ is even.}
\step{6}{$m$ is even.}
\begin{proof}
	\pf\ Theorem \ref{thm:product_odd}.
\end{proof}
\step{7}{\pflet{$k = m/2$}}
\step{8}{$4k^2 = 2n^2$}
\step{9}{$n^2 = 2k^2$}
\step{10}{$n^2$ is even.}
\step{11}{$n$ is even.}
\begin{proof}
	\pf\ Theorem \ref{thm:product_odd}.
\end{proof}
\qedstep
\begin{proof}
	\pf\ \stepref{3}, \stepref{6} and \stepref{11} form a contradiction.
\end{proof}
\qed
\end{proof}

\begin{thm}
$\mathbb{Q}$ is countably infinite.
\end{thm}

\begin{proof}
\pf\ The function $\mathbb{Z} \times \mathbb{N} \rightarrow \mathbb{Q}$ that maps $(m,n)$ to $m/(n+1)$ is a surjection. \qed
\end{proof}

\section{Algebraic Numbers}

%TODO A polynomial has only finitely many roots

\begin{df}[Algebraic Number]
A real number $r$ is \emph{algebraic} iff there exists a natural number $n$ and rational numbers $a_0$, $a_1$, \ldots, $a_{n-1}$ such that
\[ r^n + a_{n-1} r^{n-1} + \cdots + a_1 r + a_0 = 0 \]
Otherwise, $r$ is \emph{transcendental}.
\end{df}

\begin{prop}
The set of algebraic numbers is countably infinite.
\end{prop}

\begin{proof}
\pf\ There are countably many finite sequences of rational numbers, and each corresponding polynomial has only finitely many roots. \qed
\end{proof}

\begin{cor}
The set of transcendental numbers is uncountable.
\end{cor}

\chapter{Monoid Theory}

\begin{df}[Monoid]
A \emph{monoid} is a category with one object.
\end{df}

\begin{df}
Let $\mathcal{C}$ be a category and $X \in \mathcal{C}$. The monoid $\mathrm{End}_\mathcal{C}(X)$ is the set of all morphisms $X \rightarrow X$ under composition.
\end{df}

\begin{prop}
For any functor $F : \mathcal{C} \rightarrow \mathcal{D}$ and $X \in \mathcal{C}$, we have that $F : \mathrm{End}_\mathcal{C}(X) \rightarrow \mathrm{End}_\mathcal{D}(FX)$ is a monoid homomorphism.
\end{prop}

\begin{proof}
\pf\ Since $F \id{X} = \id{FX}$ and $F(g \circ f) = Fg \circ Ff$. \qed
\end{proof}

\chapter{Group Theory}

\begin{df}
Let $\mathbf{Grp}$ be the category of small groups and group homomorphisms.
\end{df}

\begin{df}
We identify any group $G$ with the category with one object whose morphisms are the elements of $G$ with composition given by the multiplication in $G$.
\end{df}

\begin{prop}
The trivial group is a zero object in $\mathbf{Grp}$.
\end{prop}

\begin{proof}
\pf\ Easy. \qed
\end{proof}

The zero morphism $G \rightarrow H$ maps every element in $G$ to $e$.

\begin{df}
Let $\mathcal{C}$ be a category and $X \in \mathcal{C}$. We write $\mathrm{Aut}_\mathcal{C}(X)$ for the set of all isomorphisms $X \cong X$ under composition.
\end{df}

\begin{prop}
Let $F : \mathcal{C} \rightarrow \mathcal{D}$ be a functor and $X \in \mathcal{C}$. Then $F : \mathrm{Aut}_\mathcal{C}(X) \rightarrow \mathrm{Aut}_\mathcal{D}(FX)$ is a group homomorphism.
\end{prop}

\begin{proof}
\pf\ Since $F \id{X} = \id{FX}$, $F(g \circ f) = Fg \circ Ff$, and $F \inv{f} = \inv{(Ff)}$. \qed
\end{proof}

\begin{prop}
$\mathbf{Grp}$ has products.
\end{prop}

\begin{df}[Free Product]
The product of a family of groups in $\mathbf{Grp}$ is called the \emph{free product}.
\end{df}

\begin{prop}
$\mathbf{Ab}$ has products given by direct sums.
\end{prop}

\chapter{Ring Theory}

\begin{df}
Let $\mathbf{Ring}$ be the concrete category of rings and ring homomorphisms.
\end{df}

\begin{df}[Spectrum]
Let $R$ be a commutative ring. The \emph{spectrum} of $R$, $\spec R$, is the set of all prime ideals of $R$.
\end{df}

\begin{df}[Zariski Topology]
Let $R$ be a commutative ring. The \emph{Zariski topology} on $\spec R$ is the topology where the closed sets are the sets of the form
\[ VE := \{ p \in \spec R : E \subseteq p \} \]
for any $E \in \mathcal{P} R$.

We prove this is a topology.
\end{df}

\begin{proof}
\pf
\step{1}{\pflet{$\mathcal{C} = \{VE : E \in \mathcal{P} R\}$}}
\step{2}{For all $\mathcal{A} \subseteq \mathcal{C}$ we have $\bigcap \mathcal{A} \in \mathcal{C}$}
\begin{proof}
	\step{a}{\pflet{$\mathcal{A} \subseteq \mathcal{C}$}}
	\step{b}{\pflet{$E = \bigcup \{ E' \in \mathcal{P} R : VE' \in \mathcal{A} \}$} \prove{$VE = \bigcap \mathcal{A}$}}
	\step{c}{For all $p \in \spec R$, if $E \subseteq p$ then $p \in \bigcap \mathcal{A}$}
	\begin{proof}
		\step{i}{\pflet{$p \in \spec R$}}
		\step{ii}{\assume{$E \subseteq p$}}
		\step{iii}{\pflet{$E' \in \mathcal{P} R$ with $VE' \in \mathcal{A}$}}
		\step{iv}{$E' \subseteq E$}
		\step{v}{$E' \subseteq p$}
		\step{vi}{$p \in VE'$}
	\end{proof}
	\step{d}{For all $p \in \spec R$, if $p \in \bigcap \mathcal{A}$ then $E \subseteq p$}
	\begin{proof}
		\step{i}{\pflet{$p \in \bigcap \mathcal{A}$}}
		\step{ii}{For all $E' \in \mathcal{P} R$ with $VE' \in \mathcal{A}$ we have $E' \subseteq p$}
		\step{iii}{$E \subseteq p$}
	\end{proof}
\end{proof}
\step{3}{For all $C,D \in \mathcal{C}$ we have $C \cup D \in \mathcal{C}$.}
\begin{proof}
	\pf\ Since $VE \cup VE' = V(E \cap E')$
\end{proof}
\step{4}{$\emptyset \in \mathcal{C}$}
\begin{proof}
	\step{a}{$VR = \emptyset$}
	\begin{proof}
		\pf\ If $p \in VR$ then $R \subseteq p$ contradicting the fact that $p$ is a prime ideal.
	\end{proof}
\end{proof}
\qed
\end{proof}

\begin{df}
For any ring $R$, let $R-\mathbf{Mod}$ be the category of small $R$-modules and $R$-module homomorphisms.
\end{df}

\begin{prop}
$R-\mathbf{Mod}$ has products and coproducts.
\end{prop}

\chapter{Field Theory}

\begin{prop}
$\mathbf{Field}$ does not have binary products.
\end{prop}

\begin{proof}
\pf\ There cannot be a field $K$ with field homomorphisms $K \rightarrow \mathbb{Z}_2$ and $K \rightarrow \mathbb{Z}_3$, because its characteristic would be both 2 and 3. \qed
\end{proof}

\chapter{Linear Algebra}

\begin{df}[Span]
Let $V$ be a vector space and $A \subseteq V$. The \emph{span} of $A$ is the set of all linear combinations of elements of $A$.
\end{df}

\begin{df}[Independent]
Let $V$ be a vector space and $A \subseteq V$. Then $A$ is \emph{linearly independent} iff, whenever
\[ \alpha_1 v_1 + \cdots + \alpha_n v_n = 0 \]
where $v_1, \ldots, v_n \in A$, then
\[ \alpha_1 = \cdots = \alpha_n = 0 \enspace . \]
\end{df}

\begin{prop}
\label{prop:extend_linearly_independent}
Let $V$ be a vector space, $A \subseteq V$ and $v \in V$. If $A$ is linearly independent and $v \notin \spn A$, then $A \cup \{ v \}$ is independent.
\end{prop}

\begin{proof}
\pf
\step{1}{\pflet{$\alpha_1 v_1 + \cdots + \alpha_n v_n + \beta v = 0$ where $v_1, \ldots, v_n \in A$}}
\step{2}{$\beta = 0$}
\begin{proof}
	\pf\ Otherwise $v = (\alpha_1 / \beta) v_1 + \cdots + (\alpha_n / \beta) v_n \in \spn A$.
\end{proof}
\step{3}{$\alpha_1 = \cdots = \alpha_n = 0$}
\begin{proof}
	\pf\ Since $A$ is linearly independent.
\end{proof}
\qed
\end{proof}

\begin{thm}
Every vector space has a basis.
\end{thm}

\begin{proof}
\pf
\step{1}{\pflet{$V$ be a vector space.}}
\step{2}{\pick\ a maximal linearly independent set $\mathcal{B}$.}
\begin{proof}
	\pf\ By Tukey's Lemma.
\end{proof}
\step{3}{$\spn \mathcal{B} = V$}
\begin{proof}
	\pf\ Proposition \ref{prop:extend_linearly_independent}.
\end{proof}
\qed
\end{proof}

\begin{df}
For any field $K$, we write $\mathbf{Vect}_K$ for $K-\mathbf{Mod}$.
\end{df}

Dual space functor $\mathbf{Vect}_K^{\mathrm{op}} \rightarrow \mathbf{Vect}_K$.

\chapter{Topology}

\section{Topological Spaces}

\begin{df}[Topological Space]
Let $X$ be a set and $\mathcal{O} \subseteq \mathcal{P} X$. Then we say $(X, \mathcal{O})$ is a \emph{topological space} iff:
\begin{itemize}
\item For any $\mathcal{U} \subseteq \mathcal{O}$ we have $\bigcup \mathcal{U} \in \mathcal{O}$.
\item For any $U, V \in \mathcal{O}$ we have $U \cap V \in \mathcal{O}$.
\item $X \in \mathcal{O}$
\end{itemize}
We call $\mathcal{O}$ the \emph{topology} of the toplogical space, and call its elements \emph{open} sets. We shall often write $X$ for the topological space $(X, \mathcal{O})$.
\end{df}

\begin{df}[Discrete Topology]
For any set $X$, the power set $\mathcal{P} X$ is called the \emph{discrete} topology on $X$.
\end{df}

\begin{prop}
For any set $X$, the discrete topology on $X$ is a topology on $X$.
\end{prop}

\begin{df}[Indiscrete Topology]
For any set $X$, the \emph{indiscrete} or \emph{trivial} topology on $X$ is  $\{ \emptyset, X \}$.
\end{df}

\begin{prop}
For any set $X$, the indiscrete topology on $X$ is a topology on $X$.
\end{prop}

\begin{df}[Cofinite Topology]
For any set $X$, the \emph{cofinite} topology is $\{ X - U : U \subseteq X \text{ is finite} \}$.
\end{df}

\begin{df}[Cocountable Topology]
For any set $X$, the \emph{cocountable} topology is $\{ X - U : U \subseteq X \text{ is countable} \}$.
\end{df}

\begin{df}[Sierpi\'{n}ski Two-Point Space]
The \emph{Sierpi\'{n}ski two-point space} is $\{0,1\}$ under the topology $\{ \emptyset, \{1\}, \{0,1\} \}$.
\end{df}

\begin{prop}
Let $X$ be a topological space and $U \subseteq X$. Then $U$ is open if and only if, for all $x \in U$, there exists an open set $V$ such that $x \in V \subseteq U$.
\end{prop}

\begin{prop}
The intersection of a set of topologies on a set $X$ is a topology on $X$.
\end{prop}

\begin{df}[Closed Set]
Let $X$ be a topological space and $A \subseteq X$. Then $A$ is \emph{closed} iff $X - A$ is open.
\end{df}

\begin{prop}
A set $B$ is open if and only if $X - B$ is closed.
\end{prop}

\begin{prop}
Let $X$ be a set and $\mathcal{C} \subseteq \mathcal{P} X$. Then there exists a topology $\mathcal{O}$ on $X$ such that $\mathcal{C}$ is the set of closed sets if and only if:
\begin{itemize}
\item For any $\mathcal{D} \subseteq \mathcal{C}$ we have $\bigcap \mathcal{D} \in \mathcal{C}$
\item For any $C, D \in \mathcal{C}$ we have $C \cup D \in \mathcal{C}$.
\item $\emptyset \in \mathcal{C}$
\end{itemize}
In this case, $\mathcal{O}$ is unique and is given by $\mathcal{O} = \{ X - C : C \in \mathcal{C} \}$.
\end{prop}

\begin{thm}
Let $X$ be a set. Let $\mathcal{C} \subseteq \mathcal{P} X$. Then there exists a topology on $X$ such that $\mathcal{C}$ is the set of closed sets if and only if:
\begin{enumerate}
\item $\emptyset \in \mathcal{C}$
\item $\forall \mathcal{A} \subseteq \mathcal{C}. \bigcap \mathcal{A} \in \mathcal{C}$
\item $\forall C,D \in \mathcal{C}. C \cup D \in \mathcal{C}$
\end{enumerate}
In this case, the topology is unique, and is $\{ X - C : C \in \mathcal{C} \}$.
\end{thm}

\begin{proof}
\pf\ Straightforward.
\end{proof}

\begin{thm}
There are infinitely many primes.
\end{thm}

\begin{proof}
Furstenberg's proof:

\pf
\step{1}{For $a \in \mathbb{Z} - \{0\}$ and $b \in \mathbb{Z}$, \pflet{$S(a,b) := \{ a n + b : n \in \mathbb{N} \}$}}
\step{2}{\pflet{$\mathcal{T}$ be the topology generated by the basis $\{ S(a,b) : a \in \mathbb{Z} - \{0\}, b \in \mathbb{Z} \}$}}
\begin{proof}
	\step{a}{For every $n \in \mathbb{Z}$, there exist $a$, $b$ such that $n \in S(a,b)$.}
	\begin{proof}
		\pf\ $n \in S(n,0)$
	\end{proof}
	\step{b}{If $n \in S(a_1,b_1) \cap S(a_2,b_2)$ then there exist $a_3$, $b_3$ such that $n \in S(a_3,b_3) \subseteq S_(a_1,b_1) \cap S(a_2,b_2)$}
	\begin{proof}
		\step{i}{\pflet{$d = \mathrm{lcm}(a_1,a_2)$} \prove{$S(d,n) \subseteq S(a_1,b_1) \cap S(a_2,b_2)$}}
		\step{ii}{\pflet{$d = a_1 k = a_2 l$}}
		\step{iii}{\pflet{$n = a_1 c + b_1 = a_2 d + b_2$}}
		\step{iv}{\pflet{$z \in \mathbb{Z}$} \prove{$dz + n \in S(a_1,b_1) \cap S(a_2,b_2)$}}
		\step{v}{$dz+n \in S(a_1,b_1)$}
		\begin{proof}
			\pf
			\begin{align*}
				dz + n & = a_1 k z + a_1 c + b_1 \\
				& = a_1 (kz + c) + b_1
			\end{align*}
		\end{proof}
		\step{vi}{$dz+n \in S(a_2, b_2)$}
		\begin{proof}
			\pf\ Similar.
		\end{proof}
	\end{proof}
\end{proof}
\step{4}{For all $a \in \mathbb{Z} - \{0\}$ and $b \in \mathbb{Z}$ we have $S(a,b)$ is closed.}
\begin{proof}
	\step{a}{\pflet{$a \in \mathbb{Z} - \{0\}$ and $b \in \mathbb{Z}$}}
	\step{b}{\pflet{$n \in \mathbb{Z} - S(a,b)$}}
	\step{c}{$n \in S(a,n) \subseteq \mathbb{Z} - S(a,b)$}
	\begin{proof}
		\step{i}{\pflet{$x \in S(a,n)$}}
		\step{ii}{\assume{for a contradiction $x \in S(a,b)$}}
		\step{iii}{\pick\ $m$ such that $x = am+b$}
		\step{iv}{\pick\ $l$ such that $x = al + n$}
		\step{v}{$n = a(m-l) + b$}
		\step{vi}{$n \in S(a,b)$}
		\qedstep
		\begin{proof}
			\pf\ This contradicts \stepref{b}.
		\end{proof}
	\end{proof}
\end{proof}
\step{5}{\[ \mathbb{Z} - \{ 1, -1 \} = \bigcup_{p \text{ prime}} S(p,0) \]}
\begin{proof}
	\pf\ Since every integer except 1 and $-1$ is divisible by a prime.
\end{proof}
\step{6}{No nonempty finite set is open.}
\begin{proof}
	\step{a}{\pflet{$U$ be a nonempty open set}}
	\step{b}{\pick\ $n \in U$}
	\step{c}{There exist $a$, $b$ such that $n \in S(a,b) \subseteq U$}
	\step{d}{$U$ is infinite.}
\end{proof}
\step{7}{$\mathbb{Z} - \{ 1, -1 \}$ is not closed.}
\step{8}{$\bigcup_{p \text{ prime}} S(p,0)$ is not closed.}
\step{9}{The union of finitely many closed sets is closed.}
\step{10}{There are infinitely many primes.}
\qed
\end{proof}

\begin{prop}
In a discrete topological space, every set is closed.
\end{prop}

\begin{proof}
\pf\ Immediate from definitions. \qed
\end{proof}

\begin{prop}
In a linearly ordered set under the order topology, every closed interval and closed ray is closed.
\end{prop}

\begin{proof}
\pf
\step{1}{\pflet{$X$ be a linearly ordered set under the order topology.}}
\step{2}{Every closed interval in $X$ is closed.}
\begin{proof}
	\pf\ Since $X - [a,b] = (-\infty, a) \cup (b, +\infty)$.
\end{proof}
\step{3}{Every closed ray in $X$ is closed.}
\begin{proof}
	\pf\ Since $X - [a,+\infty) = (-\infty,a)$ and $X - (-\infty,a] = (a,+\infty)$.
\end{proof}
\qed
\end{proof}

\begin{prop}
Let $X$ be a topological space and $Y$ a subspace of $X$. Let $A \subseteq Y$. Then $A$ is closed in $Y$ if and only if there exists a closed set $B$ in $X$ such that $A = B \cap Y$.
\end{prop}

\begin{proof}
\pf
\begin{align*}
A \text{ is closed in } Y & \Leftrightarrow Y - A \text{ is open in } Y \\
& \Leftrightarrow \exists U \text{ open in } X. Y - A = U \cap Y \\
& \Leftrightarrow \exists C \text{ closed in } X. Y - A = Y - C \\
& \Leftrightarrow \exists C \text{ closed in } X. A = Y \cap C & \qed
\end{align*}
\end{proof}

\begin{prop}
Let $X$ be a topological space and $Y$ a subspace of $X$. Let $A \subseteq Y$. If $A$ is closed in $Y$ and $Y$ is closed in $X$ then $A$ is closed in $X$.
\end{prop}

\begin{proof}
\pf
\step{1}{\pick\ $C$ closed in $X$ such that $A = C \cap Y$.}
\step{2}{$A$ is closed in $X$.}
\begin{proof}
	\pf\ It is the intersection of two closed sets in $X$.
\end{proof}
\qed
\end{proof}

\begin{df}[Neighbourhood]
Let $X$ be a topological space, $Sx \in X$ and $U \subseteq X$. Then $U$ is a \emph{neighbourhood} of $x$, and $x$ is an \emph{interior} point of $U$, iff there exists an open set $V$ such that $x \in V \subseteq U$.
\end{df}

\begin{prop}
A set $B$ is open if and only if it is a neighbourhood of each of its points.
\end{prop}

\begin{prop}
Let $X$ be a set and $\mathcal{N} : X \rightarrow \mathcal{P} X$. Then there exists a topology $\mathcal{O}$ on $X$ such that, for all $x \in X$, we have $\mathcal{N}_x$ is the set of neighbourhoods of $x$, if and only if:
\begin{itemize}
\item For all $x \in X$ and $N \in \mathcal{N}_x$ we have $x \in N$
\item For all $x \in X$ we have $X \in \mathcal{N}_x$
\item For all $x \in X$, $N \in \mathcal{N}_x$ and $V \subseteq \mathcal{P} X$, if $N \subseteq V$ then $V \in \mathcal{N}_x$
\item For all $x \in X$ and $M, N \in \mathcal{N}_x$ we have $M \cap N \in \mathcal{N}_x$
\item For all $x \in X$ and $N \in \mathcal{N}_x$, there exists $M \in \mathcal{N}_x$ such that $M \subseteq N$ and $\forall y \in M. M \in \mathcal{N}_y$.
\end{itemize}
In this case, $\mathcal{O}$ is unique and is given by $\mathcal{O} = \{ U : \forall x \in U. U \in \mathcal{N}_x \}$.
\end{prop}

\begin{df}[Exterior Point]
Let $X$ be a topological space, $x \in X$ and $B \subseteq X$. Then $x$ is an \emph{exterior point} of $B$ iff $B - X$ is a neighbourhood of $x$.
\end{df}

\begin{df}[Boundary Point]
Let $X$ be a topological space, $x \in X$ and $B \subseteq X$. Then $x$ is a \emph{boundary point} of $B$ iff it is neither an interior point nor an exterior point of $B$.
\end{df}

\begin{df}[Interior]
Let $X$ be a topological space and $B \subseteq X$. The \emph{interior} of $B$, $B^\circ$, is the set of all interior points of $B$.
\end{df}

\begin{prop}
The interior of $B$ is the union of all the open sets included in $B$.
\end{prop}

\begin{df}[Closure]
Let $X$ be a topological space and $B \subseteq X$. The \emph{closure} of $B$, $\overline{B}$, is the set of all points that are not exterior points of $B$.
\end{df}

\begin{prop}
The closure of $B$ is the intersection of all the closed sets that include $B$.
\end{prop}

\begin{prop}
A set $B$ is open iff $X - B = \overline{X - B}$.
\end{prop}

\begin{prop}[Kuratowski Closure Axioms]
Let $X$ be a set and $\overline{\ } : \mathcal{P} X \rightarrow \mathcal{P} X$. Then there exists a topology $\mathcal{O}$ such that, for all $B \subseteq X$, $\overline{B}$ is the closure of $B$, if and only if:
\begin{itemize}
\item $\overline{\emptyset} = \emptyset$
\item For all $A \subseteq X$ we have $A \subseteq \overline{A}$
\item For all $A \subseteq X$ we have $\overline{\overline{A}} = \overline{A}$
\item For all $A, B \subseteq X$ we have $\overline{A \cup B} = \overline{A} \cup \overline{B}$
\end{itemize}
In this case, $\mathcal{O}$ is unique and is defined by $\mathcal{O} = \{ U : X - U = \overline{X - U} \}$.
\end{prop}

\begin{df}[Finer, Coarser]
Let $\mathcal{T}$ and $\mathcal{T}'$ be topologies on the set $X$. Then $\mathcal{T}$ is \emph{coarser}, \emph{smaller} or \emph{weaker} than $\mathcal{T}'$, or $\mathcal{T}'$ is \emph{finer}, \emph{larger} or \emph{weaker} than $\mathcal{T}$, iff $\mathcal{T} \subseteq \mathcal{T}'$.
\end{df}

\section{Topological Disjoint Union}

\begin{df}[Coproduct Topology]
Let $\{ X_\alpha \}_{\alpha \in A}$ be a family of topological spaces. The \emph{coproduct topology} on $\coprod_{\alpha \in A} X_\alpha$ is 

\[ \mathcal{T} = \left\{ \coprod_{\alpha \in A} U_\alpha : \{ U_\alpha \}_{\alpha \in A} \text{ is a family with $U_\alpha$ open in $X_\alpha$ for all $\alpha$} \right\} \enspace . \]

We prove this is a topology.
\end{df}

\begin{proof}
\pf
\step{1}{For all $\mathcal{U} \subseteq \mathcal{T}$ we have $\bigcup \mathcal{U} \in \mathcal{T}$}
\begin{proof}
	\pf
	\[ \bigcup_{i \in I} \coprod_{\alpha \in A} U_{i\alpha} = \coprod_{\alpha \in A} \bigcup_{i \in I} U_{i \alpha} \]
\end{proof}
\step{2}{For all $U,V \in \mathcal{T}$ we have $U \cap V \in \mathcal{T}$}
\begin{proof}
	\pf
	\[ \coprod_{\alpha \in A} U_\alpha \cap \coprod_{\alpha \in A} V_\alpha = \coprod_{\alpha \in A} (U_\alpha \cap V_\alpha) \]
\end{proof}
\step{3}{$\coprod_{\alpha \in A} X_\alpha \in \mathcal{T}$}
\begin{proof}
	\pf\ Since every $X_\alpha$ is open in $X_\alpha$.
\end{proof}
\qed
\end{proof}

\begin{prop}
\label{prop:coproduct_finest}
The coproduct topology is the finest topology on $\coprod_{\alpha \in A} X_\alpha$ such that every injection $\kappa_\alpha : X_\alpha \rightarrow \coprod_{\alpha \in A} X_\alpha$ is continuous.
\end{prop}

\begin{proof}
\pf
\step{0}{\pflet{$P = \coprod_{\alpha \in A} X_\alpha$}}
\step{1}{\pflet{$\mathcal{T}_c$ be the coproduct topology.}}
\step{2}{\pflet{$\mathcal{T}$ be any topology on $P$}}
\step{3}{For all $\alpha \in A$, the injection $\kappa_\alpha : X_\alpha \rightarrow (P, \mathcal{T}_c)$ is continuous.}
\begin{proof}
	\step{a}{\pflet{$\alpha \in A$}}
	\step{b}{\pflet{$\{ U_\alpha \}_{\alpha \in A}$ be a family with each $U_\alpha$ open in $X_\alpha$.}}
	\step{c}{For all $\alpha \in A$, we have $\inv{\kappa_\alpha}(\coprod_{\alpha \in A} U_\alpha)$ is open in $X_\alpha$.}
	\begin{proof}
		\pf\ Since $\inv{\kappa_\alpha}(\coprod_{\alpha \in A} U_\alpha) = U_\alpha$.
	\end{proof}
\end{proof}
\step{4}{If, for all $\alpha \in A$, the injection $\kappa_\alpha : X_\alpha \rightarrow (P, \mathcal{T})$ is continuous, then $\mathcal{T} \subseteq \mathcal{T}_c$.}
\begin{proof}
	\step{a}{\assume{For all $\alpha \in A$, the injection $\kappa_\alpha : X_\alpha \rightarrow (P, \mathcal{T})$ is continuous.}}
	\step{b}{\pflet{$U \in \mathcal{T}$}}
	\step{c}{For all $\alpha \in a$, we have $\inv{\kappa_\alpha}(U)$ is open in $X_\alpha$.}
	\step{d}{$U = \coprod_{\alpha \in A} \inv{\kappa_\alpha}(U) \in \mathcal{T}_c$}
\end{proof}
\qed
\end{proof}

\begin{thm}
Let $\{ X_\alpha \}_{\alpha \in A}$ be a family of topological spaces. The coproduct topology is the unique topology on $\coprod_{\alpha \in A} X_\alpha$ such that, for every topological space $Z$ and function $f : \coprod_{\alpha \in A} X_\alpha \rightarrow Z$, we have $f$ is continuous if and only if $\forall \alpha \in A. f \circ \kappa_\alpha$ is continuous.
\end{thm}

\begin{proof}
\pf
\step{1}{\pflet{$X = \coprod_{\alpha \in A} X_\alpha$}}
\step{2}{\pflet{$\mathcal{T}_c$ be the coproduct topology.}}
\step{3}{For every topological space $Z$ and function $f : (X, \mathcal{T}_c) \rightarrow Z$, we have $f$ is continuous if and only if $\forall \alpha \in A. f \circ \kappa_\alpha$ is continuous.}
\begin{proof}
	\step{a}{\pflet{$Z$ be a topological space.}}
	\step{b}{\pflet{$f : X \rightarrow Z$}}
	\step{c}{If $f$ is continuous then $\forall \alpha \in A. f \circ \kappa_\alpha$ is continuous.}
	\begin{proof}
		\pf\ Because the composite of two continuous functions is continuous.
	\end{proof}
	\step{d}{If $\forall \alpha \in A. f \circ \kappa_\alpha$ is continuous then $f$ is continuous.}
	\begin{proof}
		\step{i}{\assume{$\forall \alpha \in A. f \circ \kappa_\alpha$ is continuous.}}
		\step{ii}{\pflet{$U$ be open in $Z$}}
		\step{iii}{For all $\alpha \in A$ we have $\inv{\kappa_\alpha}(\inv{f}(U))$ is open in $X_\alpha$}
		\step{iv}{$\inv{f}(U) = \coprod_{\alpha \in A} \inv{\kappa_\alpha}(\inv{f}(U))$}
		\step{v}{$\inv{f}(U)$ is open in $X$}
	\end{proof}
\end{proof}
\step{4}{For any topology $\mathcal{T}$ on $X$, if for every topological space $Z$ and function $f : (X, \mathcal{T}) \rightarrow Z$, we have $f$ is continuous if and only if $\forall \alpha \in A. f \circ \kappa_\alpha$ is continuous, then $\mathcal{T} = \mathcal{T}_c$.}
\begin{proof}
	\step{a}{\pflet{$\mathcal{T}$ be a topology on $X$.}}
	\step{b}{\assume{For every topological space $Z$ and function $f : (X, \mathcal{T}) \rightarrow Z$, we have $f$ is continuous if and only if $\forall \alpha \in A. f \circ \kappa_\alpha$ is continuous.}}
	\step{c}{$\mathcal{T} \subseteq \mathcal{T}_c$}
	\begin{proof}
		\step{i}{For all $\alpha \in A$ we have $\kappa_\alpha : X_\alpha \rightarrow (X, \mathcal{T})$ is continuous.}
		\begin{proof}
			\pf\ From \stepref{a} since $\id{X}$ is continuous.
		\end{proof}
		\step{ii}{$\mathcal{T} \subseteq \mathcal{T}_c$}
		\begin{proof}
			\pf\ Proposition \ref{prop:coproduct_finest}.
		\end{proof}
	\end{proof}
	\step{d}{$\mathcal{T}_c \subseteq \mathcal{T}$}
	\begin{proof}
		\step{i}{\pflet{$f : (X, \mathcal{T}) \rightarrow (X, \mathcal{T}_c)$ be the identity function.}}
		\step{ii}{$f \circ \kappa_\alpha$ is continuous for all $\alpha$.}
		\step{iii}{$f$ is continuous.}
		\begin{proof}
			\pf\ \stepref{a}
		\end{proof}
		\step{iv}{$\mathcal{T}_c \subseteq \mathcal{T}$}
	\end{proof}
\end{proof}
\qed
\end{proof}

\section{Bases}

\begin{df}[Basis]
Let $X$ be a topological space. A \emph{basis} for the topology on $X$ is a set of open sets $\mathcal{B}$ such that every open set is the union of a subset of $\mathcal{B}$. The elements of $\mathcal{B}$ are called \emph{basic open neighbourhoods} of their elements.
\end{df}

\begin{prop}
Let $X$ be a set. The set of all one-element subsets of $X$ is a basis for the discrete topology on $X$.
\end{prop}

\begin{prop}
Let $X$ be a topological space.
Let $\mathcal{B}$ be a basis for the topology on $X$.
Then the topology on $X$ is the coarsest topology that includes $\mathcal{B}$.
\end{prop}

\begin{prop}
Let $X$ and $Y$ be topological spaces. Let $\mathcal{B}$ be a basis for the topology on $X$ and $\mathcal{C}$ a basis for the topology on $Y$. Then
\[ \{ B \times C : B \in \mathcal{B}, C \in \mathcal{C} \} \]
is a basis for the product topology on $X \times Y$.
\end{prop}

\begin{df}[Order Topology]
Let $X$ be a linearly ordered set. The \emph{order topology} on $X$ is the topology generated by the open interval $(a,b)$ as well as the open rays $(a, + \infty)$ and $(-\infty, b)$ for $a,b \in X$.

The \emph{standard topology} on $\mathbb{R}$ is the order topology.
\end{df}

\begin{prop}
Let $X$ be a linearly ordered set. Then the order topology is generated by the basis consisting of:
\begin{itemize}
\item all open intervals $(a,b)$
\item all intervals of the form $[\bot, b)$ where $\bot$ is the least element of $X$, if any
\item all intervals of the form $(a, \top]$ where $\top$ is the greatest element of $X$, if any.
\end{itemize}
\end{prop}

\begin{prop}
Let $X$ be a linearly ordered set. The open rays in $X$ form a subbasis for the order topology.
\end{prop}

\begin{df}[Lower Limit Topology]
The \emph{lower limit topology}, \emph{Sorgenfrey topology}, \emph{uphill topology} or \emph{half-open topology} is the topology on $\mathbb{R}$ generated by the basis consisting of all half-open intervals $[a,b)$.

We write $\mathbb{R}_l$ for $\mathbb{R}$ under the lower limit topology.
\end{df}

\begin{df}[$K$-topology]
Let $K = \{ 1/n : n \in \mathbb{Z}_+ \}$. The \emph{$K$-topology} on $\mathbb{R}$ is the topology generated by the basis consisting of all open intervals $(a,b)$ and all sets of the form $(a,b) - K$.

We write $\mathbb{R}_K$ for $\mathbb{R}$ under the $K$ -topology.
\end{df}

\begin{df}[Ordered Square]
The \emph{ordered square} $I_o^2$ is the set $[0,1]^2$ under the order topology induced by the dictionary order.
\end{df}

\begin{prop}
Let $X$ be a linearly ordered set under the order topology. Let $Y \subseteq X$ be convex. Then the order topology on $Y$ is the same as the subspace topology.
\end{prop}

\begin{proof}
\pf
\step{1}{The order topology is coarser than the subspace topology.}
\begin{proof}
	\step{a}{For all $a \in Y$, the open ray $\{ y \in Y : a < y \}$ is open in the subspace topology.}
	\begin{proof}
		\pf\ It is $(a, +\infty) \cap Y$.
	\end{proof}
	\step{b}{For all $a \in Y$, the open ray $\{ y \in Y : y < a \}$ is open in the subspace topology.}
	\begin{proof}
		\pf\ It is $(-\infty, a) \cap Y$.
	\end{proof}
\end{proof}
\step{2}{The subspace topology is coarser than the order topology.}
\begin{proof}
	\step{b}{For all $a \in X$, the set $(-\infty, a) \cap Y$ is open in the order topology.}
	\begin{proof}
		\step{i}{\case{$a \in Y$}}
		\begin{proof}
			\pf\ Then $(-\infty, a) \cap Y = \{ y \in Y : y < a\}$ is an open ray in $Y$.
		\end{proof}
		\step{ii}{\case{$a$ is an upper bound for $Y$}}
		\begin{proof}
			\pf\ Then $(-\infty, a) \cap Y = Y$. 
		\end{proof}
		\step{iii}{\case{$a$ is a lower bound for $Y$}}
		\begin{proof}
			\pf\ Then $(-\infty, a) \cap Y = \emptyset$. 
		\end{proof}
		\qedstep
		\begin{proof}
			\pf\ These are the only three cases because $Y$ is convex.
		\end{proof}
	\end{proof}
	\step{c}{For all $a \in X$, the set $(a, +\infty) \cap Y$ is open in the order topology.}
	\begin{proof}
		\pf\ Similar.
	\end{proof}
\end{proof}
\qed
\end{proof}

\begin{ex}
We cannot remove the hypothesis that the set $Y$ is convex.

Let $X = \mathbb{R}$ and $Y = [0,1) \cup \{2\}$. Then $\{2\}$ is open in the subspace topology but not in the order topology on $Y$.
\end{ex}

\begin{prop}
Let $X$ be a topological space. Let $\mathcal{B}$ be a basis for the topology on $X$ and $U \subseteq X$. Then $U$ is open if and only if, for all $x \in U$, there exists $B \in \mathcal{B}$ such that $x \in B \subseteq U$.
\end{prop}

\begin{prop}
Let $X$ be a topological space and $\mathcal{B} \subseteq X$. Assume that, for every open set $U$ and element $x \in U$, there exists $B \in \mathcal{B}$ such that $x \in B \subseteq U$. Then $\mathcal{B}$ is a basis for the topology on $X$.
\end{prop}

\begin{prop}
Let $X$ be a topological space and $\mathcal{B} \subseteq \mathcal{P} X$. Then $\mathcal{B}$ is a basis for a topology on $X$ if and only if:
\begin{enumerate}
\item $\bigcup \mathcal{B} = X$
\item For all $A, B \in \mathcal{B}$ and $x \in A \cap B$, there exists $C \in \mathcal{B}$ such that $x \in C \subseteq A \cap B$.
\end{enumerate}
In this case, the topology is unique and is the set of all unions of subsets of $\mathcal{B}$. We call it the topology \emph{generated} by $\mathcal{B}$.
\end{prop}

\begin{prop}
Let $\mathcal{B}$ and $\mathcal{B}'$ be bases for the topologies $\mathcal{T}$ and $\mathcal{T}'$, respectively, on $X$. Then $\mathcal{T}'$ is finer than $\mathcal{T}$ if and only if, for every $B \in \mathcal{B}$ and $x \in B$, there exists $B' \in \mathcal{B}'$ such that $x \in B' \subseteq B$.
\end{prop}

\begin{cor}
The topologies of $\mathbb{R}_l$ and $\mathbb{R}_K$ are strictly finer than the standard topology on $\mathbb{R}$ but are not comparable to one another.
\end{cor}

\subsection{Subspaces}

\begin{prop}
\label{prop:basis_subspace}
Let $X$ be a topological space. Let $Y$ be a subspace of $X$. Let $\mathcal{B}$ be a basis for the topology on $X$. Then $\{ B \cap Y : B \in \mathcal{B} \}$ is a basis for the topology on $Y$.
\end{prop}

\begin{proof}
\pf
\step{1}{For all $B \in \mathcal{B}$ we have $B \cap Y$ is open in $Y$.}
\begin{proof}
	\pf\ Since $B$ is open in $X$.
\end{proof}
\step{2}{For any open set $V$ in $Y$ and $y \in V$, there exists $B \in \mathcal{B}$ such that $y \in B \cap Y \subseteq V$.}
\begin{proof}
	\step{a}{\pflet{$V$ be open in $Y$.}}
	\step{b}{\pflet{$y \in V$}}
	\step{c}{\pick\ $U$ open in $X$ such that $V = U \cap Y$.}
	\step{d}{\pick\ $B \in \mathcal{B}$ such that $y \in B \subseteq U$.}
	\step{e}{$y \in B \cap Y \subseteq V$}
\end{proof}
\qed
\end{proof}

\subsection{Product Topology}

\begin{prop}
Let $\{X_i\}_{i \in I}$ be a family of topological spaces. For all $i \in I$, let $\mathcal{B}_i$ be a basis for the topology on $X_i$. Then $\mathcal{B} = \left\{ \prod_{i \in I} B_i : \text{for finitely many $i \in I$ we have $B_i \in \mathcal{B}_i$, and $B_i = X_i$ for all other $i$} \right\}$ is a basis for the product topology on $\prod_{i \in I} X_i$.
\end{prop}

\begin{proof}
\pf
\step{1}{Every $B \in \mathcal{B}$ is open in the product topology.}
\begin{proof}
	\pf\ Since every element of $\mathcal{B}_i$ is open in $X_i$.
\end{proof}
\step{2}{For any open set $U$ in the product topology and $x \in U$, there exists $B \in \mathcal{B}$ such that $x \in B \subseteq U$.}
\begin{proof}
	\step{a}{\pflet{$U$ be a set open in the box topology.}}
	\step{b}{\pflet{$x \in U$}}
	\step{c}{\pick\ a family $\{U_i\}_{i \in I}$ where $U_i$ is open in $X_i$ for $i = i_1, \ldots, i_n$, and $U_i = X_i$ for all other $i$, such that $x \in \prod_{i \in I} U_i \subseteq U$}
	\step{d}{For $i = i_1, \ldots, i_n$, choose $B_i \in \mathcal{B}_i$ such that $x_i \in B_i \subseteq U_i$. Let $B_i = X_i$ for all other $i$.}
	\step{e}{$\prod_{i \in I} B_i \in \mathcal{B}$}
	\step{f}{$x \in \prod_{i \in I} B_i \subseteq \prod_{i \in I} U_i \subseteq U$}
\end{proof}
\qed
\end{proof}

\section{Subbases}

\begin{df}[Subbasis]
Let $X$ be a topological space. A \emph{subbasis} for the topology on $X$ is a set $\mathcal{S}$ of open sets such that every open set is a union of finite intersections of $\mathcal{S}$.
\end{df}

\begin{prop}
Let $X$ be a set and $\mathcal{S} \subseteq X$. Then $\mathcal{S}$ is a subbasis for a topology on $X$ if and only if $\bigcup \mathcal{S} = X$, in which case the topology is unique and is the set of all unions of finite intersections of elements of $\mathcal{S}$.
\end{prop}

\begin{prop}
Let $X$ be a topological space.
Let $\mathcal{S}$ be a subbasis for the topology on $X$.
Then the topology on $X$ is the coarsest topology that includes $\mathcal{S}$.
\end{prop}


\begin{prop}
Let $X$ and $Y$ be topological spaces. Then
\[ \mathcal{S} = \{ \inv{\pi_1}(U) : U \text{ is open in } X \} \cup \{ \inv{\pi_2}(V) : V \text{ is open in } Y \} \]
is a subbasis for the product topology on $X \times Y$.
\end{prop}

\begin{proof}
\pf
\step{1}{Every element of $\mathcal{S}$ is open.}
\begin{proof}
	\pf\ Since $\inv{\pi_1}(U) = U \times Y$ and $\inv{\pi_2}(V) = X \times V$.
\end{proof}
\step{2}{Every open set is a union of finite intersections of elements of $\mathcal{S}$.}
\begin{proof}
	\pf\ Since, for $U$ open in $X$ and $V$ open in $Y$, we have $U \times V = \inv{\pi_1}(U) \cap \inv{\pi_2}(V)$.
\end{proof}
\qed
\end{proof}

\begin{df}[Space with Basepoint]
A \emph{space with basepoint} is a pair $(X,x)$ where $X$ is a topological space and $x \in X$.
\end{df}

\section{Neighbourhood Bases}

\begin{df}[Neighbourhood Basis]
Let $X$ be a topological space and $x_0 \in X$. A \emph{neighbourhood basis} of $x_0$ is a set $\mathcal{U}$ of neighbourhoods of $x_0$ such that every neighbourhood of $x_0$ includes an element of $\mathcal{U}$.
\end{df}

\section{First Countable Spaces}

\begin{df}[First Countable]
A topological space is \emph{first countable} iff every point has a countable neighbourhood basis.
\end{df}

\begin{prop}
$\mathbb{R}_l$ is first countable.
\end{prop}

\begin{proof}
\pf\ For any $x \in \mathbb{R}$ we have $\{ [x, x + 1/n) : n \in \mathbb{Z}_+ \}$ is a countable local basis. \qed
\end{proof}

\begin{prop}
The ordered square is first countable.
\end{prop}

\begin{proof}
\pf
\step{1}{Every point $(a,b)$ with $0 < b < 1$ has a countable local basis.}
\begin{proof}
	\pf\ The set of all intervals $((a,q),(a,r))$ where $q$ and $r$ are rational and $0 \leq q < b < r \leq 1$ is a countable local basis.
\end{proof}
\step{2}{Every point $(a,0)$ has a countable local basis with $a > 0$.}
\begin{proof}
	\pf\ The set of all intervals $((q,0),(a,r))$ where $q$ and $r$ are rational with $0 \leq q < a$ and $0 < r \leq 1$ is a countable local basis.
\end{proof}
\step{3}{Every point $(a,1)$ has a countable local basis with $a < 1$.}
\begin{proof}
	\pf\ The set of all intervals $((a,q),(r,1))$ with $q$ and $r$ rational and $0 \leq q < 1$, $a < r \leq 1$ is a countable local basis.
\end{proof}
\step{4}{$(0,0)$ has a countable local basis.}
\begin{proof}
	\pf\ The set of all intervals $[(0,0),(0,r))$ with $r$ rational and $0 < r \leq 1$ is a countable local basis.
\end{proof}
\step{5}{$(1,1)$ has a countable local basis.}
\begin{proof}
	\pf\ The set of all intervals $((1,q),(1,1)]$ with $q$ rational and $0 \leq q < 1$ is a countable local basis.
\end{proof}
\qed
\end{proof}
\section{Second Countable Spaces}

\begin{df}[Second Countable]
A topological space is \emph{second countable} iff it has a countable basis.
\end{df}

Every second countable space is first countable.

A subspace of a first countable space is first countable.

A subspace of a second countable space is second countable.

$\mathbb{R}^n$ is second countable.

An uncountable discrete space is first countable but not second countable.

\begin{prop}
Let $\{ X_\lambda \}_{\lambda \in \Lambda}$ be a family of topological spaces such that no $X_\lambda$ is indiscrete. If $\Lambda$ is uncountable, then $\prod_{\lambda \in \Lambda} X_\lambda$ is not first countable.
\end{prop}

\begin{proof}
\pf
\step{1}{For all $\lambda \in \Lambda$, \pick\ $U_\lambda$ open in $X_\lambda$ such that $\emptyset \neq U_\lambda \neq X_\lambda$.}
\step{2}{For all $\lambda \in \Lambda$, \pick\ $x_\lambda \in U_\lambda$.}
\step{3}{\assume{for a contradiction $B$ is a countable neighbourhood basis for $(x_\lambda)_{\lambda \in \Lambda}$.}}
\step{4}{\pick\ $\lambda \in \Lambda$ such that, for all $U \in B$, we have $\pi_\lambda(U) = X_\lambda$}
\step{5}{There is no $U \in \lambda$ such that $U \subseteq \pi_\lambda^{-1}(U_\lambda)$}
\qedstep
\begin{proof}
\pf\ This is a contradiction.
\end{proof}
\qed
\end{proof}

\section{Interior}

\begin{df}[Interior]
Let $X$ be a topological space. Let $A \subseteq X$. The \emph{interior} of $A$, $A^\circ$, is the union of all the open sets included in $A$.
\end{df}

\section{Closure}

\begin{df}[Closure]
Let $X$ be a topological space. Let $A \subseteq X$. The \emph{closure} of $A$, $\overline{A}$, is the intersection of all the closed sets that include $A$.
\end{df}

\begin{prop}
\label{prop:closure}
Let $X$ be a topological space, $A \subseteq X$ and $x \in X$. Then $x \in \overline{A}$ if and only if every open set that contains $x$ intersects $A$.
\end{prop}

\begin{proof}
\pf
\begin{align*}
x \in \overline{A} & \Leftrightarrow \text{for every closed set $C$, if $A \subseteq C$ then $x \in C$} \\
& \Leftrightarrow \text{for every open set $U$, if $A \subseteq X - U$ then $x \in X - U$} \\
& \Leftrightarrow \text{for every open set $U$, if $A \cap U = \emptyset$ then $x \notin U$} \\
& \Leftrightarrow \text{for every open set $U$, if $x \in U$ then $A$ intersects $U$} & \qed
\end{align*}
\end{proof}

\begin{prop}
\label{prop:closure_monotone}
Let $X$ be a topological space. Let $A \subseteq B \subseteq X$. Then $\overline{A} \subseteq \overline{B}$.
\end{prop}

\begin{proof}
\pf\ Since every closed set that includes $B$ is a closed set that includes $A$. \qed
\end{proof}

\begin{prop}
Let $X$ be a topological space. Let $A,B \subseteq X$. Then $\overline{A \cup B} = \overline{A} \cup \overline{B}$.
\end{prop}

\begin{proof}
\pf
\step{1}{$\overline{A \cup B} \subseteq \overline{A} \cup \overline{B}$}
\begin{proof}
\pf\ Since $\overline{A} \cup \overline{B}$ is a closed set that includes $A \cup B$.
\end{proof}
\step{2}{$\overline{A} \cup \overline{B} \subseteq \overline{A \cup B}$}
\begin{proof}
	\pf\ Since $\overline{A} \subseteq \overline{A \cup B}$ and $\overline{B} \subseteq \overline{A \cup B}$ by Proposition \ref{prop:closure_monotone}.
\end{proof}
\qed
\end{proof}

\begin{prop}
Let $X$ be a topological space. Let $\mathcal{A} \subseteq \mathcal{P} X$. Then
\[ \bigcup \{ \overline{A} : A \in \mathcal{A} \} \subseteq \overline{\bigcup \mathcal{A}} \enspace . \]
\end{prop}

\begin{proof}
\pf
For all $A \in \mathcal{A}$ we have $\overline{A} \subseteq \overline{\bigcup \mathcal{A}}$ by Proposition \ref{prop:closure_monotone}. \qed
\end{proof}

\begin{ex}
The converse does not always hold. In $\mathbb{R}$, let $\mathcal{A} = \{ \{ x \} : 0 < x < 1 \}$. Then $\bigcup \{ \overline{A} : A \in \mathcal{A} \} = (0,1)$ but $\overline{\bigcup \mathcal{A}} = [0,1]$.
\end{ex}

\begin{prop}
Let $X$ be a topological space. Let $\mathcal{A} \subseteq \mathcal{P} X$. Then $\overline{\bigcap \mathcal{A}} \subseteq \bigcap \{ \overline{A} : A \in \mathcal{A} \}$.
\end{prop}

\begin{proof}
\pf\ Since $\overline{\bigcap \mathcal{A}} \subseteq \overline{A}$ for all $A \in \mathcal{A}$ by Proposition \ref{prop:closure_monotone}. \qed
\end{proof}

\begin{ex}
The converse does not always hold. In $\mathbb{R}$, if $A$ is the set of all rational numbers and $B$ is the set of all irrational numbers then $\bigcap{A \cap B} = \emptyset$ but $\bigcap{A} \cap \bigcap{B} = \mathbb{R}$.
\end{ex}

\subsection{Bases}

\begin{prop}
\label{prop:closure_basis}
Let $X$ be a topological space, $A \subseteq X$ and $x \in X$. Let $\mathcal{B}$ be a basis for the topology on $X$. Then $x \in \overline{A}$ if and only if, for all $B \in \mathcal{B}$, if $x \in B$ then $B$ intersects $A$.
\end{prop}

\begin{proof}
\pf
\step{1}{If $x \in \overline{A}$ then, for all $B \in \mathcal{B}$, if $x \in B$ then $B$ intersects $A$.}
\begin{proof}
	\pf\ Proposition \ref{prop:closure} since every element of $\mathcal{B}$ is open.
\end{proof}
\step{2}{If, for all $B \in \mathcal{B}$, if $x \in B$ then $B$ intersects $A$, then $x \in \overline{A}$.}
\begin{proof}
	\step{a}{\assume{For all $B \in \mathcal{B}$, if $x \in B$ then $B$ intersects $A$.}}
	\step{b}{\pflet{$U$ be an open set that contains $x$.}}
	\step{c}{\pick\ $B \in \mathcal{B}$ such that $x \in B \subseteq U$.}
	\step{d}{$B$ intersects $A$.}
	\begin{proof}
		\pf\ \stepref{a}
	\end{proof}
	\step{e}{$U$ intersects $A$.}
\end{proof}
\qed
\end{proof}

\subsection{Subspaces}

\begin{prop}
Let $X$ be a topological space. Let $Y$ be a subspace of $X$. Let $A \subseteq Y$. Let $\overline{A}$ be the closure of $A$ in $X$. Then the closure of $A$ in $Y$ is $\overline{A} \cap Y$.
\end{prop}

\begin{proof}
\pf
\step{1}{$\overline{A} \cap Y$ is the closed in $Y$.}
\begin{proof}
	\pf\ Since $\overline{A}$ is closed in $X$.
\end{proof}
\step{2}{For any closed set $B$ in $Y$, if $A \subseteq B$ then $\overline{A} \cap Y \subseteq B$.}
\begin{proof}
	\step{a}{\pflet{$B$ be closed in $Y$.}}
	\step{b}{\assume{$A \subseteq B$}}
	\step{c}{\pick\ $C$ closed in $X$ such that $B = C \cap Y$.}
	\step{d}{$A \subseteq C$}
	\step{e}{$\overline{A} \subseteq C$}
	\step{f}{$\overline{A} \cap Y \subseteq B$}
\end{proof}
\qed
\end{proof}

\subsection{Product Topology}

\begin{prop}
Let $X$ and $Y$ be topological spaces. Let $A \subseteq X$ and $B \subseteq Y$. Then $\overline{A \times B} = \overline{A} \times \overline{B}$.
\end{prop}

\begin{proof}
\pf
\step{1}{$\overline{A \times B} \subseteq \overline{A} \times \overline{B}$}
\begin{proof}
	\pf\ Since $\overline{A} \times \overline{B}$ is a closed set that includes $A \times B$ by Proposition \ref{prop:closed_product}.
\end{proof}
\step{2}{$\overline{A} \times \overline{B} \subseteq \overline{A \times B}$}
\begin{proof}
	\step{a}{\pflet{$x \in \overline{A}$ and $y \in \overline{B}$.}}
	\step{b}{\pflet{$U$ be an open set that contains $(x,y)$.}}
	\step{c}{\pick\ open sets $V$ in $X$ and $W$ in $Y$ such that $(x,y) \in V \times W \subseteq U$.}
	\step{d}{$V$ intersects $A$ and $W$ intersects $B$.}
	\step{e}{$U$ intersects $A \times B$.}
\end{proof}
\qed
\end{proof}

\subsection{Interior}

\begin{prop}
Let $X$ be a topological space and $A \subseteq X$. Then
\[ X - A^\circ = \overline{X - A} \]
\end{prop}

\begin{proof}
\pf
\begin{align*}
X - A^\circ & = X - \bigcup \{ U \text{ open in } X : U \subseteq A \} \\
& = \bigcap \{ X - U : U \text{ open in } X, U \subseteq A \} & (\text{De Morgan's Law}) \\
& = \bigcap \{ C : C \text{ closed in } X, X - A \subseteq C \} \\
& = \overline{X-A} & \qed
\end{align*}
\end{proof}

\begin{prop}
Let $X$ be a topological space and $A \subseteq X$. Then
\[ X - \overline{A} = (X - A)^\circ \]
\end{prop}

\begin{proof}
\pf\ Dual. \qed
\end{proof}

\section{Boundary}

\begin{df}[Boundary]
Let $X$ be a topological space. Let $A \subseteq X$. The \emph{boundary} of $A$ is
\[ \partial A := \overline{A} \cap \overline{X - A} \enspace . \]
\end{df}

\begin{prop}
Let $X$ be a topological space. Let $A \subseteq X$. Then
\[ A^\circ \cap \partial A = \emptyset \enspace . \]
\end{prop}

\begin{proof}
\pf
\step{1}{$A^\circ \subseteq A$}
\step{2}{$X - A \subseteq X - A^\circ$}
\step{3}{$\overline{X - A} \subseteq X - A^\circ$}
\step{4}{$\partial A \subseteq X - A^\circ$}
\qed
\end{proof}

\begin{prop}
\label{prop:closure_boundary}
Let $X$ be a topological space. Let $A \subseteq X$. Then
\[ \overline{A} = A^\circ \cup \partial A \]
\end{prop}

\begin{proof}
\step{1}{$A^\circ \subseteq \overline{A}$}
\begin{proof}
	\pf\ Since $A^\circ \subseteq A \subseteq \overline{A}$.
\end{proof}
\step{2}{$\partial A \subseteq \overline{A}$}
\begin{proof}
	\pf\ Definition of $\partial A$.
\end{proof}
\step{3}{$\overline{A} \subseteq A^\circ \cup \partial A$}
\begin{proof}
	\step{a}{\pflet{$x \in \overline{A}$}}
	\step{b}{\assume{$x \notin A^\circ$} \prove{$x \in \partial A$}}
	\step{c}{$x \in \overline{X-A}$}
	\begin{proof}
		\pf\ Since $\overline{X-A} = X - A^\circ$.
	\end{proof}
	\step{d}{$x \in \partial A$}
	\begin{proof}
		\pf\ Since $\partial A = \overline{A} \cap \overline{X-A}$.
	\end{proof}
\end{proof}
\qed
\end{proof}

\begin{prop}
Let $X$ be a topological space. Let $A \subseteq X$. Then $\partial A = \emptyset$ if and only if $A$ is both open and closed.
\end{prop}

\begin{proof}
\pf
\step{1}{If $\partial A = \emptyset$ then $A$ is open and closed.}
\begin{proof}
	\step{a}{\assume{$\partial A = \emptyset$}}
	\step{b}{$\overline{A} = A^\circ$}
	\begin{proof}
		\pf\ Proposition \ref{prop:closure_boundary}.
	\end{proof}
	\step{c}{$\overline{A} = A = A^\circ$}
\end{proof}
\step{2}{If $A$ is open and closed then $\partial A = \emptyset$.}
\begin{proof}
	\pf\ If $A$ is open and closed then
	\begin{align*}
	\partial A & = \overline{A} \cap \overline{X-A} \\
	& = \overline{A} \cap (X - A^\circ) \\
	& = A \cap (X - A) \\
	& = \emptyset
	\end{align*}
\end{proof}
\qed
\end{proof}

\begin{prop}
Let $X$ be a topological space. Let $U \subseteq X$. Then $U$ is open if and only if $\partial U = \overline{U} - U$.
\end{prop}

\begin{proof}
\pf
\step{1}{If $U$ is open then $\partial U = \overline{U} - U$}
\begin{proof}
	\pf\ If $U$ is open then
	\begin{align*}
		\partial U & = \overline{U} \cap \overline{X-U} \\
		& = \overline{U} \cap (X - U^\circ) \\
		& = \overline{U} - U^\circ \\
		& = \overline{U} - U 
	\end{align*}
\end{proof}
\step{2}{If $\partial U = \overline{U} - U$ then $U$ is open.}
\begin{proof}
	\step{a}{\assume{$\partial U = \overline{U} - U$}}
	\step{b}{$\overline{U} - U^\circ = \overline{U} - U$}
	\step{c}{$U \subseteq U^\circ$}
	\step{d}{$U = U^\circ$}
\end{proof}
\qed
\end{proof}

\section{Limit Points}

\begin{df}[Limit Point]
Let $X$ be a topological space, $x \in X$ and $A \subseteq X$. Then $x$ is a \emph{limit point}, \emph{cluster point} or \emph{point of accumulation} of $A$ iff every neighbourhood of $x$ intersects $A - \{x\}$.
\end{df}

\begin{prop}
Let $X$ be a topological space. Let $A \subseteq X$. Let $A'$ be the set of limit points of $A$. Then
\[ \overline{A} = A \cup A' \]
\end{prop}

\begin{proof}
\pf
\step{1}{$\overline{A} \subseteq A \cup A'$}
\begin{proof}
	\step{a}{\pflet{$x \in \overline{A}$}}
	\step{b}{\assume{$x \notin A$} \prove{$x \in A'$}}
	\step{c}{\pflet{$U$ be a neighbourhood of $x$.}}
	\step{d}{\pick\ $y \in U \cap A$}
	\begin{proof}
		\pf\ Proposition \ref{prop:closure}.
	\end{proof}
	\step{e}{$y \neq x$}
\end{proof}
\step{2}{$A \subseteq \overline{A}$}
\begin{proof}
	\pf\ Immediate from the definition of $\overline{A}$.
\end{proof}
\step{3}{$A' \subseteq \overline{A}$}
\begin{proof}
	\pf\ From Proposition \ref{prop:closure}.
\end{proof}
\qed
\end{proof}

\begin{cor}
A set is closed if and only if it contains all its limit points.
\end{cor}

\section{Continuous Functions}

\begin{df}[Continuous]
Let $X$ and $Y$ be topological spaces. A function $f : X \rightarrow Y$ is \emph{continuous} iff, for every open set $V$ in $Y$, the inverse image $\inv{f}(V)$ is open in $X$.
\end{df}

\begin{prop}
The composite of two continuous functions is continuous.
\end{prop}

\begin{proof}
\pf
\step{1}{\pflet{$f : X \rightarrow Y$ and $g : Y \rightarrow Z$ be continuous.}}
\step{2}{\pflet{$U$ be open in $Z$.}}
\step{3}{$\inv{g}(U)$ is open in $Y$.}
\step{4}{$\inf{f}(\inv{g}(U))$ is open in $X$.}
\qed
\end{proof}

\begin{prop}
\begin{enumerate}
\item $\id{X}$ is continuous
\item If $f : X \rightarrow Y$ is continuous and $X_0 \subseteq X$ then $f \restriction X_0 : X_0 \rightarrow Y$ is continuous.
\item If $f : X + Y \rightarrow Z$, then $f$ is continuous iff $f \circ \kappa_1 : X \rightarrow Z$ and $f \circ \kappa_2 : Y \rightarrow Z$ are continuous.
\item If $f : Z \rightarrow X \times Y$, then $f$ is continuous iff $\pi_1 \circ f$ and $\pi_2 \circ f$ are continuous.
\end{enumerate}
\end{prop}

\begin{prop}
Let $X$ and $Y$ be topological spaces. Let $f : X \rightarrow Y$. Then the following are equivalent.
\begin{enumerate}
\item $f$ is continuous.
\item For all $A \subseteq X$ we have $f(\overline{A}) \subseteq \overline{f(A)}$.
\item For every closed $B$ in $Y$, we have $\inv{f}(B)$ is closed in $X$.
\end{enumerate}
\end{prop}

\begin{proof}
\pf
\step{1}{$1 \Rightarrow 2$}
\begin{proof}
	\step{a}{\assume{$f$ is continuous.}}
	\step{b}{\pflet{$A \subseteq X$}}
	\step{c}{\pflet{$x \in \overline{A}$} \prove{$f(x) \in \overline{f(A)}$}}
	\step{d}{\pflet{$V$ be a neighbourhood of $f(x)$.} \prove{$V$ intersects $f(A)$.}}
	\step{e}{$\inv{f}(V)$ is a neighbourhood of $x$.}
	\step{f}{\pick\ $y \in \inv{f}(V) \cap A$}
	\step{g}{$f(y) \in V \cap f(A)$}
\end{proof}
\step{2}{$2 \Rightarrow 3$}
\begin{proof}
	\step{a}{\assume{2}}
	\step{b}{\pflet{$B$ be closed in $Y$}}
	\step{c}{\pflet{$A = \inv{f}(B)$} \prove{$\overline{A} = A$}}
	\step{d}{$f(A) \subseteq B$}
	\step{e}{$\overline{A} \subseteq A$}
	\begin{proof}
		\step{i}{\pflet{$x \in \overline{A}$}}
		\step{ii}{$f(x) \in B$}
		\begin{proof}
			\pf
			\begin{align*}
				f(x) & \in f(\overline{A}) \\
				& \subseteq \overline{f(A)} & (\text{\stepref{a}}) \\
				& \subseteq \overline{B} & (\text{\stepref{d}}) \\
				& = B & (\text{\stepref{b}})
			\end{align*}
		\end{proof}
	\end{proof}
\end{proof}
\step{3}{$3 \Rightarrow 1$}
\begin{proof}
	\step{a}{\assume{3}}
	\step{b}{\pflet{$V$ be open in $Y$.}}
	\step{c}{$\inv{f}(Y - V)$ is closed in $X$.}
	\step{d}{$X - \inv{f}(V)$ is closed in $X$.}
	\step{e}{$\inv{f}(V)$ is open in $X$.}
\end{proof}
\qed
\end{proof}

\begin{prop}
Let $X$ and $Y$ be topological spaces. Any constant function $X \rightarrow Y$ is continuous.
\end{prop}

\begin{proof}
\pf
\step{1}{\pflet{$b \in Y$}}
\step{2}{\pflet{$f : X \rightarrow Y$ be the constant function with value $b$.}}
\step{3}{\pflet{$V \subseteq Y$ be open.}}
\step{4}{$\inv{f}(V)$ is either $\emptyset$ or $X$.}
\step{5}{$\inv{f}(V)$ is open.}
\qed
\end{proof}

\begin{prop}
\label{prop:continuous_basis}
Let $X$ and $Y$ be topological spaces. Let $f : X \rightarrow Y$. Let $\mathcal{B}$ be a basis for $Y$. Then $f$ is continuous if and only if, for all $B \in \mathcal{B}$, we have $\inv{f}(B)$ is open in $X$.
\end{prop}

\begin{proof}
\pf
\step{1}{If $f$ is continuous then, for all $B \in \mathcal{B}$, we have $\inv{f}(B)$ is open in $X$.}
\begin{proof}
	\pf\ Since every element of $\mathcal{B}$ is open in $Y$.
\end{proof}
\step{2}{If, for all $B \in \mathcal{B}$, we have $\inv{f}(B)$ is open in $X$, then $f$ is continuous.}
\begin{proof}
	\step{a}{\assume{For all $B \in \mathcal{B}$, we have $\inv{f}(B)$ is open in $X$.}}
	\step{b}{\pflet{$U$ be open in $Y$.}}
	\step{c}{\pflet{$x \in \inv{f}(U)$}}
	\step{d}{\pick\ $B \in \mathcal{B}$ such that $f(x) \in B \subseteq U$.}
	\step{e}{$x \in \inv{f}(B) \subseteq \inv{f}(U)$}
\end{proof}
\qed
\end{proof}

\begin{prop}
Let $X$ and $Y$ be topological spaces.
Let $f : X \rightarrow Y$. Let $\mathcal{S}$ be a subbasis for the topology on $Y$. Then $f$ is continuous if and only if, for all $V \in \mathcal{S}$, we have $\inv{f}(V)$ is open in $X$.
\end{prop}

\begin{proof}
\pf
\step{1}{If $f$ is continuous then, for all $V \in \mathcal{S}$, we have $\inv{f}(V)$ is open in $X$.}
\begin{proof}
	\pf\ Immediate from definitions.
\end{proof}
\step{2}{If, for all $V \in \mathcal{S}$, we have $\inv{f}(V)$ is open in $X$, then $f$ is continuous.}
\begin{proof}
	\step{a}{\assume{For all $V \in \mathcal{S}$, we have $\inv{f}(V)$ is open in $X$.}}
	\step{b}{For all $V_1, \ldots, V_n \in \mathcal{S}$ we have $\inv{f}(V_1 \cap \cdots \cap V_n)$ is open in $X$.}
	\begin{proof}
		\pf\ Since $\inv{f}(V_1 \cap \cdots \cap V_n) = \inv{f}(V_1) \cap \cdots \cap \inv{f}(V_n)$.
	\end{proof}
	\qedstep
	\begin{proof}
		\pf\ By Proposition \ref{prop:continuous_basis} since the set of all finite intersections of elements of $\mathcal{S}$ forms a basis for the topology on $Y$.
	\end{proof}
\end{proof}
\qed
\end{proof}

\begin{prop}
\label{prop:continuous_subbasis}
Let $f : \mathbb{R} \rightarrow \mathbb{R}$. Then $f$ is continuous if and only if, for all $x \in \mathbb{R}$ and $\epsilon > 0$, there exists $\delta > 0$ such that, for all $y \in \mathbb{R}$, if $|y-x| < \delta$ then $|f(y) - f(x)| < \epsilon$.
\end{prop}

\begin{proof}
\pf
\step{1}{If $f$ is continuous then, for all $x \in \mathbb{R}$ and $\epsilon > 0$, there exists $\delta > 0$ such that, for all $y \in \mathbb{R}$, if $|y-x| < \delta$ then $|f(y) - f(x)| < \epsilon$.}
\begin{proof}
	\step{a}{\assume{$f$ is continuous.}}
	\step{b}{\pflet{$x \in \mathbb{R}$}}
	\step{c}{\pflet{$\epsilon > 0$}}
	\step{d}{$f^{-1}((f(x) - \epsilon, f(x) + \epsilon))$ is open in $X$.}
	\step{e}{\pick\ $a$, $b$ such that $x \in (a,b) \subseteq f^{-1}((f(x) - \epsilon, f(x) + \epsilon))$.}
	\step{f}{\pflet{$\delta = \min(x - a, b - x)$}}
	\step{g}{\pflet{$y \in \mathbb{R}$}}
	\step{h}{\assume{$|y-x| < \delta$}}
	\step{i}{$y \in (a,b)$}
	\step{j}{$f(y) \in (f(x) - \epsilon, f(x) + \epsilon)$}
	\step{k}{$|f(y) - f(x)| < \epsilon$}
\end{proof}
\step{2}{If, for all $x \in \mathbb{R}$ and $\epsilon > 0$, there exists $\delta > 0$ such that, for all $y \in \mathbb{R}$, if $|y-x| < \delta$ then $|f(y) - f(x)| < \epsilon$, then $f$ is continuous.}
\begin{proof}
	\step{a}{\assume{For all $x \in \mathbb{R}$ and $\epsilon > 0$, there exists $\delta > 0$ such that, for all $y \in \mathbb{R}$, if $|y-x| < \delta$ then $|f(y) - f(x)| < \epsilon$.}}
	\step{b}{For all $a \in \mathbb{R}$ we have $\inv{f}((a, +\infty))$ is open.}
	\begin{proof}
		\step{i}{\pflet{$a \in \mathbb{R}$}}
		\step{ii}{\pflet{$x \in \inv{f}((a, +\infty))$}}
		\step{iii}{\pflet{$\epsilon = f(x) - a$}}
		\step{iv}{\pick\ $\delta > 0$ such that, for all $y \in \mathbb{R}$, if $|y-x| < \delta$ then $|f(y) - f(x)| < \epsilon$}
		\step{v}{$x \in (x - \delta, x + \delta) \subseteq \inv{f}((a,+\infty))$}
	\end{proof}
	\step{c}{For all $a \in \mathbb{R}$ we have $\inv{f}((-\infty, a))$ is open.}
	\begin{proof}
		\pf\ Similar.
	\end{proof}
	\qedstep
	\begin{proof}
		\pf\ Proposition \ref{prop:continuous_subbasis}.
	\end{proof}
\end{proof}
\qed
\end{proof}

\begin{df}[Continuity at a Point]
Let $X$ and $Y$ be topological spaces. Let $f : X \rightarrow Y$. Let $a \in X$. Then $f$ is \emph{continuous at $a$} iff, for every neighbourhood $V$ of $f(a)$, there exists a neighbourhood $U$ of $a$ such that $f(U) \subseteq V$.
\end{df}

\begin{prop}
Let $X$ and $Y$ be topological spaces. Let $f : X \rightarrow Y$. Then $f$ is continuous if and only if $f$ is continuous at every point in $X$.
\end{prop}

\begin{proof}
\step{4}{If $f$ is continuous then $f$ is continuous at every point in $X$.}
\begin{proof}
	\step{a}{\assume{$f$ is continuous.}}
	\step{b}{\pflet{$a \in X$}}
	\step{c}{\pflet{$V$ be a neighbourhood of $f(a)$}}
	\step{d}{\pflet{$U = \inv{f}(V)$}}
	\step{e}{$U$ is a neighbourhood of $a$.}
	\step{f}{$f(U) \subseteq V$}
\end{proof}
\step{5}{If $f$ is continuous at every point in $X$ then $f$ is continuous.}
\begin{proof}
	\step{a}{\assume{$f$ is continuous at every point in $X$.}}
	\step{b}{\pflet{$V$ be open in $Y$.}}
	\step{c}{\pflet{$x \in \inv{f}(V)$}}
	\step{d}{$V$ is a neighbourhood of $f(x)$}
	\step{e}{\pick\ a neighbourhood $U$ of $x$ such that $f(U) \subseteq V$}
	\step{f}{$x \in U \subseteq \inv{f}(V)$}
\end{proof}
\qed
\end{proof}

\begin{df}[Homeomorphism]
Let $X$ and $Y$ be topological spaces. A \emph{homeomorphism} between $X$ and $Y$ is a bijection $f : X \approx Y$ such that $f$ and $\inv{f}$ are continuous.
\end{df}

\begin{prop}
Let $X$ and $Y$ be topological spaces. Let $f : X \rightarrow Y$. Then $f$ is a homeomorphism iff $f$ is bijective and, for all $U \subseteq X$, we have $f(U)$ is open if and only if $U$ is open.
\end{prop}

\begin{proof}
\pf\ Immediate from definitions. \qed
\end{proof}

\begin{df}[Topological Property]
A property $P$ of topological spaces is a \emph{topological} property iff, for any topological spaces $X$ and $Y$, if $P[X]$ and $X \cong Y$ then $P[Y]$.
\end{df}

\begin{df}[Retraction]
Let $X$ be a topological space and $A$ a subspace of $X$. A continuous function $\rho : X \rightarrow A$ is a \emph{retraction} iff $\rho \restriction A = \id{A}$. We say $A$ is a \emph{retract} of $X$ iff there exists a retraction.
\end{df}

\begin{df}
Let $\mathbf{Top}$ be the category of small topological spaces and continuous functions.
\end{df}

\begin{prop}
$\emptyset$ is initial in $\mathbf{Top}$.
\end{prop}

\begin{prop}
$1$ is terminal in $\mathbf{Top}$.
\end{prop}

Forgetful functor $\mathbf{Top} \rightarrow \mathbf{Set}$.

Basepoint preserving continuous functor.

\begin{prop}
Let $(X, \mathcal{T})$ be a topological space. Let $S$ be the Sierpi\'{n}ski two-point space. Define $\Phi : \mathcal{T} \rightarrow \Top[X,S]$ by $\Phi(U)(x) = 1$ iff $x \in U$. Then $\Phi$ is a bijection.
\end{prop}

\begin{proof}
\pf
\step{1}{For all $U \in \mathcal{T}$ we have $\Phi(U)$ is continuous.}
\begin{proof}
	\step{a}{\pflet{$U \in \mathcal{T}$}}
	\step{b}{$\Phi(U)(\{1\})$ is open.}
	\begin{proof}
		\pf\ Since $\Phi(U)(\{1\}) = U$.
	\end{proof}
\end{proof}
\step{2}{$\Phi$ is injective.}
\begin{proof}
	\pf\ If $\Phi(U) = \Phi(V)$ then we have $\forall x (x \in U \Leftrightarrow \Phi(U)(x) = 1 \Leftrightarrow \Phi(V)(x) = 1 \Leftrightarrow x \in V)$.
\end{proof}
\step{3}{$\Phi$ is surjective.}
\begin{proof}
	\pf\ Given $f : X \rightarrow S$ continuous we have $\Phi(\inv{f}(1)) = f$.
\end{proof}
\qed
\end{proof}

\subsection{Paths}

\begin{df}[Path]
A \emph{path} in a topological space $X$ is a continuous function $[0,1] \rightarrow X$.
\end{df}

\subsection{Loops}

\begin{df}[Loop]
A \emph{loop} in a topological space $X$ is a path $\alpha : [0,1] \rightarrow X$ such that $\alpha(0) = \alpha(1)$.
\end{df}

\section{Convergence}

\begin{df}[Convergence]
Let $X$ be a topological space. Let $(x_n)$ be a sequence in $X$. A point $a \in X$ is a \emph{limit} of the sequence iff, for every neighbourhood $U$ of $a$, there exists $n_0$ such that $\forall n \geq n_0. x_n \in U$.
\end{df}

\begin{prop}
\label{prop:continuous_converge}
If $f : X \rightarrow Y$ is continuous and $x_n \rightarrow l$ in $X$ then $f(x_n) \rightarrow f(l)$ in $Y$.
\end{prop}

%TODO

\begin{ex}
The converse does not hold.

Let $X$ be the set of all continuous functions $[0,1] \rightarrow [-1,1]$ under the product topology. Let $i : X \rightarrow L^2([0,1])$ be the inclusion.

If $f_n \rightarrow f$ then $i(f_n) \rightarrow i(f)$ --- Lebesgue convergence theorem.

We prove that $i$ is not continuous.

Assume for a contradiction $i$ is continuous. Choose a neighbourhood $K$ of 0 in $X$ such that $\forall \phi \in K _\epsilon. \int \phi^2 < 1/2$. Let $K = \prod_{\lambda \in [0,1]} U_\lambda$ where $U_\lambda = [-1,1]$ except for $\lambda = \lambda_1, \ldots, \lambda_n$. Let $\phi$ be the function that is 0 at $\lambda_1$, \ldots, $\lambda_n$ and 1 everywhere else. Then $\phi \in K$ but $\int \phi^2 = 1$.
\end{ex}

\begin{prop}
The converse does hold for first countable spaces. If $f : X \rightarrow Y$ where $X$ is first countable, and $Y$ is a topological space, and whenever $x_n \rightarrow x$ then $f(x_n) \rightarrow f(x)$, then $f$ is continuous.
\end{prop}

\begin{prop}
If $(s_n)$ is an increasing sequence of real numbers bounded above, then $(s_n)$ converges.
\end{prop}

\begin{proof}
\pf
\step{1}{\pflet{$s$ be the supremum of $\{ s_n : n \in \mathbb{N} \}$.} \prove{$s_n \rightarrow s$ as $n \rightarrow \infty$.}}
\step{2}{\pflet{$\epsilon > 0$}}
\step{3}{\pick\ $N$ such that $s_N > s - \epsilon$.}
\step{4}{$\forall n \geq N. s - \epsilon \leq s_n \leq s$}
\step{5}{$\forall n \geq N. |s_n - s| < \epsilon$}
\qed
\end{proof}

\subsection{Closure}

\begin{prop}
Let $X$ be a topological space. Let $A \subseteq X$. Let $(a_n)$ be a sequence in $A$ and $l \in X$. If $a_n \rightarrow l$ as $n \rightarrow \infty$, then $l \in \overline{A}$.
\end{prop}

\begin{proof}
\pf
\step{1}{\pflet{$U$ be a neighbourhood of $l$.}}
\step{2}{\pick\ $N$ such that $\forall n \in N. a_n \in U$}
\step{3}{$a_N \in A \cap U$}
\qed
\end{proof}

\subsection{Continuous Functions}

\begin{prop}
Let $X$ and $Y$ be topological spaces. Let $f : X \rightarrow Y$ be continuous. Let $x_n \rightarrow x$ as $n \rightarrow \infty$ in $X$. Then $f(x_n) \rightarrow f(x)$ as $n \rightarrow \infty$ in $Y$.
\end{prop}

\begin{proof}
\pf
\step{1}{\pflet{$V$ be a neighbourhood of $f(x)$.}}
\step{2}{\pick\ $N$ such that $\forall n \geq N. x_n \in \inv{f}(V)$}
\step{3}{$\forall n \geq N. f(x_n) \in V$}
\qed
\end{proof}

\subsection{Infinite Series}

\begin{df}[Series]
Let $(a_n)$ be a sequence of real numbers. We say that the infinite series $\sum_{n=0}^\infty a_n$ \emph{converges} to $s$, and write
\[ \sum_{n=0}^\infty a_n = s \]
iff $\sum_{n=0}^N a_n \rightarrow s$ as $N \rightarrow \infty$.
\end{df}

\section{Strong Continuity}

\begin{df}[Strong Continuity]
Let $X$ and $Y$ be topological spaces. Let $f : X \rightarrow Y$. Then $f$ is \emph{strongly continuous} iff, for every $V \subseteq Y$, we have $V$ is open in $Y$ if and only if $\inv{f}(V)$ is open in $X$.
\end{df}

\begin{prop}
Let $X$ and $Y$ be topological spaces. Let $f : X \rightarrow Y$. Then $f$ is strongly continuous if and only if, for all $C \subseteq Y$, we have $C$ is closed in $Y$ if and only if $\inv{f}(C)$ is closed in $X$.
\end{prop}

\begin{proof}
\pf
\begin{align*}
f \text{ is continuous} & \Leftrightarrow \forall V \subseteq Y ( V \text{ is open in } Y \Leftrightarrow \inv{f}(V) \text{ is open in } X) \\
& \Leftrightarrow \forall C \subseteq Y (Y - C \text{ is open in } Y \Leftrightarrow \inv{f}(Y - C) \text{ is open in } X) \\
& \Leftrightarrow \forall C \subseteq Y (C \text{ is closed in } Y \Leftrightarrow \inv{f}(C) \text{ is closed in } X) & \qed
\end{align*}
\end{proof}

\section{Subspaces}

\begin{df}[Subspace]
Let $X$ be a topological space, $Y$ a set, and $f : Y \rightarrow X$. The \emph{subspace topology} on $Y$ induced by $f$ is $\mathcal{T} = \{ \inv{i}(U) : U \text{ is open in } X \}$.

We prove this is a topology.
\end{df}

\begin{proof}
\pf
\step{1}{For all $\mathcal{U} \subseteq \mathcal{T}$ we have $\bigcup \mathcal{U} \in \mathcal{T}$}
\begin{proof}
	\pf\ Since $\bigcup \mathcal{U} = \inv{f}(\bigcup \{ V : \inv{f}(V) \in \mathcal{U}\})$.
\end{proof}
\step{2}{For all $U,V \in \mathcal{T}$ we have $U \cap V \in \mathcal{T}$}
\begin{proof}
	\pf\ Since $\inv{f}(U) \cap \inv{f}(V) = \inv{f}(U 
\cap V)$.
\end{proof}
\step{3}{$Y \in \mathcal{T}$}
\begin{proof}
	\pf\ Since $Y = \inv{f}(X)$.
\end{proof}
\qed
\end{proof}

\begin{prop}
Let $X$ be a topological space, $Y$ a set and $f : Y \rightarrow X$ a function. Then the subspace topology on $Y$ is the coarsest topology such that $f$ is continuous.
\end{prop}

\begin{proof}
\pf\ Immediate from definition. \qed
\end{proof}

\begin{prop}[Local Formulation of Continuity]
Let $X$ and $Y$ be topological spaces. Let $f : X \rightarrow Y$. Let $\mathcal{U}$ be a set of open subspaces of $X$ such that $X = \bigcup \mathcal{U}$. If $f \restriction U : U \rightarrow Y$ is continuous for all $U \in \mathcal{U}$, then $f$ is continuous.
\end{prop}

\begin{proof}
\pf
\step{1}{\pflet{$x \in X$} \prove{$f$ is continuous at $x$.}}
\step{2}{\pflet{$V$ be a neighbourhood of $f(x)$.}}
\step{3}{\pick\ $U \in \mathcal{U}$ such that $x \in U$.}
\step{4}{\pick\ $W$ open in $U$ such that $x \in W$ and $f(W) \subseteq V$.}
\step{5}{$W$ is open in $X$.}
\qed
\end{proof}

\begin{df}[Unit Circle]
The \emph{unit circle} $S^1$ is $\{ (x,y) \in \mathbb{R}^2 : x^2 + y^2 = 1 \}$ as a subspace of $\mathbb{R}^2$.
\end{df}

\begin{ex}
The \emph{unit sphere} $S^2$ is $\{ x \in \mathbb{R}^3 : \| x \| = 1 \}$ as a subspace of $\mathbb{R}^3$.
\end{ex}

\begin{thm}
\label{thm:subspace_universal}
Let $X$ be a topological space and $(Y,i)$ a subset of $X$. Then the subspace topology on $Y$ is the unique topology such that, for every topological space $Z$ and function $f : Z \rightarrow Y$, we have $f$ is continuous if and only if $i \circ f : Z \rightarrow X$ is continuous.
\end{thm}

\begin{proof}
\pf
\step{1}{If we give $Y$ the subspace topology then, for every topological space $Z$ and function $f : Z \rightarrow Y$, we have $f$ is continuous if and only if $i \circ f$ is continuous.}
\begin{proof}
	\step{a}{Given $Y$ the subspace topology.}
	\step{b}{\pflet{$Z$ be a topological space.}}
	\step{c}{\pflet{$f : Z \rightarrow Y$}}
	\step{d}{If $f$ is continuous then $i \circ f$ is continuous.}
	\begin{proof}
		\pf\ Since $i$ is continuous.
	\end{proof}
	\step{e}{If $i \circ f$ is continuous then $f$ is continuous.}
	\begin{proof}
		\step{i}{\assume{$i \circ f$ is continuous.}}
		\step{ii}{\pflet{$U$ be open in $Y$.}}
		\step{iii}{$\inv{f}(\inv{i}(i(U))$ is open in $Z$.}
		\step{iv}{$\inv{f}(U)$ is open in $Z$.}
	\end{proof}
\end{proof}
\step{2}{If, for every topological space $Z$ and function $f : Z \rightarrow Y$, we have $f$ is continuous if and only if $i \circ f$ is continuous.}
\begin{proof}
	\step{a}{\assume{For every topological space $Z$ and function $f : Z \rightarrow Y$, we have $f$ is continuous if and only if $i \circ f$ is continuous.}}
	\step{b}{$i$ is continuous.}
	\step{c}{For every open set $U$ in $X$, we have $\inv{i}(X)$ is open in $Y$}
	\step{b}{\pflet{$Z$ be the set $Y$ under the subspace topology and $f : Z \rightarrow Y$ the identity function.}}
	\step{c}{$i \circ f$ is continuous.}
	\step{d}{$f$ is continuous.}
	\step{e}{Every set open in $Y$ is open in $Z$.}
\end{proof}
\qed
\end{proof}

\begin{prop}
Let $X$ be a topological space, $Y$ a subspace of $X$ and $U \subseteq Y$. If $Y$ is open in $X$ and $U$ is open in $Y$ then $U$ is open in $X$.
\end{prop}

\begin{proof}
\pf
\step{1}{\pick\ $V$ open in $X$ such that $U = V \cap Y$}
\step{2}{$U$ is open in $X$.}
\begin{proof}
	\pf\ It is the intersection of two open sets in $X$.
\end{proof}
\qed
\end{proof}

\begin{prop}
Let $Y$ be a subspace of $X$ and $A \subseteq Y$. Then the subspace topology on $A$ as a subspace of $Y$ is the same as the subspace topology on $A$ as a subspace of $X$.
\end{prop}

\begin{proof}
\pf
\step{1}{\pflet{$\mathcal{T}_Y$ be the subspace topology on $A$ as a subspace of $Y$.}}
\step{2}{\pflet{$\mathcal{T}_X$ be the subspace topology on $A$ as a subspace of $X$.}}
\step{3}{\pflet{$U \subseteq A$}}
\step{4}{$U \in \mathcal{T}_Y \Leftrightarrow U \in \mathcal{T}_X$}
\begin{proof}
	\pf
	\begin{align*}
		U \in \mathcal{T}_Y & \Leftrightarrow \exists V \text{ open in } Y. U = V \cap A \\
		& \Leftrightarrow \exists V. \exists W \text{ open in } X. (V = Y \cap W \wedge U = V \cap A) \\
		& \Leftrightarrow \exists W \text{ open in } X. U = Y \cap W \cap A \\
		& \Leftrightarrow \exists W \text{ open in } X. U = W \cap A \\
		& \Leftrightarrow U \in \mathcal{T}_X
	\end{align*}
\end{proof}
\qed
\end{proof}

\begin{prop}
Let $X$ be a topological space. Let $\mathcal{B}$ be a basis for the topology on $X$. Let $Y \subseteq X$. Then $\mathcal{B}' = \{ B \cap Y : B \in \mathcal{B} \}$ is a basis for the topology on $Y$.
\end{prop}

\begin{proof}
\pf
\step{1}{Every element of $\mathcal{B}'$ is open.}
\begin{proof}
	\pf\ For all $B \in \mathcal{B}$, we have $B$ is open in $X$, so $B \cap Y$ is open in $Y$.
\end{proof}
\step{2}{For any open set $V$ in $Y$ and $y \in V$, there exists $B' \in \mathcal{B}'$ such that $y \in B' \subseteq V$}
\begin{proof}
	\step{a}{\pflet{$V$ be open in $Y$.}}
	\step{b}{\pflet{$y \in V$}}
	\step{c}{\pick\ $U$ open in $X$ such that $V = U \cap Y$.}
	\step{d}{\pick\ $B \in \mathcal{B}$ such that $y \in B \subseteq U$}
	\step{e}{$B \cap Y \in \mathcal{B}'$ and $y \in B \cap Y \subseteq V$}
\end{proof}
\qed
\end{proof}

\subsection{Product Topology}

\begin{prop}
Let $\{X_i\}_{i \in I}$ be a family of topological spaces. Let $Y_i$ be a subspace of $X_i$ for all $i \in I$. Then the product topology on $\prod_{i \in I} Y_i$ is the same as the subspace topology on $\prod_{i \in I} Y_i$ as a subspace of $\prod_{i \in I} X_i$.
\end{prop}

\begin{proof}
\pf
\step{1}{Given $\prod_{i \in I} Y_i$ the subspace topology.}
\step{11}{\pflet{$\iota : \prod_{i \in I} Y_i$ be the inclusion.}}
\step{2}{\pflet{$Z$ be any topological space.}}
\step{3}{\pflet{$f : Z \rightarrow \prod_{i \in I} Y_i$}}
\step{4}{$f$ is continuous if and only if, for all $i \in I$, we have $\pi_i \circ f$ is continuous.}
\begin{proof}
	\pf
	\begin{align*}
		f \text{ is continuous}
		& \Leftrightarrow \iota \circ f : Z \rightarrow \prod_{i \in I} X_i \text{ is continuous} & (\text{Theorem \ref{thm:subspace_universal}}) \\
		& \Leftrightarrow \forall i \in I. \pi_i \circ \iota \circ f : Z \rightarrow X_i \text{ is continuous} & (\text{Theorem \ref{thm:product_universal}}) \\
		& \Leftrightarrow \forall i \in I. \iota_i \circ \pi_i \circ f : Z \rightarrow X_i \text{ is continuous} \\
		& \Leftrightarrow \forall i \in I. \pi_i \circ f : Z \rightarrow Y_i \text{ is continuous}
		& (\text{Theorem \ref{thm:subspace_universal}})
	\end{align*}
	where $\iota_i$ is the inclusion $Y_i \rightarrow X_i$.
\end{proof}
\qed
\end{proof}

\section{Embedding}

\begin{df}[Embedding]
Let $X$ and $Y$ be topological spaces and $f : X \rightarrow Y$. Then $f$ is an \emph{embedding} iff $f$ is injective and the topology on $X$ is the subspace induced by $f$.
\end{df}

\begin{prop}
Every embedding is continuous.
\end{prop}

\begin{proof}
\pf\ Theorem \ref{thm:subspace_universal}. \qed
\end{proof}

\begin{prop}
Let $X$ and $Y$ be topological spaces. Let $b \in Y$. The function $\kappa : X \rightarrow X \times Y$ that maps $x$ to $(x,b)$ is an embedding.
\end{prop}

\begin{proof}
\pf
\step{1}{For all $U$ open in $X$, we have $U = \inv{\kappa}(V)$ for some $V$ open in $X \times Y$.}
\begin{proof}
	\pf\ Take $V = U \times Y$.
\end{proof}
\step{2}{For all $V$ open in $X \times Y$ we have $\inv{\kappa}(V)$ is open in $X$.}
\begin{proof}
	\pf\ Since $\pi_1 \circ \kappa = \id{X}$ and $\pi_2 \circ \kappa$ (which is the constant function with value $b$) are both continuous, hence $\kappa$ is continuous.
\end{proof}
\qed
\end{proof}

\section{Open Maps}

\begin{df}[Open Map]
Let $X$ and $Y$ be topological spaces and $f : X \rightarrow Y$. Then $f$ is an \emph{open map} iff, for all $U$ open in $X$, we have $f(U)$ is open in $Y$.
\end{df}

\begin{prop}
Let $X$ and $Y$ be topological spaces. The projections $\pi_1 : X \times Y \rightarrow X$ and $\pi_2 : X \times Y \rightarrow Y$ are open maps.
\end{prop}

\begin{proof}
\pf
\step{1}{$\pi_1$ is an open map.}
\begin{proof}
	\step{a}{\pflet{$U$ be open in $X \times Y$.}}
	\step{b}{\pflet{$x \in \pi_1(U)$}}
	\step{c}{\pick\ $y$ such that $(x,y) \in U$}
	\step{d}{\pick\ $V$ and $W$ open in $X$ and $Y$ respectively such that $(x,y) \in V \times W \subseteq U$}
	\step{e}{$x \in V \subseteq \pi_1(U)$}
\end{proof}
\step{2}{$\pi_2$ is an open map.}
\begin{proof}
	\pf\ Similar.
\end{proof}
\qed
\end{proof}

\section{Locally Finite}

\begin{df}[Locally Finite]
Let $X$ be a topological space. Let $\{A_i\}_{i \in I}$ be a family of subsets of $X$. Then $\{A_i\}_{i \in I}$ is \emph{locally finite} iff, for every $x \in X$, there exist only finitely many $i \in I$ such that $x \in A_i$.
\end{df}

\begin{thm}[Pasting Lemma]
Let $X$ and $Y$ be topological spaces. Let $f : X \rightarrow Y$. Let $\{A_i\}_{i \in I}$ be a locally finite family of closed subspaces of $X$ such that $X = \bigcup_{i \in I} A_i$. If $f \restriction A_i : A_i \rightarrow Y$ is continuous for all $i \in I$, then $f$ is continuous.
\end{thm}

\begin{proof}
\pf
\step{1}{\pflet{$B$ be closed in $Y$.}}
\step{2}{\pflet{$A = \inv{f}(B)$} \prove{$A$ is closed in $X$.}}
\step{3}{$A = \bigcup_{i \in I} \inv{f \restriction A_i}(B)$}
\step{4}{\pflet{$x \in X - A$} \prove{There exists a neighbourhood $U'$ of $x$ such that $U' \subseteq X - A$.}}
\step{5}{\pick\ a neighbourhood $U$ of $x$ such that $U$ intersects $A_i$ for only finitely many $i \in I$.}
\step{6}{\pflet{$i_1$, \ldots, $i_n$ be the elements of $I$ such that $U$ intersects $A_{i_1}$, \ldots, $A_{i_n}$.}}
\step{7}{For $j = 1, \ldots, n$, \pflet{$S_j = \inv{f \restriction A_{i_j}}(B)$}}
\step{8}{For $j = 1, \ldots, n$, we have $S_j$ is closed in $X$.}
\step{9}{For $j = 1, \ldots, n$, we have $x \notin S_j$.}
\step{10}{\pflet{$U' = U \cap \bigcap_{j=1}^n (X - S_j)$}}
\step{11}{$U'$ is a neighbourhood of $x$.}
\step{12}{$U' \subseteq X - A$}
\qed
\end{proof}

\section{Closed Maps}

\begin{df}[Closed Map]
Let $X$ and $Y$ be topological spaces. Let $f : X \rightarrow Y$. Then $f$ is a \emph{closed map} iff, for every closed set $C$ in $X$, we have $f(C)$ is closed in $Y$.
\end{df}

\section{Product Topology}

\begin{df}[Product Topology]
Let $\{ X_\lambda \}_{\lambda \in \Lambda}$ be a family of topological spaces. The \emph{product topology} on $\prod_{\lambda \in \Lambda} X_\lambda$ is the coarsest topology such that every projection onto $X_\lambda$ is continuous.
\end{df}

\subsection{Closed Sets}

\begin{prop}
\label{prop:closed_product}
Let $X$ and $Y$ be topological spaces. Let $A$ be a closed set in $X$ and $B$ a closed set in $Y$. Then $A \times B$ is closed in $X \times Y$.
\end{prop}

\begin{proof}
\pf\ Since $(X \times Y) - (A \times B) = ((X - A) \times Y) \cup (X \times (Y - B))$. \qed
\end{proof}

\begin{prop}
\label{prop:product_basis}
Let $\{ X_\alpha \}_{\alpha \in A}$ be a family of topological spaces. The product topology on $\prod_{\alpha \in A} X_\alpha$ is the topology generated by the basis $\mathcal{B} = \{ \prod_{\alpha \in A} U_\alpha : \text{for all } \alpha \in A, U_\alpha \text{ is open in } X_\alpha \text{ and } U_\alpha = X_\alpha \text{ for all but finitely many } \alpha \in A \}$.
\end{prop}

\begin{proof}
\pf
\step{1}{$\mathcal{B}$ is a basis for a topology.}
\step{2}{\pflet{$\mathcal{T}$ be the topology generated by $\mathcal{B}$.}}
\step{3}{\pflet{$\mathcal{T}_p$ be the product topology.}}
\step{4}{$\mathcal{T} \subseteq \mathcal{T}_p$}
\begin{proof}
	\step{a}{\pflet{$B \in \mathcal{B}$}}
	\step{b}{\pflet{$B = \prod_{\alpha \in A} U_\alpha$ with each $U_\alpha$ open in $X_\alpha$ and $U_\alpha = X_\alpha$ except for $\alpha = \alpha_1, \ldots, \alpha_n$}}
	\step{c}{$B = \inv{\pi_{\alpha_1}}(U_{\alpha_1}) \cap \cdots \cap \inv{\pi_{\alpha_n}}(U_{\alpha_n})$}
	\step{d}{$B \in \mathcal{T}_p$}
\end{proof}
\step{5}{$\mathcal{T}_p \subseteq \mathcal{T}$}
\begin{proof}
	\step{a}{For every $\alpha \in A$ we have $\pi_\alpha$ is continuous.}
	\begin{proof}
		\pf\ Since $\inv{\pi}(U)$ is open for every $U$ open in $X_\alpha$.
	\end{proof}
\end{proof}
\qed
\end{proof}

\begin{thm}
\label{thm:product_universal}
Let $\{ X_\alpha \}_{\alpha \in A}$ be a family of topological spaces. Then the product topology on $\prod_{\alpha \in A} X_\alpha$ is the unique topology such that, for every topological space $Z$ and function $f : Z \rightarrow \prod_{\alpha \in A} X_\alpha$, we have $f$ is continuous if and only if, for all $\alpha \in A$, we have $\pi_\alpha \circ f : Z \rightarrow X_\alpha$ is continuous.
\end{thm}

\begin{proof}
\pf
\step{1}{If we give $\prod_{\alpha \in A} X_\alpha$ the product topology, then for every topological space $Z$ and function $f : Z \rightarrow \prod_{\alpha \in A} X_\alpha$, we have $f$ is continuous if and only if, for all $\alpha \in A$, we have $\pi_\alpha \circ f$ is continuous.}
\begin{proof}
	\step{a}{Give $\prod_{\alpha \in A} X_\alpha$ the product topology.}
	\step{b}{\pflet{$Z$ be a topological space.}}
	\step{c}{\pflet{$f : Z \rightarrow \prod_{\alpha \in A} X_\alpha$}}
	\step{d}{If $f$ is continuous then, for all $\alpha \in A$, we have $\pi_\alpha \circ f$ is continuous.}
	\begin{proof}
		\pf\ Since the composite of two continuous functions is continuous.
	\end{proof}
	\step{e}{If, for all $\alpha \in A$, we have $\pi_\alpha \circ f$ is continuous, then $f$ is continuous.}
	\begin{proof}
		\step{i}{\assume{For all $\alpha \in A$ we have $\pi_\alpha \circ f$ is continuous.}}
		\step{ii}{\pflet{$\{ U_\alpha \}_{\alpha \in A}$ be a family with $U_\alpha$ open in $X_\alpha$ such that $U_\alpha = X_\alpha$ for all $\alpha$ except $\alpha = \alpha_1, \ldots, \alpha_n$.}}
		\step{iii}{For all $\alpha$ we have $\inv{f}(\inv{\pi_\alpha}(U_\alpha))$ is open in $Z$.}
		\step{iv}{$\inv{f}(\prod_\alpha U_\alpha)$ is open in $Z$}
		\begin{proof}
			\pf\ Since $\inv{f}(\prod_\alpha U_\alpha) = \inv{f}(\inv{\pi_{\alpha_1}}(U_{\alpha_1})) \cap \cdots \cap \inv{f}(\inv{\pi_{\alpha_n}}(U_{\alpha_n}))$.
		\end{proof}
	\end{proof}
\end{proof}
\step{2}{If $\mathcal{T}$ is a topology on $\prod_{\alpha \in A} X_\alpha$ such that, for every topological pace $Z$ and function $f : Z \rightarrow \prod_{\alpha \in A} X_\alpha$, we have $f$ is continuous if and only if, for all $\alpha \in A$, we have $\pi_\alpha \circ f$ is continuous, then $\mathcal{T}$ is the product topology.}
\begin{proof}
	\step{a}{\assume{$\mathcal{T}$ is a topology on $\prod_{\alpha \in A} X_\alpha$ such that, for every topological pace $Z$ and function $f : Z \rightarrow \prod_{\alpha \in A} X_\alpha$, we have $f$ is continuous if and only if, for all $\alpha \in A$, we have $\pi_\alpha \circ f$ is continuous.}}
	\step{b}{\pflet{$\mathcal{T}_p$ be the product topology.}}
	\step{c}{$\mathcal{T} \subseteq \mathcal{T}_p$}
	\begin{proof}
		\step{i}{\pflet{$Z = (\prod_\alpha X_\alpha, \mathcal{T}_p)$}}
		\step{ii}{\pflet{$f : Z \rightarrow \prod_\alpha X_\alpha$ be the identity function}}
		\step{iii}{For all $\alpha$ we have $\pi_\alpha \circ f$ is continuous.}
		\step{iv}{$f$ is continuous.}
		\begin{proof}
			\pf\ \stepref{a}
		\end{proof}
		\step{v}{Every set open in $\mathcal{T}$ is open in $\mathcal{T}_p$}
	\end{proof}
	\step{d}{$\mathcal{T}_p \subseteq \mathcal{T}$}
	\begin{proof}
		\step{i}{$\id{\prod_\alpha X_\alpha}$ is continuous.}
		\step{ii}{For all $\alpha$ we have $\pi_\alpha$ is continuous.}
		\begin{proof}
			\pf\ \stepref{a}
		\end{proof}
		\step{iii}{$\mathcal{T}_p \subseteq \mathcal{T}$}
		\begin{proof}
			\pf\ Since $\mathcal{T}_p$ is the coarsest topology such that every $\pi_\alpha$ is continuous.
		\end{proof}
	\end{proof}
\end{proof}
\qed
\end{proof}

\begin{ex}
It is not true that, for any function $f : \prod_{\alpha \in A} X_\alpha \rightarrow Y$, if $f$ is continuous in every variable separately then $f$ is continuous.

Define $f : \mathbb{R}^2 \rightarrow \mathbb{R}$ by
\[ f(x,y) = \begin{cases}
\frac{xy}{x^2 + y^2} & \text{if } (x,y) \neq (0,0) \\
0 & \text{if } x = y = 0
\end{cases} \]
Then $f$ is continuous in $x$ and in $y$, but is not continuous.
\end{ex}

\begin{prop}
Let $\{X_i\}_{i \in I}$ be a nonempty family of topological spaces. The product topology on $\prod_{i \in I}$ is the topology generated by the subbasis $\{ \inv{\pi_i}(U) : i \in I, U \text{ is open in } X_i \}$.
\end{prop}

\begin{proof}
\pf
\step{1}{$\{ \inv{\pi_i}(U) : i \in I, U \text{ is open in } X_i \}$ is a subbasis for a topology on $\prod_{i \in I} X_i$.}
\begin{proof}
	\step{a}{\pick\ $i_0 \in I$}
	\step{b}{$\prod_{i \in I} X_i = \inv{\pi_{i_0}}(X_{i_0})$}
\end{proof}
\step{2}{The topology generated by this subbasis is the product topology.}
\begin{proof}
	\pf\ Since the basis in Proposition \ref{prop:product_basis} is the set of all finite intersections of elements of this subbasis.
\end{proof}
\qed
\end{proof}

\subsection{Closure}

\begin{prop}
Let $\{X_i\}_{i \in I}$ be a family of topological spaces. Let $A_i \subseteq X_i$ for all $i \in I$. Then
\[ \prod_{i \in I} \overline{A_i} = \overline{\prod_{i \in I} A_i} \enspace . \]
\end{prop}

\begin{proof}
\pf
\step{1}{$\prod_{i \in I} \overline{A_i} \subseteq \overline{\prod_{i \in I} A_i}$}
\begin{proof}
	\step{a}{\pflet{$x \in \prod_{i \in I} \overline{A_i}$}}
	\step{b}{For any family $\{U_i\}_{i \in I}$ where each $U_i$ is open in $X_i$, and $U_i = X_i$ for all but finitely many $i \in I$, if $x \in \prod_{i \in I} U_i$ then $\prod_{i \in I} U_i$ intersects $\prod_{i \in I} A_i$.}
	\begin{proof}
		\step{i}{\pflet{$\{U_i\}_{i \in I}$ be a family where each $U_i$ is open in $X_i$, and $U_i = X_i$ for all but finitely many $i$.}}
		\step{ii}{\assume{$x \in \prod_{i \in I}$}}
		\step{iii}{For all $i \in I$ we have $U_i$ intersects $A_i$}
		\begin{proof}
			\pf\ Since $\pi_i(x) \in \overline{A_i}$ and $U_i$ is a neighbourhood of $\pi_i(x)$.
		\end{proof}
		\step{iv}{$\prod_{i \in I} U_i$ intersects $\prod_{i \in I} A_i$}
\end{proof}		
	\step{c}{$x \in \overline{\prod_{i \in I} A_i}$}
	\begin{proof}
		\pf\ Proposition \ref{prop:closure_basis}.
	\end{proof}
\end{proof}
\step{2}{$\overline{\prod_{i \in I} A_i} \subseteq \prod_{i \in I} \overline{A_i}$}
\begin{proof}
	\step{a}{\pflet{$x \in \overline{\prod_{i \in I} A_i}$}}
	\step{b}{\pflet{$i \in I$} \prove{$\pi_i(x) \in \overline{A_i}$}}
	\step{c}{\pflet{$U$ be a neighbourhood of $\pi_i(x)$ in $X_i$}}
	\step{d}{$\inv{\pi_i}(U)$ is a neighbourhood of $x$ in $\prod_{i \in I} X_i$}
	\step{e}{\pick\ $y \in \inv{\pi_i}(U) \cap \prod_{i \in I} A_i$}
	\step{f}{$\pi_i(y) \in U \cap A_i$}
\end{proof}
\qed
\end{proof}

\subsection{Convergence}

\begin{prop}
Let $\{X_i\}_{i \in I}$ be a family of topological spaces. Let $(x_n)$ be a sequence of points in $\prod_{i \in I} X_i$ and $l \in \prod_{i \in I} X_i$. Then $x_n \rightarrow l$ as $n \rightarrow \infty$ if and only if, for all $i \in I$, we have $\pi_i(x_n) \rightarrow \pi_i(l)$ as $n \rightarrow \infty$.
\end{prop}

\begin{proof}
\pf
\step{1}{If $x_n \rightarrow l$ as $n \rightarrow \infty$ then, for all $i \in I$, we have $\pi_i(x_n) \rightarrow \pi_i(l)$ as $n \rightarrow \infty$.}
\begin{proof}
	\pf\ Proposition \ref{prop:continuous_converge}.
\end{proof}
\step{2}{If, for all $i \in I$, we have $\pi_i(x_n) \rightarrow \pi_i(l)$ as $n \rightarrow \infty$, then $x_n \rightarrow l$ as $n \rightarrow \infty$.}
\begin{proof}
	\step{a}{\assume{For all $i \in I$ we have $\pi_i(x_n) \rightarrow \pi_i(l)$ as $n \rightarrow \infty$.}}
	\step{b}{\pflet{$U$ be a neighbourhood of $l$.}}
	\step{c}{\pick\ $i_1, \ldots, i_n \in I$ and open sets $U_j$ in $X_{i_j}$ for $j = 1, \ldots, n$ such that $l \in \inv{\pi_{i_1}}(U_1) \cap \cdots \cap \inv{\pi_{i_n}}(U_n) \subseteq U$}
	\step{d}{For $j=1, \ldots, n$ we have $\pi_{i_j}(l) \in U_j$}
	\step{e}{\pick\ $N$ such that, for all $m \geq N$, we have $\pi_{i_j}(x_m) \in U_j$}
	\step{f}{$\forall m \geq N. x_m \in U$}
\end{proof}
\qed
\end{proof}

\section{Quotient Spaces}

\begin{df}[Quotient Topology]
Let $X$ be a topological space, $S$ a set, and $\pi : X \twoheadrightarrow S$ be a surjection. The \emph{quotient topology} on $S$ induced by $\pi$ is $\mathcal{T} = \{ U \in \mathcal{P} S : \inv{\pi}(U) \text{ is open in } X \}$.

We prove this is a topology.
\end{df}

\begin{proof}
\pf
\step{1}{For all $\mathcal{U} \subseteq \mathcal{T}$ we have $\bigcup \mathcal{U} \in \mathcal{T}$.}
\begin{proof}
	\pf\ Since $\inv{\pi}(\bigcup \mathcal{U}) = \bigcup \{ \inv{\pi}(U) : U \in \mathcal{U} \}$.
\end{proof}
\step{2}{For all $U,V \in \mathcal{T}$ we have $U \cap V \in \mathcal{T}$.}
\begin{proof}
	\pf\ Since $\inv{\pi}(U \cap V) = \inv{\pi}(U) \cap \inv{\pi}(V)$.
\end{proof}
\step{3}{$X \in \mathcal{T}$}
\begin{proof}
	\pf\ Since $X = \inv{\pi}(Y)$.
\end{proof}
\qed
\end{proof}

\begin{prop}
Let $X$ be a topological space, $S$ a set and $\pi : X \twoheadrightarrow S$ a surjection. Then the quotient topology on $S$ is the finest topology such that $\pi$ is continuous.
\end{prop}

\begin{proof}
\pf\ Immediate from definitions. \qed
\end{proof}

\begin{thm}
Let $X$ be a topological space, let $S$ be a set, and let $\pi : X \twoheadrightarrow S$ be surjective. Then the quotient topology on $S$ is the unique topology such that, for every topological space $Z$ and function $f : S \rightarrow Z$, we have $f$ is continuous if and only if $f \circ \pi$ is continuous.
\end{thm}

\begin{proof}
\pf
\step{1}{If $S$ is given the quotient topology, then for every topological space $Z$ and function $f : S \rightarrow Z$, we have $f$ is continuous if and only if $f \circ \pi$ is continuous.}
\begin{proof}
	\step{a}{Give $S$ the quotient topology.}
	\step{b}{\pflet{$Z$ be a topological space.}}
	\step{c}{\pflet{$f : S \rightarrow Z$}}
	\step{d}{If $f$ is continuous then $f \circ \pi$ is continuous.}
	\begin{proof}
		\pf\ The composite of two continuous functions is continuous.
	\end{proof}
	\step{e}{If $f \circ \pi$ is continuous then $f$ is continuous.}
	\begin{proof}
		\step{i}{\assume{$f \circ \pi$ is continuous.}}
		\step{ii}{\pflet{$U$ be open in $Z$.}}
		\step{iii}{$\inv{\pi}(\inv{f}(U))$ is open in $X$.}
		\step{iv}{$\inv{f}(U)$ is open in $S$.}
	\end{proof}
\end{proof}
\step{2}{If $S$ is given a topology such that, for every topological space $Z$ and function $f : S \rightarrow Z$, we have $f$ is continuous if and only if $f \circ \pi$ is continuous, then that topology is the quotient topology.}
\begin{proof}
	\step{a}{Give $S$ a topology such that, for every topological space $Z$ and function $f : S \rightarrow Z$, we have $f$ is continuous if and only if $f \circ \pi$ is continuous.}
	\step{b}{\pflet{$U \subseteq S$}}
	\step{c}{If $\inv{\pi}(U)$ is open in $X$ then $U$ is open in $S$.}
	\begin{proof}
		\step{i}{\pflet{$Z$ be $S$ under the quotient topology induced by $\pi$.}}
		\step{ii}{\pflet{$f : S \rightarrow Z$ be the identity function.}}
		\step{iii}{$f \circ \pi$ is continuous.}
		\step{iv}{$f$ is continuous.}
		\begin{proof}
			\pf\ \stepref{a}
		\end{proof}
		\step{v}{$U$ is open in $Z$.}
		\step{vi}{$U$ is open in $X$.}
	\end{proof}
	\step{d}{If $U$ is open in $S$ then $\inv{\pi}(U)$ is open in $X$.}
	\begin{proof}
		\pf\ Since $\pi$ is continuous (taking $Z = S$ and $f = \id{S}$ in \stepref{a}).
	\end{proof}
\end{proof}
\qed
\end{proof}

\begin{cor}
Let $\pi : X \twoheadrightarrow S$ be a quotient map. Let $Z$ be a topological space. Let $f : X \rightarrow Z$ be continuous. Then there exists a continuous map $g : S \rightarrow Z$ such that $f = g \circ \pi$ if and only if, for all $s \in S$, we have $f$ is constant on $\inv{\pi}(s)$.
\end{cor}

\subsection{Quotient Maps}

\begin{df}[Quotient Map]
Let $X$ and $S$ be topological spaces and $\pi : X \rightarrow S$. Then $\pi$ is a \emph{quotient map} iff $\pi$ is surjective and the topology on $S$ is the quotient topology induced by $\pi$.
\end{df}

\begin{prop}
Let $X$ and $Y$ be topological spaces. Let $f : X \rightarrow Y$. Then $f$ is a quotient map if and only if $f$ is surjective and strongly continuous.
\end{prop}

\begin{proof}
\pf\ Immediate from definition. \qed
\end{proof}

\begin{prop}
Let $X$ and $Y$ be topological spaces. Let $p : X \twoheadrightarrow Y$ be surjective. Then the following are equivalent.
\begin{enumerate}
\item $p$ is a quotient map.
\item $p$ is continuous and maps saturated open sets to open sets.
\item $p$ is continuous and maps saturated closed sets to closed sets.
\end{enumerate}
\end{prop}

\begin{proof}
\pf
\step{1}{$1 \Rightarrow 2$}
\begin{proof}
	\step{a}{\assume{$p$ is a quotient map.}}
	\step{b}{$p$ is continuous.}
	\step{c}{$p$ maps saturated open sets to open sets.}
	\begin{proof}
		\step{i}{\pflet{$U \subseteq X$ be a saturated open set.}}
		\step{2}{$\inv{p}(p(U)) = U$}
		\step{3}{$\inv{p}(p(U))$ is open in $X$.}
		\step{4}{$p(U)$ is open in $Y$.}
	\end{proof}
\end{proof}
\step{2}{$2 \Rightarrow 3$}
\begin{proof}
	\step{a}{\assume{$p$ is continuous and maps saturated open sets to open sets.}}
	\step{b}{\pflet{$C$ be a saturated closed set in $X$.}}
	\step{c}{$X - C$ is a saturated open set.}
	\step{d}{$Y - p(C)$ is open.}
	\step{e}{$p(C)$ is closed.}
\end{proof}
\step{3}{$3 \Rightarrow 1$}
\begin{proof}
	\step{a}{\assume{$p$ is continuous and maps closed sets to closed sets.}}
	\step{b}{\pflet{$C \subseteq Y$}}
	\step{c}{\assume{$\inv{p}(C)$ is closed in $X$.} \prove{$C$ is closed in $Y$.}}
	\step{d}{$\inv{p}(C)$ is saturated.}
	\step{e}{$p(\inv{p}(C))$ is closed.}
	\step{f}{$C$ is closed.}
\end{proof}
\qed
\end{proof}

\begin{cor}
Let $X$ and $Y$ be topological spaces. Let $p : X \rightarrow Y$ be continuous and surjective. If $p$ is either an open map or a closed map, then $p$ is a quotient map.
\end{cor}

\begin{ex}
The converse does not hold.

Let $A = \{ (x,y) \in \mathbb{R}^2 : x \geq 0 \vee y = 0 \}$. Then the first projection $\pi_1 : A \rightarrow \mathbb{R}$ is a quotient map that is neither an open map nor a closed map.
\end{ex}

\begin{proof}
\pf
\step{1}{$\pi_1$ is a quotient map.}
\begin{proof}
	\step{a}{\pflet{$U \subseteq \mathbb{R}$}}
	\step{b}{If $U$ is open then $\inv{\pi_1}(U)$ is open.}
	\begin{proof}
		\pf\ Since $\inv{\pi_1}(U) = (U \times \mathbb{R}) \cap A$.
	\end{proof}
	\step{c}{If $\inv{\pi_1}(U)$ is open then $U$ is open.}
	\begin{proof}
		\step{i}{\assume{$\inv{\pi_1}(U)$ is open.}}
		\step{ii}{\pflet{$x \in U$}}
		\step{iii}{$(x,0) \in \inv{\pi_1}(U)$}
		\step{iv}{\pick\ open neighbourhoods $V$ of $x$ and $W$ of $0$ such that $V \times W \subseteq \inv{\pi_1}(U)$}
		\step{v}{$V \subseteq U$}
		\begin{proof}
			\pf\ For all $x' \in V$ we have $(x',0) \in V \times W \subseteq \inv{\pi_1}(U)$.
		\end{proof}
	\end{proof}
\end{proof}
\step{2}{$\pi_1$ is not an open map.}
\begin{proof}
	\pf\ $\pi_1(((-1,1) \times (1,2)) \cap A) = [0,1)$ which is not open in $\mathbb{R}$.
\end{proof}
\step{3}{$\pi_1$ is not a closed map.}
\begin{proof}
	\pf\ $\pi_1(\{(x,1/x) \in \mathbb{R}^2 : x > 0 \}) = (0, + \infty)$ is not closed in $\mathbb{R}$.
\end{proof}
\qed
\end{proof}

\begin{prop}
Let $Z$ be a topological space. Define $\pi : [0,1] \rightarrow S^1$ by $\pi(t) = (\cos 2 \pi t, \sin 2 \pi t)$. Given any continuous function $f : S^1 \rightarrow Z$, we have $f \circ \pi$ is a loop in $Z$. This defines a bijection between $\Top[S^1,Z]$ and the set of loops in $Z$.
\end{prop}

\begin{proof}
\pf\ Since $\pi$ is a quotient map. \qed
\end{proof}

\begin{df}[Projective Space]
The \emph{projective space} $\mathbb{RP}^n$ is the quotient of $\mathbb{R}^{n+1} - \{0\}$ by $\sim$ where $x \sim \lambda x$ for all $x \in \mathbb{R}^{n+1} - \{0\}$ and $\lambda \in \mathbb{R}$.
\end{df}

\begin{df}[Torus]
The \emph{torus} $T$ is the quotient of $[0,1]^2$ by $\sim$ where $(x,0) \sim (x,1)$ and $(0,y) \sim (1,y)$.
\end{df}

\begin{df}[M\"{o}bius Band]
The \emph{M\"{o}bius band} is the quotient of $[0,1]^2$ by $\sim$ where $(0,y) \sim (1,1-y)$.
\end{df}

\begin{df}[Klein Bottle]
The \emph{Klein bottle} is the quotient of $[0,1]^2$ by $\sim$ where $(x,0) \sim (x,1)$ and $(0,y) \sim (1,1-y)$.
\end{df}

\begin{prop}
$\mathbb{RP}^2$ is the quotient of $[0,1]^2$ by $\sim$ where $(x,0) \sim (1-x,1)$ and $(0,y) \sim (1,1-y)$.
\end{prop}

\begin{proof}
\pf TODO
\end{proof}

\begin{ex}
Let $\{X_i\}_{i \in I}$ be a family of topological spaces and $\{Y_i\}_{i \in I}$ a family of sets. Let $q_i : X_i \twoheadrightarrow Y_i$ be a surjective function for all $i \in I$. Give each $Y_i$ the quotient topology. It is not true in general that the product topology on $\prod_{i \in I} Y_i$ is the same as the quotient topology induced by $\prod_{i \in I} q_i : \prod_{i \in I} X_i \twoheadrightarrow \prod_{i \in I} Y_i$.
\end{ex}

\begin{proof}
\pf
\step{1}{\pflet{$X^* = \mathbb{R} - \mathbb{Z}_+ + \{b\}$ be the quotient space obtained from $\mathbb{R}$ by identifying the subset $\mathbb{Z}_+$ to the point $b$.}}
\step{2}{\pflet{$p : \mathbb{R} \rightarrow X^*$ be the quotient map.} \prove{$p \times \id{\mathbb{Q}} : \mathbb{R} \times \mathbb{Q} \rightarrow X^* \times \mathbb{Q}$ is not a quotient map.}}
\step{3}{For $n \in \mathbb{Z}_+$, \pflet{$c_n = \sqrt{2} / n$}}
\step{4}{For $n \in \mathbb{Z}_+$, \pflet{$U_n = \{ (x,y) \in \mathbb{Q} \times \mathbb{R} : n-1/4 < x < n+1/4 \text{ and } ((y > x + c_n - n \text{ and } y > -x + c_n + n) \text{ or } (y < x + c_n - n \text{ and } y < -x + c_n + n))\}$}}
\step{5}{For all $n \in \mathbb{Z}_+$, $U_n$ is open in $\mathbb{R} \times \mathbb{Q}$}
\step{6}{For all $n \in \mathbb{Z}_+$ we have $\{n\} \times \mathbb{Q} \subseteq U_n$}
\step{7}{\pflet{$U = \bigcup_{n \in \mathbb{Z}_+} U_n$}}
\step{8}{$U$ is open in $\mathbb{R} \times \mathbb{Q}$.}
\step{9}{$U$ is saturated with respect to $p \times \id{\mathbb{Q}}$.}
\step{10}{\pflet{$U' = (p \times \id{\mathbb{Q}})(U)$}}
\step{11}{\assume{for a contradiction $U'$ is open in $X^* \times \mathbb{Q}$.}}
%TODO
\end{proof}

\section{Box Topology}

\begin{df}[Box Topology]
Let $\{ X_i \}_{i \in I}$ be a family of topological spaces. The \emph{box topology} on $X = \prod_{i \in I} X_i$ is the topology generated by the basis $\mathcal{B} = \{ \prod_{i \in I} U_i: \{U_i\}_{i \in I} \text{ is a family with each } U_i \text{ an open set in } X_i \}$.

We prove this is a basis for a topology.
\end{df}

\begin{proof}
\pf
\step{1}{$\bigcup \mathcal{B} = X$}
\begin{proof}
	\pf\ Since $\prod_{i \in I} X_i \in \mathcal{B}$.
\end{proof}
\step{2}{For all $B_1,B_2 \in \mathcal{B}$ and $x \in B_1 \cap B_2$, there exists $B_3 \in \mathcal{B}$ such that $x \in B_3 \subseteq B_1 \cap B_2$.}
\begin{proof}
	\step{a}{\pflet{$B_1, B_2 \in \mathcal{B}$}}
	\step{b}{\pflet{$x \in B_1 \cap B_2$}}
	\step{c}{\pick\ a family $\{U_i\}_{i \in I}$ such that $B_1 = \prod_{i \in I} U_i$.}
	\step{d}{\pick\ a family $\{V_i\}_{i \in I}$ such that $B_2 = \prod_{i \in I} V_i$.}
	\step{e}{\pflet{$B_3 = \prod_{i \in I} (U_i \cap V_i)$}}
	\step{d}{$x \in B_3 \subseteq B_1 \cap B_2$}
\end{proof}
\qed
\end{proof}

\begin{prop}
The box topology is finer than the product topology.
\end{prop}

\begin{proof}
\pf\ Immediate from definitions. \qed
\end{proof}

\begin{prop}
On a finite family of topological spaces, the box topology and the product topology are the same.
\end{prop}

\begin{proof}
\pf\ Immediate from definitions. \qed
\end{proof}

\begin{prop}
The box topology is strictly finer than the product topology on the Hilbert cube.
\end{prop}

\begin{proof}
\pf\ The set $\prod_{n=0}^\infty (0, 1/(n+1)^2)$ is open in the box topology but not in the product topology. \qed
\end{proof}

\subsection{Bases}

\begin{prop}
Let $\{X_i\}_{i \in I}$ be a family of topological spaces. For all $i \in I$, let $\mathcal{B}_i$ be a basis for the topology on $X_i$. Then $\mathcal{B} = \left\{ \prod_{i \in I} B_i : \forall i \in I. B_i \in \mathcal{B}_i \right\}$ is a basis for the box topology on $\prod_{i \in I} X_i$.
\end{prop}

\begin{proof}
\pf
\step{1}{For every family $\{B_i\}_{i \in I}$ where $\forall i \in I. B_i \in \mathcal{B}_i$, we have $\prod_{i \in I} B_i$ is open in the box topology.}
\begin{proof}
	\pf\ Since each $B_i$ is open in $X_i$.
\end{proof}
\step{2}{For any open set $U$ in the box topology and $x \in U$, there exists $B \in \mathcal{B}$ such that $x \in B \subseteq U$.}
\begin{proof}
	\step{a}{\pflet{$U$ be a set open in the box topology.}}
	\step{b}{\pflet{$x \in U$}}
	\step{c}{\pick\ a family $\{U_i\}_{i \in I}$ where each $U_i$ is open in $X_i$ such that $x \in \prod_{i \in I} U_i \subseteq U$}
	\step{d}{For $i \in I$, choose $B_i \in \mathcal{B}_i$ such that $x_i \in B_i \subseteq U_i$.}
	\step{e}{$\prod_{i \in I} B_i \in \mathcal{B}$}
	\step{f}{$x \in \prod_{i \in I} B_i \subseteq \prod_{i \in I} U_i \subseteq U$}
\end{proof}
\qed
\end{proof}

\subsection{Subspaces}

\begin{prop}
Let $\{X_i\}_{i \in I}$ be a family of topological spaces. Let $Y_i$ be a subspace of $X_i$ for all $i \in I$. Then the box topology on $\prod_{i \in I} Y_i$ is the same as the subspace topology that $\prod_{i \in I} Y_i$ inherits as a subspace of $\prod_{i \in I} X_i$ under the box topology.
\end{prop}

\begin{proof}
\pf\ A basis for the box topology is
\begin{align*}
& \{ \prod_{i \in I} V_i : V_i \text{ open in } Y_i \} \\
= & \{ \prod_{i \in I} (U_i \cap Y_i) : U_i \text{ open in } X_i \} \\
& = \{ \prod_{i \in I} U_i \cap \prod_{i \in I} Y_i : U_i \text{ open in } X_i \}
\end{align*}
which is a basis for the subspace topology by Proposition \ref{prop:basis_subspace}. \qed
\end{proof}

\subsection{Closure}

\begin{prop}
Let $\{X_i\}_{i \in I}$ be a family of topological spaces. Give $\prod_{i \in I} X_i$ the box topology. Let $A_i \subseteq X_i$ for all $i \in I$. Then
\[ \prod_{i \in I} \overline{A_i} = \overline{\prod_{i \in I} A_i} \enspace . \]
\end{prop}

\begin{proof}
\pf
\step{1}{$\prod_{i \in I} \overline{A_i} \subseteq \overline{\prod_{i \in I} A_i}$}
\begin{proof}
	\step{a}{\pflet{$x \in \prod_{i \in I} \overline{A_i}$}}
	\step{b}{For any family $\{U_i\}_{i \in I}$ where each $U_i$ is open in $X_i$, if $x \in \prod_{i \in I} U_i$ then $\prod_{i \in I} U_i$ intersects $\prod_{i \in I} A_i$.}
	\begin{proof}
		\step{i}{\pflet{$\{U_i\}_{i \in I}$ be a family where each $U_i$ is open in $X_i$.}}
		\step{ii}{\assume{$x \in \prod_{i \in I}$}}
		\step{iii}{For all $i \in I$ we have $U_i$ intersects $A_i$}
		\begin{proof}
			\pf\ Since $\pi_i(x) \in \overline{A_i}$ and $U_i$ is a neighbourhood of $\pi_i(x)$.
		\end{proof}
		\step{iv}{$\prod_{i \in I} U_i$ intersects $\prod_{i \in I} A_i$}
\end{proof}		
	\step{c}{$x \in \overline{\prod_{i \in I} A_i}$}
	\begin{proof}
		\pf\ Proposition \ref{prop:closure_basis}.
	\end{proof}
\end{proof}
\step{2}{$\overline{\prod_{i \in I} A_i} \subseteq \prod_{i \in I} \overline{A_i}$}
\begin{proof}
	\step{a}{\pflet{$x \in \overline{\prod_{i \in I} A_i}$}}
	\step{b}{\pflet{$i \in I$} \prove{$\pi_i(x) \in \overline{A_i}$}}
	\step{c}{\pflet{$U$ be a neighbourhood of $\pi_i(x)$ in $X_i$}}
	\step{d}{$\inv{\pi_i}(U)$ is a neighbourhood of $x$ in $\prod_{i \in I} X_i$}
	\step{e}{\pick\ $y \in \inv{\pi_i}(U) \cap \prod_{i \in I} A_i$}
	\step{f}{$\pi_i(y) \in U \cap A_i$}
\end{proof}
\qed
\end{proof}

\section{Connected Spaces}

\begin{df}[Connected]
A topological space is \emph{connected} iff it is not the union of two nonempty open disjoint subsets.
\end{df}

\begin{prop}
The continuous image of a connected space is connected.
\end{prop}

\begin{prop}
Let $X$ be a topological space and $A,B \subseteq X$. If $X = A \cup B$, $A \cap B \neq \emptyset$, and $A$ and $B$ are connected, then $X$ is connected.
\end{prop}

\begin{prop}
If $X$ and $Y$ are nonempty topological spaces, then
$X \times Y$ is connected if and only if $X$ and $Y$ are connected.
\end{prop}

\begin{df}[Path-connected]
A topological space $X$ is \emph{path-connected} iff, for any points $a,b \in X$, there exists a continuous function $\alpha : [0,1] \rightarrow X$, called a \emph{path}, such that $\alpha(0) = a$ and $\alpha(1) = b$.
\end{df}

\begin{prop}
The continuous image of a path connected space is path connected.
\end{prop}

\begin{prop}
Let $X$ be a topological space and $A,B \subseteq X$. If $X = A \cup B$, $A \cap B \neq \emptyset$, and $A$ and $B$ are path connected, then $X$ is path connected.
\end{prop}

\begin{prop}
If $X$ and $Y$ are nonempty topological spaces, then
$X \times Y$ is path connected if and only if $X$ and $Y$ are path connected.
\end{prop}

\section{$T_1$ Spaces}

\begin{df}[$T_1$]
A topological space is $T_1$ iff every one-point set is closed.
\end{df}

\begin{prop}
A topological space is $T_1$ iff every finite set is closed.
\end{prop}

\begin{proof}
\pf\ Since the union of finitely many closed sets is closed. \qed
\end{proof}

\begin{prop}
Let $X$ be a topological space. Then $X$ is $T_1$ if and only if, for all $x,y \in X$, if $x \neq y$ then there exists a neighbourhood of $x$ that does not contain $y$, and there exists a neighbourhood of $y$ that does not contain $x$.
\end{prop}

\begin{proof}
\pf
\step{1}{If $X$ is $T_1$ then, for all $x,y \in X$, if $x \neq y$ then there exists a neighbourhood of $x$ that does not contain $y$, and there exists a neighbourhood of $y$ that does not contain $x$.}
\begin{proof}
	\step{a}{\assume{$X$ is $T_1$.}}
	\step{b}{\pflet{$x,y \in X$}}
	\step{c}{\assume{$x \neq y$}}
	\step{d}{$X - \{y\}$ is a neighbourhood of $x$ that does not contain $y$.}
	\step{e}{$X - \{x\}$ is a neighbourhood of $y$ that does not contain $x$.}
\end{proof}
\step{2}{If, for all $x,y \in X$, if $x \neq y$ then there exists a neighbourhood of $x$ that does not contain $y$, and there exists a neighbourhood of $y$ that does not contain $x$, then $X$ is $T_1$.}
\begin{proof}
	\step{a}{\assume{For all $x,y \in X$, if $x \neq y$ then there exists a neighbourhood of $x$ that does not contain $y$, and there exists a neighbourhood of $y$ that does not contain $x$.}}
	\step{b}{\pflet{$x \in X$} \prove{$\{x\}$ is closed.}}
	\step{c}{\pflet{$y \in X - \{x\}$}}
	\step{d}{\pick\ a neighbourhood $U$ of $y$ that does not contain $x$.}
	\step{e}{$y \in U \subseteq X - \{x\}$}
\end{proof}
\qed
\end{proof}

\subsection{Limit Points}

\begin{prop}
Let $X$ be a $T_1$ space. Let $A \subseteq X$ and $l \in X$. Then $l$ is a limit point of $A$ if and only if every neighbourhood of $l$ contains infinitely many points of $A$.
\end{prop}

\begin{proof}
\pf
\step{1}{If $l$ is a limit point of $A$ then every neighbourhood of $l$ contains infinitely many points of $A$.}
\begin{proof}
	\step{a}{\assume{$l$ is a limit point of $A$.}}
	\step{b}{\pflet{$U$ be a neighbourhood of $l$.}}
	\step{c}{\assume{for a contradiction $U \cap A - \{l\}$ is finite.}}
	\step{d}{$U \cap A - \{l\}$ is closed.}
	\begin{proof}
		\pf\ Since $X$ is $T_1$.
	\end{proof}
	\step{d}{$U - (A - \{l\})$ is a neighbourhood of $l$.}
	\step{e}{$U - (A - \{l\})$ intersects $A$.}
	\qedstep
\end{proof}
\step{2}{If every neighbourhood of $l$ contains infinitely many points of $A$ then $l$ is a limit point of $A$.}
\begin{proof}
	\pf\ Immediate from definitions.
\end{proof}
\qed
\end{proof}

\section{Hausdorff Spaces}

\begin{df}[Hausdorff]
A topological space is a \emph{Hausdorff} space or a \emph{$T_2$} space iff any two distinct points have disjoint neighbourhoods.
\end{df}

\begin{prop}
In a Hausdorff space, a sequence has at most one limit.
\end{prop}

\begin{proof}
\pf
\step{1}{\pflet{$X$ be a Hausdorff space.}}
\step{2}{\pflet{$(a_n)$ be a sequence in $X$ and $l,m \in X$}}
\step{3}{\assume{$a_n \rightarrow l$ and $a_n \rightarrow m$}}
\step{4}{\assume{for a contradiction $l \neq m$}}
\step{5}{\pick\ disjoint open sets $U$ and $V$ with $l \in U$ and $m \in V$}
\step{6}{\pick\ $M$, $N$ such that $\forall n \geq M. a_n \in U$ and $\forall n \geq N. a_n \in V$}
\step{7}{$a_{\max(M,N)} \in U \cap V$}
\qedstep
\begin{proof}
	\pf\ This contradicts the fact that $U \cap V = \emptyset$.
\end{proof}
\qed
\end{proof}

\begin{ex}
We cannot weaken the hypothesis from being Hausdorff to being $T_1$.

In the cofinite topology on any infinite set, every sequence converges to every point.
\end{ex}

\begin{prop}
Any linearly ordered set is Hausdorff under the order topology.
\end{prop}

\begin{proof}
\pf
\step{1}{\pflet{$X$ be a linearly ordered set under the order topology.}}
\step{2}{\pflet{$a,b \in X$ with $a \neq b$.}}
\step{3}{\assume{w.l.o.g. $a < b$.}}
\step{4}{\case{There exists $c \in X$ such that $a < c < b$.}}
\begin{proof}
	\step{i}{\pflet{$U = (-\infty, c)$}}
	\step{ii}{\pflet{$V = (c, + \infty)$}}
	\step{iii}{$U$ and $V$ are disjoint open sets with $a \in U$ and $b \in V$}
\end{proof}
\step{5}{\case{There is no $c \in X$ such that $a < c < b$.}}
\begin{proof}
	\step{i}{\pflet{$U = (-\infty, b)$}}
	\step{ii}{\pflet{$V = (a, + \infty)$}}
	\step{iii}{$U$ and $V$ are disjoint open sets with $a \in U$ and $b \in V$}
\end{proof}
\qed
\end{proof}

\begin{prop}
A subspace of a Hausdorff space is Hausdorff.
\end{prop}

\begin{proof}
\pf
\step{1}{\pflet{$X$ be a Hausdorff space.}}
\step{2}{\pflet{$Y$ be a subspace of $X$.}}
\step{3}{\pflet{$a,b \in Y$ with $a \neq b$.}}
\step{4}{\pick\ disjoint open sets $U$ and $V$ in $X$ with $a \in U$ and $b \in V$.}
\step{5}{$U \cap Y$ and $V \cap Y$ are disjoint open sets in $Y$ with $a \in U \cap Y$ and $b \in V \cap Y$.}
\qed
\end{proof}

\begin{prop}
The disjoint union of two Hausdorff spaces is Hausdorff.
\end{prop}

\begin{prop}
Let $A$ be a topological space and $B$ a Hausdorff space. Let $f,g : A \rightarrow B$ be continuous. Let $X \subseteq A$ be dense. If $f$ and $g$ agree on $X$, then $f = g$.
\end{prop}

\begin{proof}
\pf
\step{1}{\assume{for a contradiction $a \in A$ and $f(a) \neq g(a)$.}}
\step{2}{\pick\ disjoint neighbourhoods $U$ and $V$ of $f(a)$ and $g(a)$ respectively.}
\step{3}{\pick\ $x \in \inv{f}(U) \cap \inv{g}(V)$}
\step{4}{$f(x) = g(x) \in U \cap V$}
\qedstep
\begin{proof}
\pf\ This is a contradiction.
\end{proof}
\qed
\end{proof}

\subsection{Product Topology}

\begin{prop}
The product of a family of Hausdorff spaces is Hausdorff.
\end{prop}

\begin{proof}
\pf
\step{1}{\pflet{$\{X_i\}_{i \in I}$ be a family of Hausdorff spaces.}}
\step{2}{\pflet{$x,y \in \prod_{i \in I} X_i$ with $x \neq y$.}}
\step{3}{\pick\ $i \in I$ such that $\pi_i(x) \neq \pi_i(y)$}
\step{4}{\pick\ disjoint open sets $U$ and $V$ in $X_i$ such that $\pi_i(x) \in U$ and $\pi_i(y) \in V$.}
\step{5}{$x \in \inv{\pi_i}(U)$ and $y \in \inv{\pi_i}(V)$.}
\qed
\end{proof}

\subsection{Box Topology}

\begin{prop}
The box product of a family of Hausdorff spaces is Hausdorff.
\end{prop}

\begin{proof}
\pf
\step{1}{\pflet{$\{X_i\}_{i \in I}$ be a family of Hausdorff spaces.}}
\step{2}{\pflet{$x,y \in \prod_{i \in I} X_i$ with $x \neq y$.}}
\step{3}{\pick\ $i \in I$ such that $\pi_i(x) \neq \pi_i(y)$}
\step{4}{\pick\ disjoint open sets $U$ and $V$ in $X_i$ such that $\pi_i(x) \in U$ and $\pi_i(y) \in V$.}
\step{5}{$x \in \inv{\pi_i}(U)$ and $y \in \inv{\pi_i}(V)$.}
\qed
\end{proof}

\subsection{$T_1$ Spaces}

\begin{prop}
Every Hausdorff space is $T_1$.
\end{prop}

\begin{proof}
\pf
\step{1}{\pflet{$X$ be a Hausdorff space.}}
\step{2}{\pflet{$a \in X$} \prove{$X - \{a\}$ is open.}}
\step{3}{\pflet{$x \in X - \{a\}$}}
\step{4}{\pick\ disjoint open sets $U$ and $V$ with $a \in U$ and $x \in V$}
\step{5}{$x \in V \subseteq X - U \subseteq X - \{a\}$}
\qed
\end{proof}

\begin{ex}
The converse does not hold. If $X$ is an infinite set under the cofinite topology, then $X$ is $T_1$ but not Hausdorff.
\end{ex}

\begin{prop}
Let $X$ and $Y$ be metric spaces. Let $f : X \rightarrow Y$ be uniformly continuous. Let $\hat{X}$ and $\hat{Y}$ be the completions of $X$ and $Y$. Then $f$ extends uniquely to a continuous map $\hat{X} \rightarrow \hat{Y}$.
\end{prop}

\begin{proof}
\pf\ The extension maps $\lim_{n \rightarrow \infty} x_n$ to $\lim_{n \rightarrow \infty} f(x_n)$. \qed
\end{proof}

\begin{prop}
Let $X$ be a topological space. Then $X$ is Hausdorff if and only if the diagonal $\Delta = \{ (x,x) : x \in X \}$ is closed in $X^2$.
\end{prop}

\begin{proof}
\pf
\begin{align*}
& \Delta \text{ is closed} \\
\Leftrightarrow & X^2 - \Delta \text{ is open} \\
\Leftrightarrow & \forall x,y \in X ((x,y) \notin \Delta \Rightarrow \exists V,W \text{ open in } X (x \in V \wedge y \in W \wedge V \times W \subseteq X^2 - \Delta)) \\
\Leftrightarrow & \forall x,y \in X (x \neq y \Rightarrow \exists V,W \text{ open in } X (x \in V \wedge y \in W \wedge V \cap W = \emptyset)) \\
\Leftrightarrow & X \text{ is Hausdorff} & \qed
\end{align*}
\end{proof}

\section{Separable Spaces}

\begin{df}[Separable]
A topological space is \emph{separable} iff it has a countable dense subset.
\end{df}

Every second countable space is separable.

\section{Sequential Compactness}

\begin{df}[Sequentially Compact]
A topological space is \emph{sequentially compact} iff every sequence has a convergent subsequence.
\end{df}

\section{Compactness}

\begin{df}[Compact]
A topological space is \emph{compact} iff every open cover has a finite subcover.
\end{df}

\begin{prop}
Let $X$ be a compact topological space. Let $P$ be a set of open sets such that, for all $U,V \in P$, we have $U \cup V \in P$. Assume that every point has an open neighbourhood in $P$. Then $X \in P$.
\end{prop}

\begin{proof}
\pf
\step{1}{$P$ is an open cover of $X$}
\step{2}{\pick\ a finite subcover $U_1, \ldots, U_n \in P$}
\step{3}{$X = U_1 \cup \cdots \cup U_n \in P$}
\qed
\end{proof}

\begin{cor}
Let $f$ be a compact space and $f : X \rightarrow \mathbb{R}$ be locally bounded. Then $f$ is bounded.
\end{cor}

\begin{proof}
\pf\ Take $P = \{ U \text{ open in } X : f \text{ is bounded on } U \}$. \qed
\end{proof}

\begin{prop}
The continuous image of a compact space is compact.
\end{prop}

\begin{prop}
A closed subspace of a compact space is compact.
\end{prop}

\begin{prop}
Let $X$ and $Y$ be nonempty spaces. Then the following are equivalent.
\begin{enumerate}
\item $X$ and $Y$ are compact.
\item $X + Y$ is compact.
\item $X \times Y$ is compact.
\end{enumerate}
\end{prop}

\begin{prop}
A compact subspace of a Hausdorff space is closed.
\end{prop}

\begin{prop}
A continuous bijection from a compact space to a Hausdorff space is a homeomorphism.
\end{prop}

\begin{prop}
A first countable compact space is sequentially compact.
\end{prop}


\section{Quotient Spaces}

\begin{df}[Quotient Space]
Let $X$ be a topological space and $\sim$ an equivalence relation on $X$. The \emph{quotient topology} on $X / \sim$ is defined by: $U \in \mathcal{P} X$ is open in $X / \sim$ if and only if $\inv{\pi}(U)$ is open in $X$.
\end{df}

\begin{prop}
\label{prop:map_from_quotient_continuous}
Let $X$ and $Y$ be topological spaces. Let $\sim$ be an equivalence relation on $X$. Let $f : X / \sim \rightarrow Y$. Then $f$ is continuous if and only if $f \circ \pi$ is continuous.
\end{prop}

\begin{prop}
Let $X$ and $Y$ be topological spaces. Let $\sim$ be an equivalence relation on $X$. Let $\phi : Y \rightarrow X / \sim$.

Assume that, for all $y \in Y$, there exists a neighbourhood $U$ of $y$ and a continuous function $\Phi : U \rightarrow X$ such that $\pi \circ \Phi = \phi \restriction U$. Then $\phi$ is continuous.
\end{prop}

\begin{prop}
A quotient of a connected space is connected.
\end{prop}

\begin{prop}
A quotient of a path connected space is path connected.
\end{prop}

\begin{prop}
Let $X$ be a topological space and $\sim$ an equivalence relation on $X$. If $X / \sim$ is Hausdorff then every equivalence class of $\sim$ is closed in $X$.
\end{prop}

\begin{df}
Let $X$ be a topological space and $A_1, \ldots, A_r \subseteq X$. Then $X / A_1, \ldots, A_r$ is the quotient space of $X$ with respect to $\sim$ where $x \sim y$ iff $x = y$ or $\exists i (x \in A_i \wedge y \in A_i)$.
\end{df}

\begin{df}[Cone]
Let $X$ be a topological space. The \emph{cone over $X$} is the space $(X \times [0,1]) / (X \times \{1\})$.
\end{df}

\begin{df}[Suspension]
Let $X$ be a topological space. The \emph{suspension} of $X$ is the space
\[ \Sigma X := (X \times [-1,1]) / (X \times \{-1\}),(X \times \{1\}) \]
\end{df}

\begin{df}[Wedge Product]
Let $x_0 \in X$ and $y_0 \in Y$. The \emph{wedge product} $X \vee Y$ is $(X \times \{y_0\}) \cup (\{x_0\} \times Y)$ as a subspace of $X \times Y$.
\end{df}

\begin{df}[Smash Product]
Let $x_0 \in X$ and $y_0 \in Y$. The \emph{smash product} $X \wedge Y$ is $(X \times Y) / (X \vee Y)$.
\end{df}

\begin{ex}
$D^n / S^{n-1} \cong S^n$
\end{ex}

\begin{proof}
\pf
\step{1}{\pflet{$\phi : D^n / S^{n-1} \rightarrow S^n$ be the function induced by the map $D^n \rightarrow S^n$ that maps the radii of $D^n$ onto the meridians of $S^n$ from the north to the south pole.}}
\step{2}{$\phi$ is a bijection.}
\step{3}{$\phi$ is a homeomorphism.}
\begin{proof}
	\pf\ Since $D^n / S^{n-1}$ is compact and $S^n$ is Hausdorff.
\end{proof}
\qed
\end{proof}

\section{Gluing}

\begin{df}[Gluing]
Let $X$ and $Y$ be topological spaces, $X_0 \subseteq X$ and $\phi : X_0 \rightarrow Y$ a continuous map. Then $Y \cup_\phi X$ is the quotient space $(X + Y)/ \sim$, where $\sim$ is the equivalence relation generated by $x \sim \phi(x)$ for all $x \in X$.
\end{df}

\begin{prop}
$Y$ is a subspace of $Y \cup_\phi X$.
\end{prop}

\begin{df}
Let $X$ be a topological space and $\alpha : X \cong X$ a homeomorphism. Then $(X \times [0,1]) / \alpha$ is the quotient space of $X \times [0,1]$ by the equivalence relation generated by $(x,0) \sim (\alpha(x),1)$ for all $x \in X$.
\end{df}

\begin{df}[M\"{o}bius Strip]
The \emph{M\"{o}bius strip} is $([-1,1] \times [0,1])/ \alpha$ where $\alpha(x) = -x$.
\end{df}

\begin{df}[Klein Bottle]
The \emph{Klein bottle} is $(S^1 \times [0,1]) / \alpha$ where $\alpha(z) = \overline{z}$.
\end{df}

\begin{prop}
Let $M$ be the M\"{o}bius strip and $K$ the Klein bottle. Then $M \cup_{\id{\partial M}} M \cong K$.
\end{prop}

\begin{proof}
\pf
\step{1}{\pflet{$f : ([-1,1] \times [0,1]) + ([-1,1] \times [0,1]) \rightarrow S^1 \times [0,1]$ be the function that maps $\kappa_1(\theta,t)$ to $(e^{\pi i \theta / 2}, t)$ and $\kappa_2(\theta,t)$ to $(-e^{- \pi i \theta / 2}, t)$.}}
\step{2}{$f$ induces a bijection $M \cup_{\id{\partial M}} M \approx K$}
\step{3}{$f$ is a homeomorphism.}
\qed
\end{proof}

\chapter{Metric Spaces}

%TODO Define real numbers
\begin{df}[Metric Space]
Let $X$ be a set and $d : X^2 \rightarrow \mathbb{R}$. We say $(X,d)$ is a \emph{metric space} iff:
\begin{itemize}
\item For all $x,y \in X$ we have $d(x,y) \geq 0$
\item For all $x,y \in X$ we have $d(x,y) = 0$ iff $x = y$
\item For all $x,y \in X$ we have $d(x,y) = d(y,x)$
\item (\emph{Triangle Inequality}) For all $x,y,z \in X$ we have $d(x,z) \leq d(x,y) + d(y,z)$
\end{itemize}
We call $d$ the \emph{metric} of the metric space $(X,d)$. We often write $X$ for the metric space $(X,d)$.
\end{df}

\begin{df}[Discrete Metric]
On any set $X$, define the \emph{discrete} metric by $d(x,y) = 0$ if $x = y$, 1 if $x \neq y$.
\end{df}

\begin{df}[Standard Metric]
The \emph{standard metric} on $\mathbb{R}$ is defined by $d(x,y) = |x-y|$.
\end{df}

\begin{df}[Square Metric]
The \emph{square metric} $\rho$ on $\mathbb{R}^n$ is defined by
\[ \rho(\vec{x}, \vec{y}) = \max(|x_1 - y_1|, \ldots, |x_n - y_n|) \enspace . \]

We prove this is a metric.
\end{df}

\begin{proof}
\pf
\step{1}{For all $\vec{x}, \vec{y} \in \mathbb{R}^n$ we have $\rho(\vec{x}, \vec{y}) \geq 0$.}
\begin{proof}
	\pf\ Immediate from definition.
\end{proof}
\step{2}{For all $\vec{x}, \vec{y} \in \mathbb{R}^n$ we have $\rho(\vec{x}, \vec{y}) = 0$ iff $\vec{x} = \vec{y}$.}
\begin{proof}
	\pf
	\begin{align*}
		\rho(\vec{x}, \vec{y}) = 0 & \Leftrightarrow \max(|x_1 - y_1|, \ldots, |x_n - y_n|) = 0 \\
		& \Leftrightarrow |x_1 - y_1| = \cdots = |x_n - y_n| = 0 \\
		& \Leftrightarrow x_1 = y_1 \wedge \cdots \wedge x_n = y_n \\
		& \Leftrightarrow \vec{x} = \vec{y}
	\end{align*}
\end{proof}
\step{3}{For all $\vec{x}, \vec{y} \in \mathbb{R}^n$ we have $\rho(\vec{x}, \vec{y}) = \rho(\vec{y}, \vec{x})$.}
\begin{proof}
	\pf\ Immediate from definition.
\end{proof}
\step{4}{For all $\vec{x}, \vec{y}, \vec{z} \in \mathbb{R}^n$ we have $\rho(\vec{x}, \vec{z}) \leq \rho(\vec{x}, \vec{y}) + \rho(\vec{y}, \vec{z})$.}
\begin{proof}
	\pf
	\begin{align*}
		& max(|x_1 - z_1|, \ldots, |x_n - z_n|) \\
		\leq & \max(|x_1 - y_1| + |y_1 - z_1|, \ldots, |x_n - y_n| + |y_n - z_n|) \\
		\leq & \max(|x_1 - y_1|, \ldots, |x_n - y_n|) + \max(|y_1 - z_1|, \ldots, |y_n - z_n|) \\
		= & \rho(\vec{x}, \vec{y}) + \rho(\vec{y}, \vec{z})
	\end{align*}
\end{proof}
\qed
\end{proof}

\subsection{Balls}

\begin{df}[Ball]
Let $X$ be a metric space. Let $x \in X$ and $r > 0$. The \emph{ball} with \emph{centre} $x$ and \emph{radius} $r$ is
\[ B(x,r) = \{ y \in X \mid d(x,y) < r \} \enspace . \]
\end{df}

\begin{df}[Metric Topology]
Let $(X,d)$ be a metric space. The \emph{metric topology} on $X$ is the topology generated by the basis consisting of the balls.

We prove this is a basis for a topology.
\end{df}

\begin{proof}
\pf
\step{1}{Every point is a member of some ball.}
\begin{proof}
	\pf\ Since $x \in B(x,1)$.
\end{proof}
\step{2}{If $B_1$ and $B_2$ are balls and $x \in B_1 \cap B_2$, then there exists a ball $B_3$ such that $x \in B_3 \subseteq B_1 \cap B_2$.}
\begin{proof}
	\step{a}{\pflet{$x \in B(a,\epsilon_1) \cap B(b,\epsilon_2)$}}
	\step{b}{\pflet{$\epsilon = \min(\epsilon_1 - d(x,a), \epsilon_2 - d(x,b))$} \prove{$x \in B(x, \epsilon) \subseteq B(a,\epsilon_1) \cap B(b,\epsilon_2)$}}
	\step{c}{$B(x,\epsilon) \subseteq B(a, \epsilon_1)$}
	\begin{proof}
		\step{i}{\pflet{$y \in B(x, \epsilon)$}}
		\step{ii}{$d(y,a) < \epsilon_1$}
		\begin{proof}
			\pf
			\begin{align*}
				d(y,a) & \leq d(y,x) + d(x,a) & (\text{Triangle Inequality}) \\
				& < \epsilon + d(x,a) & (\text{\stepref{i}}) \\
				& \leq \epsilon_1 & (\text{\stepref{b}})
			\end{align*}
		\end{proof}
	\end{proof}
	\step{d}{$B(x,\epsilon) \subseteq B(b, \epsilon_2)$}
	\begin{proof}
		\pf\ Similar.
	\end{proof}
\end{proof}
\qed
\end{proof}

\begin{prop}
The discrete metric on a set $X$ induces the discrete topology.
\end{prop}

\begin{proof}
\pf\ Since $B(x, 1/2) = \{x\}$ for all $x \in X$. \qed
\end{proof}

\begin{prop}
The standard metric on $\mathbb{R}$ induces the standard topology.
\end{prop}

\begin{proof}
\pf
\step{1}{Every ball is open in the standard topology.}
\begin{proof}
	\pf\ Since $B(a, \epsilon) = (a - \epsilon, a + \epsilon)$.
\end{proof}
\step{2}{Every open ray is open in the metric topology.}
\begin{proof}
	\pf\ If $x \in (a, +\infty)$ then $x \in B(x,x-a) \subseteq (a, + \infty)$. Similarly for $(-\infty, a)$.
\end{proof}
\qed
\end{proof}

\begin{prop}
The square metric on $\mathbb{R}^n$ induces the product topology.
\end{prop}

\begin{proof}
\pf
\step{1}{For any real numbers $a_1$, \ldots, $a_n$, $b_1$, \ldots, $b_n$ with $a_1 < b_1$, \ldots, $a_n < b_n$, we have $(a_1,b_1) \times \cdots \times (a_n,b_n)$ is open in the metric topology.}
\begin{proof}
	\step{a}{\pflet{$\vec{x} \in (a_1,b_1) \times \cdots \times (a_n,b_n)$}}
	\step{b}{\pflet{$\epsilon = \min(x_1 - a_1, b_1 - x_1, \ldots, x_n - a_n, b_n - x_n)$}}
	\step{c}{$B(\vec{x},\epsilon) \subseteq (a_1,b_1) \times \cdots \times (a_n,b_n)$}
\end{proof}
\step{2}{For any $\vec{a} \in \mathbb{R}^n$ and $\epsilon > 0$, we have $B(\vec{a}, \epsilon)$ is open in the product topology.}
\begin{proof}
	\pf\ Since $B(\vec{a}, \epsilon) = (a_1 - \epsilon, a_1 + \epsilon) \times \cdots \times (a_n - \epsilon, a_n + \epsilon)$.
\end{proof}
\qed
\end{proof}

\begin{prop}
Addition is a continuous function $\mathbb{R}^2 \rightarrow \mathbb{R}$.
\end{prop}

\begin{proof}
\pf
\step{1}{\pflet{$(x,y) \in \mathbb{R}^2$ and $\epsilon > 0$}}
\step{2}{\pflet{$\delta = \epsilon / 2$}}
\step{3}{\pflet{$(x',y') \in \mathbb{R}^2$ with $\rho((x,y),(x',y')) < \delta$}}
\step{4}{$|x-x'|,|y-y'| < \delta$}
\step{5}{$|(x+y)-(x'+y')| < \epsilon$}
\begin{proof}
	\pf
	\begin{align*}
		|(x+y)-(x'+y')| & \leq |x-x'| + |y-y'| \\
		& < \delta + \delta & (\text{\stepref{4}}) \\
		& = \epsilon & (\text{\stepref{2}})
	\end{align*}
\end{proof}
\qed
\end{proof}

\begin{prop}
Multiplication is a continuous function $\mathbb{R}^2 \rightarrow \mathbb{R}$.
\end{prop}

\begin{proof}
\pf
\step{1}{\pflet{$(x,y) \in \mathbb{R}^2$ and $\epsilon > 0$}}
\step{2}{\pflet{$\delta = \min(\epsilon / (|x| + |y| + 1), 1)$}}
\step{3}{\pflet{$(x',y') \in \mathbb{R}^2$ with $\rho((x,y),(x',y')) < \delta$}}
\step{4}{$|x-x'|,|y-y'| < \delta$}
\step{5}{$|xy-x'y'| < \epsilon$}
\begin{proof}
	\pf
	\begin{align*}
		|xy - x'y'| & = |xy - xy' + xy - x'y - xy + x'y + xy' - x'y'| \\
		& \leq |xy-xy'| + |xy-x'y| + |xy-x'y-xy'+xy'y|
		& = |x||y-y'| + |x-x'||y| + |x-x'||y-y'| \\
		& < |x|\delta + |y|\delta + \delta^2 & (\text{\stepref{4}}) \\
		& \leq |x|\delta + |y|\delta + \delta & (\text{\stepref{2}}) \\
		& = (|x| + |y| + 1) \delta \\
		& \leq \epsilon & (\text{\stepref{2}})
	\end{align*}
\end{proof}
\qed
\end{proof}

\begin{cor}
The unit circle $S^1$ is a closed subset of $\mathbb{R}^2$.
\end{cor}

\begin{proof}
\pf\ The function $f$ that maps $(x,y)$ to $x^2 + y^2$ is continuous, and $S^1 = \inv{f}(\{1\})$. \qed
\end{proof}

\begin{cor}
The unit ball $B^2$ is a closed subset of $\mathbb{R}^2$.
\end{cor}

\begin{proof}
\pf\ The function $f$ that maps $(x,y)$ to $x^2 + y^2$ is continuous, and $B^2 = \inv{f}([0,1])$. \qed
\end{proof}

\begin{prop}
Let $(a_n)$ and $(b_n)$ be sequences of real numbers. Let $c,s,t \in \mathbb{R}$. Assume
\[ \sum_{n=0}^\infty a_n = s \text{ and } \sum_{n=0}^\infty b_n = t \enspace . \]
Then
\[ \sum_{n=0}^\infty (c a_n + b_n) = cs + t \enspace . \]
\end{prop}

\begin{proof}
\pf
\[ \sum_{n=0}^N (c a_n + b_n) = c \sum_{n=0}^N a_n + \sum_{n=0}^N b_n \rightarrow cs + t \text{ as } n \rightarrow \infty \qquad \qed \]
\end{proof}

\begin{prop}[Comparison Test]
Let $(a_n)$ and $(b_n)$ be sequences of real numbers. Assume $|a_n| \leq b_n$ for all $n$. Assume $\sum_{n=0}^\infty b_n$ converges. Then $\sum_{n=0}^\infty a_n$ converges.
\end{prop}

\begin{proof}
\pf
\step{1}{For all $n$, \pflet{$c_n = |a_n| + a_n$}}
\step{2}{$\sum_{n=0}^\infty |a_n|$ converges.}
\begin{proof}
	\pf\ Since $(\sum_{n=0}^N |a_n|)_N$ is an increasing sequence of real numbers bounded above by $\sum_{n=0}^\infty b_n$.
\end{proof}
\step{3}{$\sum_{n=0}^\infty c_n$ converges.}
\begin{proof}
	\pf\ Since $(\sum_{n=0}^N c_n)_N$ is an increasing sequence of real numbers bounded above by $2 \sum_{n=0}^\infty a_n$.
\end{proof}
\step{4}{$\sum_{n=0}^\infty a_n$ converges.}
\begin{proof}
	\pf\ Since $a_n = c_n - |a_n|$.
\end{proof}
\qed
\end{proof}

\begin{prop}
\label{prop:metric_open}
Let $X$ be a metric space. Let $U \subseteq X$. Then $U$ is open if and only if, for all $x \in U$, there exists $\epsilon > 0$ such that $B(x, \epsilon) \subseteq U$.
\end{prop}

\begin{proof}
\pf
\step{1}{If $U$ is open then, for all $x \in U$, there exists $\epsilon > 0$ such that $B(x, \epsilon) \subseteq U$.}
\begin{proof}
	\step{a}{\assume{$U$ is open.}}
	\step{b}{\pflet{$x \in U$}}
	\step{c}{\pick\ a ball $B(a, \delta)$ such that $x \in B(a, \delta) \subseteq U$}
	\step{d}{\pflet{$\epsilon = \delta - d(a,x)$} \prove{$B(x, \epsilon) \subseteq U$}}
	\step{e}{\pflet{$y \in B(x, \epsilon)$}}
	\step{f}{$y \in B(a, \delta)$}
	\begin{proof}
		\pf
		\begin{align*}
			d(a,y) & \leq d(a,x) + d(x,y) & (\text{Triangle Inequality}) \\
			& < d(a,x) + \epsilon & (\text{\stepref{e}}) \\
			& = \delta
		\end{align*}
	\end{proof}
	\step{g}{$y \in U$}
	\begin{proof}
		\pf\ \stepref{c}
	\end{proof}
\end{proof}
\step{2}{If, for all $x \in U$, there exists $\epsilon > 0$ such that $B(x, \epsilon) \subseteq U$, then $U$ is open.}
\begin{proof}
	\pf\ Immediate from definition of the metric topology.
\end{proof}
\qed
\end{proof}

\begin{prop}
\label{prop:distance_between_distances}
Let $X$ be a metric space. Let $a,b,c \in X$. Then
\[ |d(a,b) - d(a,c)| \leq d(b,c) \enspace . \]
\end{prop}

\begin{proof}
\pf
\step{1}{$d(a,b) - d(a,c) \leq d(b,c)$}
\begin{proof}
	\pf\ Triangle Inequality.
\end{proof}
\step{2}{$d(a,c) - d(a,b) \leq d(b,c)$}
\begin{proof}
	\pf\ Triangle Inequality.
\end{proof}
\qed
\end{proof}

\begin{prop}
Let $(X,d)$ be a metric space. Then the metric topology on $X$ is the coarsest topology such that $d : X^2 \rightarrow \mathbb{R}$ is continuous.
\end{prop}

\begin{proof}
\pf
\step{1}{$d$ is continuous with respect to the metric topology.}
\begin{proof}
	\step{1}{\pflet{$(a,b) \in X^2$}}
	\step{2}{\pflet{$V$ be a neighbourhood of $d(a,b)$.}}
	\step{3}{\pick\ $\epsilon > 0$ such that $(d(a,b) - \epsilon, d(a,b) + \epsilon) \subseteq V$.}
	\step{4}{\pflet{$U = B(a,\epsilon / 2) \times B(b,	\epsilon / 2)$}}
	\step{5}{\pflet{$(x,y) \in U$}}
	\step{6}{$|d(x,y) - d(a,b)| < \epsilon$}
	\begin{proof}
		\pf
		\begin{align*}
			|d(x,y) - d(a,b)| & \leq |d(x,y) - d(a,y)| + |d(a,y) - d(a,b)| \\
			& \leq d(a,x) + d(b,y) & (\text{Proposition \ref{prop:distance_between_distances}}) \\
			& < \epsilon
		\end{align*}
	\end{proof}
	\step{7}{$d(x,y) \in V$}
\end{proof}
\step{2}{If $\mathcal{T}$ is a topology on $X$ with respect to which $d$ is continuous then $\mathcal{T}$ is finer than the metric topology.}
\begin{proof}
	\step{a}{\pflet{$\mathcal{T}$ be a topology on $X$ with respect to which $d$ is continuous.}}
	\step{b}{\pflet{$a \in X$ and $\epsilon > 0$.} \prove{$B(a,\epsilon) \in \mathcal{T}$}}
	\step{c}{\pflet{$x \in B(a,\epsilon)$}}
	\step{d}{$(a,x) \in \inv{d}((0, \epsilon))$}
	\step{e}{\pick\ $U,V \in \mathcal{T}$ such that $(a,x) \in U \times V \subseteq \inv{d}((0, \epsilon))$}
	\step{f}{$x \in V \subseteq B(a, \epsilon)$}
\end{proof}
\qed
\end{proof}

\begin{prop}
\label{prop:metric_finer}
Let $d$ and $d'$ be two metrics on the same set $X$. Let $\mathcal{T}$ and $\mathcal{T}'$ be the topologies they induce. Then $\mathcal{T} \subseteq \mathcal{T}'$ if and only if, for all $x \in X$ and $\epsilon > 0$, there exists $\delta > 0$ such that
\[ B_{d'}(x, \delta) \subseteq B_d(x, \epsilon) \enspace . \]
\end{prop}

\begin{proof}
\pf
\step{1}{If $\mathcal{T} \subseteq \mathcal{T}'$ then, for all $x \in X$ and $\epsilon > 0$, there exists $\delta > 0$ such that $B_{d'}(x, \delta) \subseteq B_d(x, \epsilon)$.}
\begin{proof}
	\step{a}{\assume{$\mathcal{T} \subseteq \mathcal{T}'$}}
	\step{b}{\pflet{$x \in X$ and $\epsilon > 0$}}
	\step{c}{$x \in B_d(x, \epsilon) \in \mathcal{T}'$}
	\step{d}{There exists $\delta > 0$ such that $B_{d'}(x, \delta) \subseteq B_d(x, \epsilon)$}
	\begin{proof}
		\pf\ Proposition \ref{prop:metric_open}.
	\end{proof}
\end{proof}
\step{2}{If, for all $x \in X$ and $\epsilon > 0$, there exists $\delta > 0$ such that $B_{d'}(x, \delta) \subseteq B_d(x, \epsilon)$, then $\mathcal{T} \subseteq \mathcal{T}'$.}
\begin{proof}
	\step{a}{\assume{For all $x \in X$ and $\epsilon > 0$, there exists $\delta > 0$ such that $B_{d'}(x, \delta) \subseteq B_d(x, \epsilon)$.}}
	\step{b}{\pflet{$U \in \mathcal{T}$}}
	\step{c}{For all $x \in U$, there exists $\delta > 0$ such that $B_{d'}(x, \delta) \subseteq U$}
	\begin{proof}
		\step{i}{\pflet{$x \in U$}}
		\step{ii}{\pick\ $\epsilon > 0$ such that $B_d(x, \epsilon) \subseteq U$}
		\begin{proof}
			\pf\ Proposition \ref{prop:metric_open}.
		\end{proof}
		\step{iii}{\pick\ $\delta > 0$ such that $B_{d'}(x, \delta) \subseteq B_d(x, \epsilon)$.}
		\begin{proof}
			\pf\ \stepref{a}
		\end{proof}
		\step{iv}{$B_{d'}(x, \delta) \subseteq U$}
	\end{proof}
	\step{d}{$U \in \mathcal{T}'$}
	\begin{proof}
		\pf\ Proposition \ref{prop:metric_open}.
	\end{proof}
\end{proof}
\qed
\end{proof}

\begin{df}[Metrizable]
A topological space is \emph{metrizable} iff there exists a metric that induces its topology.
\end{df}

\begin{prop}
$\mathbb{R}^2$ under the dictionary order is metrizable.
\end{prop}

\begin{proof}
\pf
\step{1}{\pflet{$d : (\mathbb{R}^2)^2 \rightarrow \mathbb{R}$ be defined by
\[ d((x_1,y_1),(x_2,y_2)) = \begin{cases}
\min(|y_2 - y_1|,1) & \text{if } x_1 = x_2 \\
1 & \text{if } x_1 \neq x_2
\end{cases} \]}}
\step{2}{$d$ is a metric.}
\begin{proof}
	\step{a}{For all $x,y \in \mathbb{R}^2$ we have $d(x,y) \geq 0$.}
	\begin{proof}
		\pf\ Immediate from definition.
	\end{proof}
	\step{b}{For all $x,y \in \mathbb{R}^2$ we have $d(x,y) = 0$ iff $x=y$.}
	\begin{proof}
		\pf\ Immediate from definition.
	\end{proof}
	\step{c}{For all $x,y \in \mathbb{R}^2$ we have $d(x,y) = d(y,x)$.}
	\begin{proof}
		\pf\ Immediate from definition.
	\end{proof}
	\step{d}{For all $x,y,z \in \mathbb{R}^2$ we have $d(x,z) \leq d(x,y) + d(y,z)$.}
	\begin{proof}
		\pf\ Easy.
	\end{proof}
\end{proof}
\step{3}{The metric topology induced by $d$ is finer than the order topology.}
\begin{proof}
	\step{a}{\pflet{$a,b \in \mathbb{R}^2$}}
	\step{b}{\pflet{$x \in (a,b)$}}
	\step{c}{\case{$\pi_1(x) = \pi_1(a) = \pi_1(b)$}}
	\begin{proof}
		\step{i}{\pflet{$\epsilon = \min(\pi_2(x) - \pi_2(a), \pi_2(b) - \pi_2(x))$}}
		\step{ii}{$B(x,\epsilon) \subseteq (a,b)$}
	\end{proof}
	\step{d}{\case{$\pi_1(a) = \pi_1(x) < \pi_1(b)$}}
	\begin{proof}
		\step{i}{\pflet{$\epsilon = \pi_2(x) - \pi_2(a)$}}
		\step{ii}{$B(x,\epsilon) \subseteq (a,b)$}
	\end{proof}
	\step{dd}{\case{$\pi_1(a) < \pi_1(x) = \pi_1(b)$}}
	\begin{proof}
		\pf\ Similar.
	\end{proof}
	\step{e}{\case{$\pi_1(a) < \pi_1(x) < \pi_1(b)$}}
	\begin{proof}
		\pf\ Then $B(x,\epsilon) \subseteq (a,b)$.
	\end{proof}
\end{proof}
\step{4}{The order topology is finer than the metric topology.}
\begin{proof}
	\pf\ Since $B((a,b),\epsilon) = ((a,b-\epsilon),(a,b+\epsilon))$ if $\epsilon \leq 1$, and $\mathbb{R}^2$ if $\epsilon > 1$.
\end{proof}
\qed
\end{proof}

Every metrizable space is first countable.

A metric space is compact if and only if it is sequentially compact.

A metric space is separable if and only if it is second countable.

\subsection{Subspaces}

\begin{prop}
Let $(X,d)$ be a metric space and $Y \subseteq X$. Then $d \restriction Y^2$ is a metric on $Y$ that induces the subspace topology.
\end{prop}

\begin{proof}
\pf
\step{1}{\pflet{$d' = d \restriction Y^2 : Y^2 \rightarrow \mathbb{R}$}}
\step{2}{$d'$ is a metric.}
\begin{proof}
	\pf\ Each of the axioms follows from the axiom in $X$.
\end{proof}
\step{3}{The metric topology induced by $d'$ is finer than the subspace topology.}
\begin{proof}
	\step{a}{\pflet{$U$ be open in $X$} \prove{$U \cap Y$ is open in the $d'$-topology.}}
	\step{b}{\pflet{$y \in U \cap Y$}}
	\step{c}{\pick\ $\epsilon > 0$ such that $B_d(y, \epsilon) \subseteq U$}
	\step{d}{$B_{d'}(y, \epsilon) \subseteq U \cap Y$}
\end{proof}
\step{4}{The subspace topology is finer than the metric topology induced by $d'$.}
\begin{proof}
	\step{a}{\pflet{$y \in Y$ and $\epsilon > 0$} \prove{$B_{d'}(y, \epsilon)$ is open in the subspace topology.}}
	\step{b}{$B_{d'}(y, \epsilon) = B_d(y, \epsilon) \cap Y$}
\end{proof}
\qed
\end{proof}

\subsection{Convergence}

\begin{prop}[Sequence Lemma]
Let $X$ be a metric space. Let $A \subseteq X$. Let $l \in \overline{A}$. Then there exists a sequence in $A$ that converges to $l$.
\end{prop}

\begin{proof}
\pf
\step{1}{For $n \in \mathbb{N}$, \pick\ $a_n \in B(l, 1/(n+1)) \cap A$.}
\step{2}{$a_n \rightarrow l$ as $n \rightarrow \infty$.}
\qed
\end{proof}

\begin{cor}
$\mathbb{R}^\omega$ under the box topology is not first countable.
\end{cor}

\begin{proof}
\pf
\step{1}{\pflet{$A$ be the set of all sequences of positive reals.}}
\step{2}{$0 \in \overline{A}$}
\step{3}{\pflet{$(a_n)$ be a sequence in $A$} \prove{$(a_n)$ does not converge to 0.}}
\step{4}{For all $n \in \mathbb{N}$, \pflet{$a_n = (x_{nm})$}}
\step{5}{\pflet{$B' = \prod_{n=0}^\infty (-x_{nn},x_{nn})$}}
\step{6}{$B'$ is open in the box topology.}
\step{6}{$0 \in B'$}
\step{7}{For all $n$ we have $a_n \notin B'$}
\qed
\end{proof}

\begin{cor}
If $J$ is an uncountable set then $\mathbb{R}^J$ under the product topology is not first countable.
\end{cor}

\begin{proof}
\pf
\step{1}{\pflet{$A = \{ x \in \mathbb{R}^J : \pi_j(x) = 1 \text{ for all but finitely many } j \in J \}$}}
\step{2}{$0 \in \overline{A}$}
\step{3}{\pflet{$(a_n)$ be a sequence in $A$.} \prove{$(a_n)$ does not converge to 0.}}
\step{4}{For $n \in \mathbb{N}$, \pflet{$J_n = \{ j \in J : \pi_j(a_n) \neq 1 \}$}}
\step{5}{$\bigcup_{n \in \mathbb{N}} J_n$ is countable.}
\step{6}{\pick\ $\beta \in J - \bigcup_{n \in \mathbb{N}} J_n$}
\step{7}{$\forall n \in \mathbb{N}. \pi_\beta(a_n) = 1$}
\step{8}{\pflet{$U = \inv{\pi_\beta}((-1,1))$}}
\step{9}{$0 \in U$}
\step{10}{$\forall n \in \mathbb{N}. a_n \notin U$}
\step{11}{$(a_n)$ does not converge to 0.}
\qed
\end{proof}

\subsection{Continuous Functions}

\begin{prop}
Let $X$ and $Y$ be metric spaces. Let $f : X \rightarrow Y$. Then $f$ is continuous if and only if, for all $x \in X$ and $\epsilon > 0$, there exists $\delta > 0$ such that, for all $y \in X$, if $d(x,y) < \delta$ then $d(f(x),f(y)) < \epsilon$.
\end{prop}

\begin{proof}
\pf
\step{1}{If $f$ is continuous then, for all $x \in X$ and $\epsilon > 0$, there exists $\delta > 0$ such that, for all $y \in X$, if $d(x,y) < \delta$ then $d(f(x),f(y)) < \epsilon$.}
\begin{proof}
	\step{a}{\assume{$f$ is continuous.}}
	\step{b}{\pflet{$x \in X$}}
	\step{c}{\pflet{$\epsilon > 0$}}
	\step{d}{$x \in \inv{f}(B(f(x),\epsilon)$}
	\step{e}{There exists $\delta > 0$ such that $B(x, \delta) \subseteq \inv{f}(B(f(x),\epsilon)$.}
\end{proof}
\step{2}{If, for all $x \in X$ and $\epsilon > 0$, there exists $\delta > 0$ such that, for all $y \in X$, if $d(x,y) < \delta$ then $d(f(x),f(y)) < \epsilon$, then $f$ is continuous.}
\begin{proof}
	\step{a}{\assume{For all $x \in X$ and $\epsilon > 0$, there exists $\delta > 0$ such that, for all $y \in X$, if $d(x,y) < \delta$ then $d(f(x),f(y)) < \epsilon$.}}
	\step{b}{\pflet{$V$ be open in $Y$}}
	\step{c}{\pflet{$x \in \inv{f}(V)$}}
	\step{d}{\pick\ $\epsilon > 0$ such that $B(f(x),\epsilon) \subseteq V$}
	\step{e}{\pick\ $\delta > 0$ such that, for all $y \in X$, if $d(x,y) < \delta$ then $d(f(x),f(y)) < \epsilon$.}
	\step{f}{$B(x,\delta) \subseteq \inv{f}(V)$}
\end{proof}
\qed
\end{proof}

\begin{prop}
Let $X$ be a metrizable space and $Y$ a topological space. Let $f : X \rightarrow Y$. Assume that, for every sequence $(x_n)$ in $X$ and $l \in X$, if $x_n \rightarrow l$ as $n \rightarrow \infty$ then $f(x_n) \rightarrow f(l)$ as $n \rightarrow \infty$. Then $f$ is continuous.
\end{prop}

\begin{proof}
\pf
\step{1}{\pflet{$A \subseteq X$} \prove{$f(\overline{A}) \subseteq \overline{f(A)}$}}
\step{2}{\pflet{$l \in \overline{A}$} \prove{$f(l) \in \overline{f(A)}$}}
\step{3}{\pick\ a sequence $(x_n)$ in $A$ such that $x_n \rightarrow l$ as $n \rightarrow \infty$.}
\step{4}{$f(x_n) \rightarrow f(l)$ as $n \rightarrow \infty$.}
\step{5}{$f(l) \in \overline{f(A)}$}
\qed
\end{proof}

\begin{prop}
The function $i : \mathbb{R} - \{0\} \rightarrow \mathbb{R}$ that maps $x$ to $\inv{x}$ is continuous.
\end{prop}

\begin{proof}
\pf
\step{1}{\pflet{$a,b \in \mathbb{R}$ with $a < b$} \prove{$\inv{i}((a,b))$ is open.}}
\step{2}{\case{$0 < a$}}
\begin{proof}
	\pf\ $\inv{i}((a,b)) = (\inv{b}, \inv{a})$
\end{proof}
\step{3}{\case{$a = 0$}}
\begin{proof}
	\pf\ $\inv{i}((a,b)) = (\inv{b}, + \infty)$
\end{proof}
\step{4}{\case{$a < 0 < b$}}
\begin{proof}
	\pf\ $\inv{i}((a,b)) = (-\infty, \inv{a}) \cup (\inv{b}, +\infty)$
\end{proof}
\step{5}{\case{$b = 0$}}
\begin{proof}
	\pf\ $\inv{i}((a,b)) = (-\infty, \inv{a})$
\end{proof}
\step{6}{\case{$b < 0$}}
\begin{proof}
	\pf\ $\inv{i}((a,b)) = (\inv{b}, \inv{a})$
\end{proof}
\qed
\end{proof}

\begin{prop}
Subtraction is a continuous function $\mathbb{R}^2 \rightarrow \mathbb{R}$.
\end{prop}

\begin{proof}
\pf\ Since $a-b$ = $a + (-1)b$ and both addition and multiplication are continuous. \qed
\end{proof}

\begin{prop}
Division is a continuous function $\mathbb{R} \times (\mathbb{R} - \{0\}) \rightarrow \mathbb{R}$.
\end{prop}

\begin{proof}
\pf\ Since both multiplication and the function that maps $x$ to $\inv{x}$ are continuous. \qed
\end{proof}

\subsection{First Countable Spaces}

\begin{prop}
Every metrizable space is first countable.
\end{prop}

\begin{proof}
\pf\ For any point $x$, the set $\{ B(x, 1/n) : n \in \mathbb{Z}_+ \}$ is a countable basis at $x$. \qed
\end{proof}

\begin{cor}
$\mathbb{R}^\omega$ under the box topology is not metrizable.
\end{cor}

\begin{cor}
If $J$ is an uncountable set then $\mathbb{R}^J$ under the product topology is not metrizable.
\end{cor}

\subsection{Hausdorff Spaces}

\begin{prop}
Every metric space is Hausdorff.
\end{prop}

\begin{proof}
\pf
\step{1}{\pflet{$X$ be a metric space.}}
\step{2}{\pflet{$x,y \in X$ with $x \neq y$.}}
\step{3}{\pflet{$\epsilon = d(x,y)$}}
\step{4}{$B(x,\epsilon / 2)$ and $B(y, \epsilon / 2)$ are disjoint neighbourhoods of $x$ and $y$.}
\qed
\end{proof}

\subsection{Bounded Sets}

\begin{df}[Bounded]
Let $X$ be a metric space. Let $A \subseteq X$. Then $A$ is \emph{bounded} iff there exists $M$ such that $\forall x,y \in A. d(x,y) \leq M$. Its \emph{diameter} is then defined to be
\[ \diam A := \sup \{ d(x,y) : x,y \in A \} \enspace . \]
\end{df}

\subsection{Uniform Convergence}

\begin{df}[Uniform Convergence]
Let $X$ be a set and $Y$ a metric space. Let $(f_n)$ be a sequence of functions $X \rightarrow Y$, and $f : X \rightarrow Y$. Then $(f_n)$ \emph{converges uniformly} to $f$ iff, for all $\epsilon > 0$, there exists $N$ such that
\[ \forall n \geq N. \forall x \in X. d(f_n(x),f(x)) < \epsilon \enspace . \]
\end{df}

\begin{ex}
For $n \in \mathbb{N}$ define $f_n : [0,1] \rightarrow \mathbb{R}$ by $f_n(x) = x^n$. Define $f : [0,1] \rightarrow \mathbb{R}$ by $f(x) = 0$ for $x < 1$, $f(1) = 1$. Then $f_n$ converges pointwise to $f$, but does not converge uniformly to $f$.

We prove that, for all $N$, there exists $n \geq N$ and $x \in [0,1]$ such that $|x^n - f(x)| \geq 1/2$. Take $n = N$ and $x$ to be the $N$th root of $3/4$.
\end{ex}

\begin{ex}
For $n \in \mathbb{N}$, define $f_n : \mathbb{R} \rightarrow \mathbb{R}$ by
\[ f_n(x) = \frac{1}{n^3[x-(1/n)]^2 + 1} \enspace . \]
Then for all $x \in \mathbb{R}$ we have $f_n(x) \rightarrow 0$ as $n \rightarrow \infty$, but $(f_n)$ does not converge uniformly to 0.

We prove that, for all $N$, there exists $n \geq N$ and $x \in \mathbb{R}$ such that $|f_n(x)| \geq 1/2$. Take $n = N$ and $x = 1/N$. We have $f_N(1/N) = 1$.
\end{ex}

\begin{thm}[Uniform Limit Theorem]
Let $X$ be a topological space and $Y$ a metric space. Let $(f_n)$ be a sequence of functions $X \rightarrow Y$, and $f : X \rightarrow Y$. If every $f_n$ is continuous and $(f_n)$ converges uniformly to $f$, then $f$ is continuous.
\end{thm}

\begin{proof}
\pf
\step{1}{\pflet{$V$ be open in $Y$.}}
\step{2}{\pflet{$x_0 \in \inv{f}(V)$} \prove{There exists a neighbourhood $U$ of $x_0$ such that $f(U) \subseteq V$.}}
\step{3}{\pflet{$y_0 = f(x_0)$}}
\step{4}{\pick\ $\epsilon > 0$ such that $B(y_0, \epsilon) \subseteq V$.}
\step{5}{\pick\ $N$ such that $\forall n \geq N. \forall x \in X. d(f_n(x),f(x)) < \epsilon / 3$.}
\step{7}{\pick\ a neighbourhood $U$ of $x_0$ such that $f_N(U_2) \subseteq B(f_N(x_0), \epsilon / 3)$. \prove{$f(U) \subseteq V$}}
\step{9}{\pflet{$y \in U$}}
\step{10}{$d(f(y),y_0) < \epsilon$}
\begin{proof}
	\pf
	\begin{align*}
		d(f(y),y_0) & \leq d(f(y),f_N(y)) + d(f_N(y),f_N(x_0)) + d(f_N(x_0),y_0) \\
		& < \epsilon / 3 + \epsilon / 3 + \epsilon / 3 & (\text{\stepref{5}, \stepref{7}}) l\\
		& = \epsilon
	\end{align*}
\end{proof}
\step{11}{$f(y) in V$}
\begin{proof}
	\pf\ \stepref{4}
\end{proof}
\qed
\end{proof}

\begin{prop}
Let $X$ be a topological space. Let $Y$ be a metric space. Let $f_n$ be a sequence of functions $X \rightarrow Y$ and $f : X \rightarrow Y$. Let $x_n$ be a sequence of points in $X$ and $l \in X$. If $f_n$ converges uniformly to $f$, $x_n$ converges to $l$, and $f$ is continuous, then $f_n(x_n)$ converges to $f(l)$.
\end{prop}

\begin{proof}
\pf
\step{0}{$f$ is continuous.}
\step{1}{\pflet{$\epsilon > 0$}}
\step{3}{\pick\ $\delta > 0$ such that $\forall y \in X. d(y,l) < \delta \Rightarrow d(f(y),f(l)) < \epsilon / 2$}
\step{2}{\pick\ $N$ such that $\forall n \geq N. \forall x \in X. d(f_n(x),f(x)) < \epsilon / 2$ and $\forall n \geq N. d(x_n,l) < \delta$}
\step{3}{For all $n \geq N$ we have $d(f_n(x_n),f(l)) < \epsilon$}
\begin{proof}
	\pf
	\begin{align*}
		d(f_n(x_n),f(l)) & \leq d(f_n(x_n),f(x_n)) + d(f(x_n),f(l)) \\
		& < \epsilon / 2 + \epsilon / 2 \\
		& = \epsilon
	\end{align*}
\end{proof}
\qed
\end{proof}

\begin{thm}[Weierstrass $M$-Test]
Let $X$ be a set. Let $(f_n)$ be a sequence of functions $X \rightarrow \mathbb{R}$. Let $(M_n)$ be a sequence of real numbers. For $n \in \mathbb{N}$, let
\[ s_n(x) = \sum_{i=0}^n f_i(x) \enspace . \]
Assume that $\forall n \in \mathbb{N}. \forall x \in X. |f_n(x)| \leq M_n$. Assume that $\sum_{n=0}^\infty M_n$ converges. Then $(s_n)$ uniformly converges to $s$ where $s(x) = \sum_{n=0}^\infty f_n(x)$.
\end{thm}

\begin{proof}
\pf
\step{0}{For all $x \in X$ we have $\sum_{n=0}^\infty f_n(x)$ converges.}
\begin{proof}
	\pf\ By the Comparison Test.
\end{proof}
\step{1}{For $n \in \mathbb{N}$, \pflet{$r_n = \sum_{i=n+1}^\infty M_i$.}}
\step{2}{For all $k,n \in \mathbb{N}$ and $x \in X$, if $k > n$ then $|s_k(x) - s_n(x)| \leq r_n$.}
\begin{proof}
	\pf
	\begin{align*}
		|s_k(x) - s_n(x)| & = \left| \sum_{i=n+1}^k f_i(x) \right| \\
		& \leq \sum_{i=n+1}^k |f_i(x)| \\
		& \leq \sum_{i=n+1}^k M_i \\
		& \leq \sum_{i=n+1}^\infty M_i \\
		& = r_n
	\end{align*}
\end{proof}
\step{4}{For all $n \in \mathbb{N}$ we have $|s(x) - s_n(x)| \leq r_n$.}
\begin{proof}
	\pf\ Taking the limit $k \rightarrow \infty$ in \stepref{2}.
\end{proof}
\step{5}{$(s_n)$ converges uniformly to $s$.}
\begin{proof}
	\pf\ We have $\overline{\rho}(s_n,s) \leq r_n$ and so $\overline{\rho}(s_n,s) \rightarrow 0$ as $n \rightarrow \infty$ by the Sandwich Theorem.
\end{proof}
\qed
\end{proof}

\subsection{Standard Bounded Metric}

\begin{df}[Standard Bounded Metric]
Let $(X,d)$ be a metric space. The \emph{standard bounded metric} corresponding to $d$ is
\[ \overline{d}(x,y) := \min(d(x,y),1) \enspace . \]
\end{df}

\begin{prop}
The standard bounded metric associated with $d$ induces the same topology as $d$.
\end{prop}

\begin{proof}
\pf
\step{0}{\pflet{$(X,d)$ be a metric space.}}
\step{1}{Every $d$-ball is open under the topology induced by $\overline{d}$.}
\begin{proof}
	\step{a}{\pflet{$a \in X$ and $\epsilon > 0$}}
	\step{b}{\pflet{$x \in B_d(a, \epsilon)$}}
	\step{c}{\pflet{$\delta = \min(\epsilon - d(a,x), 1/2)$}}
	\step{d}{$B_{\overline{d}}(x, \delta) \subseteq B_d(a, \epsilon)$}
\end{proof}
\step{2}{Every $\overline{d}$-ball is open under the topology induced by $d$.}
\begin{proof}
	\pf\ Since $B_{\overline{d}}(a, \epsilon) = B_d(a, \epsilon)$ if $\epsilon \leq 1$, and $X$ if $\epsilon > 1$.
\end{proof}
\qed
\end{proof}

\subsection{Product Spaces}

\begin{prop}
The product of a countable family of metrizable spaces is metrizable.
\end{prop}

\begin{proof}
\pf
\step{1}{\pflet{$(X_n, d_n)$ be a sequence of metric spaces.}}
\step{2}{For $n \in \mathbb{N}$, \pflet{$\overline{d_n}$ be the standard bounded metric associated with $d_n$.}}
\step{3}{\pflet{$X = \prod_{n \in \mathbb{N}} X_n$}}
\step{4}{Define $D : X^2 \rightarrow \mathbb{R}$ by $D(x,y) = \sup_{n \in \mathbb{N}} \overline{d_n}(\pi_n(x), \pi_n(y)) / (n+1)$.}
\step{5}{$D$ is a metric on $X$.}
\begin{proof}
	\step{a}{For all $x,y \in X$ we have $D(x,y) \geq 0$.}
	\step{b}{For all $x,y \in X$ we have $D(x,y) = 0$ iff $x = y$.}
	\step{c}{For all $x,y \in X$ we have $D(x,y) = D(y,x)$.}
	\step{d}{For all $x,y,z \in X$ we have $D(x,z) \leq D(x,y) + D(y,z)$.}
\end{proof}
\step{6}{The product topology is finer than the metric topology induced by $D$.}
\begin{proof}
	\step{a}{\pflet{$a \in X$ and $\epsilon > 0$.}}
	\step{b}{\pflet{$x \in B(a, \epsilon)$}}
	\step{c}{\pflet{$\delta = \epsilon - D(a,x)$}}
	\step{c}{\pick\ $N \in \mathbb{N}$ such that $1 / (N+1) < \delta$}
	\step{d}{$x \in \prod_{n=0}^N B_{\overline{d_n}}(\pi_n(a), n \delta) \times \prod_{n=N+1}^\infty \subseteq B(a, \epsilon)$}
\end{proof}
\step{7}{The metric topology induced by $D$ is finer than the product topology.}
\begin{proof}
	\step{a}{\pflet{$n \in \mathbb{N}$ and $U$ be an open set in $X_n$.} \prove{$\inv{\pi_n}(U)$ is open in the metric topology.}}
	\step{b}{\pflet{$x \in \inv{\pi_n}(U)$}}
	\step{c}{\pick\ $\epsilon > 0$ such that $B_{\overline{d_n}}(\pi_n(x),\epsilon) \subseteq U$}
	\step{d}{$B(x, \epsilon / (n+1)) \subseteq \inv{\pi_n}(U)$}
\end{proof}
\qed
\end{proof}

\subsection{Uniform Metric}

\begin{df}[Uniform Metric]
Let $J$ be a nonempty set.
The \emph{uniform metric} $\overline{\rho}$ on $\mathbb{R}^J$ is defined by
\[ \overline{\rho}(x,y) = \sup_{j \in J} \overline{d}(x_j,y_j) \]
where $\overline{d}$ is the standard bounded metric associated with the standard metric on $\mathbb{R}$.

The topology it induces is called the \emph{uniform topology}.

We prove this is a metric.
\end{df}

\begin{proof}
\pf
\step{1}{For all $x,y \in \mathbb{R}^\omega$ we have $\overline{\rho}(x,y) \geq 0$.}
\begin{proof}
	\pf\ Pick $j_0 \in J$. Then
	\begin{align*}
		\overline{\rho}(x,y) & = \sup_j \overline{d}(x_j,y_j) \\
		& \geq \overline{d}(x_{j_0},y_{j_0}) \\
		& \geq 0
	\end{align*}
\end{proof}
\step{2}{For all $x,y \in \mathbb{R}^\omega$ we have $\overline{\rho}(x,y) = 0$ iff $x = y$.}
\begin{proof}
	\pf
	\begin{align*}
		\overline{\rho}(x,y) = 0 & \Leftrightarrow \sup_j \overline{d}(x_j,y_j) = 0 \\
		& \Leftrightarrow \forall j. \overline{d}(x_j,y_j) = 0 \\
		& \Leftrightarrow \forall j. x_j = y_j \\
		& \Leftrightarrow x = y
	\end{align*}
\end{proof}
\step{3}{For all $x,y \in \mathbb{R}^\omega$ we have $\overline{\rho}(x,y) = \overline{\rho}(y,x)$.}
\begin{proof}
	\pf
	\begin{align*}
		\overline{\rho}(x,y) & = \sup_j \overline{d}(x_j,y_j) \\
		& = \sup_j \overline{d}(y_j,x_j) \\
		& = \overline{\rho}(y,x)
	\end{align*}
\end{proof}
\step{4}{For all $x,y,z \in \mathbb{R}^\omega$ we have $\overline{\rho}(x,z) \leq \overline{\rho}(x,y) + \overline{\rho}(y,z)$.}
\begin{proof}
	\pf
	\begin{align*}
		\overline{\rho}(x,z) & = \sup_j \overline{d}(x_j,z_j) \\
		& \leq \sup_j (\overline{d}(x_j,y_j) + \overline{d}(y_j,z_j)) \\
		& \leq \sup_j \overline{d}(x_j,y_j) + \sup_j \overline{d}(y_j,z_j) \\
		& = \overline{\rho}(x,y) + \overline{\rho}(y,z)
	\end{align*}
\end{proof}
\qed
\end{proof}

\begin{prop}
The uniform topology is finer than the product topology. It is strictly finer iff $J$ is infinite.
\end{prop}

\begin{proof}
\pf
\step{1}{The uniform topology is finer than the product topology.}
\begin{proof}
	\step{a}{\pflet{$U$ be open in $\mathbb{R}$ and $j \in J$} \prove{$\inv{\pi_j}(U)$ is open in the uniform topology.}}
	\step{b}{\pflet{$x \in \inv{\pi_j}(U)$}}
	\step{c}{$\pi_j(x) \in U$}
	\step{d}{\pick\ $\epsilon > 0$ such that $B_{\overline{d}}(\pi_j(x), \epsilon) \subseteq U$}
	\step{e}{$B_{\overline{\rho}}(x, \epsilon) \subseteq \inv{\pi_j}(U)$}
\end{proof}
\step{2}{If $J$ is finite then the uniform topology is equal to the product topology.}
\begin{proof}
	\pf\ In $\mathbb{R}^n$, the uniform topology is the square topology.
\end{proof}
\step{3}{If $J$ is infinite then the uniform topology is not equal to the product topology.}
\begin{proof}
	\pf\ If $J$ is infinite then $B(0,1)$ is not open in the product topology.
\end{proof}
\qed
\end{proof}

\begin{prop}
The uniform topology is coarser than the box topology. It is strictly coarser iff $J$ is infinite.
\end{prop}

\begin{proof}
\pf
\step{1}{The uniform topology is coarser than the box topology.}
\begin{proof}
	\step{a}{\pflet{$U$ be open in the uniform topology.} \prove{$U$ is open in the box topology.}}
	\step{b}{\pflet{$x \in U$}}
	\step{c}{\pick\ $\epsilon > 0$ such that $B(x, \epsilon) \subseteq U$}
	\step{d}{$\prod_{j \in J} (x_j - \epsilon, x_j + \epsilon) \subseteq U$}
\end{proof}
\step{2}{If $J$ is finite then the uniform topology is equal to the box topology.}
\begin{proof}
	\pf\ On $\mathbb{R}^n$, the uniform metric is the square metric.
\end{proof}
\step{3}{If $J$ is infinite then the uniform topology is not equal to the box topology.}
\begin{proof}
	\step{a}{\assume{$J$ is infinite.}}
	\step{b}{\pick\ a sequence $(j_n)$ of distinct elements in $J$.}
	\step{c}{\pflet{$U = \prod_j U_j$ where $J_{j_n} = (-1/(n+1),1/(n+1))$ for $n \in \mathbb{N}$ and $J_j = (-1,1)$ for all other $j$.}}
	\step{d}{$U$ is not open in the uniform topology.}
\end{proof}
\qed
\end{proof}

\begin{prop}
The uniform topology on $\mathbb{R}^\infty$ is strictly finer than the product topology.
\end{prop}

\begin{proof}
\pf\ The set of all sequences $(x_n) \in \mathbb{R}^\infty$ such that $\forall n. |x_n| < 1$ is open in the uniform topology but not in the product topology. \qed
\end{proof}

\begin{prop}
The uniform topology on $\mathbb{R}^\infty$ is strictly coarser than the box topology.
\end{prop}

\begin{proof}
\pf\ The set of sequences $(x_n) \in \mathbb{R}^\infty$ such that $\forall n. |x_n| < 1/n$ is open in the box topology but not in the uniform topology. \qed
\end{proof}

\begin{prop}
The uniform topology on the Hilbert cube is the same as the product topology.
\end{prop}

\begin{proof}
\pf
\step{1}{\pflet{$(x_n)$ be in the Hilbert cube $H$ and $\epsilon > 0$.} \prove{$B((x_n), \epsilon) \cap H$ is open in the product topology.}}
\step{2}{\pick\ $N$ such that $1/N < \epsilon$}
\step{3}{$B((x_n), \epsilon) = (\prod_{n=0}^N (x_n - \epsilon, x_n + \epsilon) \times \prod_{n=N+1}^\infty [0,1/(n+1)]) \cap H$}
\qed
\end{proof}

\begin{cor}
The uniform topology on the Hilbert cube is strictly finer than the box topology.
\end{cor}

\begin{prop}
Let $X$ be a set and $Y$ a metric space. Let $(f_n)$ be a sequence of functions $X \rightarrow Y$, and $f : X \rightarrow Y$. Then $(f_n)$ converges uniformly to $f$ iff $(f_n)$ converges to $f$ in $Y^X$ under the uniform topology.
\end{prop}

\begin{proof}
\pf
\step{1}{If $(f_n)$ converges uniformly to $f$ then $(f_n)$ converges to $f$ in $Y^X$ under the uniform topology.}
\begin{proof}
	\step{a}{\assume{$(f_n)$ converges uniformly to $f$.}}
	\step{b}{\pflet{$\epsilon > 0$}}
	\step{c}{\pick\ $N$ such that $\forall n \geq N. \forall x \in X. d(f_n(x),f(x)) < \epsilon / 2$}
	\step{d}{$\forall n \geq N. \overline{\rho}(f_n,f) \leq \epsilon / 2$}
	\step{e}{$\forall n \geq N. \overline{\rho}(f_n,f) < \epsilon$}
\end{proof}
\step{2}{If $(f_n)$ converges to $f$ in $Y^X$ under the uniform topology then $(f_n)$ converges uniformly to $f$.}
\begin{proof}
	\step{a}{\assume{$(f_n)$ converges to $f$ in $Y^X$ under the uniform topology.}}
	\step{b}{\pflet{$\epsilon > 0$}}
	\step{c}{\pick\ $N$ such that $\forall n \geq N. \overline{\rho}(f_n,f) < \epsilon$}
	\step{d}{$\forall n \geq N. \forall x \in X. d(f_n(x),f(x)) < \epsilon$}
\end{proof}
\qed
\end{proof}

\subsection{Products}

\begin{df}[Euclidean Metric]
Let $X$ and $Y$ be metric spaces. The \emph{Euclidean metric} on $X \times Y$ is
\[ d((x_1, y_1), (x_2, y_2)) = \sqrt{d(x_1,x_2)^2 + d(y_1,y_2)^2} \enspace .\]
We write $X \times Y$ for the set $X \times Y$ under this metric.

We prove this is a metric.
\end{df}

\begin{proof}
\pf
\step{1}{$d((x_1,y_1),(x_2,y_2)) \geq 0$}
\begin{proof}
	\pf\ Immediate from definition.
\end{proof}
\step{2}{$d((x_1,y_1),(x_2,y_2)) = 0$ iff $(x_1,y_1) = (x_2,y_2)$}
\begin{proof}
	\pf\ $\sqrt{d(x_1,x_2)^2 + d(y_1,y_2)^2} = 0$ iff $d(x_1,x_2) = d(y_1,y_2) = 0$ iff $x_1 = x_2$ and $y_1 = y_2$.
\end{proof}
\step{3}{$d((x_1,y_1),(x_2,y_2)) = d((x_2,y_2),(x_1,y_1))$}
\begin{proof}
	\pf\ Since $\sqrt{d(x_1,x_2)^2 + d(y_1,y_2)^2} = \sqrt{d(x_2,x_1)^2 + d(y_2,y_1)^2}$.
\end{proof}
\step{4}{The triangle inequality holds.}
\begin{proof}
	\pf
	\begin{align*}
		& (d((x_1,y_1),(x_2,y_2)) + d((x_2,y_2),(x_3,y_3)))^2 \\
		= & d((x_1,y_1),(x_2,y_2))^2 + 2 d((x_1,y_1),(x_2,y_2)) d((x_2,y_2),(x_3,y_3)) + d((x_2,y_2),(x_3,y_3))^2 \\
		= & d(x_1,x_2)^2 + d(y_1,y_2)^2 + 2 \sqrt{(d(x_1,x_2)^2 + d(y_1,y_2)^2)(d(x_2,x_3)^2 + d(y_2,y_3)^2)} + d(x_2,x_3)^2 + d(y_2,y_3)^2\\
		\geq & d(x_1,x_2)^2 + d(x_2,x_3)^2 + d(y_1,y_2)^2 + d(y_2,y_3)^ + 2(d(x_1,x_2)d(x_2,x_3) + d(y_1,y_2) d(y_2,y_3)) \\
		& (\text{Cauchy-Schwarz}) \\
		= & (d(x_1,x_2) + d(x_2,x_3))^2 + (d(y_1,y_2) + d(y_2,y_3))^2 \\
		\geq & d(x_1,x_3)^2 + d(y_1,y_3)^2 \\
		= & d((x_1,y_1),(x_3,y_3))^2
	\end{align*}
\end{proof}
\qed
\end{proof}

\begin{prop}
Let $X$ and $Y$ be metric spaces. The Euclidean metric on $X \times Y$ induces the product topology on $X \times Y$.
\end{prop}

\begin{proof}
\pf
\step{1}{Every open ball is open in the product topology.}
\begin{proof}
	\step{a}{\pflet{$(x,y) \in B((a,b),\epsilon)$} \prove{$B(x, \sqrt{\epsilon}) \times B(y, \sqrt{\epsilon}) \subseteq B((a,b),\epsilon)$}}
	\step{b}{\pflet{$x' \in B(x, \sqrt{(\epsilon - d((x,y),(a,b)))^2/2})$ and $y' \in B(y, \sqrt{(\epsilon - d((x,y),(a,b)))^2/2})$} \prove{$d((x',y'),(a,b)) < \epsilon$}}
	\step{c}{$d((x',y'),(x,y)) < \epsilon - d((x,y),(a,b))$}
	\begin{proof}
		\pf
		\begin{align*}
			d((x',y'),(x,y)) & = \sqrt{d(x',x)^2 + d(y',y)^2} \\
			& < \sqrt{(\epsilon - d((x,y),(a,b)))^2/2 + (\epsilon - d((x,y),(a,b))^2/2} \\
			& = \epsilon - d((x,y),(a,b))
		\end{align*}
	\end{proof}
	\step{d}{$d((x',y'),(a,b)) < \epsilon$}
	\begin{proof}
		\pf
		\begin{align*}
			d((x',y'),(a,b)) & \leq d((x',y'),(x,y)) + d((x,y),(a,b)) & (\text{Triangle Inequality}) \\
			& < \epsilon & (\text{\stepref{c}})
		\end{align*}
	\end{proof}
\end{proof}
\step{2}{If $U$ is open in $X$ and $V$ is open in $Y$ then $U \times V$ is open under the Euclidean metric.}
\begin{proof}
	\step{a}{\pflet{$(x,y) \in U \times V$}}
	\step{b}{\pick\ $\delta, \epsilon > 0$ such that $B(x,\delta) \subseteq U$ and $B(y,\epsilon) \subseteq V$ \prove{$(B((x,y),\min(\delta, \epsilon)) \subseteq U \times V$}}
	\step{c}{\pflet{$(x',y') \in B((x,y),\min(\delta, \epsilon))$}}
	\step{d}{$d(x',x) < \delta$}
	\begin{proof}
		\step{i}{$d((x',y'),(x,y)) < \min(\delta, \epsilon)$}
		\step{ii}{$d(x',x)^2 + d(y',y)^2 < \delta^2$}
		\step{iii}{$d(x',x)^2 < \delta^2$}
	\end{proof}
	\step{e}{$d(y',y) < \epsilon$}
	\begin{proof}
		\pf\ Similar.
	\end{proof}
	\step{f}{$(x',y') \in U \times V$}
\end{proof}
\qed
\end{proof}

\begin{prop}
The square metric on $\mathbb{R}^n$ induces the product topology.
\end{prop}

\begin{proof}
\pf
\step{1}{\pflet{$d$ be the Euclidean metric on $\mathbb{R}^n$ and $\rho$ the square metric.}}
\step{2}{For all $x \in X$ and $\epsilon > 0$, there exists $\delta > 0$ such that $B_d(x, \delta) \subseteq B_\rho(x, \epsilon)$}
\begin{proof}
	\pf\ If $d(x,y) < \epsilon$ then $\rho(x,y) < \epsilon$.
\end{proof}
\step{3}{For all $x \in X$ and $\epsilon > 0$, there exists $\delta > 0$ such that $B_\rho(x, \delta) \subseteq B_d(x, \epsilon)$}
\begin{proof}
	\pf\ If $\rho(x,y) < \epsilon / \sqrt{n}$ then $d(x,y) < \epsilon$.
\end{proof}
\step{4}{$d$ and $\rho$ induce the same topology.}
\begin{proof}
	\pf\ Proposition \ref{prop:metric_finer}.
\end{proof}
\qed
\end{proof}

\section{Isometric Embeddings}

\begin{df}[Isometric Embedding]
Let $X$ and $Y$ be metric spaces. Let $f : X \rightarrow Y$. Then $f$ is an \emph{isometric embedding} of $X$ in $Y$ iff, for all $x,y \in X$, we have $d(f(x),f(y)) = d(x,y)$.
\end{df}

\begin{prop}
Every isometric embedding is an embedding.
\end{prop}

\begin{proof}
\pf
\step{1}{\pflet{$X$ and $Y$ be metric spaces.}}
\step{2}{\pflet{$f : X \rightarrow Y$ be an isometric embedding.}}
\step{3}{$f$ is injective.}
\step{4}{The subspace topology induced by $f$ is finer than the metric topology.}
\begin{proof}
	\step{a}{\pflet{$x \in X$ and $\epsilon > 0$} \prove{$B(x,\epsilon)$ is open in the subspace topology.}}
	\step{b}{$B(x, \epsilon) = \inv{f}(B(f(x),\epsilon))$}
\end{proof}
\step{5}{The metric topology is finer than the subspace topology induced by $f$.}
\begin{proof}
	\step{a}{\pflet{$V$ be open in $Y$} \prove{$\inv{f}(V)$ is open in $X$}}
	\step{b}{\pflet{$x \in \inv{f}(V)$}}
	\step{c}{\pick\ $\epsilon > 0$ such that $B(f(x), \epsilon) \subseteq V$}
	\step{d}{$B(x, \epsilon) \subseteq \inv{f}(V)$}
\end{proof}
\qed
\end{proof}

\section{Complete Metric Spaces}

\begin{df}[Complete]
A metric space is \emph{complete} iff every Cauchy sequence converges.
\end{df}

\begin{ex}
$\mathbb{R}$ is complete.
\end{ex}

\begin{prop}
The product of two complete metric spaces is complete.
\end{prop}

\begin{prop}
Every compact metric space is complete.
\end{prop}

\begin{prop}
Let $X$ be a complete metric space and $A \subseteq X$. Then $A$ is complete if and only if $A$ is closed.
\end{prop}

\begin{df}[Completion]
Let $X$ be a metric space. A \emph{completion} of $X$ is a complete metric space $\hat{X}$ and injection $i : X \rightarrowtail \hat{X}$ such that:
\begin{itemize}
\item The metric on $X$ is the restriction of the metric on $\hat{X}$
\item $X$ is dense in $\hat{X}$.
\end{itemize}
\end{df}

\begin{prop}
Let $i_1 : X \rightarrow Y_1$ and $i_2 : X \rightarrow Y_2$ be completions of $X$. Then there exists a unique isometry $\phi : Y_1 \cong Y_2$ such that $\phi \circ i_1 = i_2$.
\end{prop}

\begin{proof}
\pf\ Define $\phi(\lim_{n \rightarrow \infty} i_1(x_n)) = \lim_{n \rightarrow \infty} i_2(x_n)$. \qed
\end{proof}

\begin{thm}
Every metric space has a completion.
\end{thm}

\begin{proof}
\pf\ Let $\hat{X}$ be the set of Cauchy sequences in $X$ quotiented by $\sim$ where $(x_n) \sim (y_n)$ if and only if $d(x_n, y_n) \rightarrow 0$. \qed
\end{proof}

\section{Manifolds}

\begin{df}[Manifold]
An \emph{$n$-dimensional manifold} is a second countable Hausdorff space locally homeomorphic to $\mathbb{R}^n$.
\end{df}

\chapter{Homotopy Theory}

\section{Homotopies}

\begin{df}[Homotopy]
Let $X$ and $Y$ be topological spaces. Let $f,g : X \rightarrow Y$ be continuous. A \emph{homotopy} between $f$ and $g$ is a continuous function $h : X \times [0,1] \rightarrow Y$ such that
\begin{itemize}
\item $\forall x \in X. h(x,0) = f(x)$
\item $\forall x \in X. h(x,1) = g(x)$
\end{itemize}
We say $f$ and $g$ are \emph{homotopic}, $f \simeq g$, iff there exists a homotopy between them.

Let $[X,Y]$ be the set of all homotopy classes of functions $X \rightarrow Y$.
\end{df}

\begin{prop}
Let $f,f' : X \rightarrow Y$ and $g,g' : Y \rightarrow Z$ be continuous. If $f \simeq f'$ and $g \simeq g'$ then $g \circ f \simeq g' \circ f'$.
\end{prop}

\begin{df}
Let $\mathbf{HTop}$ be the category whose objects are the small topological spaces and whose morphisms are the homotopy classes of continuous functions.

A \emph{homotopy functor} is a functor $\Top \rightarrow \mathcal{C}$ that factors through the canonical functor $\Top \rightarrow \mathbf{HTop}$.
\end{df}

\begin{df}
A functor $F : \mathbf{Top} \rightarrow \mathcal{C}$ is \emph{homotopy invariant} iff, for any topological spaces $X$, $Y$ and continuous functions $f,g : X \rightarrow Y$, if $f \simeq g$ then $Hf = Hg$.
\end{df}

Basepoint-preserving homotopy.

\section{Homotopy Equivalence}

\begin{df}[Homotopy Equivalence]
Let $X$ and $Y$ be topological spaces. A \emph{homotopy equivalence} between $X$ and $Y$, $f : X \simeq Y$, is a continuous function $f : X \rightarrow Y$ such that there exists a continuous function $g : Y \rightarrow X$, the \emph{homotopy inverse} to $f$, such that $g \circ f \simeq \id{X}$ and $f \circ g \simeq \id{Y}$.
\end{df}

\begin{df}[Contractible]
A topological space $X$ is \emph{contractible} iff $X \simeq 1$.
\end{df}

\begin{ex}
$\mathbb{R}^n$ is contractible.
\end{ex}

\begin{ex}
$D^n$ is contractible.
\end{ex}

\begin{df}[Deformation Retract]
Let $X$ be a topological space and $A$ a subspace of $X$. A retraction $\rho : X \rightarrow A$ is a \emph{deformation retraction} iff $i \circ \rho \simeq \id{X}$, where $i$ is the inclusion $A \rightarrowtail X$. We say $A$ is a \emph{deformation retract} of $X$ iff there exists a deformation retraction.
\end{df}

\begin{df}[Strong Deformation Retract]
Let $X$ be a topological space and $A$ a subspace of $X$. A \emph{strong deformation retraction} $\rho : X \rightarrow A$ is a continuous function such that there exists a homotopy $h : X \times [0,1] \rightarrow X$ between $i \circ \rho$ and $\id{X}$ such that, for all $a \in X$ and $t \in [0,1]$, we have $h(a,t) = a$.

We say $A$ is a \emph{strong deformation retract} of $X$ iff a strong deformation retraction exists.
\end{df}

\begin{ex}
$\{0\}$ is a strong deformation retract of $\mathbb{R}^n$ and of $D^n$.
\end{ex}

\begin{ex}
$S^1$ is a strong deformation retract of the torus $S^1 \times D^2$.
\end{ex}

\begin{ex}
$S^{n-1}$ is a strong deformation retract of $D^n - \{0\}$.
\end{ex}

\begin{ex}
For any topological space $X$, the singleton consisting of the vertex is a strong deformation retract of the cone over $X$.
\end{ex}

\chapter{Simplicial Complexes}

\begin{df}[Simplex]
A \emph{$k$-dimensional simplex} or \emph{$k$-simplex} in $\mathbb{R}^n$ is the convex hull $s(x_0, \ldots, x_k)$ of $k+1$ points in general position.
\end{df}

\begin{df}[Face]
A \emph{sub-simplex} or \emph{face} of $s(x_0, \ldots, x_k)$ is the convex hull of a subset of $\{x_0, \ldots, x_k\}$.
\end{df}

\begin{df}[Simplicial Complex]
A \emph{simplicial complex} in $\mathbb{R}^n$ is a set $K$ of simplices such that:
\begin{itemize}
\item for every simplex $s$ in $K$, every face of $s$ is in $K$.
\item The intersection of two simplices $s_1, s_2 \in K$ is either empty or is a face of both $s_1$ and $s_2$.
\item $K$ is locally finite, i.e. every point of $\mathbb{R}^n$ has a neighbourhood that only intersects finitely many elements of $K$.
\end{itemize}

The topological space \emph{underlying} $K$ is $|K| = \bigcup K$ as a subspace of $\mathbb{R}^n$.
\end{df}

\section{Cell Decompositions}

\begin{df}[$n$-cell]
An \emph{$n$-cell} is a topological space homeomorphic to $\mathbb{R}^n$.
\end{df}

\begin{df}[Cell Decomposition]
Let $X$ be a topological space. A \emph{cell decomposition} of $X$ is a partition of $X$ into subspaces that are $n$-cells.
\end{df}

\begin{df}[$n$-skeleton]
Given a cell decomposition of $X$, the \emph{$n$-skeleton} $X^n$ is the union of all the cells of dimension $\leq n$.
\end{df}

\section{CW-complexes}

\begin{df}[CW-Complex]
A \emph{CW-complex} consists of a topological space $X$ and a cell decomposition $\mathcal{E}$ of $X$ such that:
\begin{enumerate}
\item \emph{Characteristic Maps} For every $n$-cell $e \in \mathcal{E}$, there exists a continuous map $\Phi_e : D^n \rightarrow X$ such that $\Phi((D^n)^\circ) = e$, the corestriction $\Phi_e : (D^n)^\circ \approx e$ is a homeomorphism, and $\Phi_e(S^n)$ is the union of all the cells in $\mathcal{E}$ of dimension $< n$.
\item \emph{Closure Finiteness} For all $e \in \mathcal{E}$, we have $\overline{e}$ intersects only finitely many other cells in $\mathcal{E}$.
\item \emph{Weak Topology} Given $A \subseteq X$, we have $A$ is closed iff for all $e \in \mathcal{E}$, $A \cap \overline{e}$ is closed.
\end{enumerate}
\end{df}

\begin{prop}
If a cell decomposition $\mathcal{E}$ satisfies the Characteristic Maps axiom, then for every $n$-cell $e \in \mathcal{E}$ we have $\overline{e} = \Phi_e(D^n)$. Therefore $\overline{e}$ is compact and $\overline{e} - e = \Phi_e(S^{n-1}) \subseteq X^{n-1}$.
\end{prop}

\begin{proof}
\pf
\step{1}{$e \subseteq \Phi_e(D^n) \subseteq \overline{e}$}
\begin{proof}
	\pf
	\begin{align*}
	e & = \Phi_e((D^n)^\circ) \\
	& \subseteq \Phi_e(D^n) \\
	& = \Phi_e(\overline{(D^n)^\circ}) \\
	& \subseteq \overline{\Phi_e((D^n)^\circ)} \\
	& = \overline{e}
	\end{align*}
\end{proof}
\step{2}{$\Phi_e(D^n)$ is compact.}
\begin{proof}
	\pf\ Because $D^n$ is compact.
\end{proof}
\step{3}{$\Phi_e(D^n)$ is closed.}
\step{4}{$\Phi_e(D^n) = \overline{e}$}
\qed
\end{proof}

\chapter{Topological Groups}

\begin{df}[Topological Group]
A \emph{topological group} is a group $G$ with a topology such that the function $G^2 \rightarrow G$ that maps $(x,y)$ to $x\inv{y}$ is continuous.
\end{df}

\begin{ex}
$GL(n,\mathbb{R})$ and $GL(n,\mathbb{C})$ are topological groups.
\end{ex}

\begin{prop}
Any subgroup of a topological group is a topological group under the subspace topology.
\end{prop}

\begin{df}[Homogeneous Space]
A \emph{homogeneous space} is a topological space of the form $G/H$, where $G$ is a topological group and $H$ is a normal subgroup of $G$, under the quotient topology.
\end{df}

\begin{prop}
Let $G$ be a topological group and $H$ a normal subgroup of $G$. Then $G/H$ is Hausdorff if and only if $H$ is closed.
\end{prop}

\begin{proof}
\pf\ See Bourbaki, N., General Topology. III.12 \qed
\end{proof}

\section{Continuous Actions}

\begin{df}[Continuous Action]
Let $G$ be a topological group and $X$ a topological space. A \emph{continuous action} of $G$ on $X$ is a continuous function $\cdot : G \times X \rightarrow X$ such that:
\begin{itemize}
\item $\forall x \in X. ex = x$
\item $\forall g,h \in G. \forall x \in X. g(hx) = (gh)x$
\end{itemize}

A \emph{$G$-space} consists of a topological space $X$ and a continuous action of $G$ on $X$.
\end{df}

\begin{df}[Orbit]
Let $X$ be a $G$-space and $x \in X$. The \emph{orbit} of $x$ is $\{ gx : g \in G \}$.

The \emph{orbit space} $X / G$ is the set of all orbits under the quotient topology.
\end{df}

\begin{prop}
Define an action of $SO(2)$ on $S^2$ by 
\[ g(x_1, x_2, x_3) = (g(x_1, x_2), x_3) \enspace . \] Then $S^2 / SO(2) \cong [-1,1]$.
\end{prop}

\begin{proof}
\pf
\step{1}{\pflet{$f_3 : S^2 / SO(2) \rightarrow [-1,1]$ be the function induced by $\pi_3 : S^2 \rightarrow [-1,1]$}}
\step{2}{$f_3$ is bijective.}
\step{3}{$S^2 / SO(2)$ is compact.}
\begin{proof}
	\pf\ It is the continuous image of $S^2$ which is compact.
\end{proof}
\step{4}{$[-1,1]$ is Hausdorff.}
\step{5}{$f_3$ is a homeomorphism.}
\qed
\end{proof}

\begin{df}[Stabilizer]
Let $X$ be a $G$-space and $x \in X$. The \emph{stabilizer} of $x$ is $G_x := \{ g \in G : gx = x \}$.
\end{df}

\begin{prop}
The function that maps $gG_x$ to $gx$ is a continuous bijection from $G / G_x$ to $Gx$.
\end{prop}

\begin{proof}
\pf
\step{1}{If $gG_x = hG_x$ then $gx = hx$.}
\begin{proof}
	\step{a}{\assume{$gG_x = hG_x$}}
	\step{b}{$\inv{g}h \in G_x$}
	\step{c}{$\inv{g}h x = x$}
	\step{d}{$gx = hx$}
\end{proof}
\step{2}{If $gx = hx$ then $gG_x = hG_x$.}
\begin{proof}
	\pf\ Similar.
\end{proof}
\step{3}{The function is continuous.}
\begin{proof}
	\pf\ Proposition \ref{prop:map_from_quotient_continuous}.
\end{proof}
\qed
\end{proof}

\chapter{Topological Vector Spaces}

\begin{df}[Topological Vector Space]
Let $K$ be either $\mathbb{R}$ or $\mathbb{C}$. A \emph{topological vector space} over $K$ consists of a 	vector space $E$ over $K$ and a topology on $E$ such that:
\begin{itemize}
\item Substraction is a continuous function $E^2 \rightarrow E$
\item Multiplication is a continuous function $K \times E \rightarrow E$
\end{itemize}
\end{df}

\begin{prop}
Every topological vector space is a topological group under addition.
\end{prop}

\begin{proof}
\pf\ Immediate from the definition. \qed
\end{proof}

\begin{thm}
The usual topology on a finite dimensional vector space over $K$ is the only one that makes it into a Hausdorff topological vector space.
\end{thm}

\begin{proof}
\pf\ See Bourbaki. Elements de Mathematique, Livre V: Espaces Vectoriels Topologiques, Th. 2, p. 18 \qed
\end{proof}

\begin{prop}
Let $E$ be a topological vector space and $E_0$ a subspace of $E$. Then $\overline{E_0}$ is a subspace of $E$.
\end{prop}

\begin{df}
Let $E$ be a topological vector space. The topological space \emph{associated} with $E$ is $E / \overline{\{0\}}$.
\end{df}

\section{Cauchy Sequences}

\begin{df}[Cauchy Sequence]
Let $E$ be a topological vector space. A sequence $(x_n)$ in $E$ is a \emph{Cauchy sequence} iff, for every neighbourhood $U$ of 0, there exists $n_0$ such that $\forall m,n \geq n_0. x_n - x_m \in U$.
\end{df}

\begin{df}[Complete Topological Vector Space]
A topological vector space is \emph{complete} iff every Cauchy sequence converges.
\end{df}

\section{Seminorms}

\begin{df}[Seminorm]
Let $E$ be a vector space over $K$. A \emph{seminorm} on $E$ is a function $\|\ \| : E \rightarrow \mathbb{R}$ such that:
\begin{enumerate}
\item $\forall x \in E. \| x \| \geq 0$
\item $\forall \alpha \in K. \forall x \in E. \| \alpha x \| = |\alpha| \|x\|$
\item \emph{Triangle Inequality} $\forall x,y : \in E. \| x + y \| \leq \| x \| + \| y \|$
\end{enumerate}
\end{df}

\begin{ex}
The function that maps $(x_1, \ldots, x_n)$ to $|x_i|$ is a seminorm on $\mathbb{R}^n$.
\end{ex}

\begin{df}
Let $E$ be a vector space over $K$.
Let $\Lambda$ be a set of seminorms on $E$. The topology \emph{generated} by $\Lambda$ is the topology generated by the subbasis consisting of all sets of the form $B_\epsilon^\lambda(x) = \{ y \in E : \lambda(y-x) < \epsilon \}$ for $\epsilon > 0$, $\lambda \in \Lambda$ and $x \in E$.
\end{df}

\begin{prop}
$E$ is a topological vector space under this topology. It is Hausdorff iff, for all $x \in E$, if $\forall \lambda \in \Lambda. \lambda(x) = 0$ then $x = 0$.
\end{prop}

\section{Fr\'{e}chet Spaces}

\begin{df}[Pre-Fr\'{e}chet Space]
A \emph{pre-Fr\'{e}chet space} is a Hausdorff topological vector space whose topology is generated by a countable set of seminorms.
\end{df}

\begin{prop}
Let $E$ be a pre-Fr\'{e}chet space whose topology is generated by the family of seminorms $\{ \|\ \|_n : n \in \mathbb{Z}^+ \}$. Then
\[ d(x,y) = \sum_{n=1}^\infty \frac{1}{2^n} \frac{\|x-y\|_n}{1 + \|x-y\|_n} \]
is a metric that induces the same topology. The two definitions of Cauchy sequence agree.
\end{prop}

\begin{df}[Fr\'{e}chet Space]
A \emph{Fr\'{e}chet space} is a complete pre-Fr\'{e}chet space.
\end{df}

\section{Normed Spaces}

\begin{df}[Normed Space]
Let $E$ be a vector space over $K$. A \emph{norm} on $E$ is a function $\|\ \| : E \rightarrow \mathbb{R}$ is a seminorm such that, $\forall x \in E. \| x \| = 0 \Leftrightarrow x = 0$.

A \emph{normed space} consists of a vector space with a norm.
\end{df}

\begin{prop}
If $E$ is a normed space then $d(x,y) = \| x - y \|$ is a metric on $E$ that makes $E$ into a topological vector space. The two definitions of Cauchy sequence agree on $E$.
\end{prop}

\begin{df}[$p$-norm]
For any $p \geq 1$, the \emph{$p$-norm} on $\mathbb{R}^n$ is defined by
\[ \| \vec{x} \|_p := \left( \sum_{i=1}^n |x_i|^p \right)^{\frac{1}{p}} \enspace . \]

We prove this is a norm.
\end{df}

\begin{proof}
\pf
\step{1}{For all $\vec{x} \in \mathbb{R}^n$ we have $\| \vec{x} \|_p \geq 0$}
\begin{proof}
	\pf\ Immediate from definition.
\end{proof}
\step{2}{For all $\alpha \in \mathbb{R}$ and $\vec{x} \in \mathbb{R}^n$ we have $\| \alpha \vec{x} \|_p = |\alpha| \| \vec{x} \|_p$}
\begin{proof}
	\pf
	\begin{align*}
		\| \alpha (x_1, \ldots, x_n) \|
		& = \| (\alpha x_1, \ldots, \alpha x_n) \| \\
		& = \left( \sum_{i=1}^n (\alpha x_i)^p \right)^{\frac{1}{p}} \\
		& = \left( |\alpha|^p \sum_{i=1}^n x_i^p \right)^{\frac{1}{p}} \\
		& = |\alpha| \left( \sum_{i=1}^n x_i^p \right)^{\frac{1}{p}} \\
		& = |\alpha| \|\vec{x}\|_p
	\end{align*}
\end{proof}
\step{3}{The triangle inequality holds.}
\begin{proof}
	\pf
	\begin{align*}
		\| \vec{x} + \vec{y} \|_p^p
		& = \sum_{i=1}^n |x_i + y_i|^p \\
		& = \sum_{i=1}^n |x_i + y_i| |x_i + y_i|^{p-1} \\
		& \leq \sum_{i=1}^n (|x_i| + |y_i|) |x_i + y_i|^{p-1} \\
		& = \sum_{i=1}^n |x_i| |x_i + y_i|^{p-1} + \sum_{i=1}^n |y_i| |x_i + y_i|^{p-1} \\
		& \leq \left( \sum_{i=1}^n |x_i|^p \right)^{\frac{1}{p}} \left( \sum_{i=1}^n |x_i + y_i|^p \right)^{\frac{p-1}{p}} + \left( \sum_{i=1}^n |y_i|^p \right)^{\frac{1}{p}} \left( \sum_{i=1}^n |x_i + y_i|^p \right)^{\frac{p-1}{p}} & (\text{H\"{o}lder's Inequality}) \\
		& = ( \|\vec{x}\|_p + \|\vec{y}\|_p) \| \vec{x} + \vec{y} \|^{p-1}
	\end{align*}
	Assuming w.l.o.g. $\| \vec{x} + \vec{y} \|^{p-1} \neq 0$ (using \stepref{4}) we have $\| \vec{x} + \vec{y} \|_p \leq \| \vec{x} \|_p + \| \vec{y} \|_p$. 	
\end{proof}
\step{4}{For any $\vec{x} \in \mathbb{R}^n$, we have $\| \vec{x} \| = 0$ iff $\vec{x} = \vec{0}$.}
\begin{proof}
	\pf $\sum_{i=1}^n x_i^p = 0$ iff $x_1 = \cdots = x_n  = 0$.
\end{proof}
\qed
\end{proof}

\begin{prop}
The $p$-norm on $\mathbb{R}^n$ induces the product topology.
\end{prop}

\begin{proof}
\pf
\step{0}{\pflet{$d$ be the metric induced by the $p$-norm and $\rho$ the square metric on $\mathbb{R}^n$.}}
\step{1}{The metric topology is finer than the product topology.}
\begin{proof}
	\step{a}{\pflet{$\vec{x} \in \mathbb{R}^n$ and $\epsilon > 0$}} 
	\step{b}{\pflet{$\delta = \epsilon / n^{\frac{1}{p}}$}\prove{$B_\rho(\vec{x}, \delta) \subseteq B_d(\vec{x}, \epsilon)$}}
	\step{c}{\pflet{$\vec{y} \in B_\rho(\vec{x}, \delta)$}}
	\step{d}{$\forall i. |x_i - y_i| < \delta$}
	\step{e}{$d(\vec{x}, \vec{y}) < \epsilon$}
	\begin{proof}
		\pf
		\begin{align*}
			d(\vec{x}, \vec{y}) & = \left( \sum_{i=1}^n |x_i - y_i|^p \right)^{\frac{1}{p}} \\
			& < \left( \sum_{i=1}^n \delta^p \right)^{\frac{1}{p}} & (\text{\stepref{d}}) \\
			& = n^{\frac{1}{p}} \delta \\
			& = \epsilon & (\text{\stepref{b}})
		\end{align*}
	\end{proof}
\end{proof}
\step{2}{The product topology is finer than the metric topology.}
\begin{proof}
	\step{a}{\pflet{$\vec{x} \in \mathbb{R}^n$ and $\epsilon > 0$}}
	\step{c}{\pflet{$\vec{y} \in B_d(\vec{x}, \epsilon)$}}
	\step{d}{$d(\vec{x}, \vec{y}) < \epsilon$}
	\step{e}{$\sum_{i=1}^n |x_i - y_i|^p < \epsilon^p$}
	\step{f}{$\forall i. |x_i - y_i|^p < \epsilon^p$}
	\step{g}{$\forall i. |x_i - y_i| < \epsilon$}
	\step{h}{$\rho(\vec{x},\vec{y}) < \epsilon$}
\end{proof}
\qed
\end{proof}

\begin{df}[Sup-norm]
The \emph{sup-norm} on $\mathbb{R}^n$ is defined by
\[ \| (x_1, \ldots, x_n) \|_\infty := \max(|x_1|, \ldots, |x_n|) \enspace . \]
\end{df}

\begin{prop}
The 2-norm on $\mathbb{R}^n$ induces the standard metric.
\end{prop}

\begin{proof}
\pf\ Immediate from definitions. \qed
\end{proof}

\begin{df}
For $p \geq 1$, the normed space $l_p$ is the set of all sequences $(x_n)$ in $\mathbb{R}$ such that $\sum_{n=1}^\infty x_n^p$ converges, under
\[ \| (x_n) \|_p := \left( \sum_{i=1}^\infty |x_i|^p \right)^{\frac{1}{p}} \enspace . \]
\end{df}

\begin{prop}
The spaces $l_p$ for $p \geq 1$ are all homeomorphic.
\end{prop}

\begin{proof}
\pf\ See Kadets, Mikhail Iosifovich. 1967. Proof of the topological equivalence of all separable
infinite-dimensional banach spaces. Functional Analysis and Its Applications 1 (1): 53–62.
http://dx.doi.org/10.1007/BF01075865.
\end{proof}

\begin{prop}
The metric topology on $l_2$ is strictly finer than the uniform topology.
\end{prop}

\begin{proof}
\pf
\step{o}{\pflet{$d$ be the metric induced by the $l^2$-norm and $\overline{\rho}$ the uniform topology.}}
\step{1}{The metric topology is finer than the uniform topology.}
\begin{proof}
	\step{a}{\pflet{$x \in l_2$}}
	\step{b}{\pflet{$\epsilon > 0$}}
	\step{c}{\pflet{$\delta = \epsilon / 2$}}
	\step{d}{\pflet{$y \in B_d(x, \delta)$}}
	\step{e}{$\sum_{n=0}^\infty (x_n - y_n)^2 < \delta^2$}
	\step{f}{$\forall n. (x_n - y_n)^2 < \delta^2$}
	\step{g}{$\forall n. |x_n - y_n| < \delta$}
	\step{h}{$\forall n. \overline{d}(x_n,y_n) < \delta$}
	\step{i}{$\overline{\rho}(x,y) \leq \delta$}
	\step{j}{$\overline{\rho}(x,y) < \epsilon$}
	\step{k}{$y \in B_{\overline{\rho}}(x, \epsilon)$}
\end{proof}
\step{2}{The metric topology is not the same as the uniform topology.}
\begin{proof}
	\step{a}{\assume{for a contradiction $B_d(0,1)$ is open in the uniform topology.}}
	\step{b}{\pick\ $\epsilon > 0$ such that $B_{\overline{\rho}}(0, \epsilon) \subseteq B_d(0,1)$}
	\step{c}{\pick\ an integer $N$ such that $1/N < \epsilon^2/4$}
	\step{d}{\pflet{$(x_n)$ be the sequence with $x_n = \epsilon / 2$ for $n < N$ and $x_n = 0$ for $n \geq N$}}
	\step{e}{$(x_n) \in l_2$}
	\step{f}{$(x_n) \in B_{\overline{\rho}}(0, \epsilon)$}
	\begin{proof}
		\pf\ Since $\overline{\rho}((x_n),0) = \epsilon / 2$.
	\end{proof}
	\step{g}{$d((x_n),0) > 1$}
	\begin{proof}
		\pf
		\begin{align*}
			d((x_n),0)^2 & = \sum_{n=0}^\infty x_n^2 \\
			& = N \epsilon^2 / 4 \\
			& > 1 
		\end{align*}
	\end{proof}
\end{proof}
\qed
\end{proof}

\begin{prop}
The metric topology on $l_2$ is strictly coarser than the box topology.
\end{prop}

\begin{proof}
\pf
\step{1}{The box topology is finer than the metric topology.}
\begin{proof}
	\step{a}{\pflet{$(x_n) \in l_2$ and $\epsilon > 0$.}}
	\step{b}{\pflet{$(y_n) \in B((x_n), \epsilon)$}}
	\step{c}{\pick\ a sequence of real numbers $(\delta_n)$ such that $\sum_{n=0}^\infty \delta_n^2 < (\epsilon - d((x_n),(y_n)))^2$}
	\step{d}{\pflet{$U = \prod_n (y_n - \delta_n, y_n + \delta_n)$} \prove{$U \subseteq B((x_n),\epsilon)$}}
	\step{e}{\pflet{$(z_n) \in U$}}
	\step{f}{$d((z_n),(y_n)) < \epsilon - d((x_n),(y_n))$}
	\begin{proof}
		\pf
		\begin{align*}
			d((z_n),(y_n))^2 & = \sum_{n=0}^\infty (z_n - y_n)^2 \\
			& < \sum_{n=0}^\infty \delta_n^2 \\
			& < (\epsilon - d((x_n),(y_n)))^2
		\end{align*}
	\end{proof}
	\step{g}{$d((z_n),(x_n)) < \epsilon$}
\end{proof}
\step{2}{The box topology is not equal to the metric topology.}
\begin{proof}
	\step{a}{\pflet{$U = \prod_n (-1/n,1/n)$}}
	\step{b}{\assume{for a contradiction $U$ is open in the metric topology.}}
	\step{c}{\pick\ $\epsilon > 0$ such that $B(0,\epsilon) \subseteq U$}
	\step{d}{\pick\ $N$ such that $1/N < \epsilon/2$.}
	\step{e}{\pflet{$(x_n)$ be the sequence with $x_N = \epsilon/2$ and $x_n = 0$ for all other $n$.}}
	\step{f}{$d((x_n),0) = \epsilon / 2$}
	\step{g}{$(x_n) \notin U$}
\end{proof}
\qed
\end{proof}

\begin{prop}
The $l^2$-topology on $\mathbb{R}^\infty$ is strictly finer than the uniform topology.
\end{prop}

\begin{proof}
\pf
	\step{a}{\assume{for a contradiction $B_d(0,1) \cap \mathbb{R}^\infty$ is open in the uniform topology.}}
	\step{b}{\pick\ $\epsilon > 0$ such that $B_{\overline{\rho}}(0, \epsilon) \cap \mathbb{R}^\infty \subseteq B_d(0,1) \cap \mathbb{R}^\infty$}
	\step{c}{\pick\ an integer $N$ such that $1/N < \epsilon^2/4$}
	\step{d}{\pflet{$(x_n)$ be the sequence with $x_n = \epsilon / 2$ for $n < N$ and $x_n = 0$ for $n \geq N$}}
	\step{e}{$(x_n) \in \mathbb{R}^\infty$}
	\step{f}{$(x_n) \in B_{\overline{\rho}}(0, \epsilon)$}
	\begin{proof}
		\pf\ Since $\overline{\rho}((x_n),0) = \epsilon / 2$.
	\end{proof}
	\step{g}{$d((x_n),0) > 1$}
	\begin{proof}
		\pf
		\begin{align*}
			d((x_n),0)^2 & = \sum_{n=0}^\infty x_n^2 \\
			& = N \epsilon^2 / 4 \\
			& > 1 
		\end{align*}
	\end{proof}
\qed
\end{proof}

\begin{prop}
The $l^2$-topology on $\mathbb{R}^\infty$ is strictly coarser than the box topology.
\end{prop}

\begin{proof}
	\step{a}{\pflet{$U = \prod_n (-1/n,1/n) \cap \mathbb{R}^\infty$}}
	\step{b}{\assume{for a contradiction $U$ is open in the metric topology.}}
	\step{c}{\pick\ $\epsilon > 0$ such that $B(0,\epsilon) \cap \mathbb{R}^\infty \subseteq U \cap \mathbb{R}^\infty$}
	\step{d}{\pick\ $N$ such that $1/N < \epsilon/2$.}
	\step{e}{\pflet{$(x_n)$ be the sequence with $x_N = \epsilon/2$ and $x_n = 0$ for all other $n$.}}
	\step{f}{$d((x_n),0) = \epsilon / 2$}
	\step{g}{$(x_n) \notin U$}
	\qed
\end{proof}

\begin{prop}
The $l^2$-topology on the Hilbert cube the same as the product topology.
\end{prop}

\begin{proof}
\pf
\step{1}{For every $(x_n) \in H$ and $\epsilon > 0$, there exists a neighbourhood $U$ of $(x_n)$ in the product topology such that $U \subseteq B((x_n),\epsilon)$.}
\begin{proof}
	\step{a}{\pflet{$(x_n) \in H$}}
	\step{b}{\pflet{$\epsilon > 0$}}
	\step{c}{\pick\ $N$ such that $\sum_{i=N+1}^\infty 1/i^2 < \epsilon^2 / 2$}
	\step{d}{\pflet{$B' = (\prod_{i=0}^N (x_i - \epsilon / \sqrt{2N}, x_i + \epsilon / \sqrt{2N}) \times \prod_{i=N+1}^\infty [0, 1/(i+1)]) \cap H$} \prove{$B' \subseteq B((x_n), \epsilon)$}}
	\step{e}{\pflet{$(y_n) \in B'$}}
	\step{f}{$d((x_n),(y_n)) < \epsilon$}
	\begin{proof}
		\pf
		\begin{align*}
			d((x_n),(y_n))^2 & = \sum_{i=0}^\infty |x_n - y_n|^2 \\
			& < \sum_{i=0}^N \epsilon^2 / 2N + \sum_{i=N+1}^\infty 1/(i+1) 1/(i+1)^2 \\
			& < \epsilon^2 / 2 + \epsilon^2 / 2 \\
			& = \epsilon^2
		\end{align*}
	\end{proof}
\end{proof}
\step{2}{The product topology is finer than the $l^2$-topology.}
\begin{proof}
	\step{a}{\pflet{$(x_n) \in H$ and $\epsilon > 0$} \prove{$B((x_n), \epsilon) \cap H$ is open in the product topology.}}
	\step{b}{\pflet{$(y_n) \in B((x_n), \epsilon)$}}
	\step{c}{\pick\ a neighbourhood $U$ of $(y_n)$ in the product topology such that $U \subseteq B((y_n), \epsilon - d((x_n),(y_n)))$}
	\step{d}{$U \subseteq B((x_n), \epsilon)$}
\end{proof}
\qed
\end{proof}

\begin{df}
Let $l_\infty$ be the set of all bounded sequences in $\mathbb{R}$ under
\[ \| (x_n) \| := \sup_n |x_n| \]
\end{df}

\begin{prop}
For all $p \geq 1$ we have $l_p$ is not homeomorphic to $l_\infty$.
\end{prop}

%TODO

\begin{prop}
Let $\|\ \|$ be a seminorm on the vector space $E$. Then $\|\ \|$ defines a norm on $E / \overline{\{0\}}$.
\end{prop}

\begin{prop}
Let $E$ and $F$ be normed spaces. Any continuous linear map $E \rightarrow F$ is uniformly continuous.
\end{prop}

\begin{df}
For $p \geq 1$. let $\mathcal{L}^p(\mathbb{R}^n)$ be the vector space of all Lebesgue-measurable functions $f : \mathbb{R}^n \rightarrow \mathbb{R}$ such that $|f|^p$ is Lebesgue-integrable. Then
\[ \| f \|_p := \sqrt{p}{\int_{\mathbb{R}^n} |f(x)|^p dx} \]
defines a seminorm on $\mathcal{L}^p(\mathbb{R}^n)$. Let
\[ L^p(\mathbb{R}^n) := \mathcal{L}^p(\mathbb{R}^n) / \overline{\{0\}} \enspace . \]
\end{df}

\section{Inner Product Spaces}

\begin{df}[Inner Product]
Given $\vec{x}, \vec{y} \in \mathbb{R}^n$, define
\[ \vec{x} \cdot \vec{y} = x_1 y_1 + \cdots + x_n y_n \enspace . \]
\end{df}

\begin{prop}
\[ \vec{x} \cdot (\vec{y} + \vec{z}) = \vec{x} \cdot \vec{y} + \vec{x} \cdot \vec{z} \]
\end{prop}

\begin{proof}
\pf
\begin{align*}
	\vec{x} \cdot (\vec{y} + \vec{z}) & = x_1 (y_1 + z_1) + \cdots + x_n (y_n + z_n) \\
	& = x_1 y_1 + x_1 z_1 + \cdots + x_n y_n + x_n z_n \\
	& = \vec{x} \cdot \vec{y} + \vec{x} \cdot \vec{z} & \qed
\end{align*}
\end{proof}

\begin{prop}
\label{prop:dot_leq_norms}
For all $\vec{x}, \vec{y} \in \mathbb{R}^n$ we have
\[ | \vec{x} \cdot \vec{y} | \leq \| \vec{x} \| \| \vec{y} \| \enspace . \]
\end{prop}

\begin{proof}
\pf
\step{1}{\assume{w.l.o.g. $\vec{x} \neq \vec{0} \neq \vec{y}$}}
\step{2}{\pflet{$a = 1 / \|x\|$}}
\step{3}{\pflet{$b = 1 / \|y\|$}}
\step{4}{$\| a\vec{x} + b\vec{y} \| \geq 0$}
\step{5}{$a^2 \| \vec{x}\|^2 + 2 ab \vec{x} \cdot \vec{y} + b^2 \| \vec{y}\|^2 \geq 0$}
\step{6}{$ab \vec{x} \cdot \vec{y} \geq -1$}
\step{7}{$\| a \vec{x} - b \vec{y} \| \geq 0$}
\step{8}{$ab \vec{x} \cdot \vec{y} \leq 1$}
\step{9}{$|\vec{x} \cdot \vec{y}| \leq 1/ab$}
\qed
\end{proof}

\begin{prop}
Let $(x_n)$, $(y_n)$ be sequences of real numbers. If $\sum_{n=0}^\infty x_n^2$ and $\sum_{n=0}^\infty y_n^2$ converge then $\sum_{n=0}^\infty |x_n y_n|$ converges.
\end{prop}

\begin{proof}
\pf
\begin{align*}
\sum_{n=0}^N |x_n y_n| & \leq \sqrt{\sum_{n=0}^N x_n^2 \sum_{n=0}^N y_n^2} & (\text{Proposition \ref{prop:dot_leq_norms}}) \\
& \leq \sqrt{\sum_{n=0}^\infty x_n^2 \sum_{n=0}^\infty y_n^2} & \qed
\end{align*}
\end{proof}


\begin{prop}
If $E$ is an inner product space then $\| x \| = \sqrt{\langle x,x \rangle}$ is a norm on $E$.
\end{prop}

\section{Banach Spaces}

\begin{df}[Banach Space]
A \emph{Banach space} is a complete normed space.
\end{df}

\begin{ex}
For any topological space $X$, the set $C(X)$ of bounded continuous functions $X \rightarrow \mathbb{R}$ is a Banach space under $\| f \| = \sup_{x \in X} |f(x)|$.
\end{ex}

\begin{prop}
The completion of a normed space is a Banach space.
\end{prop}

\begin{prop}
Let $E$ and $F$ be normed spaces. Let $f : E \rightarrow F$ be a continuous linear map. Then the extension to the completions $\hat{E} \rightarrow \hat{F}$ is linear.
\end{prop}

\begin{prop}
$L^p(\mathbb{R}^n)$ is a Banach space.
\end{prop}

\begin{prop}
$C(\mathbb{R})$ is first countable but not second countable.
\end{prop}

\begin{proof}
\pf\ For every sequence of 0s and 1s $s = (s_n)$, let $f_s$ be a continuous bounded function whose value at $n$ is $s_n$. Then the set of all $f_s$ is an uncountable discrete set in $C(\mathbb{R})$. Hence $C(\mathbb{R})$ is not second countable.

It is first countable because it is metrizable. \qed
\end{proof}

\section{Hilbert Spaces}

\begin{df}[Hilbert Space]
A \emph{Hilbert space} is a complete inner product space.
\end{df}

\begin{ex}
The set of \emph{square-integrable functions} is the set of Lebesgue integrable functions $[-\pi,\pi] \rightarrow \mathbb{R}$ quotiented by: $f \sim g$ iff $\{ x \in [-\pi,\pi] : f(x) \neq g(x) \}$ has measure 0. This is a Hilbert space under
\[ \langle f,g \rangle = \frac{1}{\pi} \int_{- \pi}{\pi} f(x) g(x) dx \enspace . \]
\end{ex}

\begin{prop}
The completion of an inner product space is a Hilbert space.
\end{prop}

An infinite dimensional Hilbert space with the weak topology is not first countable.

\section{Locally Convex Spaces}

\begin{df}[Locally Convex Space]
A topological vector space is \emph{locally convex} iff every neighbourhood of 0 includes a convex neighbourhood of 0.
\end{df}

\begin{prop}
A topological vector space is locally convex if and only if its topology is generated by a set of seminorms.
\end{prop}

\begin{proof}
\pf\ See K\"{o}the, G. Topological Vector Spaces 1. Section 18. \qed
\end{proof}

\begin{prop}
A locally convex topological vector space is a pre-Fr\'{e}chet space if and only if it is metrizable.
\end{prop}

\begin{proof}
\pf\ See K\"{o}the, G. Topological Vector Spaces 1. Section 18. \qed
\end{proof}

\begin{ex}
Let $E$ be an infinite dimensional Hilbert space. Let $E'$ be the same vector space under the \emph{weak topology}, the coarsest topology such that every continuous linear map $E \rightarrow \mathbb{R}$ is continuous as a map $E' \rightarrow \mathbb{R}$. Then $E$ is locally convex Hausdorff but not metrizable.

Proof: See Dieudonne, J. A., Treatise on Analysis, Vol. II, New York and London: Academic Press, 1970, p. 76.
\end{ex}

\begin{df}[Thom Space]
Let $E$ be a vector bundle with a Riemannian metric, $DE = \{ x \in E : \| x \| \leq 1 \}$ its disc bundle and $SE := \{ v \in E : \| v \| = 1 \}$ its sphere bundle. The \emph{Thom space} of $E$ is the quotient space $DE / SE$.
\end{df}

\end{document}
