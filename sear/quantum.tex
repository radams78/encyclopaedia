\chapter{The Wave Function}

\section{The Schr\"{o}dinger Equation}

\begin{df}[Planck's constant]
\emph{Planck's constant} is
\[ h = 6.62607015 \times 10^{-3} \mathrm{Js} \enspace . \]
\end{df}

\begin{df}[Reduced Planck's constant]
The \emph{reduced Planck's constant} is
\[ \hbar = h / 2 \pi \enspace . \]
\end{df}

Consider a particle of mass $m$ moving in one dimension under a force given by a potential energy function $V(x,t) : \mathbb{R} \times [0,+\infty) \rightarrow \mathbb{R}$. Associated with the particle is a \emph{wave function} $\Psi(x,t) : \mathbb{R} \times [0,+\infty) \rightarrow \mathbb{C}$ that is differentiable in $t$, twice differentiable in $x$, satisfies the \emph{(time-dependent) Schr\"{o}dinger equation}: for all $x$ and $t$,

\[ i \hbar \frac{\partial \Psi}{\partial t} = - \frac{\hbar^2}{2 m} \frac{\partial^2 \Psi}{\partial x^2} + V(x,t) \Psi(x,t) \]

and satisfies

\[ \int_{- \infty}^\infty |\Psi(x,0)|^2 dx = 1 \enspace . \]

\begin{prop}
\label{prop:differentiate_psi_squared}
\[ \frac{\partial}{\partial t} |\Psi(x,t)|^2 = \frac{i \hbar}{2m} \frac{\partial}{\partial x} \left(
\Psi^*(x,t) \frac{\partial \Psi}{\partial x} - \frac{\partial \Psi^*}{\partial x} \Psi(x,t) \right) \]
\end{prop}

\begin{proof}
\pf
\step{1}{\[ \frac{\partial \Psi}{\partial t} = \frac{i \hbar}{2m} \frac{\partial^2 \Psi}{\partial x^2} - \frac{i}{\hbar} V(x,t) \Psi(x,t) \]}
\begin{proof}
	\pf\ Schr\"{o}dinger equation.
\end{proof}
\step{2}{\[ \frac{\partial \Psi^*}{\partial t} = - \frac{i \hbar}{2m} \frac{\partial^2 \Psi^*}{\partial x^2} + \frac{i}{\hbar} V(x,t) \Psi(x,t)^* \]}
\begin{proof}
	\pf\ Taking complex conjugates in \stepref{1}.
\end{proof}
\step{3}{\[ \frac{\partial}{\partial t} |\Psi(x,t)|^2 = \frac{i \hbar}{2m} \frac{\partial}{\partial x} \left(
\Psi^*(x,t) \frac{\partial \Psi}{\partial x} - \frac{\partial \Psi^*}{\partial x} \Psi(x,t) \right) \]}
\begin{proof}
	\pf
	\begin{align*}
		\frac{\partial}{\partial t} |\Psi(x,t)|^2
		& = \frac{\partial}{\partial t} (\Psi(x,t)^* \Psi(x,t)) \\
		& = \Psi^* \frac{\partial \Psi}{\partial t} + \frac{\partial \Psi^*}{\partial t} \Psi \\
		& = \frac{i \hbar}{2m} \left( \Psi^* \frac{\partial^2 \Psi}{\partial x^2} - \frac{\partial^2 \Psi^*}{\partial x^2} \Psi \right) & (\text{\stepref{1}, \stepref{2}}) \\
		& = \frac{i \hbar}{2m} \frac{\partial}{\partial x} \left(
\Psi^*(x,t) \frac{\partial \Psi}{\partial x} - \frac{\partial \Psi^*}{\partial x} \Psi(x,t) \right)
	\end{align*}
\end{proof}
\qed
\end{proof}

\begin{prop}
For all $t \in [0, + \infty)$ we have
\[ \int_{- \infty}^\infty |\Psi(x,t)|^2 dx = 1 \enspace . \]
\end{prop}

\begin{proof}
\pf
\step{3}{\[ \frac{d}{dt} \int_{-\infty}^\infty |\Psi(x,t)|^2 dx = 0 \]}
\begin{proof}
\pf
\begin{align*}
\frac{d}{dt} \int_{-\infty}^\infty |\Psi(x,t)|^2 dx
& = \int_{-\infty}^\infty \frac{\partial}{\partial t} |\Psi(x,t)|^2 dx \\
& = \frac{i \hbar}{2m} \left[ \Psi^* \frac{\partial \Psi}{\partial x} - \frac{\partial \Psi^*}{\partial x} \Psi \right]_{- \infty}^\infty & (\text{Proposition \ref{prop:differentiate_psi_squared}}) \\
& = 0 & (\Psi \rightarrow 0 \text{ as } x \rightarrow \pm \infty)
\end{align*}
\step{4}{$\int_{-\infty}^\infty |\Psi(x,t)|^2 dx$ is constant.}
\end{proof}
\qed
\end{proof}

\section{Statistical Interpretation}

Born's statistical interpretation of the wave function:

The \emph{position} on the particle at time $t$ is a random variable $x$ with probability density function $|\Psi(x,t)|^2$ at time $t$.

\begin{prop}
The expected value of position is
\[ \langle x \rangle = \int_{- \infty}^{+ \infty} x |\Psi(x,t)|^2 dx = \int_{-\infty}^{+ \infty} \Psi(x,t)^* x \Psi(x,t) dx \]
\end{prop}

\begin{proof}
\pf\ Immediate from definitions. \qed
\end{proof}

\section{Momentum}

Associated with any \emph{observable} quantity $Q$ is an operator
\[ \hat{Q} : \mathcal{C}(\mathbb{R}, \mathbb{C}) \rightarrow \mathcal{C}(\mathbb{R}, \mathbb{C}) \enspace .\]
%TODO What is the space of the functions?

The \emph{expected value} of $Q$ at time $t$ is then
\[ \langle Q \rangle = \int_{-\infty}^{+ \infty} \Psi(x,t)^* \hat{Q}(\lambda x. \Psi(x,t)) dx \enspace . \]

Position $x$ is represented by the operator $\hat{x}$, multiplication by $x$.

Momentum $p$ is represented by the operator $\hat{p} = - i \hbar \frac{\partial}{\partial x}$.

Kinetic energy $T$ is represented by
\[ \hat{T} = \frac{\hat{p}^2}{2m} = - \frac{\hbar^2}{2m} \frac{\partial^2}{\partial x^2} \enspace . \]

\begin{prop}[Ehrenfest's Theorem]
\begin{enumerate}
\item \[ \langle p \rangle = m \frac{d \langle x \rangle}{dt} \]
\item \[ \frac{d \langle p \rangle}{dt} = \left\langle - \frac{\partial V}{\partial x} \right\rangle \]
\end{enumerate}
\end{prop}

\begin{proof}
\pf
\begin{align*}
m \frac{d \langle x \rangle}{dt} & = m \frac{d}{dt} \int_{- \infty}^{+ \infty} x |\Psi^2| dx \\
& = m \int_{-\infty}^{+ \infty} x \frac{\partial}{\partial t} |\Psi^2| dx \\
& = \frac{i \hbar}{2} \int_{-\infty}^{+ \infty} x \frac{\partial}{\partial x} \left( \Psi^* \frac{\partial \Psi}{\partial x} - \frac{\partial \Psi^*}{\partial x} \Psi \right) dx & (\text{Proposition \ref{prop:differentiate_psi_squared}}) \\
& = - \frac{i \hbar}{2} \int_{-\infty}^{+ \infty} \left( \Psi^* \frac{\partial \Psi}{\partial x} - \frac{\partial \Psi^*}{\partial x} \Psi \right) dx \\
& + \left[ x \left( \Psi^* \frac{\partial \Psi}{\partial x} - \frac{\partial \Psi^*}{\partial x} \Psi \right) \right]_{- \infty}^{+ \infty} & (\text{integrating by parts}) \\
& = - \frac{i \hbar}{2} \int_{-\infty}^{+ \infty} \left( \Psi^* \frac{\partial \Psi}{\partial x} - \frac{\partial \Psi^*}{\partial x} \Psi \right) dx \\
& = - i \hbar \int_{- \infty}^{+ \infty} \Psi^* \frac{\partial \Psi}{\partial x} dx & (\text{integrating by parts}) \\
& = \langle p \rangle \\
\frac{d \langle p \rangle}{dt} & = - i \hbar \frac{d}{dt} \int_{-\infty}^{+ \infty} \Psi(x,t)^* \frac{\partial \Psi}{\partial x} dx \\
& = - i \hbar \int_{-\infty}^{+ \infty} \frac{\partial}{\partial t} \left( \Psi(x,t)^* \frac{\partial \Psi}{\partial x} \right) dx \\
& = - \i \hbar \int_{- \infty}^{+ \infty} \left( \Psi^* \frac{\partial^2 \Psi}{\partial t \partial x} + \frac{\partial \Psi^*}{\partial t} \frac{\partial \Psi}{\partial x} \right) dx \\
\frac{\partial \Psi}{\partial t} & = \frac{i \hbar}{2m} \frac{\partial^2 \Psi}{\partial x^2} - \frac{i}{\hbar} V \Psi & (\text{Schr\"{o}dinger equation}) \\
\therefore \frac{\partial^2 \Psi}{\partial x \partial t} & = \frac{i \hbar}{2m} \frac{\partial^3 \Psi} {\partial x^3} - \frac{i}{\hbar} \frac{\partial V}{\partial x} \Psi - \frac{i}{\hbar} V \frac{\partial \Psi}{\partial x} \\
\frac{\partial \Psi^*}{\partial t} & = - \frac{i \hbar}{2m} \frac{\partial^2 \Psi^*}{\partial x^2} + \frac{i}{\hbar} V \Psi^* \\
\therefore \Psi^* \frac{\partial^2 \Psi}{\partial t \partial x} + \frac{\partial \Psi^*}{\partial t} \frac{\partial \Psi}{\partial x}
& = \frac{i \hbar}{2m} \Psi^* \frac{\partial^3 \Psi}{\partial x^3} - \frac{i}{\hbar} \frac{\partial V}{\partial x} \Psi^* \Psi - \frac{i}{\hbar} V \Psi^* \frac{\partial \Psi}{\partial x} \\
& - \frac{i \hbar}{2m} \frac{\partial^2 \Psi^*}{\partial x^2} \frac{\partial \Psi}{\partial x} + \frac{i}{\hbar} V \Psi^* \frac{\partial \Psi}{\partial x} \\
\therefore \frac{d \langle p \rangle}{dt}
& = \frac{\hbar^2}{2m} \int_{-\infty}^{+ \infty} \left( \Psi^* \frac{\partial^3 \Psi}{\partial x^3} - \frac{\partial^2 \Psi^*}{\partial x^2} \frac{\partial \Psi}{\partial x} \right) dx \\
& + \int_{-\infty}^{+ \infty} \Psi^* \left(- \frac{\partial V}{\partial x} \right) \Psi dx \\
& = - \frac{\hbar^2}{2m} \int_{-\infty}^{+ \infty} \left( \frac{\partial \Psi^*}{\partial x} \frac{\partial^2 \Psi}{\partial x^3} + \frac{\partial^2 \Psi^*}{\partial x^2} \frac{\partial \Psi}{\partial x} \right) dx \\
& + \left[ \Psi^* \frac{\partial^2 \Psi}{\partial x^2} \right]_{- \infty}^{+ \infty} 
+ \left \langle - \frac{\partial V}{\partial x} \right\rangle \\ \\
& = - \frac{\hbar^2}{2m} \left[ \frac{\partial \Psi^*}{\partial x} \frac{\partial \Psi}{\partial x} \right]_{-\infty}^{+\infty} + \left\langle - \frac{\partial V}{\partial x} \right\rangle \\
& = \left\langle - \frac{\partial V}{\partial x} \right\rangle %TODO Justify this & \qed
\end{align*}
\end{proof}