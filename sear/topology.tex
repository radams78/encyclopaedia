
\chapter{Topology}

\section{Topological Spaces}

\begin{df}[Topological Space]
Let $X$ be a set and $\mathcal{O} \subseteq \mathcal{P} X$. Then we say $(X, \mathcal{O})$ is a \emph{topological space} iff:
\begin{itemize}
\item For any $\mathcal{U} \subseteq \mathcal{O}$ we have $\bigcup \mathcal{U} \in \mathcal{O}$.
\item For any $U, V \in \mathcal{O}$ we have $U \cap V \in \mathcal{O}$.
\item $X \in \mathcal{O}$
\end{itemize}
We call $\mathcal{O}$ the \emph{topology} of the toplogical space, and call its elements \emph{open} sets. We shall often write $X$ for the topological space $(X, \mathcal{O})$.
\end{df}

\begin{ex}[Discrete Topology]
For any set $X$, the power set $\mathcal{P} X$ is called the \emph{discrete} topology on $X$.
\end{ex}

\begin{ex}[Indiscrete Topology]
For any set $X$, the \emph{indiscrete} or \emph{trivial} topology on $X$ is  $\{ \emptyset, X \}$.
\end{ex}

\begin{ex}[Cofinite Topology]
For any set $X$, the \emph{cofinite} topology is $\mathcal{T} = \{\emptyset\} \cup \{ X - U : U \subseteq X \text{ is finite} \}$.

We prove this is a topology.
\end{ex}

\begin{df}[Cocountable Topology]
For any set $X$, the \emph{cocountable} topology is $\{ X - U : U \subseteq X \text{ is countable} \}$.
\end{df}

\begin{df}[Sierpi\'{n}ski Two-Point Space]
The \emph{Sierpi\'{n}ski two-point space} is $\{0,1\}$ under the topology $\{ \emptyset, \{1\}, \{0,1\} \}$.
\end{df}

\begin{prop}
Let $X$ be a topological space and $U \subseteq X$. Then $U$ is open if and only if, for all $x \in U$, there exists an open set $V$ such that $x \in V \subseteq U$.
\end{prop}

\begin{prop}
The intersection of a set of topologies on a set $X$ is a topology on $X$.
\end{prop}

\begin{df}[Closed Set]
Let $X$ be a topological space and $A \subseteq X$. Then $A$ is \emph{closed} iff $X - A$ is open.
\end{df}

\begin{prop}
A set $B$ is open if and only if $X - B$ is closed.
\end{prop}

\begin{prop}
Let $X$ be a set and $\mathcal{C} \subseteq \mathcal{P} X$. Then there exists a topology $\mathcal{O}$ on $X$ such that $\mathcal{C}$ is the set of closed sets if and only if:
\begin{itemize}
\item For any $\mathcal{D} \subseteq \mathcal{C}$ we have $\bigcap \mathcal{D} \in \mathcal{C}$
\item For any $C, D \in \mathcal{C}$ we have $C \cup D \in \mathcal{C}$.
\item $\emptyset \in \mathcal{C}$
\end{itemize}
In this case, $\mathcal{O}$ is unique and is given by $\mathcal{O} = \{ X - C : C \in \mathcal{C} \}$.
\end{prop}

\begin{thm}
Let $X$ be a set. Let $\mathcal{C} \subseteq \mathcal{P} X$. Then there exists a topology on $X$ such that $\mathcal{C}$ is the set of closed sets if and only if:
\begin{enumerate}
\item $\emptyset \in \mathcal{C}$
\item $\forall \mathcal{A} \subseteq \mathcal{C}. \bigcap \mathcal{A} \in \mathcal{C}$
\item $\forall C,D \in \mathcal{C}. C \cup D \in \mathcal{C}$
\end{enumerate}
In this case, the topology is unique, and is $\{ X - C : C \in \mathcal{C} \}$.
\end{thm}

\begin{proof}
\pf\ Straightforward.
\end{proof}

\begin{thm}
There are infinitely many primes.
\end{thm}

\begin{proof}
Furstenberg's proof:

\pf
\step{1}{For $a \in \mathbb{Z} - \{0\}$ and $b \in \mathbb{Z}$, \pflet{$S(a,b) := \{ a n + b : n \in \mathbb{N} \}$}}
\step{2}{\pflet{$\mathcal{T}$ be the topology generated by the basis $\{ S(a,b) : a \in \mathbb{Z} - \{0\}, b \in \mathbb{Z} \}$}}
\begin{proof}
	\step{a}{For every $n \in \mathbb{Z}$, there exist $a$, $b$ such that $n \in S(a,b)$.}
	\begin{proof}
		\pf\ $n \in S(n,0)$
	\end{proof}
	\step{b}{If $n \in S(a_1,b_1) \cap S(a_2,b_2)$ then there exist $a_3$, $b_3$ such that $n \in S(a_3,b_3) \subseteq S_(a_1,b_1) \cap S(a_2,b_2)$}
	\begin{proof}
		\step{i}{\pflet{$d = \mathrm{lcm}(a_1,a_2)$} \prove{$S(d,n) \subseteq S(a_1,b_1) \cap S(a_2,b_2)$}}
		\step{ii}{\pflet{$d = a_1 k = a_2 l$}}
		\step{iii}{\pflet{$n = a_1 c + b_1 = a_2 d + b_2$}}
		\step{iv}{\pflet{$z \in \mathbb{Z}$} \prove{$dz + n \in S(a_1,b_1) \cap S(a_2,b_2)$}}
		\step{v}{$dz+n \in S(a_1,b_1)$}
		\begin{proof}
			\pf
			\begin{align*}
				dz + n & = a_1 k z + a_1 c + b_1 \\
				& = a_1 (kz + c) + b_1
			\end{align*}
		\end{proof}
		\step{vi}{$dz+n \in S(a_2, b_2)$}
		\begin{proof}
			\pf\ Similar.
		\end{proof}
	\end{proof}
\end{proof}
\step{4}{For all $a \in \mathbb{Z} - \{0\}$ and $b \in \mathbb{Z}$ we have $S(a,b)$ is closed.}
\begin{proof}
	\step{a}{\pflet{$a \in \mathbb{Z} - \{0\}$ and $b \in \mathbb{Z}$}}
	\step{b}{\pflet{$n \in \mathbb{Z} - S(a,b)$}}
	\step{c}{$n \in S(a,n) \subseteq \mathbb{Z} - S(a,b)$}
	\begin{proof}
		\step{i}{\pflet{$x \in S(a,n)$}}
		\step{ii}{\assume{for a contradiction $x \in S(a,b)$}}
		\step{iii}{\pick\ $m$ such that $x = am+b$}
		\step{iv}{\pick\ $l$ such that $x = al + n$}
		\step{v}{$n = a(m-l) + b$}
		\step{vi}{$n \in S(a,b)$}
		\qedstep
		\begin{proof}
			\pf\ This contradicts \stepref{b}.
		\end{proof}
	\end{proof}
\end{proof}
\step{5}{\[ \mathbb{Z} - \{ 1, -1 \} = \bigcup_{p \text{ prime}} S(p,0) \]}
\begin{proof}
	\pf\ Since every integer except 1 and $-1$ is divisible by a prime.
\end{proof}
\step{6}{No nonempty finite set is open.}
\begin{proof}
	\step{a}{\pflet{$U$ be a nonempty open set}}
	\step{b}{\pick\ $n \in U$}
	\step{c}{There exist $a$, $b$ such that $n \in S(a,b) \subseteq U$}
	\step{d}{$U$ is infinite.}
\end{proof}
\step{7}{$\mathbb{Z} - \{ 1, -1 \}$ is not closed.}
\step{8}{$\bigcup_{p \text{ prime}} S(p,0)$ is not closed.}
\step{9}{The union of finitely many closed sets is closed.}
\step{10}{There are infinitely many primes.}
\qed
\end{proof}

\begin{prop}
In a discrete topological space, every set is closed.
\end{prop}

\begin{proof}
\pf\ Immediate from definitions. \qed
\end{proof}

\begin{prop}
In a linearly ordered set under the order topology, every closed interval and closed ray is closed.
\end{prop}

\begin{proof}
\pf
\step{1}{\pflet{$X$ be a linearly ordered set under the order topology.}}
\step{2}{Every closed interval in $X$ is closed.}
\begin{proof}
	\pf\ Since $X - [a,b] = (-\infty, a) \cup (b, +\infty)$.
\end{proof}
\step{3}{Every closed ray in $X$ is closed.}
\begin{proof}
	\pf\ Since $X - [a,+\infty) = (-\infty,a)$ and $X - (-\infty,a] = (a,+\infty)$.
\end{proof}
\qed
\end{proof}

\begin{prop}
Let $X$ be a topological space and $Y$ a subspace of $X$. Let $A \subseteq Y$. Then $A$ is closed in $Y$ if and only if there exists a closed set $B$ in $X$ such that $A = B \cap Y$.
\end{prop}

\begin{proof}
\pf
\begin{align*}
A \text{ is closed in } Y & \Leftrightarrow Y - A \text{ is open in } Y \\
& \Leftrightarrow \exists U \text{ open in } X. Y - A = U \cap Y \\
& \Leftrightarrow \exists C \text{ closed in } X. Y - A = Y - C \\
& \Leftrightarrow \exists C \text{ closed in } X. A = Y \cap C & \qed
\end{align*}
\end{proof}

\begin{prop}
Let $X$ be a topological space and $Y$ a subspace of $X$. Let $A \subseteq Y$. If $A$ is closed in $Y$ and $Y$ is closed in $X$ then $A$ is closed in $X$.
\end{prop}

\begin{proof}
\pf
\step{1}{\pick\ $C$ closed in $X$ such that $A = C \cap Y$.}
\step{2}{$A$ is closed in $X$.}
\begin{proof}
	\pf\ It is the intersection of two closed sets in $X$.
\end{proof}
\qed
\end{proof}

\begin{df}[Neighbourhood]
Let $X$ be a topological space, $Sx \in X$ and $U \subseteq X$. Then $U$ is a \emph{neighbourhood} of $x$, and $x$ is an \emph{interior} point of $U$, iff there exists an open set $V$ such that $x \in V \subseteq U$.
\end{df}

\begin{prop}
A set $B$ is open if and only if it is a neighbourhood of each of its points.
\end{prop}

\begin{prop}
Let $X$ be a set and $\mathcal{N} : X \rightarrow \mathcal{P} X$. Then there exists a topology $\mathcal{O}$ on $X$ such that, for all $x \in X$, we have $\mathcal{N}_x$ is the set of neighbourhoods of $x$, if and only if:
\begin{itemize}
\item For all $x \in X$ and $N \in \mathcal{N}_x$ we have $x \in N$
\item For all $x \in X$ we have $X \in \mathcal{N}_x$
\item For all $x \in X$, $N \in \mathcal{N}_x$ and $V \subseteq \mathcal{P} X$, if $N \subseteq V$ then $V \in \mathcal{N}_x$
\item For all $x \in X$ and $M, N \in \mathcal{N}_x$ we have $M \cap N \in \mathcal{N}_x$
\item For all $x \in X$ and $N \in \mathcal{N}_x$, there exists $M \in \mathcal{N}_x$ such that $M \subseteq N$ and $\forall y \in M. M \in \mathcal{N}_y$.
\end{itemize}
In this case, $\mathcal{O}$ is unique and is given by $\mathcal{O} = \{ U : \forall x \in U. U \in \mathcal{N}_x \}$.
\end{prop}

\begin{df}[Exterior Point]
Let $X$ be a topological space, $x \in X$ and $B \subseteq X$. Then $x$ is an \emph{exterior point} of $B$ iff $B - X$ is a neighbourhood of $x$.
\end{df}

\begin{df}[Boundary Point]
Let $X$ be a topological space, $x \in X$ and $B \subseteq X$. Then $x$ is a \emph{boundary point} of $B$ iff it is neither an interior point nor an exterior point of $B$.
\end{df}

\begin{df}[Interior]
Let $X$ be a topological space and $B \subseteq X$. The \emph{interior} of $B$, $B^\circ$, is the set of all interior points of $B$.
\end{df}

\begin{prop}
The interior of $B$ is the union of all the open sets included in $B$.
\end{prop}

\begin{df}[Closure]
Let $X$ be a topological space and $B \subseteq X$. The \emph{closure} of $B$, $\overline{B}$, is the set of all points that are not exterior points of $B$.
\end{df}

\begin{prop}
The closure of $B$ is the intersection of all the closed sets that include $B$.
\end{prop}

\begin{prop}
A set $B$ is open iff $X - B = \overline{X - B}$.
\end{prop}

\begin{prop}[Kuratowski Closure Axioms]
Let $X$ be a set and $\overline{\ } : \mathcal{P} X \rightarrow \mathcal{P} X$. Then there exists a topology $\mathcal{O}$ such that, for all $B \subseteq X$, $\overline{B}$ is the closure of $B$, if and only if:
\begin{itemize}
\item $\overline{\emptyset} = \emptyset$
\item For all $A \subseteq X$ we have $A \subseteq \overline{A}$
\item For all $A \subseteq X$ we have $\overline{\overline{A}} = \overline{A}$
\item For all $A, B \subseteq X$ we have $\overline{A \cup B} = \overline{A} \cup \overline{B}$
\end{itemize}
In this case, $\mathcal{O}$ is unique and is defined by $\mathcal{O} = \{ U : X - U = \overline{X - U} \}$.
\end{prop}

\begin{df}[Finer, Coarser]
Let $\mathcal{T}$ and $\mathcal{T}'$ be topologies on the set $X$. Then $\mathcal{T}$ is \emph{coarser}, \emph{smaller} or \emph{weaker} than $\mathcal{T}'$, or $\mathcal{T}'$ is \emph{finer}, \emph{larger} or \emph{weaker} than $\mathcal{T}$, iff $\mathcal{T} \subseteq \mathcal{T}'$.
\end{df}

\section{Bases}

\begin{df}[Basis]
Let $X$ be a topological space. A \emph{basis} for the topology on $X$ is a set of open sets $\mathcal{B}$ such that every open set is the union of a subset of $\mathcal{B}$. The elements of $\mathcal{B}$ are called \emph{basic open neighbourhoods} of their elements.
\end{df}

\begin{prop}
Let $X$ be a set. The set of all one-element subsets of $X$ is a basis for the discrete topology on $X$.
\end{prop}

\begin{prop}
Let $X$ be a topological space.
Let $\mathcal{B}$ be a basis for the topology on $X$.
Then the topology on $X$ is the coarsest topology that includes $\mathcal{B}$.
\end{prop}

\begin{prop}
Let $X$ and $Y$ be topological spaces. Let $\mathcal{B}$ be a basis for the topology on $X$ and $\mathcal{C}$ a basis for the topology on $Y$. Then
\[ \{ B \times C : B \in \mathcal{B}, C \in \mathcal{C} \} \]
is a basis for the product topology on $X \times Y$.
\end{prop}

\section{Order Topology}

\begin{df}[Order Topology]
Let $X$ be a linearly ordered set. The \emph{order topology} on $X$ is the topology generated by the open interval $(a,b)$ as well as the open rays $(a, + \infty)$ and $(-\infty, b)$ for $a,b \in X$.

The \emph{standard topology} on $\mathbb{R}$ is the order topology.
\end{df}

\begin{prop}
Let $X$ be a linearly ordered set. Then the order topology is generated by the basis consisting of:
\begin{itemize}
\item all open intervals $(a,b)$
\item all intervals of the form $[\bot, b)$ where $\bot$ is the least element of $X$, if any
\item all intervals of the form $(a, \top]$ where $\top$ is the greatest element of $X$, if any.
\end{itemize}
\end{prop}

\begin{prop}
Let $X$ be a linearly ordered set. The open rays in $X$ form a subbasis for the order topology.
\end{prop}

\begin{df}[Lower Limit Topology]
The \emph{lower limit topology}, \emph{Sorgenfrey topology}, \emph{uphill topology} or \emph{half-open topology} is the topology on $\mathbb{R}$ generated by the basis consisting of all half-open intervals $[a,b)$.

We write $\mathbb{R}_l$ for $\mathbb{R}$ under the lower limit topology.
\end{df}

\begin{df}[$K$-topology]
Let $K = \{ 1/n : n \in \mathbb{Z}_+ \}$. The \emph{$K$-topology} on $\mathbb{R}$ is the topology generated by the basis consisting of all open intervals $(a,b)$ and all sets of the form $(a,b) - K$.

We write $\mathbb{R}_K$ for $\mathbb{R}$ under the $K$ -topology.
\end{df}

\begin{prop}
Let $X$ be a linearly ordered set under the order topology. Let $Y \subseteq X$ be convex. Then the order topology on $Y$ is the same as the subspace topology.
\end{prop}

\begin{proof}
\pf
\step{1}{The order topology is coarser than the subspace topology.}
\begin{proof}
	\step{a}{For all $a \in Y$, the open ray $\{ y \in Y : a < y \}$ is open in the subspace topology.}
	\begin{proof}
		\pf\ It is $(a, +\infty) \cap Y$.
	\end{proof}
	\step{b}{For all $a \in Y$, the open ray $\{ y \in Y : y < a \}$ is open in the subspace topology.}
	\begin{proof}
		\pf\ It is $(-\infty, a) \cap Y$.
	\end{proof}
\end{proof}
\step{2}{The subspace topology is coarser than the order topology.}
\begin{proof}
	\step{b}{For all $a \in X$, the set $(-\infty, a) \cap Y$ is open in the order topology.}
	\begin{proof}
		\step{i}{\case{$a \in Y$}}
		\begin{proof}
			\pf\ Then $(-\infty, a) \cap Y = \{ y \in Y : y < a\}$ is an open ray in $Y$.
		\end{proof}
		\step{ii}{\case{$a$ is an upper bound for $Y$}}
		\begin{proof}
			\pf\ Then $(-\infty, a) \cap Y = Y$. 
		\end{proof}
		\step{iii}{\case{$a$ is a lower bound for $Y$}}
		\begin{proof}
			\pf\ Then $(-\infty, a) \cap Y = \emptyset$. 
		\end{proof}
		\qedstep
		\begin{proof}
			\pf\ These are the only three cases because $Y$ is convex.
		\end{proof}
	\end{proof}
	\step{c}{For all $a \in X$, the set $(a, +\infty) \cap Y$ is open in the order topology.}
	\begin{proof}
		\pf\ Similar.
	\end{proof}
\end{proof}
\qed
\end{proof}

\begin{ex}
We cannot remove the hypothesis that the set $Y$ is convex.

Let $X = \mathbb{R}$ and $Y = [0,1) \cup \{2\}$. Then $\{2\}$ is open in the subspace topology but not in the order topology on $Y$.
\end{ex}

\begin{prop}
Let $X$ be a topological space. Let $\mathcal{B}$ be a basis for the topology on $X$ and $U \subseteq X$. Then $U$ is open if and only if, for all $x \in U$, there exists $B \in \mathcal{B}$ such that $x \in B \subseteq U$.
\end{prop}

\begin{prop}
Let $X$ be a topological space and $\mathcal{B} \subseteq X$. Assume that, for every open set $U$ and element $x \in U$, there exists $B \in \mathcal{B}$ such that $x \in B \subseteq U$. Then $\mathcal{B}$ is a basis for the topology on $X$.
\end{prop}

\begin{prop}
Let $X$ be a topological space and $\mathcal{B} \subseteq \mathcal{P} X$. Then $\mathcal{B}$ is a basis for a topology on $X$ if and only if:
\begin{enumerate}
\item $\bigcup \mathcal{B} = X$
\item For all $A, B \in \mathcal{B}$ and $x \in A \cap B$, there exists $C \in \mathcal{B}$ such that $x \in C \subseteq A \cap B$.
\end{enumerate}
In this case, the topology is unique and is the set of all unions of subsets of $\mathcal{B}$. We call it the topology \emph{generated} by $\mathcal{B}$.
\end{prop}

\begin{prop}
Let $\mathcal{B}$ and $\mathcal{B}'$ be bases for the topologies $\mathcal{T}$ and $\mathcal{T}'$, respectively, on $X$. Then $\mathcal{T}'$ is finer than $\mathcal{T}$ if and only if, for every $B \in \mathcal{B}$ and $x \in B$, there exists $B' \in \mathcal{B}'$ such that $x \in B' \subseteq B$.
\end{prop}

\begin{cor}
The topologies of $\mathbb{R}_l$ and $\mathbb{R}_K$ are strictly finer than the standard topology on $\mathbb{R}$ but are not comparable to one another.
\end{cor}

\subsection{Subspaces}

\begin{prop}
\label{prop:basis_subspace}
Let $X$ be a topological space. Let $Y$ be a subspace of $X$. Let $\mathcal{B}$ be a basis for the topology on $X$. Then $\{ B \cap Y : B \in \mathcal{B} \}$ is a basis for the topology on $Y$.
\end{prop}

\begin{proof}
\pf
\step{1}{For all $B \in \mathcal{B}$ we have $B \cap Y$ is open in $Y$.}
\begin{proof}
	\pf\ Since $B$ is open in $X$.
\end{proof}
\step{2}{For any open set $V$ in $Y$ and $y \in V$, there exists $B \in \mathcal{B}$ such that $y \in B \cap Y \subseteq V$.}
\begin{proof}
	\step{a}{\pflet{$V$ be open in $Y$.}}
	\step{b}{\pflet{$y \in V$}}
	\step{c}{\pick\ $U$ open in $X$ such that $V = U \cap Y$.}
	\step{d}{\pick\ $B \in \mathcal{B}$ such that $y \in B \subseteq U$.}
	\step{e}{$y \in B \cap Y \subseteq V$}
\end{proof}
\qed
\end{proof}

\subsection{Product Topology}

\begin{prop}
Let $\{X_i\}_{i \in I}$ be a family of topological spaces. For all $i \in I$, let $\mathcal{B}_i$ be a basis for the topology on $X_i$. Then $\mathcal{B} = \left\{ \prod_{i \in I} B_i : \text{for finitely many $i \in I$ we have $B_i \in \mathcal{B}_i$, and $B_i = X_i$ for all other $i$} \right\}$ is a basis for the product topology on $\prod_{i \in I} X_i$.
\end{prop}

\begin{proof}
\pf
\step{1}{Every $B \in \mathcal{B}$ is open in the product topology.}
\begin{proof}
	\pf\ Since every element of $\mathcal{B}_i$ is open in $X_i$.
\end{proof}
\step{2}{For any open set $U$ in the product topology and $x \in U$, there exists $B \in \mathcal{B}$ such that $x \in B \subseteq U$.}
\begin{proof}
	\step{a}{\pflet{$U$ be a set open in the box topology.}}
	\step{b}{\pflet{$x \in U$}}
	\step{c}{\pick\ a family $\{U_i\}_{i \in I}$ where $U_i$ is open in $X_i$ for $i = i_1, \ldots, i_n$, and $U_i = X_i$ for all other $i$, such that $x \in \prod_{i \in I} U_i \subseteq U$}
	\step{d}{For $i = i_1, \ldots, i_n$, choose $B_i \in \mathcal{B}_i$ such that $x_i \in B_i \subseteq U_i$. Let $B_i = X_i$ for all other $i$.}
	\step{e}{$\prod_{i \in I} B_i \in \mathcal{B}$}
	\step{f}{$x \in \prod_{i \in I} B_i \subseteq \prod_{i \in I} U_i \subseteq U$}
\end{proof}
\qed
\end{proof}

\section{Subbases}

\begin{df}[Subbasis]
Let $X$ be a topological space. A \emph{subbasis} for the topology on $X$ is a set $\mathcal{S}$ of open sets such that every open set is a union of finite intersections of $\mathcal{S}$.
\end{df}

\begin{prop}
Let $X$ be a set and $\mathcal{S} \subseteq X$. Then $\mathcal{S}$ is a subbasis for a topology on $X$ if and only if $\bigcup \mathcal{S} = X$, in which case the topology is unique and is the set of all unions of finite intersections of elements of $\mathcal{S}$.
\end{prop}

\begin{prop}
Let $X$ be a topological space.
Let $\mathcal{S}$ be a subbasis for the topology on $X$.
Then the topology on $X$ is the coarsest topology that includes $\mathcal{S}$.
\end{prop}


\begin{prop}
Let $X$ and $Y$ be topological spaces. Then
\[ \mathcal{S} = \{ \inv{\pi_1}(U) : U \text{ is open in } X \} \cup \{ \inv{\pi_2}(V) : V \text{ is open in } Y \} \]
is a subbasis for the product topology on $X \times Y$.
\end{prop}

\begin{proof}
\pf
\step{1}{Every element of $\mathcal{S}$ is open.}
\begin{proof}
	\pf\ Since $\inv{\pi_1}(U) = U \times Y$ and $\inv{\pi_2}(V) = X \times V$.
\end{proof}
\step{2}{Every open set is a union of finite intersections of elements of $\mathcal{S}$.}
\begin{proof}
	\pf\ Since, for $U$ open in $X$ and $V$ open in $Y$, we have $U \times V = \inv{\pi_1}(U) \cap \inv{\pi_2}(V)$.
\end{proof}
\qed
\end{proof}

\begin{df}[Space with Basepoint]
A \emph{space with basepoint} is a pair $(X,x)$ where $X$ is a topological space and $x \in X$.
\end{df}

\section{Neighbourhood Bases}

\begin{df}[Neighbourhood Basis]
Let $X$ be a topological space and $x_0 \in X$. A \emph{neighbourhood basis} of $x_0$ is a set $\mathcal{U}$ of neighbourhoods of $x_0$ such that every neighbourhood of $x_0$ includes an element of $\mathcal{U}$.
\end{df}

\section{First Countable Spaces}

\begin{df}[First Countable]
A topological space is \emph{first countable} iff every point has a countable neighbourhood basis.
\end{df}

\begin{prop}
$\mathbb{R}_l$ is first countable.
\end{prop}

\begin{proof}
\pf\ For any $x \in \mathbb{R}$ we have $\{ [x, x + 1/n) : n \in \mathbb{Z}_+ \}$ is a countable local basis. \qed
\end{proof}

\begin{prop}
The ordered square is first countable.
\end{prop}

\begin{proof}
\pf
\step{1}{Every point $(a,b)$ with $0 < b < 1$ has a countable local basis.}
\begin{proof}
	\pf\ The set of all intervals $((a,q),(a,r))$ where $q$ and $r$ are rational and $0 \leq q < b < r \leq 1$ is a countable local basis.
\end{proof}
\step{2}{Every point $(a,0)$ has a countable local basis with $a > 0$.}
\begin{proof}
	\pf\ The set of all intervals $((q,0),(a,r))$ where $q$ and $r$ are rational with $0 \leq q < a$ and $0 < r \leq 1$ is a countable local basis.
\end{proof}
\step{3}{Every point $(a,1)$ has a countable local basis with $a < 1$.}
\begin{proof}
	\pf\ The set of all intervals $((a,q),(r,1))$ with $q$ and $r$ rational and $0 \leq q < 1$, $a < r \leq 1$ is a countable local basis.
\end{proof}
\step{4}{$(0,0)$ has a countable local basis.}
\begin{proof}
	\pf\ The set of all intervals $[(0,0),(0,r))$ with $r$ rational and $0 < r \leq 1$ is a countable local basis.
\end{proof}
\step{5}{$(1,1)$ has a countable local basis.}
\begin{proof}
	\pf\ The set of all intervals $((1,q),(1,1)]$ with $q$ rational and $0 \leq q < 1$ is a countable local basis.
\end{proof}
\qed
\end{proof}
\section{Second Countable Spaces}

\begin{df}[Second Countable]
A topological space is \emph{second countable} iff it has a countable basis.
\end{df}

Every second countable space is first countable.

A subspace of a first countable space is first countable.

A subspace of a second countable space is second countable.

$\mathbb{R}^n$ is second countable.

An uncountable discrete space is first countable but not second countable.

\begin{prop}
Let $\{ X_\lambda \}_{\lambda \in \Lambda}$ be a family of topological spaces such that no $X_\lambda$ is indiscrete. If $\Lambda$ is uncountable, then $\prod_{\lambda \in \Lambda} X_\lambda$ is not first countable.
\end{prop}

\begin{proof}
\pf
\step{1}{For all $\lambda \in \Lambda$, \pick\ $U_\lambda$ open in $X_\lambda$ such that $\emptyset \neq U_\lambda \neq X_\lambda$.}
\step{2}{For all $\lambda \in \Lambda$, \pick\ $x_\lambda \in U_\lambda$.}
\step{3}{\assume{for a contradiction $B$ is a countable neighbourhood basis for $(x_\lambda)_{\lambda \in \Lambda}$.}}
\step{4}{\pick\ $\lambda \in \Lambda$ such that, for all $U \in B$, we have $\pi_\lambda(U) = X_\lambda$}
\step{5}{There is no $U \in \lambda$ such that $U \subseteq \pi_\lambda^{-1}(U_\lambda)$}
\qedstep
\begin{proof}
\pf\ This is a contradiction.
\end{proof}
\qed
\end{proof}

\begin{prop}
The long line cannot be embedded in $\mathbb{R}^n$ for any $n$.
\end{prop}

\begin{proof}
\pf\ Since the long line is not second countable but $\mathbb{R}^n$ is. \qed
\end{proof}

\section{Interior}

\begin{df}[Interior]
Let $X$ be a topological space. Let $A \subseteq X$. The \emph{interior} of $A$, $A^\circ$, is the union of all the open sets included in $A$.
\end{df}

\section{Closure}

\begin{df}[Closure]
Let $X$ be a topological space. Let $A \subseteq X$. The \emph{closure} of $A$, $\overline{A}$, is the intersection of all the closed sets that include $A$.
\end{df}

\begin{prop}
\label{prop:closure}
Let $X$ be a topological space, $A \subseteq X$ and $x \in X$. Then $x \in \overline{A}$ if and only if every open set that contains $x$ intersects $A$.
\end{prop}

\begin{proof}
\pf
\begin{align*}
x \in \overline{A} & \Leftrightarrow \text{for every closed set $C$, if $A \subseteq C$ then $x \in C$} \\
& \Leftrightarrow \text{for every open set $U$, if $A \subseteq X - U$ then $x \in X - U$} \\
& \Leftrightarrow \text{for every open set $U$, if $A \cap U = \emptyset$ then $x \notin U$} \\
& \Leftrightarrow \text{for every open set $U$, if $x \in U$ then $A$ intersects $U$} & \qed
\end{align*}
\end{proof}

\begin{prop}
\label{prop:closure_monotone}
Let $X$ be a topological space. Let $A \subseteq B \subseteq X$. Then $\overline{A} \subseteq \overline{B}$.
\end{prop}

\begin{proof}
\pf\ Since every closed set that includes $B$ is a closed set that includes $A$. \qed
\end{proof}

\begin{prop}
Let $X$ be a topological space. Let $A,B \subseteq X$. Then $\overline{A \cup B} = \overline{A} \cup \overline{B}$.
\end{prop}

\begin{proof}
\pf
\step{1}{$\overline{A \cup B} \subseteq \overline{A} \cup \overline{B}$}
\begin{proof}
\pf\ Since $\overline{A} \cup \overline{B}$ is a closed set that includes $A \cup B$.
\end{proof}
\step{2}{$\overline{A} \cup \overline{B} \subseteq \overline{A \cup B}$}
\begin{proof}
	\pf\ Since $\overline{A} \subseteq \overline{A \cup B}$ and $\overline{B} \subseteq \overline{A \cup B}$ by Proposition \ref{prop:closure_monotone}.
\end{proof}
\qed
\end{proof}

\begin{prop}
Let $X$ be a topological space. Let $\mathcal{A} \subseteq \mathcal{P} X$. Then
\[ \bigcup \{ \overline{A} : A \in \mathcal{A} \} \subseteq \overline{\bigcup \mathcal{A}} \enspace . \]
\end{prop}

\begin{proof}
\pf
For all $A \in \mathcal{A}$ we have $\overline{A} \subseteq \overline{\bigcup \mathcal{A}}$ by Proposition \ref{prop:closure_monotone}. \qed
\end{proof}

\begin{ex}
The converse does not always hold. In $\mathbb{R}$, let $\mathcal{A} = \{ \{ x \} : 0 < x < 1 \}$. Then $\bigcup \{ \overline{A} : A \in \mathcal{A} \} = (0,1)$ but $\overline{\bigcup \mathcal{A}} = [0,1]$.
\end{ex}

\begin{prop}
Let $X$ be a topological space. Let $\mathcal{A} \subseteq \mathcal{P} X$. Then $\overline{\bigcap \mathcal{A}} \subseteq \bigcap \{ \overline{A} : A \in \mathcal{A} \}$.
\end{prop}

\begin{proof}
\pf\ Since $\overline{\bigcap \mathcal{A}} \subseteq \overline{A}$ for all $A \in \mathcal{A}$ by Proposition \ref{prop:closure_monotone}. \qed
\end{proof}

\begin{ex}
The converse does not always hold. In $\mathbb{R}$, if $A$ is the set of all rational numbers and $B$ is the set of all irrational numbers then $\bigcap{A \cap B} = \emptyset$ but $\bigcap{A} \cap \bigcap{B} = \mathbb{R}$.
\end{ex}

\subsection{Bases}

\begin{prop}
\label{prop:closure_basis}
Let $X$ be a topological space, $A \subseteq X$ and $x \in X$. Let $\mathcal{B}$ be a basis for the topology on $X$. Then $x \in \overline{A}$ if and only if, for all $B \in \mathcal{B}$, if $x \in B$ then $B$ intersects $A$.
\end{prop}

\begin{proof}
\pf
\step{1}{If $x \in \overline{A}$ then, for all $B \in \mathcal{B}$, if $x \in B$ then $B$ intersects $A$.}
\begin{proof}
	\pf\ Proposition \ref{prop:closure} since every element of $\mathcal{B}$ is open.
\end{proof}
\step{2}{If, for all $B \in \mathcal{B}$, if $x \in B$ then $B$ intersects $A$, then $x \in \overline{A}$.}
\begin{proof}
	\step{a}{\assume{For all $B \in \mathcal{B}$, if $x \in B$ then $B$ intersects $A$.}}
	\step{b}{\pflet{$U$ be an open set that contains $x$.}}
	\step{c}{\pick\ $B \in \mathcal{B}$ such that $x \in B \subseteq U$.}
	\step{d}{$B$ intersects $A$.}
	\begin{proof}
		\pf\ \stepref{a}
	\end{proof}
	\step{e}{$U$ intersects $A$.}
\end{proof}
\qed
\end{proof}

\subsection{Subspaces}

\begin{prop}
Let $X$ be a topological space. Let $Y$ be a subspace of $X$. Let $A \subseteq Y$. Let $\overline{A}$ be the closure of $A$ in $X$. Then the closure of $A$ in $Y$ is $\overline{A} \cap Y$.
\end{prop}

\begin{proof}
\pf
\step{1}{$\overline{A} \cap Y$ is the closed in $Y$.}
\begin{proof}
	\pf\ Since $\overline{A}$ is closed in $X$.
\end{proof}
\step{2}{For any closed set $B$ in $Y$, if $A \subseteq B$ then $\overline{A} \cap Y \subseteq B$.}
\begin{proof}
	\step{a}{\pflet{$B$ be closed in $Y$.}}
	\step{b}{\assume{$A \subseteq B$}}
	\step{c}{\pick\ $C$ closed in $X$ such that $B = C \cap Y$.}
	\step{d}{$A \subseteq C$}
	\step{e}{$\overline{A} \subseteq C$}
	\step{f}{$\overline{A} \cap Y \subseteq B$}
\end{proof}
\qed
\end{proof}

\subsection{Product Topology}

\begin{prop}
Let $X$ and $Y$ be topological spaces. Let $A \subseteq X$ and $B \subseteq Y$. Then $\overline{A \times B} = \overline{A} \times \overline{B}$.
\end{prop}

\begin{proof}
\pf
\step{1}{$\overline{A \times B} \subseteq \overline{A} \times \overline{B}$}
\begin{proof}
	\pf\ Since $\overline{A} \times \overline{B}$ is a closed set that includes $A \times B$ by Proposition \ref{prop:closed_product}.
\end{proof}
\step{2}{$\overline{A} \times \overline{B} \subseteq \overline{A \times B}$}
\begin{proof}
	\step{a}{\pflet{$x \in \overline{A}$ and $y \in \overline{B}$.}}
	\step{b}{\pflet{$U$ be an open set that contains $(x,y)$.}}
	\step{c}{\pick\ open sets $V$ in $X$ and $W$ in $Y$ such that $(x,y) \in V \times W \subseteq U$.}
	\step{d}{$V$ intersects $A$ and $W$ intersects $B$.}
	\step{e}{$U$ intersects $A \times B$.}
\end{proof}
\qed
\end{proof}

\subsection{Interior}

\begin{prop}
Let $X$ be a topological space and $A \subseteq X$. Then
\[ X - A^\circ = \overline{X - A} \]
\end{prop}

\begin{proof}
\pf
\begin{align*}
X - A^\circ & = X - \bigcup \{ U \text{ open in } X : U \subseteq A \} \\
& = \bigcap \{ X - U : U \text{ open in } X, U \subseteq A \} & (\text{De Morgan's Law}) \\
& = \bigcap \{ C : C \text{ closed in } X, X - A \subseteq C \} \\
& = \overline{X-A} & \qed
\end{align*}
\end{proof}

\begin{prop}
Let $X$ be a topological space and $A \subseteq X$. Then
\[ X - \overline{A} = (X - A)^\circ \]
\end{prop}

\begin{proof}
\pf\ Dual. \qed
\end{proof}

\section{Boundary}

\begin{df}[Boundary]
Let $X$ be a topological space. Let $A \subseteq X$. The \emph{boundary} of $A$ is
\[ \partial A := \overline{A} \cap \overline{X - A} \enspace . \]
\end{df}

\begin{prop}
Let $X$ be a topological space. Let $A \subseteq X$. Then
\[ A^\circ \cap \partial A = \emptyset \enspace . \]
\end{prop}

\begin{proof}
\pf
\step{1}{$A^\circ \subseteq A$}
\step{2}{$X - A \subseteq X - A^\circ$}
\step{3}{$\overline{X - A} \subseteq X - A^\circ$}
\step{4}{$\partial A \subseteq X - A^\circ$}
\qed
\end{proof}

\begin{prop}
\label{prop:closure_boundary}
Let $X$ be a topological space. Let $A \subseteq X$. Then
\[ \overline{A} = A^\circ \cup \partial A \]
\end{prop}

\begin{proof}
\step{1}{$A^\circ \subseteq \overline{A}$}
\begin{proof}
	\pf\ Since $A^\circ \subseteq A \subseteq \overline{A}$.
\end{proof}
\step{2}{$\partial A \subseteq \overline{A}$}
\begin{proof}
	\pf\ Definition of $\partial A$.
\end{proof}
\step{3}{$\overline{A} \subseteq A^\circ \cup \partial A$}
\begin{proof}
	\step{a}{\pflet{$x \in \overline{A}$}}
	\step{b}{\assume{$x \notin A^\circ$} \prove{$x \in \partial A$}}
	\step{c}{$x \in \overline{X-A}$}
	\begin{proof}
		\pf\ Since $\overline{X-A} = X - A^\circ$.
	\end{proof}
	\step{d}{$x \in \partial A$}
	\begin{proof}
		\pf\ Since $\partial A = \overline{A} \cap \overline{X-A}$.
	\end{proof}
\end{proof}
\qed
\end{proof}

\begin{prop}
Let $X$ be a topological space. Let $A \subseteq X$. Then $\partial A = \emptyset$ if and only if $A$ is both open and closed.
\end{prop}

\begin{proof}
\pf
\step{1}{If $\partial A = \emptyset$ then $A$ is open and closed.}
\begin{proof}
	\step{a}{\assume{$\partial A = \emptyset$}}
	\step{b}{$\overline{A} = A^\circ$}
	\begin{proof}
		\pf\ Proposition \ref{prop:closure_boundary}.
	\end{proof}
	\step{c}{$\overline{A} = A = A^\circ$}
\end{proof}
\step{2}{If $A$ is open and closed then $\partial A = \emptyset$.}
\begin{proof}
	\pf\ If $A$ is open and closed then
	\begin{align*}
	\partial A & = \overline{A} \cap \overline{X-A} \\
	& = \overline{A} \cap (X - A^\circ) \\
	& = A \cap (X - A) \\
	& = \emptyset
	\end{align*}
\end{proof}
\qed
\end{proof}

\begin{prop}
Let $X$ be a topological space. Let $U \subseteq X$. Then $U$ is open if and only if $\partial U = \overline{U} - U$.
\end{prop}

\begin{proof}
\pf
\step{1}{If $U$ is open then $\partial U = \overline{U} - U$}
\begin{proof}
	\pf\ If $U$ is open then
	\begin{align*}
		\partial U & = \overline{U} \cap \overline{X-U} \\
		& = \overline{U} \cap (X - U^\circ) \\
		& = \overline{U} - U^\circ \\
		& = \overline{U} - U 
	\end{align*}
\end{proof}
\step{2}{If $\partial U = \overline{U} - U$ then $U$ is open.}
\begin{proof}
	\step{a}{\assume{$\partial U = \overline{U} - U$}}
	\step{b}{$\overline{U} - U^\circ = \overline{U} - U$}
	\step{c}{$U \subseteq U^\circ$}
	\step{d}{$U = U^\circ$}
\end{proof}
\qed
\end{proof}

\section{Limit Points}

\begin{df}[Limit Point]
Let $X$ be a topological space, $x \in X$ and $A \subseteq X$. Then $x$ is a \emph{limit point}, \emph{cluster point} or \emph{point of accumulation} of $A$ iff every neighbourhood of $x$ intersects $A - \{x\}$.
\end{df}

\begin{prop}
Let $X$ be a topological space. Let $A \subseteq X$. Let $A'$ be the set of limit points of $A$. Then
\[ \overline{A} = A \cup A' \]
\end{prop}

\begin{proof}
\pf
\step{1}{$\overline{A} \subseteq A \cup A'$}
\begin{proof}
	\step{a}{\pflet{$x \in \overline{A}$}}
	\step{b}{\assume{$x \notin A$} \prove{$x \in A'$}}
	\step{c}{\pflet{$U$ be a neighbourhood of $x$.}}
	\step{d}{\pick\ $y \in U \cap A$}
	\begin{proof}
		\pf\ Proposition \ref{prop:closure}.
	\end{proof}
	\step{e}{$y \neq x$}
\end{proof}
\step{2}{$A \subseteq \overline{A}$}
\begin{proof}
	\pf\ Immediate from the definition of $\overline{A}$.
\end{proof}
\step{3}{$A' \subseteq \overline{A}$}
\begin{proof}
	\pf\ From Proposition \ref{prop:closure}.
\end{proof}
\qed
\end{proof}

\begin{cor}
A set is closed if and only if it contains all its limit points.
\end{cor}

\section{Continuous Functions}

\begin{df}[Continuous]
Let $X$ and $Y$ be topological spaces. A function $f : X \rightarrow Y$ is \emph{continuous} iff, for every open set $V$ in $Y$, the inverse image $\inv{f}(V)$ is open in $X$.
\end{df}

\begin{prop}
The composite of two continuous functions is continuous.
\end{prop}

\begin{proof}
\pf
\step{1}{\pflet{$f : X \rightarrow Y$ and $g : Y \rightarrow Z$ be continuous.}}
\step{2}{\pflet{$U$ be open in $Z$.}}
\step{3}{$\inv{g}(U)$ is open in $Y$.}
\step{4}{$\inf{f}(\inv{g}(U))$ is open in $X$.}
\qed
\end{proof}

\begin{prop}
\begin{enumerate}
\item $\id{X}$ is continuous
\item If $f : X \rightarrow Y$ is continuous and $X_0 \subseteq X$ then $f \restriction X_0 : X_0 \rightarrow Y$ is continuous.
\item If $f : X + Y \rightarrow Z$, then $f$ is continuous iff $f \circ \kappa_1 : X \rightarrow Z$ and $f \circ \kappa_2 : Y \rightarrow Z$ are continuous.
\item If $f : Z \rightarrow X \times Y$, then $f$ is continuous iff $\pi_1 \circ f$ and $\pi_2 \circ f$ are continuous.
\end{enumerate}
\end{prop}

\begin{prop}
Let $X$ and $Y$ be topological spaces. Let $f : X \rightarrow Y$. Then the following are equivalent.
\begin{enumerate}
\item $f$ is continuous.
\item For all $A \subseteq X$ we have $f(\overline{A}) \subseteq \overline{f(A)}$.
\item For every closed $B$ in $Y$, we have $\inv{f}(B)$ is closed in $X$.
\end{enumerate}
\end{prop}

\begin{proof}
\pf
\step{1}{$1 \Rightarrow 2$}
\begin{proof}
	\step{a}{\assume{$f$ is continuous.}}
	\step{b}{\pflet{$A \subseteq X$}}
	\step{c}{\pflet{$x \in \overline{A}$} \prove{$f(x) \in \overline{f(A)}$}}
	\step{d}{\pflet{$V$ be a neighbourhood of $f(x)$.} \prove{$V$ intersects $f(A)$.}}
	\step{e}{$\inv{f}(V)$ is a neighbourhood of $x$.}
	\step{f}{\pick\ $y \in \inv{f}(V) \cap A$}
	\step{g}{$f(y) \in V \cap f(A)$}
\end{proof}
\step{2}{$2 \Rightarrow 3$}
\begin{proof}
	\step{a}{\assume{2}}
	\step{b}{\pflet{$B$ be closed in $Y$}}
	\step{c}{\pflet{$A = \inv{f}(B)$} \prove{$\overline{A} = A$}}
	\step{d}{$f(A) \subseteq B$}
	\step{e}{$\overline{A} \subseteq A$}
	\begin{proof}
		\step{i}{\pflet{$x \in \overline{A}$}}
		\step{ii}{$f(x) \in B$}
		\begin{proof}
			\pf
			\begin{align*}
				f(x) & \in f(\overline{A}) \\
				& \subseteq \overline{f(A)} & (\text{\stepref{a}}) \\
				& \subseteq \overline{B} & (\text{\stepref{d}}) \\
				& = B & (\text{\stepref{b}})
			\end{align*}
		\end{proof}
	\end{proof}
\end{proof}
\step{3}{$3 \Rightarrow 1$}
\begin{proof}
	\step{a}{\assume{3}}
	\step{b}{\pflet{$V$ be open in $Y$.}}
	\step{c}{$\inv{f}(Y - V)$ is closed in $X$.}
	\step{d}{$X - \inv{f}(V)$ is closed in $X$.}
	\step{e}{$\inv{f}(V)$ is open in $X$.}
\end{proof}
\qed
\end{proof}

\begin{prop}
Let $X$ and $Y$ be topological spaces. Any constant function $X \rightarrow Y$ is continuous.
\end{prop}

\begin{proof}
\pf
\step{1}{\pflet{$b \in Y$}}
\step{2}{\pflet{$f : X \rightarrow Y$ be the constant function with value $b$.}}
\step{3}{\pflet{$V \subseteq Y$ be open.}}
\step{4}{$\inv{f}(V)$ is either $\emptyset$ or $X$.}
\step{5}{$\inv{f}(V)$ is open.}
\qed
\end{proof}

\begin{prop}
\label{prop:continuous_basis}
Let $X$ and $Y$ be topological spaces. Let $f : X \rightarrow Y$. Let $\mathcal{B}$ be a basis for $Y$. Then $f$ is continuous if and only if, for all $B \in \mathcal{B}$, we have $\inv{f}(B)$ is open in $X$.
\end{prop}

\begin{proof}
\pf
\step{1}{If $f$ is continuous then, for all $B \in \mathcal{B}$, we have $\inv{f}(B)$ is open in $X$.}
\begin{proof}
	\pf\ Since every element of $\mathcal{B}$ is open in $Y$.
\end{proof}
\step{2}{If, for all $B \in \mathcal{B}$, we have $\inv{f}(B)$ is open in $X$, then $f$ is continuous.}
\begin{proof}
	\step{a}{\assume{For all $B \in \mathcal{B}$, we have $\inv{f}(B)$ is open in $X$.}}
	\step{b}{\pflet{$U$ be open in $Y$.}}
	\step{c}{\pflet{$x \in \inv{f}(U)$}}
	\step{d}{\pick\ $B \in \mathcal{B}$ such that $f(x) \in B \subseteq U$.}
	\step{e}{$x \in \inv{f}(B) \subseteq \inv{f}(U)$}
\end{proof}
\qed
\end{proof}

\begin{prop}
Let $X$ and $Y$ be topological spaces.
Let $f : X \rightarrow Y$. Let $\mathcal{S}$ be a subbasis for the topology on $Y$. Then $f$ is continuous if and only if, for all $V \in \mathcal{S}$, we have $\inv{f}(V)$ is open in $X$.
\end{prop}

\begin{proof}
\pf
\step{1}{If $f$ is continuous then, for all $V \in \mathcal{S}$, we have $\inv{f}(V)$ is open in $X$.}
\begin{proof}
	\pf\ Immediate from definitions.
\end{proof}
\step{2}{If, for all $V \in \mathcal{S}$, we have $\inv{f}(V)$ is open in $X$, then $f$ is continuous.}
\begin{proof}
	\step{a}{\assume{For all $V \in \mathcal{S}$, we have $\inv{f}(V)$ is open in $X$.}}
	\step{b}{For all $V_1, \ldots, V_n \in \mathcal{S}$ we have $\inv{f}(V_1 \cap \cdots \cap V_n)$ is open in $X$.}
	\begin{proof}
		\pf\ Since $\inv{f}(V_1 \cap \cdots \cap V_n) = \inv{f}(V_1) \cap \cdots \cap \inv{f}(V_n)$.
	\end{proof}
	\qedstep
	\begin{proof}
		\pf\ By Proposition \ref{prop:continuous_basis} since the set of all finite intersections of elements of $\mathcal{S}$ forms a basis for the topology on $Y$.
	\end{proof}
\end{proof}
\qed
\end{proof}

\begin{prop}
\label{prop:continuous_subbasis}
Let $f : \mathbb{R} \rightarrow \mathbb{R}$. Then $f$ is continuous if and only if, for all $x \in \mathbb{R}$ and $\epsilon > 0$, there exists $\delta > 0$ such that, for all $y \in \mathbb{R}$, if $|y-x| < \delta$ then $|f(y) - f(x)| < \epsilon$.
\end{prop}

\begin{proof}
\pf
\step{1}{If $f$ is continuous then, for all $x \in \mathbb{R}$ and $\epsilon > 0$, there exists $\delta > 0$ such that, for all $y \in \mathbb{R}$, if $|y-x| < \delta$ then $|f(y) - f(x)| < \epsilon$.}
\begin{proof}
	\step{a}{\assume{$f$ is continuous.}}
	\step{b}{\pflet{$x \in \mathbb{R}$}}
	\step{c}{\pflet{$\epsilon > 0$}}
	\step{d}{$f^{-1}((f(x) - \epsilon, f(x) + \epsilon))$ is open in $X$.}
	\step{e}{\pick\ $a$, $b$ such that $x \in (a,b) \subseteq f^{-1}((f(x) - \epsilon, f(x) + \epsilon))$.}
	\step{f}{\pflet{$\delta = \min(x - a, b - x)$}}
	\step{g}{\pflet{$y \in \mathbb{R}$}}
	\step{h}{\assume{$|y-x| < \delta$}}
	\step{i}{$y \in (a,b)$}
	\step{j}{$f(y) \in (f(x) - \epsilon, f(x) + \epsilon)$}
	\step{k}{$|f(y) - f(x)| < \epsilon$}
\end{proof}
\step{2}{If, for all $x \in \mathbb{R}$ and $\epsilon > 0$, there exists $\delta > 0$ such that, for all $y \in \mathbb{R}$, if $|y-x| < \delta$ then $|f(y) - f(x)| < \epsilon$, then $f$ is continuous.}
\begin{proof}
	\step{a}{\assume{For all $x \in \mathbb{R}$ and $\epsilon > 0$, there exists $\delta > 0$ such that, for all $y \in \mathbb{R}$, if $|y-x| < \delta$ then $|f(y) - f(x)| < \epsilon$.}}
	\step{b}{For all $a \in \mathbb{R}$ we have $\inv{f}((a, +\infty))$ is open.}
	\begin{proof}
		\step{i}{\pflet{$a \in \mathbb{R}$}}
		\step{ii}{\pflet{$x \in \inv{f}((a, +\infty))$}}
		\step{iii}{\pflet{$\epsilon = f(x) - a$}}
		\step{iv}{\pick\ $\delta > 0$ such that, for all $y \in \mathbb{R}$, if $|y-x| < \delta$ then $|f(y) - f(x)| < \epsilon$}
		\step{v}{$x \in (x - \delta, x + \delta) \subseteq \inv{f}((a,+\infty))$}
	\end{proof}
	\step{c}{For all $a \in \mathbb{R}$ we have $\inv{f}((-\infty, a))$ is open.}
	\begin{proof}
		\pf\ Similar.
	\end{proof}
	\qedstep
	\begin{proof}
		\pf\ Proposition \ref{prop:continuous_subbasis}.
	\end{proof}
\end{proof}
\qed
\end{proof}

\begin{df}[Continuity at a Point]
Let $X$ and $Y$ be topological spaces. Let $f : X \rightarrow Y$. Let $a \in X$. Then $f$ is \emph{continuous at $a$} iff, for every neighbourhood $V$ of $f(a)$, there exists a neighbourhood $U$ of $a$ such that $f(U) \subseteq V$.
\end{df}

\begin{prop}
Let $X$ and $Y$ be topological spaces. Let $f : X \rightarrow Y$. Then $f$ is continuous if and only if $f$ is continuous at every point in $X$.
\end{prop}

\begin{proof}
\step{4}{If $f$ is continuous then $f$ is continuous at every point in $X$.}
\begin{proof}
	\step{a}{\assume{$f$ is continuous.}}
	\step{b}{\pflet{$a \in X$}}
	\step{c}{\pflet{$V$ be a neighbourhood of $f(a)$}}
	\step{d}{\pflet{$U = \inv{f}(V)$}}
	\step{e}{$U$ is a neighbourhood of $a$.}
	\step{f}{$f(U) \subseteq V$}
\end{proof}
\step{5}{If $f$ is continuous at every point in $X$ then $f$ is continuous.}
\begin{proof}
	\step{a}{\assume{$f$ is continuous at every point in $X$.}}
	\step{b}{\pflet{$V$ be open in $Y$.}}
	\step{c}{\pflet{$x \in \inv{f}(V)$}}
	\step{d}{$V$ is a neighbourhood of $f(x)$}
	\step{e}{\pick\ a neighbourhood $U$ of $x$ such that $f(U) \subseteq V$}
	\step{f}{$x \in U \subseteq \inv{f}(V)$}
\end{proof}
\qed
\end{proof}

\begin{df}[Homeomorphism]
Let $X$ and $Y$ be topological spaces. A \emph{homeomorphism} between $X$ and $Y$ is a bijection $f : X \approx Y$ such that $f$ and $\inv{f}$ are continuous.
\end{df}

\begin{prop}
Let $X$ and $Y$ be topological spaces. Let $f : X \rightarrow Y$. Then $f$ is a homeomorphism iff $f$ is bijective and, for all $U \subseteq X$, we have $f(U)$ is open if and only if $U$ is open.
\end{prop}

\begin{proof}
\pf\ Immediate from definitions. \qed
\end{proof}

\begin{df}[Topological Property]
A property $P$ of topological spaces is a \emph{topological} property iff, for any topological spaces $X$ and $Y$, if $P[X]$ and $X \cong Y$ then $P[Y]$.
\end{df}

\begin{df}[Retraction]
Let $X$ be a topological space and $A$ a subspace of $X$. A continuous function $\rho : X \rightarrow A$ is a \emph{retraction} iff $\rho \restriction A = \id{A}$. We say $A$ is a \emph{retract} of $X$ iff there exists a retraction.
\end{df}

\begin{df}
Let $\mathbf{Top}$ be the category of small topological spaces and continuous functions.
\end{df}

\begin{prop}
$\emptyset$ is initial in $\mathbf{Top}$.
\end{prop}

\begin{prop}
$1$ is terminal in $\mathbf{Top}$.
\end{prop}

Forgetful functor $\mathbf{Top} \rightarrow \mathbf{Set}$.

Basepoint preserving continuous functor.

\begin{prop}
Let $(X, \mathcal{T})$ be a topological space. Let $S$ be the Sierpi\'{n}ski two-point space. Define $\Phi : \mathcal{T} \rightarrow \Top[X,S]$ by $\Phi(U)(x) = 1$ iff $x \in U$. Then $\Phi$ is a bijection.
\end{prop}

\begin{proof}
\pf
\step{1}{For all $U \in \mathcal{T}$ we have $\Phi(U)$ is continuous.}
\begin{proof}
	\step{a}{\pflet{$U \in \mathcal{T}$}}
	\step{b}{$\Phi(U)(\{1\})$ is open.}
	\begin{proof}
		\pf\ Since $\Phi(U)(\{1\}) = U$.
	\end{proof}
\end{proof}
\step{2}{$\Phi$ is injective.}
\begin{proof}
	\pf\ If $\Phi(U) = \Phi(V)$ then we have $\forall x (x \in U \Leftrightarrow \Phi(U)(x) = 1 \Leftrightarrow \Phi(V)(x) = 1 \Leftrightarrow x \in V)$.
\end{proof}
\step{3}{$\Phi$ is surjective.}
\begin{proof}
	\pf\ Given $f : X \rightarrow S$ continuous we have $\Phi(\inv{f}(1)) = f$.
\end{proof}
\qed
\end{proof}

\subsection{Order Topology}

\begin{prop}
Let $X$ and $Y$ be linearly ordered sets under the order topology. Let $f : X \rightarrow Y$ be strictly monotone and surjective. Then $f$ is a homeomorphism.
\end{prop}

\begin{proof}
\pf
\step{1}{$f$ is continuous.}
\begin{proof}
	\step{a}{For all $b \in Y$ we have $\inv{f}((b, +\infty))$ is open in $X$.}
	\begin{proof}
		\step{i}{\pflet{$b \in Y$}}
		\step{ii}{\pflet{$a$ be the element of $X$ such that $f(a) = b$.}}
		\step{iii}{$\inv{f}((b, +\infty)) = (a, + \infty)$}
	\end{proof}
	\step{b}{For all $b \in Y$ we have $\inv{f}((-\infty, b))$ is open in $X$.}
	\begin{proof}
		\pf\ Similar.
	\end{proof}
\end{proof}
\step{2}{$\inv{f}$ is continuous.}
\begin{proof}
	\pf\ Similar.
\end{proof}
\qed
\end{proof}

\begin{cor}
For $n$ a positive integer, the $n$th root function $\overline{\mathbb{R}_+} \rightarrow \overline{\mathbb{R}_+}$ is continuous.
\end{cor}

\subsection{Paths}

\begin{df}[Path]
A \emph{path} in a topological space $X$ is a continuous function $[0,1] \rightarrow X$.
\end{df}

\begin{df}[Constant Path]
Let $X$ be a topological space and $a \in X$. The \emph{constant} path at $a$ is the path $p : [0,1] \rightarrow X$ with $p(t) = a$ for all $t \in [0,1]$.
\end{df}

\begin{df}[Reverse Path]
Let $X$ be a topological space and $p : [0,1] \rightarrow X$. The \emph{reverse} of $p$ is the path $q : [0,1] \rightarrow X$ with $q(t) = p(1-t)$ for all $t \in [0,1]$.
\end{df}

\begin{df}[Concatenation]
Let $X$ be a topological space and $p, q : [0,1] \rightarrow X$ be paths in $X$ with $p(1) = q(0)$. The \emph{concatenation} of $p$ and $q$ is the path $r : [0,1] \rightarrow X$ with
\[ r(t) = \begin{cases}
p(2t) & \text{if } 0 \leq t \leq 1/2 \\
q(2t-1) & \text{if } 1/2 \leq t \leq 1
\end{cases} \]
\end{df}

\subsection{Loops}

\begin{df}[Loop]
A \emph{loop} in a topological space $X$ is a path $\alpha : [0,1] \rightarrow X$ such that $\alpha(0) = \alpha(1)$.
\end{df}

\section{Convergence}

\begin{df}[Convergence]
Let $X$ be a topological space. Let $(x_n)$ be a sequence in $X$. A point $a \in X$ is a \emph{limit} of the sequence iff, for every neighbourhood $U$ of $a$, there exists $n_0$ such that $\forall n \geq n_0. x_n \in U$.
\end{df}

\begin{prop}
\label{prop:continuous_converge}
If $f : X \rightarrow Y$ is continuous and $x_n \rightarrow l$ in $X$ then $f(x_n) \rightarrow f(l)$ in $Y$.
\end{prop}

%TODO

\begin{ex}
The converse does not hold.

Let $X$ be the set of all continuous functions $[0,1] \rightarrow [-1,1]$ under the product topology. Let $i : X \rightarrow L^2([0,1])$ be the inclusion.

If $f_n \rightarrow f$ then $i(f_n) \rightarrow i(f)$ --- Lebesgue convergence theorem.

We prove that $i$ is not continuous.

Assume for a contradiction $i$ is continuous. Choose a neighbourhood $K$ of 0 in $X$ such that $\forall \phi \in K _\epsilon. \int \phi^2 < 1/2$. Let $K = \prod_{\lambda \in [0,1]} U_\lambda$ where $U_\lambda = [-1,1]$ except for $\lambda = \lambda_1, \ldots, \lambda_n$. Let $\phi$ be the function that is 0 at $\lambda_1$, \ldots, $\lambda_n$ and 1 everywhere else. Then $\phi \in K$ but $\int \phi^2 = 1$.
\end{ex}

\begin{prop}
The converse does hold for first countable spaces. If $f : X \rightarrow Y$ where $X$ is first countable, and $Y$ is a topological space, and whenever $x_n \rightarrow x$ then $f(x_n) \rightarrow f(x)$, then $f$ is continuous.
\end{prop}

\begin{prop}
If $(s_n)$ is an increasing sequence of real numbers bounded above, then $(s_n)$ converges.
\end{prop}

\begin{proof}
\pf
\step{1}{\pflet{$s$ be the supremum of $\{ s_n : n \in \mathbb{N} \}$.} \prove{$s_n \rightarrow s$ as $n \rightarrow \infty$.}}
\step{2}{\pflet{$\epsilon > 0$}}
\step{3}{\pick\ $N$ such that $s_N > s - \epsilon$.}
\step{4}{$\forall n \geq N. s - \epsilon \leq s_n \leq s$}
\step{5}{$\forall n \geq N. |s_n - s| < \epsilon$}
\qed
\end{proof}

\subsection{Closure}

\begin{prop}
Let $X$ be a topological space. Let $A \subseteq X$. Let $(a_n)$ be a sequence in $A$ and $l \in X$. If $a_n \rightarrow l$ as $n \rightarrow \infty$, then $l \in \overline{A}$.
\end{prop}

\begin{proof}
\pf
\step{1}{\pflet{$U$ be a neighbourhood of $l$.}}
\step{2}{\pick\ $N$ such that $\forall n \in N. a_n \in U$}
\step{3}{$a_N \in A \cap U$}
\qed
\end{proof}

\subsection{Continuous Functions}

\begin{prop}
Let $X$ and $Y$ be topological spaces. Let $f : X \rightarrow Y$ be continuous. Let $x_n \rightarrow x$ as $n \rightarrow \infty$ in $X$. Then $f(x_n) \rightarrow f(x)$ as $n \rightarrow \infty$ in $Y$.
\end{prop}

\begin{proof}
\pf
\step{1}{\pflet{$V$ be a neighbourhood of $f(x)$.}}
\step{2}{\pick\ $N$ such that $\forall n \geq N. x_n \in \inv{f}(V)$}
\step{3}{$\forall n \geq N. f(x_n) \in V$}
\qed
\end{proof}

\subsection{Infinite Series}

\begin{df}[Series]
Let $(a_n)$ be a sequence of real numbers. We say that the infinite series $\sum_{n=0}^\infty a_n$ \emph{converges} to $s$, and write
\[ \sum_{n=0}^\infty a_n = s \]
iff $\sum_{n=0}^N a_n \rightarrow s$ as $N \rightarrow \infty$.
\end{df}

\section{Strong Continuity}

\begin{df}[Strong Continuity]
Let $X$ and $Y$ be topological spaces. Let $f : X \rightarrow Y$. Then $f$ is \emph{strongly continuous} iff, for every $V \subseteq Y$, we have $V$ is open in $Y$ if and only if $\inv{f}(V)$ is open in $X$.
\end{df}

\begin{prop}
Let $X$ and $Y$ be topological spaces. Let $f : X \rightarrow Y$. Then $f$ is strongly continuous if and only if, for all $C \subseteq Y$, we have $C$ is closed in $Y$ if and only if $\inv{f}(C)$ is closed in $X$.
\end{prop}

\begin{proof}
\pf
\begin{align*}
f \text{ is continuous} & \Leftrightarrow \forall V \subseteq Y ( V \text{ is open in } Y \Leftrightarrow \inv{f}(V) \text{ is open in } X) \\
& \Leftrightarrow \forall C \subseteq Y (Y - C \text{ is open in } Y \Leftrightarrow \inv{f}(Y - C) \text{ is open in } X) \\
& \Leftrightarrow \forall C \subseteq Y (C \text{ is closed in } Y \Leftrightarrow \inv{f}(C) \text{ is closed in } X) & \qed
\end{align*}
\end{proof}

\section{Subspaces}

\begin{df}[Subspace]
Let $X$ be a topological space, $Y$ a set, and $f : Y \rightarrow X$. The \emph{subspace topology} on $Y$ induced by $f$ is $\mathcal{T} = \{ \inv{i}(U) : U \text{ is open in } X \}$.

We prove this is a topology.
\end{df}

\begin{proof}
\pf
\step{1}{For all $\mathcal{U} \subseteq \mathcal{T}$ we have $\bigcup \mathcal{U} \in \mathcal{T}$}
\begin{proof}
	\pf\ Since $\bigcup \mathcal{U} = \inv{f}(\bigcup \{ V : \inv{f}(V) \in \mathcal{U}\})$.
\end{proof}
\step{2}{For all $U,V \in \mathcal{T}$ we have $U \cap V \in \mathcal{T}$}
\begin{proof}
	\pf\ Since $\inv{f}(U) \cap \inv{f}(V) = \inv{f}(U 
\cap V)$.
\end{proof}
\step{3}{$Y \in \mathcal{T}$}
\begin{proof}
	\pf\ Since $Y = \inv{f}(X)$.
\end{proof}
\qed
\end{proof}

\begin{prop}
Let $X$ be a topological space, $Y$ a set and $f : Y \rightarrow X$ a function. Then the subspace topology on $Y$ is the coarsest topology such that $f$ is continuous.
\end{prop}

\begin{proof}
\pf\ Immediate from definition. \qed
\end{proof}

\begin{prop}[Local Formulation of Continuity]
Let $X$ and $Y$ be topological spaces. Let $f : X \rightarrow Y$. Let $\mathcal{U}$ be a set of open subspaces of $X$ such that $X = \bigcup \mathcal{U}$. If $f \restriction U : U \rightarrow Y$ is continuous for all $U \in \mathcal{U}$, then $f$ is continuous.
\end{prop}

\begin{proof}
\pf
\step{1}{\pflet{$x \in X$} \prove{$f$ is continuous at $x$.}}
\step{2}{\pflet{$V$ be a neighbourhood of $f(x)$.}}
\step{3}{\pick\ $U \in \mathcal{U}$ such that $x \in U$.}
\step{4}{\pick\ $W$ open in $U$ such that $x \in W$ and $f(W) \subseteq V$.}
\step{5}{$W$ is open in $X$.}
\qed
\end{proof}

\begin{thm}
\label{thm:subspace_universal}
Let $X$ be a topological space and $(Y,i)$ a subset of $X$. Then the subspace topology on $Y$ is the unique topology such that, for every topological space $Z$ and function $f : Z \rightarrow Y$, we have $f$ is continuous if and only if $i \circ f : Z \rightarrow X$ is continuous.
\end{thm}

\begin{proof}
\pf
\step{1}{If we give $Y$ the subspace topology then, for every topological space $Z$ and function $f : Z \rightarrow Y$, we have $f$ is continuous if and only if $i \circ f$ is continuous.}
\begin{proof}
	\step{a}{Given $Y$ the subspace topology.}
	\step{b}{\pflet{$Z$ be a topological space.}}
	\step{c}{\pflet{$f : Z \rightarrow Y$}}
	\step{d}{If $f$ is continuous then $i \circ f$ is continuous.}
	\begin{proof}
		\pf\ Since $i$ is continuous.
	\end{proof}
	\step{e}{If $i \circ f$ is continuous then $f$ is continuous.}
	\begin{proof}
		\step{i}{\assume{$i \circ f$ is continuous.}}
		\step{ii}{\pflet{$U$ be open in $Y$.}}
		\step{iii}{$\inv{f}(\inv{i}(i(U))$ is open in $Z$.}
		\step{iv}{$\inv{f}(U)$ is open in $Z$.}
	\end{proof}
\end{proof}
\step{2}{If, for every topological space $Z$ and function $f : Z \rightarrow Y$, we have $f$ is continuous if and only if $i \circ f$ is continuous.}
\begin{proof}
	\step{a}{\assume{For every topological space $Z$ and function $f : Z \rightarrow Y$, we have $f$ is continuous if and only if $i \circ f$ is continuous.}}
	\step{b}{$i$ is continuous.}
	\step{c}{For every open set $U$ in $X$, we have $\inv{i}(X)$ is open in $Y$}
	\step{b}{\pflet{$Z$ be the set $Y$ under the subspace topology and $f : Z \rightarrow Y$ the identity function.}}
	\step{c}{$i \circ f$ is continuous.}
	\step{d}{$f$ is continuous.}
	\step{e}{Every set open in $Y$ is open in $Z$.}
\end{proof}
\qed
\end{proof}

\begin{prop}
\label{prop:open_in_open}
Let $X$ be a topological space, $Y$ a subspace of $X$ and $U \subseteq Y$. If $Y$ is open in $X$ and $U$ is open in $Y$ then $U$ is open in $X$.
\end{prop}

\begin{proof}
\pf
\step{1}{\pick\ $V$ open in $X$ such that $U = V \cap Y$}
\step{2}{$U$ is open in $X$.}
\begin{proof}
	\pf\ It is the intersection of two open sets in $X$.
\end{proof}
\qed
\end{proof}

\begin{prop}
Let $Y$ be a subspace of $X$ and $A \subseteq Y$. Then the subspace topology on $A$ as a subspace of $Y$ is the same as the subspace topology on $A$ as a subspace of $X$.
\end{prop}

\begin{proof}
\pf
\step{1}{\pflet{$\mathcal{T}_Y$ be the subspace topology on $A$ as a subspace of $Y$.}}
\step{2}{\pflet{$\mathcal{T}_X$ be the subspace topology on $A$ as a subspace of $X$.}}
\step{3}{\pflet{$U \subseteq A$}}
\step{4}{$U \in \mathcal{T}_Y \Leftrightarrow U \in \mathcal{T}_X$}
\begin{proof}
	\pf
	\begin{align*}
		U \in \mathcal{T}_Y & \Leftrightarrow \exists V \text{ open in } Y. U = V \cap A \\
		& \Leftrightarrow \exists V. \exists W \text{ open in } X. (V = Y \cap W \wedge U = V \cap A) \\
		& \Leftrightarrow \exists W \text{ open in } X. U = Y \cap W \cap A \\
		& \Leftrightarrow \exists W \text{ open in } X. U = W \cap A \\
		& \Leftrightarrow U \in \mathcal{T}_X
	\end{align*}
\end{proof}
\qed
\end{proof}

\begin{prop}
Let $X$ be a topological space. Let $\mathcal{B}$ be a basis for the topology on $X$. Let $Y \subseteq X$. Then $\mathcal{B}' = \{ B \cap Y : B \in \mathcal{B} \}$ is a basis for the topology on $Y$.
\end{prop}

\begin{proof}
\pf
\step{1}{Every element of $\mathcal{B}'$ is open.}
\begin{proof}
	\pf\ For all $B \in \mathcal{B}$, we have $B$ is open in $X$, so $B \cap Y$ is open in $Y$.
\end{proof}
\step{2}{For any open set $V$ in $Y$ and $y \in V$, there exists $B' \in \mathcal{B}'$ such that $y \in B' \subseteq V$}
\begin{proof}
	\step{a}{\pflet{$V$ be open in $Y$.}}
	\step{b}{\pflet{$y \in V$}}
	\step{c}{\pick\ $U$ open in $X$ such that $V = U \cap Y$.}
	\step{d}{\pick\ $B \in \mathcal{B}$ such that $y \in B \subseteq U$}
	\step{e}{$B \cap Y \in \mathcal{B}'$ and $y \in B \cap Y \subseteq V$}
\end{proof}
\qed
\end{proof}

\subsection{Product Topology}

\begin{prop}
Let $\{X_i\}_{i \in I}$ be a family of topological spaces. Let $Y_i$ be a subspace of $X_i$ for all $i \in I$. Then the product topology on $\prod_{i \in I} Y_i$ is the same as the subspace topology on $\prod_{i \in I} Y_i$ as a subspace of $\prod_{i \in I} X_i$.
\end{prop}

\begin{proof}
\pf
\step{1}{Given $\prod_{i \in I} Y_i$ the subspace topology.}
\step{11}{\pflet{$\iota : \prod_{i \in I} Y_i$ be the inclusion.}}
\step{2}{\pflet{$Z$ be any topological space.}}
\step{3}{\pflet{$f : Z \rightarrow \prod_{i \in I} Y_i$}}
\step{4}{$f$ is continuous if and only if, for all $i \in I$, we have $\pi_i \circ f$ is continuous.}
\begin{proof}
	\pf
	\begin{align*}
		f \text{ is continuous}
		& \Leftrightarrow \iota \circ f : Z \rightarrow \prod_{i \in I} X_i \text{ is continuous} & (\text{Theorem \ref{thm:subspace_universal}}) \\
		& \Leftrightarrow \forall i \in I. \pi_i \circ \iota \circ f : Z \rightarrow X_i \text{ is continuous} & (\text{Theorem \ref{thm:product_universal}}) \\
		& \Leftrightarrow \forall i \in I. \iota_i \circ \pi_i \circ f : Z \rightarrow X_i \text{ is continuous} \\
		& \Leftrightarrow \forall i \in I. \pi_i \circ f : Z \rightarrow Y_i \text{ is continuous}
		& (\text{Theorem \ref{thm:subspace_universal}})
	\end{align*}
	where $\iota_i$ is the inclusion $Y_i \rightarrow X_i$.
\end{proof}
\qed
\end{proof}

\section{Embedding}

\begin{df}[Embedding]
Let $X$ and $Y$ be topological spaces and $f : X \rightarrow Y$. Then $f$ is an \emph{embedding} iff $f$ is injective and the topology on $X$ is the subspace induced by $f$.
\end{df}

\begin{prop}
Every embedding is continuous.
\end{prop}

\begin{proof}
\pf\ Theorem \ref{thm:subspace_universal}. \qed
\end{proof}

\begin{prop}
Let $X$ and $Y$ be topological spaces. Let $b \in Y$. The function $\kappa : X \rightarrow X \times Y$ that maps $x$ to $(x,b)$ is an embedding.
\end{prop}

\begin{proof}
\pf
\step{1}{For all $U$ open in $X$, we have $U = \inv{\kappa}(V)$ for some $V$ open in $X \times Y$.}
\begin{proof}
	\pf\ Take $V = U \times Y$.
\end{proof}
\step{2}{For all $V$ open in $X \times Y$ we have $\inv{\kappa}(V)$ is open in $X$.}
\begin{proof}
	\pf\ Since $\pi_1 \circ \kappa = \id{X}$ and $\pi_2 \circ \kappa$ (which is the constant function with value $b$) are both continuous, hence $\kappa$ is continuous.
\end{proof}
\qed
\end{proof}

\section{Open Maps}

\begin{df}[Open Map]
Let $X$ and $Y$ be topological spaces and $f : X \rightarrow Y$. Then $f$ is an \emph{open map} iff, for all $U$ open in $X$, we have $f(U)$ is open in $Y$.
\end{df}

\begin{prop}
Let $X$ and $Y$ be topological spaces. The projections $\pi_1 : X \times Y \rightarrow X$ and $\pi_2 : X \times Y \rightarrow Y$ are open maps.
\end{prop}

\begin{proof}
\pf
\step{1}{$\pi_1$ is an open map.}
\begin{proof}
	\step{a}{\pflet{$U$ be open in $X \times Y$.}}
	\step{b}{\pflet{$x \in \pi_1(U)$}}
	\step{c}{\pick\ $y$ such that $(x,y) \in U$}
	\step{d}{\pick\ $V$ and $W$ open in $X$ and $Y$ respectively such that $(x,y) \in V \times W \subseteq U$}
	\step{e}{$x \in V \subseteq \pi_1(U)$}
\end{proof}
\step{2}{$\pi_2$ is an open map.}
\begin{proof}
	\pf\ Similar.
\end{proof}
\qed
\end{proof}

\subsection{Subspaces}

\begin{prop}
Let $X$ and $Y$ be topological spaces. Let $p : X \rightarrow Y$ be an open map. Let $A$ be an open set in $X$. Then $p \restriction A : A \rightarrow p(A)$ is an open map.
\end{prop}

\begin{proof}
\pf
\step{1}{\pflet{$U$ be open in $A$.}}
\step{2}{$U$ is open in $X$.}
\begin{proof}
	\pf\ Proposition \ref{prop:open_in_open}.
\end{proof}
\step{3}{$p(U)$ is open in $Y$.}
\step{5}{$p(U)$ is open in $p(A)$.}
\begin{proof}
	\pf\ Since $p(U) = p(U) \cap p(A)$.
\end{proof}
\qed
\end{proof}

\section{Locally Finite}

\begin{df}[Locally Finite]
Let $X$ be a topological space. Let $\{A_i\}_{i \in I}$ be a family of subsets of $X$. Then $\{A_i\}_{i \in I}$ is \emph{locally finite} iff, for every $x \in X$, there exist only finitely many $i \in I$ such that $x \in A_i$.
\end{df}

\begin{thm}[Pasting Lemma]
Let $X$ and $Y$ be topological spaces. Let $f : X \rightarrow Y$. Let $\{A_i\}_{i \in I}$ be a locally finite family of closed subspaces of $X$ such that $X = \bigcup_{i \in I} A_i$. If $f \restriction A_i : A_i \rightarrow Y$ is continuous for all $i \in I$, then $f$ is continuous.
\end{thm}

\begin{proof}
\pf
\step{1}{\pflet{$B$ be closed in $Y$.}}
\step{2}{\pflet{$A = \inv{f}(B)$} \prove{$A$ is closed in $X$.}}
\step{3}{$A = \bigcup_{i \in I} \inv{f \restriction A_i}(B)$}
\step{4}{\pflet{$x \in X - A$} \prove{There exists a neighbourhood $U'$ of $x$ such that $U' \subseteq X - A$.}}
\step{5}{\pick\ a neighbourhood $U$ of $x$ such that $U$ intersects $A_i$ for only finitely many $i \in I$.}
\step{6}{\pflet{$i_1$, \ldots, $i_n$ be the elements of $I$ such that $U$ intersects $A_{i_1}$, \ldots, $A_{i_n}$.}}
\step{7}{For $j = 1, \ldots, n$, \pflet{$S_j = \inv{f \restriction A_{i_j}}(B)$}}
\step{8}{For $j = 1, \ldots, n$, we have $S_j$ is closed in $X$.}
\step{9}{For $j = 1, \ldots, n$, we have $x \notin S_j$.}
\step{10}{\pflet{$U' = U \cap \bigcap_{j=1}^n (X - S_j)$}}
\step{11}{$U'$ is a neighbourhood of $x$.}
\step{12}{$U' \subseteq X - A$}
\qed
\end{proof}

\section{Closed Maps}

\begin{df}[Closed Map]
Let $X$ and $Y$ be topological spaces. Let $f : X \rightarrow Y$. Then $f$ is a \emph{closed map} iff, for every closed set $C$ in $X$, we have $f(C)$ is closed in $Y$.
\end{df}

\section{Product Topology}

\begin{df}[Product Topology]
Let $\{ X_\lambda \}_{\lambda \in \Lambda}$ be a family of topological spaces. The \emph{product topology} on $\prod_{\lambda \in \Lambda} X_\lambda$ is the coarsest topology such that every projection onto $X_\lambda$ is continuous.
\end{df}

\subsection{Closed Sets}

\begin{prop}
\label{prop:closed_product}
Let $X$ and $Y$ be topological spaces. Let $A$ be a closed set in $X$ and $B$ a closed set in $Y$. Then $A \times B$ is closed in $X \times Y$.
\end{prop}

\begin{proof}
\pf\ Since $(X \times Y) - (A \times B) = ((X - A) \times Y) \cup (X \times (Y - B))$. \qed
\end{proof}

\begin{prop}
\label{prop:product_basis}
Let $\{ X_\alpha \}_{\alpha \in A}$ be a family of topological spaces. The product topology on $\prod_{\alpha \in A} X_\alpha$ is the topology generated by the basis $\mathcal{B} = \{ \prod_{\alpha \in A} U_\alpha : \text{for all } \alpha \in A, U_\alpha \text{ is open in } X_\alpha \text{ and } U_\alpha = X_\alpha \text{ for all but finitely many } \alpha \in A \}$.
\end{prop}

\begin{proof}
\pf
\step{1}{$\mathcal{B}$ is a basis for a topology.}
\step{2}{\pflet{$\mathcal{T}$ be the topology generated by $\mathcal{B}$.}}
\step{3}{\pflet{$\mathcal{T}_p$ be the product topology.}}
\step{4}{$\mathcal{T} \subseteq \mathcal{T}_p$}
\begin{proof}
	\step{a}{\pflet{$B \in \mathcal{B}$}}
	\step{b}{\pflet{$B = \prod_{\alpha \in A} U_\alpha$ with each $U_\alpha$ open in $X_\alpha$ and $U_\alpha = X_\alpha$ except for $\alpha = \alpha_1, \ldots, \alpha_n$}}
	\step{c}{$B = \inv{\pi_{\alpha_1}}(U_{\alpha_1}) \cap \cdots \cap \inv{\pi_{\alpha_n}}(U_{\alpha_n})$}
	\step{d}{$B \in \mathcal{T}_p$}
\end{proof}
\step{5}{$\mathcal{T}_p \subseteq \mathcal{T}$}
\begin{proof}
	\step{a}{For every $\alpha \in A$ we have $\pi_\alpha$ is continuous.}
	\begin{proof}
		\pf\ Since $\inv{\pi}(U)$ is open for every $U$ open in $X_\alpha$.
	\end{proof}
\end{proof}
\qed
\end{proof}

\begin{thm}
\label{thm:product_universal}
Let $\{ X_\alpha \}_{\alpha \in A}$ be a family of topological spaces. Then the product topology on $\prod_{\alpha \in A} X_\alpha$ is the unique topology such that, for every topological space $Z$ and function $f : Z \rightarrow \prod_{\alpha \in A} X_\alpha$, we have $f$ is continuous if and only if, for all $\alpha \in A$, we have $\pi_\alpha \circ f : Z \rightarrow X_\alpha$ is continuous.
\end{thm}

\begin{proof}
\pf
\step{1}{If we give $\prod_{\alpha \in A} X_\alpha$ the product topology, then for every topological space $Z$ and function $f : Z \rightarrow \prod_{\alpha \in A} X_\alpha$, we have $f$ is continuous if and only if, for all $\alpha \in A$, we have $\pi_\alpha \circ f$ is continuous.}
\begin{proof}
	\step{a}{Give $\prod_{\alpha \in A} X_\alpha$ the product topology.}
	\step{b}{\pflet{$Z$ be a topological space.}}
	\step{c}{\pflet{$f : Z \rightarrow \prod_{\alpha \in A} X_\alpha$}}
	\step{d}{If $f$ is continuous then, for all $\alpha \in A$, we have $\pi_\alpha \circ f$ is continuous.}
	\begin{proof}
		\pf\ Since the composite of two continuous functions is continuous.
	\end{proof}
	\step{e}{If, for all $\alpha \in A$, we have $\pi_\alpha \circ f$ is continuous, then $f$ is continuous.}
	\begin{proof}
		\step{i}{\assume{For all $\alpha \in A$ we have $\pi_\alpha \circ f$ is continuous.}}
		\step{ii}{\pflet{$\{ U_\alpha \}_{\alpha \in A}$ be a family with $U_\alpha$ open in $X_\alpha$ such that $U_\alpha = X_\alpha$ for all $\alpha$ except $\alpha = \alpha_1, \ldots, \alpha_n$.}}
		\step{iii}{For all $\alpha$ we have $\inv{f}(\inv{\pi_\alpha}(U_\alpha))$ is open in $Z$.}
		\step{iv}{$\inv{f}(\prod_\alpha U_\alpha)$ is open in $Z$}
		\begin{proof}
			\pf\ Since $\inv{f}(\prod_\alpha U_\alpha) = \inv{f}(\inv{\pi_{\alpha_1}}(U_{\alpha_1})) \cap \cdots \cap \inv{f}(\inv{\pi_{\alpha_n}}(U_{\alpha_n}))$.
		\end{proof}
	\end{proof}
\end{proof}
\step{2}{If $\mathcal{T}$ is a topology on $\prod_{\alpha \in A} X_\alpha$ such that, for every topological pace $Z$ and function $f : Z \rightarrow \prod_{\alpha \in A} X_\alpha$, we have $f$ is continuous if and only if, for all $\alpha \in A$, we have $\pi_\alpha \circ f$ is continuous, then $\mathcal{T}$ is the product topology.}
\begin{proof}
	\step{a}{\assume{$\mathcal{T}$ is a topology on $\prod_{\alpha \in A} X_\alpha$ such that, for every topological pace $Z$ and function $f : Z \rightarrow \prod_{\alpha \in A} X_\alpha$, we have $f$ is continuous if and only if, for all $\alpha \in A$, we have $\pi_\alpha \circ f$ is continuous.}}
	\step{b}{\pflet{$\mathcal{T}_p$ be the product topology.}}
	\step{c}{$\mathcal{T} \subseteq \mathcal{T}_p$}
	\begin{proof}
		\step{i}{\pflet{$Z = (\prod_\alpha X_\alpha, \mathcal{T}_p)$}}
		\step{ii}{\pflet{$f : Z \rightarrow \prod_\alpha X_\alpha$ be the identity function}}
		\step{iii}{For all $\alpha$ we have $\pi_\alpha \circ f$ is continuous.}
		\step{iv}{$f$ is continuous.}
		\begin{proof}
			\pf\ \stepref{a}
		\end{proof}
		\step{v}{Every set open in $\mathcal{T}$ is open in $\mathcal{T}_p$}
	\end{proof}
	\step{d}{$\mathcal{T}_p \subseteq \mathcal{T}$}
	\begin{proof}
		\step{i}{$\id{\prod_\alpha X_\alpha}$ is continuous.}
		\step{ii}{For all $\alpha$ we have $\pi_\alpha$ is continuous.}
		\begin{proof}
			\pf\ \stepref{a}
		\end{proof}
		\step{iii}{$\mathcal{T}_p \subseteq \mathcal{T}$}
		\begin{proof}
			\pf\ Since $\mathcal{T}_p$ is the coarsest topology such that every $\pi_\alpha$ is continuous.
		\end{proof}
	\end{proof}
\end{proof}
\qed
\end{proof}

\begin{ex}
It is not true that, for any function $f : \prod_{\alpha \in A} X_\alpha \rightarrow Y$, if $f$ is continuous in every variable separately then $f$ is continuous.

Define $f : \mathbb{R}^2 \rightarrow \mathbb{R}$ by
\[ f(x,y) = \begin{cases}
\frac{xy}{x^2 + y^2} & \text{if } (x,y) \neq (0,0) \\
0 & \text{if } x = y = 0
\end{cases} \]
Then $f$ is continuous in $x$ and in $y$, but is not continuous.
\end{ex}

\begin{prop}
Let $\{X_i\}_{i \in I}$ be a nonempty family of topological spaces. The product topology on $\prod_{i \in I}$ is the topology generated by the subbasis $\{ \inv{\pi_i}(U) : i \in I, U \text{ is open in } X_i \}$.
\end{prop}

\begin{proof}
\pf
\step{1}{$\{ \inv{\pi_i}(U) : i \in I, U \text{ is open in } X_i \}$ is a subbasis for a topology on $\prod_{i \in I} X_i$.}
\begin{proof}
	\step{a}{\pick\ $i_0 \in I$}
	\step{b}{$\prod_{i \in I} X_i = \inv{\pi_{i_0}}(X_{i_0})$}
\end{proof}
\step{2}{The topology generated by this subbasis is the product topology.}
\begin{proof}
	\pf\ Since the basis in Proposition \ref{prop:product_basis} is the set of all finite intersections of elements of this subbasis.
\end{proof}
\qed
\end{proof}

\subsection{Closure}

\begin{prop}
Let $\{X_i\}_{i \in I}$ be a family of topological spaces. Let $A_i \subseteq X_i$ for all $i \in I$. Then
\[ \prod_{i \in I} \overline{A_i} = \overline{\prod_{i \in I} A_i} \enspace . \]
\end{prop}

\begin{proof}
\pf
\step{1}{$\prod_{i \in I} \overline{A_i} \subseteq \overline{\prod_{i \in I} A_i}$}
\begin{proof}
	\step{a}{\pflet{$x \in \prod_{i \in I} \overline{A_i}$}}
	\step{b}{For any family $\{U_i\}_{i \in I}$ where each $U_i$ is open in $X_i$, and $U_i = X_i$ for all but finitely many $i \in I$, if $x \in \prod_{i \in I} U_i$ then $\prod_{i \in I} U_i$ intersects $\prod_{i \in I} A_i$.}
	\begin{proof}
		\step{i}{\pflet{$\{U_i\}_{i \in I}$ be a family where each $U_i$ is open in $X_i$, and $U_i = X_i$ for all but finitely many $i$.}}
		\step{ii}{\assume{$x \in \prod_{i \in I}$}}
		\step{iii}{For all $i \in I$ we have $U_i$ intersects $A_i$}
		\begin{proof}
			\pf\ Since $\pi_i(x) \in \overline{A_i}$ and $U_i$ is a neighbourhood of $\pi_i(x)$.
		\end{proof}
		\step{iv}{$\prod_{i \in I} U_i$ intersects $\prod_{i \in I} A_i$}
\end{proof}		
	\step{c}{$x \in \overline{\prod_{i \in I} A_i}$}
	\begin{proof}
		\pf\ Proposition \ref{prop:closure_basis}.
	\end{proof}
\end{proof}
\step{2}{$\overline{\prod_{i \in I} A_i} \subseteq \prod_{i \in I} \overline{A_i}$}
\begin{proof}
	\step{a}{\pflet{$x \in \overline{\prod_{i \in I} A_i}$}}
	\step{b}{\pflet{$i \in I$} \prove{$\pi_i(x) \in \overline{A_i}$}}
	\step{c}{\pflet{$U$ be a neighbourhood of $\pi_i(x)$ in $X_i$}}
	\step{d}{$\inv{\pi_i}(U)$ is a neighbourhood of $x$ in $\prod_{i \in I} X_i$}
	\step{e}{\pick\ $y \in \inv{\pi_i}(U) \cap \prod_{i \in I} A_i$}
	\step{f}{$\pi_i(y) \in U \cap A_i$}
\end{proof}
\qed
\end{proof}

\subsection{Convergence}

\begin{prop}
Let $\{X_i\}_{i \in I}$ be a family of topological spaces. Let $(x_n)$ be a sequence of points in $\prod_{i \in I} X_i$ and $l \in \prod_{i \in I} X_i$. Then $x_n \rightarrow l$ as $n \rightarrow \infty$ if and only if, for all $i \in I$, we have $\pi_i(x_n) \rightarrow \pi_i(l)$ as $n \rightarrow \infty$.
\end{prop}

\begin{proof}
\pf
\step{1}{If $x_n \rightarrow l$ as $n \rightarrow \infty$ then, for all $i \in I$, we have $\pi_i(x_n) \rightarrow \pi_i(l)$ as $n \rightarrow \infty$.}
\begin{proof}
	\pf\ Proposition \ref{prop:continuous_converge}.
\end{proof}
\step{2}{If, for all $i \in I$, we have $\pi_i(x_n) \rightarrow \pi_i(l)$ as $n \rightarrow \infty$, then $x_n \rightarrow l$ as $n \rightarrow \infty$.}
\begin{proof}
	\step{a}{\assume{For all $i \in I$ we have $\pi_i(x_n) \rightarrow \pi_i(l)$ as $n \rightarrow \infty$.}}
	\step{b}{\pflet{$U$ be a neighbourhood of $l$.}}
	\step{c}{\pick\ $i_1, \ldots, i_n \in I$ and open sets $U_j$ in $X_{i_j}$ for $j = 1, \ldots, n$ such that $l \in \inv{\pi_{i_1}}(U_1) \cap \cdots \cap \inv{\pi_{i_n}}(U_n) \subseteq U$}
	\step{d}{For $j=1, \ldots, n$ we have $\pi_{i_j}(l) \in U_j$}
	\step{e}{\pick\ $N$ such that, for all $m \geq N$, we have $\pi_{i_j}(x_m) \in U_j$}
	\step{f}{$\forall m \geq N. x_m \in U$}
\end{proof}
\qed
\end{proof}

\section{Topological Disjoint Union}

\begin{df}[Coproduct Topology]
Let $\{ X_\alpha \}_{\alpha \in A}$ be a family of topological spaces. The \emph{coproduct topology} on $\coprod_{\alpha \in A} X_\alpha$ is 

\[ \mathcal{T} = \left\{ \coprod_{\alpha \in A} U_\alpha : \{ U_\alpha \}_{\alpha \in A} \text{ is a family with $U_\alpha$ open in $X_\alpha$ for all $\alpha$} \right\} \enspace . \]

We prove this is a topology.
\end{df}

\begin{proof}
\pf
\step{1}{For all $\mathcal{U} \subseteq \mathcal{T}$ we have $\bigcup \mathcal{U} \in \mathcal{T}$}
\begin{proof}
	\pf
	\[ \bigcup_{i \in I} \coprod_{\alpha \in A} U_{i\alpha} = \coprod_{\alpha \in A} \bigcup_{i \in I} U_{i \alpha} \]
\end{proof}
\step{2}{For all $U,V \in \mathcal{T}$ we have $U \cap V \in \mathcal{T}$}
\begin{proof}
	\pf
	\[ \coprod_{\alpha \in A} U_\alpha \cap \coprod_{\alpha \in A} V_\alpha = \coprod_{\alpha \in A} (U_\alpha \cap V_\alpha) \]
\end{proof}
\step{3}{$\coprod_{\alpha \in A} X_\alpha \in \mathcal{T}$}
\begin{proof}
	\pf\ Since every $X_\alpha$ is open in $X_\alpha$.
\end{proof}
\qed
\end{proof}

\begin{prop}
\label{prop:coproduct_finest}
The coproduct topology is the finest topology on $\coprod_{\alpha \in A} X_\alpha$ such that every injection $\kappa_\alpha : X_\alpha \rightarrow \coprod_{\alpha \in A} X_\alpha$ is continuous.
\end{prop}

\begin{proof}
\pf
\step{0}{\pflet{$P = \coprod_{\alpha \in A} X_\alpha$}}
\step{1}{\pflet{$\mathcal{T}_c$ be the coproduct topology.}}
\step{2}{\pflet{$\mathcal{T}$ be any topology on $P$}}
\step{3}{For all $\alpha \in A$, the injection $\kappa_\alpha : X_\alpha \rightarrow (P, \mathcal{T}_c)$ is continuous.}
\begin{proof}
	\step{a}{\pflet{$\alpha \in A$}}
	\step{b}{\pflet{$\{ U_\alpha \}_{\alpha \in A}$ be a family with each $U_\alpha$ open in $X_\alpha$.}}
	\step{c}{For all $\alpha \in A$, we have $\inv{\kappa_\alpha}(\coprod_{\alpha \in A} U_\alpha)$ is open in $X_\alpha$.}
	\begin{proof}
		\pf\ Since $\inv{\kappa_\alpha}(\coprod_{\alpha \in A} U_\alpha) = U_\alpha$.
	\end{proof}
\end{proof}
\step{4}{If, for all $\alpha \in A$, the injection $\kappa_\alpha : X_\alpha \rightarrow (P, \mathcal{T})$ is continuous, then $\mathcal{T} \subseteq \mathcal{T}_c$.}
\begin{proof}
	\step{a}{\assume{For all $\alpha \in A$, the injection $\kappa_\alpha : X_\alpha \rightarrow (P, \mathcal{T})$ is continuous.}}
	\step{b}{\pflet{$U \in \mathcal{T}$}}
	\step{c}{For all $\alpha \in a$, we have $\inv{\kappa_\alpha}(U)$ is open in $X_\alpha$.}
	\step{d}{$U = \coprod_{\alpha \in A} \inv{\kappa_\alpha}(U) \in \mathcal{T}_c$}
\end{proof}
\qed
\end{proof}

\begin{thm}
Let $\{ X_\alpha \}_{\alpha \in A}$ be a family of topological spaces. The coproduct topology is the unique topology on $\coprod_{\alpha \in A} X_\alpha$ such that, for every topological space $Z$ and function $f : \coprod_{\alpha \in A} X_\alpha \rightarrow Z$, we have $f$ is continuous if and only if $\forall \alpha \in A. f \circ \kappa_\alpha$ is continuous.
\end{thm}

\begin{proof}
\pf
\step{1}{\pflet{$X = \coprod_{\alpha \in A} X_\alpha$}}
\step{2}{\pflet{$\mathcal{T}_c$ be the coproduct topology.}}
\step{3}{For every topological space $Z$ and function $f : (X, \mathcal{T}_c) \rightarrow Z$, we have $f$ is continuous if and only if $\forall \alpha \in A. f \circ \kappa_\alpha$ is continuous.}
\begin{proof}
	\step{a}{\pflet{$Z$ be a topological space.}}
	\step{b}{\pflet{$f : X \rightarrow Z$}}
	\step{c}{If $f$ is continuous then $\forall \alpha \in A. f \circ \kappa_\alpha$ is continuous.}
	\begin{proof}
		\pf\ Because the composite of two continuous functions is continuous.
	\end{proof}
	\step{d}{If $\forall \alpha \in A. f \circ \kappa_\alpha$ is continuous then $f$ is continuous.}
	\begin{proof}
		\step{i}{\assume{$\forall \alpha \in A. f \circ \kappa_\alpha$ is continuous.}}
		\step{ii}{\pflet{$U$ be open in $Z$}}
		\step{iii}{For all $\alpha \in A$ we have $\inv{\kappa_\alpha}(\inv{f}(U))$ is open in $X_\alpha$}
		\step{iv}{$\inv{f}(U) = \coprod_{\alpha \in A} \inv{\kappa_\alpha}(\inv{f}(U))$}
		\step{v}{$\inv{f}(U)$ is open in $X$}
	\end{proof}
\end{proof}
\step{4}{For any topology $\mathcal{T}$ on $X$, if for every topological space $Z$ and function $f : (X, \mathcal{T}) \rightarrow Z$, we have $f$ is continuous if and only if $\forall \alpha \in A. f \circ \kappa_\alpha$ is continuous, then $\mathcal{T} = \mathcal{T}_c$.}
\begin{proof}
	\step{a}{\pflet{$\mathcal{T}$ be a topology on $X$.}}
	\step{b}{\assume{For every topological space $Z$ and function $f : (X, \mathcal{T}) \rightarrow Z$, we have $f$ is continuous if and only if $\forall \alpha \in A. f \circ \kappa_\alpha$ is continuous.}}
	\step{c}{$\mathcal{T} \subseteq \mathcal{T}_c$}
	\begin{proof}
		\step{i}{For all $\alpha \in A$ we have $\kappa_\alpha : X_\alpha \rightarrow (X, \mathcal{T})$ is continuous.}
		\begin{proof}
			\pf\ From \stepref{a} since $\id{X}$ is continuous.
		\end{proof}
		\step{ii}{$\mathcal{T} \subseteq \mathcal{T}_c$}
		\begin{proof}
			\pf\ Proposition \ref{prop:coproduct_finest}.
		\end{proof}
	\end{proof}
	\step{d}{$\mathcal{T}_c \subseteq \mathcal{T}$}
	\begin{proof}
		\step{i}{\pflet{$f : (X, \mathcal{T}) \rightarrow (X, \mathcal{T}_c)$ be the identity function.}}
		\step{ii}{$f \circ \kappa_\alpha$ is continuous for all $\alpha$.}
		\step{iii}{$f$ is continuous.}
		\begin{proof}
			\pf\ \stepref{a}
		\end{proof}
		\step{iv}{$\mathcal{T}_c \subseteq \mathcal{T}$}
	\end{proof}
\end{proof}
\qed
\end{proof}

\section{Quotient Spaces}

\begin{df}[Quotient Topology]
Let $X$ be a topological space, $S$ a set, and $\pi : X \twoheadrightarrow S$ be a surjection. The \emph{quotient topology} on $S$ induced by $\pi$ is $\mathcal{T} = \{ U \in \mathcal{P} S : \inv{\pi}(U) \text{ is open in } X \}$.

We prove this is a topology.
\end{df}

\begin{proof}
\pf
\step{1}{For all $\mathcal{U} \subseteq \mathcal{T}$ we have $\bigcup \mathcal{U} \in \mathcal{T}$.}
\begin{proof}
	\pf\ Since $\inv{\pi}(\bigcup \mathcal{U}) = \bigcup \{ \inv{\pi}(U) : U \in \mathcal{U} \}$.
\end{proof}
\step{2}{For all $U,V \in \mathcal{T}$ we have $U \cap V \in \mathcal{T}$.}
\begin{proof}
	\pf\ Since $\inv{\pi}(U \cap V) = \inv{\pi}(U) \cap \inv{\pi}(V)$.
\end{proof}
\step{3}{$X \in \mathcal{T}$}
\begin{proof}
	\pf\ Since $X = \inv{\pi}(Y)$.
\end{proof}
\qed
\end{proof}

\begin{prop}
Let $X$ be a topological space, $S$ a set and $\pi : X \twoheadrightarrow S$ a surjection. Then the quotient topology on $S$ is the finest topology such that $\pi$ is continuous.
\end{prop}

\begin{proof}
\pf\ Immediate from definitions. \qed
\end{proof}

\begin{thm}
\label{thm:quotient_topology_universal}
Let $X$ be a topological space, let $S$ be a set, and let $\pi : X \twoheadrightarrow S$ be surjective. Then the quotient topology on $S$ is the unique topology such that, for every topological space $Z$ and function $f : S \rightarrow Z$, we have $f$ is continuous if and only if $f \circ \pi$ is continuous.
\end{thm}

\begin{proof}
\pf
\step{1}{If $S$ is given the quotient topology, then for every topological space $Z$ and function $f : S \rightarrow Z$, we have $f$ is continuous if and only if $f \circ \pi$ is continuous.}
\begin{proof}
	\step{a}{Give $S$ the quotient topology.}
	\step{b}{\pflet{$Z$ be a topological space.}}
	\step{c}{\pflet{$f : S \rightarrow Z$}}
	\step{d}{If $f$ is continuous then $f \circ \pi$ is continuous.}
	\begin{proof}
		\pf\ The composite of two continuous functions is continuous.
	\end{proof}
	\step{e}{If $f \circ \pi$ is continuous then $f$ is continuous.}
	\begin{proof}
		\step{i}{\assume{$f \circ \pi$ is continuous.}}
		\step{ii}{\pflet{$U$ be open in $Z$.}}
		\step{iii}{$\inv{\pi}(\inv{f}(U))$ is open in $X$.}
		\step{iv}{$\inv{f}(U)$ is open in $S$.}
	\end{proof}
\end{proof}
\step{2}{If $S$ is given a topology such that, for every topological space $Z$ and function $f : S \rightarrow Z$, we have $f$ is continuous if and only if $f \circ \pi$ is continuous, then that topology is the quotient topology.}
\begin{proof}
	\step{a}{Give $S$ a topology such that, for every topological space $Z$ and function $f : S \rightarrow Z$, we have $f$ is continuous if and only if $f \circ \pi$ is continuous.}
	\step{b}{\pflet{$U \subseteq S$}}
	\step{c}{If $\inv{\pi}(U)$ is open in $X$ then $U$ is open in $S$.}
	\begin{proof}
		\step{i}{\pflet{$Z$ be $S$ under the quotient topology induced by $\pi$.}}
		\step{ii}{\pflet{$f : S \rightarrow Z$ be the identity function.}}
		\step{iii}{$f \circ \pi$ is continuous.}
		\step{iv}{$f$ is continuous.}
		\begin{proof}
			\pf\ \stepref{a}
		\end{proof}
		\step{v}{$U$ is open in $Z$.}
		\step{vi}{$U$ is open in $X$.}
	\end{proof}
	\step{d}{If $U$ is open in $S$ then $\inv{\pi}(U)$ is open in $X$.}
	\begin{proof}
		\pf\ Since $\pi$ is continuous (taking $Z = S$ and $f = \id{S}$ in \stepref{a}).
	\end{proof}
\end{proof}
\qed
\end{proof}

\subsection{Quotient Maps}

\begin{df}[Quotient Map]
Let $X$ and $S$ be topological spaces and $\pi : X \rightarrow S$. Then $\pi$ is a \emph{quotient map} iff $\pi$ is surjective and the topology on $S$ is the quotient topology induced by $\pi$.
\end{df}

\begin{prop}
Let $X$ and $Y$ be topological spaces. Let $f : X \rightarrow Y$. Then $f$ is a quotient map if and only if $f$ is surjective and strongly continuous.
\end{prop}

\begin{proof}
\pf\ Immediate from definition. \qed
\end{proof}

\begin{prop}
Let $X$ and $Y$ be topological spaces. Let $p : X \twoheadrightarrow Y$ be surjective. Then the following are equivalent.
\begin{enumerate}
\item $p$ is a quotient map.
\item $p$ is continuous and maps saturated open sets to open sets.
\item $p$ is continuous and maps saturated closed sets to closed sets.
\end{enumerate}
\end{prop}

\begin{proof}
\pf
\step{1}{$1 \Rightarrow 2$}
\begin{proof}
	\step{a}{\assume{$p$ is a quotient map.}}
	\step{b}{$p$ is continuous.}
	\step{c}{$p$ maps saturated open sets to open sets.}
	\begin{proof}
		\step{i}{\pflet{$U \subseteq X$ be a saturated open set.}}
		\step{2}{$\inv{p}(p(U)) = U$}
		\step{3}{$\inv{p}(p(U))$ is open in $X$.}
		\step{4}{$p(U)$ is open in $Y$.}
	\end{proof}
\end{proof}
\step{2}{$2 \Rightarrow 3$}
\begin{proof}
	\step{a}{\assume{$p$ is continuous and maps saturated open sets to open sets.}}
	\step{b}{\pflet{$C$ be a saturated closed set in $X$.}}
	\step{c}{$X - C$ is a saturated open set.}
	\step{d}{$Y - p(C)$ is open.}
	\step{e}{$p(C)$ is closed.}
\end{proof}
\step{3}{$3 \Rightarrow 1$}
\begin{proof}
	\step{a}{\assume{$p$ is continuous and maps closed sets to closed sets.}}
	\step{b}{\pflet{$C \subseteq Y$}}
	\step{c}{\assume{$\inv{p}(C)$ is closed in $X$.} \prove{$C$ is closed in $Y$.}}
	\step{d}{$\inv{p}(C)$ is saturated.}
	\step{e}{$p(\inv{p}(C))$ is closed.}
	\step{f}{$C$ is closed.}
\end{proof}
\qed
\end{proof}

\begin{cor}
Let $X$ and $Y$ be topological spaces. Let $p : X \rightarrow Y$ be continuous and surjective. If $p$ is either an open map or a closed map, then $p$ is a quotient map.
\end{cor}

\begin{ex}
The converse does not hold.

Let $A = \{ (x,y) \in \mathbb{R}^2 : x \geq 0 \vee y = 0 \}$. Then the first projection $\pi_1 : A \rightarrow \mathbb{R}$ is a quotient map that is neither an open map nor a closed map.
\end{ex}

\begin{proof}
\pf
\step{1}{$\pi_1$ is a quotient map.}
\begin{proof}
	\step{a}{\pflet{$U \subseteq \mathbb{R}$}}
	\step{b}{If $U$ is open then $\inv{\pi_1}(U)$ is open.}
	\begin{proof}
		\pf\ Since $\inv{\pi_1}(U) = (U \times \mathbb{R}) \cap A$.
	\end{proof}
	\step{c}{If $\inv{\pi_1}(U)$ is open then $U$ is open.}
	\begin{proof}
		\step{i}{\assume{$\inv{\pi_1}(U)$ is open.}}
		\step{ii}{\pflet{$x \in U$}}
		\step{iii}{$(x,0) \in \inv{\pi_1}(U)$}
		\step{iv}{\pick\ open neighbourhoods $V$ of $x$ and $W$ of $0$ such that $V \times W \subseteq \inv{\pi_1}(U)$}
		\step{v}{$V \subseteq U$}
		\begin{proof}
			\pf\ For all $x' \in V$ we have $(x',0) \in V \times W \subseteq \inv{\pi_1}(U)$.
		\end{proof}
	\end{proof}
\end{proof}
\step{2}{$\pi_1$ is not an open map.}
\begin{proof}
	\pf\ $\pi_1(((-1,1) \times (1,2)) \cap A) = [0,1)$ which is not open in $\mathbb{R}$.
\end{proof}
\step{3}{$\pi_1$ is not a closed map.}
\begin{proof}
	\pf\ $\pi_1(\{(x,1/x) \in \mathbb{R}^2 : x > 0 \}) = (0, + \infty)$ is not closed in $\mathbb{R}$.
\end{proof}
\qed
\end{proof}

\begin{cor}
\label{cor:product_quotient_maps}
Let $\{ X_i \}_{i \in I}$ and $\{ Y_i \}_{i \in I}$ be families of topological spaces and $p_i : X_i \rightarrow Y_i$ for all $i \in I$.
\begin{enumerate}
\item If every $p_i$ is an open quotient map, then $\prod_{i \in I} p_i : \prod_{i \in I} X_i \twoheadrightarrow \prod_{i \in I} Y_i$ is an open quotient map.
\item If every $p_i$ is a closed quotient map, then $\prod_{i \in I} p_i : \prod_{i \in I} X_i \twoheadrightarrow \prod_{i \in I} Y_i$ is a closed quotient map.
\end{enumerate}
\end{cor}

\begin{ex}
The product of two quotient maps is not necessarily a quotient map.

Let $Y$ be the quotient space of $\mathbb{R}_K$ obtained by collapsing the set $K$ to a point. Let $p : \mathbb{R}_K \twoheadrightarrow Y$ be the quotient map. Then $q \times q : \mathbb{R}_K^2 \rightarrow Y^2$ is not a quotient map.
\end{ex}

\begin{proof}
\pf
\step{1}{\pflet{$\Delta = \{(y,y) : y \in Y\}$}}
\step{2}{$Y$ is not Hausdorff.}
\begin{proof}
	\step{a}{\pflet{$*_K \in Y$ be the point such thta $q(K) = \{*_K\}$}}
	\step{a}{\assume{for a contradiction $U$ and $V$ are disjoint neighbourhoods of 0 and $*_K$}}
	\step{b}{$\inv{q}(U)$ and $\inv{q}(V)$ are disjoint open sets with $0 \in \inv{q}(U)$ and $K \subseteq \inv{q}(V)$}
	\qedstep
	\begin{proof}
		\pf\ This is a contradiction.
	\end{proof}
\end{proof}
\step{3}{$\Delta$ is not closed in $Y^2$.}
\step{4}{$\inv{(q \times q)}(\Delta)$ is closed in $\mathbb{R}_K^2$}
\begin{proof}
	\pf\ It is $\{ (x,x) : x \in \mathbb{R} \} \cup K^2$.
\end{proof}
\qed
\end{proof}

\begin{prop}
Let $\pi : X \twoheadrightarrow S$ be a quotient map. Let $Z$ be a topological space. Let $f : X \rightarrow Z$ be continuous. Then there exists a continuous map $g : S \rightarrow Z$ such that $f = g \circ \pi$ if and only if, for all $s \in S$, we have $f$ is constant on $\inv{\pi}(s)$.
\end{prop}

\begin{proof}
\pf\ From Theorem \ref{thm:quotient_topology_universal}. \qed
\end{proof}

\begin{prop}
Let $Z$ be a topological space. Define $\pi : [0,1] \rightarrow S^1$ by $\pi(t) = (\cos 2 \pi t, \sin 2 \pi t)$. Given any continuous function $f : S^1 \rightarrow Z$, we have $f \circ \pi$ is a loop in $Z$. This defines a bijection between $\Top[S^1,Z]$ and the set of loops in $Z$.
\end{prop}

\begin{proof}
\pf\ Since $\pi$ is a quotient map. \qed
\end{proof}

\begin{df}[Projective Space]
The \emph{projective space} $\mathbb{RP}^n$ is the quotient of $\mathbb{R}^{n+1} - \{0\}$ by $\sim$ where $x \sim \lambda x$ for all $x \in \mathbb{R}^{n+1} - \{0\}$ and $\lambda \in \mathbb{R}$.
\end{df}

\begin{df}[Torus]
The \emph{torus} $T$ is the quotient of $[0,1]^2$ by $\sim$ where $(x,0) \sim (x,1)$ and $(0,y) \sim (1,y)$.
\end{df}

\begin{df}[M\"{o}bius Band]
The \emph{M\"{o}bius band} is the quotient of $[0,1]^2$ by $\sim$ where $(0,y) \sim (1,1-y)$.
\end{df}

\begin{df}[Klein Bottle]
The \emph{Klein bottle} is the quotient of $[0,1]^2$ by $\sim$ where $(x,0) \sim (x,1)$ and $(0,y) \sim (1,1-y)$.
\end{df}

\begin{prop}
$\mathbb{RP}^2$ is the quotient of $[0,1]^2$ by $\sim$ where $(x,0) \sim (1-x,1)$ and $(0,y) \sim (1,1-y)$.
\end{prop}

\begin{proof}
\pf TODO
\end{proof}

\begin{ex}
Let $\{X_i\}_{i \in I}$ be a family of topological spaces and $\{Y_i\}_{i \in I}$ a family of sets. Let $q_i : X_i \twoheadrightarrow Y_i$ be a surjective function for all $i \in I$. Give each $Y_i$ the quotient topology. It is not true in general that the product topology on $\prod_{i \in I} Y_i$ is the same as the quotient topology induced by $\prod_{i \in I} q_i : \prod_{i \in I} X_i \twoheadrightarrow \prod_{i \in I} Y_i$.
\end{ex}

\begin{proof}
\pf
\step{1}{\pflet{$X^* = \mathbb{R} - \mathbb{Z}_+ + \{b\}$ be the quotient space obtained from $\mathbb{R}$ by identifying the subset $\mathbb{Z}_+$ to the point $b$.}}
\step{2}{\pflet{$p : \mathbb{R} \rightarrow X^*$ be the quotient map.} \prove{$p \times \id{\mathbb{Q}} : \mathbb{R} \times \mathbb{Q} \rightarrow X^* \times \mathbb{Q}$ is not a quotient map.}}
\step{3}{For $n \in \mathbb{Z}_+$, \pflet{$c_n = \sqrt{2} / n$}}
\step{4}{For $n \in \mathbb{Z}_+$, \pflet{$U_n = \{ (x,y) \in \mathbb{Q} \times \mathbb{R} : n-1/4 < x < n+1/4 \text{ and } ((y > x + c_n - n \text{ and } y > -x + c_n + n) \text{ or } (y < x + c_n - n \text{ and } y < -x + c_n + n))\}$}}
\step{5}{For all $n \in \mathbb{Z}_+$, $U_n$ is open in $\mathbb{R} \times \mathbb{Q}$}
\step{6}{For all $n \in \mathbb{Z}_+$ we have $\{n\} \times \mathbb{Q} \subseteq U_n$}
\step{7}{\pflet{$U = \bigcup_{n \in \mathbb{Z}_+} U_n$}}
\step{8}{$U$ is open in $\mathbb{R} \times \mathbb{Q}$.}
\step{9}{$U$ is saturated with respect to $p \times \id{\mathbb{Q}}$.}
\step{10}{\pflet{$U' = (p \times \id{\mathbb{Q}})(U)$}}
\step{11}{\assume{for a contradiction $U'$ is open in $X^* \times \mathbb{Q}$.}}
%TODO
\end{proof}


\begin{prop}
Let $X$ and $Y$ be topological spaces. Let $\sim$ be an equivalence relation on $X$. Let $\phi : Y \rightarrow X / \sim$.

Assume that, for all $y \in Y$, there exists a neighbourhood $U$ of $y$ and a continuous function $\Phi : U \rightarrow X$ such that $\pi \circ \Phi = \phi \restriction U$. Then $\phi$ is continuous.
\end{prop}

\begin{prop}
Let $X$ be a topological space and $\sim$ an equivalence relation on $X$. If $X / \sim$ is Hausdorff then every equivalence class of $\sim$ is closed in $X$.
\end{prop}

\begin{df}
Let $X$ be a topological space and $A_1, \ldots, A_r \subseteq X$. Then $X / A_1, \ldots, A_r$ is the quotient space of $X$ with respect to $\sim$ where $x \sim y$ iff $x = y$ or $\exists i (x \in A_i \wedge y \in A_i)$.
\end{df}

\begin{df}[Cone]
Let $X$ be a topological space. The \emph{cone over $X$} is the space $(X \times [0,1]) / (X \times \{1\})$.
\end{df}

\begin{df}[Suspension]
Let $X$ be a topological space. The \emph{suspension} of $X$ is the space
\[ \Sigma X := (X \times [-1,1]) / (X \times \{-1\}),(X \times \{1\}) \]
\end{df}

\begin{df}[Wedge Product]
Let $x_0 \in X$ and $y_0 \in Y$. The \emph{wedge product} $X \vee Y$ is $(X \times \{y_0\}) \cup (\{x_0\} \times Y)$ as a subspace of $X \times Y$.
\end{df}

\begin{df}[Smash Product]
Let $x_0 \in X$ and $y_0 \in Y$. The \emph{smash product} $X \wedge Y$ is $(X \times Y) / (X \vee Y)$.
\end{df}

\begin{ex}
$D^n / S^{n-1} \cong S^n$
\end{ex}

\begin{proof}
\pf
\step{1}{\pflet{$\phi : D^n / S^{n-1} \rightarrow S^n$ be the function induced by the map $D^n \rightarrow S^n$ that maps the radii of $D^n$ onto the meridians of $S^n$ from the north to the south pole.}}
\step{2}{$\phi$ is a bijection.}
\step{3}{$\phi$ is a homeomorphism.}
\begin{proof}
	\pf\ Since $D^n / S^{n-1}$ is compact and $S^n$ is Hausdorff.
\end{proof}
\qed
\end{proof}

\section{Box Topology}

\begin{df}[Box Topology]
Let $\{ X_i \}_{i \in I}$ be a family of topological spaces. The \emph{box topology} on $X = \prod_{i \in I} X_i$ is the topology generated by the basis $\mathcal{B} = \{ \prod_{i \in I} U_i: \{U_i\}_{i \in I} \text{ is a family with each } U_i \text{ an open set in } X_i \}$.

We prove this is a basis for a topology.
\end{df}

\begin{proof}
\pf
\step{1}{$\bigcup \mathcal{B} = X$}
\begin{proof}
	\pf\ Since $\prod_{i \in I} X_i \in \mathcal{B}$.
\end{proof}
\step{2}{For all $B_1,B_2 \in \mathcal{B}$ and $x \in B_1 \cap B_2$, there exists $B_3 \in \mathcal{B}$ such that $x \in B_3 \subseteq B_1 \cap B_2$.}
\begin{proof}
	\step{a}{\pflet{$B_1, B_2 \in \mathcal{B}$}}
	\step{b}{\pflet{$x \in B_1 \cap B_2$}}
	\step{c}{\pick\ a family $\{U_i\}_{i \in I}$ such that $B_1 = \prod_{i \in I} U_i$.}
	\step{d}{\pick\ a family $\{V_i\}_{i \in I}$ such that $B_2 = \prod_{i \in I} V_i$.}
	\step{e}{\pflet{$B_3 = \prod_{i \in I} (U_i \cap V_i)$}}
	\step{d}{$x \in B_3 \subseteq B_1 \cap B_2$}
\end{proof}
\qed
\end{proof}

\begin{prop}
The box topology is finer than the product topology.
\end{prop}

\begin{proof}
\pf\ Immediate from definitions. \qed
\end{proof}

\begin{prop}
On a finite family of topological spaces, the box topology and the product topology are the same.
\end{prop}

\begin{proof}
\pf\ Immediate from definitions. \qed
\end{proof}

\begin{prop}
The box topology is strictly finer than the product topology on the Hilbert cube.
\end{prop}

\begin{proof}
\pf\ The set $\prod_{n=0}^\infty (0, 1/(n+1)^2)$ is open in the box topology but not in the product topology. \qed
\end{proof}

\subsection{Bases}

\begin{prop}
Let $\{X_i\}_{i \in I}$ be a family of topological spaces. For all $i \in I$, let $\mathcal{B}_i$ be a basis for the topology on $X_i$. Then $\mathcal{B} = \left\{ \prod_{i \in I} B_i : \forall i \in I. B_i \in \mathcal{B}_i \right\}$ is a basis for the box topology on $\prod_{i \in I} X_i$.
\end{prop}

\begin{proof}
\pf
\step{1}{For every family $\{B_i\}_{i \in I}$ where $\forall i \in I. B_i \in \mathcal{B}_i$, we have $\prod_{i \in I} B_i$ is open in the box topology.}
\begin{proof}
	\pf\ Since each $B_i$ is open in $X_i$.
\end{proof}
\step{2}{For any open set $U$ in the box topology and $x \in U$, there exists $B \in \mathcal{B}$ such that $x \in B \subseteq U$.}
\begin{proof}
	\step{a}{\pflet{$U$ be a set open in the box topology.}}
	\step{b}{\pflet{$x \in U$}}
	\step{c}{\pick\ a family $\{U_i\}_{i \in I}$ where each $U_i$ is open in $X_i$ such that $x \in \prod_{i \in I} U_i \subseteq U$}
	\step{d}{For $i \in I$, choose $B_i \in \mathcal{B}_i$ such that $x_i \in B_i \subseteq U_i$.}
	\step{e}{$\prod_{i \in I} B_i \in \mathcal{B}$}
	\step{f}{$x \in \prod_{i \in I} B_i \subseteq \prod_{i \in I} U_i \subseteq U$}
\end{proof}
\qed
\end{proof}

\subsection{Subspaces}

\begin{prop}
Let $\{X_i\}_{i \in I}$ be a family of topological spaces. Let $Y_i$ be a subspace of $X_i$ for all $i \in I$. Then the box topology on $\prod_{i \in I} Y_i$ is the same as the subspace topology that $\prod_{i \in I} Y_i$ inherits as a subspace of $\prod_{i \in I} X_i$ under the box topology.
\end{prop}

\begin{proof}
\pf\ A basis for the box topology is
\begin{align*}
& \{ \prod_{i \in I} V_i : V_i \text{ open in } Y_i \} \\
= & \{ \prod_{i \in I} (U_i \cap Y_i) : U_i \text{ open in } X_i \} \\
& = \{ \prod_{i \in I} U_i \cap \prod_{i \in I} Y_i : U_i \text{ open in } X_i \}
\end{align*}
which is a basis for the subspace topology by Proposition \ref{prop:basis_subspace}. \qed
\end{proof}

\subsection{Closure}

\begin{prop}
Let $\{X_i\}_{i \in I}$ be a family of topological spaces. Give $\prod_{i \in I} X_i$ the box topology. Let $A_i \subseteq X_i$ for all $i \in I$. Then
\[ \prod_{i \in I} \overline{A_i} = \overline{\prod_{i \in I} A_i} \enspace . \]
\end{prop}

\begin{proof}
\pf
\step{1}{$\prod_{i \in I} \overline{A_i} \subseteq \overline{\prod_{i \in I} A_i}$}
\begin{proof}
	\step{a}{\pflet{$x \in \prod_{i \in I} \overline{A_i}$}}
	\step{b}{For any family $\{U_i\}_{i \in I}$ where each $U_i$ is open in $X_i$, if $x \in \prod_{i \in I} U_i$ then $\prod_{i \in I} U_i$ intersects $\prod_{i \in I} A_i$.}
	\begin{proof}
		\step{i}{\pflet{$\{U_i\}_{i \in I}$ be a family where each $U_i$ is open in $X_i$.}}
		\step{ii}{\assume{$x \in \prod_{i \in I}$}}
		\step{iii}{For all $i \in I$ we have $U_i$ intersects $A_i$}
		\begin{proof}
			\pf\ Since $\pi_i(x) \in \overline{A_i}$ and $U_i$ is a neighbourhood of $\pi_i(x)$.
		\end{proof}
		\step{iv}{$\prod_{i \in I} U_i$ intersects $\prod_{i \in I} A_i$}
\end{proof}		
	\step{c}{$x \in \overline{\prod_{i \in I} A_i}$}
	\begin{proof}
		\pf\ Proposition \ref{prop:closure_basis}.
	\end{proof}
\end{proof}
\step{2}{$\overline{\prod_{i \in I} A_i} \subseteq \prod_{i \in I} \overline{A_i}$}
\begin{proof}
	\step{a}{\pflet{$x \in \overline{\prod_{i \in I} A_i}$}}
	\step{b}{\pflet{$i \in I$} \prove{$\pi_i(x) \in \overline{A_i}$}}
	\step{c}{\pflet{$U$ be a neighbourhood of $\pi_i(x)$ in $X_i$}}
	\step{d}{$\inv{\pi_i}(U)$ is a neighbourhood of $x$ in $\prod_{i \in I} X_i$}
	\step{e}{\pick\ $y \in \inv{\pi_i}(U) \cap \prod_{i \in I} A_i$}
	\step{f}{$\pi_i(y) \in U \cap A_i$}
\end{proof}
\qed
\end{proof}

\section{Separations}

\begin{df}[Separation]
Let $X$ be a topological space. A \emph{separation} of $X$ is a pair $(U,V)$ of disjoint nonempty oped subsets in $X$ such that $U \cup V = X$.
\end{df}

\subsubsection{Subspaces}

\begin{prop}
\label{prop:separation_limit_points}
Let $X$ be a topological space and $Y$ a subspace of $X$. Then a separation of $Y$ is a pair $(A,B)$ of disjoint nonempty subsets of $Y$, neither of which contains a limit point of the other, such that $A \cup B = Y$.
\end{prop}

\begin{proof}
\pf\ Since the following are equivalent:
\begin{itemize}
\item Neither of $A$ and $B$ contains a limit point of the other.
\item $A$ contains all its own limit points in $Y$, and $B$ contains all its own limit points in $Y$.
\item $A$ and $B$ are closed in $Y$.
\end{itemize}
\qed
\end{proof}

\section{Connected Spaces}

\begin{df}[Connected]
A topological space is \emph{connected} iff it has no separation.
\end{df}

\subsection{The Real Numbers}

\begin{ex}
The space $\mathbb{R}_l$ is disconnected. The sets $(-\infty, 0)$ and $[0, +\infty)$ form a separation.
\end{ex}

\subsection{The Indiscrete Topology}

\begin{ex}
Any indiscrete space is connected.
\end{ex}

\subsection{The Cofinite Topology}

\begin{ex}
Any infinite set under the cofinite topology is connected.
\end{ex}

\begin{proof}
\pf
\step{1}{\pflet{$X$ be an infinite set under the cofinite topology.}}
\step{2}{\assume{for a contradiction $(C,D)$ is a separation of $X$.}}
\step{3}{$X = (X - C) \cup (X - D) \cup (C \cap D)$}
\qedstep
\begin{proof}
	\pf\ This is a contradiction since $X$ is infinite, $X - C$ and $X - D$ are finite, and $C \cap D = \emptyset$.
\end{proof}
\qed
\end{proof}

\begin{ex}
The rationals are disconnected. For any irrational $a$, we have $(- \infty, a) \cap \mathbb{Q}$ and $(a, +\infty) \cap \mathbb{Q}$ form a separation of $\mathbb{Q}$.
\end{ex}

\begin{ex}
$\mathbb{R}^\omega$ under the box topology is not connected. The set of bounded sequences and the set of unbounded sequences form a separation.
\end{ex}

\begin{prop}
A topological space $X$ is connected if and only if the only sets that are both open and closed are $\emptyset$ and $X$.
\end{prop}

\begin{proof}
\pf\ Since $(U,V)$ is a separation of $X$ iff $U$ is both open and closed and $V = X - U$. \qed
\end{proof}

\subsection{Finer and Coarser}

\begin{prop}
Let $\mathcal{T}$ and $\mathcal{T}'$ be topologies on the same set $X$. Assume $\mathcal{T} \subseteq \mathcal{T}'$. If $\mathcal{T}'$ is connected then $\mathcal{T}$ is connected.
\end{prop}

\begin{proof}
\pf\ If $(C,D)$ is a separation of $(X, \mathcal{T})$ then it is a separation of $(X, \mathcal{T}')$. \qed
\end{proof}

\subsection{Boundary}

\begin{prop}
Let $X$ be a topological space. Let $A \subseteq X$. Let $C$ be a connected subspace of $X$. If $C$ intersects $A$ and $X - A$ then $C$ intersects $\partial A$.
\end{prop}

\begin{proof}
\pf\ Otherwise $(C \cap \overline{A}, C \cap \overline{X - A})$ would be a separation of $C$. \qed
\end{proof}

\subsection{Continuous Functions}

\begin{prop}
The continuous image of a connected space is connected.
\end{prop}

\begin{proof}
\pf
\step{1}{\pflet{$X$ and $Y$ be topological spaces.}}
\step{2}{\pflet{$f : X \twoheadrightarrow Y$ be a surjective continuous function.}}
\step{3}{\pflet{$(C,D)$ be a separation of $Y$.}}
\step{4}{$(\inv{f}(C), \inv{f}(D))$ is a separation of $X$.}
\qed
\end{proof}

\subsection{Subspaces}

\begin{prop}
\label{prop:connected_subspace}
Let $X$ be a topological space. Let $(C,D)$ be a separation of $X$. Let $Y$ be a connected subspace of $X$. Then either $Y \subseteq C$ or $Y \subseteq D$.
\end{prop}

\begin{proof}
\pf\ Otherwise $(Y \cap C, Y \cap D)$ would be a separation of $Y$. \qed
\end{proof}

\begin{prop}
\label{prop:connected_union}
Let $X$ be a topological space. Let $\mathcal{A}$ be a set of connected subspaces of $X$ and $B$ a connected subspace of $X$. Assume that, for all $A \in \mathcal{A}$, we have $A \cap B \neq \emptyset$. Then $\bigcup \mathcal{A} \cup B$ is connected.
\end{prop}

\begin{proof}
\pf
\step{1}{\assume{for a contradiction $(C,D)$ is a separation of $\bigcup \mathcal{A} \cup B$.}}
\step{2}{\assume{w.l.o.g. $B \subseteq C$}}
\begin{proof}
	\pf\ Proposition \ref{prop:connected_subspace}.
\end{proof}
\step{4}{For all $A \in \mathcal{A}$ we have $A \subseteq C$}
\begin{proof}
	\pf\ Proposition \ref{prop:connected_subspace}.
\end{proof}
\step{5}{$D = \emptyset$}
\qedstep
\begin{proof}
	\pf\ This is a contradiction.
\end{proof}
\qed
\end{proof}

\begin{prop}
\label{prop:connected_closure}
Let $X$ be a topological space. Let $A$ be a connected subspace of $X$. Let $B$ be a subspace of $X$. If $A \subseteq B \subseteq \overline{A}$ then $B$ is connected.
\end{prop}

\begin{proof}
\pf
\step{1}{\assume{for a contradiction $(C,D)$ is a separation of $B$.}}
\step{2}{\assume{w.l.o.g. $A \subseteq C$}}
\begin{proof}
	\pf\ Proposition \ref{prop:connected_subspace}.
\end{proof}
\step{3}{$\overline{A} \subseteq \overline{C}$}
\step{4}{$\overline{C} \cap D = \emptyset$}
\step{5}{$B \cap D = \emptyset$}
\qedstep
\begin{proof}
	\pf\ This is a contradiction.
\end{proof}
\qed
\end{proof}

\begin{cor}
The topologist's sine curve is connected.
\end{cor}

\begin{proof}
\pf\ The set $\{ (x, \sin 1/x) : 0 < x \leq 1 \}$ is connected, since it is the continuous image of the connected set $(0,1]$. The topologist's sine curve is its closure, hence connected by Proposition \ref{prop:connected_closure}. \qed
\end{proof}

\begin{prop}
Let $X$ be a topological space. Let $(A_n)$ be a sequence of connected subspaces of $X$ such that, for all $n$, we have $A_n \cap A_{n+1} \neq \emptyset$. Then $\bigcup_n A_n$ is connected.
\end{prop}

\begin{proof}
\pf
\step{1}{\assume{for a contradiction $(C,D)$ is a separation of $\bigcup_n A_n$}}
\step{2}{\assume{w.l.o.g. $A_0 \subseteq C$}}
\begin{proof}
	\pf\ Proposition \ref{prop:connected_subspace}.
\end{proof}
\step{3}{$\forall n. A_n \subseteq C$}
\begin{proof}
	\step{a}{\assume{as induction hypothesis $A_n \subseteq C$}}
	\step{b}{\pick\ $x \in A_n \cap A_{n+1}$}
	\step{c}{$x \in C$}
	\step{d}{$A_{n+1} \subseteq C$}
	\begin{proof}
		\pf\ Proposition \ref{prop:connected_subspace}.
	\end{proof}
\end{proof}
\step{4}{$\bigcup_n A_n \subseteq C$}
\qedstep
\begin{proof}
	\pf\ This is a contradiction.
\end{proof}
\qed
\end{proof}

\begin{prop}
Let $X$ be a connected topological space. Let $Y \subseteq X$ be connected. Let $(A,B)$ be a separation of $X - Y$. Then $Y \cup A$ and $Y \cup B$ are connected.
\end{prop}

\begin{proof}
\pf
\step{1}{$Y \cup A$ is connected.}
\begin{proof}
	\step{a}{\assume{for a contradiction $(C,D)$ is a separation of $Y \cup A$}}
	\step{b}{\assume{w.l.o.g. $Y \subseteq C$}}
	\step{c}{\pick\ $C'$ and $D'$ open in $X$ such that $C = C' \cap (Y \cup A)$ and $D = D' \cap (Y \cup A)$}
	\step{d}{$D = D' \cap A$}
	\step{e}{$C' \cap D' \cap A = \emptyset$}
	\step{f}{$A \subseteq C' \cup D'$}
	\step{g}{\pick\ $A'$ and $B'$ open in $X$ such that $A = A' - Y$ and $B = B' - Y$}
	\step{h}{$A' \cap B' \subseteq Y$}
	\step{i}{$X - Y \subseteq A' \cup B'$}
	\step{j}{$A' \subseteq C' \cup D'$}
	\step{k}{$(D' \cap A', B' \cup C')$ is a separation of $X$.}
\end{proof}
\step{2}{$Y \cup B$ is connected.}
\begin{proof}
	\pf\ Similar.
\end{proof}
\qed
\end{proof}

\subsection{Order Topology}

\begin{prop}
\label{prop:continuum_connected}
Let $L$ be a linearly ordered set under the order topology. Then $L$ is connected if and only if $X$ is a linear continuum.
\end{prop}

\begin{proof}
\pf
\step{1}{If $L$ is a linear continuum then $L$ is connected.}
\begin{proof}
	\step{1}{\pflet{$L$ be a linear continuum.}}
	\step{2}{\assume{for a contradiction $(A,B)$ is a 	separation of $L$.}}
	\step{3}{\pick\ $a \in A$ and $b \in B$.}
	\step{4}{\assume{w.l.o.g. $a < b$}}
	\step{5}{\pflet{$c = \sup \{ x \in A : x < b \}$}}
	\step{6}{$c \notin A$}
	\begin{proof}
		\step{a}{\assume{for a contradiction $c \in A$.}}
		\step{b}{\pick\ $e > c$ such that $[c,e) \subseteq A$.}
		\step{c}{\pick\ $z$ such that $c < z < e$.}
		\step{d}{$z \in A$}
		\qedstep
		\begin{proof}
			\pf\ This contradicts \stepref{5}.
		\end{proof}
	\end{proof}
	\step{7}{$c \notin B$}
	\begin{proof}
		\step{a}{\assume{for a contradictis $c \in B$.}}
		\step{b}{\pick\ $d < c$ such that $(d,c] \subseteq B$.}
		\step{c}{\pick\ $z$ such that $d < z < c$}
		\step{d}{$z$ is an upper bound for $\{ x \in A : x < b \}$}
		\qedstep
		\begin{proof}
			\pf\ This contradicts \stepref{5}.
		\end{proof}
	\end{proof}
	\qedstep
	\begin{proof}
		\pf\ This is a contradiction.
	\end{proof}
\end{proof}
\step{2}{If $L$ is connected then $L$ is a linear continuum.}
\begin{proof}
	\step{a}{\assume{$L$ is connected.}}
	\step{b}{$L$ is dense.}
	\begin{proof}
		\step{i}{\pflet{$a,b \in L$ with $a < b$.}}
		\step{ii}{\assume{for a contradiction there is no $c$ such that $a < c < b$.}}
		\step{iii}{$((-\infty, b),(a, + \infty))$ is a separation of $L$.}
	\end{proof}
	\step{c}{$L$ has the least upper bound property.}
	\begin{proof}
		\step{i}{\assume{for a contradiction $S \subseteq L$ is a nonempty set bounded above with no least upper bound.}}
		\step{ii}{\pflet{$S \uparrow$ be the set of upper bounds for $S$.}}
		\step{iii}{\pflet{$S \uparrow \downarrow$ be the set of lower bounds for $S \uparrow$.} \prove{$(S \uparrow \downarrow, S \uparrow)$ is a separation of $L$.}}
		\step{iv}{$S \uparrow \neq \emptyset$}
		\begin{proof}
			\pf\ Since $S$ is bounded above.
		\end{proof}
		\step{v}{$S \uparrow \downarrow \neq \emptyset$}
		\begin{proof}
			\pf\ Since $\emptyset \neq S \subseteq S \uparrow \downarrow$.
		\end{proof}
		\step{vi}{$S \uparrow$ is open.}
		\begin{proof}
			\step{one}{\pflet{$u \in S \uparrow$}}
			\step{two}{\pick\ $v \in S \uparrow$ such that $v < u$}
			\begin{proof}
				\pf\ Since $u$ is not the least upper bound for $S$.
			\end{proof}
			\step{three}{$u \in (v, +\infty) \subseteq S \uparrow$}
		\end{proof}
		\step{vii}{$S \uparrow \downarrow$ is open.}
		\begin{proof}
			\step{one}{\pflet{$l \in S \uparrow \downarrow$}}
			\step{two}{$l \notin S \uparrow$}
			\begin{proof}
				\pf\ Since $l$ is not the least upper bound for $S$.
			\end{proof}
			\step{three}{\pick\ $s \in S$ such that $l < s$}
			\step{four}{$l \in (-\infty, s) \subseteq S \uparrow \downarrow$}
		\end{proof}
		\step{viii}{$S \uparrow \cap S \uparrow \downarrow \neq \emptyset$}
		\begin{proof}
			\pf\ An element of both would be a least upper bound for $S$.
		\end{proof}
		\step{ix}{$S \uparrow \cup S \uparrow \downarrow = L$}
		\begin{proof}
			\step{one}{\pflet{$x \in L$}}
			\step{two}{\assume{$x \notin S \uparrow$}}
			\step{three}{There exists $s \in S$ such that $x < s$.}
			\step{four}{$\forall u \in S \uparrow. x < u$}
			\step{five}{$x \in S \uparrow \downarrow$}
		\end{proof}
	\end{proof}
\end{proof}
\qed
\end{proof}

\begin{thm}[Intermediate Value Theorem]
Let $X$ be a connected space. Let $Y$ be a linearly ordered set under the order topology. Let $f : X \rightarrow Y$ be continuous. Let $a, b \in X$ and $r \in Y$. If $f(a) < r < f(b)$, then there exists $c \in X$ such that $f(c) = r$.
\end{thm}

\begin{proof}
\pf\ Otherwise $\{ x \in X : f(x) < r \}$ and $\{x \in X : f(x) > r \}$ would form a separation of $X$. \qed
\end{proof}

\begin{cor}
Every continuous function $[0,1] \rightarrow [0,1]$ has a fixed point.
\end{cor}

\begin{proof}
\pf
\step{1}{\pflet{$f : [0,1] \rightarrow [0,1]$ be continuous.}}
\step{2}{\pflet{$g : [0,1] \rightarrow [-1,1]$ be the function $g(x) = f(x) - x$.}}
\step{3}{$g(0) \geq 0$}
\step{4}{$g(1) \leq 0$}
\step{5}{There exists $x \in [0,1]$ such that $g(x) = 0$.}
\begin{proof}
	\pf\ Intermediate Value Theorem.
\end{proof}
\step{6}{There exists $x \in [0,1]$ such that $f(x) = x$.}
\qed
\end{proof}
\subsection{Product Topology}

\begin{prop}
The product of a family of connected spaces is connected.
\end{prop}

\begin{proof}
\pf
\step{0}{The product of two connected spaces is connected.}
\begin{proof}
\pf
\step{1}{\pflet{$X$ and $Y$ be connected topological spaces.}}
\step{3}{\assume{w.l.o.g. $X$ and $Y$ are nonempty.}}
\step{3}{\pick\ $(a,b) \in X \times Y$}
\step{4}{$X \times \{b\}$ is connected.}
\begin{proof}
	\pf\ It is homeomorphic to $X$.
\end{proof}
\step{5}{For all $x \in X$ we have $\{x\} \times Y$ is connected.}
\begin{proof}
	\pf\ It is homeomorphic to $Y$.
\end{proof}
\step{6}{For all $x \in X$ we have $(X \times \{b\}) \cup (\{x\} \times Y)$ is connected.}
\begin{proof}
	\pf\ Proposition \ref{prop:connected_union}.
\end{proof}
\step{7}{$X \cup Y$ is connected.}
\begin{proof}
	\pf\ Proposition \ref{prop:connected_union} since $X \cup Y = \bigcup_{x \in X} ((X \times \{b\}) \cup (\{x\} \times Y))$ and the subspaces all have the point $(a,b)$ in common.
\end{proof}
\end{proof}
\step{1}{\pflet{$\{X_i\}_{i \in I}$ be a family of connected spaces.}}
\step{2}{\pflet{$X = \prod_{i \in I} X_i$}}
\step{3}{\assume{w.l.o.g. each $X_i$ is nonempty.}}
\step{4}{\pick\ $a \in X$}
\step{5}{For every finite $K \subseteq I$, \pflet{$X_K = \{x \in X : \forall i \notin K. \pi_i(x) = \pi_i(a) \}$}}
\step{6}{For every finite $K \subseteq I$, we have $X_K$ is connected.}
\begin{proof}
	\pf\ It is homeomorphic to $\prod_{i \in K} X_i$ which is connected by \stepref{0}.
\end{proof}
\step{7}{\pflet{$Y = \bigcup_{K \text{ a finite subset of } I} X_K$}}
\step{8}{$Y$ is connected.}
\begin{proof}
	\pf\ Proposition \ref{prop:connected_union} since $a \in X_K$ for all $K$.
\end{proof}
\step{9}{$X = \overline{Y}$}
\begin{proof}
	\step{a}{\pflet{$x \in X$}}
	\step{b}{\pflet{$U$ be a neighbourhood of $x$.} \prove{$U$ intersects $Y$.}}
	\step{c}{\pick\ a finite subset $K$ of $I$ and $U_i$ open in each $X_i$ such that $U_i = X_i$ for all $i \notin K$, and $x \in \prod_i U_i \subseteq U$}
	\step{d}{\pflet{$y \in X$ be the point with $\pi_i(y) = \pi_i(x)$ for $i \in K$ and $\pi_i(y) = \pi_i(a)$ for $i \notin K$}}
	\step{e}{$y \in U \cap Y$}
\end{proof}
\step{10}{$X$ is connected.}
\begin{proof}
	\pf\ Proposition \ref{prop:connected_closure}.
\end{proof}
\qed
\end{proof}

\begin{prop}
Let $X$ and $Y$ be topological spaces. Let $A$ be a proper subset of $X$ and $B$ a proper subset of $Y$. Then $(X \times Y) - (A \times B)$ is connected.
\end{prop}

\begin{proof}
\pf
\step{1}{\pick\ $x_0 \in X - A$}
\step{2}{\pick\ $y_0 \in Y - B$}
\step{3}{\pflet{$C = ((X - A) \times Y) \cup (X \times \{y_0\})$}}
\step{4}{\pflet{$D = (\{x_0\} \times Y) \cup (X \times (Y - B))$}}
\step{5}{$C$ is connected.}
\begin{proof}
	\step{a}{$C = \bigcup_{x \in X - A} (\{x\} \times Y) \cup (X \times \{y_0\})$}
	\step{b}{For all $x \in X - A$ we have $\{x\} \times Y$ is connected.}
	\begin{proof}
		\pf\ It is homeomorphic to $Y$.
	\end{proof}
	\step{c}{$X \times \{y_0\}$ is connected.}
	\begin{proof}
		\pf\ It is homeomorphic to $X$.
	\end{proof}
	\step{d}{For all $x \in X - A$ we have $(x,y_0) \in (\{x\} \times Y) \cap (X \times \{y_0\})$}
	\step{e}{$C$ is connected.}
	\begin{proof}
		\pf\ Proposition \ref{prop:connected_union}.
	\end{proof}
\end{proof}
\step{6}{$D$ is connected.}
\begin{proof}
	\pf\ Similar.
\end{proof}
\step{7}{$(X \times Y) - (A \times B) = C \cup D$}
\step{8}{$(X \times Y) - (A \times B)$ is connected.}
\begin{proof}
	\pf\ Proposition \ref{prop:connected_union} since $(x_0,y_0) \in C \cap D$.
\end{proof}
\qed
\end{proof}

\subsection{Quotient Spaces}

\begin{prop}
A quotient of a connected space is connected.
\end{prop}

\begin{proof}
\pf
\step{1}{\pflet{$p : X \twoheadrightarrow Y$ be a quotient map.}}
\step{2}{If $(C,D)$ is a separation of $Y$ then $(\inv{p}(C), \inv{p}(D))$ is a separation of $X$.}
\qed
\end{proof}

\begin{prop}
Let $p : X \twoheadrightarrow Y$ be a quotient map. Assume that $Y$ is connected, for all $y \in Y$, we have $\inv{p}(y)$ is connected. Then $X$ is connected.
\end{prop}

\begin{proof}
\pf
\step{1}{\assume{for a contradiction $(A,B)$ is a separation of $X$.}}
\step{2}{For all $y \in Y$, either $\inv{p}(y) \subseteq A$ or $\inv{p}(y) \subseteq B$.}
\step{3}{$(\{y \in Y : \inv{p}(y) \subseteq A\}, \{y \in Y : \inv{p}(y) \subseteq B\})$ form a separation of $Y$.}
\qedstep
\begin{proof}
	\pf\ This is a contradiction.
\end{proof}
\qed
\end{proof}

\section{$T_1$ Spaces}

\begin{df}[$T_1$]
A topological space is $T_1$ iff every one-point set is closed.
\end{df}

\begin{prop}
A topological space is $T_1$ iff every finite set is closed.
\end{prop}

\begin{proof}
\pf\ Since the union of finitely many closed sets is closed. \qed
\end{proof}

\begin{prop}
Let $X$ be a topological space. Then $X$ is $T_1$ if and only if, for all $x,y \in X$, if $x \neq y$ then there exists a neighbourhood of $x$ that does not contain $y$, and there exists a neighbourhood of $y$ that does not contain $x$.
\end{prop}

\begin{proof}
\pf
\step{1}{If $X$ is $T_1$ then, for all $x,y \in X$, if $x \neq y$ then there exists a neighbourhood of $x$ that does not contain $y$, and there exists a neighbourhood of $y$ that does not contain $x$.}
\begin{proof}
	\step{a}{\assume{$X$ is $T_1$.}}
	\step{b}{\pflet{$x,y \in X$}}
	\step{c}{\assume{$x \neq y$}}
	\step{d}{$X - \{y\}$ is a neighbourhood of $x$ that does not contain $y$.}
	\step{e}{$X - \{x\}$ is a neighbourhood of $y$ that does not contain $x$.}
\end{proof}
\step{2}{If, for all $x,y \in X$, if $x \neq y$ then there exists a neighbourhood of $x$ that does not contain $y$, and there exists a neighbourhood of $y$ that does not contain $x$, then $X$ is $T_1$.}
\begin{proof}
	\step{a}{\assume{For all $x,y \in X$, if $x \neq y$ then there exists a neighbourhood of $x$ that does not contain $y$, and there exists a neighbourhood of $y$ that does not contain $x$.}}
	\step{b}{\pflet{$x \in X$} \prove{$\{x\}$ is closed.}}
	\step{c}{\pflet{$y \in X - \{x\}$}}
	\step{d}{\pick\ a neighbourhood $U$ of $y$ that does not contain $x$.}
	\step{e}{$y \in U \subseteq X - \{x\}$}
\end{proof}
\qed
\end{proof}

\subsection{Limit Points}

\begin{prop}
Let $X$ be a $T_1$ space. Let $A \subseteq X$ and $l \in X$. Then $l$ is a limit point of $A$ if and only if every neighbourhood of $l$ contains infinitely many points of $A$.
\end{prop}

\begin{proof}
\pf
\step{1}{If $l$ is a limit point of $A$ then every neighbourhood of $l$ contains infinitely many points of $A$.}
\begin{proof}
	\step{a}{\assume{$l$ is a limit point of $A$.}}
	\step{b}{\pflet{$U$ be a neighbourhood of $l$.}}
	\step{c}{\assume{for a contradiction $U \cap A - \{l\}$ is finite.}}
	\step{d}{$U \cap A - \{l\}$ is closed.}
	\begin{proof}
		\pf\ Since $X$ is $T_1$.
	\end{proof}
	\step{d}{$U - (A - \{l\})$ is a neighbourhood of $l$.}
	\step{e}{$U - (A - \{l\})$ intersects $A$.}
	\qedstep
\end{proof}
\step{2}{If every neighbourhood of $l$ contains infinitely many points of $A$ then $l$ is a limit point of $A$.}
\begin{proof}
	\pf\ Immediate from definitions.
\end{proof}
\qed
\end{proof}

\section{Hausdorff Spaces}

\begin{df}[Hausdorff]
A topological space is a \emph{Hausdorff} space or a \emph{$T_2$} space iff any two distinct points have disjoint neighbourhoods.
\end{df}

\begin{prop}
In a Hausdorff space, a sequence has at most one limit.
\end{prop}

\begin{proof}
\pf
\step{1}{\pflet{$X$ be a Hausdorff space.}}
\step{2}{\pflet{$(a_n)$ be a sequence in $X$ and $l,m \in X$}}
\step{3}{\assume{$a_n \rightarrow l$ and $a_n \rightarrow m$}}
\step{4}{\assume{for a contradiction $l \neq m$}}
\step{5}{\pick\ disjoint open sets $U$ and $V$ with $l \in U$ and $m \in V$}
\step{6}{\pick\ $M$, $N$ such that $\forall n \geq M. a_n \in U$ and $\forall n \geq N. a_n \in V$}
\step{7}{$a_{\max(M,N)} \in U \cap V$}
\qedstep
\begin{proof}
	\pf\ This contradicts the fact that $U \cap V = \emptyset$.
\end{proof}
\qed
\end{proof}

\begin{ex}
We cannot weaken the hypothesis from being Hausdorff to being $T_1$.

In the cofinite topology on any infinite set, every sequence converges to every point.
\end{ex}

\begin{prop}
Any linearly ordered set is Hausdorff under the order topology.
\end{prop}

\begin{proof}
\pf
\step{1}{\pflet{$X$ be a linearly ordered set under the order topology.}}
\step{2}{\pflet{$a,b \in X$ with $a \neq b$.}}
\step{3}{\assume{w.l.o.g. $a < b$.}}
\step{4}{\case{There exists $c \in X$ such that $a < c < b$.}}
\begin{proof}
	\step{i}{\pflet{$U = (-\infty, c)$}}
	\step{ii}{\pflet{$V = (c, + \infty)$}}
	\step{iii}{$U$ and $V$ are disjoint open sets with $a \in U$ and $b \in V$}
\end{proof}
\step{5}{\case{There is no $c \in X$ such that $a < c < b$.}}
\begin{proof}
	\step{i}{\pflet{$U = (-\infty, b)$}}
	\step{ii}{\pflet{$V = (a, + \infty)$}}
	\step{iii}{$U$ and $V$ are disjoint open sets with $a \in U$ and $b \in V$}
\end{proof}
\qed
\end{proof}

\begin{prop}
A subspace of a Hausdorff space is Hausdorff.
\end{prop}

\begin{proof}
\pf
\step{1}{\pflet{$X$ be a Hausdorff space.}}
\step{2}{\pflet{$Y$ be a subspace of $X$.}}
\step{3}{\pflet{$a,b \in Y$ with $a \neq b$.}}
\step{4}{\pick\ disjoint open sets $U$ and $V$ in $X$ with $a \in U$ and $b \in V$.}
\step{5}{$U \cap Y$ and $V \cap Y$ are disjoint open sets in $Y$ with $a \in U \cap Y$ and $b \in V \cap Y$.}
\qed
\end{proof}

\begin{prop}
The disjoint union of two Hausdorff spaces is Hausdorff.
\end{prop}

\begin{prop}
Let $A$ be a topological space and $B$ a Hausdorff space. Let $f,g : A \rightarrow B$ be continuous. Let $X \subseteq A$ be dense. If $f$ and $g$ agree on $X$, then $f = g$.
\end{prop}

\begin{proof}
\pf
\step{1}{\assume{for a contradiction $a \in A$ and $f(a) \neq g(a)$.}}
\step{2}{\pick\ disjoint neighbourhoods $U$ and $V$ of $f(a)$ and $g(a)$ respectively.}
\step{3}{\pick\ $x \in \inv{f}(U) \cap \inv{g}(V)$}
\step{4}{$f(x) = g(x) \in U \cap V$}
\qedstep
\begin{proof}
\pf\ This is a contradiction.
\end{proof}
\qed
\end{proof}

\subsection{Product Topology}

\begin{prop}
The product of a family of Hausdorff spaces is Hausdorff.
\end{prop}

\begin{proof}
\pf
\step{1}{\pflet{$\{X_i\}_{i \in I}$ be a family of Hausdorff spaces.}}
\step{2}{\pflet{$x,y \in \prod_{i \in I} X_i$ with $x \neq y$.}}
\step{3}{\pick\ $i \in I$ such that $\pi_i(x) \neq \pi_i(y)$}
\step{4}{\pick\ disjoint open sets $U$ and $V$ in $X_i$ such that $\pi_i(x) \in U$ and $\pi_i(y) \in V$.}
\step{5}{$x \in \inv{\pi_i}(U)$ and $y \in \inv{\pi_i}(V)$.}
\qed
\end{proof}

\subsection{Box Topology}

\begin{prop}
The box product of a family of Hausdorff spaces is Hausdorff.
\end{prop}

\begin{proof}
\pf
\step{1}{\pflet{$\{X_i\}_{i \in I}$ be a family of Hausdorff spaces.}}
\step{2}{\pflet{$x,y \in \prod_{i \in I} X_i$ with $x \neq y$.}}
\step{3}{\pick\ $i \in I$ such that $\pi_i(x) \neq \pi_i(y)$}
\step{4}{\pick\ disjoint open sets $U$ and $V$ in $X_i$ such that $\pi_i(x) \in U$ and $\pi_i(y) \in V$.}
\step{5}{$x \in \inv{\pi_i}(U)$ and $y \in \inv{\pi_i}(V)$.}
\qed
\end{proof}

\subsection{$T_1$ Spaces}

\begin{prop}
Every Hausdorff space is $T_1$.
\end{prop}

\begin{proof}
\pf
\step{1}{\pflet{$X$ be a Hausdorff space.}}
\step{2}{\pflet{$a \in X$} \prove{$X - \{a\}$ is open.}}
\step{3}{\pflet{$x \in X - \{a\}$}}
\step{4}{\pick\ disjoint open sets $U$ and $V$ with $a \in U$ and $x \in V$}
\step{5}{$x \in V \subseteq X - U \subseteq X - \{a\}$}
\qed
\end{proof}

\begin{ex}
The converse does not hold. If $X$ is an infinite set under the cofinite topology, then $X$ is $T_1$ but not Hausdorff.
\end{ex}

\begin{prop}
Let $X$ and $Y$ be metric spaces. Let $f : X \rightarrow Y$ be uniformly continuous. Let $\hat{X}$ and $\hat{Y}$ be the completions of $X$ and $Y$. Then $f$ extends uniquely to a continuous map $\hat{X} \rightarrow \hat{Y}$.
\end{prop}

\begin{proof}
\pf\ The extension maps $\lim_{n \rightarrow \infty} x_n$ to $\lim_{n \rightarrow \infty} f(x_n)$. \qed
\end{proof}

\begin{prop}
Let $X$ be a topological space. Then $X$ is Hausdorff if and only if the diagonal $\Delta = \{ (x,x) : x \in X \}$ is closed in $X^2$.
\end{prop}

\begin{proof}
\pf
\begin{align*}
& \Delta \text{ is closed} \\
\Leftrightarrow & X^2 - \Delta \text{ is open} \\
\Leftrightarrow & \forall x,y \in X ((x,y) \notin \Delta \Rightarrow \exists V,W \text{ open in } X (x \in V \wedge y \in W \wedge V \times W \subseteq X^2 - \Delta)) \\
\Leftrightarrow & \forall x,y \in X (x \neq y \Rightarrow \exists V,W \text{ open in } X (x \in V \wedge y \in W \wedge V \cap W = \emptyset)) \\
\Leftrightarrow & X \text{ is Hausdorff} & \qed
\end{align*}
\end{proof}

\section{Separable Spaces}

\begin{df}[Separable]
A topological space is \emph{separable} iff it has a countable dense subset.
\end{df}

Every second countable space is separable.

\section{Sequential Compactness}

\begin{df}[Sequentially Compact]
A topological space is \emph{sequentially compact} iff every sequence has a convergent subsequence.
\end{df}

\section{Compactness}

\begin{df}[Compact]
A topological space is \emph{compact} iff every open cover has a finite subcover.
\end{df}

\begin{prop}
Let $X$ be a compact topological space. Let $P$ be a set of open sets such that, for all $U,V \in P$, we have $U \cup V \in P$. Assume that every point has an open neighbourhood in $P$. Then $X \in P$.
\end{prop}

\begin{proof}
\pf
\step{1}{$P$ is an open cover of $X$}
\step{2}{\pick\ a finite subcover $U_1, \ldots, U_n \in P$}
\step{3}{$X = U_1 \cup \cdots \cup U_n \in P$}
\qed
\end{proof}

\begin{cor}
Let $f$ be a compact space and $f : X \rightarrow \mathbb{R}$ be locally bounded. Then $f$ is bounded.
\end{cor}

\begin{proof}
\pf\ Take $P = \{ U \text{ open in } X : f \text{ is bounded on } U \}$. \qed
\end{proof}

\begin{prop}
The continuous image of a compact space is compact.
\end{prop}

\begin{prop}
A closed subspace of a compact space is compact.
\end{prop}

\begin{prop}
Let $X$ and $Y$ be nonempty spaces. Then the following are equivalent.
\begin{enumerate}
\item $X$ and $Y$ are compact.
\item $X + Y$ is compact.
\item $X \times Y$ is compact.
\end{enumerate}
\end{prop}

\begin{prop}
A compact subspace of a Hausdorff space is closed.
\end{prop}

\begin{prop}
A continuous bijection from a compact space to a Hausdorff space is a homeomorphism.
\end{prop}

\begin{prop}
A first countable compact space is sequentially compact.
\end{prop}

\section{Gluing}

\begin{df}[Gluing]
Let $X$ and $Y$ be topological spaces, $X_0 \subseteq X$ and $\phi : X_0 \rightarrow Y$ a continuous map. Then $Y \cup_\phi X$ is the quotient space $(X + Y)/ \sim$, where $\sim$ is the equivalence relation generated by $x \sim \phi(x)$ for all $x \in X$.
\end{df}

\begin{prop}
$Y$ is a subspace of $Y \cup_\phi X$.
\end{prop}

\begin{df}
Let $X$ be a topological space and $\alpha : X \cong X$ a homeomorphism. Then $(X \times [0,1]) / \alpha$ is the quotient space of $X \times [0,1]$ by the equivalence relation generated by $(x,0) \sim (\alpha(x),1)$ for all $x \in X$.
\end{df}

\begin{df}[M\"{o}bius Strip]
The \emph{M\"{o}bius strip} is $([-1,1] \times [0,1])/ \alpha$ where $\alpha(x) = -x$.
\end{df}

\begin{df}[Klein Bottle]
The \emph{Klein bottle} is $(S^1 \times [0,1]) / \alpha$ where $\alpha(z) = \overline{z}$.
\end{df}

\begin{prop}
Let $M$ be the M\"{o}bius strip and $K$ the Klein bottle. Then $M \cup_{\id{\partial M}} M \cong K$.
\end{prop}

\begin{proof}
\pf
\step{1}{\pflet{$f : ([-1,1] \times [0,1]) + ([-1,1] \times [0,1]) \rightarrow S^1 \times [0,1]$ be the function that maps $\kappa_1(\theta,t)$ to $(e^{\pi i \theta / 2}, t)$ and $\kappa_2(\theta,t)$ to $(-e^{- \pi i \theta / 2}, t)$.}}
\step{2}{$f$ induces a bijection $M \cup_{\id{\partial M}} M \approx K$}
\step{3}{$f$ is a homeomorphism.}
\qed
\end{proof}

\section{Homogeneous Spaces}

\begin{df}[Homogeneous]
A topological space $X$ is \emph{homogeneous} iff, for any $x,y \in X$, there exists a homeomorphism $f : X \cong X$ such that $f(x) = y$.
\end{df}

\section{Regular Spaces}

\begin{df}[Regular]
A topological space $X$ is \emph{regular} iff it is $T_1$ and, for every closed set $A$ and point $x \notin A$, there exist disjoint open sets $U$ and $V$ with $A \subseteq U$ and $x \in V$.
\end{df}

\section{Totally Disconnected Spaces}

\begin{df}[Totally Disconnected]
A topological space $X$ is \emph{totally disconnected} iff the only connected subspaces are the one-point subspaces.
\end{df}

\begin{ex}
Every discrete space is totally disconnected.
\end{ex}

\begin{ex}
The rationals are totally disconnected.
\end{ex}

\section{Path Connected Spaces}

\begin{df}[Path-connected]
A topological space $X$ is \emph{path-connected} iff, for any points $a,b \in X$, there exists a continuous function $\alpha : [0,1] \rightarrow X$, called a \emph{path}, such that $\alpha(0) = a$ and $\alpha(1) = b$.
\end{df}

\subsection{The Ordered Square}

\begin{prop}
The ordered square is not path connected.
\end{prop}

\begin{proof}
\pf
\step{1}{\assume{for a contradiction $p : [a,b] \rightarrow I_o^2$ is a path from $(0,0)$ to $(1,1)$.}}
\step{2}{$p$ is surjective.}
\begin{proof}
	\pf\ Intermediate Value Theorem.
\end{proof}
\step{3}{For all $x \in [0,1]$, the set $\inv{p}(\{x\} \times (0,1))$ is a nonempty open set in $[0,1]$.}
\step{4}{For all $x \in [0,1]$ choose a rational $q_x \in \inv{p}(\{x\} \times (0,1))$.}
\step{5}{The mapping that maps $x$ to $q_x$ is an injective function $[0,1] \rightarrow \mathbb{Q}$}
\qedstep
\begin{proof}
	\pf\ This contradicts the fact that $[0,1]$ is uncountable and $\mathbb{Q}$ is countable.
\end{proof}
\qed
\end{proof}

\subsection{Punctured Euclidean Space}

\begin{prop}
For $n > 1$, the punctured Euclidean space $\mathbb{R}^n - \{0\}$ is path connected.
\end{prop}

\begin{proof}
\pf\ Given points $x$ and $y$, take the straight line from $x$ to $y$ if this does not pass through 0. Otherwise pick a point $z$ not on this line, and take the two straight lines from $x$ to $z$ then from $z$ to $y$. \qed
\end{proof}

\subsection{The Topologist's Sine Curve}

\begin{prop}
The topologist's sine curve is not path connected.
\end{prop}

\begin{proof}
\pf
\step{1}{\pflet{$S = \{ (x, \sin 1/x) : 0 < x \leq 1\}$}}
\step{2}{\assume{for a contradiction $p : [0,1] \rightarrow \overline{S}$ is a path from $(0,0)$ to $(1, \sin 1)$.}}
\step{3}{\pflet{$b$ be the largest element of $\inv{p}(\{0\} \times [-1,1])$}}
\step{4}{For $n$ a positive integer, choose $t_n$ such that $b < t_n < ((n-1)b+1)/n$ and $\pi_2(p(t_t)) = (-1)^n$}
\step{5}{$t_n \rightarrow b$ as $n \rightarrow \infty$}
\step{6}{$(p(t_n))$ does not converge.}
\qedstep
\begin{proof}
	\pf\ This is a contradiction.
\end{proof}
\qed
\end{proof}

\subsection{The Long Line}

\begin{prop}
The long line is path connected.
\end{prop}

\begin{proof}
\pf
\step{0}{\pflet{$L = S_\Omega \times [0,1)$ be the long line.}}
\step{2}{\pflet{$(a,b),(c,d) \in L$}}
\step{3}{\pick\ $e$ such that $a < e$ and $c < e$}
\step{4}{$(a,b),(c,d) \in [(0,0),(e,0)) \cong [0,1)$}
\begin{proof}
	\pf\ Using Proposition \ref{prop:long_line_01}.
\end{proof}
\step{5}{There is a path from $(a,b)$ to $(c,d)$.}
\qed
\end{proof}

\subsection{Continuous Functions}

\begin{prop}
The continuous image of a path connected space is path connected.
\end{prop}

\begin{proof}
\pf
\step{1}{\pflet{$X$ be a path connected space and $Y$ a topological space.}}
\step{2}{\pflet{$f : X \twoheadrightarrow Y$ be a surjective continuous function.} \prove{$Y$ is path connected.}}
\step{3}{\pflet{$a,b \in Y$}}
\step{4}{\pick\ $x, y \in X$ with $f(x) = a$ and $f(y) = b$.}
\step{5}{\pick\ a path $p : [0,1] \rightarrow X$ from $x$ to $y$.}
\step{6}{$f \circ p$ is a path from $a$ to $b$.}
\qed
\end{proof}

\subsection{Subspaces}

\begin{prop}
Let $\{X\}$ be a topological space. Let $\mathcal{A}$ be a set of connected subspaces of $X$. If $\bigcap \mathcal{A} \neq \emptyset$ then $\bigcup \mathcal{A}$ is connected.
\end{prop}

\begin{proof}
\pf
\step{1}{\pick\ $a \in \bigcap \mathcal{A}$}
\step{2}{\pick\ $x,y \in \bigcup \mathcal{A}$}
\step{3}{\pick\ $A, B \in \mathcal{A}$ with $x \in A$ and $y \in B$.}
\step{4}{\pick\ a path $p$ from $x$ to $a$ in $A$, and a path $q$ from $a$ to $y$ in $B$.}
\step{5}{The concatenation of $p$ and $q$ is a path from $x$ to $y$ in $\bigcup \mathcal{A}$.}
\qed
\end{proof}

\begin{prop}
A quotient of a path connected space is path connected.
\end{prop}

\subsection{Product Topology}

\begin{prop}
The product of a family of path connected spaces is path connected.
\end{prop}

\begin{proof}
\pf
\step{1}{\pflet{$\{X_i\}_{i \in I}$ be a family of path connected spaces.}}
\step{2}{\pflet{$x,y \in \prod_{i \in I} X_i$}}
\step{3}{For $i \in I$, \pick\ a path $p_i : [0,1] \rightarrow X_i$ from $\pi_i(x)$ to $\pi_i(y)$}
\step{4}{$\lambda t \in [0,1]. \lambda i \in I. p_i(t)$ is a path from $x$ to $y$ in $\prod_{i \in I} X_i$.}
\qed
\end{proof}

\begin{prop}
Let $A \subseteq \mathbb{R}^2$. If $A$ is countable then $\mathbb{R}^2 - A$ is path connected.
\end{prop}

\begin{proof}
\pf
\step{1}{\pflet{$x,y \in \mathbb{R}^2 - A$}}
\step{2}{\pick\ two non-parallel lines $L$ through $x$ and $L'$ through $y$ that do not pass through any points in $A$.}
\begin{proof}
	\pf\ These exist since uncountably many lines pass through any point.
\end{proof}
\step{3}{There exists a path from $x$ to $y$ that follows $L$ from $x$ to the point of intersection of $L$ and $L'$, and then follows $L'$ to $y$.}
\qed
\end{proof}

\subsection{Connected Spaces}

\begin{prop}
Every path connected space is connected.
\end{prop}

\begin{proof}
\pf
\step{1}{\pflet{$X$ be a path connected space.}}
\step{2}{\assume{for a contradiction $(A,B)$ is a separation of $X$.}}
\step{3}{\pick\ $a \in A$ and $b \in B$}
\step{4}{\pick\ a path $p : [0,1] \rightarrow X$ from $a$ to $b$.}
\step{5}{$(\inv{p}(A), \inv{p}(B))$ is a separation of $[0,1]$.}
\qedstep
\begin{proof}
	\pf\ This contradicts Proposition \ref{prop:continuum_connected}.
\end{proof}
\qed
\end{proof}

\begin{cor}
For $n > 1$, we have $\mathbb{R}^n$ and $\mathbb{R}$ are not homeomorphic.
\end{cor}

\begin{proof}
\pf\ Removing a point from $\mathbb{R}$ gives a disconnected space. \qed
\end{proof}

\begin{prop}
Every open connected subspace of $\mathbb{R}^2$ is path connected.
\end{prop}

\begin{proof}
\pf
\step{1}{\pflet{$U$ be an open connected subspace of $\mathbb{R}^2$.}}
\step{2}{\assume{w.l.o.g. $U \neq \emptyset$}}
\step{3}{\pick\ $x_0 \in U$}
\step{4}{\pflet{$V = \{ x \in U : \text{there exists a path from $x_0$ to $x$} \}$}}
\step{5}{$V$ is open in $U$.}
\begin{proof}
	\step{a}{\pflet{$x \in V$}}
	\step{b}{\pick\ $\epsilon > 0$ such that $B(x, \epsilon) \subseteq U$}
	\step{c}{$B(x, \epsilon) \subseteq V$}
	\begin{proof}
		\pf\ For all $y \in B(x,\epsilon)$, take a path from $x_0$ to $x$ and then a straight line from $x$ to $y$.
	\end{proof}
\end{proof}
\step{6}{$V$ is closed in $U$.}
\begin{proof}
	\step{a}{\pflet{$x \in U - V$}}
	\step{b}{\pick\ $\epsilon > 0$ such that $B(x,\epsilon) \subseteq U$}
	\step{c}{$B(x, \epsilon) \subseteq U - V$}
	\begin{proof}
		\step{i}{\pflet{$y \in B(x, \epsilon)$}}
		\step{ii}{There is a path from $y$ to $x$.}
		\step{iii}{There is no path from $x_0$ to $y$.}
	\end{proof}
\end{proof}
\step{7}{$V = U$}
\begin{proof}
	\pf\ $U$ is connected.
\end{proof}
\qed
\end{proof}

\section{Locally Homeomorphic}

\begin{df}
Let $X$ and $Y$ be topological spaces. Then $X$ is \emph{locally homeomorphic} to $Y$ if and only if every point in $X$ has a neighbourhood that is homeomorphic to an open set in $Y$.
\end{df}

\subsection{The Long Line}

\begin{prop}
The long line is locally homeomorphic to $[0,1)$.
\end{prop}

\begin{proof}
\pf\ By Proposition \ref{prop:long_line_01}. \qed
\end{proof}

\section{Components}

\begin{df}[(Connected) Component]
Let $X$ be a topological space. Define the equivalence relation $\sim$ on $X$ by: $x \sim y$ iff there exists a connected $C \subseteq X$ such that $x \in C$ and $y \in C$. The \emph{components} of $X$ are the equivalence classes with respect to $\sim$.

We prove this is an equivalence relation.
\end{df}

\begin{proof}
\pf
\step{1}{$\sim$ is reflexive.}
\begin{proof}
	\pf\ For any $x \in X$, we have $\{x\}$ is connected and $x \in \{x\}$, hence $x \sim x$.
\end{proof}
\step{2}{$\sim$ is symmetric.}
\begin{proof}
	\pf\ Immediate from definition.
\end{proof}
\step{3}{$\sim$ is transitive.}
\begin{proof}
	\step{a}{\assume{$x \sim y$ and $y \sim z$}}
	\step{b}{\pick\ connected subspaces $C$ and $D$ of $X$ with $x \in C$, $y \in C$, $y \in D$ and $z \in D$.}
	\step{c}{$C \cup D$ is connected.}
	\begin{proof}
		\pf\ Proposition \ref{prop:connected_union}.
	\end{proof}
	\step{d}{$x \in C \cup D$ and $z \in C \cup D$.}
	\step{e}{$x \sim z$}
\end{proof}
\qed
\end{proof}

\begin{ex}
The components of $\mathbb{Q}$ are the singleton subsets.
\end{ex}

\begin{ex}
The components of $\mathbb{R}_l$ are the singleton subsets.
\end{ex}

\begin{prop}
Every component of a topological space is connected.
\end{prop}

\begin{proof}
\pf
\step{1}{\pflet{$C$ be a component of the topological space $X$.}}
\step{2}{\assume{for a contradiction $(A,B)$ is a separation of $C$.}}
\step{3}{\pick\ $a \in A$ and $b \in B$.}
\step{4}{$a \sim b$}
\step{5}{\pick\ a connected subspace $D$ of $X$ such that $a \in D$ and $b \in D$.}
\step{6}{$D \subseteq C$}
\step{7}{$(A \cap D, B \cap D)$ is a separation of $D$.}
\qedstep
\begin{proof}
	\pf\ This is a contradiction.
\end{proof}
\qed
\end{proof}

\begin{prop}
\label{prop:connected_in_component}
Let $X$ be a topological space. Let $A$ be a nonempty connected subspace of $X$. Then there exists a unique component $C$ of $X$ such that $A \subseteq C$.
\end{prop}

\begin{proof}
\pf
\step{1}{\pick\ $a \in A$}
\step{2}{\pflet{$C$ be the $\sim$-equivalence class of $a$.}}
\step{3}{$A \subseteq C$}
\begin{proof}
	\pf\ For all $x \in A$ we have $a \sim x$ hence $x \in C$.
\end{proof}
\step{4}{For any component $C'$, if $A \subseteq C'$ then $C' = C$.}
\begin{proof}
	\pf\ Since the components are pairwise disjoint.
\end{proof}
\qed
\end{proof}

\begin{prop}
Every component of a topological space is closed.
\end{prop}

\begin{proof}
\pf
\step{1}{\pflet{$X$ be a topological space.}}
\step{2}{\pflet{$C$ be a component of $X$.}}
\step{3}{$\overline{C}$ is connected.}
\begin{proof}
	\pf\ Proposition \ref{prop:connected_closure}.
\end{proof}
\step{4}{$\overline{C} \subseteq C$}
\begin{proof}
	\pf\ Proposition \ref{prop:connected_in_component}.
\end{proof}
\step{5}{$C = \overline{C}$}
\qed
\end{proof}

\begin{cor}
If a topological space has only finitely many components, then its components are open.
\end{cor}

\section{Path Components}

\begin{df}[Path Component]
Let $X$ be a topological space. Define the equivalence relation $\sim$ on $X$ by: $x \sim y$ iff there exists a path from $x$ to $y$. The \emph{path components} of $X$ are the equivalence classes with respect to $\sim$.

We prove $\sim$ is an equivalence relation.
\end{df}

\begin{proof}
\pf
\step{1}{$\sim$ is reflexive.}
\begin{proof}
	\pf\ For any $a \in X$ the constant path at $a$ is a path from $a$ to $a$.
\end{proof}
\step{2}{$\sim$ is symmetric.}
\begin{proof}
	\pf\ If $p$ is a path from $a$ to $b$ then the reverse of $p$ is a path from $b$ to $a$.
\end{proof}
\step{3}{$\sim$ is transitive.}
\begin{proof}
	\pf\ If $p$ is a path from $a$ to $b$ and $q$ is a path from $b$ to $c$ then the concatenation of $p$ and $q$ is a path from $a$ to $c$.
\end{proof}
\qed
\end{proof}

\begin{ex}
The topologist's sine curve has two path components, namely $\{0\} \times [0,1]$ (which is closed and not open) and $\{ (x, \sin 1/x) : 0 < x \leq 1 \}$ (which is open and not closed).
\end{ex}

\begin{prop}
Every path component is path connected.
\end{prop}

\begin{proof}
\pf\ If $x$ and $y$ are in the same path component then $x \sim y$ so there is a path from $x$ to $y$. \qed
\end{proof}

\begin{cor}
Every path component is a subset of a component.
\end{cor}

\begin{prop}
Let $X$ be a topological space. Let $A$ be a nonempty path connected subspace of $X$. Then there exists a unique path component $C$ of $X$ such that $A \subseteq C$.
\end{prop}

\begin{proof}
\pf
\step{1}{\pick\ $a \in A$}
\step{2}{\pflet{$C$ be the path component of $a$.}}
\step{3}{$A \subseteq C$}
\begin{proof}
	\pf\ For all $x \in A$ we have $a \sim x$ (because $A$ is path connected) hence $x \in C$.
\end{proof}
\step{4}{For any path component $C'$, if $A \subseteq C'$ then $C = C'$.}
\begin{proof}
	\pf\ This holds because the path components are pairwise disjoint.
\end{proof}
\qed
\end{proof}

\begin{ex}
In $\mathbb{R}^\omega$ under the box topology, $\vec{x}$ and $\vec{y}$ are in the same component if and only if $\vec{x} - \vec{y}$ is eventually zero.
\end{ex}

\begin{proof}
\pf
\step{1}{\pflet{$B$ be the set of sequences that are eventually zero.}}
\step{2}{$B$ is connected.}
\begin{proof}
	\pf\ For $\vec{x} \in B$, the straight line path from 0 to $\vec{x}$ is continuous.
\end{proof}
\step{3}{$B$ is maximally connected.}
\begin{proof}
	\pf\ Since $(B, \mathbb{R}^\omega - B)$ form a separation of $\mathbb{R}^\omega$.
\end{proof}
\step{4}{For all $\vec{y} \in \mathbb{R}^\omega$, the component that contains $\vec{y}$ is $\{ \vec{x} \in \mathbb{R}^\omega : \vec{x} - \vec{y} \text{ is eventually zero} \}$.}
\begin{proof}
	\pf\ Since the function that maps $\vec{x}$ to $\vec{x} + \vec{y}$ is a homeomorphism of $\mathbb{R}^\omega$ with itself.
\end{proof}
\qed
\end{proof}


\section{Local Connectedness}

\begin{df}[Locally Connected]
Let $X$ be a topological space and $x \in X$. Then $X$ is \emph{locally connected} at $x$ iff, for every neighbourhood $U$ of $x$, there exists a connected neighbourhood $V$ of $x$ such that $V \subseteq U$.

The space $X$ is \emph{locally connected} iff it is locally connected at every point.
\end{df}

\begin{ex}
Every interval and ray in the real line is connected and locally connected.
\end{ex}

\begin{ex}
The space $[-1,0) \cup (0,1]$ is locally connected but not connected.
\end{ex}

\begin{ex}
The topologist's sine curve is connected but not locally connected.
\end{ex}

\begin{ex}
The rationals $\mathbb{Q}$ are neither connected nor locally connected.
\end{ex}

\begin{thm}
Let $X$ be a topological space. Then $X$ is locally connected if and only if, for every open set $U$ in $X$, every component of $U$ is open in $X$.
\end{thm}

\begin{proof}
\pf
\step{1}{If $X$ is locally connected then, for every open set $U$ in $X$, every component of $U$ is open in $X$.}
\begin{proof}
	\step{a}{\assume{$X$ is locally connected.}}
	\step{b}{\pflet{$U$ be an open set in $X$.}}
	\step{c}{\pflet{$C$ be a component of $U$.}}
	\step{d}{\pflet{$x \in C$}}
	\step{e}{\pick\ a connected neighbourhood $V$ of $x$ in $X$ such that $V \subseteq U$}
	\step{f}{$x \in V \subseteq C$}
\end{proof}
\step{2}{If, for every open set $U$ in $X$, every component of $U$ is open in $X$, then $X$ is locally connected.}
\begin{proof}
	\step{a}{\assume{For every open set $U$ in $X$, every component of $U$ is open in $X$.}}
	\step{b}{\pflet{$x \in X$}}
	\step{c}{\pflet{$U$ be a neighbourhood of $x$.}}
	\step{d}{\pflet{$V$ be the component of $U$ that contains $x$.}}
	\step{e}{$V$ is a connected neighbourhood of $x$ and $V \subseteq U$.}
\end{proof}
\qed
\end{proof}

\section{Local Path Connectedness}

\begin{df}[Locally Path Connected]
Let $X$ be a topological space and $x \in X$. Then $X$ is \emph{locally path connected} at $x$ iff, for every neighbourhood $U$ of $x$, there exists a path connected neighbourhood $V$ of $x$ such that $V \subseteq U$.

The space $X$ is \emph{locally path connected} iff it is locally connected at every point.
\end{df}

\begin{thm}
\label{thm:locally_path_connected_path_components_open}
Let $X$ be a topological space. Then $X$ is locally path connected if and only if, for every open set $U$ in $X$, every path component of $U$ is open in $X$.
\end{thm}

\begin{proof}
\pf
\step{1}{If $X$ is locally path connected then, for every open set $U$ in $X$, every path component of $U$ is open in $X$.}
\begin{proof}
	\step{a}{\assume{$X$ is locally path connected.}}
	\step{b}{\pflet{$U$ be an open set in $X$.}}
	\step{c}{\pflet{$C$ be a path component of $U$.}}
	\step{d}{\pflet{$x \in C$}}
	\step{e}{\pick\ a path connected neighbourhood $V$ of $x$ in $X$ such that $V \subseteq U$}
	\step{f}{$x \in V \subseteq C$}
\end{proof}
\step{2}{If, for every open set $U$ in $X$, every path component of $U$ is open in $X$, then $X$ is locally path connected.}
\begin{proof}
	\step{a}{\assume{For every open set $U$ in $X$, every path component of $U$ is open in $X$.}}
	\step{b}{\pflet{$x \in X$}}
	\step{c}{\pflet{$U$ be a neighbourhood of $x$.}}
	\step{d}{\pflet{$V$ be the path component of $U$ that contains $x$.}}
	\step{e}{$V$ is a path connected neighbourhood of $x$ and $V \subseteq U$.}
\end{proof}
\qed
\end{proof}

\begin{thm}
In a locally path connected space, the components are the same as the path components.
\end{thm}

\begin{proof}
\pf
\step{1}{\pflet{$X$ be a locally path connected space.}}
\step{2}{\pflet{$P$ be a path component of $X$.}}
\step{3}{\pflet{$C$ be the component that includes $P$.} \prove{$P = C$}}
\step{4}{\pflet{$Q$ be the union of all the path components of $C$ other than $P$.}}
\step{5}{$P$ and $Q$ are open in $C$.}
\begin{proof}
	\pf\ Theorem \ref{thm:locally_path_connected_path_components_open}.
\end{proof}
\step{6}{$P \cup Q = C$ and $P \cap Q = \emptyset$}
\step{7}{$Q = \emptyset$}
\begin{proof}
	\pf\ Otherwise $(P,Q)$ would be a separation of $C$.
\end{proof}
\step{8}{$P = C$}
\qed
\end{proof}

\begin{ex}
The converse does not hold. In $\mathbb{Q}$, the components are the same as the path components, namely the one-point sets, but $\mathbb{Q}$ is not locally path connected.
\end{ex}

\chapter{Metric Spaces}

%TODO Define real numbers
\begin{df}[Metric Space]
Let $X$ be a set and $d : X^2 \rightarrow \mathbb{R}$. We say $(X,d)$ is a \emph{metric space} iff:
\begin{itemize}
\item For all $x,y \in X$ we have $d(x,y) \geq 0$
\item For all $x,y \in X$ we have $d(x,y) = 0$ iff $x = y$
\item For all $x,y \in X$ we have $d(x,y) = d(y,x)$
\item (\emph{Triangle Inequality}) For all $x,y,z \in X$ we have $d(x,z) \leq d(x,y) + d(y,z)$
\end{itemize}
We call $d$ the \emph{metric} of the metric space $(X,d)$. We often write $X$ for the metric space $(X,d)$.
\end{df}

\begin{df}[Discrete Metric]
On any set $X$, define the \emph{discrete} metric by $d(x,y) = 0$ if $x = y$, 1 if $x \neq y$.
\end{df}

\begin{df}[Standard Metric]
The \emph{standard metric} on $\mathbb{R}$ is defined by $d(x,y) = |x-y|$.
\end{df}

\begin{df}[Square Metric]
The \emph{square metric} $\rho$ on $\mathbb{R}^n$ is defined by
\[ \rho(\vec{x}, \vec{y}) = \max(|x_1 - y_1|, \ldots, |x_n - y_n|) \enspace . \]

We prove this is a metric.
\end{df}

\begin{proof}
\pf
\step{1}{For all $\vec{x}, \vec{y} \in \mathbb{R}^n$ we have $\rho(\vec{x}, \vec{y}) \geq 0$.}
\begin{proof}
	\pf\ Immediate from definition.
\end{proof}
\step{2}{For all $\vec{x}, \vec{y} \in \mathbb{R}^n$ we have $\rho(\vec{x}, \vec{y}) = 0$ iff $\vec{x} = \vec{y}$.}
\begin{proof}
	\pf
	\begin{align*}
		\rho(\vec{x}, \vec{y}) = 0 & \Leftrightarrow \max(|x_1 - y_1|, \ldots, |x_n - y_n|) = 0 \\
		& \Leftrightarrow |x_1 - y_1| = \cdots = |x_n - y_n| = 0 \\
		& \Leftrightarrow x_1 = y_1 \wedge \cdots \wedge x_n = y_n \\
		& \Leftrightarrow \vec{x} = \vec{y}
	\end{align*}
\end{proof}
\step{3}{For all $\vec{x}, \vec{y} \in \mathbb{R}^n$ we have $\rho(\vec{x}, \vec{y}) = \rho(\vec{y}, \vec{x})$.}
\begin{proof}
	\pf\ Immediate from definition.
\end{proof}
\step{4}{For all $\vec{x}, \vec{y}, \vec{z} \in \mathbb{R}^n$ we have $\rho(\vec{x}, \vec{z}) \leq \rho(\vec{x}, \vec{y}) + \rho(\vec{y}, \vec{z})$.}
\begin{proof}
	\pf
	\begin{align*}
		& max(|x_1 - z_1|, \ldots, |x_n - z_n|) \\
		\leq & \max(|x_1 - y_1| + |y_1 - z_1|, \ldots, |x_n - y_n| + |y_n - z_n|) \\
		\leq & \max(|x_1 - y_1|, \ldots, |x_n - y_n|) + \max(|y_1 - z_1|, \ldots, |y_n - z_n|) \\
		= & \rho(\vec{x}, \vec{y}) + \rho(\vec{y}, \vec{z})
	\end{align*}
\end{proof}
\qed
\end{proof}

\subsection{Balls}

\begin{df}[Ball]
Let $X$ be a metric space. Let $x \in X$ and $r > 0$. The \emph{ball} with \emph{centre} $x$ and \emph{radius} $r$ is
\[ B(x,r) = \{ y \in X \mid d(x,y) < r \} \enspace . \]
\end{df}

\begin{df}[Closed Ball]
Let $X$ be a metric space. Let $x \in X$ and $r > 0$. The \emph{closed ball} with \emph{centre} $x$ and \emph{radius} $r$ is
\[ \overline{B(x,r)} = \{ y \in X \mid d(x,y) < r \} \enspace . \]
\end{df}

\begin{df}[Metric Topology]
Let $(X,d)$ be a metric space. The \emph{metric topology} on $X$ is the topology generated by the basis consisting of the balls.

We prove this is a basis for a topology.
\end{df}

\begin{proof}
\pf
\step{1}{Every point is a member of some ball.}
\begin{proof}
	\pf\ Since $x \in B(x,1)$.
\end{proof}
\step{2}{If $B_1$ and $B_2$ are balls and $x \in B_1 \cap B_2$, then there exists a ball $B_3$ such that $x \in B_3 \subseteq B_1 \cap B_2$.}
\begin{proof}
	\step{a}{\pflet{$x \in B(a,\epsilon_1) \cap B(b,\epsilon_2)$}}
	\step{b}{\pflet{$\epsilon = \min(\epsilon_1 - d(x,a), \epsilon_2 - d(x,b))$} \prove{$x \in B(x, \epsilon) \subseteq B(a,\epsilon_1) \cap B(b,\epsilon_2)$}}
	\step{c}{$B(x,\epsilon) \subseteq B(a, \epsilon_1)$}
	\begin{proof}
		\step{i}{\pflet{$y \in B(x, \epsilon)$}}
		\step{ii}{$d(y,a) < \epsilon_1$}
		\begin{proof}
			\pf
			\begin{align*}
				d(y,a) & \leq d(y,x) + d(x,a) & (\text{Triangle Inequality}) \\
				& < \epsilon + d(x,a) & (\text{\stepref{i}}) \\
				& \leq \epsilon_1 & (\text{\stepref{b}})
			\end{align*}
		\end{proof}
	\end{proof}
	\step{d}{$B(x,\epsilon) \subseteq B(b, \epsilon_2)$}
	\begin{proof}
		\pf\ Similar.
	\end{proof}
\end{proof}
\qed
\end{proof}

\begin{prop}
The discrete metric on a set $X$ induces the discrete topology.
\end{prop}

\begin{proof}
\pf\ Since $B(x, 1/2) = \{x\}$ for all $x \in X$. \qed
\end{proof}

\begin{prop}
The standard metric on $\mathbb{R}$ induces the standard topology.
\end{prop}

\begin{proof}
\pf
\step{1}{Every ball is open in the standard topology.}
\begin{proof}
	\pf\ Since $B(a, \epsilon) = (a - \epsilon, a + \epsilon)$.
\end{proof}
\step{2}{Every open ray is open in the metric topology.}
\begin{proof}
	\pf\ If $x \in (a, +\infty)$ then $x \in B(x,x-a) \subseteq (a, + \infty)$. Similarly for $(-\infty, a)$.
\end{proof}
\qed
\end{proof}

\begin{prop}
The square metric on $\mathbb{R}^n$ induces the product topology.
\end{prop}

\begin{proof}
\pf
\step{1}{For any real numbers $a_1$, \ldots, $a_n$, $b_1$, \ldots, $b_n$ with $a_1 < b_1$, \ldots, $a_n < b_n$, we have $(a_1,b_1) \times \cdots \times (a_n,b_n)$ is open in the metric topology.}
\begin{proof}
	\step{a}{\pflet{$\vec{x} \in (a_1,b_1) \times \cdots \times (a_n,b_n)$}}
	\step{b}{\pflet{$\epsilon = \min(x_1 - a_1, b_1 - x_1, \ldots, x_n - a_n, b_n - x_n)$}}
	\step{c}{$B(\vec{x},\epsilon) \subseteq (a_1,b_1) \times \cdots \times (a_n,b_n)$}
\end{proof}
\step{2}{For any $\vec{a} \in \mathbb{R}^n$ and $\epsilon > 0$, we have $B(\vec{a}, \epsilon)$ is open in the product topology.}
\begin{proof}
	\pf\ Since $B(\vec{a}, \epsilon) = (a_1 - \epsilon, a_1 + \epsilon) \times \cdots \times (a_n - \epsilon, a_n + \epsilon)$.
\end{proof}
\qed
\end{proof}

\begin{prop}
\label{prop:addition_continuous}
Addition is a continuous function $\mathbb{R}^2 \rightarrow \mathbb{R}$.
\end{prop}

\begin{proof}
\pf
\step{1}{\pflet{$(x,y) \in \mathbb{R}^2$ and $\epsilon > 0$}}
\step{2}{\pflet{$\delta = \epsilon / 2$}}
\step{3}{\pflet{$(x',y') \in \mathbb{R}^2$ with $\rho((x,y),(x',y')) < \delta$}}
\step{4}{$|x-x'|,|y-y'| < \delta$}
\step{5}{$|(x+y)-(x'+y')| < \epsilon$}
\begin{proof}
	\pf
	\begin{align*}
		|(x+y)-(x'+y')| & \leq |x-x'| + |y-y'| \\
		& < \delta + \delta & (\text{\stepref{4}}) \\
		& = \epsilon & (\text{\stepref{2}})
	\end{align*}
\end{proof}
\qed
\end{proof}

\begin{prop}
\label{prop:multiply_continuous}
Multiplication is a continuous function $\mathbb{R}^2 \rightarrow \mathbb{R}$.
\end{prop}

\begin{proof}
\pf
\step{1}{\pflet{$(x,y) \in \mathbb{R}^2$ and $\epsilon > 0$}}
\step{2}{\pflet{$\delta = \min(\epsilon / (|x| + |y| + 1), 1)$}}
\step{3}{\pflet{$(x',y') \in \mathbb{R}^2$ with $\rho((x,y),(x',y')) < \delta$}}
\step{4}{$|x-x'|,|y-y'| < \delta$}
\step{5}{$|xy-x'y'| < \epsilon$}
\begin{proof}
	\pf
	\begin{align*}
		|xy - x'y'| & = |xy - xy' + xy - x'y - xy + x'y + xy' - x'y'| \\
		& \leq |xy-xy'| + |xy-x'y| + |xy-x'y-xy'+xy'y|
		& = |x||y-y'| + |x-x'||y| + |x-x'||y-y'| \\
		& < |x|\delta + |y|\delta + \delta^2 & (\text{\stepref{4}}) \\
		& \leq |x|\delta + |y|\delta + \delta & (\text{\stepref{2}}) \\
		& = (|x| + |y| + 1) \delta \\
		& \leq \epsilon & (\text{\stepref{2}})
	\end{align*}
\end{proof}
\qed
\end{proof}

\begin{cor}
The unit circle $S^1$ is a closed subset of $\mathbb{R}^2$.
\end{cor}

\begin{proof}
\pf\ The function $f$ that maps $(x,y)$ to $x^2 + y^2$ is continuous, and $S^1 = \inv{f}(\{1\})$. \qed
\end{proof}

\begin{cor}
The unit ball $B^2$ is a closed subset of $\mathbb{R}^2$.
\end{cor}

\begin{proof}
\pf\ The function $f$ that maps $(x,y)$ to $x^2 + y^2$ is continuous, and $B^2 = \inv{f}([0,1])$. \qed
\end{proof}

\begin{prop}
Let $(a_n)$ and $(b_n)$ be sequences of real numbers. Let $c,s,t \in \mathbb{R}$. Assume
\[ \sum_{n=0}^\infty a_n = s \text{ and } \sum_{n=0}^\infty b_n = t \enspace . \]
Then
\[ \sum_{n=0}^\infty (c a_n + b_n) = cs + t \enspace . \]
\end{prop}

\begin{proof}
\pf
\[ \sum_{n=0}^N (c a_n + b_n) = c \sum_{n=0}^N a_n + \sum_{n=0}^N b_n \rightarrow cs + t \text{ as } n \rightarrow \infty \qquad \qed \]
\end{proof}

\begin{prop}[Comparison Test]
Let $(a_n)$ and $(b_n)$ be sequences of real numbers. Assume $|a_n| \leq b_n$ for all $n$. Assume $\sum_{n=0}^\infty b_n$ converges. Then $\sum_{n=0}^\infty a_n$ converges.
\end{prop}

\begin{proof}
\pf
\step{1}{For all $n$, \pflet{$c_n = |a_n| + a_n$}}
\step{2}{$\sum_{n=0}^\infty |a_n|$ converges.}
\begin{proof}
	\pf\ Since $(\sum_{n=0}^N |a_n|)_N$ is an increasing sequence of real numbers bounded above by $\sum_{n=0}^\infty b_n$.
\end{proof}
\step{3}{$\sum_{n=0}^\infty c_n$ converges.}
\begin{proof}
	\pf\ Since $(\sum_{n=0}^N c_n)_N$ is an increasing sequence of real numbers bounded above by $2 \sum_{n=0}^\infty a_n$.
\end{proof}
\step{4}{$\sum_{n=0}^\infty a_n$ converges.}
\begin{proof}
	\pf\ Since $a_n = c_n - |a_n|$.
\end{proof}
\qed
\end{proof}

\begin{prop}
\label{prop:metric_open}
Let $X$ be a metric space. Let $U \subseteq X$. Then $U$ is open if and only if, for all $x \in U$, there exists $\epsilon > 0$ such that $B(x, \epsilon) \subseteq U$.
\end{prop}

\begin{proof}
\pf
\step{1}{If $U$ is open then, for all $x \in U$, there exists $\epsilon > 0$ such that $B(x, \epsilon) \subseteq U$.}
\begin{proof}
	\step{a}{\assume{$U$ is open.}}
	\step{b}{\pflet{$x \in U$}}
	\step{c}{\pick\ a ball $B(a, \delta)$ such that $x \in B(a, \delta) \subseteq U$}
	\step{d}{\pflet{$\epsilon = \delta - d(a,x)$} \prove{$B(x, \epsilon) \subseteq U$}}
	\step{e}{\pflet{$y \in B(x, \epsilon)$}}
	\step{f}{$y \in B(a, \delta)$}
	\begin{proof}
		\pf
		\begin{align*}
			d(a,y) & \leq d(a,x) + d(x,y) & (\text{Triangle Inequality}) \\
			& < d(a,x) + \epsilon & (\text{\stepref{e}}) \\
			& = \delta
		\end{align*}
	\end{proof}
	\step{g}{$y \in U$}
	\begin{proof}
		\pf\ \stepref{c}
	\end{proof}
\end{proof}
\step{2}{If, for all $x \in U$, there exists $\epsilon > 0$ such that $B(x, \epsilon) \subseteq U$, then $U$ is open.}
\begin{proof}
	\pf\ Immediate from definition of the metric topology.
\end{proof}
\qed
\end{proof}

\begin{prop}
\label{prop:distance_between_distances}
Let $X$ be a metric space. Let $a,b,c \in X$. Then
\[ |d(a,b) - d(a,c)| \leq d(b,c) \enspace . \]
\end{prop}

\begin{proof}
\pf
\step{1}{$d(a,b) - d(a,c) \leq d(b,c)$}
\begin{proof}
	\pf\ Triangle Inequality.
\end{proof}
\step{2}{$d(a,c) - d(a,b) \leq d(b,c)$}
\begin{proof}
	\pf\ Triangle Inequality.
\end{proof}
\qed
\end{proof}

\begin{prop}
Let $(X,d)$ be a metric space. Then the metric topology on $X$ is the coarsest topology such that $d : X^2 \rightarrow \mathbb{R}$ is continuous.
\end{prop}

\begin{proof}
\pf
\step{1}{$d$ is continuous with respect to the metric topology.}
\begin{proof}
	\step{1}{\pflet{$(a,b) \in X^2$}}
	\step{2}{\pflet{$V$ be a neighbourhood of $d(a,b)$.}}
	\step{3}{\pick\ $\epsilon > 0$ such that $(d(a,b) - \epsilon, d(a,b) + \epsilon) \subseteq V$.}
	\step{4}{\pflet{$U = B(a,\epsilon / 2) \times B(b,	\epsilon / 2)$}}
	\step{5}{\pflet{$(x,y) \in U$}}
	\step{6}{$|d(x,y) - d(a,b)| < \epsilon$}
	\begin{proof}
		\pf
		\begin{align*}
			|d(x,y) - d(a,b)| & \leq |d(x,y) - d(a,y)| + |d(a,y) - d(a,b)| \\
			& \leq d(a,x) + d(b,y) & (\text{Proposition \ref{prop:distance_between_distances}}) \\
			& < \epsilon
		\end{align*}
	\end{proof}
	\step{7}{$d(x,y) \in V$}
\end{proof}
\step{2}{If $\mathcal{T}$ is a topology on $X$ with respect to which $d$ is continuous then $\mathcal{T}$ is finer than the metric topology.}
\begin{proof}
	\step{a}{\pflet{$\mathcal{T}$ be a topology on $X$ with respect to which $d$ is continuous.}}
	\step{b}{\pflet{$a \in X$ and $\epsilon > 0$.} \prove{$B(a,\epsilon) \in \mathcal{T}$}}
	\step{c}{\pflet{$x \in B(a,\epsilon)$}}
	\step{d}{$(a,x) \in \inv{d}((0, \epsilon))$}
	\step{e}{\pick\ $U,V \in \mathcal{T}$ such that $(a,x) \in U \times V \subseteq \inv{d}((0, \epsilon))$}
	\step{f}{$x \in V \subseteq B(a, \epsilon)$}
\end{proof}
\qed
\end{proof}

\begin{prop}
\label{prop:metric_finer}
Let $d$ and $d'$ be two metrics on the same set $X$. Let $\mathcal{T}$ and $\mathcal{T}'$ be the topologies they induce. Then $\mathcal{T} \subseteq \mathcal{T}'$ if and only if, for all $x \in X$ and $\epsilon > 0$, there exists $\delta > 0$ such that
\[ B_{d'}(x, \delta) \subseteq B_d(x, \epsilon) \enspace . \]
\end{prop}

\begin{proof}
\pf
\step{1}{If $\mathcal{T} \subseteq \mathcal{T}'$ then, for all $x \in X$ and $\epsilon > 0$, there exists $\delta > 0$ such that $B_{d'}(x, \delta) \subseteq B_d(x, \epsilon)$.}
\begin{proof}
	\step{a}{\assume{$\mathcal{T} \subseteq \mathcal{T}'$}}
	\step{b}{\pflet{$x \in X$ and $\epsilon > 0$}}
	\step{c}{$x \in B_d(x, \epsilon) \in \mathcal{T}'$}
	\step{d}{There exists $\delta > 0$ such that $B_{d'}(x, \delta) \subseteq B_d(x, \epsilon)$}
	\begin{proof}
		\pf\ Proposition \ref{prop:metric_open}.
	\end{proof}
\end{proof}
\step{2}{If, for all $x \in X$ and $\epsilon > 0$, there exists $\delta > 0$ such that $B_{d'}(x, \delta) \subseteq B_d(x, \epsilon)$, then $\mathcal{T} \subseteq \mathcal{T}'$.}
\begin{proof}
	\step{a}{\assume{For all $x \in X$ and $\epsilon > 0$, there exists $\delta > 0$ such that $B_{d'}(x, \delta) \subseteq B_d(x, \epsilon)$.}}
	\step{b}{\pflet{$U \in \mathcal{T}$}}
	\step{c}{For all $x \in U$, there exists $\delta > 0$ such that $B_{d'}(x, \delta) \subseteq U$}
	\begin{proof}
		\step{i}{\pflet{$x \in U$}}
		\step{ii}{\pick\ $\epsilon > 0$ such that $B_d(x, \epsilon) \subseteq U$}
		\begin{proof}
			\pf\ Proposition \ref{prop:metric_open}.
		\end{proof}
		\step{iii}{\pick\ $\delta > 0$ such that $B_{d'}(x, \delta) \subseteq B_d(x, \epsilon)$.}
		\begin{proof}
			\pf\ \stepref{a}
		\end{proof}
		\step{iv}{$B_{d'}(x, \delta) \subseteq U$}
	\end{proof}
	\step{d}{$U \in \mathcal{T}'$}
	\begin{proof}
		\pf\ Proposition \ref{prop:metric_open}.
	\end{proof}
\end{proof}
\qed
\end{proof}

\begin{df}[Metrizable]
A topological space is \emph{metrizable} iff there exists a metric that induces its topology.
\end{df}

\begin{prop}
$\mathbb{R}^2$ under the dictionary order is metrizable.
\end{prop}

\begin{proof}
\pf
\step{1}{\pflet{$d : (\mathbb{R}^2)^2 \rightarrow \mathbb{R}$ be defined by
\[ d((x_1,y_1),(x_2,y_2)) = \begin{cases}
\min(|y_2 - y_1|,1) & \text{if } x_1 = x_2 \\
1 & \text{if } x_1 \neq x_2
\end{cases} \]}}
\step{2}{$d$ is a metric.}
\begin{proof}
	\step{a}{For all $x,y \in \mathbb{R}^2$ we have $d(x,y) \geq 0$.}
	\begin{proof}
		\pf\ Immediate from definition.
	\end{proof}
	\step{b}{For all $x,y \in \mathbb{R}^2$ we have $d(x,y) = 0$ iff $x=y$.}
	\begin{proof}
		\pf\ Immediate from definition.
	\end{proof}
	\step{c}{For all $x,y \in \mathbb{R}^2$ we have $d(x,y) = d(y,x)$.}
	\begin{proof}
		\pf\ Immediate from definition.
	\end{proof}
	\step{d}{For all $x,y,z \in \mathbb{R}^2$ we have $d(x,z) \leq d(x,y) + d(y,z)$.}
	\begin{proof}
		\pf\ Easy.
	\end{proof}
\end{proof}
\step{3}{The metric topology induced by $d$ is finer than the order topology.}
\begin{proof}
	\step{a}{\pflet{$a,b \in \mathbb{R}^2$}}
	\step{b}{\pflet{$x \in (a,b)$}}
	\step{c}{\case{$\pi_1(x) = \pi_1(a) = \pi_1(b)$}}
	\begin{proof}
		\step{i}{\pflet{$\epsilon = \min(\pi_2(x) - \pi_2(a), \pi_2(b) - \pi_2(x))$}}
		\step{ii}{$B(x,\epsilon) \subseteq (a,b)$}
	\end{proof}
	\step{d}{\case{$\pi_1(a) = \pi_1(x) < \pi_1(b)$}}
	\begin{proof}
		\step{i}{\pflet{$\epsilon = \pi_2(x) - \pi_2(a)$}}
		\step{ii}{$B(x,\epsilon) \subseteq (a,b)$}
	\end{proof}
	\step{dd}{\case{$\pi_1(a) < \pi_1(x) = \pi_1(b)$}}
	\begin{proof}
		\pf\ Similar.
	\end{proof}
	\step{e}{\case{$\pi_1(a) < \pi_1(x) < \pi_1(b)$}}
	\begin{proof}
		\pf\ Then $B(x,\epsilon) \subseteq (a,b)$.
	\end{proof}
\end{proof}
\step{4}{The order topology is finer than the metric topology.}
\begin{proof}
	\pf\ Since $B((a,b),\epsilon) = ((a,b-\epsilon),(a,b+\epsilon))$ if $\epsilon \leq 1$, and $\mathbb{R}^2$ if $\epsilon > 1$.
\end{proof}
\qed
\end{proof}

Every metrizable space is first countable.

A metric space is compact if and only if it is sequentially compact.

A metric space is separable if and only if it is second countable.

\subsection{Subspaces}

\begin{prop}
Let $(X,d)$ be a metric space and $Y \subseteq X$. Then $d \restriction Y^2$ is a metric on $Y$ that induces the subspace topology.
\end{prop}

\begin{proof}
\pf
\step{1}{\pflet{$d' = d \restriction Y^2 : Y^2 \rightarrow \mathbb{R}$}}
\step{2}{$d'$ is a metric.}
\begin{proof}
	\pf\ Each of the axioms follows from the axiom in $X$.
\end{proof}
\step{3}{The metric topology induced by $d'$ is finer than the subspace topology.}
\begin{proof}
	\step{a}{\pflet{$U$ be open in $X$} \prove{$U \cap Y$ is open in the $d'$-topology.}}
	\step{b}{\pflet{$y \in U \cap Y$}}
	\step{c}{\pick\ $\epsilon > 0$ such that $B_d(y, \epsilon) \subseteq U$}
	\step{d}{$B_{d'}(y, \epsilon) \subseteq U \cap Y$}
\end{proof}
\step{4}{The subspace topology is finer than the metric topology induced by $d'$.}
\begin{proof}
	\step{a}{\pflet{$y \in Y$ and $\epsilon > 0$} \prove{$B_{d'}(y, \epsilon)$ is open in the subspace topology.}}
	\step{b}{$B_{d'}(y, \epsilon) = B_d(y, \epsilon) \cap Y$}
\end{proof}
\qed
\end{proof}

\subsection{Convergence}

\begin{prop}[Sequence Lemma]
Let $X$ be a metric space. Let $A \subseteq X$. Let $l \in \overline{A}$. Then there exists a sequence in $A$ that converges to $l$.
\end{prop}

\begin{proof}
\pf
\step{1}{For $n \in \mathbb{N}$, \pick\ $a_n \in B(l, 1/(n+1)) \cap A$.}
\step{2}{$a_n \rightarrow l$ as $n \rightarrow \infty$.}
\qed
\end{proof}

\begin{cor}
$\mathbb{R}^\omega$ under the box topology is not first countable.
\end{cor}

\begin{proof}
\pf
\step{1}{\pflet{$A$ be the set of all sequences of positive reals.}}
\step{2}{$0 \in \overline{A}$}
\step{3}{\pflet{$(a_n)$ be a sequence in $A$} \prove{$(a_n)$ does not converge to 0.}}
\step{4}{For all $n \in \mathbb{N}$, \pflet{$a_n = (x_{nm})$}}
\step{5}{\pflet{$B' = \prod_{n=0}^\infty (-x_{nn},x_{nn})$}}
\step{6}{$B'$ is open in the box topology.}
\step{6}{$0 \in B'$}
\step{7}{For all $n$ we have $a_n \notin B'$}
\qed
\end{proof}

\begin{cor}
If $J$ is an uncountable set then $\mathbb{R}^J$ under the product topology is not first countable.
\end{cor}

\begin{proof}
\pf
\step{1}{\pflet{$A = \{ x \in \mathbb{R}^J : \pi_j(x) = 1 \text{ for all but finitely many } j \in J \}$}}
\step{2}{$0 \in \overline{A}$}
\step{3}{\pflet{$(a_n)$ be a sequence in $A$.} \prove{$(a_n)$ does not converge to 0.}}
\step{4}{For $n \in \mathbb{N}$, \pflet{$J_n = \{ j \in J : \pi_j(a_n) \neq 1 \}$}}
\step{5}{$\bigcup_{n \in \mathbb{N}} J_n$ is countable.}
\step{6}{\pick\ $\beta \in J - \bigcup_{n \in \mathbb{N}} J_n$}
\step{7}{$\forall n \in \mathbb{N}. \pi_\beta(a_n) = 1$}
\step{8}{\pflet{$U = \inv{\pi_\beta}((-1,1))$}}
\step{9}{$0 \in U$}
\step{10}{$\forall n \in \mathbb{N}. a_n \notin U$}
\step{11}{$(a_n)$ does not converge to 0.}
\qed
\end{proof}

\subsection{Continuous Functions}

\begin{prop}
Let $X$ and $Y$ be metric spaces. Let $f : X \rightarrow Y$. Then $f$ is continuous if and only if, for all $x \in X$ and $\epsilon > 0$, there exists $\delta > 0$ such that, for all $y \in X$, if $d(x,y) < \delta$ then $d(f(x),f(y)) < \epsilon$.
\end{prop}

\begin{proof}
\pf
\step{1}{If $f$ is continuous then, for all $x \in X$ and $\epsilon > 0$, there exists $\delta > 0$ such that, for all $y \in X$, if $d(x,y) < \delta$ then $d(f(x),f(y)) < \epsilon$.}
\begin{proof}
	\step{a}{\assume{$f$ is continuous.}}
	\step{b}{\pflet{$x \in X$}}
	\step{c}{\pflet{$\epsilon > 0$}}
	\step{d}{$x \in \inv{f}(B(f(x),\epsilon)$}
	\step{e}{There exists $\delta > 0$ such that $B(x, \delta) \subseteq \inv{f}(B(f(x),\epsilon)$.}
\end{proof}
\step{2}{If, for all $x \in X$ and $\epsilon > 0$, there exists $\delta > 0$ such that, for all $y \in X$, if $d(x,y) < \delta$ then $d(f(x),f(y)) < \epsilon$, then $f$ is continuous.}
\begin{proof}
	\step{a}{\assume{For all $x \in X$ and $\epsilon > 0$, there exists $\delta > 0$ such that, for all $y \in X$, if $d(x,y) < \delta$ then $d(f(x),f(y)) < \epsilon$.}}
	\step{b}{\pflet{$V$ be open in $Y$}}
	\step{c}{\pflet{$x \in \inv{f}(V)$}}
	\step{d}{\pick\ $\epsilon > 0$ such that $B(f(x),\epsilon) \subseteq V$}
	\step{e}{\pick\ $\delta > 0$ such that, for all $y \in X$, if $d(x,y) < \delta$ then $d(f(x),f(y)) < \epsilon$.}
	\step{f}{$B(x,\delta) \subseteq \inv{f}(V)$}
\end{proof}
\qed
\end{proof}

\begin{prop}
Let $X$ be a metrizable space and $Y$ a topological space. Let $f : X \rightarrow Y$. Assume that, for every sequence $(x_n)$ in $X$ and $l \in X$, if $x_n \rightarrow l$ as $n \rightarrow \infty$ then $f(x_n) \rightarrow f(l)$ as $n \rightarrow \infty$. Then $f$ is continuous.
\end{prop}

\begin{proof}
\pf
\step{1}{\pflet{$A \subseteq X$} \prove{$f(\overline{A}) \subseteq \overline{f(A)}$}}
\step{2}{\pflet{$l \in \overline{A}$} \prove{$f(l) \in \overline{f(A)}$}}
\step{3}{\pick\ a sequence $(x_n)$ in $A$ such that $x_n \rightarrow l$ as $n \rightarrow \infty$.}
\step{4}{$f(x_n) \rightarrow f(l)$ as $n \rightarrow \infty$.}
\step{5}{$f(l) \in \overline{f(A)}$}
\qed
\end{proof}

\begin{prop}
\label{prop:inverse_continuous}
The function $i : \mathbb{R} - \{0\} \rightarrow \mathbb{R}$ that maps $x$ to $\inv{x}$ is continuous.
\end{prop}

\begin{proof}
\pf
\step{1}{\pflet{$a,b \in \mathbb{R}$ with $a < b$} \prove{$\inv{i}((a,b))$ is open.}}
\step{2}{\case{$0 < a$}}
\begin{proof}
	\pf\ $\inv{i}((a,b)) = (\inv{b}, \inv{a})$
\end{proof}
\step{3}{\case{$a = 0$}}
\begin{proof}
	\pf\ $\inv{i}((a,b)) = (\inv{b}, + \infty)$
\end{proof}
\step{4}{\case{$a < 0 < b$}}
\begin{proof}
	\pf\ $\inv{i}((a,b)) = (-\infty, \inv{a}) \cup (\inv{b}, +\infty)$
\end{proof}
\step{5}{\case{$b = 0$}}
\begin{proof}
	\pf\ $\inv{i}((a,b)) = (-\infty, \inv{a})$
\end{proof}
\step{6}{\case{$b < 0$}}
\begin{proof}
	\pf\ $\inv{i}((a,b)) = (\inv{b}, \inv{a})$
\end{proof}
\qed
\end{proof}

\begin{prop}
Subtraction is a continuous function $\mathbb{R}^2 \rightarrow \mathbb{R}$.
\end{prop}

\begin{proof}
\pf\ Since $a-b$ = $a + (-1)b$ and both addition and multiplication are continuous. \qed
\end{proof}

\begin{prop}
Division is a continuous function $\mathbb{R} \times (\mathbb{R} - \{0\}) \rightarrow \mathbb{R}$.
\end{prop}

\begin{proof}
\pf\ Since both multiplication and the function that maps $x$ to $\inv{x}$ are continuous. \qed
\end{proof}

\subsection{First Countable Spaces}

\begin{prop}
Every metrizable space is first countable.
\end{prop}

\begin{proof}
\pf\ For any point $x$, the set $\{ B(x, 1/n) : n \in \mathbb{Z}_+ \}$ is a countable basis at $x$. \qed
\end{proof}

\begin{cor}
$\mathbb{R}^\omega$ under the box topology is not metrizable.
\end{cor}

\begin{cor}
If $J$ is an uncountable set then $\mathbb{R}^J$ under the product topology is not metrizable.
\end{cor}

\subsection{Hausdorff Spaces}

\begin{prop}
Every metric space is Hausdorff.
\end{prop}

\begin{proof}
\pf
\step{1}{\pflet{$X$ be a metric space.}}
\step{2}{\pflet{$x,y \in X$ with $x \neq y$.}}
\step{3}{\pflet{$\epsilon = d(x,y)$}}
\step{4}{$B(x,\epsilon / 2)$ and $B(y, \epsilon / 2)$ are disjoint neighbourhoods of $x$ and $y$.}
\qed
\end{proof}

\subsection{Bounded Sets}

\begin{df}[Bounded]
Let $X$ be a metric space. Let $A \subseteq X$. Then $A$ is \emph{bounded} iff there exists $M$ such that $\forall x,y \in A. d(x,y) \leq M$. Its \emph{diameter} is then defined to be
\[ \diam A := \sup \{ d(x,y) : x,y \in A \} \enspace . \]
\end{df}

\subsection{Uniform Convergence}

\begin{df}[Uniform Convergence]
Let $X$ be a set and $Y$ a metric space. Let $(f_n)$ be a sequence of functions $X \rightarrow Y$, and $f : X \rightarrow Y$. Then $(f_n)$ \emph{converges uniformly} to $f$ iff, for all $\epsilon > 0$, there exists $N$ such that
\[ \forall n \geq N. \forall x \in X. d(f_n(x),f(x)) < \epsilon \enspace . \]
\end{df}

\begin{ex}
For $n \in \mathbb{N}$ define $f_n : [0,1] \rightarrow \mathbb{R}$ by $f_n(x) = x^n$. Define $f : [0,1] \rightarrow \mathbb{R}$ by $f(x) = 0$ for $x < 1$, $f(1) = 1$. Then $f_n$ converges pointwise to $f$, but does not converge uniformly to $f$.

We prove that, for all $N$, there exists $n \geq N$ and $x \in [0,1]$ such that $|x^n - f(x)| \geq 1/2$. Take $n = N$ and $x$ to be the $N$th root of $3/4$.
\end{ex}

\begin{ex}
For $n \in \mathbb{N}$, define $f_n : \mathbb{R} \rightarrow \mathbb{R}$ by
\[ f_n(x) = \frac{1}{n^3[x-(1/n)]^2 + 1} \enspace . \]
Then for all $x \in \mathbb{R}$ we have $f_n(x) \rightarrow 0$ as $n \rightarrow \infty$, but $(f_n)$ does not converge uniformly to 0.

We prove that, for all $N$, there exists $n \geq N$ and $x \in \mathbb{R}$ such that $|f_n(x)| \geq 1/2$. Take $n = N$ and $x = 1/N$. We have $f_N(1/N) = 1$.
\end{ex}

\begin{thm}[Uniform Limit Theorem]
Let $X$ be a topological space and $Y$ a metric space. Let $(f_n)$ be a sequence of functions $X \rightarrow Y$, and $f : X \rightarrow Y$. If every $f_n$ is continuous and $(f_n)$ converges uniformly to $f$, then $f$ is continuous.
\end{thm}

\begin{proof}
\pf
\step{1}{\pflet{$V$ be open in $Y$.}}
\step{2}{\pflet{$x_0 \in \inv{f}(V)$} \prove{There exists a neighbourhood $U$ of $x_0$ such that $f(U) \subseteq V$.}}
\step{3}{\pflet{$y_0 = f(x_0)$}}
\step{4}{\pick\ $\epsilon > 0$ such that $B(y_0, \epsilon) \subseteq V$.}
\step{5}{\pick\ $N$ such that $\forall n \geq N. \forall x \in X. d(f_n(x),f(x)) < \epsilon / 3$.}
\step{7}{\pick\ a neighbourhood $U$ of $x_0$ such that $f_N(U_2) \subseteq B(f_N(x_0), \epsilon / 3)$. \prove{$f(U) \subseteq V$}}
\step{9}{\pflet{$y \in U$}}
\step{10}{$d(f(y),y_0) < \epsilon$}
\begin{proof}
	\pf
	\begin{align*}
		d(f(y),y_0) & \leq d(f(y),f_N(y)) + d(f_N(y),f_N(x_0)) + d(f_N(x_0),y_0) \\
		& < \epsilon / 3 + \epsilon / 3 + \epsilon / 3 & (\text{\stepref{5}, \stepref{7}}) l\\
		& = \epsilon
	\end{align*}
\end{proof}
\step{11}{$f(y) in V$}
\begin{proof}
	\pf\ \stepref{4}
\end{proof}
\qed
\end{proof}

\begin{prop}
Let $X$ be a topological space. Let $Y$ be a metric space. Let $f_n$ be a sequence of functions $X \rightarrow Y$ and $f : X \rightarrow Y$. Let $x_n$ be a sequence of points in $X$ and $l \in X$. If $f_n$ converges uniformly to $f$, $x_n$ converges to $l$, and $f$ is continuous, then $f_n(x_n)$ converges to $f(l)$.
\end{prop}

\begin{proof}
\pf
\step{0}{$f$ is continuous.}
\step{1}{\pflet{$\epsilon > 0$}}
\step{3}{\pick\ $\delta > 0$ such that $\forall y \in X. d(y,l) < \delta \Rightarrow d(f(y),f(l)) < \epsilon / 2$}
\step{2}{\pick\ $N$ such that $\forall n \geq N. \forall x \in X. d(f_n(x),f(x)) < \epsilon / 2$ and $\forall n \geq N. d(x_n,l) < \delta$}
\step{3}{For all $n \geq N$ we have $d(f_n(x_n),f(l)) < \epsilon$}
\begin{proof}
	\pf
	\begin{align*}
		d(f_n(x_n),f(l)) & \leq d(f_n(x_n),f(x_n)) + d(f(x_n),f(l)) \\
		& < \epsilon / 2 + \epsilon / 2 \\
		& = \epsilon
	\end{align*}
\end{proof}
\qed
\end{proof}

\begin{thm}[Weierstrass $M$-Test]
Let $X$ be a set. Let $(f_n)$ be a sequence of functions $X \rightarrow \mathbb{R}$. Let $(M_n)$ be a sequence of real numbers. For $n \in \mathbb{N}$, let
\[ s_n(x) = \sum_{i=0}^n f_i(x) \enspace . \]
Assume that $\forall n \in \mathbb{N}. \forall x \in X. |f_n(x)| \leq M_n$. Assume that $\sum_{n=0}^\infty M_n$ converges. Then $(s_n)$ uniformly converges to $s$ where $s(x) = \sum_{n=0}^\infty f_n(x)$.
\end{thm}

\begin{proof}
\pf
\step{0}{For all $x \in X$ we have $\sum_{n=0}^\infty f_n(x)$ converges.}
\begin{proof}
	\pf\ By the Comparison Test.
\end{proof}
\step{1}{For $n \in \mathbb{N}$, \pflet{$r_n = \sum_{i=n+1}^\infty M_i$.}}
\step{2}{For all $k,n \in \mathbb{N}$ and $x \in X$, if $k > n$ then $|s_k(x) - s_n(x)| \leq r_n$.}
\begin{proof}
	\pf
	\begin{align*}
		|s_k(x) - s_n(x)| & = \left| \sum_{i=n+1}^k f_i(x) \right| \\
		& \leq \sum_{i=n+1}^k |f_i(x)| \\
		& \leq \sum_{i=n+1}^k M_i \\
		& \leq \sum_{i=n+1}^\infty M_i \\
		& = r_n
	\end{align*}
\end{proof}
\step{4}{For all $n \in \mathbb{N}$ we have $|s(x) - s_n(x)| \leq r_n$.}
\begin{proof}
	\pf\ Taking the limit $k \rightarrow \infty$ in \stepref{2}.
\end{proof}
\step{5}{$(s_n)$ converges uniformly to $s$.}
\begin{proof}
	\pf\ We have $\overline{\rho}(s_n,s) \leq r_n$ and so $\overline{\rho}(s_n,s) \rightarrow 0$ as $n \rightarrow \infty$ by the Sandwich Theorem.
\end{proof}
\qed
\end{proof}

\subsection{Standard Bounded Metric}

\begin{df}[Standard Bounded Metric]
Let $(X,d)$ be a metric space. The \emph{standard bounded metric} corresponding to $d$ is
\[ \overline{d}(x,y) := \min(d(x,y),1) \enspace . \]
\end{df}

\begin{prop}
The standard bounded metric associated with $d$ induces the same topology as $d$.
\end{prop}

\begin{proof}
\pf
\step{0}{\pflet{$(X,d)$ be a metric space.}}
\step{1}{Every $d$-ball is open under the topology induced by $\overline{d}$.}
\begin{proof}
	\step{a}{\pflet{$a \in X$ and $\epsilon > 0$}}
	\step{b}{\pflet{$x \in B_d(a, \epsilon)$}}
	\step{c}{\pflet{$\delta = \min(\epsilon - d(a,x), 1/2)$}}
	\step{d}{$B_{\overline{d}}(x, \delta) \subseteq B_d(a, \epsilon)$}
\end{proof}
\step{2}{Every $\overline{d}$-ball is open under the topology induced by $d$.}
\begin{proof}
	\pf\ Since $B_{\overline{d}}(a, \epsilon) = B_d(a, \epsilon)$ if $\epsilon \leq 1$, and $X$ if $\epsilon > 1$.
\end{proof}
\qed
\end{proof}

\subsection{Product Spaces}

\begin{prop}
The product of a countable family of metrizable spaces is metrizable.
\end{prop}

\begin{proof}
\pf
\step{1}{\pflet{$(X_n, d_n)$ be a sequence of metric spaces.}}
\step{2}{For $n \in \mathbb{N}$, \pflet{$\overline{d_n}$ be the standard bounded metric associated with $d_n$.}}
\step{3}{\pflet{$X = \prod_{n \in \mathbb{N}} X_n$}}
\step{4}{Define $D : X^2 \rightarrow \mathbb{R}$ by $D(x,y) = \sup_{n \in \mathbb{N}} \overline{d_n}(\pi_n(x), \pi_n(y)) / (n+1)$.}
\step{5}{$D$ is a metric on $X$.}
\begin{proof}
	\step{a}{For all $x,y \in X$ we have $D(x,y) \geq 0$.}
	\step{b}{For all $x,y \in X$ we have $D(x,y) = 0$ iff $x = y$.}
	\step{c}{For all $x,y \in X$ we have $D(x,y) = D(y,x)$.}
	\step{d}{For all $x,y,z \in X$ we have $D(x,z) \leq D(x,y) + D(y,z)$.}
\end{proof}
\step{6}{The product topology is finer than the metric topology induced by $D$.}
\begin{proof}
	\step{a}{\pflet{$a \in X$ and $\epsilon > 0$.}}
	\step{b}{\pflet{$x \in B(a, \epsilon)$}}
	\step{c}{\pflet{$\delta = \epsilon - D(a,x)$}}
	\step{c}{\pick\ $N \in \mathbb{N}$ such that $1 / (N+1) < \delta$}
	\step{d}{$x \in \prod_{n=0}^N B_{\overline{d_n}}(\pi_n(a), n \delta) \times \prod_{n=N+1}^\infty \subseteq B(a, \epsilon)$}
\end{proof}
\step{7}{The metric topology induced by $D$ is finer than the product topology.}
\begin{proof}
	\step{a}{\pflet{$n \in \mathbb{N}$ and $U$ be an open set in $X_n$.} \prove{$\inv{\pi_n}(U)$ is open in the metric topology.}}
	\step{b}{\pflet{$x \in \inv{\pi_n}(U)$}}
	\step{c}{\pick\ $\epsilon > 0$ such that $B_{\overline{d_n}}(\pi_n(x),\epsilon) \subseteq U$}
	\step{d}{$B(x, \epsilon / (n+1)) \subseteq \inv{\pi_n}(U)$}
\end{proof}
\qed
\end{proof}

\begin{df}
For $n \geq 1$, the \emph{unit ball} $B^n$ is the closed ball $\overline{B(0,1)}$ in $\mathbb{R}^n$ under the Euclidean metric.
\end{df}

\section{Uniform Metric}

\begin{df}[Uniform Metric]
Let $J$ be a nonempty set.
The \emph{uniform metric} $\overline{\rho}$ on $\mathbb{R}^J$ is defined by
\[ \overline{\rho}(x,y) = \sup_{j \in J} \overline{d}(x_j,y_j) \]
where $\overline{d}$ is the standard bounded metric associated with the standard metric on $\mathbb{R}$.

The topology it induces is called the \emph{uniform topology}.

We prove this is a metric.
\end{df}

\begin{proof}
\pf
\step{1}{For all $x,y \in \mathbb{R}^\omega$ we have $\overline{\rho}(x,y) \geq 0$.}
\begin{proof}
	\pf\ Pick $j_0 \in J$. Then
	\begin{align*}
		\overline{\rho}(x,y) & = \sup_j \overline{d}(x_j,y_j) \\
		& \geq \overline{d}(x_{j_0},y_{j_0}) \\
		& \geq 0
	\end{align*}
\end{proof}
\step{2}{For all $x,y \in \mathbb{R}^\omega$ we have $\overline{\rho}(x,y) = 0$ iff $x = y$.}
\begin{proof}
	\pf
	\begin{align*}
		\overline{\rho}(x,y) = 0 & \Leftrightarrow \sup_j \overline{d}(x_j,y_j) = 0 \\
		& \Leftrightarrow \forall j. \overline{d}(x_j,y_j) = 0 \\
		& \Leftrightarrow \forall j. x_j = y_j \\
		& \Leftrightarrow x = y
	\end{align*}
\end{proof}
\step{3}{For all $x,y \in \mathbb{R}^\omega$ we have $\overline{\rho}(x,y) = \overline{\rho}(y,x)$.}
\begin{proof}
	\pf
	\begin{align*}
		\overline{\rho}(x,y) & = \sup_j \overline{d}(x_j,y_j) \\
		& = \sup_j \overline{d}(y_j,x_j) \\
		& = \overline{\rho}(y,x)
	\end{align*}
\end{proof}
\step{4}{For all $x,y,z \in \mathbb{R}^\omega$ we have $\overline{\rho}(x,z) \leq \overline{\rho}(x,y) + \overline{\rho}(y,z)$.}
\begin{proof}
	\pf
	\begin{align*}
		\overline{\rho}(x,z) & = \sup_j \overline{d}(x_j,z_j) \\
		& \leq \sup_j (\overline{d}(x_j,y_j) + \overline{d}(y_j,z_j)) \\
		& \leq \sup_j \overline{d}(x_j,y_j) + \sup_j \overline{d}(y_j,z_j) \\
		& = \overline{\rho}(x,y) + \overline{\rho}(y,z)
	\end{align*}
\end{proof}
\qed
\end{proof}

\begin{prop}
The uniform topology is finer than the product topology. It is strictly finer iff $J$ is infinite.
\end{prop}

\begin{proof}
\pf
\step{1}{The uniform topology is finer than the product topology.}
\begin{proof}
	\step{a}{\pflet{$U$ be open in $\mathbb{R}$ and $j \in J$} \prove{$\inv{\pi_j}(U)$ is open in the uniform topology.}}
	\step{b}{\pflet{$x \in \inv{\pi_j}(U)$}}
	\step{c}{$\pi_j(x) \in U$}
	\step{d}{\pick\ $\epsilon > 0$ such that $B_{\overline{d}}(\pi_j(x), \epsilon) \subseteq U$}
	\step{e}{$B_{\overline{\rho}}(x, \epsilon) \subseteq \inv{\pi_j}(U)$}
\end{proof}
\step{2}{If $J$ is finite then the uniform topology is equal to the product topology.}
\begin{proof}
	\pf\ In $\mathbb{R}^n$, the uniform topology is the square topology.
\end{proof}
\step{3}{If $J$ is infinite then the uniform topology is not equal to the product topology.}
\begin{proof}
	\pf\ If $J$ is infinite then $B(0,1)$ is not open in the product topology.
\end{proof}
\qed
\end{proof}

\begin{prop}
The uniform topology is coarser than the box topology. It is strictly coarser iff $J$ is infinite.
\end{prop}

\begin{proof}
\pf
\step{1}{The uniform topology is coarser than the box topology.}
\begin{proof}
	\step{a}{\pflet{$U$ be open in the uniform topology.} \prove{$U$ is open in the box topology.}}
	\step{b}{\pflet{$x \in U$}}
	\step{c}{\pick\ $\epsilon > 0$ such that $B(x, \epsilon) \subseteq U$}
	\step{d}{$\prod_{j \in J} (x_j - \epsilon, x_j + \epsilon) \subseteq U$}
\end{proof}
\step{2}{If $J$ is finite then the uniform topology is equal to the box topology.}
\begin{proof}
	\pf\ On $\mathbb{R}^n$, the uniform metric is the square metric.
\end{proof}
\step{3}{If $J$ is infinite then the uniform topology is not equal to the box topology.}
\begin{proof}
	\step{a}{\assume{$J$ is infinite.}}
	\step{b}{\pick\ a sequence $(j_n)$ of distinct elements in $J$.}
	\step{c}{\pflet{$U = \prod_j U_j$ where $J_{j_n} = (-1/(n+1),1/(n+1))$ for $n \in \mathbb{N}$ and $J_j = (-1,1)$ for all other $j$.}}
	\step{d}{$U$ is not open in the uniform topology.}
\end{proof}
\qed
\end{proof}

\begin{prop}
The uniform topology on $\mathbb{R}^\infty$ is strictly finer than the product topology.
\end{prop}

\begin{proof}
\pf\ The set of all sequences $(x_n) \in \mathbb{R}^\infty$ such that $\forall n. |x_n| < 1$ is open in the uniform topology but not in the product topology. \qed
\end{proof}

\begin{prop}
The uniform topology on $\mathbb{R}^\infty$ is strictly coarser than the box topology.
\end{prop}

\begin{proof}
\pf\ The set of sequences $(x_n) \in \mathbb{R}^\infty$ such that $\forall n. |x_n| < 1/n$ is open in the box topology but not in the uniform topology. \qed
\end{proof}

\begin{prop}
The uniform topology on the Hilbert cube is the same as the product topology.
\end{prop}

\begin{proof}
\pf
\step{1}{\pflet{$(x_n)$ be in the Hilbert cube $H$ and $\epsilon > 0$.} \prove{$B((x_n), \epsilon) \cap H$ is open in the product topology.}}
\step{2}{\pick\ $N$ such that $1/N < \epsilon$}
\step{3}{$B((x_n), \epsilon) = (\prod_{n=0}^N (x_n - \epsilon, x_n + \epsilon) \times \prod_{n=N+1}^\infty [0,1/(n+1)]) \cap H$}
\qed
\end{proof}

\begin{cor}
The uniform topology on the Hilbert cube is strictly finer than the box topology.
\end{cor}

\begin{prop}
Let $X$ be a set and $Y$ a metric space. Let $(f_n)$ be a sequence of functions $X \rightarrow Y$, and $f : X \rightarrow Y$. Then $(f_n)$ converges uniformly to $f$ iff $(f_n)$ converges to $f$ in $Y^X$ under the uniform topology.
\end{prop}

\begin{proof}
\pf
\step{1}{If $(f_n)$ converges uniformly to $f$ then $(f_n)$ converges to $f$ in $Y^X$ under the uniform topology.}
\begin{proof}
	\step{a}{\assume{$(f_n)$ converges uniformly to $f$.}}
	\step{b}{\pflet{$\epsilon > 0$}}
	\step{c}{\pick\ $N$ such that $\forall n \geq N. \forall x \in X. d(f_n(x),f(x)) < \epsilon / 2$}
	\step{d}{$\forall n \geq N. \overline{\rho}(f_n,f) \leq \epsilon / 2$}
	\step{e}{$\forall n \geq N. \overline{\rho}(f_n,f) < \epsilon$}
\end{proof}
\step{2}{If $(f_n)$ converges to $f$ in $Y^X$ under the uniform topology then $(f_n)$ converges uniformly to $f$.}
\begin{proof}
	\step{a}{\assume{$(f_n)$ converges to $f$ in $Y^X$ under the uniform topology.}}
	\step{b}{\pflet{$\epsilon > 0$}}
	\step{c}{\pick\ $N$ such that $\forall n \geq N. \overline{\rho}(f_n,f) < \epsilon$}
	\step{d}{$\forall n \geq N. \forall x \in X. d(f_n(x),f(x)) < \epsilon$}
\end{proof}
\qed
\end{proof}

\begin{prop}
In $\mathbb{R}^\omega$ under the uniform topology, $\vec{x}$ and $\vec{y}$ lie in the same component if and only if $\vec{x} - \vec{y}$ is bounded.
\end{prop}

\begin{proof}
\pf
\step{1}{The set of bounded sequences form a component of $\mathbb{R}^\omega$.}
\begin{proof}
	\step{a}{\pflet{$B$ be the set of bounded sequences.}}
	\step{b}{$B$ is connected.}
	\begin{proof}
		\step{i}{\pflet{$\vec{x} \in B$} \prove{The straight line path $p : [0,1] \rightarrow \mathbb{R}^\omega$ from 0 to $\vec{x}$ is continuous.}}
		\step{ii}{\pflet{$t \in [0,1]$ and $\epsilon > 0$}}
		\step{iii}{\pick\ $B > 0$ such that $\forall n. |x_n| < B$}
		\step{iv}{\pflet{$\delta = \epsilon / B$}}
		\step{v}{\pflet{$s \in [0,1]$ with $|s-t| < \delta$}}
		\step{vi}{For all $n$ we have $|p(s)_n - p(t)_n| < \epsilon / 2$}
		\begin{proof}
			\pf
			\begin{align*}
				|p(s)_n - p(t)_n| & = |s-t||x_n| \\
				& < \delta B \\
				& = \epsilon
			\end{align*}
		\end{proof}
		\step{vii}{$\overline{\rho}(p(s),p(t)) \leq \epsilon / 2$}
		\step{viii}{$\overline{\rho}(p(s),p(t)) < \epsilon$}
	\end{proof}
	\step{c}{$B$ is maximally connected.}
	\begin{proof}
		\pf\ Since $(B, \mathbb{R}^\omega - B)$ form a separation of $\mathbb{R}^\omega$.
	\end{proof}
\end{proof}
\step{2}{For any $\vec{y} \in \mathbb{R}^\omega$, the component containing $\vec{y}$ is $\{ \vec{x} \in \mathbb{R}^\omega : \vec{x} - \vec{y} \text{ is bounded} \}$.}
\begin{proof}
	\pf\ Since the function that maps $\vec{x}$ to $\vec{x} + \vec{y}$ is a homeomorphism between $\mathbb{R}^\omega$ and itself.
\end{proof}
\qed
\end{proof}

\subsection{Products}

\begin{df}[Euclidean Metric]
Let $X$ and $Y$ be metric spaces. The \emph{Euclidean metric} on $X \times Y$ is
\[ d((x_1, y_1), (x_2, y_2)) = \sqrt{d(x_1,x_2)^2 + d(y_1,y_2)^2} \enspace .\]
We write $X \times Y$ for the set $X \times Y$ under this metric.

We prove this is a metric.
\end{df}

\begin{proof}
\pf
\step{1}{$d((x_1,y_1),(x_2,y_2)) \geq 0$}
\begin{proof}
	\pf\ Immediate from definition.
\end{proof}
\step{2}{$d((x_1,y_1),(x_2,y_2)) = 0$ iff $(x_1,y_1) = (x_2,y_2)$}
\begin{proof}
	\pf\ $\sqrt{d(x_1,x_2)^2 + d(y_1,y_2)^2} = 0$ iff $d(x_1,x_2) = d(y_1,y_2) = 0$ iff $x_1 = x_2$ and $y_1 = y_2$.
\end{proof}
\step{3}{$d((x_1,y_1),(x_2,y_2)) = d((x_2,y_2),(x_1,y_1))$}
\begin{proof}
	\pf\ Since $\sqrt{d(x_1,x_2)^2 + d(y_1,y_2)^2} = \sqrt{d(x_2,x_1)^2 + d(y_2,y_1)^2}$.
\end{proof}
\step{4}{The triangle inequality holds.}
\begin{proof}
	\pf
	\begin{align*}
		& (d((x_1,y_1),(x_2,y_2)) + d((x_2,y_2),(x_3,y_3)))^2 \\
		= & d((x_1,y_1),(x_2,y_2))^2 + 2 d((x_1,y_1),(x_2,y_2)) d((x_2,y_2),(x_3,y_3)) + d((x_2,y_2),(x_3,y_3))^2 \\
		= & d(x_1,x_2)^2 + d(y_1,y_2)^2 + 2 \sqrt{(d(x_1,x_2)^2 + d(y_1,y_2)^2)(d(x_2,x_3)^2 + d(y_2,y_3)^2)} + d(x_2,x_3)^2 + d(y_2,y_3)^2\\
		\geq & d(x_1,x_2)^2 + d(x_2,x_3)^2 + d(y_1,y_2)^2 + d(y_2,y_3)^ + 2(d(x_1,x_2)d(x_2,x_3) + d(y_1,y_2) d(y_2,y_3)) \\
		& (\text{Cauchy-Schwarz}) \\
		= & (d(x_1,x_2) + d(x_2,x_3))^2 + (d(y_1,y_2) + d(y_2,y_3))^2 \\
		\geq & d(x_1,x_3)^2 + d(y_1,y_3)^2 \\
		= & d((x_1,y_1),(x_3,y_3))^2
	\end{align*}
\end{proof}
\qed
\end{proof}

\begin{prop}
Let $X$ and $Y$ be metric spaces. The Euclidean metric on $X \times Y$ induces the product topology on $X \times Y$.
\end{prop}

\begin{proof}
\pf
\step{1}{Every open ball is open in the product topology.}
\begin{proof}
	\step{a}{\pflet{$(x,y) \in B((a,b),\epsilon)$} \prove{$B(x, \sqrt{\epsilon}) \times B(y, \sqrt{\epsilon}) \subseteq B((a,b),\epsilon)$}}
	\step{b}{\pflet{$x' \in B(x, \sqrt{(\epsilon - d((x,y),(a,b)))^2/2})$ and $y' \in B(y, \sqrt{(\epsilon - d((x,y),(a,b)))^2/2})$} \prove{$d((x',y'),(a,b)) < \epsilon$}}
	\step{c}{$d((x',y'),(x,y)) < \epsilon - d((x,y),(a,b))$}
	\begin{proof}
		\pf
		\begin{align*}
			d((x',y'),(x,y)) & = \sqrt{d(x',x)^2 + d(y',y)^2} \\
			& < \sqrt{(\epsilon - d((x,y),(a,b)))^2/2 + (\epsilon - d((x,y),(a,b))^2/2} \\
			& = \epsilon - d((x,y),(a,b))
		\end{align*}
	\end{proof}
	\step{d}{$d((x',y'),(a,b)) < \epsilon$}
	\begin{proof}
		\pf
		\begin{align*}
			d((x',y'),(a,b)) & \leq d((x',y'),(x,y)) + d((x,y),(a,b)) & (\text{Triangle Inequality}) \\
			& < \epsilon & (\text{\stepref{c}})
		\end{align*}
	\end{proof}
\end{proof}
\step{2}{If $U$ is open in $X$ and $V$ is open in $Y$ then $U \times V$ is open under the Euclidean metric.}
\begin{proof}
	\step{a}{\pflet{$(x,y) \in U \times V$}}
	\step{b}{\pick\ $\delta, \epsilon > 0$ such that $B(x,\delta) \subseteq U$ and $B(y,\epsilon) \subseteq V$ \prove{$(B((x,y),\min(\delta, \epsilon)) \subseteq U \times V$}}
	\step{c}{\pflet{$(x',y') \in B((x,y),\min(\delta, \epsilon))$}}
	\step{d}{$d(x',x) < \delta$}
	\begin{proof}
		\step{i}{$d((x',y'),(x,y)) < \min(\delta, \epsilon)$}
		\step{ii}{$d(x',x)^2 + d(y',y)^2 < \delta^2$}
		\step{iii}{$d(x',x)^2 < \delta^2$}
	\end{proof}
	\step{e}{$d(y',y) < \epsilon$}
	\begin{proof}
		\pf\ Similar.
	\end{proof}
	\step{f}{$(x',y') \in U \times V$}
\end{proof}
\qed
\end{proof}

\begin{prop}
The square metric on $\mathbb{R}^n$ induces the product topology.
\end{prop}

\begin{proof}
\pf
\step{1}{\pflet{$d$ be the Euclidean metric on $\mathbb{R}^n$ and $\rho$ the square metric.}}
\step{2}{For all $x \in X$ and $\epsilon > 0$, there exists $\delta > 0$ such that $B_d(x, \delta) \subseteq B_\rho(x, \epsilon)$}
\begin{proof}
	\pf\ If $d(x,y) < \epsilon$ then $\rho(x,y) < \epsilon$.
\end{proof}
\step{3}{For all $x \in X$ and $\epsilon > 0$, there exists $\delta > 0$ such that $B_\rho(x, \delta) \subseteq B_d(x, \epsilon)$}
\begin{proof}
	\pf\ If $\rho(x,y) < \epsilon / \sqrt{n}$ then $d(x,y) < \epsilon$.
\end{proof}
\step{4}{$d$ and $\rho$ induce the same topology.}
\begin{proof}
	\pf\ Proposition \ref{prop:metric_finer}.
\end{proof}
\qed
\end{proof}

\subsection{Connected Spaces}

\begin{ex}
The space $\mathbb{R}^\omega$ under the uniform topology is disconnected. The set of bounded sequences and the set of unbounded sequences form a separation.
\end{ex}

\section{Isometric Embeddings}

\begin{df}[Isometric Embedding]
Let $X$ and $Y$ be metric spaces. Let $f : X \rightarrow Y$. Then $f$ is an \emph{isometric embedding} of $X$ in $Y$ iff, for all $x,y \in X$, we have $d(f(x),f(y)) = d(x,y)$.
\end{df}

\begin{prop}
Every isometric embedding is an embedding.
\end{prop}

\begin{proof}
\pf
\step{1}{\pflet{$X$ and $Y$ be metric spaces.}}
\step{2}{\pflet{$f : X \rightarrow Y$ be an isometric embedding.}}
\step{3}{$f$ is injective.}
\step{4}{The subspace topology induced by $f$ is finer than the metric topology.}
\begin{proof}
	\step{a}{\pflet{$x \in X$ and $\epsilon > 0$} \prove{$B(x,\epsilon)$ is open in the subspace topology.}}
	\step{b}{$B(x, \epsilon) = \inv{f}(B(f(x),\epsilon))$}
\end{proof}
\step{5}{The metric topology is finer than the subspace topology induced by $f$.}
\begin{proof}
	\step{a}{\pflet{$V$ be open in $Y$} \prove{$\inv{f}(V)$ is open in $X$}}
	\step{b}{\pflet{$x \in \inv{f}(V)$}}
	\step{c}{\pick\ $\epsilon > 0$ such that $B(f(x), \epsilon) \subseteq V$}
	\step{d}{$B(x, \epsilon) \subseteq \inv{f}(V)$}
\end{proof}
\qed
\end{proof}

\section{Complete Metric Spaces}

\begin{df}[Complete]
A metric space is \emph{complete} iff every Cauchy sequence converges.
\end{df}

\begin{ex}
$\mathbb{R}$ is complete.
\end{ex}

\begin{prop}
The product of two complete metric spaces is complete.
\end{prop}

\begin{prop}
Every compact metric space is complete.
\end{prop}

\begin{prop}
Let $X$ be a complete metric space and $A \subseteq X$. Then $A$ is complete if and only if $A$ is closed.
\end{prop}

\begin{df}[Completion]
Let $X$ be a metric space. A \emph{completion} of $X$ is a complete metric space $\hat{X}$ and injection $i : X \rightarrowtail \hat{X}$ such that:
\begin{itemize}
\item The metric on $X$ is the restriction of the metric on $\hat{X}$
\item $X$ is dense in $\hat{X}$.
\end{itemize}
\end{df}

\begin{prop}
Let $i_1 : X \rightarrow Y_1$ and $i_2 : X \rightarrow Y_2$ be completions of $X$. Then there exists a unique isometry $\phi : Y_1 \cong Y_2$ such that $\phi \circ i_1 = i_2$.
\end{prop}

\begin{proof}
\pf\ Define $\phi(\lim_{n \rightarrow \infty} i_1(x_n)) = \lim_{n \rightarrow \infty} i_2(x_n)$. \qed
\end{proof}

\begin{thm}
Every metric space has a completion.
\end{thm}

\begin{proof}
\pf\ Let $\hat{X}$ be the set of Cauchy sequences in $X$ quotiented by $\sim$ where $(x_n) \sim (y_n)$ if and only if $d(x_n, y_n) \rightarrow 0$. \qed
\end{proof}

\section{Manifolds}

\begin{df}[Manifold]
An \emph{$n$-dimensional manifold} is a second countable Hausdorff space locally homeomorphic to $\mathbb{R}^n$.
\end{df}

\chapter{Homotopy Theory}

\section{Homotopies}

\begin{df}[Homotopy]
Let $X$ and $Y$ be topological spaces. Let $f,g : X \rightarrow Y$ be continuous. A \emph{homotopy} between $f$ and $g$ is a continuous function $h : X \times [0,1] \rightarrow Y$ such that
\begin{itemize}
\item $\forall x \in X. h(x,0) = f(x)$
\item $\forall x \in X. h(x,1) = g(x)$
\end{itemize}
We say $f$ and $g$ are \emph{homotopic}, $f \simeq g$, iff there exists a homotopy between them.

Let $[X,Y]$ be the set of all homotopy classes of functions $X \rightarrow Y$.
\end{df}

\begin{prop}
Let $f,f' : X \rightarrow Y$ and $g,g' : Y \rightarrow Z$ be continuous. If $f \simeq f'$ and $g \simeq g'$ then $g \circ f \simeq g' \circ f'$.
\end{prop}

\begin{df}
Let $\mathbf{HTop}$ be the category whose objects are the small topological spaces and whose morphisms are the homotopy classes of continuous functions.

A \emph{homotopy functor} is a functor $\Top \rightarrow \mathcal{C}$ that factors through the canonical functor $\Top \rightarrow \mathbf{HTop}$.
\end{df}

\begin{df}
A functor $F : \mathbf{Top} \rightarrow \mathcal{C}$ is \emph{homotopy invariant} iff, for any topological spaces $X$, $Y$ and continuous functions $f,g : X \rightarrow Y$, if $f \simeq g$ then $Hf = Hg$.
\end{df}

Basepoint-preserving homotopy.

\section{Homotopy Equivalence}

\begin{df}[Homotopy Equivalence]
Let $X$ and $Y$ be topological spaces. A \emph{homotopy equivalence} between $X$ and $Y$, $f : X \simeq Y$, is a continuous function $f : X \rightarrow Y$ such that there exists a continuous function $g : Y \rightarrow X$, the \emph{homotopy inverse} to $f$, such that $g \circ f \simeq \id{X}$ and $f \circ g \simeq \id{Y}$.
\end{df}

\begin{df}[Contractible]
A topological space $X$ is \emph{contractible} iff $X \simeq 1$.
\end{df}

\begin{ex}
$\mathbb{R}^n$ is contractible.
\end{ex}

\begin{ex}
$D^n$ is contractible.
\end{ex}

\begin{df}[Deformation Retract]
Let $X$ be a topological space and $A$ a subspace of $X$. A retraction $\rho : X \rightarrow A$ is a \emph{deformation retraction} iff $i \circ \rho \simeq \id{X}$, where $i$ is the inclusion $A \rightarrowtail X$. We say $A$ is a \emph{deformation retract} of $X$ iff there exists a deformation retraction.
\end{df}

\begin{df}[Strong Deformation Retract]
Let $X$ be a topological space and $A$ a subspace of $X$. A \emph{strong deformation retraction} $\rho : X \rightarrow A$ is a continuous function such that there exists a homotopy $h : X \times [0,1] \rightarrow X$ between $i \circ \rho$ and $\id{X}$ such that, for all $a \in X$ and $t \in [0,1]$, we have $h(a,t) = a$.

We say $A$ is a \emph{strong deformation retract} of $X$ iff a strong deformation retraction exists.
\end{df}

\begin{ex}
$\{0\}$ is a strong deformation retract of $\mathbb{R}^n$ and of $D^n$.
\end{ex}

\begin{ex}
$S^1$ is a strong deformation retract of the torus $S^1 \times D^2$.
\end{ex}

\begin{ex}
$S^{n-1}$ is a strong deformation retract of $D^n - \{0\}$.
\end{ex}

\begin{ex}
For any topological space $X$, the singleton consisting of the vertex is a strong deformation retract of the cone over $X$.
\end{ex}

\chapter{Simplicial Complexes}

\begin{df}[Simplex]
A \emph{$k$-dimensional simplex} or \emph{$k$-simplex} in $\mathbb{R}^n$ is the convex hull $s(x_0, \ldots, x_k)$ of $k+1$ points in general position.
\end{df}

\begin{df}[Face]
A \emph{sub-simplex} or \emph{face} of $s(x_0, \ldots, x_k)$ is the convex hull of a subset of $\{x_0, \ldots, x_k\}$.
\end{df}

\begin{df}[Simplicial Complex]
A \emph{simplicial complex} in $\mathbb{R}^n$ is a set $K$ of simplices such that:
\begin{itemize}
\item for every simplex $s$ in $K$, every face of $s$ is in $K$.
\item The intersection of two simplices $s_1, s_2 \in K$ is either empty or is a face of both $s_1$ and $s_2$.
\item $K$ is locally finite, i.e. every point of $\mathbb{R}^n$ has a neighbourhood that only intersects finitely many elements of $K$.
\end{itemize}

The topological space \emph{underlying} $K$ is $|K| = \bigcup K$ as a subspace of $\mathbb{R}^n$.
\end{df}

\section{Cell Decompositions}

\begin{df}[$n$-cell]
An \emph{$n$-cell} is a topological space homeomorphic to $\mathbb{R}^n$.
\end{df}

\begin{df}[Cell Decomposition]
Let $X$ be a topological space. A \emph{cell decomposition} of $X$ is a partition of $X$ into subspaces that are $n$-cells.
\end{df}

\begin{df}[$n$-skeleton]
Given a cell decomposition of $X$, the \emph{$n$-skeleton} $X^n$ is the union of all the cells of dimension $\leq n$.
\end{df}

\section{CW-complexes}

\begin{df}[CW-Complex]
A \emph{CW-complex} consists of a topological space $X$ and a cell decomposition $\mathcal{E}$ of $X$ such that:
\begin{enumerate}
\item \emph{Characteristic Maps} For every $n$-cell $e \in \mathcal{E}$, there exists a continuous map $\Phi_e : D^n \rightarrow X$ such that $\Phi((D^n)^\circ) = e$, the corestriction $\Phi_e : (D^n)^\circ \approx e$ is a homeomorphism, and $\Phi_e(S^n)$ is the union of all the cells in $\mathcal{E}$ of dimension $< n$.
\item \emph{Closure Finiteness} For all $e \in \mathcal{E}$, we have $\overline{e}$ intersects only finitely many other cells in $\mathcal{E}$.
\item \emph{Weak Topology} Given $A \subseteq X$, we have $A$ is closed iff for all $e \in \mathcal{E}$, $A \cap \overline{e}$ is closed.
\end{enumerate}
\end{df}

\begin{prop}
If a cell decomposition $\mathcal{E}$ satisfies the Characteristic Maps axiom, then for every $n$-cell $e \in \mathcal{E}$ we have $\overline{e} = \Phi_e(D^n)$. Therefore $\overline{e}$ is compact and $\overline{e} - e = \Phi_e(S^{n-1}) \subseteq X^{n-1}$.
\end{prop}

\begin{proof}
\pf
\step{1}{$e \subseteq \Phi_e(D^n) \subseteq \overline{e}$}
\begin{proof}
	\pf
	\begin{align*}
	e & = \Phi_e((D^n)^\circ) \\
	& \subseteq \Phi_e(D^n) \\
	& = \Phi_e(\overline{(D^n)^\circ}) \\
	& \subseteq \overline{\Phi_e((D^n)^\circ)} \\
	& = \overline{e}
	\end{align*}
\end{proof}
\step{2}{$\Phi_e(D^n)$ is compact.}
\begin{proof}
	\pf\ Because $D^n$ is compact.
\end{proof}
\step{3}{$\Phi_e(D^n)$ is closed.}
\step{4}{$\Phi_e(D^n) = \overline{e}$}
\qed
\end{proof}

\chapter{Topological Groups}

\section{Topological Groups}

\begin{df}[Topological Group]
A \emph{topological group} is a group $G$ with a topology such that the function $G^2 \rightarrow G$ that maps $(x,y)$ to $x\inv{y}$ is continuous.
\end{df}

\begin{ex}
$\mathbb{Z}$ is a topological group under addition.
\end{ex}

\begin{proof}
\pf\ The function that sends $(x,y)$ to $x \inv{y}$ is continuous because the topology on $\mathbb{Z}$ is discrete. \qed
\end{proof}

\begin{ex}
$\mathbb{R}$ is a topological group under addition.
\end{ex}

\begin{proof}
\pf\ From Propositions \ref{prop:addition_continuous} and \ref{prop:multiply_continuous}. \qed
\end{proof}

\begin{ex}
$\mathbb{R}_+$ is a topological group under multiplication.
\end{ex}

\begin{proof}
\pf\ From Propositions \ref{prop:multiply_continuous} and \ref{prop:inverse_continuous}. \qed
\end{proof}

\begin{ex}
$S^1$ as a subspace of $\mathbb{C}$ is a topological group under multiplication.
\end{ex}

\begin{proof}
\pf
\step{1}{\pflet{$f : S^1 \rightarrow S^1$ be the function $f(x,y) = x\inv{y}$}}
\step{2}{\pflet{$U$ be an open set in $S^1$} \prove{$\inv{f}(U)$ is open in $(S^1)^2$}}
\step{3}{\pflet{$(x,y) \in \inv{f}(U)$}}
\step{4}{$x\inv{y} \in U$}
\step{5}{\pflet{$x = e^{i\phi}$ and $y = e^{i\psi}$}}
\step{6}{$x\inv{y} = e^{i(\phi - \psi)} \in U$}
\step{7}{\pick\ $\epsilon > 0$ such that, for all $t$, if $|\phi - \psi - t| < \epsilon$ then $e^{it} \in U$}
\step{8}{$(x,y) \in \{e^{it} : |\phi - t| < \epsilon / 2 \} \times \{e^{it} : |\psi - t | < \epsilon / 2 \} \subseteq \inv{f}(U)$}
\qed
\end{proof}

\begin{ex}
$GL(n, \mathbb{R})$ is a topological group considered as a subspace of $\mathbb{R}^{n^2}$.
\end{ex}

\begin{proof}
\pf\ Since the calculations for matrix multiplication and inverse are compositions of continuous functions. \qed
\end{proof}

\begin{ex}
$GL(n,\mathbb{R})$ and $GL(n,\mathbb{C})$ are topological groups.
\end{ex}

\begin{prop}
Let $G$ be a group with a topology. Then $G$ is a topological group if and only if the functions $m : G^2 \rightarrow G$ that sends $(x,y)$ to $xy$ and the function $i : G \rightarrow G$ that sends $x$ to $\inv{x}$ are continuous.
\end{prop}

\begin{proof}
\pf
\step{1}{If $G$ is a topological group then $i$ is continuous.}
\begin{proof}
	\pf\ Since $\inv{x} = e \inv{x}$.
\end{proof}
\step{2}{If $G$ is a topological group then $m$ is continuous.}
\begin{proof}
	\pf\ Since $xy = x\inv{(\inv{y})}$.
\end{proof}
\step{3}{If $m$ and $i$ are continuous then $G$ is a topological group.}
\begin{proof}
	\pf\ Since $x\inv{y} = m(x,i(y))$.
\end{proof}
\qed
\end{proof}

\begin{prop}
Let $G$ be a topological group. Let $\alpha \in G$. The function that maps $x$ to $\alpha x$ is a homeomorphism between $G$ and itself.
\end{prop}

\begin{proof}
\pf
\step{1}{For any $\alpha \in G$, the function that maps $x$ to $\alpha x$ is continuous.}
\begin{proof}
	\pf\ From the definition of topological group.
\end{proof}
\step{2}{For any $\alpha \in G$, the function that maps $x$ to $\alpha x$ is a homeomorphism between $G$ and itself.}
\begin{proof}
	\pf\ Its inverse is the function that maps $x$ to $\inv{\alpha} x$.
\end{proof}
\qed
\end{proof}

\begin{cor}
Every topological group is homogeneous.
\end{cor}

\begin{prop}
Let $G$ be a topological group. Let $\alpha \in G$. The function that maps $x$ to $x \alpha$ is a homeomorphism between $G$ and itself.
\end{prop}

\begin{proof}
\pf\ Similar. \qed
\end{proof}

\subsection{Subgroups}

\begin{prop}
Any subgroup of a topological group is a topological group under the subspace topology.
\end{prop}

\begin{proof}
\pf\ Since the restriction of continuous functions is continuous. \qed
\end{proof}

\begin{prop}
Let $G$ be a topological group and $H$ a subgroup of $G$. Then $\overline{H}$ is a topological group under the subspace topology.
\end{prop}

\begin{proof}
\pf
\step{1}{\pflet{$x,y \in \overline{H}$} \prove{$x\inv{y} \in \overline{H}$}}
\step{2}{\pflet{$U$ be a neighbourhood of $x \inv{y}$.} \prove{$U$ intersects $H$.}}
\step{3}{\pflet{$f : G^2 \rightarrow G$ be the function that maps $(x,y)$ to $x \inv{y}$.}}
\step{4}{$\inv{f}(U)$ is a neighbourhood of $(x,y)$}
\step{5}{\pick\ neighbourhoods $V$ of $x$ and $W$ of $y$ such that $V \times W \subseteq \inv{f}(U)$.}
\step{6}{\pick\ elements $x' \in V \cap H$ and $y' \in W \cap H$}
\step{7}{$x'\inv{y'} \in U \cap H$}
\qed
\end{proof}

\subsection{Left Cosets}

\begin{prop}
Let $G$ be a topological group and $H$ a subgroup of $G$. Give $G/H$ the quotient topology. Let $\alpha \in G$. Define $f_\alpha : G/H \rightarrow G/H$ by
\[ f_\alpha(xH) = \alpha x H \enspace . \]
Then $f_\alpha$ is a homeomorphism.
\end{prop}

\begin{proof}
\pf
\step{1}{For all $\alpha \in G$ we have $f_\alpha$ is well defined.}
\begin{proof}
	\step{a}{\pflet{$x,y \in G$}}
	\step{b}{\assume{$xH = yH$} \prove{$\alpha x H = \alpha y H$}}
	\step{c}{$\inv{x}y \in H$}
	\step{d}{$\inv{x}\inv{\alpha}\alpha y \in H$}
	\step{e}{$\alpha x H = \alpha y H$}
\end{proof}
\step{2}{For all $\alpha \in G$ we have $f_\alpha$ is injective.}
\begin{proof}
	\step{a}{\pflet{$x,y \in G$}}
	\step{b}{\assume{$\alpha x H = \alpha y H$} \prove{$xH = yH$}}
	\step{c}{$\inv{\alpha x} \alpha y \in H$}
	\step{d}{$\inv{x} y \in H$}
	\step{e}{$xH = yH$}
\end{proof}
\step{3}{For all $\alpha \in G$ we have $f_\alpha$ is surjective.}
\begin{proof}
	\pf\ For all $x \in G$ we have $xH = f_\alpha(\inv{\alpha} x H)$.
\end{proof}
\step{4}{For all $\alpha \in G$ we have $f_\alpha$ is continuous.}
\begin{proof}
	\step{b}{\pflet{$V$ be open in $G/H$}}
	\step{c}{$\inv{\pi}(\inv{f_\alpha}(V))$ is open in $G$.}
	\begin{proof}
		\pf\ It is $\inv{g_\alpha}(\inv{\pi}(V))$ where $g_\alpha : V \rightarrow V$ is the homeomorphism $g_\alpha(x) = \alpha x$.
	\end{proof}
	\step{d}{$\inv{f_\alpha}(V)$ is open in $G/H$.}
\end{proof}
\step{5}{For all $\alpha \in G$ we have $\inv{f_\alpha}$ is continuous.}
\begin{proof}
	\pf\ It is $f_{\inv{\alpha}}$.
\end{proof}
\qed
\end{proof}

\begin{cor}
Let $G$ be a topological group and $H$ a subgroup of $G$. Then $G/H$ is a homogeneous space.
\end{cor}

\begin{prop}
Let $G$ be a $T_1$ topological group and $H$ a closed subgroup of $G$. Then $G/H$ is $T_1$.
\end{prop}

\begin{proof}
\pf
\step{1}{\pflet{$x \in G$} \prove{$xH$ is closed.}}
\step{2}{$\inv{\pi}(xH)$ is closed in $G$.}
\begin{proof}
	\pf\ It is $f_x(H)$ and $f_x$ is a homeomorphism.
\end{proof}
\step{3}{$xH$ is closed in $G/H$.}
\qed
\end{proof}

\begin{prop}
\label{prop:pi_open_map}
Let $G$ be a topological group and $H$ a subgroup of $G$. Then the canonical map $\pi : G \twoheadrightarrow G / H$ is an open map.
\end{prop}

\begin{proof}
\pf
\step{1}{\pflet{$U$ be open in $G$.}}
\step{2}{$\forall h \in H. Uh$ is open in $G$.}
\begin{proof}
	\pf\ Since the function that maps $g$ to $gh$ is an automorphism of $G$.
\end{proof}
\step{3}{$UH$ is open in $G$}
\begin{proof}
	\pf\ It is $\bigcup_{h \in H} Uh$.
\end{proof}
\step{4}{$UH = \inv{\pi}(\pi(U))$}
\begin{proof}
	\pf
	\begin{align*}
		\inv{\pi}(\pi(U)) & = \{ x \in G : \exists y \in U. xH = yH \} \\
		& = \{ x \in G : \exists y \in U. \inv{x} y \in H \} \\
		& = \{ x \in G : \exists y \in U. \exists h \in H. \inv{y} x= h \} \\
		& = \{ x \in G : \exists y \in U. \exists h \in H. x = yh \} \\
		& = UH
	\end{align*}
\end{proof}
\step{5}{$\inv{\pi}(\pi(U))$ is open in $G$.}
\step{6}{$\pi(U)$ is open in $G/H$.}
\qed
\end{proof}

\begin{prop}
Let $G$ be a topological group. Let $H$ be a normal subgroup of $G$. Then $G / H$ is a topological group.
\end{prop}

\begin{proof}
\pf
\step{1}{\pflet{$f : G^2 \rightarrow G$ be the map $f(x,y) = x \inv{y}$}}
\step{2}{\pflet{$g : (G/H)^2 \rightarrow G/H$ be the map $g(xH,yH) = x \inv{y} H$}}
\step{3}{$g \circ (\pi \times \pi) = \pi \circ f : G^2 \rightarrow G / H$}
\step{4}{$g \circ (\pi \times \pi)$ is continuous.}
\begin{proof}
	\pf\ Since $\pi$ and $f$ are continuous.
\end{proof}
\step{5}{$\pi$ is an open quotient map.}
\begin{proof}
	\pf\ Proposition \ref{prop:pi_open_map}.
\end{proof}
\step{6}{$\pi \times \pi$ is an open quotient map.}
\begin{proof}
	\pf\ Corollary \ref{cor:product_quotient_maps}.
\end{proof}
\step{7}{$g$ is continuous.}
\begin{proof}
	\pf\ Theorem \ref{thm:quotient_topology_universal}.
\end{proof}
\qed
\end{proof}

\subsection{Homogeneous Spaces}

\begin{df}[Homogeneous Space]
A \emph{homogeneous space} is a topological space of the form $G/H$, where $G$ is a topological group and $H$ is a normal subgroup of $G$, under the quotient topology.
\end{df}

\begin{prop}
Let $G$ be a topological group and $H$ a normal subgroup of $G$. Then $G/H$ is Hausdorff if and only if $H$ is closed.
\end{prop}

\begin{proof}
\pf\ See Bourbaki, N., General Topology. III.12 \qed
\end{proof}

\section{Symmetric Neighbourhoods}

\begin{df}[Symmetric Neighbourhood]
Let $G$ be a topological group. Let $V$ be a neighbourhood of $e$. Then $V$ is \emph{symmetric} iff $V = \inv{V}$.
\end{df}

\begin{prop}
Let $G$ be a topological group. Let $U$ be a neighbourhood of $e$. Then there exists a symmetric neighbourhood $V$ of $e$ such that $VV \subseteq U$.
\end{prop}

\begin{proof}
\pf
\step{1}{\pick\ a neighbourhood $V'$ of $e$ such that $V' V' \subseteq U$.}
\begin{proof}
	\step{a}{\pflet{$m : G^2 \rightarrow G$ be the function $m(x,y) = xy$}}
	\step{b}{$\inv{m}(U)$ is open in $G^2$}
	\step{c}{$(e,e) \in \inv{m}(U)$}
	\step{d}{\pick\ neighbourhoods $V_1$, $V_2$ of $e$ such that $V_1 \times V_2 \subseteq \inv{m}(U)$}
	\step{e}{\pflet{$V' = V_1 \cap V_2$}}
\end{proof}
\step{2}{\pick\ a neighbourhood $W$ of $e$ such that $W \inv{W} \subseteq V'$}
\begin{proof}
	\step{a}{\pflet{$f : G^2 \rightarrow G$ be the function $m(x,y) = x\inv{y}$}}
	\step{b}{$\inv{f}(V')$ is open in $G^2$}
	\step{c}{$(e,e) \in \inv{m}(V')$}
	\step{d}{\pick\ neighbourhoods $W_1$, $W_2$ of $e$ such that $W_1 \times W_2 \subseteq \inv{f}(V')$}
	\step{e}{\pflet{$W = W_1 \cap W_2$}}
\end{proof}
\step{3}{\pflet{$V = W \inv{W}$}}
\step{4}{$V$ is a neighbourhood of $e$.}
\step{5}{$V$ is symmetric.}
\step{6}{$VV \subseteq U$}
\qed
\end{proof}

\begin{prop}
Every $T_1$ topological group is regular.
\end{prop}

\begin{proof}
\pf
\step{1}{\pflet{$G$ be a $T_1$ topological group.}}
\step{2}{\pflet{$A$ be a closed set in $G$ and $x \in G - A$.}}
\step{3}{$G - A \inv{x}$ is a neighbourhood of $e$.}
\step{4}{\pick\ a symmetric neighbourhood $V$ of $e$ such that $VV \subseteq G - A \inv{x}$.}
\step{5}{\pflet{$U = VA$ and $U' = Vx$}}
\step{6}{$U$ and $U'$ are disjoint open sets with $A \subseteq U$ and $x \in U'$.}
\qed
\end{proof}

\begin{prop}
Let $G$ be a $T_1$ topological group. Let $H$ be a closed subgroup of $G$. Then $G/H$ is regular.
\end{prop}

\begin{proof}
\pf
\step{1}{\pflet{$A$ be a closed set in $G/H$ and $xH \in G/H - A$.}}
\step{2}{$G - \inv{\pi}(A)\inv{x}$ is a neighbourhood of $e$.}
\step{3}{\pick\ a symmetric neighbourhood $V$ of $e$ such that $VV \subseteq G - \inv{\pi}(A) \inv{x}$.}
\step{4}{\pflet{$U = \pi(V) A$ and $U' = \pi(V) (xH)$.}}
\step{5}{$U$ and $U'$ are disjoint open sets with $A \subseteq U$ and $xH \in U'$}
\begin{proof}
	\step{a}{\assume{for a contradiction $U \cap U' \neq \emptyset$.}}
	\step{b}{\pick\ $v_1, v_2 \in V$ and $a \in G$ such that $aH \in A$ and $v_1 a H = v_2 x H$.}
	\step{c}{$\inv{a} \inv{v_1} v2 x \in H$}
	\step{d}{$\inv{v_1} v_2 \in \inv{\pi}(A) \inv{x}$}
	\qedstep
	\begin{proof}
		\pf\ This contradicts \stepref{3}.
	\end{proof}
\end{proof}
\qed
\end{proof}

\section{Continuous Actions}

\begin{df}[Continuous Action]
Let $G$ be a topological group and $X$ a topological space. A \emph{continuous action} of $G$ on $X$ is a continuous function $\cdot : G \times X \rightarrow X$ such that:
\begin{itemize}
\item $\forall x \in X. ex = x$
\item $\forall g,h \in G. \forall x \in X. g(hx) = (gh)x$
\end{itemize}

A \emph{$G$-space} consists of a topological space $X$ and a continuous action of $G$ on $X$.
\end{df}

\begin{df}[Orbit]
Let $X$ be a $G$-space and $x \in X$. The \emph{orbit} of $x$ is $\{ gx : g \in G \}$.

The \emph{orbit space} $X / G$ is the set of all orbits under the quotient topology.
\end{df}

\begin{prop}
Define an action of $SO(2)$ on $S^2$ by 
\[ g(x_1, x_2, x_3) = (g(x_1, x_2), x_3) \enspace . \] Then $S^2 / SO(2) \cong [-1,1]$.
\end{prop}

\begin{proof}
\pf
\step{1}{\pflet{$f_3 : S^2 / SO(2) \rightarrow [-1,1]$ be the function induced by $\pi_3 : S^2 \rightarrow [-1,1]$}}
\step{2}{$f_3$ is bijective.}
\step{3}{$S^2 / SO(2)$ is compact.}
\begin{proof}
	\pf\ It is the continuous image of $S^2$ which is compact.
\end{proof}
\step{4}{$[-1,1]$ is Hausdorff.}
\step{5}{$f_3$ is a homeomorphism.}
\qed
\end{proof}

\begin{df}[Stabilizer]
Let $X$ be a $G$-space and $x \in X$. The \emph{stabilizer} of $x$ is $G_x := \{ g \in G : gx = x \}$.
\end{df}

\begin{prop}
The function that maps $gG_x$ to $gx$ is a continuous bijection from $G / G_x$ to $Gx$.
\end{prop}

\begin{proof}
\pf
\step{1}{If $gG_x = hG_x$ then $gx = hx$.}
\begin{proof}
	\step{a}{\assume{$gG_x = hG_x$}}
	\step{b}{$\inv{g}h \in G_x$}
	\step{c}{$\inv{g}h x = x$}
	\step{d}{$gx = hx$}
\end{proof}
\step{2}{If $gx = hx$ then $gG_x = hG_x$.}
\begin{proof}
	\pf\ Similar.
\end{proof}
\step{3}{The function is continuous.}
\begin{proof}
	\pf\ Theorem \ref{thm:quotient_topology_universal}.
\end{proof}
\qed
\end{proof}

\chapter{Topological Vector Spaces}

\begin{df}[Topological Vector Space]
Let $K$ be either $\mathbb{R}$ or $\mathbb{C}$. A \emph{topological vector space} over $K$ consists of a 	vector space $E$ over $K$ and a topology on $E$ such that:
\begin{itemize}
\item Substraction is a continuous function $E^2 \rightarrow E$
\item Multiplication is a continuous function $K \times E \rightarrow E$
\end{itemize}
\end{df}

\begin{prop}
Every topological vector space is a topological group under addition.
\end{prop}

\begin{proof}
\pf\ Immediate from the definition. \qed
\end{proof}

\begin{thm}
The usual topology on a finite dimensional vector space over $K$ is the only one that makes it into a Hausdorff topological vector space.
\end{thm}

\begin{proof}
\pf\ See Bourbaki. Elements de Mathematique, Livre V: Espaces Vectoriels Topologiques, Th. 2, p. 18 \qed
\end{proof}

\begin{prop}
Let $E$ be a topological vector space and $E_0$ a subspace of $E$. Then $\overline{E_0}$ is a subspace of $E$.
\end{prop}

\begin{df}
Let $E$ be a topological vector space. The topological space \emph{associated} with $E$ is $E / \overline{\{0\}}$.
\end{df}

\section{Cauchy Sequences}

\begin{df}[Cauchy Sequence]
Let $E$ be a topological vector space. A sequence $(x_n)$ in $E$ is a \emph{Cauchy sequence} iff, for every neighbourhood $U$ of 0, there exists $n_0$ such that $\forall m,n \geq n_0. x_n - x_m \in U$.
\end{df}

\begin{df}[Complete Topological Vector Space]
A topological vector space is \emph{complete} iff every Cauchy sequence converges.
\end{df}

\section{Seminorms}

\begin{df}[Seminorm]
Let $E$ be a vector space over $K$. A \emph{seminorm} on $E$ is a function $\|\ \| : E \rightarrow \mathbb{R}$ such that:
\begin{enumerate}
\item $\forall x \in E. \| x \| \geq 0$
\item $\forall \alpha \in K. \forall x \in E. \| \alpha x \| = |\alpha| \|x\|$
\item \emph{Triangle Inequality} $\forall x,y : \in E. \| x + y \| \leq \| x \| + \| y \|$
\end{enumerate}
\end{df}

\begin{ex}
The function that maps $(x_1, \ldots, x_n)$ to $|x_i|$ is a seminorm on $\mathbb{R}^n$.
\end{ex}

\begin{df}
Let $E$ be a vector space over $K$.
Let $\Lambda$ be a set of seminorms on $E$. The topology \emph{generated} by $\Lambda$ is the topology generated by the subbasis consisting of all sets of the form $B_\epsilon^\lambda(x) = \{ y \in E : \lambda(y-x) < \epsilon \}$ for $\epsilon > 0$, $\lambda \in \Lambda$ and $x \in E$.
\end{df}

\begin{prop}
$E$ is a topological vector space under this topology. It is Hausdorff iff, for all $x \in E$, if $\forall \lambda \in \Lambda. \lambda(x) = 0$ then $x = 0$.
\end{prop}

\section{Fr\'{e}chet Spaces}

\begin{df}[Pre-Fr\'{e}chet Space]
A \emph{pre-Fr\'{e}chet space} is a Hausdorff topological vector space whose topology is generated by a countable set of seminorms.
\end{df}

\begin{prop}
Let $E$ be a pre-Fr\'{e}chet space whose topology is generated by the family of seminorms $\{ \|\ \|_n : n \in \mathbb{Z}^+ \}$. Then
\[ d(x,y) = \sum_{n=1}^\infty \frac{1}{2^n} \frac{\|x-y\|_n}{1 + \|x-y\|_n} \]
is a metric that induces the same topology. The two definitions of Cauchy sequence agree.
\end{prop}

\begin{df}[Fr\'{e}chet Space]
A \emph{Fr\'{e}chet space} is a complete pre-Fr\'{e}chet space.
\end{df}

\section{Normed Spaces}

\begin{df}[Normed Space]
Let $E$ be a vector space over $K$. A \emph{norm} on $E$ is a function $\|\ \| : E \rightarrow \mathbb{R}$ is a seminorm such that, $\forall x \in E. \| x \| = 0 \Leftrightarrow x = 0$.

A \emph{normed space} consists of a vector space with a norm.
\end{df}

\begin{prop}
If $E$ is a normed space then $d(x,y) = \| x - y \|$ is a metric on $E$ that makes $E$ into a topological vector space. The two definitions of Cauchy sequence agree on $E$.
\end{prop}

\begin{df}[$p$-norm]
For any $p \geq 1$, the \emph{$p$-norm} on $\mathbb{R}^n$ is defined by
\[ \| \vec{x} \|_p := \left( \sum_{i=1}^n |x_i|^p \right)^{\frac{1}{p}} \enspace . \]

We prove this is a norm.
\end{df}

\begin{proof}
\pf
\step{1}{For all $\vec{x} \in \mathbb{R}^n$ we have $\| \vec{x} \|_p \geq 0$}
\begin{proof}
	\pf\ Immediate from definition.
\end{proof}
\step{2}{For all $\alpha \in \mathbb{R}$ and $\vec{x} \in \mathbb{R}^n$ we have $\| \alpha \vec{x} \|_p = |\alpha| \| \vec{x} \|_p$}
\begin{proof}
	\pf
	\begin{align*}
		\| \alpha (x_1, \ldots, x_n) \|
		& = \| (\alpha x_1, \ldots, \alpha x_n) \| \\
		& = \left( \sum_{i=1}^n (\alpha x_i)^p \right)^{\frac{1}{p}} \\
		& = \left( |\alpha|^p \sum_{i=1}^n x_i^p \right)^{\frac{1}{p}} \\
		& = |\alpha| \left( \sum_{i=1}^n x_i^p \right)^{\frac{1}{p}} \\
		& = |\alpha| \|\vec{x}\|_p
	\end{align*}
\end{proof}
\step{3}{The triangle inequality holds.}
\begin{proof}
	\pf
	\begin{align*}
		\| \vec{x} + \vec{y} \|_p^p
		& = \sum_{i=1}^n |x_i + y_i|^p \\
		& = \sum_{i=1}^n |x_i + y_i| |x_i + y_i|^{p-1} \\
		& \leq \sum_{i=1}^n (|x_i| + |y_i|) |x_i + y_i|^{p-1} \\
		& = \sum_{i=1}^n |x_i| |x_i + y_i|^{p-1} + \sum_{i=1}^n |y_i| |x_i + y_i|^{p-1} \\
		& \leq \left( \sum_{i=1}^n |x_i|^p \right)^{\frac{1}{p}} \left( \sum_{i=1}^n |x_i + y_i|^p \right)^{\frac{p-1}{p}} + \left( \sum_{i=1}^n |y_i|^p \right)^{\frac{1}{p}} \left( \sum_{i=1}^n |x_i + y_i|^p \right)^{\frac{p-1}{p}} & (\text{H\"{o}lder's Inequality}) \\
		& = ( \|\vec{x}\|_p + \|\vec{y}\|_p) \| \vec{x} + \vec{y} \|^{p-1}
	\end{align*}
	Assuming w.l.o.g. $\| \vec{x} + \vec{y} \|^{p-1} \neq 0$ (using \stepref{4}) we have $\| \vec{x} + \vec{y} \|_p \leq \| \vec{x} \|_p + \| \vec{y} \|_p$. 	
\end{proof}
\step{4}{For any $\vec{x} \in \mathbb{R}^n$, we have $\| \vec{x} \| = 0$ iff $\vec{x} = \vec{0}$.}
\begin{proof}
	\pf $\sum_{i=1}^n x_i^p = 0$ iff $x_1 = \cdots = x_n  = 0$.
\end{proof}
\qed
\end{proof}

\begin{prop}
The $p$-norm on $\mathbb{R}^n$ induces the product topology.
\end{prop}

\begin{proof}
\pf
\step{0}{\pflet{$d$ be the metric induced by the $p$-norm and $\rho$ the square metric on $\mathbb{R}^n$.}}
\step{1}{The metric topology is finer than the product topology.}
\begin{proof}
	\step{a}{\pflet{$\vec{x} \in \mathbb{R}^n$ and $\epsilon > 0$}} 
	\step{b}{\pflet{$\delta = \epsilon / n^{\frac{1}{p}}$}\prove{$B_\rho(\vec{x}, \delta) \subseteq B_d(\vec{x}, \epsilon)$}}
	\step{c}{\pflet{$\vec{y} \in B_\rho(\vec{x}, \delta)$}}
	\step{d}{$\forall i. |x_i - y_i| < \delta$}
	\step{e}{$d(\vec{x}, \vec{y}) < \epsilon$}
	\begin{proof}
		\pf
		\begin{align*}
			d(\vec{x}, \vec{y}) & = \left( \sum_{i=1}^n |x_i - y_i|^p \right)^{\frac{1}{p}} \\
			& < \left( \sum_{i=1}^n \delta^p \right)^{\frac{1}{p}} & (\text{\stepref{d}}) \\
			& = n^{\frac{1}{p}} \delta \\
			& = \epsilon & (\text{\stepref{b}})
		\end{align*}
	\end{proof}
\end{proof}
\step{2}{The product topology is finer than the metric topology.}
\begin{proof}
	\step{a}{\pflet{$\vec{x} \in \mathbb{R}^n$ and $\epsilon > 0$}}
	\step{c}{\pflet{$\vec{y} \in B_d(\vec{x}, \epsilon)$}}
	\step{d}{$d(\vec{x}, \vec{y}) < \epsilon$}
	\step{e}{$\sum_{i=1}^n |x_i - y_i|^p < \epsilon^p$}
	\step{f}{$\forall i. |x_i - y_i|^p < \epsilon^p$}
	\step{g}{$\forall i. |x_i - y_i| < \epsilon$}
	\step{h}{$\rho(\vec{x},\vec{y}) < \epsilon$}
\end{proof}
\qed
\end{proof}

\begin{df}[Sup-norm]
The \emph{sup-norm} on $\mathbb{R}^n$ is defined by
\[ \| (x_1, \ldots, x_n) \|_\infty := \max(|x_1|, \ldots, |x_n|) \enspace . \]
\end{df}

\begin{prop}
The 2-norm on $\mathbb{R}^n$ induces the standard metric.
\end{prop}

\begin{proof}
\pf\ Immediate from definitions. \qed
\end{proof}

\begin{df}
For $p \geq 1$, the normed space $l_p$ is the set of all sequences $(x_n)$ in $\mathbb{R}$ such that $\sum_{n=1}^\infty x_n^p$ converges, under
\[ \| (x_n) \|_p := \left( \sum_{i=1}^\infty |x_i|^p \right)^{\frac{1}{p}} \enspace . \]
\end{df}

\begin{prop}
The spaces $l_p$ for $p \geq 1$ are all homeomorphic.
\end{prop}

\begin{proof}
\pf\ See Kadets, Mikhail Iosifovich. 1967. Proof of the topological equivalence of all separable
infinite-dimensional banach spaces. Functional Analysis and Its Applications 1 (1): 53–62.
http://dx.doi.org/10.1007/BF01075865.
\end{proof}

\begin{prop}
The metric topology on $l_2$ is strictly finer than the uniform topology.
\end{prop}

\begin{proof}
\pf
\step{o}{\pflet{$d$ be the metric induced by the $l^2$-norm and $\overline{\rho}$ the uniform topology.}}
\step{1}{The metric topology is finer than the uniform topology.}
\begin{proof}
	\step{a}{\pflet{$x \in l_2$}}
	\step{b}{\pflet{$\epsilon > 0$}}
	\step{c}{\pflet{$\delta = \epsilon / 2$}}
	\step{d}{\pflet{$y \in B_d(x, \delta)$}}
	\step{e}{$\sum_{n=0}^\infty (x_n - y_n)^2 < \delta^2$}
	\step{f}{$\forall n. (x_n - y_n)^2 < \delta^2$}
	\step{g}{$\forall n. |x_n - y_n| < \delta$}
	\step{h}{$\forall n. \overline{d}(x_n,y_n) < \delta$}
	\step{i}{$\overline{\rho}(x,y) \leq \delta$}
	\step{j}{$\overline{\rho}(x,y) < \epsilon$}
	\step{k}{$y \in B_{\overline{\rho}}(x, \epsilon)$}
\end{proof}
\step{2}{The metric topology is not the same as the uniform topology.}
\begin{proof}
	\step{a}{\assume{for a contradiction $B_d(0,1)$ is open in the uniform topology.}}
	\step{b}{\pick\ $\epsilon > 0$ such that $B_{\overline{\rho}}(0, \epsilon) \subseteq B_d(0,1)$}
	\step{c}{\pick\ an integer $N$ such that $1/N < \epsilon^2/4$}
	\step{d}{\pflet{$(x_n)$ be the sequence with $x_n = \epsilon / 2$ for $n < N$ and $x_n = 0$ for $n \geq N$}}
	\step{e}{$(x_n) \in l_2$}
	\step{f}{$(x_n) \in B_{\overline{\rho}}(0, \epsilon)$}
	\begin{proof}
		\pf\ Since $\overline{\rho}((x_n),0) = \epsilon / 2$.
	\end{proof}
	\step{g}{$d((x_n),0) > 1$}
	\begin{proof}
		\pf
		\begin{align*}
			d((x_n),0)^2 & = \sum_{n=0}^\infty x_n^2 \\
			& = N \epsilon^2 / 4 \\
			& > 1 
		\end{align*}
	\end{proof}
\end{proof}
\qed
\end{proof}

\begin{prop}
The metric topology on $l_2$ is strictly coarser than the box topology.
\end{prop}

\begin{proof}
\pf
\step{1}{The box topology is finer than the metric topology.}
\begin{proof}
	\step{a}{\pflet{$(x_n) \in l_2$ and $\epsilon > 0$.}}
	\step{b}{\pflet{$(y_n) \in B((x_n), \epsilon)$}}
	\step{c}{\pick\ a sequence of real numbers $(\delta_n)$ such that $\sum_{n=0}^\infty \delta_n^2 < (\epsilon - d((x_n),(y_n)))^2$}
	\step{d}{\pflet{$U = \prod_n (y_n - \delta_n, y_n + \delta_n)$} \prove{$U \subseteq B((x_n),\epsilon)$}}
	\step{e}{\pflet{$(z_n) \in U$}}
	\step{f}{$d((z_n),(y_n)) < \epsilon - d((x_n),(y_n))$}
	\begin{proof}
		\pf
		\begin{align*}
			d((z_n),(y_n))^2 & = \sum_{n=0}^\infty (z_n - y_n)^2 \\
			& < \sum_{n=0}^\infty \delta_n^2 \\
			& < (\epsilon - d((x_n),(y_n)))^2
		\end{align*}
	\end{proof}
	\step{g}{$d((z_n),(x_n)) < \epsilon$}
\end{proof}
\step{2}{The box topology is not equal to the metric topology.}
\begin{proof}
	\step{a}{\pflet{$U = \prod_n (-1/n,1/n)$}}
	\step{b}{\assume{for a contradiction $U$ is open in the metric topology.}}
	\step{c}{\pick\ $\epsilon > 0$ such that $B(0,\epsilon) \subseteq U$}
	\step{d}{\pick\ $N$ such that $1/N < \epsilon/2$.}
	\step{e}{\pflet{$(x_n)$ be the sequence with $x_N = \epsilon/2$ and $x_n = 0$ for all other $n$.}}
	\step{f}{$d((x_n),0) = \epsilon / 2$}
	\step{g}{$(x_n) \notin U$}
\end{proof}
\qed
\end{proof}

\begin{prop}
The $l^2$-topology on $\mathbb{R}^\infty$ is strictly finer than the uniform topology.
\end{prop}

\begin{proof}
\pf
	\step{a}{\assume{for a contradiction $B_d(0,1) \cap \mathbb{R}^\infty$ is open in the uniform topology.}}
	\step{b}{\pick\ $\epsilon > 0$ such that $B_{\overline{\rho}}(0, \epsilon) \cap \mathbb{R}^\infty \subseteq B_d(0,1) \cap \mathbb{R}^\infty$}
	\step{c}{\pick\ an integer $N$ such that $1/N < \epsilon^2/4$}
	\step{d}{\pflet{$(x_n)$ be the sequence with $x_n = \epsilon / 2$ for $n < N$ and $x_n = 0$ for $n \geq N$}}
	\step{e}{$(x_n) \in \mathbb{R}^\infty$}
	\step{f}{$(x_n) \in B_{\overline{\rho}}(0, \epsilon)$}
	\begin{proof}
		\pf\ Since $\overline{\rho}((x_n),0) = \epsilon / 2$.
	\end{proof}
	\step{g}{$d((x_n),0) > 1$}
	\begin{proof}
		\pf
		\begin{align*}
			d((x_n),0)^2 & = \sum_{n=0}^\infty x_n^2 \\
			& = N \epsilon^2 / 4 \\
			& > 1 
		\end{align*}
	\end{proof}
\qed
\end{proof}

\begin{prop}
The $l^2$-topology on $\mathbb{R}^\infty$ is strictly coarser than the box topology.
\end{prop}

\begin{proof}
	\step{a}{\pflet{$U = \prod_n (-1/n,1/n) \cap \mathbb{R}^\infty$}}
	\step{b}{\assume{for a contradiction $U$ is open in the metric topology.}}
	\step{c}{\pick\ $\epsilon > 0$ such that $B(0,\epsilon) \cap \mathbb{R}^\infty \subseteq U \cap \mathbb{R}^\infty$}
	\step{d}{\pick\ $N$ such that $1/N < \epsilon/2$.}
	\step{e}{\pflet{$(x_n)$ be the sequence with $x_N = \epsilon/2$ and $x_n = 0$ for all other $n$.}}
	\step{f}{$d((x_n),0) = \epsilon / 2$}
	\step{g}{$(x_n) \notin U$}
	\qed
\end{proof}

\begin{prop}
The $l^2$-topology on the Hilbert cube the same as the product topology.
\end{prop}

\begin{proof}
\pf
\step{1}{For every $(x_n) \in H$ and $\epsilon > 0$, there exists a neighbourhood $U$ of $(x_n)$ in the product topology such that $U \subseteq B((x_n),\epsilon)$.}
\begin{proof}
	\step{a}{\pflet{$(x_n) \in H$}}
	\step{b}{\pflet{$\epsilon > 0$}}
	\step{c}{\pick\ $N$ such that $\sum_{i=N+1}^\infty 1/i^2 < \epsilon^2 / 2$}
	\step{d}{\pflet{$B' = (\prod_{i=0}^N (x_i - \epsilon / \sqrt{2N}, x_i + \epsilon / \sqrt{2N}) \times \prod_{i=N+1}^\infty [0, 1/(i+1)]) \cap H$} \prove{$B' \subseteq B((x_n), \epsilon)$}}
	\step{e}{\pflet{$(y_n) \in B'$}}
	\step{f}{$d((x_n),(y_n)) < \epsilon$}
	\begin{proof}
		\pf
		\begin{align*}
			d((x_n),(y_n))^2 & = \sum_{i=0}^\infty |x_n - y_n|^2 \\
			& < \sum_{i=0}^N \epsilon^2 / 2N + \sum_{i=N+1}^\infty 1/(i+1) 1/(i+1)^2 \\
			& < \epsilon^2 / 2 + \epsilon^2 / 2 \\
			& = \epsilon^2
		\end{align*}
	\end{proof}
\end{proof}
\step{2}{The product topology is finer than the $l^2$-topology.}
\begin{proof}
	\step{a}{\pflet{$(x_n) \in H$ and $\epsilon > 0$} \prove{$B((x_n), \epsilon) \cap H$ is open in the product topology.}}
	\step{b}{\pflet{$(y_n) \in B((x_n), \epsilon)$}}
	\step{c}{\pick\ a neighbourhood $U$ of $(y_n)$ in the product topology such that $U \subseteq B((y_n), \epsilon - d((x_n),(y_n)))$}
	\step{d}{$U \subseteq B((x_n), \epsilon)$}
\end{proof}
\qed
\end{proof}

\begin{df}
Let $l_\infty$ be the set of all bounded sequences in $\mathbb{R}$ under
\[ \| (x_n) \| := \sup_n |x_n| \]
\end{df}

\begin{prop}
For all $p \geq 1$ we have $l_p$ is not homeomorphic to $l_\infty$.
\end{prop}

%TODO

\begin{prop}
Let $\|\ \|$ be a seminorm on the vector space $E$. Then $\|\ \|$ defines a norm on $E / \overline{\{0\}}$.
\end{prop}

\begin{prop}
Let $E$ and $F$ be normed spaces. Any continuous linear map $E \rightarrow F$ is uniformly continuous.
\end{prop}

\begin{df}
For $p \geq 1$. let $\mathcal{L}^p(\mathbb{R}^n)$ be the vector space of all Lebesgue-measurable functions $f : \mathbb{R}^n \rightarrow \mathbb{R}$ such that $|f|^p$ is Lebesgue-integrable. Then
\[ \| f \|_p := \sqrt{p}{\int_{\mathbb{R}^n} |f(x)|^p dx} \]
defines a seminorm on $\mathcal{L}^p(\mathbb{R}^n)$. Let
\[ L^p(\mathbb{R}^n) := \mathcal{L}^p(\mathbb{R}^n) / \overline{\{0\}} \enspace . \]
\end{df}

\section{Unit Ball}

\begin{prop}
Let $n$ be a positive integer.
Every open ball $B(\vec{x}, \epsilon)$ in $\mathbb{R}^n$ is path connected.
\end{prop}

\begin{proof}
\pf
\step{1}{\pflet{$\vec{y}, \vec{z} \in B(\vec{x}, \epsilon)$}}
\step{2}{\pflet{$\vec{p} : [0,1] \rightarrow B(\vec{x}, \epsilon)$ be the path $\vec{p}(t) = (1-t) \vec{y} + t \vec{z}$.}}
\begin{proof}
	\step{a}{\pflet{$t \in [0,1]$} \prove{$\vec{p}(t) \in B(\vec{x}, \epsilon)$}}
	\step{b}{$d(\vec{p}(t), \vec{x}) < \epsilon$}
	\begin{proof}
		\pf
		\begin{align*}
			d(\vec{p}(t),\vec{x})
			& = \| (1-t) \vec{y} + t \vec{z} - \vec{x} \| \\
			& = \| (1-t) (\vec{y} - \vec{x}) + t (\vec{z} - \vec{x}) \| \\
			& \leq (1-t) \| \vec{y} - \vec{x} \| + t \| \vec{z} - \vec{x} \| \\
			& < (1 - t) \epsilon + t \epsilon \\
			& = \epsilon
		\end{align*}
	\end{proof}
\end{proof}
\step{3}{$\vec{p}$ is a path from $\vec{x}$ to $\vec{y}$.}
\qed
\end{proof}

\begin{prop}
Let $n$ be a positive integer.
Every closed ball $B(\vec{x}, \epsilon)$ in $\mathbb{R}^n$ is path connected.
\end{prop}

\begin{proof}
\pf
\step{1}{\pflet{$\vec{y}, \vec{z} \in \overline{B(\vec{x}, \epsilon)}$}}
\step{2}{\pflet{$\vec{p} : [0,1] \rightarrow \overline{B(\vec{x}, \epsilon)}$ be the path $\vec{p}(t) = (1-t) \vec{y} + t \vec{z}$.}}
\begin{proof}
	\step{a}{\pflet{$t \in [0,1]$} \prove{$\vec{p}(t) \in \overline{B(\vec{x}, \epsilon)}$}}
	\step{b}{$d(\vec{p}(t), \vec{x}) \leq \epsilon$}
	\begin{proof}
		\pf
		\begin{align*}
			d(\vec{p}(t),\vec{x})
			& = \| (1-t) \vec{y} + t \vec{z} - \vec{x} \| \\
			& = \| (1-t) (\vec{y} - \vec{x}) + t (\vec{z} - \vec{x}) \| \\
			& \leq (1-t) \| \vec{y} - \vec{x} \| + t \| \vec{z} - \vec{x} \| \\
			& \leq (1 - t) \epsilon + t \epsilon \\
			& = \epsilon
		\end{align*}
	\end{proof}
\end{proof}
\step{3}{$\vec{p}$ is a path from $\vec{x}$ to $\vec{y}$.}
\qed
\end{proof}

\section{Unit Sphere}

\begin{df}[Unit Sphere]
Let $n$ be a positive integer. The \emph{unit sphere} $S^{n-1}$ is
\[ S^{n-1} := \{ \vec{x} \in \mathbb{R}^n : \| \vec{x} \| = 1 \} \enspace . \]
\end{df}

\begin{prop}
For $n > 1$. the unit sphere $S^{n-1}$ is path connected.
\end{prop}

\begin{proof}
\pf\ The map $g : \mathbb{R}^n - \{\vec{0}\} \twoheadrightarrow S^{n-1}$ defined by $g(\vec{x}) = \vec{x} / \| \vec{x}\|$ is continuous and surjective. Hence $S^{n-1}$ is the continuous image of a path connected space. \qed
\end{proof}

\section{Inner Product Spaces}

\begin{df}[Inner Product]
Given $\vec{x}, \vec{y} \in \mathbb{R}^n$, define
\[ \vec{x} \cdot \vec{y} = x_1 y_1 + \cdots + x_n y_n \enspace . \]
\end{df}

\begin{prop}
\[ \vec{x} \cdot (\vec{y} + \vec{z}) = \vec{x} \cdot \vec{y} + \vec{x} \cdot \vec{z} \]
\end{prop}

\begin{proof}
\pf
\begin{align*}
	\vec{x} \cdot (\vec{y} + \vec{z}) & = x_1 (y_1 + z_1) + \cdots + x_n (y_n + z_n) \\
	& = x_1 y_1 + x_1 z_1 + \cdots + x_n y_n + x_n z_n \\
	& = \vec{x} \cdot \vec{y} + \vec{x} \cdot \vec{z} & \qed
\end{align*}
\end{proof}

\begin{prop}
\label{prop:dot_leq_norms}
For all $\vec{x}, \vec{y} \in \mathbb{R}^n$ we have
\[ | \vec{x} \cdot \vec{y} | \leq \| \vec{x} \| \| \vec{y} \| \enspace . \]
\end{prop}

\begin{proof}
\pf
\step{1}{\assume{w.l.o.g. $\vec{x} \neq \vec{0} \neq \vec{y}$}}
\step{2}{\pflet{$a = 1 / \|x\|$}}
\step{3}{\pflet{$b = 1 / \|y\|$}}
\step{4}{$\| a\vec{x} + b\vec{y} \| \geq 0$}
\step{5}{$a^2 \| \vec{x}\|^2 + 2 ab \vec{x} \cdot \vec{y} + b^2 \| \vec{y}\|^2 \geq 0$}
\step{6}{$ab \vec{x} \cdot \vec{y} \geq -1$}
\step{7}{$\| a \vec{x} - b \vec{y} \| \geq 0$}
\step{8}{$ab \vec{x} \cdot \vec{y} \leq 1$}
\step{9}{$|\vec{x} \cdot \vec{y}| \leq 1/ab$}
\qed
\end{proof}

\begin{prop}
Let $(x_n)$, $(y_n)$ be sequences of real numbers. If $\sum_{n=0}^\infty x_n^2$ and $\sum_{n=0}^\infty y_n^2$ converge then $\sum_{n=0}^\infty |x_n y_n|$ converges.
\end{prop}

\begin{proof}
\pf
\begin{align*}
\sum_{n=0}^N |x_n y_n| & \leq \sqrt{\sum_{n=0}^N x_n^2 \sum_{n=0}^N y_n^2} & (\text{Proposition \ref{prop:dot_leq_norms}}) \\
& \leq \sqrt{\sum_{n=0}^\infty x_n^2 \sum_{n=0}^\infty y_n^2} & \qed
\end{align*}
\end{proof}


\begin{prop}
If $E$ is an inner product space then $\| x \| = \sqrt{\langle x,x \rangle}$ is a norm on $E$.
\end{prop}

\section{Banach Spaces}

\begin{df}[Banach Space]
A \emph{Banach space} is a complete normed space.
\end{df}

\begin{ex}
For any topological space $X$, the set $C(X)$ of bounded continuous functions $X \rightarrow \mathbb{R}$ is a Banach space under $\| f \| = \sup_{x \in X} |f(x)|$.
\end{ex}

\begin{prop}
The completion of a normed space is a Banach space.
\end{prop}

\begin{prop}
Let $E$ and $F$ be normed spaces. Let $f : E \rightarrow F$ be a continuous linear map. Then the extension to the completions $\hat{E} \rightarrow \hat{F}$ is linear.
\end{prop}

\begin{prop}
$L^p(\mathbb{R}^n)$ is a Banach space.
\end{prop}

\begin{prop}
$C(\mathbb{R})$ is first countable but not second countable.
\end{prop}

\begin{proof}
\pf\ For every sequence of 0s and 1s $s = (s_n)$, let $f_s$ be a continuous bounded function whose value at $n$ is $s_n$. Then the set of all $f_s$ is an uncountable discrete set in $C(\mathbb{R})$. Hence $C(\mathbb{R})$ is not second countable.

It is first countable because it is metrizable. \qed
\end{proof}

\section{Hilbert Spaces}

\begin{df}[Hilbert Space]
A \emph{Hilbert space} is a complete inner product space.
\end{df}

\begin{ex}
The set of \emph{square-integrable functions} is the set of Lebesgue integrable functions $[-\pi,\pi] \rightarrow \mathbb{R}$ quotiented by: $f \sim g$ iff $\{ x \in [-\pi,\pi] : f(x) \neq g(x) \}$ has measure 0. This is a Hilbert space under
\[ \langle f,g \rangle = \frac{1}{\pi} \int_{- \pi}{\pi} f(x) g(x) dx \enspace . \]
\end{ex}

\begin{prop}
The completion of an inner product space is a Hilbert space.
\end{prop}

An infinite dimensional Hilbert space with the weak topology is not first countable.

\section{Locally Convex Spaces}

\begin{df}[Locally Convex Space]
A topological vector space is \emph{locally convex} iff every neighbourhood of 0 includes a convex neighbourhood of 0.
\end{df}

\begin{prop}
A topological vector space is locally convex if and only if its topology is generated by a set of seminorms.
\end{prop}

\begin{proof}
\pf\ See K\"{o}the, G. Topological Vector Spaces 1. Section 18. \qed
\end{proof}

\begin{prop}
A locally convex topological vector space is a pre-Fr\'{e}chet space if and only if it is metrizable.
\end{prop}

\begin{proof}
\pf\ See K\"{o}the, G. Topological Vector Spaces 1. Section 18. \qed
\end{proof}

\begin{ex}
Let $E$ be an infinite dimensional Hilbert space. Let $E'$ be the same vector space under the \emph{weak topology}, the coarsest topology such that every continuous linear map $E \rightarrow \mathbb{R}$ is continuous as a map $E' \rightarrow \mathbb{R}$. Then $E$ is locally convex Hausdorff but not metrizable.

Proof: See Dieudonne, J. A., Treatise on Analysis, Vol. II, New York and London: Academic Press, 1970, p. 76.
\end{ex}

\begin{df}[Thom Space]
Let $E$ be a vector bundle with a Riemannian metric, $DE = \{ x \in E : \| x \| \leq 1 \}$ its disc bundle and $SE := \{ v \in E : \| v \| = 1 \}$ its sphere bundle. The \emph{Thom space} of $E$ is the quotient space $DE / SE$.
\end{df}
