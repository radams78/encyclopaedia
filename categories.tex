\newcommand{\dom}{\ensuremath{\operatorname{dom}}}
\newcommand{\cod}{\ensuremath{\operatorname{cod}}}
\newcommand{\Cat}{\ensuremath{\mathbf{Cat}}}

\chapter{Foundations}

This is a placeholder --- I am not sure what foundation I want to use for this
project yet. I will try to work in a way which is foundation-independent. What
I do could be formalized in ZFC, ETCS, or some other system. I will assume the
usual set theoretic constructions as needed. 

\section{Relations}

\begin{df}[Reflexive]
A relation $R$ on a class $A$ is \emph{reflexive} iff, for all $x \in A$, we have $xRx$.
\end{df}

\begin{df}[Transitive]
A relation $R$ on a class $A$ is \emph{transitive} iff, whenever $xRy$ and $yRz$, then $xRz$.
\end{df}

\chapter{Categories}

\section{Definition}

\begin{df}[Category]
    A \emph{category} $\mathcal{C}$ consists of:
    \begin{itemize}
        \item A class $|\mathcal{C}|$ of \emph{objects}. We write $A \in \mathcal{C}$ for $A
                  \in |\mathcal{C}|$.
        \item For any objects $A$, $B$, a set $\mathcal{C}[A,B]$ of \emph{morphisms} from $A$
              to $B$. We write $f : A \rightarrow B$ for $f \in \mathcal{C}[A,B]$.
        \item For any object $A$, a morphism $\id{A} : A \rightarrow A$, the \emph{identity}
              morphism on $A$.
        \item For any morphisms $f : A \rightarrow B$ and $g : B \rightarrow C$, a morphism
              $g \circ f : A \rightarrow C$, the \emph{composite} of $f$ and $g$.
    \end{itemize}
    such that:
    \begin{description}
        \item[Associativity] Given $f : A \rightarrow B$, $g : B \rightarrow C$ and $h : C
                \rightarrow D$, we have $h \circ (g \circ f) = (h \circ g) \circ f$
        \item[Left Unit Law] For any morphism $f : A \rightarrow B$, we have $\id{B} \circ f
                = f$.
        \item[Right Unit Law] For any morphism $f : A \rightarrow B$, we have $f \circ \id{A}
                = f$.
    \end{description}
\end{df}

\section{Examples}

\begin{ex}[Category of Sets]
    The \emph{category of sets} $\Set$ has objects all sets and morphisms all functions.
\end{ex}

\begin{ex}[Category of Finite Sets]
The \emph{category of finite sets} $\Set_{\mathbf{fin}}$ has objects all finite sets and morphisms all functions.
\end{ex}

\begin{ex}[Category of Sets and Relations]
The \emph{category of sets and relations} $\mathbf{Rel}$ has:
\begin{itemize}
\item objects all sets
\item morphism $A \rightarrow B$ all relations between $A$ and $B$
\item the identity on $A$ is $\{ (a,a) : a \in A \}$
\item given $R \subseteq A \times B$ and $S \subseteq B \times C$, we define
\[ S \circ R = \{ (a,c) \in A \times C : \exists b \in B. aRb \wedge bSc \} \enspace . \]
\end{itemize}
\end{ex}

%Basic Properties

%Ways of constructing categories

%Special kinds of categories

\section{Subcategories}

\begin{df}[Subcategory]
A category $\mathcal{C}$ is a \emph{subcategory} of a category $\mathcal{D}$ iff:
\begin{itemize}
\item $|\mathcal{C}| \subseteq |\mathcal{D}|$
\item for all $A,B \in \mathcal{C}$, we have $\mathcal{C}[A,B] \subseteq \mathcal{D}[A,B]$
\item for all $A \in \mathcal{C}$, the identity on $A$ is the same in $\mathcal{C}$ and $\mathcal{D}$
\item composition in $\mathcal{C}$ and composition in $\mathcal{D}$ agree on composable pairs of morphisms from $\mathcal{C}$.
\end{itemize}
It is a \emph{full} subcategory iff, for all $A, B \in \mathcal{C}$, we have $\mathcal{C}[A,B] = \mathcal{D}[A,B]$.
\end{df}

\chapter{Morphisms}

\begin{df}[Endomorphism]
    In a category $\mathcal{C}$, an \emph{endomorphism} on an object $A$ is a morphism $A \rightarrow A$. We write $\mathrm{End}_\mathcal{C}(A)$ for $\mathcal{C}[A,A]$.
\end{df}

\section{Monomorphisms and Epimorphisms}

\begin{df}[Monomorphism]
    In a category, let $f : A \rightarrow B$. Then $f$ is a \emph{monomorphism} or \emph{monic} iff, for every object $X$ and morphism $x,y : X \rightarrow A$, if $fx = fy$ then $x=y$.
\end{df}

\begin{df}[Epimorphism]
    In a category, let $f : A \rightarrow B$. Then $f$ is a \emph{epimorphism} or \emph{epi} iff, for every object $X$ and morphism $x,y : B \rightarrow X$, if $xf = yf$ then $x=y$.
\end{df}

\begin{prop}
    The composite of two monomorphism is monic.
\end{prop}

\begin{proof}
    \pf
    \step{1}{\pflet{$f : A \rightarrowtail B$ and $g : B \rightarrowtail C$ be monic.}}
    \step{2}{\pflet{$x,y : X \rightarrow A$}}
    \step{3}{\assume{$g \circ f \circ x = g \circ f \circ y$}}
    \step{4}{$f \circ x = f \circ y$}
    \step{5}{$x = y$}
    \qed
\end{proof}

\begin{prop}
    The composite of two epimorphisms is epi.
\end{prop}

\begin{proof}
    \pf\ Dual. \qed
\end{proof}

\begin{prop}
    Let $f : A \rightarrow B$ and $g : B \rightarrow C$. If $g \circ f$ is monic then $f$ is monic.
\end{prop}

\begin{proof}
    \pf\ If $f \circ x = f \circ y$ then $gfx = gfy$ and so $x = y$. \qed
\end{proof}

\begin{prop}
    Let $f : A \rightarrow B$ and $g : B \rightarrow C$. If $g \circ f$ is epi then $g$ is epi.
\end{prop}

\begin{proof}
    \pf\ Dual. \qed
\end{proof}

\begin{prop}
    A function is a monomorphism in $\Set$ iff it is injective.
\end{prop}

\begin{proof}
    \pf
    \step{1}{\pflet{$f : A \rightarrow B$}}
    \step{2}{If $f$ is monic then $f$ is injective.}
    \begin{proof}
        \step{a}{\assume{$f$ is monic.}}
        \step{b}{\pflet{$x,y \in A$}}
        \step{c}{\assume{$f(x) = f(y)$}}
        \step{d}{\pflet{$\overline{x}, \overline{y} : 1 \rightarrow A$ be the functions such that $\overline{x}(*) = x$ and $\overline{y}(*) = y$}}
        \step{e}{$f \circ \overline{x} = f \circ \overline{y}$}
        \step{f}{$\overline{x} = \overline{y}$}
        \begin{proof}
            \pf\ By \stepref{a}.
        \end{proof}
        \step{g}{$x = y$}
    \end{proof}
    \step{3}{If $f$ is injective then $f$ is monic.}
    \begin{proof}
        \step{a}{\assume{$f$ is injective.}}
        \step{b}{\pflet{$X$ be a set and $x,y : X \rightarrow A$.}}
        \step{c}{\assume{$f \circ x = f \circ y$} \prove{$x = y$}}
        \step{d}{\pflet{$t \in X$} \prove{$x(t) = y(t)$}}
        \step{e}{$f(x(t)) = f(y(t))$}
        \step{f}{$x(t) = y(t)$}
        \begin{proof}
            \pf\ By \stepref{a}.
        \end{proof}
    \end{proof}
    \qed
\end{proof}

\begin{prop}
    A function is an epimorphism in $\Set$ iff it is surjective.
\end{prop}

\begin{proof}
    \pf
    \step{1}{\pflet{$f : A \rightarrow B$}}
    \step{2}{If $f$ is an epimorphism then $f$ is surjective.}
    \begin{proof}
        \step{a}{\assume{$f$ is an epimorphism.}}
        \step{b}{\pflet{$b \in B$}}
        \step{d}{\pflet{$x,y : B \rightarrow 2$ be defined by $x(b) = 1$ and $x(t) = 0$ for all other $t \in B$, $y(t) = 0$ for all $t \in B$.}}
        \step{f}{$x \neq y$}
        \step{e}{$x \circ f \neq y \circ f$}
        \step{g}{There exists $a \in A$ such that $f(a) = b$.}
    \end{proof}
    \step{3}{If $f$ is surjective then $f$ is an epimorphism.}
    \begin{proof}
        \step{a}{\assume{$f$ is surjective.}}
        \step{b}{\pflet{$x,y : B \rightarrow X$}}
        \step{c}{\assume{$x \circ f = y \circ f$} \prove{$x = y$}}
        \step{d}{\pflet{$b \in B$} \prove{$x(b) = y(b)$}}
        \step{e}{\pick\ $a \in A$ such that $f(a) = b$}
        \step{f}{$x(f(a)) = y(f(a))$}
        \step{g}{$x(b) = y(b)$}
    \end{proof}
    \qed
\end{proof}

\begin{prop}
    In a preorder, every morphism is monic and epi.
\end{prop}

\begin{proof}
    \pf\ Immediate from definitions. \qed
\end{proof}

\section{Sections and Retractions}

\begin{df}[Section, Retraction]
    In a category, let $r : A \rightarrow B$ and $s : B \rightarrow A$. Then $r$ is a \emph{retraction} of $s$, and $s$ is a \emph{section} of $r$, iff $r \circ s = \id{B}$.
\end{df}

\begin{prop}
    Every identity morphism is a section and retraction of itself.
\end{prop}

\begin{proof}
    \pf\ Immediate from definitions. \qed
\end{proof}

\begin{prop}
    \label{prop:retraction-is-section}
    Let $r,r' : A \rightarrow B$ and $s : B \rightarrow A$.
    If $r$ is a retraction of $s$ and $r'$ is a section of $s$ then $r = r'$.
\end{prop}

\begin{proof}
    \pf
    \begin{align*}
        r & = r \circ \id{A}            \\
          & = r \circ s \circ r'        \\
          & = \id{B} \circ r'           \\
          & = r'                 & \qed
    \end{align*}
\end{proof}

\begin{prop}
    \label{prop:retraction-comp}
    Let $r_1 : A \rightarrow B$, $r_2 : B \rightarrow C$, $s_1 : B \rightarrow A$ and $s_2 : C \rightarrow B$. If $r_1$ is a retraction of $s_1$ and $r_2$ is a retraction of $s_2$ then $r_2 \circ r_1$ is a retraction of $s_1 \circ s_2$.
\end{prop}

\begin{proof}
    \pf
    \begin{align*}
        r_2 \circ r_1 \circ s_1 \circ s_2 & = r_2 \circ \id{B} \circ s_2        \\
                                          & = r_2 \circ s_2                     \\
                                          & = \id{C}                     & \qed
    \end{align*}
\end{proof}

\begin{prop}
    Every section is monic.
\end{prop}

\begin{proof}
    \pf
    \step{1}{\pflet{$s : A \rightarrow B$ be a section of $r : B \rightarrow A$.}}
    \step{2}{\pflet{$x,y : X \rightarrow A$ satisfy $sx = sy$.}}
    \step{3}{$rsx = rsy$}
    \step{4}{$x = y$}
    \qed
\end{proof}

\begin{prop}
    Every retraction is epi.
\end{prop}

\begin{proof}
    \pf\ Dual. \qed
\end{proof}

\begin{prop}
    In $\Set$, every epimorphism has a retraction.
\end{prop}

\begin{proof}
    \pf\ By the Axiom of Choice. \qed
\end{proof}

\begin{ex}
    It is not true in general that every monomorphism in any category has a section. nor that every epimorphism in any category has a retraction.

    In the category $\mathbf{2}$, the morphism $0 \leq 1$ is monic and epi but has
    no retraction or section.
\end{ex}

%TODO This is not true in Grp

\section{Isomorphisms}

\begin{df}[Isomorphism]
    In a category $\mathcal{C}$, a morphism $f : A \rightarrow B$ is an \emph{isomorphism}, denoted $f : A \cong B$, iff there exists a morphism $f^{-1} : B \rightarrow A$, the \emph{inverse} of $f$, such that $f^{-1} \circ f = \id{A}$ and $f \circ f^{-1} = \id{B}$.

    An \emph{automorphism} on an object $A$ is an isomorphism between $A$ and
    itself. We write $\mathrm{Aut}_\mathcal{C}(A)$ for the set of all automorphisms
    on $A$.

    Objects $A$ and $B$ are \emph{isomorphic}, $A \cong B$, iff there exists an
    isomorphism between them.
\end{df}

\begin{prop}
    \label{prop:inv-unique}
    The inverse of an isomorphism is unique.
\end{prop}

\begin{proof}
    \pf\ Proposition \ref{prop:retraction-is-section}. \qed
\end{proof}

\begin{prop}
    For any object $A$ we have $\id{A} : A \cong A$ and $\id{A}^{-1} = \id{A}$.
\end{prop}

\begin{proof}
    \pf\ Since $\id{A} \circ \id{A} = \id{A}$ by the Unit Laws. \qed
\end{proof}

\begin{prop}
    If $f : A \cong B$ then $f^{-1} : B \cong A$ and $(f^{-1})^{-1} = f$.
\end{prop}

\begin{proof}
    \pf\ Immediate from definitions. \qed
\end{proof}

\begin{prop}
    If $f : A \cong B$ and $g : B \cong C$ then $g \circ f : A \cong C$ and $\inv{(g \circ f)} = \inv{f} \circ \inv{g}$.
\end{prop}

\begin{proof}
    \pf\ From Proposition \ref{prop:retraction-comp}. \qed
\end{proof}

\begin{df}[Groupoid]
    A \emph{groupoid} is a category in which every morphism is an isomorphism.
\end{df}

\section{Initial and Terminal Objects}

\begin{df}[Initial Object]
    An object $I$ in a category is \emph{initial} iff, for any object $X$, there is exactly one morphism $I \rightarrow X$.
\end{df}

\begin{ex}
    The empty set is the initial object in $\Set$.
\end{ex}

\begin{df}[Terminal Object]
    An object $T$ in a category is \emph{terminal} iff, for any object $X$, there is exactly one morphism $X \rightarrow T$.
\end{df}

\begin{ex}
    Every singleton is terminal in $\Set$.
\end{ex}

\begin{prop}
    If $I$ and $J$ are initial in a category, then there exists a unique isomorphism $I \cong J$.
\end{prop}

\begin{proof}
    \pf
    \step{1}{\pflet{$i$ be the unique morphism $I \rightarrow J$.}}
    \step{2}{\pflet{$\inv{i}$ be the unique morphism $J \rightarrow I$.}}
    \step{3}{$i \circ \inv{i} = \id{J}$}
    \begin{proof}
        \pf\ Since there is only one morphism $J \rightarrow J$.
    \end{proof}
    \step{4}{$\inv{i} \circ i = \id{I}$}
    \begin{proof}
        \pf\ Since there is only one morphism $I \rightarrow I$.
    \end{proof}
    \qed
\end{proof}

\begin{prop}
    If $S$ and $T$ are terminal in a category, then there exists a unique isomorphism $S \cong T$.
\end{prop}

\begin{proof}
    \pf\ Dual. \qed
\end{proof}

\section{Comma Categories}

\begin{df}[Comma Category]
    Let $F : \mathcal{C} \rightarrow \mathcal{E}$ and $G : \mathcal{D} \rightarrow \mathcal{E}$ be functors. The \emph{comma category} $F \downarrow G$ is the category with:
    \begin{itemize}
        \item objects all pairs $(C,D,f)$ where $C \in \mathcal{C}$, $D \in \mathcal{D}$ and
              $f : FC \rightarrow GD : \mathcal{E}$
        \item morphisms $(u,v) : (C,D,f) \rightarrow (C',D',g)$ all pairs $u : C \rightarrow
                  C' : \mathcal{C}$ and $v : D \rightarrow D' : \mathcal{D}$ such that the
              following diagram commutes:

              \begin{tikzcd}
                  FC \arrow[r,"f"] \arrow[d,"Fu"] & GD \arrow[d,"Gv"] \\
                  FC' \arrow[r,"g"] & GD'
              \end{tikzcd}
    \end{itemize}
\end{df}

\begin{df}[Slice Category]
    Let $\mathcal{C}$ be a category and $A \in \mathcal{C}$. The \emph{slice category} over $A$, denoted $\mathcal{C} / A$, is the comma category $1_\mathcal{C} \downarrow K^{\mathbf{1}} A$.
\end{df}

\begin{df}[Coslice Category]
    Let $\mathcal{C}$ be a category and $A \in \mathcal{C}$. The \emph{coslice category} over $A$, denoted $\mathcal{C} \backslash A$, is the comma category $K^{\mathbf{1}} A \downarrow 1_\mathcal{C}$.
\end{df}

\begin{df}[Pointed Sets]
    The \emph{category of pointed sets} $\Set_*$ is the coslice category $\Set \backslash 1$.
\end{df}

\chapter{Functors}

\begin{df}[Functor]
    Let $\mathcal{C}$ and $\mathcal{D}$ be categories. A \emph{functor} $F : \mathcal{C} \rightarrow \mathcal{D}$ consists of:
    \begin{itemize}
        \item for every object $A \in \mathcal{C}$, an object $FA \in \mathcal{D}$
        \item for any morphism $f : A \rightarrow B : \mathcal{C}$, a morphism $Ff : FA
                  \rightarrow FB : \mathcal{D}$
    \end{itemize}
    such that:
    \begin{itemize}
        \item $F \id{A} = \id{FA}$
        \item $F(g \circ f) = Fg \circ Ff$
    \end{itemize}
\end{df}

\begin{df}[Identity Functor]
    For any category $\mathcal{C}$, the \emph{identity functor} $1_\mathcal{C} : \mathcal{C} \rightarrow \mathcal{C}$ is defined by
    \begin{align*}
        1_\mathcal{C} A & = A \\
        1_\mathcal{C} f & = f
    \end{align*}
\end{df}

\begin{df}[Constant Functor]
    Given categories $\mathcal{C}$, $\mathcal{D}$ and an object $D \in \mathcal{D}$, the \emph{constant functor} $K^\mathcal{C} D : \mathcal{C} \rightarrow \mathcal{D}$ is the functor defined by
    \begin{align*}
        K^\mathcal{C} D C & = D      \\
        K^\mathcal{C} D f & = \id{D}
    \end{align*}
\end{df}

\begin{df}[Composition of Functors]
Given functors $F : \mathcal{C} \rightarrow \mathcal{D}$ and $G : \mathcal{D} \rightarrow \mathcal{E}$, define the \emph{composite} functor $G \circ F : \mathcal{C} \rightarrow \mathcal{E}$ by
\begin{align*}
(G \circ F) A & = G (F A) \\
(G \circ F) f & = G (F f)
\end{align*}
\end{df}

\begin{df}[Category of Categories]
For any universe $\mathcal{U}$, let $\mathbf{Cat}_\mathcal{U}$ be the category of categories whose sets of objects and morphisms are in $\mathcal{U}$, and functors.
\end{df}

\chapter{Constructions of Categories}

\section{Opposite Category}

\begin{df}[Opposite Category]
    For any category $\mathcal{C}$, the \emph{opposite} category $\mathcal{C}^\mathrm{op}$ is the category with the same objects as $\mathcal{C}$ and
    \[ \mathcal{C}^\mathrm{op}[A,B] = \mathcal{C}[B,A] \]
\end{df}

\section{Product Categories}

\begin{df}[Product Category]
Given categories $\mathcal{C}$ and $\mathcal{D}$, the \emph{product category} $\mathcal{C} \times \mathcal{D}$ has:
\begin{itemize}
\item objects $|\mathcal{C} \times \mathcal{D}| = |\mathcal{C}| \times |\mathcal{D}|$
\item a morphism $(C,D) \rightarrow (C',D')$ is a pair $(f,g)$ where $f : C \rightarrow C' : \mathcal{C}$ and $g : D \rightarrow D' : \mathcal{D}$
\item $\id{(C,D)} = (\id{C},\id{D})$
\item $(g',f') \circ (g,f) = (g' \circ g, f' \circ f)$
\end{itemize}

Define the \emph{projection} functors $\pi_1 : \mathcal{C} \times \mathcal{D} \rightarrow \mathcal{C}$ and $\pi_2 : \mathcal{C} \times \mathcal{D} \rightarrow \mathcal{D}$ by
\begin{align*}
\pi_1(C,D) & = C & \pi_2(C,D) & = D \\
\pi_1(f,g) & = f & \pi_2(f,g) & = g
\end{align*}
\end{df}

\section{Arrow Category}

\begin{df}[Arrow Category]
For any category $\mathcal{C}$, the \emph{arrow category} $\mathcal{C}^\rightarrow$ has:
\begin{itemize}
\item objects all triples $(A,B,f)$ where $A,B \in \mathcal{C}$ and $f : A \rightarrow B : \mathcal{C}$
\item morphisms $(u,v) : (A,B,f) \rightarrow (C,D,g)$ all pairs $(u : A \rightarrow C, v : B \rightarrow D)$ such that the following diagram commutes.
\[ \begin{tikzcd}
A \arrow[r,"f"] \arrow[d,"u"] & B \arrow[d,"v"] \\
C \arrow[r,"g"] & D
\end{tikzcd} \]
\end{itemize}

We have the domain and codomain functors $\dom : \mathcal{C}^\rightarrow \rightarrow \mathcal{C}$ and $\cod : \mathcal{C}^\rightarrow \rightarrow \mathcal{C}$ given by
\begin{align*}
\dom (A,B,f) & = A & \cod (A,B,f) & = B \\
\dom(u,v) & = u & \cod (u,v) & = v
\end{align*}
\end{df}

\section{Slice Category}

\begin{df}[Slice Category]
Let $\mathcal{C}$ be a category and $A \in \mathcal{C}$. The \emph{slice category} $\mathcal{C} / A$ is the category with
\begin{itemize}
\item objects all pairs $(B,f)$ such that $f : B \rightarrow A$
\item morphisms $u : (B,f) \rightarrow (C,g)$ all morphisms $u : B \rightarrow C$ such that $g \circ u = f$.
\end{itemize}
\end{df}

\begin{df}[Slice Category]
Let $\mathcal{C}$ be a category and $A \in \mathcal{C}$. The \emph{coslice category} $A \backslash \mathcal{C}$ is the category with
\begin{itemize}
\item objects all pairs $(B,f)$ such that $f : A \rightarrow B$
\item morphisms $u : (B,f) \rightarrow (C,g)$ all morphisms $u : B \rightarrow C$ such that $u \circ f = g$.
\end{itemize}
\end{df}

\begin{ex}
The \emph{category of pointed sets} $\mathbf{Set}_*$ is $\mathbf{1} \backslash \mathbf{Set}$.
\end{ex}

\chapter{Preorders}

\section{Definition}

\begin{df}[Thin Category]
A category $\mathcal{C}$ is \emph{thin} or a \emph{preorder} iff, for any objects $A$ and $B$, there is at most one morphism $A \rightarrow B$. We write $A \leq B$ iff there exists a morphism $A \rightarrow B$; this is called the \emph{ordering relation} on $\mathcal{C}$.
\end{df}

\begin{prop}
For any preorder $\mathcal{C}$, the relation $\leq$ is reflexive and transitive. Conversely, given any class $A$ and relation $\leq$ on $A$ that is reflexive and transitive, there exists a preorder $A$ with class of objects $A$, unique up to unique isomorphism that is the identity on objects, such that $\leq$ is the ordering relation on $A$.
\end{prop}

\begin{proof}
\pf\ All parts are immediate from definitions. \qed
\end{proof}

\begin{prop}
Let $\mathcal{C}$ and $\mathcal{D}$ be preorders and $F : \mathcal{C} \rightarrow \mathcal{D}$ be a functor. Then $F$ is \emph{monotone}: for all $x,y \in \mathcal{C}$, if $x \leq y$ then $F(x) \leq F(y)$.

Conversely, given any monotone function $f$ from the objects of $\mathcal{C}$ to the objects of $\mathcal{D}$, there exists a unique functor whose action on objects is $f$.
\end{prop}

\begin{proof}
\pf\ Immediate from definitions. \qed
\end{proof}

\begin{ex}[Discrete Category]
For any set $A$, the \emph{discrete} category $A$ is the preorder with objects the elements of $A$ and order relation $=$.
\end{ex}

\begin{ex}
For any ordinal $\alpha$, let $\mathbf{\alpha}$ be the preorder $\{ \beta : \beta < \alpha \}$ under $\leq$.
\end{ex}

%Examples

%Basic Properties

%Ways of constructing preordered sets

\section{Partial Orders}

\subsection{Definition}

\begin{df}[Partial Order]
A \emph{partial order}, \emph{partially ordered set} or \emph{poset} is a preorder such that, for any $x$ and $y$, if $x \leq y$ and $y \leq x$ then $x = y$.
\end{df}

\begin{ex}
Every discrete category is a poset.
\end{ex}

\begin{ex}
For any ordinal $\alpha$, the preorder $\mathbf{\alpha}$ is a poset.
\end{ex}

\begin{df}[Category of Posets]
Let $\mathbf{Pos}$ be full subcategory of $\mathbf{Cat}$ whose objects are the posets.
\end{df}

% Examples
% Basic Properties
% Relation with Previous Concepts
% Ways of Constructing X
% Special Kinds

\chapter{Objects}


\section{Terminal and Initial Objects}

\begin{df}[Terminal Object]
Let $\mathcal{C}$ be a category.
An object $T \in \mathcal{C}$ is \emph{terminal} iff, for every object $X$, there is exactly one morphism $X \rightarrow T$.
\end{df}

\begin{ex}$ $
\begin{itemize}
\item The terminal objects in $\Set$ are the singletons.
\item $\mathbf{1}$ is terminal in $\Cat$.
\end{itemize}
\end{ex}

\begin{df}[Initial Object]
Let $\mathcal{C}$ be a category.
An object $I \in \mathcal{C}$ is \emph{initial} iff, for every object $X$, there is exactly one morphism $I \rightarrow X$.
\end{df}

\begin{ex}$ $
\begin{itemize}
\item
The empty set is the initial object in $\Set$.
\item
$\mathbf{0}$ is initial in $\Cat$.
\end{itemize}
\end{ex}