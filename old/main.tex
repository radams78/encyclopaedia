\documentclass{book}

\usepackage{amsmath}
\usepackage{amssymb}
\usepackage{amsthm}
\let\proof\relax
\let\endproof\relax
\let\qed\relax
\usepackage{pf2}
\usepackage{hyperref}

\title{Mathematics}
\author{Robin Adams}

\newtheorem{ax}{Axiom}[section]
\newtheorem{axs}[ax]{Axiom Schema}
\newtheorem{prop}[ax]{Proposition}
\newtheorem{props}[ax]{Proposition Schema}
\newtheorem{cor}{Corollary}[ax]
\newtheorem{cors}{Corollary Schema}[ax]
\newtheorem{thm}[ax]{Theorem}
\newtheorem{thms}[ax]{Theorem Schema}
\newtheorem{lm}[ax]{Lemma}
\theoremstyle{definition}
\newtheorem{df}[ax]{Definition}
\newtheorem{dfs}[ax]{Definition Schema}

\newcommand{\dom}{\ensuremath{\operatorname{dom}}}
\newcommand{\ran}{\ensuremath{\operatorname{ran}}}
\newcommand{\spn}{\ensuremath{\operatorname{span}}}
\newcommand{\cl}{\ensuremath{\operatorname{cl}}}
\newcommand{\sgn}{\ensuremath{\operatorname{sgn}}}
\newcommand{\seg}{\ensuremath{\operatorname{seg}}}
\newcommand{\id}[1]{\ensuremath{\mathrm{id}_{#1}}}
\newcommand{\rank}{\ensuremath{\operatorname{rank}}}
\newcommand{\cf}{\ensuremath{\operatorname{cf}}}
\newcommand{\ssup}{\ensuremath{\operatorname{ssup}}}
\newcommand{\supp}{\ensuremath{\operatorname{supp}}}
\newcommand{\tr}{\ensuremath{\operatorname{tr}}}
\newcommand{\card}{\ensuremath{\operatorname{card}}}

\begin{document}

\maketitle
\tableofcontents

\chapter{Sets and Classes}

\section{Classes}

Our language is the language of first-order logic with equality over one primitive binary predicate $\in$. We call all the objects we reason about \emph{sets}. When $a \in b$, we say $a$ is a \emph{member} or \emph{element} of $b$, or $b$ \emph{contains} $a$. We write $b \ni a$ for $a \in b$, and $a \notin b$ for $\neg(a \in b)$. We write $\forall x \in a. \phi$ as an abbreviation for $\forall x(x \in a \rightarrow \phi)$, and $\exists x \in a. \phi$ as an abbreviation for $\exists x(x \in a \wedge \phi)$.

We shall speak informally of \emph{classes} as an abbreviation for talking about predicates. A \emph{class} is determined by a unary predicate $\phi[x]$ (possibly with parameters). We write $\{ x \mid \phi[x] \}$ or $\{ x : \phi[x] \}$ for the class determined by $\phi[x]$. We write '$a$ is an element of $\{x \mid \phi[x]\}$' or '$a \in \{x \mid \phi[x]\}$' for $\phi[a]$.

We write $\{t[x_1, \ldots, x_n] \mid P[x_1, \ldots, x_n] \}$ for 
\[ \{y \mid \exists x_1, \ldots, x_n (y = t[x_1, \ldots, x_n] \wedge P[x_1, \ldots, x_n]) \} \enspace . \]

We say two classes $\mathbf{A}$ and $\mathbf{B}$ are \emph{equal}, and write $\mathbf{A} = \mathbf{B}$, iff $\forall x (x \in \mathbf{A} \leftrightarrow x \in \mathbf{B})$.

\begin{props}
For any class $\mathbf{A}$, the following is a theorem.

\[ \mathbf{A} = \mathbf{A} \]
\end{props}

\begin{proof}
\pf\ We have $\forall x (x \in \mathbf{A} \Leftrightarrow x \in \mathbf{A})$. \qed
\end{proof}

\begin{props}
For any classes $\mathbf{A}$ and $\mathbf{B}$, the following is a theorem.

If $\mathbf{A} = \mathbf{B}$ then $\mathbf{B} = \mathbf{A}$.
\end{props}

\begin{proof}
\pf\ If $\forall x (x \in \mathbf{A} \Leftrightarrow x \in \mathbf{B})$ then $\forall x (x \in \mathbf{B} \Leftrightarrow x \in \mathbf{A})$. \qed
\end{proof}

\begin{props}
For any classes $\mathbf{A}$, $\mathbf{B}$ and $\mathbf{C}$, the following is a theorem.

If $\mathbf{A} = \mathbf{B}$ and $\mathbf{B} = \mathbf{C}$ then $\mathbf{A} = \mathbf{C}$.
\end{props}

\begin{proof}
\pf\ If $\forall x (x \in \mathbf{A} \Leftrightarrow x \in \mathbf{B})$ and $\forall x (x \in \mathbf{B} \Leftrightarrow x \in \mathbf{C})$ then $\forall x (x \in \mathbf{A} \Leftrightarrow x \in \mathbf{C})$. \qed
\end{proof}

\subsection{Subclasses}

\begin{df}[Subclass]
We say a class $\mathbf{A}$ is a \emph{subclass} of $\mathbf{B}$, or $\mathbf{B}$ is a \emph{superclass} of $\mathbf{A}$, or $\mathbf{B}$ \emph{includes} $\mathbf{A}$, and write $\mathbf{A} \subseteq \mathbf{B}$ or $\mathbf{B} \supseteq \mathbf{A}$, iff every element of $\mathbf{A}$ is an element of $\mathbf{B}$. Otherwise we write $\mathbf{A} \nsubseteq \mathbf{B}$ or $\mathbf{B} \nsupseteq \mathbf{A}$.

We say $\mathbf{A}$ is a \emph{proper} subclass of $\mathbf{B}$, $\mathbf{B}$ is a \emph{proper} superclass of $\mathbf{A}$, or $\mathbf{B}$ \emph{properly} includes $\mathbf{A}$, and write $\mathbf{A} \subsetneq \mathbf{B}$ or $\mathbf{B} \supsetneq \mathbf{A}$, iff $\mathbf{A} \subseteq \mathbf{B}$ and $\mathbf{A} \neq \mathbf{B}$.
\end{df}

\begin{props}
For any class $\mathbf{A}$, the following is a theorem.

\[ \mathbf{A} \subseteq \mathbf{A} \]
\end{props}

\begin{proof}
\pf\ Every element of $\mathbf{A}$ is an element of $\mathbf{A}$. \qed
\end{proof}

\begin{props}
For any classes $\mathbf{A}$ and $\mathbf{B}$, the following is a theorem.

If $\mathbf{A} \subseteq \mathbf{B}$ and $\mathbf{B} \subseteq \mathbf{A}$ then $\mathbf{A} = \mathbf{B}$.
\end{props}

\begin{proof}
\pf\ If every element of $\mathbf{A}$ is an element of $\mathbf{B}$, and every element of $\mathbf{B}$ is an element of $\mathbf{A}$, then $\mathbf{A}$ and $\mathbf{B}$ have exactly the same elements. \qed
\end{proof}

\begin{props}
For any classes $\mathbf{A}$, $\mathbf{B}$ and $\mathbf{C}$, the following is a theorem.

If $\mathbf{A} \subseteq \mathbf{B}$ and $\mathbf{B} \subseteq \mathbf{C}$ then $\mathbf{A} \subseteq \mathbf{C}$.
\end{props}

\begin{proof}
\pf\ If every element of $\mathbf{A}$ is an element of $\mathbf{B}$ and every element of $\mathbf{B}$ is an element of $\mathbf{C}$ then every element of $\mathbf{A}$ is an element of $\mathbf{C}$. \qed
\end{proof}

\subsection{Constructions of Classes}

\begin{df}[Empty Class]
The \emph{empty class} $\emptyset$ is $\{ x \mid \bot \}$. Every other class is \emph{nonempty}.
\end{df}

\begin{df}[Universal Class]
The \emph{universal class} $\mathbf{V}$ is $\{ x \mid \top \}$.
\end{df}

\begin{df}[Enumeration]
Given objects $a_1$, \ldots, $a_n$, we define the class $\{ a_1, \ldots, a_n \}$ to be the class $\{ x \mid x = a_1 \vee \cdots \vee x = a_n \}$.
\end{df}

\begin{df}[Intersection]
For any classes $\mathbf{A}$ and $\mathbf{B}$, the \emph{intersection} $\mathbf{A} \cap \mathbf{B}$ is $\{ x \mid x \in \mathbf{A} \wedge x \in \mathbf{B}\}$.
\end{df}

\begin{df}[Union]
For any classes $\mathbf{A}$ and $\mathbf{B}$, the \emph{union} $\mathbf{A} \cup \mathbf{B}$ is $\{ x \mid x \in \mathbf{A} \vee x \in \mathbf{B} \}$.
\end{df}

\begin{df}[Relative Complement]
Let $\mathbf{A}$ and $\mathbf{B}$ be classes. The \emph{relative complement} of $\mathbf{B}$ in $\mathbf{A}$ is the class $\mathbf{A} - \mathbf{B} := \{x \in \mathbf{A} \mid x \notin \mathbf{B} \}$.
\end{df}

\begin{df}[Symmetric Difference]
For any classes $\mathbf{A}$ and $\mathbf{B}$, the \emph{symmetric difference} is the class $\mathbf{A} + \mathbf{B} := (\mathbf{A} - \mathbf{B}) \cup (\mathbf{B} - \mathbf{A})$.
\end{df}

\begin{df}[Pairwise disjoint]
Let $\mathbf{A}$ be a class. We say the elements of $\mathbf{A}$ are \emph{pairwise disjoint} iff, for all $x, y \in \mathbf{A}$, if $x \cap y \neq \emptyset$ then $x = y$.
\end{df}

\section{Sets and the Axiom of Extensionality}
	
\begin{df}[Axiom of Extensionality]
The \emph{Axiom of Extensionality} is the statement: if two sets have exactly the same members, then they are equal.
\[ \forall x, y (\forall z (z \in x \Leftrightarrow z \in y) \Rightarrow x = y) \enspace . \]

When working in a theory with the Axiom of Extensionality, we may identify a set $a$ with the class $\{x \mid x \in a\}$. Our use of the symbols $\in$ and $=$ is consistent. We say a class $\mathbf{A}$ \emph{is a set} iff there exists a set $a$ such that $a = \mathbf{A}$; that is, $\{x \mid \phi[x]\}$ is a set iff $\exists a \forall x (x \in a \leftrightarrow \phi[x])$. Otherwise, $\mathbf{A}$ is a \emph{proper class}.
\end{df}

\begin{df}[Subset]
If $A$ is a set and $A \subseteq \mathbf{B}$, we say $A$ is a \emph{subset} of $\mathbf{B}$.
\end{df}

\begin{df}[Union]
The \emph{union} of a class $\mathbf{A}$ is $\{ x \mid \exists X \in \mathbf{A}. x \in X\}$.
We write $\bigcup_{P(x)} t(x)$ for $\bigcup \{ t(x) \mid P(x) \}$.
\end{df}

\begin{df}[Intersection]
The \emph{intersection} of a class $\mathbf{A}$ is $\{ x \mid \forall X \in \mathbf{A}. x \in X \}$. We write $\bigcap_{P(x)} t(x)$ for $\bigcap \{ t(x) \mid P(x) \}$.
\end{df}

\begin{df}[Power Class]
For any class $\mathbf{A}$, the \emph{power class} $\mathcal{P} \mathbf{A}$ is $\{ X \mid X \subseteq \mathbf{A} \}$.
\end{df}

\section{The Other Axioms}

\begin{df}[Empty Set Axiom]
The \emph{Empty Set Axiom} is the statement: The empty class $\emptyset$ is a set.
\[ \exists e \forall x x \notin e \]
\end{df}

\begin{df}[Pairing Axiom]
The \emph{Pairing Axiom} or \emph{Pair Set Axiom} is the statement: for any sets $a$ and $b$, the class $\{a,b\}$ is a set.
\[ \forall a \forall b \exists c \forall x (x \in c \Leftrightarrow x = a \vee x = b) \]
\end{df}

\begin{df}[Union Axiom]
The \emph{Union Axiom} is the statement: for any set $A$, the class $\bigcup A$ is a set.
\[ \forall A \exists B \forall x (x \in B \Leftrightarrow \exists y (y \in A \wedge x \in y)) \]
\end{df}

\begin{df}[Comprehension Axiom Scheme]
The \emph{Comprehension Axiom Scheme}, \emph{Subset Axiom Scheme} or \emph{Aussonderungsaxiom Scheme} is the set of sentences of the form, for any class $\mathbf{A}$: If $\mathbf{A}$ is a subclass of a set then $\mathbf{A}$ is a set.

That is, for any property $P[x,y_1,\ldots,y_n]$:

For any sets $a_1$, \ldots, $a_n$ and $B$, the class $\{ x \in B \mid P[x,a_1, \ldots, a_n] \}$ is a set.
\[ \forall a_1, \ldots, a_n, B. \exists C. \forall x (x \in C \Leftrightarrow x \in B \wedge P[x,a_1, \ldots, a_n]) \]
\end{df}

\begin{df}[Replacement Axiom Scheme]
The \emph{Replacement Axiom Scheme} is the set of sentences of the form, for some property $P[x,y,z_1, \ldots, z_n]$:

For any sets $a_1$, \ldots, $a_n$, $B$, assume for all $x \in B$ there exists at most one $y$ such that $P[x,y,a_1, \ldots, a_n]$. Then $\{ y \mid \exists x \in B. P[x,y,a_1, \ldots, a_n]$ is a set.

\[ \forall a_1, \ldots, a_n, B (\forall x \in B. \forall y,y'(P[x,y,a_1, \ldots, a_n] \wedge P[x,y',a_1, \ldots, a_n] \Rightarrow y = y') \Rightarrow \]
\[ \exists C \forall y (y \in C \Leftrightarrow \exists x \in B. P[x,y,a_1, \ldots, a_n])) \]
\end{df}

\begin{df}[Power Set Axiom]
The \emph{Power Set Axiom} is the statement: the power class of a set is a set.
\[ \forall A \exists B \forall x (x \in B \Leftrightarrow \forall y (y \in x \Rightarrow y \in A)) \]
\end{df}

\begin{df}[Axiom of Infinity]
The \emph{Axiom of Infinity} is the statement: there exists a set $I$ such that $\emptyset \in I$ and $\forall x \in I. x \cup \{x\} \in I$.
\[ \exists I (\exists e \in I. \forall x. x \notin e \wedge \forall x \in I. \exists y \in I. \forall z (z \in y \Leftrightarrow z \in x \vee z = x)) \]
\end{df}

\begin{df}[Axiom of Choice]
The \emph{Axiom of Choice} is the statement: 
For any set $A$ of pairwise disjoint, nonempty sets, there exists a set $C$ such that, for all $x \in A$, we have $x \cap C$ has exactly one element.

\begin{align*}
& \forall A ( \forall x \in A. \exists y y \in x \wedge \\
& \forall x,y \in A. \forall z(z \in x \wedge z \in y \Rightarrow x = y) \Rightarrow \\
& \exists C. \forall x \in A. \exists y \forall z (z \in x \wedge z \in C \Leftrightarrow z = y))
\end{align*}
\end{df}

\begin{df}[Axiom of Regularity]
The \emph{Axiom of Regularity} or \emph{Axiom of Foundation} is the statement:
for any $A$, if $A$ has a member, then there exists $m \in A$ such that $m \cap A = \emptyset$.

\[ \forall A (\exists x. x \in A \Rightarrow \exists m \in A. \neg \exists x (x \in m \wedge x \in A)) \]
\end{df}

\begin{df}[Skolem Set Theory]
\emph{Skolem set theory} (S) is the theory whose axioms are:
\begin{itemize}
\item Extensionality
\item Empty Set
\item Pairing
\item Union
\item Comprehension
\item Power Set
\item Regularity
\end{itemize}
Let SC be the extension of S with the Axiom of Choice.
\end{df}

\begin{df}[Skolem-Fraenkel Set Theory]
\emph{Skolem-Fraenkel set theory} (SF) is the theory whose axioms are:
\begin{itemize}
\item Extensionality
\item Empty Set
\item Union
\item Replacement
\item Power Set
\item Regularity
\end{itemize}

Let SFC be the extension of SF with the Axiom of Choice.
\end{df}

\begin{df}[Zermelo Set Theory]
\emph{Zermelo set theory} is the theory whose axioms are:
\begin{itemize}
\item Extensionality
\item Pairing
\item Union
\item Comprehension
\item Power Set
\item Infinity
\item Regularity
\end{itemize}
We label theorems with Z when they are provable in Zermelo set theory.

Let ZC be the extension of Z with the Axiom of Choice.
\end{df}

\begin{df}[Fraenkel-Mostowski Set Theory]
\emph{Fraenkel-Mostowski set theory} (FM) is the theory whose axioms are:
\begin{itemize}
\item The \emph{Axiom of Extensionality with Urelements}: For any sets $x$ and $y$, if $x$ is nonempty and $x$ and $y$ have exactly the same elements, then $x = y$.
\item Union
\item Replacement
\item Power Set
\item Infinity
\item Regularity
\end{itemize}
We write FMC for the extension of FM with Choice.
\end{df}

\begin{df}[Zermelo-Fraenkel Set Theory]
\emph{Zermelo-Fraenkel set theory} is the theory whose axioms are:
\begin{itemize}
\item Extensionality
\item Union
\item Replacement
\item Power Set
\item Infinity
\item Regularity
\end{itemize}
We label theorems with ZF when they are provable in Zermelo-Fraenkel set theory.

Let ZFC be the extension of ZF with the Axiom of Choice.
\end{df}

We label a theorem with FOL if it can be proved in first-order logic, i.e. from no axioms.

\section{ZFC Extends Z}

\begin{prop}
The Axiom of Infinity implies the Empty Set Axiom.
\end{prop}

\begin{proof}
\pf\ Trivial. \qed
\end{proof}

\begin{cor}
Z is an extension of S.

ZF is an extension of SF.

ZC is an extension of SC.

ZFC is an extension of SFC.
\end{cor}

\begin{prop}
The Axiom of Pairing is a theorem of SF without Foundation.
\end{prop}

\begin{proof}
\pf
\step{0}{\pflet{$a,b$ be sets.}}
\step{1}{\pflet{$P(x,y)$ be the predicate $(x = \emptyset \wedge y = a) \vee (x = \mathcal{P} \emptyset \wedge y = b)$.}}
\step{2}{For all $x \in \mathcal{P} \mathcal{P} \emptyset$, there exists at most one $y$ such that $P(x,y)$.}
\begin{proof}
	\step{a}{\pflet{$x \in \mathcal{P} \mathcal{P} \emptyset$}}
	\step{b}{\pflet{$y$ and $y'$ be sets.}}
	\step{c}{\assume{$P(x,y)$ and $P(x,y')$}}
	\step{d}{$(x = \emptyset \wedge y = a) \vee (x = \mathcal{P} \emptyset \wedge y = b)$}
	\begin{proof}
		\pf\ From \stepref{c}.
	\end{proof}
	\step{e}{$(x = \emptyset \wedge y' = a) \vee (x = \mathcal{P} \emptyset \wedge y' = b)$}
	\begin{proof}
		\pf\ From \stepref{c}.
	\end{proof}
	\step{f}{$\emptyset \neq \mathcal{P} \emptyset$}
	\begin{proof}
		\pf\ Since $\emptyset \in \mathcal{P} \emptyset$ and $\emptyset \notin \emptyset$.
	\end{proof}
	\step{g}{$y = y'$}
\end{proof}
\step{3}{\pflet{$A$ be the set $\{ y \mid \exists x \in \mathcal{P} \mathcal{P} \emptyset. P(x,y)\}$.}}
\step{4}{$A = \{a,b\}$}
\qed
\end{proof}

\begin{props}
Every instance of the Comprehension Axiom Scheme is a theorem of SF.
\end{props}

\begin{proof}
\pf
\step{0}{\pflet{$P(x)$ be a predicate.}}
\step{1}{\pflet{$A$ be a set.}}
\step{2}{\pflet{$Q(x,y)$ be the predicate $P(x) \wedge y = x$.}}
\step{3}{For all $x \in A$, there exists at most one $y$ such that $Q(x,y)$.}
\begin{proof}
	\step{3a}{\pflet{$x \in A$}}
	\step{3b}{\pflet{$y$ and $y'$ be sets.}}
	\step{3c}{\assume{$Q(x,y)$ and $Q(x,y')$}}
	\step{3d}{$x \in A \wedge P(x) \wedge y = x \wedge y' = x$}
	\begin{proof}
		\pf\ From \stepref{3c}.
	\end{proof}
	\step{3e}{$y = y'$}
	\begin{proof}
		\pf\ From \stepref{3d}.
	\end{proof}
\end{proof}
\step{4}{\pflet{$B$ be the set $\{y \mid \exists x \in A. Q(x,y)\}$}}
\begin{proof}
	\pf\ This is a set by an Axiom of Replacement and \stepref{3}.
\end{proof}
\step{5}{$B = \{y \in A \mid P(y)\}$}
\begin{proof}
	\pf
	\begin{align*}
		y \in B & \Leftrightarrow \exists x \in A. Q(x,y) & (\text{\stepref{4}}) \\
		& \Leftrightarrow \exists x \in A (P(x) \wedge y = x) & (\text{\stepref{2}}) \\
		& \Leftrightarrow P(y)
	\end{align*}
\end{proof}
\qed
\end{proof}

\begin{cor}
SF without Foundation is an extension of S without Foundation.

SFC without Foundation is an extension of SC without Foundation.

ZF without Foundation is an extension of Z without Foundation.

ZFC without Foundation is an extension of ZC without Foundation.

SF is an extension of S.

SFC is an extension of SC.

ZF is an extension of Z.

ZFC is an extension of ZC.
\end{cor}

\section{Consequences of the Axioms}

\begin{prop}[S without Foundation]
The union of two sets is a set.
\end{prop}

\begin{proof}
\pf\ Because $A \cup B = \bigcup \{A,B\}$. \qed
\end{proof}

\begin{props}[S without Foundation]
For any number $n$, the following is a theorem:

For any sets $a_1$, \ldots, $a_n$, the class $\{a_1, \ldots, a_n\} = \{x \mid x = a_1 \vee \cdots \vee x = a_n\}$ is a set.
\end{props}

\begin{proof}
\pf\ The case $n=1$ follows from Pairing since $\{a\} = \{a,a\}$.

If we have proved the theorem for $n$ we have $\{a_1, \ldots, a_n, a_{n+1}\} = \{a_1, \ldots, a_n\} \cup \{a_{n+1}\}$. \qed
\end{proof}

\begin{prop}[S]
\label{prop:xnotinx}
No set is a member of itself.
\end{prop}

\begin{proof}
\pf
\step{1}{\pflet{$x$ be any set.}}
\step{2}{\pick\ $m \in \{x\}$ such that $m \cap \{x\} = \emptyset$.}
\begin{proof}
	\pf\ Axiom of Foundation.
\end{proof}
\step{3}{$m = x$}
\step{4}{$x \cap \{x\} = \emptyset$}
\step{5}{$x \notin x$}
\qed
\end{proof}

\begin{thm}[Russell's Paradox (S without Foundation)]
The universal class $\mathbf{V}$ is a proper class.
\end{thm}

\begin{proof}
\pf\ If $\mathbf{V}$ is a set then $R = \{ x \in \mathbf{V} \mid x \notin x \}$ is a set. Then we have $R \in R$ if and only if $R \notin R$, which is a contradiction. \qed
\end{proof}

\begin{prop}[S]
There are no sets $a$ and $b$ such that $a \in b$ and $b \in a$.
\end{prop}

\begin{proof}
\pf
\step{1}{\pflet{$a$ and $b$ be any sets.}}
\step{2}{\pick\ $m \in \{a,b\}$ such that $m \cap \{a,b\} = \emptyset$}
\step{3}{\case{$m = a$}}
\begin{proof}
	\pf\ Then $b \notin a$.
\end{proof}
\step{4}{\case{$m = b$}}
\begin{proof}
	\pf\ Then $a \notin b$.
\end{proof}
\qed
\end{proof}

\begin{prop}[S without Foundation]
The intersection of a set and a class is a set.
\end{prop}

\begin{proof}
\pf\ Immediate from Comprehension. \qed
\end{proof}

\begin{prop}[S without Foundation]
The relative complement of a class in a set is a set.
\end{prop}

\begin{proof}
\pf\ Immediate from Comprehension. \qed
\end{proof}

\begin{cor}[S without Foundation]
The symmetric difference of two sets is a set.
\end{cor}

\begin{prop}[S without Foundation]
The intersection of a nonempty class is a set.
\end{prop}

\begin{proof}
\pf
\step{1}{\pflet{$\mathbf{A}$ be a nonempty class.}}
\step{2}{\pick\ $B \in \mathbf{A}$}
\step{3}{$\bigcap \mathbf{A} \subseteq B$}
\step{4}{$\bigcap \mathbf{A}$ is a set.}
\begin{proof}
	\pf\ By Comprehension.
\end{proof}
\qed
\end{proof}

\begin{props}[FOL]
\label{prop:powermono}
For any classes $\mathbf{A}$ and $\mathbf{B}$, the following is a theorem:

If $\mathbf{A} \subseteq \mathbf{B}$ then $\mathcal{P} \mathbf{A} \subseteq \mathcal{P} \mathbf{B}$.
\end{props}

\begin{proof}
\pf\ Every subset of $\mathbf{A}$ is a subset of $\mathbf{B}$. \qed
\end{proof}

\begin{props}[FOL]
\label{prop:unionmonotone}
For any classes $\mathbf{A}$ and $\mathbf{B}$, the following is a theorem:

If $\mathbf{A} \subseteq \mathbf{B}$ then $\bigcup \mathbf{A} \subseteq \bigcup \mathbf{B}$.
\end{props}

\begin{proof}
\pf\ If $x \in X \in \mathbf{A}$ then $x \in X \in \mathbf{B}$. \qed
\end{proof}

\begin{props}[S without Foundation]
\label{prop:UPA}
For any class $\mathbf{A}$, the following is a theorem:

\[ \mathbf{A} = \bigcup \mathcal{P} \mathbf{A} \]
\end{props}

\begin{proof}
\pf
\step{1}{$\mathbf{A} \subseteq \bigcup \mathcal{P} \mathbf{A}$}
\begin{proof}
	\pf\ For all $x \in \mathbf{A}$ we have $x \in \{x\} \in \mathcal{P} \mathbf{A}$.
\end{proof}
\step{2}{$\bigcup \mathcal{P} \mathbf{A} \subseteq \mathbf{A}$}
\begin{proof}
	\step{a}{\pflet{$x \in \bigcup \mathcal{P} \mathbf{A}$}}
	\step{b}{\pick\ $X \in \mathcal{P} \mathbf{A}$ such that $x \in X$}
	\step{c}{$X \subseteq \mathbf{A}$}
	\step{d}{$x \in \mathbf{A}$}
\end{proof}
\qed
\end{proof}

\section{Transitive Classes}

\begin{df}[Transitive Class]
A class $\mathbf{A}$ is a \emph{transitive class} iff whenever $x \in y \in \mathbf{A}$ then $x \in \mathbf{A}$.
\end{df}

\begin{props}[FOL]
\label{prop:transitiveset}
For any class $\mathbf{A}$, the following is a theorem:

The following are equivalent.
\begin{enumerate}
\item
$\mathbf{A}$ is a transitive class.
\item
$\bigcup \mathbf{A} \subseteq \mathbf{A}$
\item
Every element of $\mathbf{A}$ is a subset of $\mathbf{A}$.
\item
$\mathbf{A} \subseteq \mathcal{P} \mathbf{A}$
\end{enumerate}
\end{props}

\begin{proof}
\pf\ Immediate from definitions. \qed
\end{proof}

\begin{props}[FOL]
\label{prop:uniontransitive}
For any class $\mathbf{A}$, the following is a theorem:

If $\mathbf{A}$ is a transitive class then $\bigcup \mathbf{A}$ is a transitive class.
\end{props}

\begin{proof}
\pf
\step{1}{\assume{$\mathbf{A}$ is a transitive class.}}
\step{2}{\pflet{$x \in y \in \bigcup \mathbf{A}$}}
\step{3}{$y \in \mathbf{A}$}
\begin{proof}
	\pf\ Since $\bigcup \mathbf{A} \subseteq \mathbf{A}$ by Proposition \ref{prop:transitiveset}.
\end{proof}
\step{4}{$x \in \bigcup \mathbf{A}$}
\qed
\end{proof}

\begin{props}[S without Foundation]
\label{prop:powtransitive}
For any class $\mathbf{A}$, the following is a theorem:

We have $\mathbf{A}$ is a transitive class if and only if $\mathcal{P} \mathbf{A}$ is a transitive class.
\end{props}

\begin{proof}
\pf
\step{1}{If $\mathbf{A}$ is a transitive class then $\mathcal{P} \mathbf{A}$ is a transitive class.}
\begin{proof}
	\step{a}{\assume{$\mathbf{A}$ is a transitive class.}}
	\step{b}{$\mathbf{A} \subseteq \mathcal{P} \mathbf{A}$}
	\begin{proof}
		\pf\ Proposition \ref{prop:transitiveset}.
	\end{proof}
	\step{c}{$\mathcal{P} \mathbf{A} \subseteq \mathcal{P} \mathcal{P} \mathbf{A}$}
	\begin{proof}
		\pf\ Proposition \ref{prop:powermono}.
	\end{proof}
	\step{d}{$\mathcal{P} \mathbf{A}$ is a transitive class.}
	\begin{proof}
		\pf\ Proposition \ref{prop:transitiveset}.
	\end{proof}
\end{proof}
\step{2}{If $\mathcal{P} \mathbf{A}$ is a transitive class then $\mathbf{A}$ is a transitive class.}
\begin{proof}
	\step{a}{\assume{$\mathcal{P} \mathbf{A}$ is a transitive class.}}
	\step{b}{$\bigcup \mathcal{P} \mathbf{A} \subseteq \mathcal{P} \mathbf{A}$}
	\begin{proof}
		\pf\ Proposition \ref{prop:transitiveset}.
	\end{proof}
	\step{c}{$\mathbf{A} \subseteq \mathcal{P} \mathbf{A}$}
	\begin{proof}
		\pf\ Proposition \ref{prop:UPA}.
	\end{proof}
	\step{d}{$\mathbf{A}$ is a transitive class.}
	\begin{proof}
		\pf\ Proposition \ref{prop:transitiveset}.
	\end{proof}
\end{proof}
\qed
\end{proof}

\begin{props}[FOL]
For any class $\mathbf{A}$, the following is a theorem:

If every member of $\mathbf{A}$ is a transitive set then $\bigcup \mathbf{A}$ is a transitive class.
\end{props}

\begin{proof}
\pf
\step{1}{\assume{Every member of $\mathbf{A}$ is a transitive set.}}
\step{2}{\pflet{$x \in y \in \bigcup \mathbf{A}$}}
\step{3}{\pick\ $A \in \mathbf{A}$ such that $y \in A$.}
\step{4}{$x \in A$}
\begin{proof}
	\pf\ Since $A$ is a transitive set.
\end{proof}
\step{5}{$x \in \bigcup \mathbf{A}$}
\qed
\end{proof}

\begin{props}[FOL]
\label{prop:inttransitive}
For any class $\mathbf{A}$, the following is a theorem:

If every member of $\mathbf{A}$ is a transitive set then $\bigcap \mathbf{A}$ is a transitive class.
\end{props}

\begin{proof}
\pf
\step{1}{\assume{Every member of $\mathbf{A}$ is a transitive set.}}
\step{2}{\pflet{$x \in y \in \bigcap \mathbf{A}$} \prove{$x \in \bigcap \mathbf{A}$}}
\step{3}{\pflet{$A \in \mathbf{A}$}}
\step{4}{$y \in A$}
\step{5}{$x \in A$}
\begin{proof}
	\pf\ Since $A$ is a transitive set.
\end{proof}
\qed
\end{proof}

\chapter{Relations}

\section{Ordered Pairs}

\begin{df}[Ordered Pair]
For any sets $a$ and $b$, the \emph{ordered pair} $(a,b)$ is defined to be $\{ \{ a \}, \{a , b \} \}$.
\end{df}

\begin{thm}[S without Foundation]
For any sets $a$, $b$, $c$, $d$, we have $(a,b) = (c,d)$ if and only if $a = c$ and $b = d$.
\end{thm}

\begin{proof}
\pf
\step{1}{If $(a,b) = (c,d)$ then $a = c$ and $b = d$.}
\begin{proof}
	\step{a}{\assume{$\{\{a\},\{a,b\}\} = \{\{c\},\{c,d\}\}$}}
	\step{b}{$\bigcap \{\{a\},\{a,b\}\} = \bigcap \{\{c\},\{c,d\}\}$}
	\step{c}{$\{a\} = \{c\}$}
	\step{d}{$a=c$}
	\step{e}{$\bigcup\{\{a\},\{a,b\}\} = \bigcup \{\{c\},\{c,d\}\}$}
	\step{f}{$\{a,b\} = \{c,d\}$}
	\step{g}{$b = c$ or $b = d$}
	\step{h}{$a = d$ or $b = d$}
	\step{i}{If $b = c$ and $a = d$ then $b = d$}
	\begin{proof}
		\pf\ By \stepref{d}.
	\end{proof}
	\step{j}{$b = d$}
	\begin{proof}
		\pf\ From \stepref{g}, \stepref{h}, \stepref{i}.
	\end{proof}
\end{proof}
\step{2}{If $a = c$ and $b = d$ then $(a,b) = (c,d)$.}
\begin{proof}
	\pf\ First-order logic.
\end{proof}
\qed
\end{proof}

\begin{df}[Cartesian Product]
The \emph{Cartesian product} of classes $\mathbf{A}$ and $\mathbf{B}$ is the class $\mathbf{A} \times \mathbf{B} := \{(x,y) \mid x \in \mathbf{A}, y \in \mathbf{B}\}$.
\end{df}

\begin{prop}[S without Foundation]
For any sets $A$ and $B$, the class $A \times B$ is a set.
\end{prop}

\begin{proof}
\pf\ It is a subset of $\mathcal{P} \mathcal{P} (A \cup B)$. \qed
\end{proof}

\begin{props}[S without Foundation]
For any classes $\mathbf{A}$, $\mathbf{B}$ and $\mathbf{C}$, the following is a theorem:

\[ \mathbf{A} \times (\mathbf{B} \cup \mathbf{C}) = (\mathbf{A} \times \mathbf{B}) \cup (\mathbf{A} \times \mathbf{C}) \]
\end{props}

\begin{proof}
\pf
\begin{align*}
(x,y) \in \mathbf{A} \times (\mathbf{B} \cup \mathbf{C}) & \Leftrightarrow x \in \mathbf{A} \wedge (y \in \mathbf{B} \vee y \in \mathbf{C}) \\
& \Leftrightarrow (x \in \mathbf{A} \wedge y \in \mathbf{B}) \vee (x \in \mathbf{A} \wedge y \in \mathbf{C}) \\
& \Leftrightarrow (x,y) \in (\mathbf{A} \times \mathbf{B}) \cup (\mathbf{A} \times \mathbf{C}) & \qed
\end{align*}
\end{proof}

\begin{props}[S without Foundation]
For any classes $\mathbf{A}$ and $\mathbf{B}$, the following is a theorem:

If $\mathbf{A} \times \mathbf{B} = \mathbf{A} \times \mathbf{C}$ and $\mathbf{A}$ is nonempty then $\mathbf{B} = \mathbf{C}$.
\end{props}

\begin{proof}
\pf
\step{1}{\pick\ $a \in \mathbf{A}$}
\step{2}{For all $x$ we have $x \in \mathbf{B}$ iff $x \in \mathbf{C}$.}
\begin{proof}
	\pf
	\begin{align*}
		x \in \mathbf{B} & \Leftrightarrow (a,x) \in \mathbf{A} \times \mathbf{B} \\
		& \Leftrightarrow (a,x) \in \mathbf{A} \times \mathbf{C} \\
		& \Leftrightarrow x \in \mathbf{C}
	\end{align*}
\end{proof}
\qed
\end{proof}

\begin{props}[S without Foundation]
For any classes $\mathbf{A}$ and $\mathbf{B}$, the following is a theorem:

\[ \mathbf{A} \times \bigcup \mathbf{B} = \{ (a,b) \mid \exists Y \in \mathbf{B}. (a \in \mathbf{A} \wedge b \in Y) \} \]
\end{props}

\begin{proof}
\pf
\begin{align*}
(x,y) \in A \times \bigcup \mathbf{B} & \Leftrightarrow x \in A \wedge \exists Y \in \mathbf{B}. y \in Y \\
& \Leftrightarrow \exists Y \in \mathbf{B} (x \in A \wedge y \in Y) & \qed
\end{align*}
\end{proof}

\section{Relations}

\begin{df}[Relation]
A \emph{relation} is a class of ordered pairs.

A \emph{relation} $\mathbf{R}$ between classes $\mathbf{A}$ and $\mathbf{B}$ is a subclass of $\mathbf{A} \times \mathbf{B}$.

A \emph{(binary) relation on $\mathbf{A}$} is a relation between $\mathbf{A}$ and $\mathbf{A}$.

We write $x \mathbf{R} y$ for $(x,y) \in \mathbf{R}$.
\end{df}

\begin{df}[Domain]
The \emph{domain} of a class $\mathbf{R}$ is the class
\[ \dom \mathbf{R} := \{ x \mid \exists y. x \mathbf{R} y \} \enspace . \]
\end{df}

\begin{prop}[S without Foundation]
The domain of a set is a set.
\end{prop}

\begin{df}[Range]
The \emph{range} of a class $\mathbf{R}$ is the class
\[ \ran \mathbf{R} := \{ y \mid \exists x. x \mathbf{R} y \} \]
\end{df}

\begin{prop}[S without Foundation]
The range of a set is a set.
\end{prop}

\subsection{Identity Functions}

\begin{df}[Identity Function]
For any class $\mathbf{A}$, the \emph{identity function} or \emph{diagonal relation} $\mathrm{id}_\mathbf{A}$ on $\mathbf{A}$ is
\[ \mathrm{id}_\mathbf{A} := \{(x,x) \mid x \in \mathbf{A} \} \enspace . \]
\end{df}

\subsection{Inverses}

\begin{df}[Inverse]
The \emph{inverse} of a relation $\mathbf{R}$ between $\mathbf{A}$ and $\mathbf{B}$ is the relation $\mathbf{R}^{-1}$ between $\mathbf{B}$ and $\mathbf{A}$ defined by
\[ b \mathbf{R}^{-1} a \Leftrightarrow a \mathbf{R} b \enspace . \]
\end{df}

\begin{props}[S without Foundation]
For any classes $\mathbf{A}$, $\mathbf{B}$ and $\mathbf{R}$, the following is a theorem:

If $\mathbf{R}$ is a relation between $\mathbf{A}$ and $\mathbf{B}$ then $(\mathbf{R}^{-1})^{-1} = \mathbf{R}$.
\end{props}

\begin{proof}
\pf
\begin{align*}
x (\mathbf{R}^{-1})^{-1} y & \Leftrightarrow y \mathbf{R}^{-1} x \\
& \Leftrightarrow x \mathbf{R} y & \qed
\end{align*}
\end{proof}

\subsection{Composition}

\begin{df}[Composition]
Let $\mathbf{R}$ be a relation between $\mathbf{A}$ and $\mathbf{B}$, and $\mathbf{S}$ be a relation between $\mathbf{B}$ and $\mathbf{C}$. The \emph{composition} $\mathbf{S} \circ \mathbf{R}$ is the relation between $\mathbf{A}$ and $\mathbf{C}$ defined by
\[ a (\mathbf{S} \circ \mathbf{R}) c \Leftrightarrow \exists b (a \mathbf{R} b \wedge b \mathbf{S} c) \enspace. \]
\end{df}

\begin{props}[S without Foundation]
For any classes $\mathbf{A}$, $\mathbf{B}$, $\mathbf{C}$, $\mathbf{R}$ and $\mathbf{S}$, the following is a theorem:

If $\mathbf{R}$ is a relation between $\mathbf{A}$ and $\mathbf{B}$, and $\mathbf{S}$ is a relation between $\mathbf{B}$ and $\mathbf{C}$, then
\[ (\mathbf{S} \circ \mathbf{R})^{-1} = \mathbf{R}^{-1} \circ \mathbf{S}^{-1} \enspace . \]
\end{props}

\begin{proof}
\pf
\begin{align*}
z(\mathbf{S} \circ \mathbf{R})^{-1}x & \Leftrightarrow x(\mathbf{S} \circ \mathbf{R})z \\
& \Leftrightarrow \exists y. (x \mathbf{R} y \wedge y \mathbf{S}z) \\
& \Leftrightarrow \exists y. (y \mathbf{R}^{-1} x \wedge z \mathbf{S}^{-1} y) \\
& \Leftrightarrow z (\mathbf{R}^{-1} \circ \mathbf{S}^{-1}) x & \qed
\end{align*}
\end{proof}

\subsection{Properties of Relaitons}

\begin{df}[Reflexive]
Let $\mathbf{R}$ be a binary relation on $\mathbf{A}$. Then $\mathbf{R}$ is \emph{reflexive} on $\mathbf{A}$ iff $\forall x \in \mathbf{A}. (x,x) \in \mathbf{R}$.
\end{df}

\begin{props}[S without Foundation]
\label{prop:invref}
For any classes $\mathbf{A}$ and $\mathbf{R}$, the following is a theorem:

If $\mathbf{R}$ is a reflexive relation on $\mathbf{A}$ then so is $\mathbf{R}^{-1}$.
\end{props}

\begin{proof}
\pf
\step{1}{\pflet{$x \in \mathbf{A}$}}
\step{2}{$x \mathbf{R} x$}
\begin{proof}
	\pf\ Since $\mathbf{R}$ is reflexive.
\end{proof}
\step{3}{$x \mathbf{R}^{-1} x$}
\qed
\end{proof}

\begin{df}[Irreflexive]
A relation $\mathbf{R}$ is \emph{irreflexive} iff there is no $x$ such that $(x,x) \in \mathbf{R}$.
\end{df}

\begin{df}[Symmetric]
A relation $\mathbf{R}$ is \emph{symmetric} iff, whenever $x\mathbf{R}y$, then $y\mathbf{R}x$.
\end{df}

\begin{df}[Antisymmetric]
A relation $\mathbf{R}$ is \emph{antisymmetric} iff, whenever $x \mathbf{R} y$ and $y \mathbf{R} x$, then $x = y$.
\end{df}

\begin{props}[S without Foundation]
\label{prop:invantisym}
For any classes $\mathbf{A}$ and $\mathbf{R}$, the following is a theorem:

If $\mathbf{R}$ is an antisymmetric relation on $\mathbf{A}$ then so is $\mathbf{R}^{-1}$.
\end{props}

\begin{proof}
\pf
\step{1}{\assume{$x \mathbf{R}^{-1} y$ and $y \mathbf{R}^{-1} x$}}
\step{2}{$y \mathbf{R} x$ and $x \mathbf{R} y$}
\step{3}{$x = y$}
\begin{proof}
	\pf\ Since $\mathbf{R}$ is antisymmetric.
\end{proof}
\qed
\end{proof}

\begin{df}[Transitive]
A relation $\mathbf{R}$ is \emph{transitive} iff, whenever $x\mathbf{R}y$ and $y \mathbf{R} z$, then $x\mathbf{R} z$.
\end{df}

\begin{props}[S without Foundation]
\label{prop:invtrans}
For any classes $\mathbf{A}$, $\mathbf{B}$ and $\mathbf{R}$, the following is a theorem:

If $\mathbf{R}$ is a transitive relation between $\mathbf{A}$ and $\mathbf{B}$ then $\mathbf{R}^{-1}$ is transitive.
\end{props}

\begin{proof}
\pf
\step{1}{\assume{$(x,y), (y,z) \in \mathbf{R}^{-1}$}}
\step{2}{$(y,x),(z,y) \in \mathbf{R}$}
\step{3}{$(z,x) \in \mathbf{R}$}
\step{4}{$(x,z) \in \mathbf{R}^{-1}$}
\qed
\end{proof}

\begin{prop}[S without Foundation]
For any relation $R$ on a set $A$, there exists a smallest transitive relation on $A$ that includes $R$.
\end{prop}

\begin{proof}
\pf\ The relation is $\bigcap \{ S \in \mathcal{P} A^2 \mid R \subseteq S, S \text{ is transitive} \}$. \qed
\end{proof}

\begin{df}[Transitive Closure]
For any relation $R$ on a set $A$, the \emph{transitive closure} of $R$ is the smallest transitive relation that includes $R$.
\end{df}

\begin{df}[Minimal]
Let $\mathbf{R}$ be a relation on $\mathbf{A}$. An element $m \in \mathbf{A}$ is \emph{minimal} iff there is no $x \in \mathbf{A}$ such that $x \mathbf{R} m$.
\end{df}

\begin{df}[Maximal]
Let $\mathbf{R}$ be a relation on $\mathbf{A}$. An element $m \in \mathbf{A}$ is \emph{maximal} iff there is no $x \in \mathbf{A}$ such that $m \mathbf{R} x$.
\end{df}

\section{n-ary Relations}

\begin{dfs}
For any sets $a_1$, \ldots, $a_n$, define the \emph{ordered $n$-tuple} $(a_1, \ldots, a_n)$ by
\begin{align*}
(a_1) & := a_1 \\
(a_1, \ldots, a_n, a_{n+1}) & = ((a_1, \ldots, a_n), a_{n+1})
\end{align*}
\end{dfs}

\begin{dfs}
An \emph{$n$-ary relation on $\mathbf{A}$} is a class of ordered $n$-tuples all of whose components are in $\mathbf{A}$.
\end{dfs}

\section{Well Founded Relations}

\begin{df}[Well Founded]
A relation $\mathbf{R}$ is \emph{well founded} iff:
\begin{itemize}
\item Every nonempty set has an $\mathbf{R}$-minimal element.
\item For every set $x$, there exists a set $u$ such that $x \subseteq u$ and, for all $w$, $y$, if $y \in u$ and $w \mathbf{R} y$ then $w \in u$.
\end{itemize}
\end{df}

\begin{props}[S without Foundation]
\label{prop:segset}
For any class $\mathbf{R}$, the following is a theorem:

Assume $\mathbf{R}$ is a well founded relation. For any set $a$, the class $\{ x \mid x \mathbf{R} a\}$ is a set.
\end{props}

\begin{proof}
\pf
\step{1}{\pick\ a set $u$ such that $\{ a \} \subseteq u$ and, for all $w$, $y$, if $y \in u$ and $w \mathbf{R} y$ then $w \in u$.}
\step{2}{$\{ x \mid x \mathbf{R} a \} \subseteq u$}
\qed
\end{proof}

\begin{props}[S without Foundation]
\label{prop:leastelement}
For any classes $\mathbf{A}$, $\mathbf{B}$ and $\mathbf{R}$, the following is a theorem:

Assume $\mathbf{R}$ is a well founded relation on $\mathbf{A}$ and $\mathbf{B} \subseteq \mathbf{A}$ is nonempty. Then $\mathbf{B}$ has an $\mathbf{R}$-minimal element.
\end{props}

\begin{proof}
\pf
\step{1}{\pick\ $b \in \mathbf{B}$}
\step{2}{\pflet{$S = \{ x \in \mathbf{B} \mid x \mathbf{R} b \}$}}
\step{3}{\case{$S = \emptyset$}}
\begin{proof}
	\pf\ In this case $b$ is an $\mathbf{R}$-minimal element of $\mathbf{B}$.
\end{proof}
\step{4}{\case{$S \neq \emptyset$}}
\begin{proof}
	\pf\ In this cases $S$ has an $\mathbf{R}$-minimal element, which is an $\mathbf{R}$-minimal element of $\mathbf{B}$.
\end{proof}
\qed
\end{proof}

\begin{props}[FOL]
\label{prop:subwellfounded}
For any classes $\mathbf{R}$ and $\mathbf{S}$, the following is a theorem:

Assume $\mathbf{R}$ is a relation.
If $\mathbf{S}$ is a well-founded relation and $\mathbf{R} \subseteq \mathbf{S}$ then $\mathbf{R}$ is a well founded relation.
\end{props}

\begin{proof}
\pf\ Immediate from definitions.
\qed
\end{proof}

\begin{thms}[Transfinite Induction Principle (S without Foundation)]
For any classes $\mathbf{A}$, $\mathbf{B}$ and $\mathbf{R}$, the following is a theorem:

Assume $\mathbf{R}$ is a well founded relation on $\mathbf{A}$ and $\mathbf{B} \subseteq \mathbf{A}$. Assume that, for all $t \in \mathbf{A}$,
\[ \{ x \in \mathbf{A} \mid x \mathbf{R} t \} \subseteq \mathbf{B} \Rightarrow t \in \mathbf{B} \enspace . \]
Then $\mathbf{B} = \mathbf{A}$.
\end{thms}

\begin{proof}
\pf
\step{1}{\assume{for a contradiction $\mathbf{B} \neq \mathbf{A}$}}
\step{2}{\pick\ an $\mathbf{R}$-minimal element $m$ of $\mathbf{A} - \mathbf{B}$.}
\begin{proof}
	\pf\ Proposition \ref{prop:leastelement}.
\end{proof}
\step{3}{$\{ x \in \mathbf{A} \mid x \mathbf{R} m \} \subseteq \mathbf{B}$}
\begin{proof}
	\pf\ By minimality of $m$.
\end{proof}
\step{4}{$m \in \mathbf{B}$}
\qedstep
\begin{proof}
	\pf\ This is a contradiction.
\end{proof}
\qed
\end{proof}

\begin{thm}[S without Foundation]
The transitive closure of a well founded relation on a set is well founded.
\end{thm}

\begin{proof}
\pf
\step{1}{\pflet{$R$ be a well founded relation on the set $A$.}}
\step{2}{\pflet{$R^t$ be the transitive closure of $R$.}}
\step{3}{For any $x,y \in A$, if $x R^t y$ then there exists $z \in A$ such that $z R y$.}
\begin{proof}
	\pf\ $\{ (x,y) \in A^2 \mid \exists z \in A. zRy \}$ is a transitive relation on $A$ that includes $R$.
\end{proof}
\step{4}{\pflet{$B$ be a nonempty subset of $A$.}}
\step{5}{\pick\ an $R$-minimal element $b$ of $B$.}
\step{6}{$b$ is $R^t$-minimal in $B$.}
\begin{proof}
	\pf\ If there exists $x$ such that $x R^t b$ then there exists $z$ such that $z R b$ by \stepref{3}.
\end{proof}
\qed
\end{proof}

\begin{df}[Initial Segment]
Let $\mathbf{R}$ be a relation on $\mathbf{A}$ and $a \in \mathbf{A}$. The \emph{initial segment} up to $a$ is
\[ \seg a := \{ x \in \mathbf{A} \mid x \mathbf{R} a \} \enspace . \]
\end{df}

\chapter{Functions}

\section{Functions}

\begin{df}[Function]
A \emph{function} from $\mathbf{A}$ to $\mathbf{B}$ is a relation $\mathbf{F}$ between $\mathbf{A}$ and $\mathbf{B}$ such that, for all $x \in \mathbf{A}$, there is only one $y$ such that $x \mathbf{F} y$. We denote this $y$ by $\mathbf{F}(x)$.

A \emph{binary operation} on a class $\mathbf{A}$ is a function $\mathbf{A}^2 \rightarrow \mathbf{A}$.
\end{df}

\begin{df}[Closed]
Let $\mathbf{F} : \mathbf{A} \rightarrow \mathbf{A}$ be a function and $\mathbf{B} \subseteq \mathbf{A}$. Then $\mathbf{B}$ is \emph{closed} under $\mathbf{F}$ iff $\forall x \in \mathbf{B}. \mathbf{F}(x) \in \mathbf{B}$.
\end{df}

\begin{prop}[S without Foundation]
For any class $\mathbf{A}$, the following is a theorem:

\[ \mathrm{id}_\mathbf{A} : \mathbf{A} \rightarrow \mathbf{A} \]
\end{prop}

\begin{proof}
\pf\ For all $x \in \mathbf{A}$, the only $y$ such that $(x,y) \in \mathrm{id}_\mathbf{A}$ is $y = x$. \qed
\end{proof}

\begin{props}[S without Foundation]
For any classes $\mathbf{A}$, $\mathbf{B}$, $\mathbf{C}$, $\mathbf{F}$ and $\mathbf{G}$, the following is a theorem:

Assume $\mathbf{F} : \mathbf{A} \rightarrow \mathbf{B}$ and $\mathbf{G} : \mathbf{B} \rightarrow \mathbf{C}$. Then $\mathbf{G} \circ \mathbf{F} : \mathbf{A} \rightarrow \mathbf{C}$ and, for all $x \in \mathbf{A}$, we have
\[ (\mathbf{G} \circ \mathbf{F})(x) = \mathbf{G}(\mathbf{F}(x)) \enspace . \]
\end{props}

\begin{proof}
\pf
\step{1}{$\forall x \in \mathbf{A}. (x, \mathbf{G}(\mathbf{F}(x))) \in \mathbf{G} \circ \mathbf{F})$}
\begin{proof}
	\pf\ Because $(x, \mathbf{F}(x)) \in \mathbf{F}$ and $(\mathbf{F}(x), \mathbf{G}(\mathbf{F}(x))) \in \mathbf{G}$.
\end{proof}
\step{2}{If $(x,z) \in \mathbf{F} \circ \mathbf{G}$ then $z = \mathbf{G}(\mathbf{F}(x))$}
\begin{proof}
	\step{a}{\pick\ $y \in \mathbf{B}$ such that $x \mathbf{F} y$ and $y \mathbf{G} z$}
	\step{b}{$y = \mathbf{F}(x)$}
	\step{c}{$z = \mathbf{G}(y)$}
	\step{d}{$z = \mathbf{G}(\mathbf{F}(x))$}
\end{proof}
\qed
\end{proof}

\begin{prop}[SC without Foundation]
For any set $A$ there exists a function $F : \mathcal{P} A - \{ \emptyset \} \rightarrow A$ (a \emph{choice function} for $A$) such that, for every nonempty $B \subseteq A$, we have $F(B) \in B$.
\end{prop}

\begin{proof}
\pf
\step{1}{\pflet{$A$ be a set.}}
\step{2}{\pflet{$\mathcal{A} = \{ \{ B \} \times B \mid B \in \mathcal{P} A - \{ \emptyset \} \}$}}
\step{3}{Every member of $\mathcal{A}$ is nonempty.}
\step{4}{Any two distinct members of $\mathcal{A}$ are disjoint.}
\step{5}{\pick\ a set $C$ such that, for all $X \in \mathcal{A}$, we have $C \cap X$ is a singleton.}
\begin{proof}
	\pf\ Axiom of Choice.
\end{proof}
\step{6}{\pflet{$F = C \cap \bigcup \mathcal{A}$}}
\step{7}{$F : \mathcal{P} A - \{ \emptyset \} \rightarrow A$}
\begin{proof}
	\step{a}{$F$ is a function.}
	\begin{proof}
		\step{i}{\pflet{$(B,b),(B,b') \in F$}}
		\step{ii}{$(B,b),(B,b') \in \{B\} \times B$}
		\begin{proof}
			\pf\ Since $(B,b),(B,b') \in \bigcup \mathcal{A}$.
		\end{proof}
		\step{iii}{$(B,b),(B,b') \in C \cap (\{B\} \times B)$}
		\step{iv}{$(B,b) = (B,b')$}
		\begin{proof}
			\pf\ From \stepref{5}.
		\end{proof}
		\step{v}{$b = b'$}
	\end{proof}
	\step{b}{$\dom F = \mathcal{P} A - \{ \emptyset \}$}
	\begin{proof}
		\pf
		\begin{align*}
			& B \in \dom F \\
			\Leftrightarrow & \exists b. (B,b) \in F \\
			\Leftrightarrow & \exists b. ((B,b) \in \bigcup \mathcal{A} \wedge (B,b) \in C) \\
			\Leftrightarrow & \exists b. \exists B' \in \mathcal{P} A - \{ \emptyset \}. ((B,b) \in \{ B' \} \times B' \wedge (B,b) \in C) \\
			\Leftrightarrow & B \in \mathcal{P} A - \{ \emptyset \} \wedge \exists b \in B. (B,b) \in C \\
			\Leftrightarrow & B \in \mathcal{P} A - \{ \emptyset \} & (\text{\stepref{5}})
		\end{align*}
	\end{proof}
	\step{c}{$\ran F \subseteq A$}
\end{proof}
\step{8}{For every nonempty $B \subseteq A$ we have $F(B) \in B$}
\qed
\end{proof}

\begin{prop}[SC without Foundation]
\label{prop:AxChoice}
For any relation $R$ between $A$ and $B$, there exists a function $H : A \rightarrow B$ such that $H \subseteq R$ (i.e. $\forall x \in A. xRH(x)$).
\end{prop}

\begin{proof}
\pf
\step{1}{\pflet{$R$ be a relation between $A$ and $B$.}}
\step{2}{\pick\ a choice function $G$ for $B$.}
\step{3}{Define $H : A \rightarrow B$ by $H(x) = G(\{y \mid xRy\})$}
\step{4}{$H \subseteq R$}
\qed
\end{proof}

\subsection{Injective Functions}

\begin{df}[Injective]
A function $\mathbf{F} : \mathbf{A} \rightarrow \mathbf{B}$ is \emph{one-to-one}, \emph{injective} or an \emph{injection}, $\mathbf{F} : \mathbf{A} \rightarrowtail \mathbf{B}$, iff, for all $x,y \in \mathbf{A}$, if $\mathbf{F}(x) = \mathbf{F}(y)$, then $x = y$.
\end{df}

\begin{prop}[S without Foundation]
\label{prop:idinj}
For any class $\mathbf{A}$, the following is a theorem:

$\id{\mathbf{A}} : \mathbf{A} \rightarrow \mathbf{A}$ is injective.
\end{prop}

\begin{proof}
\pf\ If $\id{\mathbf{A}}(x) = \id{\mathbf{A}}(y)$ then immediately $x = y$. \qed
\end{proof}

\begin{props}[S without Foundation]
\label{prop:compinj}
For any classes $\mathbf{A}$, $\mathbf{B}$, $\mathbf{C}$, $\mathbf{F}$, $\mathbf{G}$, the following is a theorem:

Assume $\mathbf{F} : \mathbf{A} \rightarrowtail \mathbf{B}$ and $\mathbf{G} : \mathbf{B} \rightarrowtail \mathbf{C}$. Then $\mathbf{G} \circ \mathbf{F} : \mathbf{A} \rightarrowtail \mathbf{C}$.
\end{props}

\begin{proof}
\pf
\step{1}{\pflet{$x,y \in \mathbf{A}$}}
\step{2}{\assume{$(\mathbf{G} \circ \mathbf{F})(x) = (\mathbf{G} \circ \mathbf{F})(y)$}}
\step{3}{$\mathbf{G}(\mathbf{F}(x)) = \mathbf{G}(\mathbf{F}(y))$}
\step{4}{$\mathbf{F}(x) = \mathbf{F}(y)$}
\begin{proof}
	\pf\ Since $\mathbf{G}$ is injective.
\end{proof}
\step{5}{$x = y$}
\begin{proof}
	\pf\ Since $\mathbf{F}$ is injective.
\end{proof}
\qed
\end{proof}

\begin{prop}[S without Foundation]
Let $F : A \rightarrow B$ where $A$ is nonempty. There exists $G : B \rightarrow A$ (a \emph{left inverse}) such that $G \circ F = \id{A}$ if and only if $F$ is one-to-one.
\end{prop}

\begin{proof}
\pf
\step{1}{If there exists $G : B \rightarrow A$ such that $G \circ F = \id{A}$ then $F$ is one-to-one.}
\begin{proof}
	\step{a}{\assume{$G : B \rightarrow A$ and $G \circ F = I_A$}}
	\step{b}{\pflet{$x,y \in A$}}
	\step{c}{\assume{$F(x) = F(y)$}}
	\step{d}{$x = y$}
	\begin{proof}
		\pf\ $x = G(F(x)) = G(F(y)) = y$
	\end{proof}
\end{proof}
\step{2}{If $F$ is one-to-one then there exists $G : B \rightarrow A$ such that $G \circ F = I_A$.}
\begin{proof}
	\step{a}{\assume{$F$ is one-to-one.}}
	\step{b}{\pick\ $a \in A$}
	\step{c}{\pflet{$G : B \rightarrow A$ be the function defined by: $G(b)$ is the (unique) $x \in A$ such that $F(x) = b$ if there exists such an $x$, $G(b) = a$ otherwise.}}
	\step{d}{For all $x \in A$ we have $G(F(x)) = x$.}
\end{proof}
\qed
\end{proof}

\subsection{Surjective Functions}

\begin{df}[Surjective]
Let $F : A \rightarrow B$. We say that $F$ is \emph{surjective}, or maps $A$ \emph{onto} $B$, and write $F : A \twoheadrightarrow B$, iff for all $y \in B$ there exists $x \in A$ such that $F(x) = y$.
\end{df}

\begin{props}[S without Foundation]
\label{prop:idsurj}
For any class $\mathbf{A}$, the following is a theorem:

$\id{\mathbf{A}} : \mathbf{A} \rightarrow \mathbf{A}$ is surjective.
\end{props}

\begin{proof}
\pf\ For any $y \in \mathbf{A}$ we have $\id{\mathbf{A}}(y) = y$. \qed
\end{proof}

\begin{props}[S without Foundation]
\label{prop:compsurj}
For any classes $\mathbf{A}$, $\mathbf{B}$, $\mathbf{C}$, $\mathbf{F}$ and $\mathbf{G}$, the following is a theorem:

If $\mathbf{F} : \mathbf{A} \twoheadrightarrow \mathbf{B}$ and $\mathbf{G} : \mathbf{B} \twoheadrightarrow \mathbf{C}$, then $\mathbf{G} \circ \mathbf{F} : \mathbf{A} \twoheadrightarrow \mathbf{C}$.
\end{props}

\begin{proof}
\pf
\step{1}{\pflet{$c \in \mathbf{C}$}}
\step{2}{\pick\ $b \in \mathbf{B}$ such that $\mathbf{G}(b) = c$.}
\step{3}{\pick\ $a \in \mathbf{A}$ such that $\mathbf{F}(a) = b$.}
\step{4}{$(\mathbf{G} \circ \mathbf{F})(a) = c$}
\qed
\end{proof}

\begin{prop}[SC without Foundation]
Let $F : A \rightarrow B$. There exists $H : B \rightarrow A$ (a \emph{right inverse}) such that $F \circ H = \id{B}$ if and only if $F$ maps $A$ onto $B$.
\end{prop}

\begin{proof}
\pf
\step{1}{If $F$ has a right inverse then $F$ is surjective.}
\begin{proof}
	\step{a}{\assume{$F$ has a right inverse $H : B \rightarrow A$.}}
	\step{b}{\pflet{$y \in B$}}
	\step{c}{$F(H(y)) = y$}
	\step{c}{There exists $x \in A$ such that $F(x) = y$}
\end{proof}
\step{2}{If $F$ is surjective then $F$ has a right inverse.}
\begin{proof}
	\step{a}{\assume{$F$ is surjective.}}
	\step{b}{\pick\ a function $H : B \rightarrow A$ such that $H \subseteq F^{-1}$}
	\begin{proof}
		\pf\ Proposition \ref{prop:AxChoice}.
	\end{proof}
	\step{d}{$F \circ H = \id{B}$}
	\begin{proof}
		\step{i}{\pflet{$y \in B$}}
		\step{ii}{$(y,H(y)) \in F^{-1}$}
		\step{iii}{$F(H(y)) = y$}
	\end{proof}
\end{proof}
\qed
\end{proof}

\subsection{Bijections}

\begin{df}[Bijection]
Let $\mathbf{F} : \mathbf{A} \rightarrow \mathbf{B}$. Then $\mathbf{F}$ is \emph{bijective} or a \emph{bijection}, $\mathbf{F} : \mathbf{A} \approx \mathbf{B}$, iff it is injective and surjective.
\end{df}

\begin{props}[S without Foundation]
\label{prop:idbij}
For any class $\mathbf{A}$, the following is a theorem:

The identity function $\id{\mathbf{A}} : \mathbf{A} \approx \mathbf{A}$ is a bijection.
\end{props}

\begin{proof}
\pf\ Proposition \ref{prop:idinj} and \ref{prop:idsurj}. \qed	
\end{proof}

\begin{props}[S without Foundation]
\label{prop:invbij}
For any classes $\mathbf{A}$, $\mathbf{B}$ and $\mathbf{F}$, the following is a theorem:

If $\mathbf{F} : \mathbf{A} \approx \mathbf{B}$ then $\mathbf{F}^{-1} : \mathbf{B} \approx \mathbf{A}$.
\end{props}

\begin{proof}
\pf
\step{1}{$\mathbf{F}^{-1} : \mathbf{B} \rightarrow \mathbf{A}$}
\begin{proof}
	\step{a}{\pflet{$b \in \mathbf{B}$}}
	\step{b}{\pick\ $a \in \mathbf{A}$ such that $\mathbf{F}(a) = b$.}
	\begin{proof}
		\pf\ Since $\mathbf{F}$ is surjective.
	\end{proof}
	\step{c}{$(b,a) \in \mathbf{F}^{-1}$}
	\step{d}{If $(b,a') \in \mathbf{F}^{-1}$ then $a' = a$.}
	\begin{proof}
		\step{i}{\pflet{$a' \in \mathbf{A}$ such that $(b,a') \in \mathbf{F}^{-1}$}}
		\step{ii}{$\mathbf{F}(a') = \mathbf{F}(a)$}
		\step{iii}{$a' = a$}
		\begin{proof}
			\pf\ Since $\mathbf{F}$ is injective.
		\end{proof}
	\end{proof}
\end{proof}
\step{2}{$\mathbf{F}^{-1}$ is injective.}
\begin{proof}
	\step{a}{\pflet{$x,y \in \mathbf{B}$}}
	\step{b}{\assume{$\mathbf{F}^{-1}(x) = \mathbf{F}^{-1}(y)$}}
	\step{c}{$x = y$}
	\begin{proof}
		\pf\ $x = \mathbf{F}(\mathbf{F}^{-1}(x)) = \mathbf{F}(\mathbf{F}^{-1}(y)) = y$.
	\end{proof}
\end{proof}
\step{3}{$\mathbf{F}^{-1}$ is surjective.}
\begin{proof}
	\pf\ For all $a \in \mathbf{A}$ we have $\mathbf{F}^{-1}(\mathbf{F}(a)) = a$.
\end{proof}
\qed
\end{proof}

\begin{props}[S without Foundation]
\label{prop:compbij}
For any classes $\mathbf{A}$, $\mathbf{B}$, $\mathbf{C}$, $\mathbf{F}$ and $\mathbf{G}$, the following is a theorem:

If $\mathbf{F} : \mathbf{A} \approx \mathbf{B}$ and $\mathbf{G} : \mathbf{B} \approx \mathbf{C}$ then $\mathbf{G} \circ \mathbf{F} : \mathbf{A} \approx \mathbf{C}$.
\end{props}

\begin{proof}
\pf\ Propositions \ref{prop:compinj} and \ref{prop:compsurj}. \qed
\end{proof}

\subsection{Restrictions}

\begin{df}[Restriction]
Let $\mathbf{F} : \mathbf{A} \rightarrow \mathbf{B}$. Let $\mathbf{C} \subseteq \mathbf{A}$. The \emph{restriction} of $\mathbf{F}$ to $\mathbf{C}$, denoted $\mathbf{F} \restriction \mathbf{C}$, is the function
\begin{align*}
\mathbf{F} \restriction \mathbf{C} & : \mathbf{C} \rightarrow \mathbf{B} \\
(\mathbf{F} \restriction \mathbf{C})(x) & = \mathbf{F}(x) & (x \in \mathbf{C})
\end{align*}
\end{df}

\subsection{Images}

\begin{df}[Image]
Let $\mathbf{F} : \mathbf{A} \rightarrow \mathbf{B}$ and $\mathbf{C} \subseteq \mathbf{A}$. The \emph{image} of $\mathbf{C}$ under $\mathbf{F}$ is the class
\[ \mathbf{F}(\mathbf{C}) := \{ \mathbf{F}(x) \mid x \in \mathbf{C} \} \enspace . \]
\end{df}

\begin{props}[S without Foundation]
For any classes $\mathbf{F}$, $\mathbf{A}$ and $\mathbf{B}$, the following is a theorem.

If $\mathbf{F} : \mathbf{A} \rightarrow \mathbf{B}$, then for any subset $S \subseteq \mathbf{A}$, the class $\mathbf{F}(S)$ is a set.
\end{props}

\begin{proof}
\pf\ By an Axiom of Replacement. \qed
\end{proof}

\begin{props}[S without Foundation]
\label{prop:imgunion}
For any classes $\mathbf{A}$, $\mathbf{B}$, $\mathbf{C}$ and $\mathbf{F}$, the following is a theorem:

Assume $\mathbf{F} : \mathbf{A} \rightarrow \mathbf{B}$ and $\mathbf{C} \subseteq \mathcal{P} \mathbf{A}$. Then
\[ \mathbf{F} \left( \bigcup \mathbf{C} \right) = \{ y \mid \exists X \in \mathbf{C}. y \in \mathbf{F}(X) \} \]
\end{props}

\begin{proof}
\pf
\begin{align*}
	y \in \mathbf{F} \left( \bigcup \mathbf{C} \right) & \Leftrightarrow \exists x \in \bigcup \mathbf{C}. y = \mathbf{F}(x) \\
	& \Leftrightarrow \exists x. \exists X. X \in \mathbf{C} \wedge x \in X \wedge y = \mathbf{F}(x) \\
	& \Leftrightarrow \exists X \in \mathbf{C}. y \in \mathbf{F}(X) & \qed
\end{align*}
\end{proof}

\begin{props}[S without Foundation]
For any classes $\mathbf{A}$, $\mathbf{B}$, $\mathbf{C}$, $\mathbf{D}$ and $\mathbf{F}$, the following is a theorem:

Assume $\mathbf{F} : \mathbf{A} \rightarrow \mathbf{B}$ and $\mathbf{C}, \mathbf{D} \subseteq \mathbf{A}$. Then
\[ \mathbf{F}(\mathbf{C} \cup \mathbf{D}) = \mathbf{F}(\mathbf{C}) \cup \mathbf{F}(\mathbf{D}) \enspace . \]
\end{props}

\begin{proof}
\pf
\begin{align*}
	y \in \mathbf{F}(\mathbf{C} \cup \mathbf{D})
	& \Leftrightarrow \exists x \in \mathbf{C} \cup \mathbf{D}. y = \mathbf{F}(x) \\
	& \Leftrightarrow \exists x \in \mathbf{C}. y = \mathbf{F}(x) \vee \exists x \in \mathbf{D}. y = \mathbf{F}(x) \\
	& \Leftrightarrow y \in \mathbf{F}(\mathbf{C}) \cup \mathbf{F}(\mathbf{D}) & \qed
\end{align*}
\end{proof}

\begin{prop}[S without Foundation]
For any classes $\mathbf{F}$, $\mathbf{A}$, $\mathbf{B}$, $\mathbf{C}$ and $\mathbf{D}$, the following is a theorem:

Assume $\mathbf{F} : \mathbf{A} \rightarrow \mathbf{B}$ and $\mathbf{C}, \mathbf{D} \subseteq \mathbf{A}$. Then

\[ \mathbf{F}(\mathbf{A} \cap \mathbf{B}) \subseteq \mathbf{F}(\mathbf{A}) \cap \mathbf{F}(\mathbf{B}) \enspace . \]

Equality holds if $\mathbf{F}$ is injective.
\end{prop}

\begin{proof}
\pf
\step{1}{$\mathbf{F}(\mathbf{A} \cap \mathbf{B}) \subseteq \mathbf{F}(\mathbf{A}) \cap \mathbf{F}(\mathbf{B})$}
\begin{proof}
	\step{a}{\pflet{$y \in \mathbf{F}(\mathbf{A} \cap \mathbf{B})$}}
	\step{b}{\pick\ $x \in \mathbf{A} \cap \mathbf{B}$ such that $y = \mathbf{F}(x)$}
	\step{c}{$y \in \mathbf{F}(\mathbf{A})$}
	\begin{proof}
		\pf\ Since $x \in \mathbf{A}$.
	\end{proof}
	\step{d}{$y \in \mathbf{F}(\mathbf{B})$}
	\begin{proof}
		\pf\ Since $x \in \mathbf{B}$.
	\end{proof}
\end{proof}
\step{2}{If $\mathbf{F}$ is injective then $\mathbf{F}(\mathbf{A} \cap \mathbf{B}) = \mathbf{F}(\mathbf{A}) \cap \mathbf{F}(\mathbf{B})$.}
\begin{proof}
	\step{a}{\assume{$\mathbf{F}$ is injective.}}
	\step{b}{\pflet{$y \in \mathbf{F}(\mathbf{A}) \cap \mathbf{F}(\mathbf{B})$}}
	\step{c}{\pick\ $x \in \mathbf{A}$ such that $y = \mathbf{F}(x)$}
	\step{d}{\pick\ $x' \in \mathbf{B}$ such that $y = \mathbf{F}(x')$}
	\step{e}{$x = x'$}
	\begin{proof}
		\pf\ \stepref{a}
	\end{proof}
	\step{f}{$x \in \mathbf{A} \cap \mathbf{B}$}
	\step{g}{$y \in \mathbf{F}(\mathbf{A} \cap \mathbf{B})$}
\end{proof}
\qed
\end{proof}

\begin{props}[S without Foundation]
\label{prop:imgint}
For any classes $\mathbf{F}$, $\mathbf{A}$, $\mathbf{B}$, and $\mathbf{C}$, the following is a theorem:

Let $\mathbf{F} : \mathbf{A} \rightarrow \mathbf{B}$ and $\mathbf{C} \subseteq \mathcal{P} \mathbf{A}$. Then

\[ \mathbf{F} \left( \bigcap \mathbf{C} \right) \subseteq \bigcap \{ \mathbf{F}(X) \mid X \in \mathbf{A} \} \enspace . \]

Equality holds if $\mathbf{F}$ is injective and $\mathbf{A}$ is nonempty.
\end{props}

\begin{proof}
\pf
\step{1}{$\mathbf{F} \left( \bigcap \mathbf{A} \right) \subseteq \bigcap \{ \mathbf{F}(X) \mid X \in \mathbf{A} \}$}
\begin{proof}
	\step{a}{\pflet{$y \in \mathbf{F} (\bigcap \mathbf{A})$}}
	\step{b}{\pick\ $x \in \bigcap \mathbf{A}$ such that $y = \mathbf{F}(x)$}
	\step{c}{\pflet{$X \in \mathbf{A}$}}
	\step{d}{$x \in X$}
	\step{e}{$y \in \mathbf{F}(X)$}
\end{proof}
\step{2}{If $\mathbf{F}$ is injective then $\mathbf{F} \left( \bigcap \mathbf{A} \right) = \bigcap \{ \mathbf{F}(X) \mid X \in \mathbf{A} \}$}
\begin{proof}
	\step{a}{\assume{$\mathbf{F}$ is injective.}}
	\step{aa}{\assume{$\mathbf{A}$ is nonempty.}}
	\step{b}{\pflet{$y \in \bigcap \{ \mathbf{F}(X) \mid X \in \mathbf{A} \}$}}
	\step{c}{\pick\ $X_0 \in \mathbf{A}$}
	\step{d}{\pick\ $x \in X_0$ such that $(x,y) \in \mathbf{F}$}
	\step{e}{$x \in \bigcap \mathbf{A}$}
	\begin{proof}
		\step{i}{\pflet{$X \in \mathbf{A}$}}
		\step{ii}{\pick\ $x' \in X$ such that $(x',y) \in \mathbf{F}$.}
		\step{iii}{$x = x'$}
		\begin{proof}
			\pf\ \stepref{a}
		\end{proof}
		\step{iv}{$x \in X$}
	\end{proof}
	\step{f}{$y \in \mathbf{F}(\bigcap \mathbf{A})$}
\end{proof}
\qed
\end{proof}

\begin{prop}[S without Foundation]
\label{prop:imgdiff}
For any classes $\mathbf{A}$, $\mathbf{B}$, $\mathbf{C}$, $\mathbf{D}$ and $\mathbf{F}$, the following is a theorem:

Assume $\mathbf{F} : \mathbf{A} \rightarrow \mathbf{B}$ and $\mathbf{C}, \mathbf{D} \subseteq \mathbf{A}$. Then
\[ \mathbf{F}(\mathbf{C}) - \mathbf{F}(\mathbf{D}) \subseteq \mathbf{F}(\mathbf{C} - \mathbf{D}) \enspace . \]
Equality holds if $\mathbf{F}$ is injective.
\end{prop}

\begin{proof}
\pf
\step{1}{$\mathbf{F}(\mathbf{C}) - \mathbf{F}(\mathbf{D}) \subseteq \mathbf{F}(\mathbf{A} - \mathbf{B})$}
\begin{proof}
	\step{a}{\pflet{$y \in \mathbf{F}(\mathbf{A}) - \mathbf{F}(\mathbf{B})$}}
	\step{b}{\pick\ $x \in \mathbf{A}$ such that $y = \mathbf{F}(x)$}
	\step{c}{$x \notin \mathbf{B}$}
	\step{d}{$x \in \mathbf{A} - \mathbf{B}$}
	\step{e}{$y \in \mathbf{F}(\mathbf{A} - \mathbf{B})$}
\end{proof}
\step{2}{If $\mathbf{F}$ is injective then $\mathbf{F}(\mathbf{A}) - \mathbf{F}(\mathbf{B}) = \mathbf{F}(\mathbf{A} - \mathbf{B})$}
\begin{proof}
	\step{a}{\assume{$\mathbf{F}$ is injective.}}
	\step{b}{\pflet{$y \in \mathbf{F}(\mathbf{A} - \mathbf{B})$}}
	\step{c}{\pick\ $x \in \mathbf{A} - \mathbf{B}$ such that $y = \mathbf{F}(x)$}
	\step{d}{$y \in \mathbf{F}(\mathbf{A})$}
	\step{e}{$y \notin \mathbf{F}(\mathbf{B})$}
	\begin{proof}
		\step{i}{\assume{for a contradiction $y \in \mathbf{F}(\mathbf{B})$}}
		\step{ii}{\pick\ $x' \in \mathbf{B}$ such that $y = \mathbf{F}(x')$}
		\step{iii}{$x = x'$}
		\begin{proof}
			\pf\ \stepref{a}
		\end{proof}
		\step{iv}{$x \in \mathbf{B}$}
		\qedstep
		\begin{proof}
			\pf\ This contradicts \stepref{c}.
		\end{proof}
	\end{proof}
\end{proof}
\qed
\end{proof}

\subsection{Inverse Images}

\begin{df}[Inverse Image]
Let $\mathbf{F} : \mathbf{A} \rightarrow \mathbf{B}$ and $\mathbf{C} \subseteq \mathbf{B}$. Then the \emph{inverse image} of $\mathbf{C}$ under $\mathbf{F}$ is
\[ \mathbf{F}^{-1}(\mathbf{C}) = \{ x \in \mathbf{A} \mid \mathbf{F}(x) \in \mathbf{C} \} \enspace . \]
\end{df}

\begin{props}[S without Foundation]
For any classes $\mathbf{A}$, $\mathbf{B}$, $\mathbf{C}$ and $\mathbf{F}$, the following is a theorem:

Assume $\mathbf{F} : \mathbf{A} \rightarrow \mathbf{B}$ and $\mathbf{C} \subseteq \mathcal{P} \mathbf{B}$. Then
\[ \mathbf{F}^{-1} \left( \bigcap \mathbf{C} \right) = \bigcap \{ \mathbf{F}^{-1}(X) \mid X \in \mathbf{C} \} \enspace . \]
\end{props}

\begin{proof}
\pf
\begin{align*}
x \in \mathbf{F}^{-1} \left( \bigcap \mathbf{C} \right)
& \Leftrightarrow \mathbf{F}(x) \in \bigcap \mathbf{C} \\
& \Leftrightarrow \forall X \in \mathbf{C}. \mathbf{F}(x) \in X \\
& \Leftrightarrow \forall X \in \mathbf{C}. x \in \mathbf{F}^{-1}(X) & \qed
\end{align*}
\end{proof}

\begin{props}[S without Foundation]
For any classes $\mathbf{A}$, $\mathbf{B}$, $\mathbf{C}$, $\mathbf{D}$ and $\mathbf{F}$, the following is a theorem:

Assume $\mathbf{F} : \mathbf{A} \rightarrow \mathbf{B}$ and $\mathbf{C}, \mathbf{D} \subseteq \mathbf{B}$. Then
\[ \mathbf{F}^{-1}(\mathbf{C} - \mathbf{D}) = \mathbf{F}^{-1}(\mathbf{C}) - \mathbf{F}^{-1}(\mathbf{D}) \enspace . \]
\end{props}

\begin{proof}
\pf
\begin{align*}
x \in \mathbf{F}^{-1}(\mathbf{C} - \mathbf{D}) & \Leftrightarrow \mathbf{F}(x) \in \mathbf{C} - \mathbf{D} \\
& \Leftrightarrow \mathbf{F}(x) \in \mathbf{C} \wedge \mathbf{F}(x) \notin \mathbf{D} \\
& \Leftrightarrow x \in \mathbf{F}^{-1}(\mathbf{C}) \wedge x \in \mathbf{F}^{-1}(\mathbf{D}) \\
& \Leftrightarrow x \in \mathbf{F}^{-1}(\mathbf{C}) - \mathbf{F}^{-1}(\mathbf{D}) & \qed
\end{align*}
\end{proof}

\subsection{Function Sets}

\begin{prop}[SF without Foundation]
For any classes $\mathbf{B}$ and $\mathbf{F}$, the following is a theorem:

Let $A$ be a set. If $\mathbf{F} : A \rightarrow \mathbf{B}$ then $\mathbf{F}$ is a set.
\end{prop}

\begin{proof}
\pf\ By an Axiom of Replacement, we have $R = \{ \mathbf{F}(x) \mid x \in A \}$ is a set. Hence $\mathbf{F}$ is a set since $\mathbf{F} \subseteq A \times R$. \qed
\end{proof}

\begin{df}[Dependent Product Class]
Let $I$ be a set and let $\mathbf{H}(i)$ be a class for all $i \in I$. We write $\prod_{i \in I} \mathbf{H}(i)$ for the class of all functions $f : I \rightarrow \bigcup_{i \in I} \mathbf{H}(i)$ such that $\forall i \in I. f(i) \in \mathbf{H}(i)$.

We write $\mathbf{B}^I$ for $\prod_{i \in I} \mathbf{B}$ where $\mathbf{B}$ does not depend on $I$.
\end{df}

\begin{props}[SF without Foundation]
Let $I$ be a set. Let $H(i)$ be a set for every $i \in I$. Then $\prod_{i \in I} H(i)$ is a set.
\end{props}

\begin{proof}
\pf
\step{1}{$\{ \mathbf{H}(i) \mid i \in I \}$ is a set.}
\begin{proof}
	\pf\ By an Axiom of Replacement.
\end{proof}
\step{2}{$\bigcup_{i \in I} \mathbf{H}(i)$ is a set.}
\step{3}{$\prod_{i \in I} \mathbf{H}(i)$ is a set.}
\begin{proof}
	\pf\ It is a subset of $\mathcal{P}\left( I \times \bigcup_{i \in I} \mathbf{H}(i) \right)$.
\end{proof}
\qed
\end{proof}

\begin{prop}[SFC without Foundation]
Let $I$ be a set. Let $H(i)$ be a set for all $i \in I$. If $\forall i \in I. H(i) \neq \emptyset$ then $\prod_{i \in I} H(i) \neq \emptyset.$
\end{prop}

\begin{proof}
\pf
\step{1}{\assume{$\forall i \in I. H(i) \neq \emptyset$}}
\step{2}{\pflet{$R = \{ (i,x) \mid i \in I, x \in H(i) \}$}}
\step{3}{\pick\ a function $f : I \rightarrow \bigcup_{i \in I} H(i)$ such that $f \subseteq R$}
\begin{proof}
	\pf\ Proposition \ref{prop:AxChoice}.
\end{proof}
\step{4}{$f \in \prod_{i \in I} H(i)$}
\qed
\end{proof}

\section{Monotone Functions}

\begin{df}[Monotone]
Let $\leq$ be a preorder on $\mathbf{A}$, and $\preccurlyeq$ a preorder on $\mathbf{B}$. Then $\mathbf{F} : \mathbf{A} \rightarrow \mathbf{B}$ is \emph{monotone} iff, whenever $x \leq y$, then $\mathbf{F}(x) \preccurlyeq \mathbf{F}(y)$.
\end{df}

\begin{prop}[S without Foundation]
\label{prop:monotone_fixed_point}
Any monotone function $\mathcal{P} X \rightarrow \mathcal{P} Y$ has a fixed point.
\end{prop}

\begin{proof}
\pf
\step{1}{\pflet{$h : \mathcal{P} X \rightarrow \mathcal{P} Y$ be monotone.}}
\step{2}{\pflet{$T_0 = \{ Y \in \mathcal{P} X \mid Y \subseteq h(Y) \}$}}
\step{3}{\pflet{$T = \bigcup T_0$}}
\step{4}{$T \subseteq h(T)$}
\begin{proof}
	\step{a}{For all $Y \in T_0$ we have $Y \subseteq h(T)$}
	\begin{proof}
		\step{a}{\pflet{$Y \in T_0$}}
		\step{c}{$Y \subseteq h(Y)$}
		\begin{proof}
			\pf\ \stepref{2}, \stepref{a}
		\end{proof}
		\step{d}{$Y \subseteq T$}
		\begin{proof}
			\pf\ \stepref{3}, \stepref{a}
		\end{proof}
		\step{f}{$h(Y) \subseteq h(T)$}
		\begin{proof}
			\pf\ \stepref{1}, \stepref{d}
		\end{proof}
		\step{g}{$Y \subseteq h(T)$}
		\begin{proof}
			\pf\ \stepref{c}, \stepref{f}
		\end{proof}
	\end{proof}
	\step{b}{$T \subseteq h(T)$}
\end{proof}
\step{5}{$h(T) \in T_0$}
\step{6}{$h(T) \subseteq T$}
\step{7}{$h(T) = T$}
\begin{proof}
	\pf\ \stepref{4}, \stepref{6}
\end{proof}
\qed
\end{proof}

\section{Equinumerosity}

\begin{df}[Equinumerous]
Sets $A$ and $B$ are \emph{equinumerous}, $A \approx B$, iff there exists a bijection between $A$ and $B$.
\end{df}

\begin{prop}[S without Foundation]
Let $2$ be any set of the form $\{a,b\}$ where $a \neq b$.
For any set $A$ we have $\mathcal{P} A \approx 2^A$.
\end{prop}

\begin{proof}
\pf\ The function $H : \mathcal{P} A \rightarrow 2^A$ defined by $H(S)(x) = a$ if $x \in S$ and $b$ if $x \notin S$ is a bijection. \qed
\end{proof}

\section{Domination}

\begin{df}[Dominate]
A set $A$ is \emph{dominated} by a set $B$, $A \preccurlyeq B$, iff there exists an injection $A \rightarrow B$.
\end{df}

\begin{prop}[S without Foundation]
\[ A \preccurlyeq A \]
\end{prop}

\begin{prop}[S without Foundation]
If $A \preccurlyeq B$ and $B \preccurlyeq C$ then $A \preccurlyeq C$.
\end{prop}

\begin{prop}[S without Foundation]
If $A \subseteq B$ then $A \preccurlyeq B$.
\end{prop}

\begin{prop}[S without Foundation]
If $A \approx B$ then $A \preccurlyeq B$.
\end{prop}

\begin{prop}[SC without Foundation]
Given sets $A$ and $B$, if $A \neq \emptyset$ or $B = \emptyset$, then we have $A \preccurlyeq B$ iff there exists a surjective function $B \rightarrow A$.
\end{prop}

\begin{proof}
\pf
\step{1}{If $A \preccurlyeq B$ and $A \neq \emptyset$ then there exists a surjective function $B \rightarrow A$.}
\begin{proof}
	\step{a}{\assume{$f : A \rightarrow B$ be injective.}}
	\step{b}{\pick\ $a \in A$}
	\step{c}{\pflet{$g : B \rightarrow A$ be the function defined by $g(b) = f^{-1}(b)$ if $b \in \ran f$, and $g(b) = a$ otherwise.}}
	\step{d}{$g$ is surjective.}
\end{proof}
\step{2}{If there exists a surjective function $B \rightarrow A$ then $A \preccurlyeq B$.}
\begin{proof}
	\step{a}{\assume{there exists a surjective function $g : B \rightarrow A$}}
	\step{x}{$\forall a \in A. \exists b \in B. g(b) = a$}
	\step{b}{Choose a function $f : A \rightarrow B$ such that $\forall a \in A. g(f(a)) = a$}
	\step{c}{$f$ is injective.}
\end{proof}
\qed
\end{proof}

\begin{thm}[Schr\"{o}der-Bernstein (S without Foundation)]
If $A \preccurlyeq B$ and $B \preccurlyeq A$ then $A \approx B$.
\end{thm}

\begin{proof}
\pf
\step{1}{\pflet{$f : A \rightarrowtail B$ and $g : B \rightarrowtail A$ be injective.}}
\step{2}{\pflet{$h : \mathcal{P} X \rightarrow \mathcal{P} X$ be the function $h(Z) =  X - g(Y - f(Z))$.}}
\step{3}{$h$ is monotone.}
\step{4}{\pick\ $T \subseteq X$ such that $h(T) = T$.}
\begin{proof}
	\pf\ Proposition \ref{prop:monotone_fixed_point}.
\end{proof}
\step{5}{\pflet{$k : A \rightarrow B$ be the function
\[ k(x) = \begin{cases}
f(x) & \text{if } x \in T \\
g^{-1}(x) & \text{if } x \notin T
\end{cases} \]}}
\begin{proof}
	\pf\ If $x \notin T = X - g(Y - f(T))$ then $x \in g(Y - f(T))$ and so $x \in \ran g$.
\end{proof}
\step{6}{$k$ is injective.}
\begin{proof}
	\step{a}{\pflet{$x,y \in A$}}
	\step{b}{\assume{$k(x) = k(y)$}}
	\step{c}{\case{$x,y \in T$}}
	\begin{proof}
		\pf\ Then $f(x) = f(y)$ so $x = y$ because $f$ is injective.
	\end{proof}
	\step{d}{\case{$x \in T$ and $y \notin T$}}
	\begin{proof}
		\step{i}{$f(x) = g^{-1}(y)$}
		\step{ii}{$y = g(f(x))$}
		\step{iii}{$y \in g(Y - f(T))$}
		\step{iv}{$f(x) \in Y - f(T)$}
		\qedstep
		\begin{proof}
			\pf\ This is a contradiction.
		\end{proof}
	\end{proof}
	\step{e}{\case{$x \notin T$ and $y \in T$}}
	\begin{proof}
		\pf\ Similar.
	\end{proof}
	\step{f}{\case{$x,y \notin T$}}
	\begin{proof}
		\pf\ Then $x = g(k(x)) = g(k(y)) = y$.
	\end{proof}
\end{proof}
\step{7}{$k$ is surjective.}
\begin{proof}
	\step{a}{\pflet{$y \in Y$}}
	\step{b}{\case{$y \in f(T)$}}
	\begin{proof}
		\pf\ Pick $x \in T$ such that $y = f(x)$. Then $y = k(x)$.
	\end{proof}
	\step{c}{\case{$y \notin f(T)$}}
	\begin{proof}
		\pf\ $y = k(g(y))$.
	\end{proof}
\end{proof}
\qed
\end{proof}
\section{Transfinite Recursion}

\begin{thms}[Transfinite Recursion Theorem Schema (SF without Foundation)]
For any classes $\mathbf{A}$, $\mathbf{R}$ and any property $G[x,y,z]$, there exists a class $\mathbf{F}$ such that, for any class $\mathbf{F}'$ the following is a theorem:

Assume that $\mathbf{R}$ is a well-founded relation on $\mathbf{A}$. Assume that, for any $f$ and $t$, there exists a unique $z$ such that $G[f,t,z]$. Then $\mathbf{F} : \mathbf{A} \rightarrow \mathbf{V}$ such that, for all $t \in \mathbf{A}$, we have $\mathbf{F} \restriction \seg t$ is a set and
\[ G[\mathbf{F} \restriction \seg t, t, \mathbf{F}(t)] \enspace . \]

If $\mathbf{F}' : \mathbf{A} \rightarrow \mathbf{V}$ satisfies that, for all $t \in \mathbf{A}$, we have $\mathbf{F}' \restriction \seg t$ is a set and $G[\mathbf{F}' \restriction \seg t, t, \mathbf{F}'(t)]$, then $\mathbf{F}' = \mathbf{F}$.
\end{thms}

\begin{proof}
\pf
\step{1}{For $B$ a subset of $\mathbf{A}$, let us say a function $v : B \rightarrow \mathbf{V}$ is \emph{acceptable} iff, for all $x \in B$, we have $\seg x \subseteq B$ and $G[v \restriction \seg x, x, v(x)]$}
\step{2}{\pflet{$\mathbf{K}$ be the class of all acceptable functions.}}
\step{3}{\pflet{$\mathbf{F} = \bigcup \mathbf{K}$}}
\step{4}{For all $B,C \subseteq \mathbf{A}$, given $v_1 : B \rightarrow \mathbf{V}$ and $v_2 : C \rightarrow \mathbf{V}$ acceptable and $x \in B \cap C$, we have $v_1(x) = v_2(x)$}
\begin{proof}
	\step{a}{\assume{as transfinite induction hypothesis $\forall y \mathbf{R} x. y \in B \cap C \Rightarrow v_1(y) = v_2(y)$}}
	\step{b}{$v_1 \restriction \seg x = v_2 \restriction \seg x$}
	\step{c}{$G[v_1 \restriction \seg x, x, v_1(x)]$}
	\step{d}{$G[v_2 \restriction \seg x, x, v_2(x)]$}
	\step{e}{$v_1(x) = v_2(x)$}
\end{proof}
\step{5}{$\mathbf{F}$ is a function.}
\begin{proof}
	\step{a}{\assume{$(x,y),(x,z) \in \mathbf{F}$}}
	\step{b}{\pick\ acceptable $v_1 : B \rightarrow \mathbf{V}$ and $v_2 : C \rightarrow \mathbf{V}$ such that $v_1(x) = y$ and $v_2(x) = z$}
	\step{c}{$y = z$}
	\begin{proof}
		\pf\ By \stepref{4}.
	\end{proof}
\end{proof}
\step{6}{For all $t \in \dom \mathbf{F}$, we have $\mathbf{F} \restriction \seg t$ is a set and $G[\mathbf{F} \restriction \seg t, t, \mathbf{F}(t)]$}
\begin{proof}
	\step{a}{\pflet{$t \in \dom \mathbf{F}$}}
	\step{b}{\pick\ an acceptable $v : A \rightarrow \mathbf{V}$ such that $t \in A$}
	\step{c}{For all $y \mathbf{R} x$ we have $v(y) = \mathbf{F}(y)$}
	\step{d}{$\mathbf{F} \restriction \seg x = v \restriction \seg x$}
	\step{e}{$G[v \restriction \seg x, x, v(x)]$}
	\step{f}{$G[\mathbf{F} \restriction \seg x, x, \mathbf{F}(x)]$}
\end{proof}
\step{7}{$\dom \mathbf{F} = \mathbf{A}$}
\begin{proof}
	\step{a}{\pflet{$x \in \mathbf{A}$}}
	\step{b}{\assume{as transfinite induction hypothesis $\forall y \mathbf{R} x. y \in \mathbf{A}$}}
	\step{c}{\assume{for a contradiction $x \notin \dom \mathbf{F}$}}
	\step{d}{$\mathbf{F} \restriction \seg x$ is a set}
	\begin{proof}
		\pf\ Axiom of Replacement, Proposition \ref{prop:segset}.
	\end{proof}
	\step{e}{$\mathbf{F} \restriction \seg x$ is acceptable}
	\step{f}{\pflet{$y$ be the unique object such that $G[\mathbf{F} \restriction \seg x, x, y]$}}
	\step{g}{$\mathbf{F} \restriction \seg x \cup \{(x,y)\}$ is acceptable}
	\step{h}{$x \in \dom \mathbf{F}$}
	\qedstep
	\begin{proof}
		\pf\ This is a contradiction.
	\end{proof}
\end{proof}
\step{8}{If $\mathbf{F}' : \mathbf{A} \rightarrow \mathbf{V}$ satisfies the theorem, then $\mathbf{F}' = \mathbf{F}$.}
\begin{proof}
	\step{a}{\pflet{$x \in \mathbf{A}$} \prove{$\mathbf{F}'(x) = \mathbf{F}(x)$}}
	\step{b}{\assume{as transfinite induction hypothesis $\forall y \mathbf{R} x. \mathbf{F}'(y) = \mathbf{F}(y)$}}
	\step{c}{$\mathbf{F} \restriction x = \mathbf{F}' \restriction x$}
	\step{d}{$G[\mathbf{F} \restriction x, x, \mathbf{F}(x)]$}
	\step{e}{$G[\mathbf{F}' \restriction x, x, \mathbf{F}'(x)]$}
	\step{f}{$\mathbf{F}(x) = \mathbf{F}'(x)$}
\end{proof}
\qed
\end{proof}

\begin{thms}[Mostowski's Isomorphism Theorem (SF without Foundation)]
For any class $\mathbf{R}$, there exist classes $\mathbf{B}$ and $\mathbf{F}$ such that the following is a theorem.

Assume $\mathbf{R}$ is a well-founded relation. Let $\mathbf{A} = \dom \mathbf{R} \cup \ran \mathbf{R}$. Assume that, for all $x,y \in \mathbf{A}$, if $\forall t (t \mathbf{R} x \Leftrightarrow t \mathbf{R} y)$, then $x = y$. Then $\mathbf{B}$ is a transitive class and $\mathbf{F}$ is an isomorphism between $(\mathbf{A}, \mathbf{R})$ and $(\mathbf{B}, \in)$.
\end{thms}

\begin{proof}
\pf
\step{1}{Define $\mathbf{F} : \mathbf{A} \rightarrow \mathbf{V}$ by transfinite recursion so that, for all $x \in \mathbf{A}$, we have
\[ \mathbf{F}(x) = \{ \mathbf{F}(y) \mid y \mathbf{R} x \} \enspace . \]}
\step{2}{\pflet{$\mathbf{B} = \ran \mathbf{F}$}}
\step{3}{$\mathbf{B}$ is a transitive class.}
\begin{proof}
	\step{a}{\pflet{$u \in v \in \mathbf{B}$}}
	\step{b}{\pick\ $y$ such that $\mathbf{F}(y) = v$}
	\step{d}{There exists $x$ such that $x \mathbf{R} y$ and $u = \mathbf{F}$}
	\step{e}{$u \in \mathbf{B}$}
\end{proof}
\step{5}{$\mathbf{F}$ is injective.}
\begin{proof}
	\step{a}{\pflet{$x \in \mathbf{A}$} \prove{$\forall y \in \mathbf{A}. \mathbf{F}(x) = \mathbf{F}(y) \Rightarrow x = y$}}
	\step{b}{\assume{as transfinite induction hypothesis $\forall x' \mathbf{R} x. \forall y \in \mathbf{A}. \mathbf{F}(x') = \mathbf{F}(y) \Rightarrow x' = y$}}
	\step{c}{\pflet{$y \in \mathbf{A}$}}
	\step{d}{\assume{$\mathbf{F}(x) = \mathbf{F}(y)$}}
	\step{e}{$\forall t. (t \mathbf{R} x \Leftrightarrow t \mathbf{R} y)$}
	\begin{proof}
		\step{i}{\pflet{$t$ be a set.}}
		\step{ii}{If $t \mathbf{R} x$ then $t \mathbf{R} y$}
		\begin{proof}
			\step{one}{\assume{$t \mathbf{R} x$}}
			\step{two}{$\mathbf{F}(t) \in \mathbf{F}(x)$}
			\step{three}{$\mathbf{F}(t) \in \mathbf{F}(y)$}
			\step{four}{\pick\ $t'$ such that $t' \mathbf{R} y$ and $\mathbf{F}(t) = \mathbf{F}(t')$}
			\step{five}{$t = t'$}
			\begin{proof}
				\pf\ \stepref{b}
			\end{proof}
			\step{six}{$t \mathbf{R} y$}
		\end{proof}
		\step{iii}{If $t \mathbf{R} y$ then $t \mathbf{R} x$}
		\begin{proof}
			\pf\ Similar.
		\end{proof}
	\end{proof}
	\step{f}{$x = y$}
\end{proof}
\step{7}{For all $x,y \in \mathbf{A}$, if $x \mathbf{R} y$ then $\mathbf{F}(x) \in \mathbf{F}(y)$.}
\step{8}{For all $x,y \in \mathbf{A}$, if $\mathbf{F}(x) \in \mathbf{F}(y)$ then $x \mathbf{R} y$.}
\qed
\end{proof}

\begin{df}[Transitive Collapse]
For any set $x$, the \emph{transitive collapse} of $x$ is the unique set $y$ such that $(x, \in) \cong (y, \in)$. The unique isomorphism between them is called the \emph{collapsing isomorphism}.
\end{df}

\chapter{Category Theory}

\section{Categories}

\begin{df}[Category]
A \emph{category} $\mathbf{C}$ consists of:
\begin{itemize}
\item a class of \emph{objects};
\item for any objects $X$, $Y$, a set $\mathbf{C}(X,Y)$ whose elements are called \emph{morphisms}. We write $f : X \rightarrow Y$ for $f \in \mathbf{C}(X,Y)$.
\item for any morphisms $f : X \rightarrow Y$ and $g : Y \rightarrow Z$, a morphism $gf : X \rightarrow Z$
\end{itemize}
such that:
\begin{itemize}
\item For all $f : X \rightarrow Y$, $g : Y \rightarrow Z$, $h : Z \rightarrow W$, we have $h(gf) = (hg)f$.
\item For every object $X$, there exists a morphism $\id{X} : X \rightarrow X$ such that:
\begin{itemize}
\item for any object $Y$ and morphism $f : X \rightarrow Y$ we have $f \id{X} = f$.
\item for any object $Y$ and morphism $f : Y \rightarrow X$ we have $\id{X} f = f$.
\end{itemize}
\end{itemize}
\end{df}

\begin{df}[Category of Sets]
Let $\mathrm{Set}$ be the category of sets and functions.
\end{df}

\begin{df}[Category of Pointed Sets]
The \emph{category of pointed sets} $\mathrm{Set}_*$ has:
\begin{itemize}
\item objects all pairs $(A,a)$ where $A$ is a set and $a \in A$;
\item morphisms $f : (A,a) \rightarrow (B,b)$ all functions $f : A \rightarrow B$ such that $f(a) = b$.
\end{itemize}
\end{df}

\begin{df}[Opposite Category]
For any category $\mathbf{C}$, the \emph{opposite category} $\mathbf{C}^{\mathrm{op}}$ is the category with the same objects as $\mathbf{C}$ and $\mathbf{C}^{\mathrm{op}}(X,Y) = \mathbf{C}(Y,X)$.
\end{df}

\section{Invertible Morphisms}

\begin{df}[Left Inverse, Right Inverse]
In any category, let $f : A \rightarrow B$ and $g : B \rightarrow A$. Then $g$ is a \emph{left inverse} of $f$, and $f$ is a \emph{right inverse} of $f$, iff $g f = \id{A}$.
\end{df}

\begin{prop}[FOL]
Let $f : A \rightarrow B$ and $g,h : B \rightarrow A$. If $g$ is a left inverse to $f$ and $h$ is a right inverse to $f$ then $g = h$.
\end{prop}

\begin{proof}
\pf\ Since $g = g \id{B} = gfh = \id{A} h = h$. \qed
\end{proof}

\begin{df}[Isomorphism]
A morphism $f : A \rightarrow B$ is an \emph{isomorphism}, $f : A \cong B$, iff it has both a left and a right inverse. In this case, its unique inverse is denoted $f^{-1}$.

Two objects $A$ and $B$ are \emph{isomorphic}, $A \cong B$, iff there exists an isomorphism between them.
\end{df}

\begin{prop}[S without Foundation]
A function is an isomorphism in $\mathrm{Set}$ iff it is a bijection.
\end{prop}

\begin{prop}[S without Foundation]
Let $f : A \rightarrow B$ in the category $\mathbf{C}$. Then the following are equivalent.
\begin{enumerate}
\item $f$ is an isomorphism.
\item For every object $X$, the function $\mathbf{C}(X,f) : \mathbf{C}(X,A) \rightarrow \mathbf{C}(X,B)$ is a bijection.
\item For every object $X$, the function $\mathbf{C}(f,X) : \mathbf{C}(B,X) \rightarrow \mathbf{C}(A,X)$ is a bijection.
\end{enumerate}
\end{prop}

\chapter{Equivalence Relations}

\begin{df}[Equivalence Relation]
An \emph{equivalence relation} on a class $\mathbf{A}$ is a binary relation on $\mathbf{A}$ that is reflexive, symmetric and transitive.
\end{df}

\begin{prop}[S without Foundation]
Equinumerosity is an equivalence relation on the class of all sets.
\end{prop}

\begin{proof}
\pf\ Propositions \ref{prop:idbij}, \ref{prop:invbij}, \ref{prop:compbij}. \qed
\end{proof}

\begin{df}[Respects]
Let $\mathbf{R}$ be an equivalence relation on $\mathbf{A}$ and $\mathbf{F} : \mathbf{A} \rightarrow \mathbf{B}$. Then $\mathbf{F}$ \emph{respects} $\mathbf{A}$ iff, whenever $(x,y) \in \mathbf{R}$, then $\mathbf{F}(x) = \mathbf{F}(y)$.
\end{df}

\begin{df}[Equivalence Class]
Let $\mathbf{R}$ be an equivalence relation on $\mathbf{A}$ and $a \in \mathbf{A}$. The \emph{equivalence class} of $a$ \emph{modulo} $\mathbf{R}$ is
\[ [a]_{\mathbf{R}} := \{ x \mid a \mathbf{R} x \} \enspace . \]
\end{df}

\begin{props}[S without Foundation]
\label{prop:eqclassequal}
For any classes $\mathbf{A}$ and $\mathbf{R}$, the following is a theorem.

Assume $\mathbf{R}$ be an equivalence relation on $\mathbf{A}$. Let $a,b \in \mathbf{A}$. Then $[a]_\mathbf{R} = [b]_\mathbf{R}$ if and only if $a \mathbf{R} b$.
\end{props}

\begin{proof}
\pf
\step{1}{If $[a]_\mathbf{R} = [b]_\mathbf{R}$ then $a \mathbf{R} b$.}
\begin{proof}
	\step{a}{\assume{$[a]_\mathbf{R} = [b]_\mathbf{R}$}}
	\step{b}{$b\mathbf{R}b$}
	\begin{proof}
		\pf\ Reflexivity
	\end{proof}
	\step{c}{$b \in [b]_\mathbf{R}$}
	\step{d}{$b \in [a]_\mathbf{R}$}
	\step{e}{$a\mathbf{R}b$}
\end{proof}
\step{2}{If $a\mathbf{R}b$ then $[a]_\mathbf{R} = [b]_\mathbf{R}$.}
\begin{proof}
	\step{a}{For all $x,y \in \mathbf{A}$, if $x \mathbf{R}y$ then $[y]_\mathbf{R} \subseteq [x]_\mathbf{R}$}
	\begin{proof}
		\step{i}{\pflet{$x,y \in \mathbf{A}$}}
		\step{ii}{\assume{$x\mathbf{R}y$}}
		\step{iii}{\pflet{$t \in [y]_\mathbf{R}$}}
		\step{iv}{$y\mathbf{R}t$}
		\step{v}{$x\mathbf{R}t$}
		\begin{proof}
			\pf\ Transitivity, \stepref{ii}, \stepref{iv}.
		\end{proof}
		\step{vi}{$t \in [x]_\mathbf{R}$}
	\end{proof}
	\step{b}{\assume{$a\mathbf{R}b$}}
	\step{c}{$[b]_\mathbf{R} \subseteq [a]_\mathbf{R}$}
	\begin{proof}
		\pf\ \stepref{a}, \stepref{b}.
	\end{proof}
	\step{d}{$b\mathbf{R}a$}
	\begin{proof}
		\pf\ Symmetry, \stepref{b}.
	\end{proof}
	\step{e}{$[a]_\mathbf{R} \subseteq [b]_\mathbf{R}$}
	\begin{proof}
		\pf\ \stepref{a}, \stepref{d}.
	\end{proof}
	\step{f}{$[a]_\mathbf{R} = [b]_\mathbf{R}$}
	\begin{proof}
		\pf\ \stepref{c}, \stepref{e}.
	\end{proof}
\end{proof}
\qed
\end{proof}

\begin{df}[Partition]
A \emph{partition} $\Pi$ of a set $A$ is a set of nonempty subsets of $A$ that is disjoint and exhaustive, i.e.
\begin{enumerate}
\item no two different sets in $\Pi$ have any common elements, and
\item each element of $A$ is in some set in $\Pi$.
\end{enumerate}
\end{df}

\begin{df}
Let $R$ be an equivalence relation on a set $A$. The \emph{quotient set} $A / R$ is the set of all equivalence classes.
\end{df}

\begin{thm}[S without Foundation]
Let $A$ be a set and $\mathbf{B}$ a class. Let $R$ be an equivalence relation on $A$ and $F : A \rightarrow \mathbf{B}$. Then $F$ respects $R$ if and only if there exists $\hat{F} : A/R \rightarrow \mathbf{B}$ such that
\[ \forall a \in A. \hat{F}([a]_R) = F(a) \enspace . \]
In this case, $\hat{F}$ is unique.
\end{thm}

\begin{proof}
\pf
\step{1}{If $F$ respects $R$ then there exists $\hat{F} : A / R \rightarrow \mathbf{B}$ such that $\forall a \in A. \hat{F}([a]_R) = F(a)$.}
\begin{proof}
	\step{a}{\assume{$F$ respects $R$.}}
	\step{b}{\pflet{$\hat{F} = \{([a]_R, F(a)) \mid a \in A \}$}}
	\step{c}{$\hat{F}$ is a function.}
	\begin{proof}
		\step{i}{\assume{$a,a' \in A$ and $[a]_R = [a']_R$} \prove{$F(a) = F(a')$}}
		\step{ii}{$(a,a') \in R$}
		\begin{proof}
			\pf\ Proposition \ref{prop:eqclassequal}.
		\end{proof}
		\step{iii}{$F(a) = F(a')$}
		\begin{proof}
			\pf\ \stepref{a}
		\end{proof}
	\end{proof}
	\step{d}{$\dom \hat{F} = A / R$}
	\step{e}{$\ran \hat{F} \subseteq \mathbf{B}$}
	\step{f}{$\forall a \in A. \hat{F}([a]_R) = F(a)$}
\end{proof}
\step{2}{If there exists $\hat{F} : A / R \rightarrow \mathbf{B}$ such that $\forall a \in A. \hat{F}([a]_R) = F(a)$ then $F$ respects $R$.}
\begin{proof}
	\step{a}{\assume{$\hat{F} : A / R \rightarrow \mathbf{B}$ and $\forall a \in A. \hat{F}([a]_R) = F(a)$}}
	\step{b}{\pflet{$a,a' \in A$}}
	\step{c}{\assume{$(a,a') \in R$}}
	\step{d}{$[a]_R = [a']_R$}
		\begin{proof}
			\pf\ Proposition \ref{prop:eqclassequal}.
		\end{proof}
	\step{e}{$F(a) = F(a')$}
	\begin{proof}
		\pf\ \stepref{a}
	\end{proof}
\end{proof}
\step{3}{If $G, H : A / R \rightarrow \mathbf{B}$ and $\forall a \in A. G([a]_R) = H([a]_R)$ then $G = H$.}
\qed
\end{proof}

\begin{prop}[S without Foundation]
Let $R$ be an equivalence relation on a set $A$. Then $A/R$ is a partition of $A$.
\end{prop}

\begin{proof}
\pf
\step{2}{Every member of $A/R$ is nonempty.}
\begin{proof}
	\pf\ Since $a \in [a]_R$ by reflexivity.
\end{proof}
\step{3}{No two different sets in $A/R$ have any common elements.}
\begin{proof}
	\step{a}{\pflet{$[a]_R, [b]_R \in A/R$}}
	\step{b}{\pflet{$c \in [a]_R \cap [b]_R$} \prove{$[a]_R = [b]_R$}}
	\step{c}{$aRc$}
	\begin{proof}
		\pf\ \stepref{b}
	\end{proof}
	\step{d}{$bRc$}
	\begin{proof}
		\pf\ \stepref{b}
	\end{proof}
	\step{e}{$cRb$}
	\begin{proof}
		\pf\ Symmetry, \stepref{d}
	\end{proof}
	\step{f}{$aRb$}
	\begin{proof}
		\pf\ Transitivity, \stepref{c}, \stepref{e}
	\end{proof}
	\step{g}{$[a]_R = [b]_R$}
	\begin{proof}
		\pf\ Proposition \ref{prop:eqclassequal}, \stepref{f}
	\end{proof}
\end{proof}
\step{4}{Each element of $A$ is in some set in $A/R$.}
\begin{proof}
	\pf\ Since $a \in [a]_R$ by reflexivity.
\end{proof}
\qed
\end{proof}

\begin{prop}[S without Foundation]
For any partition $P$ of a set $A$, there exists a unique equivalence relation $R$ on $A$ such that $A/R = P$, namely $xRy$ iff $\exists X \in P(x \in X \wedge y \in X)$.
\end{prop}

\begin{proof}
\pf\ Easy. \qed
\end{proof}

\begin{df}[Natural Map]
Let $A$ be a set and $R$ an equivalence relation on $A$. The \emph{natural map} $A \rightarrow A / R$ is the function that maps $a \in A$ to $[a]_R$.
\end{df}

\chapter{Ordering Relations}

\section{Partial Orders}

\begin{df}[Partial Ordering]
Let $\mathbf{A}$ be a class. A \emph{partial ordering} on $\mathbf{A}$ is a relation $\mathbf{R}$ on $\mathbf{A}$ that is reflexive, antisymmetric and transitive.

We often write $\leq$ for a partial ordering, and then write $x < y$ for $x \leq y \wedge x \neq y$.
\end{df}

\begin{props}[S without Foundation]
\label{prop:invposet}
For any classes $\mathbf{A}$ and $\mathbf{R}$, the following is a theorem:

If $\mathbf{R}$ is a partial order on $\mathbf{A}$ then so is $\mathbf{R}^{-1}$.
\end{props}

\begin{proof}
\pf
\step{1}{$\mathbf{R}^{-1}$ is reflexive.}
\begin{proof}
	\pf\ Proposition \ref{prop:invref}.
\end{proof}
\step{2}{$\mathbf{R}^{-1}$ is antisymmetric.}
\begin{proof}
	\pf\ Proposition \ref{prop:invantisym}.
\end{proof}
\step{3}{$\mathbf{R}^{-1}$ is transitive.}
\begin{proof}
	\step{a}{\assume{$x \mathbf{R}^{-1} y$ and $y \mathbf{R}^{-1} z$}}
	\step{b}{$y \mathbf{R} x$ and $z \mathbf{R} y$}
	\step{c}{$z \mathbf{R} x$}
	\begin{proof}
		\pf\ Since $\mathbf{R}$ is transitive.
	\end{proof}
	\step{d}{$x \mathbf{R}^{-1} z$}
\end{proof}
\qed
\end{proof}

\begin{props}[S without Foundation]
\label{prop:subposet}
For any classes $\mathbf{A}$, $\mathbf{B}$, $\mathbf{F}$ and $\mathbf{R}$, the following is a theorem:

Assume $\mathbf{R}$ is a partial order on $\mathbf{B}$ and $\mathbf{F} : \mathbf{A} \rightarrow \mathbf{B}$ is injective. Define $\mathbf{S}$ on $\mathbf{A}$ by $x \mathbf{S} y$ iff $\mathbf{F}(x) \mathbf{R} \mathbf{F}(y)$. Then $\mathbf{S}$ is a partial order on $\mathbf{A}$.
\end{props}

\begin{proof}
\pf
\step{1}{$\mathbf{S}$ is reflexive.}
\begin{proof}
	\pf\ For any $x \in \mathbf{A}$ we have $\mathbf{F}(x) \mathbf{R} \mathbf{F}(x)$.
\end{proof}
\step{2}{$\mathbf{S}$ is antisymmetric.}
\begin{proof}
	\step{a}{\pflet{$x,y \in \mathbf{A}$}}
	\step{b}{\assume{$x \mathbf{S} y$ and $y \mathbf{S} x$}}
	\step{c}{$\mathbf{F}(x) \mathbf{R} \mathbf{F}(y)$ and $\mathbf{F}(y) \mathbf{R} \mathbf{F}(x)$}
	\step{d}{$\mathbf{F}(x) = \mathbf{F}(y)$}
	\begin{proof}
		\pf\ $\mathbf{R}$ is antisymmetric.
	\end{proof}
	\step{e}{$x = y$}
\end{proof}
\step{3}{$\mathbf{S}$ is transitive.}
\qed
\end{proof}

\begin{cors}[S without Foundation]
\label{cor:subposet}
For any classes $\mathbf{A}$, $\mathbf{B}$ and $\mathbf{R}$, the following is a theorem:

Assume $\mathbf{R}$ be a partial order on $\mathbf{A}$ and $\mathbf{B} \subseteq \mathbf{A}$. Then $\mathbf{R} \cap \mathbf{B}^2$ is a partial order on $\mathbf{B}$.
\end{cors}

\begin{df}[Partially Ordered Set]
A \emph{partially ordered set} or \emph{poset} is a pair $(A, \leq)$ where $A$ is a set and $\leq$ is a partial ordering on $A$. We often write just $A$ for $(A, \leq)$.

If $(A, \leq)$ is a poset and $B \subseteq A$ we write just $B$ for the poset $(B, \leq \cap B^2)$.
\end{df}

\begin{df}[Strictly Monotone]
Let $(A,<_A)$ and $(B, <_B)$ be posets. A function $f : A \rightarrow B$ is \emph{strictly monotone} iff, whenever $x <_A y$, then $f(x) <_B f(y)$.
\end{df}

\begin{df}[Least]
Let $\leq$ be a partial order on $\mathbf{A}$. An element $m \in \mathbf{A}$ is \emph{least} iff for all $x \in \mathbf{A}$ we have $m \leq x$.
\end{df}

\begin{prop}[S without Foundation]
A partial order has at most one least element.
\end{prop}

\begin{proof}
\pf\ If $m$ and $m'$ are least then $m \leq m'$ and $m' \leq m$, so $m = m'$. \qed
\end{proof}

\begin{df}[Greatst]
Let $\leq$ be a partial order on $\mathbf{A}$. An element $m \in \mathbf{A}$ is \emph{greatest} iff for all $x \in A$ we have $x \leq m$.
\end{df}

\begin{prop}[S without Foundation]
A poset has at most one greatest element.
\end{prop}

\begin{proof}
\pf\ If $m$ and $m'$ are greatest then $m \leq m'$ and $m' \leq m$, so $m = m'$. \qed
\end{proof}

\begin{df}[Upper Bound]
Let $\leq$ be a partial ordering on $\mathbf{A}$ and $\mathbf{B} \subseteq \mathbf{A}$. Let $u \in \mathbf{A}$. Then $u$ is an \emph{upper bound} for $\mathbf{B}$ iff $\forall x \in \mathbf{B}. x \leq u$.
\end{df}

\begin{df}[Lower Bound]
Let $\leq$ be a partial ordering on $\mathbf{A}$ and $\mathbf{B} \subseteq \mathbf{A}$. Let $l \in \mathbf{A}$. Then $l$ is a \emph{lower bound} for $\mathbf{B}$ iff $\forall x \in \mathbf{B}. l \leq x$.
\end{df}

\begin{df}[Bounded Above]
Let $\leq$ be a partial ordering on $\mathbf{A}$ and $\mathbf{B} \subseteq \mathbf{A}$. Then $\mathbf{B}$ is \emph{bounded above} iff it has an upper bound.
\end{df}

\begin{df}[Bounded Below]
Let $\leq$ be a partial ordering on $\mathbf{A}$ and $\mathbf{B} \subseteq \mathbf{A}$. Then $\mathbf{B}$ is \emph{bounded below} iff it has a lower bound.
\end{df}

\begin{df}[Least Upper Bound]
Let $\leq$ be a partial ordering on $\mathbf{A}$ and $\mathbf{B} \subseteq \mathbf{A}$. Let $s \in \mathbf{A}$. Then $s$ is the \emph{least upper bound} or \emph{supremum} of $\mathbf{B}$ iff $s$ is an upper bound for $\mathbf{B}$ and, for every upper bound $u$ for $\mathbf{B}$, we have $s \leq u$.
\end{df}

\begin{df}[Greatest Lower Bound]
Let $\leq$ be a partial ordering on $\mathbf{A}$ and $\mathbf{B} \subseteq \mathbf{A}$. Let $i \in \mathbf{A}$. Then $i$ is the \emph{greatest lower bound} or \emph{infimum} of $\mathbf{B}$ iff $i$ is a lower bound for $\mathbf{B}$ and, for every lower bound $l$ for $\mathbf{B}$, we have $i \leq l$.
\end{df}

\begin{df}[Complete]
A partial order is \emph{complete} iff every nonempty subset bounded above has a supremum, and every nonempty subset bounded below has an infimum.
\end{df}

\begin{df}[Order Isomorphism]
Let $A$ and $B$ be posets. An \emph{order isomorphism} between $A$ and $B$, $f : A \cong B$, is a bijection $f : A \approx B$ such that $f$ and $f^{-1}$ are monotone.
\end{df}
	
\begin{thm}[Knaster Fixed-Point Theorem (S without Foundation)]
Let $A$ be a complete poset with a greatest and least element. Let $\phi : A \rightarrow A$ be monotone. Then there exists $a \in A$ such that $\phi(a) = a$.
\end{thm}

\begin{proof}
\pf
\step{1}{\pflet{$B = \{ x \in A \mid x \leq \phi(x) \}$}}
\step{2}{\pflet{$a = \sup B$}}
\begin{proof}
	\pf\ $B$ is nonempty because the least element of $A$ is in $B$, and it is bounded above by the greatest element of $A$.
\end{proof}
\step{3}{For all $b \in B$ we have $b \leq \phi(a)$}
\begin{proof}
	\step{a}{\pflet{$b \in B$}}
	\step{b}{$b \leq \phi(b)$}
	\step{c}{$b \leq a$}
	\step{d}{$\phi(b) \leq \phi(a)$}
	\step{e}{$b \leq \phi(a)$}
\end{proof}
\step{4}{$a \leq \phi(a)$}
\step{5}{$\phi(a) \leq \phi(\phi(a))$}
\step{6}{$\phi(a) \in B$}
\step{7}{$\phi(a) \leq a$}
\step{8}{$\phi(a) = a$}
\qed
\end{proof}

\begin{df}[Dense]
Let $\leq$ be a partial order on $\mathbf{A}$ and
$\mathbf{B} \subseteq \mathbf{A}$. Then $\mathbf{B}$ is \emph{dense} iff, for all $x,y \in \mathbf{A}$, if $x < y$ then there exists $z \in \mathbf{B}$ such that $x < z < y$.
\end{df}

\begin{prop}[S without Foundation]
\label{prop:orderisoid}
Let $A$ be a complete poset with no least element. Let $B \subseteq A$ be dense. Let $\theta : A \rightarrow A$ be a monotone map that is the identity on $B$. Then $\theta = \mathrm{id}_A$.
\end{prop}

\begin{proof}
\pf
\step{1}{\pflet{$a \in A$} \prove{$\theta(a) = a$}}
\step{3}{\pflet{$S(a) = \{b \in B \mid b < a\}$}}
\step{2}{$S(a)$ is nonempty and bounded above.}
\begin{proof}
	\step{a}{$S(a)$ is nonempty.}
	\begin{proof}
		\step{i}{\pick\ $a_1 < a$}
		\begin{proof}
			\pf\ Since $a$ is not least.
		\end{proof}
		\step{ii}{There exists $b \in B$ such that $a_1 < b < a$.}
	\end{proof}
	\step{b}{$S(a)$ is bounded above by $a$.}
\end{proof}
\step{4}{$\sup S(a) \leq a$}
\step{5}{$\sup S(a) = a$}
\begin{proof}
	\step{a}{\assume{for a contradiction $\sup S(a) < a$}}
	\step{b}{\pick\ $b \in B$ such that $\sup S(a) < b < a$}
	\step{c}{$b \in S(a)$}
	\qedstep
	\begin{proof}
		\pf\ This contradicts the fact that $\sup S(a) < b$.
	\end{proof}
\end{proof}
\step{6}{For all $b \in S(a)$ we have $b \leq \theta(a)$}
\begin{proof}
	\step{a}{\pflet{$b \in S(a)$}}
	\step{b}{$b < a$}
	\step{c}{$\theta(b) \leq \theta(a)$}
	\step{d}{$b \leq \theta(a)$}
	\begin{proof}
		\pf\ $\theta(b) = b$
	\end{proof}
\end{proof}
\step{7}{$a \leq \theta(a)$}
\begin{proof}
	\pf\ Since $a = \sup S(a)$ and $\theta(a)$ is an upper bound for $S(a)$.
\end{proof}
\step{8}{$a \nless \theta(a)$}
\begin{proof}
	\step{a}{\assume{for a contradiction $a < \theta(a)$.}}
	\step{b}{\pick\ $b \in B$ such that $a < b < \theta(a)$}
	\step{c}{$\theta(a) \leq \theta(b) = b$}
	\qedstep
	\begin{proof}
		\pf\ This contradicts the fact that $b < \theta(a)$.
	\end{proof}
\end{proof}
\step{9}{$\theta(a) = a$}
\qed
\end{proof}

\begin{thm}[S without Foundation]
\label{thm:extendorderiso}
Let $A$ and $P$ be complete posets with no least or greatest element. Let $B$ be dense in $A$ and $Q$ be dense in $P$. Every order isomorphism $\phi : B \cong Q$ extends uniquely to an order isomorphism $A \cong P$.
\end{thm}

\begin{proof}
\pf
\step{1}{For $a \in A$, let $S(a) = \{ b \in B \mid b < a \}$.}
\step{2}{Define $\overline{\phi} : A \rightarrow P$ by $\overline{\phi}(a) = \sup \phi(S(a))$.}
\begin{proof}
	\step{a}{$\phi(S(a))$ is nonempty.}
	\begin{proof}
		\step{i}{\pick\ $a_1 < a$}
		\begin{proof}
			\pf\ Since $a$ is not least.
		\end{proof}
		\step{ii}{\pick\ $b \in B$ such that $a_1 < b < a$.}
		\step{iii}{$\phi(b) \in \phi(S(a))$}
	\end{proof}
	\step{b}{$\phi(S(a))$ is bounded above.}
	\begin{proof}
		\step{i}{\pick\ $a_2 > a$}
		\begin{proof}
			\pf\ Since $a$ is not greatest.
		\end{proof}
		\step{ii}{\pick\ $b \in B$ such that $a < b < a_2$}
		\step{iii}{$\phi(b)$ is an upper bound for $\phi(S(a))$.}
	\end{proof}
\end{proof}
\step{3}{$\overline{\phi}$ is monotone.}
\begin{proof}
	\pf\ If $a \leq a'$ then $S(a) \subseteq S(a')$ and so $\overline{\phi}(a) \leq \overline{\phi}(a')$.
\end{proof}
\step{4}{$\overline{\phi}$ extends $\phi$.}
\begin{proof}
	\step{a}{\pflet{$b \in B$} \prove{$\phi(b) = \sup \phi(S(b))$}}
	\step{b}{$\phi(b)$ is an upper bound for $\phi(S(b))$}
	\step{c}{\pflet{$u$ be any upper bound for $\phi(S(b))$} \prove{$\phi(b) \leq u$}}
	\step{d}{\assume{for a contradiction $u < \phi(b)$}}
	\step{e}{\pick\ $q \in Q$ such that $u < q < \phi(b)$}
	\step{f}{\pick\ $b' \in B$ such that $\phi(b') = q$}
	\step{g}{$b' < b$}
	\step{h}{$b' \in S(b)$}
	\step{i}{$q = \phi(b') \leq u$}
	\qedstep
	\begin{proof}
		\pf\ This is a contradiction.
	\end{proof}
\end{proof}
\step{5}{\pflet{$\psi = \phi^{-1}$}}
\step{6}{\pflet{$\overline{\psi} : P \rightarrow A$ be the function $\overline{\psi}(p) = \sup \{ \psi(q) \mid q \in Q, q < p \}$}}
\step{7}{$\overline{\psi}$ is monotone and extends $\psi$}
\begin{proof}
	\pf\ Similar.
\end{proof}
\step{8}{$\overline{\psi} \circ \overline{\phi} : A \rightarrow A$ is monotone and the identity on $B$.}
\step{9}{$\overline{\psi} \circ \overline{\phi} = \mathrm{id}_A$}
\begin{proof}
	\pf\ Proposition \ref{prop:orderisoid}.
\end{proof}
\step{10}{$\overline{\phi} \circ \overline{\psi} = \mathrm{id}_B$}
\begin{proof}
	\pf\ Proposition \ref{prop:orderisoid}.
\end{proof}
\step{11}{If $\phi^* : A \cong P$ is any order isomorphism that extends $\phi$ then $\phi^* = \overline{\phi}$.}
\begin{proof}
	\step{a}{\pflet{$a \in A$} \prove{$\phi^*(a) = \sup \phi(S(a))$}}
	\step{b}{$\phi^*(a)$ is an upper bound for $\phi(S(a))$}
	\step{c}{\pflet{$u$ be any upper bound for $\phi(S(a))$} \prove{$\phi^*(a) \leq u$}}
	\step{d}{\assume{for a contradiction $u < \phi^*(a)$}}
	\step{e}{\pick\ $q \in Q$ such that $u < q < \phi^*(a)$}
	\step{f}{\pick\ $b \in B$ such that $q = \phi(b)$}
	\step{g}{$b < a$}
	\step{h}{$b \in S(a)$}
	\step{i}{$q = \phi(b) \leq u$}
	\qedstep
	\begin{proof}
		\pf\ This is a contradiction.
	\end{proof}
\end{proof}
\qed
\end{proof}

\begin{df}[Initial Segment]
Let $\leq$ be a partial order on $\mathbf{A}$ and $t \in A$. The \emph{initial segment} up to $t$ is the class
\[ \seg t := \{ x \in \mathbf{A} \mid x < t \} \enspace . \]
\end{df}

\begin{df}[Lexicographic Ordering]
Let $\mathbf{R}$ be a partial order on $\mathbf{A}$ and $\mathbf{S}$ a partial order on $\mathbf{B}$. The \emph{lexicographic ordering} $\leq$ on $\mathbf{A} \times \mathbf{B}$ is defined by:
\[ (a,b) \leq (a',b') \Leftrightarrow (a \mathbf{R} a' \wedge a \neq a') \vee (a = a' \wedge b \mathbf{S} b') \enspace . \]
\end{df}

\begin{props}[S without Foundation]
\label{prop:lexicposet}
For any classes $\mathbf{A}$, $\mathbf{B}$, $\mathbf{R}$ and $\mathbf{S}$, the following is a theorem:

If $\mathbf{R}$ is a partial order on $\mathbf{A}$ and $\mathbf{S}$ is a partial order on $\mathbf{B}$ then the lexicographic ordering on $\mathbf{A} \times \mathbf{B}$ is a partial order.
\end{props}

\begin{proof}
\pf
\step{1}{\pflet{$\leq$ be the lexicographic ordering on $\mathbf{A} \times \mathbf{B}$}}
\step{2}{$\leq$ is reflexive.}
\begin{proof}
	\pf\ For any $a \in \mathbf{A}$ and $b \in \mathbf{B}$ we have $a = a$ and $b \mathbf{S} b$, so $(a,b) \leq (a,b)$.
\end{proof}
\step{3}{$\leq$ is antisymmetric.}
\begin{proof}
	\step{a}{\assume{$(a,b) \leq (a',b')$ and $(a',b') \leq (a,b)$}}
	\step{b}{$(a \mathbf{R} a' \wedge a \neq a') \vee (a = a' \wedge b \mathbf{S} b')$}
	\step{c}{$(a' \mathbf{R} a \wedge a' \neq a) \vee (a' = a \wedge b \mathbf{S} b')$}
	\step{d}{\case{$a = a'$}}
	\begin{proof}
		\pf\ Then $b \mathbf{S} b'$ and $b' \mathbf{S} b$ hence $b = b'$ and $(a,b) = (a',b')$.
	\end{proof}
	\step{e}{\case{$a \neq a'$}}
	\begin{proof}
		\pf\ Then $a \mathbf{R} a'$ and $a' \mathbf{R} a$ hence $a = a'$ which is a contradiction.
	\end{proof}
\end{proof}
\step{4}{$\leq$ is transitive.}
\begin{proof}
	\step{a}{\assume{$(a_1,b_1) \leq (a_2,b_2) \leq (a_3,b_3)$}}
	\step{b}{$(a_1 \mathbf{R} a_2 \wedge a_1 \neq a_2) \vee (a_1 = a_2 \wedge b_1 \mathbf{S} b_2)$}
	\step{c}{$(a_2 \mathbf{R} a_3 \wedge a_2 \neq a_3) \vee (a_2 = a_3 \wedge b_2 \mathbf{S} b_3)$}
	\step{d}{\case{$a_1 \mathbf{R} a_2, a_1 \neq a_2, a_2 \mathbf{R} a_3, a_2 \neq a_3$}}
	\begin{proof}
		\step{i}{$a_1 \mathbf{R} a_3$}
		\begin{proof}
			\pf\ Since $\mathbf{R}$ is transitive.
		\end{proof}
		\step{ii}{$a_1 \neq a_3$}
		\begin{proof}
			\pf\ If $a_1 = a_3$ then $a_1 \mathbf{R} a_2$ and $a_2 \mathbf{R} a_1$ so $a_1 = a_2$ which is a contradiction.
		\end{proof}
	\end{proof}
	\step{e}{\case{$a_1 \mathbf{R} a_2, a_1 \neq a_2, a_2 = a_3, b_2 \mathbf{S} b_3$}}
	\begin{proof}
		\pf\ Then $a_1 \mathbf{R} a_3$ and $a_1 \neq a_3$.
	\end{proof}
	\step{f}{\case{$a_1 = a_2, b_1 \mathbf{S} b_2, a_2 \mathbf{R} a_3, a_2 \neq a_3$}}
	\begin{proof}
		\pf\ Then $a_1 \mathbf{R} a_3$ and $a_1 \neq a_3$.
	\end{proof}
	\step{g}{\case{$a_1 = a_2, b_1 \mathbf{S} b_2, a_2 = a_3, b_2 \mathbf{S} b_3$}}
	\begin{proof}
		\pf\ Then $a_1 = a_3$ and $b_1 \mathbf{S} b_3$.
	\end{proof}
\end{proof}
\qed
\end{proof}

\section{Linear Orders}

\begin{df}[Linear Ordering]
Let $\mathbf{A}$ be a class. A \emph{linear ordering} or \emph{total ordering} on $\mathbf{A}$ is a partial ordering $\leq$ on $\mathbf{A}$ that is \emph{total}, i.e.
\[ \forall x,y \in \mathbf{A}. x \leq y \vee y \leq x \]

We often use the symbol $<$ for a linear ordering, and then write $x < y$ for $(x,y) \in <$.
\end{df}

\begin{props}[Trichotomy (S without Foundation)]
For any classes $\mathbf{A}$ and $\leq$, the following is a theorem:

Assume $\leq$ be a linear ordering on $\mathbf{A}$. For any $x,y \in \mathbf{A}$, exactly one of $x < y$, $x = y$, $y < x$ holds.
\end{props}

\begin{proof}
\pf\ Immediate from definitions. \qed
\end{proof}

\begin{props}[S without Foundation]
\label{prop:linord}
For any classes $\mathbf{A}$ and $<$, the following is a theorem:

Let $<$ be a transitive relation on $\mathbf{A}$ that satisfies trichotomy. Define $\leq$ on $\mathbf{A}$ by $x \leq y$ iff $x < y$ or $x = y$. Then $\leq$ is a linear ordering on $\mathbf{A}$ and $x < y$ iff $x \leq y$ and $x \neq y$.
\end{props}

\begin{proof}
\pf
\step{1}{$\leq$ is reflexive.}
\begin{proof}
	\pf\ By definition we have $\forall x \in \mathbf{A}. x \leq x$.
\end{proof}
\step{2}{$\leq$ is antisymmetric.}
\begin{proof}
	\step{a}{\assume{$x \leq y$ and $y \leq x$}}
	\step{b}{$x < y$ or $x = y$}
	\step{c}{$y < x$ or $y = x$}
	\step{d}{We cannot have $x < y$ and $y < x$}
	\begin{proof}
		\pf\ Trichotomy.
	\end{proof}
	\step{e}{$x = y$}
\end{proof}
\step{3}{$\leq$ is transitive.}
\begin{proof}
	\step{a}{\assume{$x \leq y$ and $y \leq z$}}
	\step{b}{$x < y$ or $x = y$}
	\step{c}{$y < z$ or $y = z$}
	\step{d}{\case{$x < y$ and $y < z$}}
	\begin{proof}
		\pf\ Then $x < z$ by transitivity, so $x \leq z$.
	\end{proof}
	\step{e}{\case{$x = y$}}
	\begin{proof}
		\pf\ Then we have $y \leq z$ and so $x \leq z$.
	\end{proof}
	\step{f}{\case{$y = z$}}
	\begin{proof}
		\pf\ Then we have $x \leq y$ and so $x \leq z$.
	\end{proof}
\end{proof}
\step{4}{$\leq$ is total.}
\begin{proof}
	\pf\ Immediate from trichotomy.
\end{proof}
\qed
\end{proof}

\begin{props}[S without Foundation]
For any classes $\mathbf{A}$ and $\mathbf{R}$, the following is a theorem:

If $\mathbf{R}$ is a linear ordering on $\mathbf{A}$ then $\mathbf{R}^{-1}$ is also a linear ordering on $\mathbf{A}$.
\end{props}

\begin{proof}
\pf
\step{1}{$\mathbf{R}^{-1}$ is a partial order on $\mathbf{A}$.}
\begin{proof}
	\pf\ Proposition \ref{prop:invposet}.
\end{proof}
\step{2}{$\mathbf{R}^{-1}$ is total.}
\begin{proof}
	\step{a}{\pflet{$x, y \in \mathbf{A}$}}
	\step{b}{$x \mathbf{R} y$ or $y \mathbf{R} x$.}
	\step{c}{$y \mathbf{R}^{-1} x$ or $x \mathbf{R}^{-1} y$.}
\end{proof}
\qed
\end{proof}

\begin{props}[S without Foundation]
\label{prop:strictmonotoneinj}
For any classes $\mathbf{A}$, $\mathbf{B}$, $\mathbf{F}$, $\mathbf{R}$, $\mathbf{S}$, the following is a theorem:

Assume $\mathbf{R}$ is a linear order on $\mathbf{A}$, $\mathbf{S}$ is a partial order on $\mathbf{B}$, and $\mathbf{F} : \mathbf{A} \rightarrow \mathbf{B}$. If $\mathbf{F}$ is strictly monotone then it is injective.
\end{props}

\begin{proof}
\pf
\step{3}{\pflet{$x,y \in \mathbf{A}$}}
\step{6}{\assume{$x \neq y$} \prove{$\mathbf{F}(x) \neq \mathbf{F}(y)$}}
\step{5}{\assume{w.l.o.g. $x \mathbf{R} y$}}
\begin{proof}
	\pf\ $\mathbf{R}$ is total.
\end{proof}
\step{7}{$\mathbf{F}(x) \mathbf{S} \mathbf{F}(y)$ and $\mathbf{F}(x) \neq \mathbf{F}(y)$}
\begin{proof}
	\pf\ $\mathbf{F}$ is strictly monotone.
\end{proof}
\qed
\end{proof}

\begin{props}[S without Foundation]
\label{prop:strictmonotoneinv}
For any classes $\mathbf{A}$, $\mathbf{B}$, $\leq$, $\preccurlyeq$ and $\mathbf{F}$, the following is a theorem:

Assume $\leq$ is a linear order on $\mathbf{A}$ and $\preccurlyeq$ is a linear order on $\mathbf{B}$. Assume $\mathbf{F} : \mathbf{A} \rightarrow \mathbf{B}$ and $\mathbf{F}$ is strictly monotone. For all $x,y \in \mathbf{A}$, if $\mathbf{F}(x) \prec \mathbf{F}(y)$ then $x < y$.
\end{props}

\begin{proof}
\pf
\step{1}{$\mathbf{F}(x) \neq \mathbf{F}(y)$ and $\mathbf{F}(y) \not\prec \mathbf{F}(x)$}
\begin{proof}
	\pf\ Trichotomy.
\end{proof}
\step{2}{$x \neq y$ and $y \nless x$}
\begin{proof}
	\pf\ $\mathbf{F}$ is strictly monotone.
\end{proof}
\step{3}{$x < y$}
\begin{proof}
	\pf\ Trichotomy.
\end{proof}
\qed
\end{proof}

\begin{cors}[S without Foundation]
\label{cor:orderiso}
For any classes $\mathbf{A}$, $\mathbf{B}$, $\leq$, $\preccurlyeq$ and $\mathbf{F}$, the following is a theorem:

Assume $\leq$ is a linear order on $\mathbf{A}$ and $\preccurlyeq$ is a linear order on $\mathbf{B}$. Assume $\mathbf{F} : \mathbf{A} \rightarrow \mathbf{B}$ and $\mathbf{F}$ is strictly monotone. Then $\mathbf{F}$ is an order isomorphism.
\end{cors}

\begin{props}[S without Foundation]
\label{prop:subloset}
For any classes $\mathbf{A}$, $\mathbf{B}$, $\mathbf{F}$ and $\mathbf{S}$, the following is a theorem:

Assume $\mathbf{S}$ is a linear order on $\mathbf{B}$ and $\mathbf{F} : \mathbf{A} \rightarrowtail \mathbf{B}$. Define $\mathbf{R}$ on $\mathbf{A}$ by $x \mathbf{R} y$ if and only if $\mathbf{F}(x) \mathbf{S} \mathbf{F}(y)$. Then $\mathbf{R}$ is a linear order on $\mathbf{A}$.
\end{props}

\begin{proof}
\pf
\step{1}{$\mathbf{R}$ is a partial order on $\mathbf{A}$.}
\begin{proof}
	\pf\ Proposition \ref{prop:subposet}.
\end{proof}
\step{2}{$\mathbf{R}$ is total.}
\begin{proof}
	\pf\ For all $x,y \in \mathbf{A}$ we have $\mathbf{F}(x) \mathbf{S} \mathbf{F}(y)$ or $\mathbf{F}(y) \mathbf{S} \mathbf{F}(x)$.
\end{proof}
\qed
\end{proof}

\begin{cors}[S without Foundation]
\label{cor:subloset}
For any classes $\mathbf{A}$, $\mathbf{B}$ and $\mathbf{R}$, the following is a theorem:

Assume $\mathbf{R}$ be a linear order on $\mathbf{A}$ and $\mathbf{B} \subseteq \mathbf{A}$. Then $\mathbf{R} \cap \mathbf{B}^2$ is a linear order on $\mathbf{B}$.
\end{cors}

\begin{props}[S without Foundation]
\label{prop:lexicographlinear}
For any classes $\mathbf{A}$, $\mathbf{B}$, $\mathbf{R}$ and $\mathbf{S}$, the following is a theorem:

Assume $\mathbf{R}$ is a linear order on $\mathbf{A}$ and $\mathbf{S}$ is a linear order on $\mathbf{B}$. Then the lexicographic ordering is a linear order on $\mathbf{A} \times \mathbf{B}$.
\end{props}

\begin{proof}
\pf
\step{1}{\pflet{$\leq$ be the lexicographic order on $\mathbf{A} \times \mathbf{B}$}}
\step{2}{$\leq$ is a partial order.}
\begin{proof}
	\pf\ Proposition \ref{prop:lexicposet}.
\end{proof}
\step{3}{$\leq$ is total.}
\begin{proof}
	\step{a}{\pflet{$a,a' \in \mathbf{A}$ and $b, b' \in \mathbf{B}$}}
	\step{b}{\case{$a \mathbf{R} a'$ and $a \neq a'$}}
	\begin{proof}
		\pf\ Then $(a,b) \leq (a',b')$.
	\end{proof}
	\step{c}{\case{$a = a'$}}
	\begin{proof}
		\pf\ We have $b \mathbf{S} b'$ or $b' \mathbf{S} b$, so $(a,b) \leq (a',b')$ or $(a',b') \leq (a,b)$.
	\end{proof}
	\step{d}{\case{$a' \mathbf{R} a$ and $a \neq a'$}}
	\begin{proof}
		\pf\ Then $(a',b') \leq (a,b)$.
	\end{proof}
\end{proof}
\qed
\end{proof}

\section{Well Orderings}

\begin{df}[Well Ordering]
A \emph{well ordering} on a class $\mathbf{A}$ is a well-founded linear ordering on $\mathbf{A}$.
\end{df}

\begin{prop}[S without Foundation]
Let $S$ be a well ordering of the set $B$ and $f : A \rightarrow B$ a function. Define $R$ on $A$ by $xRy$ if and only if $F(x) S F(y)$. Then $R$ well orders $A$.
\end{prop}

\begin{proof}
\pf
\step{1}{$R$ linearly orders $A$.}
\begin{proof}
	\pf\ Proposition \ref{prop:subloset}.
\end{proof}
\step{3}{Every nonempty subset of $A$ has a least element.}
\begin{proof}
	\step{a}{\pflet{$C$ be a nonempty subset of $A$.}}
	\step{b}{\pflet{$y$ be the least element of $f(C)$.}}
	\step{c}{\pick\ $x \in C$ such that $f(x) = y$.}
	\step{d}{$x$ is least in $C$.}
\end{proof}
\qed
\end{proof}

\begin{props}[S without Foundation]
\label{prop:subwoset}
For any classes $\mathbf{A}$, $\mathbf{B}$ and $\mathbf{R}$, the following is a theorem:

Assume $\mathbf{R}$ well orders $\mathbf{B}$ and $\mathbf{A} \subseteq \mathbf{B}$. Then $\mathbf{R} \cap \mathbf{A}^2$ well orders $\mathbf{A}$.
\end{props}

\begin{proof}
\pf
\step{1}{\pflet{$\mathbf{R}' = \mathbf{R} \cap \mathbf{A}^2$}}
\step{2}{$\mathbf{R}'$ linearly orders $\mathbf{A}$.}
\begin{proof}
	\pf\ Corollary \ref{cor:subloset}.
\end{proof}
\step{3}{$\mathbf{R}'$ is well founded.}
\begin{proof}
	\pf\ Proposition \ref{prop:subwellfounded}.
\end{proof}
\qed
\end{proof}

\begin{props}[SF without Foundation]
For any classes $\mathbf{A}$, $\mathbf{B}$, $\mathbf{F}$ and $\mathbf{S}$, the following is a theorem:

Assume $\mathbf{S}$ well orders $\mathbf{B}$ and $\mathbf{F} : \mathbf{A} \rightarrowtail \mathbf{B}$. Define $\mathbf{R}$ on $\mathbf{A}$ by $x \mathbf{R} y$ if and only if $\mathbf{F}(x) \mathbf{S} \mathbf{F}(y)$. Then $\mathbf{R}$ well orders $\mathbf{A}$.
\end{props}

\begin{proof}
\pf
\step{1}{$\mathbf{R}$ linearly orders $\mathbf{A}$.}
\begin{proof}
	\pf\ Proposition \ref{prop:subloset}.
\end{proof}
\step{2}{For all $t \in \mathbf{A}$ we have $\{ x \in \mathbf{A} \mid x \mathbf{R} t \wedge x \neq t \}$ is a set.}
\begin{proof}
	\step{a}{\pflet{$t \in \mathbf{A}$}}
	\step{b}{\pflet{$S = \{ y \in \mathbf{B} \mid y \mathbf{S} \mathbf{F}(t) \wedge y \neq \mathbf{F}(t) \}$}}
	\step{c}{\pflet{$P(x,y)$ be the property $\mathbf{F}(y) = x$}}
	\step{d}{For all $x \in S$ there exists at most one $y$ such that $P(x,y)$}
	\begin{proof}
		\pf\ $\mathbf{F}$ is injective.
	\end{proof}
	\step{e}{\pflet{$T = \{ y \mid \exists x \in S. P(x,y) \}$}}
	\begin{proof}
		\pf\ Axiom of Replacement.
	\end{proof}
	\step{f}{$T = \{ x \in \mathbf{A} \mid x \mathbf{R} t \wedge x \neq t \}$}
\end{proof}
\step{3}{Every nonempty subset of $\mathbf{A}$ has a least element.}
\begin{proof}
	\step{a}{\pflet{$S$ be a nonempty subset of $\mathbf{A}$.}}
	\step{x}{$\mathbf{F}(S)$ is a nonempty subset of $\mathbf{B}$}
	\begin{proof}
		\pf\ Axiom of Replacement.
	\end{proof}
	\step{b}{\pflet{$y$ be the least element of $\mathbf{F}(S)$.}}
	\step{c}{\pick\ $x \in S$ such that $\mathbf{F}(x) = y$.}
	\step{d}{$x$ is least in $S$.}
\end{proof}
\qed
\end{proof}

\begin{prop}[S without Foundation]
\label{prop:lexicogwellorder}
For any well ordered sets $A$ and $B$, the lexicographic order well orders $A \times B$.
\end{prop}

\begin{proof}
\pf
\step{1}{$A \times B$ is linearly ordered.}
\begin{proof}
	\pf\ Proposition \ref{prop:lexicographlinear}.
\end{proof}
\step{2}{Every nonempty subset of $A \times B$ has a least element.}
\begin{proof}
	\step{a}{\pflet{$S$ be a nonempty subset of $A \times B$.}}
	\step{b}{\pflet{$a$ be the least element of $\{ x \in A \mid \exists y \in B. (x,y) \in S \}$.}}
	\step{c}{\pflet{$b$ be the least element of $\{ y \in B \mid (a,y) \in S \}$.}}
	\step{d}{$(a,b)$ is least in $S$.}
\end{proof}
\qed
\end{proof}

\begin{df}[End Extension]
Let $A$ and $B$ be well ordered sets. Then $B$ is an \emph{end extension} of $A$ iff $A \subseteq B$ and:
\begin{itemize}
\item Whenever $x,y \in A$ then $x \leq_A y$ iff $x \leq_B y$.
\item Whenever $x \in A$ and $y \in B - A$ then $x < y$.
\end{itemize}
\end{df}

\begin{thm}[S without Foundation]
Let $\leq$ be a linear ordering on $A$. Assume that, for any $B \subseteq A$ such that $\forall t \in A. \seg t \subseteq B \Rightarrow t \in B$, we have $B = A$. Then $\leq$ is a well ordering on $A$.
\end{thm}

\begin{proof}
\pf
\step{1}{\pflet{$C \subseteq A$ be nonempty.}}
\step{2}{\pflet{$B = \{ t \in A \mid \forall x \in C. t < x \}$}}
\step{3}{$B \cap C = \emptyset$}
\step{4}{$B \neq A$}
\step{5}{\pick\ $t \in A$ such that $\seg t \subseteq B$ and $t \notin B$}
\step{6}{$t$ is least in $C$.}
\qed
\end{proof}

\begin{props}[S without Foundation]
For any classes $\mathbf{A}$, $\mathbf{B}$, $\mathbf{F}$, $\mathbf{G}$, $\leq$ and $\preccurlyeq$, the following is a theorem:

Assume $\leq$ well orders $\mathbf{A}$ and $\preccurlyeq$ well orders $\mathbf{B}$. Assume $\mathbf{F}$ and $\mathbf{G}$ are order isomorphisms between $\mathbf{A}$ and $\mathbf{B}$. Then $\mathbf{F} = \mathbf{G}$.
\end{props}

\begin{proof}
\pf
\step{1}{For all $x \in \mathbf{A}$, if $\forall t < x. \mathbf{F}(t) = \mathbf{G}(t)$, then $\mathbf{F}(x) = \mathbf{G}(x)$}
\begin{proof}
	\step{a}{\pflet{$x \in \mathbf{A}$}}
	\step{b}{\assume{$\forall t < x. \mathbf{F}(t) = \mathbf{G}(t)$}}
	\step{e}{$\mathbf{F}(\seg x) = \mathbf{G}(\seg x)$}
	\step{c}{$\mathbf{F}(x)$ is the least element of $\mathbf{B} - \mathbf{F}(\seg x)$}
	\step{d}{$\mathbf{G}(x)$ is the least element of $\mathbf{B} - \mathbf{G}(\seg x)$}
	\step{e}{$\mathbf{F}(x) = \mathbf{G}(x)$}
\end{proof}
\step{2}{$\forall x \in \mathbf{A}. \mathbf{F}(x) = \mathbf{G}(x)$}
\begin{proof}
	\pf\ Transfinite induction.
\end{proof}
\qed
\end{proof}

\begin{thm}[ZF without Foundation]
Let $A$ and $B$ be well ordered sets. Then one of the following holds: $A \cong B$; there exists $b \in B$ such that $A \cong \seg b$; there exists $a \in A$ such that $\seg a \cong B$.
\end{thm}

\begin{proof}
\pf
\step{1}{\pick\ $e$ that is not in $A$ or $B$.}
\step{2}{\pflet{$F : A \rightarrow B \cup \{e\}$ be the function defined by transfinite recursion thus:
\[ F(t) =
\begin{cases}
\text{the least element of } B - F(\seg t) & \text{if } B - F(\seg t) \neq \emptyset \\
e & \text{if } B - F(\seg t) = \emptyset
\end{cases} \]}}
\step{3}{\case{$e \in \ran F$}}
\begin{proof}
	\step{a}{\pflet{$t$ be least such that $F(t) = e$}}
	\step{b}{$F \restriction \seg t : \seg t \cong B$}
\end{proof}
\step{4}{\case{$\ran F = B$}}
\begin{proof}
	\pf\ We have $F : A \cong B$
\end{proof}
\step{5}{\case{$\ran F \subsetneq B$}}
\begin{proof}
	\step{a}{\pflet{$b$ be the least element of $B - \ran F$}}
	\step{b}{$F : A \cong \seg b$}
\end{proof}
\qed
\end{proof}

\chapter{Ordinal Numbers}

\section{Ordinals}

\begin{df}[Ordinal]
A set $\alpha$ is an \emph{ordinal (number)} iff:
\begin{itemize}
\item $\alpha$ is a transitive set
\item $\{ (x,y) \in \alpha^2 \mid x \in y \}$ is a strict well ordering of $\alpha$.
\end{itemize}

Given ordinals $\alpha$, $\beta$, we write $\alpha < \beta$ iff $\alpha \in \beta$, and $\alpha \leq \beta$ iff $\alpha \in \beta$ or $\alpha = \beta$.

Let $\mathbf{On}$ be the class of ordinals.
\end{df}

\begin{prop}[S]
Let $\alpha$ be a set. Then $\alpha$ is an ordinal if and only if $\alpha$ is a transitive set and, for all $x,y \in \alpha$, we have $x \in y$ or $x = y$ or $y \in x$.
\end{prop}

\begin{proof}
\pf
\step{1}{\pflet{$\alpha$ be a set.}}
\step{2}{If $\alpha$ is an ordinal then $\alpha$ is a transitive set and, for all $x,y \in \alpha$, we have $x \in y$ or $x = y$ or $y \in x$.}
\begin{proof}
	\pf\ Immediate from definitions.
\end{proof}
\step{3}{If $\alpha$ is a transitive set and $\forall x,y \in \alpha(x \in y \vee x = y \vee y \in x)$, then $\alpha$ is an ordinal.}
\begin{proof}
	\step{a}{\assume{$\alpha$ is a transitive set.}}
	\step{b}{\assume{$\forall x,y \in \alpha(x \in y \vee x = y \vee y \in x)$}}
	\step{f}{Every nonempty subset of $\alpha$ has an $\in$-least element.}
	\begin{proof}
		\pf\ Axiom of Foundation.
	\end{proof}
\end{proof}
\qed
\end{proof}

\begin{prop}[S without Foundation]
\label{prop:Ontransitive}
$\mathbf{On}$ is a transitive class. That is,
if $\beta$ is an ordinal and $\alpha \in \beta$ then $\alpha$ is an ordinal.
\end{prop}

\begin{proof}
\pf
\step{1}{\pflet{$\beta$ be an ordinal.}}
\step{2}{\pflet{$\alpha \in \beta$}}
\step{3}{$\alpha$ is a transitive set.}
\begin{proof}
	\step{a}{\pflet{$x \in y \in \alpha$}}
	\step{b}{$y \in \beta$}
	\begin{proof}
		\pf\ $\beta$ is a transitive set.
	\end{proof}
	\step{c}{$x \in \beta$}
	\begin{proof}
		\pf\ $\beta$ is a transitive set.
	\end{proof}
	\step{d}{$x \in \alpha$ or $x = \alpha$ or $\alpha \in x$}
	\step{e}{$x \neq \alpha$}
	\begin{proof}
		\pf\ We cannot have $x \in y \in x$ since $\beta$ is well-ordered by $\in$.
	\end{proof}
	\step{f}{$\alpha \notin x$}
	\begin{proof}
		\pf\ We cannot have $x \in y \in \alpha \in x$ since $\beta$ is well-ordered by $\in$.
	\end{proof}
	\step{g}{$x \in \alpha$}
\end{proof}
\step{4}{$\alpha$ is well-ordered by $\in$}
\begin{proof}
	\pf\ This holds because $\alpha \subseteq \beta$ since $\beta$ is a transitive set.
\end{proof}
\qed
\end{proof}

\begin{prop}[S without Foundation]
\label{prop:ordinalsubset}
For ordinals $\alpha$ and $\beta$, we have $\alpha \subseteq \beta$ if and only if $\alpha \leq \beta$.
\end{prop}

\begin{proof}
\pf
\step{1}{If $\alpha \leq \beta$ then $\alpha \subseteq \beta$.}
\begin{proof}
	\pf\ Since $\beta$ is a transitive set.
\end{proof}
\step{2}{If $\alpha \subseteq \beta$ then $\alpha \leq \beta$.}
\begin{proof}
	\step{a}{\assume{$\alpha \subsetneq \beta$} \prove{$\alpha \in \beta$}}
	\step{b}{\pflet{$u$ be the $\in$-least element of $\beta - \alpha$}}
	\step{c}{$u \in \beta$}
	\step{d}{$u \notin \alpha$}
	\step{e}{For all $v \in \beta$, if $v \notin \alpha$ then $v \notin u$}
	\begin{proof}
		\pf\ From \stepref{b}
	\end{proof}
	\step{f}{For all $v \in u$ we have $v \in \alpha$}
	\step{g}{For all $w \in \alpha$ we have $w \in u$}
	\begin{proof}
		\step{i}{\pflet{$w \in \alpha$}}
		\step{ii}{$w \in \beta$}
		\begin{proof}
			\pf\ \stepref{a}
		\end{proof}
		\step{iii}{$w \in u$ or $w = u$ or $u \in w$}
		\begin{proof}
			\pf\ $\beta$ is an ordinal, \stepref{c}, \stepref{ii}
		\end{proof}
		\step{iv}{$w \neq u$}
		\begin{proof}
			\pf\ \stepref{d}, \stepref{i}
		\end{proof}
		\step{v}{$u \notin w$}
		\begin{proof}
			\pf\ \stepref{d}, \stepref{i}, $\alpha$ is a transitive set.
		\end{proof}
		\step{vi}{$w \in u$}
		\begin{proof}
			\pf\ \stepref{iii}, \stepref{iv}, \stepref{v}
		\end{proof}
	\end{proof}
	\step{h}{$u = \alpha$}
	\begin{proof}
		\pf\ \stepref{f}, \stepref{g}, Axiom of Extensionality.
	\end{proof}
	\step{i}{$\alpha \in \beta$}
	\begin{proof}
		\pf\ \stepref{c}, \stepref{h}
	\end{proof}
\end{proof}
\qed
\end{proof}

\begin{prop}[S without Foundation]
\label{prop:intordinal}
The intersection of any nonempty set of ordinals is an ordinal.
\end{prop}

\begin{proof}
\pf
\step{0}{\pflet{$x$ be a nonempty set of ordinals.}}
\step{1}{$\bigcap x$ is a transitive set.}
\begin{proof}
	\pf\ Proposition \ref{prop:inttransitive}.
\end{proof}
\step{2}{$\bigcap x$ is well ordered by $\in$}
\begin{proof}
	\step{a}{\pick\ $\alpha \in x$}
	\step{b}{$\alpha$ is well ordered by $\in$.}
	\step{c}{$\bigcap x \subseteq \alpha$}
\end{proof}
\qed
\end{proof}

\begin{thm}[S without Foundation]
\label{thm:ordinaltrichotomy}
For any ordinals $\alpha$ and $\beta$, we have $\alpha < \beta$ or $\alpha = \beta$ or $\beta < \alpha$.
\end{thm}

\begin{proof}
\pf
\step{1}{\pflet{$\alpha$ and $\beta$ be ordinals.}}
\step{2}{\case{$\alpha \cap \beta = \alpha$ or $\alpha \cap \beta = \beta$}}
\begin{proof}
	\step{a}{\assume{for a contradiction $\alpha \cap \beta \subsetneq \alpha$ and $\alpha \cap \beta \subsetneq \beta$}}
	\step{b}{$\alpha \cap \beta \in \alpha$ and $\alpha \cap \beta \in \beta$}
	\begin{proof}
		\pf\ Propositions \ref{prop:ordinalsubset}, \ref{prop:intordinal}.
	\end{proof}
	\step{c}{$\alpha \cap \beta \in \alpha \cap \beta$}
	\qedstep
	\begin{proof}
		\pf\ This contradicts the fact that $\alpha$ is well ordered by $\in$.
	\end{proof}
\end{proof}
\step{3}{$\alpha \subseteq \beta$ or $\beta \subseteq \alpha$}
\step{4}{$\alpha \in \beta$ or $\alpha = \beta$ or $\beta \in \alpha$}
\begin{proof}
	\pf\ Proposition \ref{prop:ordinalsubset}.
\end{proof}
\qed
\end{proof}

\begin{cor}[S without Foundation]
The class $\mathbf{On}$ is well ordered by $<$.
\end{cor}

\begin{proof}
\pf
\step{2}{For ordinals $\alpha$, $\beta$, $\gamma$, if $\alpha < \beta < \gamma$ then $\alpha < \gamma$.}
\begin{proof}
	\pf\ Because $\gamma$ is a transitive set.
\end{proof}
\step{3}{Any nonempty set of ordinals has an $\in$-least element.}
\begin{proof}
	\step{a}{\pflet{$S$ be a nonempty set of ordinals.}}
	\step{b}{\pick\ $\alpha \in S$}
	\step{c}{\assume{w.l.o.g. $\alpha$ is not least in $S$.}}
	\step{d}{$\alpha \cap S$ is a nonempty subset of $\alpha$.}
	\step{e}{\pflet{$\beta$ be the least element of $\alpha \cap S$.}}
	\step{f}{$\beta$ is least in $S$.}
\end{proof}
\qed
\end{proof}

\begin{cor}[Burali-Forti Paradox (S without Foundation)]
The class $\mathbf{On}$ is a proper class.
\end{cor}

\begin{proof}
\pf\ If $\mathbf{On}$ is a set then it is an ordinal by Proposition \ref{prop:Ontransitive} and the previous Corollary, hence $\mathbf{On} \in \mathbf{On}$ contradicting the fact that every ordinal is well ordered by $\in$. \qed
\end{proof}

\begin{cor}[S without Foundation]
\label{cor:transords}
A set is an ordinal if and only if it is a transitive set of ordinals.
\end{cor}

\begin{prop}[S without Foundation]
For any nonempty set of ordinals $x$, we have $\bigcap x$ is the infimum of $x$ in $\mathbf{On}$.
\end{prop}

\begin{proof}
\pf\ It is an ordinal by Proposition \ref{prop:intordinal}. The fact it is the infimum is
immediate from definitions. \qed
\end{proof}

\begin{prop}[S without Foundation]
For any nonempty set of ordinals $x$, we have $\bigcup x$ is the supremum of $x$ in $\mathbf{On}$.
\end{prop}

\begin{proof}
\pf
\step{1}{$\bigcup x$ is an ordinal.}
\begin{proof}
	\step{a}{$\bigcup x$ is a transitive set.}
	\begin{proof}
		\pf\ Proposition \ref{prop:uniontransitive}.
	\end{proof}
	\step{b}{$\bigcup x$ is a set of ordinals.}
\end{proof}
\qed
\end{proof}

\begin{prop}[S]
For any set $\alpha$, we have $\alpha$ is an ordinal if and only if $\alpha$ is a transitive set of transitive sets.
\end{prop}

\begin{proof}
\pf
\step{1}{If $\alpha$ is an ordinal then $\alpha$ is a transitive set of transitive sets.}
\begin{proof}
	\pf\ Proposition \ref{prop:Ontransitive}.
\end{proof}
\step{2}{If $\alpha$ is a transitive set of transitive sets then $\alpha$ is an ordinal.}
\begin{proof}
	\step{a}{\assume{$\alpha$ is a transitive set of transitive sets.}}
	\step{b}{\assume{for a contradiction there exist $x,y \in \alpha$ such that $x \notin y$ and $x \neq y$ and $y \notin x$}}
	\step{c}{$S = \{ x \in \alpha \mid \exists y \in \alpha. x \notin y \wedge x \neq y \wedge y \notin x \}$ is nonempty}
	\step{d}{\pick\ $x \in S$ such that $x \cap S = \emptyset$}
	\begin{proof}
		\pf\ Axiom of Foundation.
	\end{proof}
	\step{e}{$T = \{ y \in \alpha \mid x \notin y \wedge x \neq y \wedge y \notin x \}$ is nonempty.}
	\step{f}{\pick\ $y \in T$ such that $y \cap T = \emptyset$}
	\begin{proof}
		\pf\ Axiom of Foundation.
	\end{proof}
	\step{g}{$x \subseteq y$}
	\begin{proof}
		\step{i}{\pflet{$z \in x$}}
		\step{ii}{$z \notin S$}
		\begin{proof}
			\pf\ \stepref{d}
		\end{proof}
		\step{iii}{$z \in y$ or $z = y$ or $y \in z$}
		\step{iv}{$z \neq y$}
		\begin{proof}
			\pf\ Since $y \notin x$ (\stepref{f}).
		\end{proof}
		\step{v}{$y \notin z$}
		\begin{proof}
			\pf\ Since $x$ is transitive and $y \notin x$.
		\end{proof}
		\step{vi}{$z \in y$}
	\end{proof}
	\step{h}{$y \subseteq x$}
	\begin{proof}
		\step{i}{\pflet{$z \in y$}}
		\step{ii}{$z \notin T$}
		\begin{proof}
			\pf\ \stepref{f}
		\end{proof}
		\step{iii}{$z \in x$ or $z = x$ or $x \in z$}
		\step{iv}{$z \neq x$}
		\begin{proof}
			\pf\ Since $x \notin y$
		\end{proof}
		\step{v}{$x \notin z$}
		\begin{proof}
			\pf\ Since $y$ is a transitive set and $x \notin y$.
		\end{proof}
		\step{vi}{$z \in x$}
	\end{proof}
	\step{i}{$x = y$}
	\begin{proof}
		\pf\ Axiom of Extensionality.
	\end{proof}
	\qedstep
	\begin{proof}
		\pf\ This contradicts \stepref{f}.
	\end{proof}
\end{proof}
\qed
\end{proof}

\begin{prop}[S without Foundation]
$\emptyset$ is an ordinal.
\end{prop}

\begin{proof}
\pf\ Vacuous. \qed
\end{proof}

\begin{df}
\[ 0 := \emptyset \]
\end{df}

\begin{df}[Successor]
For any ordinal $\alpha$, the \emph{successor} of $\alpha$ is
\[ \alpha^+ := \alpha \cup \{ \alpha \} \enspace . \]
\end{df}

\begin{prop}[S without Foundation]
For any ordinal $\alpha$, the successor $\alpha^+$ is an ordinal.
\end{prop}

\begin{proof}
\pf
\step{0}{\pflet{$\alpha$ be an ordinal.}}
\step{1}{$\alpha^+$ is a transitive set.}
\begin{proof}
	\step{a}{\pflet{$x \in y \in \alpha^+$}}
	\step{b}{$y \in \alpha$ or $y = \alpha$}
	\step{c}{$x \in \alpha$}
	\begin{proof}
		\pf\ If $y \in \alpha$ then this follows because $\alpha$ is a transitive set. If $y = \alpha$ this follows immediately.
	\end{proof}
	\step{d}{$x \in \alpha^+$}
\end{proof}
\step{2}{$\alpha^+$ is a set of ordinals.}
\qed
\end{proof}

\begin{prop}[S without Foundation]
There are no ordinals $\alpha$ and $\beta$ such that $\alpha < \beta < \alpha^+$.
\end{prop}

\begin{proof}
\pf
\step{1}{\assume{for a contradiction $\alpha \in \beta \in \alpha^+$}}
\step{2}{$\beta \in \alpha$ or $\beta = \alpha$.}
\step{3}{\case{$\beta \in \alpha$}}
\begin{proof}
	\pf\ We have $\alpha \in \beta \in \alpha$ contradicting the Axiom of Foundation.
\end{proof}
\step{4}{\case{$\beta = \alpha$}}
\begin{proof}
	\pf\ We have $\alpha \in \alpha$ contradicting the Axiom of Foundation.
\end{proof}
\qed
\end{proof}

\begin{prop}[S without Foundation]
\label{prop:Peano2}
For ordinals $\alpha$ and $\beta$, if $\alpha^+ = \beta^+$ then $\alpha = \beta$.
\end{prop}

\begin{proof}
	\pf\ If $\alpha^+ = \beta^+$ then $\alpha = \bigcup (\alpha^+) = \bigcup (\beta^+) = \beta$. \qed
\end{proof}

\begin{prop}[S without Foundation]
\label{prop:succltsucc}
For ordinals $\alpha$ and $\beta$, we have $\alpha < \beta$ if and only if $\alpha^+ < \beta^+$.
\end{prop}

\begin{proof}
\pf
\begin{align*}
\alpha < \beta & \Leftrightarrow \alpha^+ \leq \beta \\
& \Leftrightarrow \alpha^+ < \beta^+ & \qed
\end{align*}
\end{proof}

\begin{df}[Successor Ordinal]
An ordinal $\alpha$ is a \emph{successor ordinal} iff there exists an ordinal $\beta$ such that $\alpha = \beta^+$.
\end{df}

\begin{df}[Limit Ordinal]
An ordinal is a \emph{limit ordinal} iff it is neither 0 nor a successor ordinal.
\end{df}

\begin{prop}[S without Foundation]
The relation $\{ (\alpha, \beta) \in \mathbf{On}^2 \mid \alpha < \beta \}$ is well founded.
\end{prop}

\begin{proof}
\pf
\step{1}{Any nonempty set has an $\in$-minimal element.}
\begin{proof}
	\pf\ Since $\mathbf{On}$ is well ordered by $\in$.
\end{proof}
\step{2}{For any set $x$, there exists a set $u$ such that $x \in u$ and for all ordinals $\alpha$, $\beta$, if $\beta \in u$ and $\alpha < \beta$ then $\alpha \in u$.}
\begin{proof}
	\step{a}{\pflet{$x$ be a set.}}
	\step{b}{\pflet{$u = \{ x \}$ if $x$ is not an ordinal, $x^+$ if $x$ is an ordinal.}}
	\step{c}{$x \in u$}
	\step{d}{\pflet{$\alpha$, $\beta$ be ordinals.}}
	\step{e}{\assume{$\beta \in u$}}
	\step{f}{\assume{$\alpha < \beta$}}
	\step{g}{$x$ is an ordinal}
	\step{h}{$\alpha < \beta \leq x$}
	\step{i}{$\alpha \in x^+ = u$}
\end{proof}
\qed
\end{proof}

\begin{prop}[S without Foundation]
If $\lambda$ is a limit ordinal and $\beta < \lambda$ then $\beta^+ < \lambda$.
\end{prop}

\begin{proof}
\pf\ Since $\beta^+ \leq \lambda$ and $\beta^+ \neq \lambda$. \qed
\end{proof}

\begin{prop}[S without Foundation]
\label{prop:ordisoeq}
For any ordinal numbers $\alpha$ and $\beta$, if $(\alpha, \in) \cong (\beta, \in)$ then $\alpha = \beta$.
\end{prop}

\begin{proof}
\pf
\step{1}{\pflet{$f : \alpha \cong \beta$}}
\step{2}{For all $x \in \alpha$, if $\forall t < x. f(t) = t$ then $f(x) = x$}
\begin{proof}
	\step{a}{$f(x) \subseteq x$}
	\begin{proof}
		\step{i}{\pflet{$y \in f(x)$}}
		\step{ii}{$y \in \beta$}
		\step{iii}{\pick\ $t \in \alpha$ such that $f(t) = y$}
		\begin{proof}
			\pf\ $f$ is surjective.
		\end{proof}
		\step{iv}{$f(t) \in f(x)$}
		\step{v}{$t \in x$}
		\begin{proof}
			\pf\ Since $f$ is an order isomorphism.
		\end{proof}
		\step{vi}{$f(t) = t$}
		\begin{proof}
			\pf\ Induction hypothesis.
		\end{proof}
		\step{viii}{$y = t$}
		\step{ix}{$y \in x$}
	\end{proof}
	\step{b}{$x \subseteq f(x)$}
	\begin{proof}
		\step{i}{\pflet{$t \in x$}}
		\step{ii}{$f(t) \in f(x)$}
		\step{iii}{$f(t) = t$}
		\step{iv}{$t \in f(x)$}
	\end{proof}
\end{proof}
\step{3}{$\forall x \in \alpha. f(x) = x$}
\begin{proof}
	\pf\ Transfinite induction.
\end{proof}
\step{4}{$\alpha = \beta$}
\begin{proof}
	\pf\ Since $\beta = \{ f(t) \mid t \in \alpha \} = \{ t \mid t \in \alpha \} = \alpha$.
\end{proof}
\qed
\end{proof}

\begin{thm}[SF without Foundation]
\label{thm:unique_ordinal}
Every well-ordered set is isomorphic to a unique ordinal.
\end{thm}

\begin{proof}
\pf
\step{1}{For any well-ordered set $A$, there exists an ordinal $\alpha$ such that $A \cong \alpha$.}
\begin{proof}
	\step{a}{\pflet{$A$ be a well-ordered set.}}
	\step{b}{Define the function $E$ on $A$ by transfinite recursion thus:
	\[ E(t) = \{ E(x) \mid x < t \} \qquad (t \in A) \enspace . \]}
	\step{c}{\pflet{$\alpha = \{ E(x) \mid x \in A \}$}}
	\step{d}{$\alpha$ is an ordinal.}
	\begin{proof}
		\step{i}{$\alpha$ is a transitive set.}
		\begin{proof}
			\step{one}{\pflet{$x \in y \in \alpha$}}
			\step{two}{\pick\ $t \in A$ such that $y = E(t)$}
			\step{three}{$x \in E(t) = \{ E(s) \mid s < t \}$}
			\step{four}{\pick\ $s < t$ such that $x = E(s)$}
			\step{five}{$x \in \alpha$}
		\end{proof}
		\step{ii}{For all $x,y \in \alpha$ we have $x \in y$ or $x = y$ or $y \in x$}
		\begin{proof}
			\step{a}{\pflet{$x,y \in \alpha$}}
			\step{b}{\pick\ $a,b \in A$ such that $E(a) = x$ and $E(b) = y$}
			\step{c}{$a < b$ or $a = b$ or $b < a$}
			\step{d}{$x \in y$ or $x = y$ or $y \in x$}
		\end{proof}
		\step{iii}{Every nonempty subset of $\alpha$ has a least element.}
		\begin{proof}
			\step{A}{\pflet{$S$ be a nonempty subset of $\alpha$}}
			\step{B}{\pflet{$T = \{ x \in A \mid E(x) \in S \}$}}
			\step{C}{\pflet{$t$ be the least element of $T$.} \prove{$E(t)$ is least in $S$}}
			\step{D}{\pflet{$y \in S$}}
			\step{E}{\pick\ $s \in T$ such that $E(s) = y$}
			\step{F}{$t \leq s$}
			\step{G}{$x \leq y$}
		\end{proof}
	\end{proof}
	\step{f}{$E$ is surjective.}
	\begin{proof}
		\pf\ By definition of $\alpha$.
	\end{proof}
	\step{g}{$E$ is strictly monotone.}
	\begin{proof}
		\pf\ If $s < t$ then $E(s) \in E(t)$ by definition of $E(t)$.
	\end{proof}
	\qedstep
	\begin{proof}
		\pf\ Corollary \ref{cor:orderiso}.
	\end{proof}
\end{proof}
\step{2}{For any ordinals $\alpha$ and $\beta$, if $\alpha \cong \beta$ then $\alpha = \beta$.}
\begin{proof}
	\pf\ Proposition \ref{prop:ordisoeq}.
\end{proof}
\qed
\end{proof}

\begin{prop}[SF without Foundation]
\label{prop:subsetleq}
Let $\alpha$ be an ordinal and $S \subseteq \alpha$. Then $S$ is well-ordered by $\in$ and the ordinal of $(S, \in)$ is $\leq \alpha$.
\end{prop}

\begin{proof}
\pf
\step{1}{$S$ is well ordered by $\in$.}
\step{2}{\pflet{$\beta$ be the ordinal of $(S, \in)$}}
\step{3}{\pflet{$E : S \approx \beta$ be the unique isomorphism.}}
\step{4}{$\forall \gamma \in S. E(\gamma) \leq \gamma$}
\begin{proof}
	\step{o}{\pflet{$\gamma \in S$}}
	\step{a}{\assume{as transfinite induction hypothesis $\forall \delta < \gamma. E(\delta) \leq \delta$}}
	\step{b}{$E(\gamma)$ is the least element of $\beta$ that is greater than $E(\delta)$ for all $\delta < \gamma$}
	\step{c}{$\gamma$ is greater than $E(\delta)$ for all $\delta < \gamma$}
	\step{d}{$E(\gamma) \leq \gamma$}
\end{proof}
\step{5}{$\beta \leq \alpha$}
\begin{proof}
	\step{a}{$\forall \gamma < \beta. \gamma < \alpha$}
	\begin{proof}
		\step{i}{\pflet{$\gamma < \beta$}}
		\step{ii}{\pick\ $\delta \in S$ such that $E(\delta) = \gamma$}
		\step{iii}{$\gamma = E(\delta) \leq \delta < \alpha$}
	\end{proof}
\end{proof}
\qed
\end{proof}

\begin{lm}[SF without Foundation]
\label{lm:endextleq}
Let $A$ and $B$ be well-ordered sets. If $B$ is an end extension of $A$ then the ordinal of $A$ is $\leq$ the ordinal of $B$.
\end{lm}

\begin{proof}
\pf
\step{1}{\pflet{$\alpha$ be the ordinal of $A$ and $\beta$ the ordinal of $B$.}}
\step{2}{\pflet{$E_A : A \cong \alpha$ and $E_B : B \cong \beta$ be the canonical isomorphisms.}}
\step{3}{$\forall a \in A. E_A(a) = E_B(a)$}
\begin{proof}
	\step{a}{\pflet{$a \in A$}}
	\step{b}{\assume{as transfinite induction hypothesis $\forall x < a. E_A(x) = E_B(x)$}}
	\step{c}{$E_A(a)$ is the least ordinal that is greater than $E_A(x)$ for all $x < a$}
	\step{d}{$E_B(a)$ is the least ordinal that is greater than $E_B(x)$ for all $x < b$}
	\step{e}{$\{ x \in A \mid x <_A a \} = \{ x \in B \mid x <_B a \}$}
	\step{f}{$E_A(a) = E_B(a)$}
\end{proof}
\step{4}{$\alpha \subseteq \beta$}
\step{5}{$\alpha \leq \beta$}
\qed
\end{proof}

\begin{lm}[SF without Foundation]
\label{lm:union_of_end_extensions}
Let $\mathcal{C}$ be a set of well ordered sets such that, for any $A,B \in \mathcal{C}$, we have that one of $A$ and $B$ is an end extension of the other. Let $W = \bigcup \mathcal{C}$ under $x \leq y$ iff there exists $A \in W$ such that $x,y \in A$ and $x \leq y$. Then $W$ is a well ordered set whose ordinal is the supremum of the ordinals of the members of $\mathcal{C}$.
\end{lm}

\begin{proof}
\pf
\step{1}{$\leq$ is reflexive on $W$.}
\begin{proof}
	\step{a}{\pflet{$x \in W$}}
	\step{b}{\pick\ $A \in W$ such that $x \in A$.}
	\step{c}{$x \leq x$}
\end{proof}
\step{2}{$\leq$ is antisymmetric on $W$.}
\begin{proof}
	\step{a}{\pflet{$x,y \in W$}}
	\step{b}{\assume{$x \leq y$ and $y \leq x$}}
	\step{c}{\pick\ $A \in W$ such that $x,y \in A$ and $x \leq_A y$, and $B \in W$ such that $x,y \in B$ and $y \leq_B x$}
	\step{d}{\assume{w.l.o.g. $B$ is an end extension of $A$}}
	\step{e}{$x \leq_B y$ and $y \leq_B x$}
	\step{f}{$x = y$}
\end{proof}
\step{3}{$\leq$ is transitive on $W$.}
\begin{proof}
	\step{a}{\assume{$x \leq y \leq z$}}
	\step{b}{\pick\ $A,B \in W$ such that $x \leq_A y$ and $y \leq_B z$}
	\step{c}{\case{$A$ is an end extension of $B$.}}
	\begin{proof}
		\step{i}{$x \leq_A y$ and $y \leq_A z$}
		\step{2}{$x \leq_A z$}
		\step{3}{$x \leq z$}
	\end{proof}
	\step{d}{\case{$B$ is an end extension of $A$.}}
	\begin{proof}
		\pf\ Similar.
	\end{proof}
\end{proof}
\step{4}{$\leq$ is total on $W$.}
\begin{proof}
	\step{a}{\pflet{$x,y \in W$}}
	\step{b}{\pick\ $A,B \in \mathcal{C}$ such that $x \in A$ and $y \in B$}
	\step{c}{\assume{w.l.o.g. $B$ is an end extension of $A$}}
	\step{d}{$x \leq_B y$ or $y \leq_B x$}
	\step{e}{$x \leq_W y$ or $y \leq_W x$}
\end{proof}
\step{5}{Every nonempty subset of $W$ has a least element.}
\begin{proof}
	\step{a}{\pflet{$S$ be a nonempty subset of $W$}}
	\step{b}{\pick\ $s \in S$}
	\step{c}{\pick\ $A \in \mathcal{C}$ such that $s \in A$}
	\step{d}{\pflet{$a$ be the $\leq_A$-least element of $S \cap A$} \prove{$a$ is least in $S$}}
	\step{e}{\pflet{$x \in S$} \prove{$a \leq x$}}
	\step{f}{\pick\ $B \in \mathcal{C}$ such that $x \in B$}
	\step{g}{\case{$A$ is an end extension of $B$}}
	\begin{proof}
		\step{i}{$a \leq_A x$}
		\step{ii}{$a \leq x$}
	\end{proof}
	\step{h}{\case{$B$ is an end extension of $A$}}
	\begin{proof}
		\step{i}{\case{$x \in A$}}
		\begin{proof}
			\step{one}{$a \leq_A x$}
			\step{two}{$a \leq x$}
		\end{proof}
		\step{ii}{\case{$x \in B - A$}}
		\begin{proof}
			\step{one}{$a \leq_B x$}
			\step{two}{$a \leq x$}
		\end{proof}
	\end{proof}
\end{proof}
\step{6}{For all $A \in \mathcal{C}$, $W$ is an end extension of $A$.}
\begin{proof}
	\step{a}{For all $x, y \in A$, we have $x \leq_A y$ if and only if $x \leq_W y$}
	\begin{proof}
		\step{i}{\pflet{$x,y \in A$}}
		\step{ii}{If $x \leq_A y$ then $x \leq_W y$}
		\begin{proof}
			\pf\ Immediate from definitions.
		\end{proof}
		\step{iii}{If $x \leq_W y$ then $x \leq_A y$}
		\begin{proof}
			\step{one}{\assume{$x \leq_W y$}}
			\step{two}{\pick\ $B \in \mathcal{C}$ such that $x \leq_B y$}
			\step{three}{\case{$A$ is an end extension of $B$}}
			\begin{proof}
				\pf\ Then $x \leq_A y$.
			\end{proof}
			\step{four}{\case{$B$ is an end extension of $A$}}
			\begin{proof}
				\pf\ Then $x \leq_A y$.
			\end{proof}
		\end{proof}
	\end{proof}
	\step{b}{For all $x \in A$ and $y \in W - A$ we have $x < y$}
	\begin{proof}
		\step{i}{\pflet{$x \in A$ and $y \in W - A$}}
		\step{ii}{\pick\ $B \in \mathcal{C}$ such that $y \in B$}
		\step{iii}{$B$ is an end extension of $A$}
		\step{iv}{$x <_B y$}
		\step{v}{$x <_W y$}
	\end{proof}
\end{proof}
\step{7}{For all $A \in \mathcal{C}$, the ordinal of $A$ is $\leq$ the ordinal of $W$.}
\begin{proof}
	\pf\ Lemma \ref{lm:endextleq}.
\end{proof}
\step{7}{For any ordinal $\alpha$, if for all $A \in \mathcal{C}$ the ordinal of $A$ is $\leq \alpha$, then the ordinal of $W$ is $\leq \alpha$.}
\begin{proof}
	\step{a}{\pflet{$\alpha$ be an ordinal.}}
	\step{b}{\assume{for all $A \in \mathcal{C}$, the ordinal of $A$ is $\leq \alpha$}}
	\step{c}{\pflet{$\beta$ be the ordinal of $W$}}
	\step{d}{\pflet{$E : W \approx \beta$ be the canonical isomorphism.}}
	\step{e}{\assume{for a contradiction $\alpha < \beta$}}
	\step{f}{\pflet{$a \in W$ be the element with $E(a) = \alpha$}}
	\step{g}{\pick\ $A \in \mathcal{C}$ such that $a \in A$}
	\step{h}{\pflet{$\gamma$ be the ordinal of $A$ and $E_A : A \cong \gamma$ be the canonical isomorphism.}}
	\step{i}{For all $x \in A$ we have $E_A(x) = E(x)$}
	\begin{proof}
		\pf\ Transfinite induction on $x$.
	\end{proof}
	\step{j}{$E_A(a) = \alpha$}
	\step{k}{$\alpha < \gamma$}
	\qedstep
	\begin{proof}
		\pf\ This contradicts \stepref{b}.
	\end{proof}
\end{proof}
\qed
\end{proof}

\section{Natural Numbers}

\begin{df}[Natural Number]
An ordinal $n$ is a \emph{natural number} iff either it is 0, or it is a successor ordinal and every element of $n$ is either 0 or a successor ordinal.

We write $\omega$ or $\mathbb{N}$ for the class of natural numbers.
\end{df}

\begin{thms}[Mathematical Induction (S without Foundation)]
For any predicate $P(x)$, the following is a theorem:

Assume $P(0)$. Assume that, for any ordinal $\alpha$, if $P(\alpha)$ then $P(\alpha^+)$. Then, for every natural number $n$, we have $P(n)$.
\end{thms}

\begin{proof}
\pf
\step{1}{\assume{$n$ is a natural number such that $\neg P(n)$}}
\step{2}{\pflet{$y = \{ \gamma \in n^+ \mid \neg P(\gamma) \}$}}
\step{3}{$y \neq \emptyset$}
\step{4}{\pick\ $\gamma_0 \in y$ such that $y \cap \gamma_0 = \emptyset$}
\step{5}{\case{$\gamma_0 = 0$}}
\begin{proof}
	\pf\ Then $\neg P(0)$
\end{proof}
\step{6}{\case{$\gamma_0 \neq 0$}}
\begin{proof}
	\step{a}{\pick\ $\gamma_1$ such that $\gamma_0 = \gamma_1^+$}
	\begin{proof}
		\pf\ Either $\gamma_0 = n$ or $\gamma_0 \in n$, and in either case $\gamma_0$ is a successor because $n$ is a natural number.
	\end{proof}
	\step{b}{$\gamma_1 \notin y$}
	\begin{proof}
		\pf\ From \stepref{4} since $\gamma_1 \in \gamma_0$.
	\end{proof}
	\step{c}{$\gamma_1 \in n^+$}
	\begin{proof}
		\pf\ Since $\gamma_1 \in \gamma_0 \in n^+$ and $n^+$ is a transitive set.
	\end{proof}
	\step{d}{$P(\gamma_1)$ and $\neg P(\gamma_1^+)$}
	\begin{proof}
		\pf\ \stepref{2}, \stepref{b}, \stepref{c}
	\end{proof}
\end{proof}
\qed
\end{proof}

\begin{thm}[Z without Foundation]
The class of natural numbers $\omega$ is a set.
\end{thm}

\begin{proof}
\pf
\step{1}{\pick\ a set $I$ such that $\emptyset \in I$ and $\forall x \in I. x \cup \{x\} \in I$.}
\begin{proof}
	\pf\ Axiom of Infinity.
\end{proof}
\step{2}{Every natural number is in $I$.}
\begin{proof}
	\pf\ By induction.
\end{proof}
\qedstep
\begin{proof}
	\pf\ By an Axiom of Comprehension.
\end{proof}
\qed
\end{proof}

\begin{thm}[Z without Foundation]
$\omega$ is a limit ordinal.
\end{thm}

\begin{proof}
\pf
\step{1}{$\omega$ is a transitive set.}
\begin{proof}
	\step{a}{\pflet{$m \in n \in \omega$} \prove{$m$ is a natural number.}}
	\step{b}{\assume{$m \neq 0$}}
	\step{c}{$m$ is a successor ordinal.}
	\begin{proof}
		\pf\ Since $m \in n$ and $n$ is a natural number.
	\end{proof}
	\step{d}{Every element of $m$ is either 0 or a successor ordinal.}
	\begin{proof}
		\step{i}{\pflet{$x \in m$}}
		\step{ii}{$x \in n$}
		\begin{proof}
			\pf\ Since $n$ is a transitive set.
		\end{proof}
		\step{iii}{$x = 0$ or $x$ is a successor ordinal.}
	\end{proof}
\end{proof}
\step{2}{$\omega$ is a set of ordinals.}
\step{3}{$\omega \neq 0$}
\begin{proof}
	\pf\ Since $0 \in \omega$.
\end{proof}
\step{4}{$\omega$ is not a successor ordinal.}
\begin{proof}
	\step{a}{\assume{for a contradiction $\omega = n^+$}}
	\step{b}{$n$ is a natural number.}
	\step{c}{$\omega$ is a natural number.}
	\step{d}{$\omega \in \omega$}
	\qedstep
	\begin{proof}
		\pf\ This contradicts the Axiom of Foundation.
	\end{proof}
\end{proof}
\qed
\end{proof}

\begin{df}[Finite]
A set is \emph{finite} iff it is equinumerous to some natural number; otherwise it is \emph{infinite}.
\end{df}

\begin{df}[Denumerable]
A set is \emph{denumerable} iff it is equinumerous with $\mathbb{N}$.
\end{df}

\begin{thm}[Pigeonhole Principle (S without Foundation)]
No natural number is equinumerous to a proper subset of itself.
\end{thm}

\begin{proof}
\pf
\step{1}{\pflet{$P(n)$ be the property: any one-to-one function $n \rightarrow n$ is surjective.}}
\step{2}{$P(0)$}
\begin{proof}
	\pf\ The only function $0 \rightarrow 0$ is injective.
\end{proof}
\step{3}{For every natural number $n$, if $P(n)$ then $P(n+1)$.}
\begin{proof}
	\step{a}{\assume{$P(n)$}}
	\step{b}{\pflet{$f$ be a one-to-one function $n+1 \rightarrow n+1$}}
	\step{c}{$f \restriction n$ is a one-to-one function $n \rightarrow n+1$}
	\step{d}{\case{$n \notin ran f$}}
	\begin{proof}
		\step{i}{$f \restriction n : n \rightarrow n$}
		\step{ii}{$\ran (f \restriction n) = n$}
		\step{iii}{$f(n) = n$}
		\begin{proof}
			\pf\ \stepref{a}.
		\end{proof}
		\step{iv}{$\ran f = n+1$}
	\end{proof}
	\step{e}{\case{$n \in \ran f$}}
	\begin{proof}
		\step{i}{\pick\ $p \in n$ such that $f(p) = n$}
		\step{ii}{\pflet{$\hat{f} : n \rightarrow n$ be the function
		\begin{align*}
		\hat{f}(p) & = f(n) \\
		\hat{f}(x) & = f(x) & (x \neq p)
		\end{align*}}}
		\step{iii}{$\hat{f}$ is one-to-one}
		\step{iv}{$\ran \hat{f} = n$}
		\begin{proof}
			\pf\ \stepref{a}
		\end{proof}
		\step{v}{$\ran f = n + 1$}
	\end{proof}
\end{proof}
\step{4}{For every natural number $n$, $P(n)$.}
\qed
\end{proof}

\begin{cor}[S without Foundation]
Let $m$ and $n$ be natural numbers. Then $m \preccurlyeq n$ if and only if $m \leq n$.
\end{cor}

\begin{cor}
$\mathbb{N}$ is infinite.
\end{cor}

\section{Transitive Closure of a Set}

\begin{df}[Transitive Closure]
Let $x$ be a set. The \emph{transitive closure} of $x$ is the class
\[ \mathrm{TC}(x) = \bigcap \{ y \mid y \text{ is a transitive set} \wedge x \in y \} \enspace . \]
\end{df}

\begin{thm}[ZF without Foundation]
For any set $x$, the transitive closure $\mathrm{TC}(x)$ is a set.
\end{thm}

\begin{proof}
\pf
\step{1}{\pflet{$E(n,y,x)$ be the predicate: There exists a function $f : n^+ \rightarrow \mathbf{V}$ such that $f(0) = y$ and $f(n) = x$ and $\forall m \in n. f(m) \in f(m^+)$}}
\step{2}{$\forall x,y. E(0,y,x) \Leftrightarrow y = x$}
\step{3}{\pflet{$x$ be a set.}}
\step{3}{\pflet{$\mathbf{W} = \{ y \mid \exists n \in \mathbb{N}. E(n,y,x) \}$}}
\step{4}{$\mathbf{W}$ is a set.}
\begin{proof}
	\step{a}{For $n \in \mathbb{N}$, \pflet{$\mathbf{W}_n = \{ y \mid E(n,y,x) \}$}}
	\step{b}{For all $n \in \mathbb{N}$, $\mathbf{W}_n$ is a set.}
	\begin{proof}
		\step{i}{$\mathbf{W}_0$ is a set.}
		\begin{proof}
			\pf\ $\mathbf{W}_0 = \{x\}$.
		\end{proof}
		\step{ii}{For all $n \in \mathbb{N}$, if $\mathbf{W}_n$ is a set then $\mathbf{W}_{n^+}$ is a set.}
		\begin{proof}
			\step{one}{$\mathbf{W}_{n^+} = \bigcup \mathbf{W}_n$}
			\begin{proof}
				\step{A}{$\mathbf{W}_{n^+} \subseteq \bigcup \mathbf{W}_n$}
				\begin{proof}
					\step{one}{\pflet{$y \in \mathbf{W}_{n^+}$}}
					\step{two}{$E(n^+,y,x)$}
					\step{three}{\pick\ $f : n^{++} \rightarrow \mathbf{V}$ such that $f(0) = y$ and $f(n^+) = x$ and $\forall m \in n^+. f(m) \in f(m^+)$}
					
					\step{four}{$y \in f(1)$ and $E(n,f(1),x)$}
					\begin{proof}
						\pf\ Define $g : n^+ \rightarrow \mathbf{V}$ by $g(m) = f(m^+)$. Then $g(0) = f(1)$ and $g(n) = x$ and $\forall m \in n. g(m) \in g(m^+)$.
					\end{proof}
					\step{five}{$y \in f(1) \in \mathbf{W}_n$}
				\end{proof}
				\step{B}{$\bigcup \mathbf{W}_n \subseteq \mathbf{W}_{n^+}$}
				\begin{proof}
					\step{one}{\pflet{$y \in z \in \mathbf{W}_n$}}
					\step{two}{\pick\ $f : n^+ \rightarrow \mathbf{V}$ such that $f(0) = z$ and $f(n) = x$ and $\forall m \in n. f(m) \in f(m^+)$}
					\step{three}{\pflet{$g : n^{++} \rightarrow \mathbf{V}$ be the function defined by $g(0) = y$, $g(m^+) = f(m)$}}
					\step{four}{$g(0) = y$ and $g(n^+) = x$ and $\forall m \in n^+. f(m) \in f(m^+)$}
					\step{five}{$E(n^+,y,x)$}
				\end{proof}
			\end{proof}
		\end{proof}
	\end{proof}
	\step{c}{$\mathbf{W} = \bigcup \{ \mathbf{W}_n \mid n \in \mathbb{N} \}$}
	\step{d}{$\{ \mathbf{W}_n \mid n \in \mathbb{N} \}$ is a set.}
	\begin{proof}
		\pf\ Axiom of Replacement.
	\end{proof}
	\step{e}{$\mathbf{W}$ is a set.}
	\begin{proof}
		\pf\ Axiom of Union.
	\end{proof}
\end{proof}
\step{4}{$\mathbf{W}$ is a transitive set.}
\begin{proof}
	\step{a}{\pflet{$y \in z \in \mathbf{W}$}}
	\step{b}{\pick\ $n \in \mathbb{N}$ and $f : n^+ \rightarrow \mathbf{V}$ such that $f(0) = z$, $f(n) = x$ and $\forall m \in n. f(m^+) \in f(m)$}
	\step{c}{\pflet{$g : n^{++} \rightarrow \mathbf{V}$ be the function with $g(0) = y$ and $g(m^+) = f(m)$ for $m \in n$}}
	\step{d}{$g(0) = y$ and $g(n^+) = x$ and $\forall m \in n^{++}. g(m^+) \in g(m)$}
	\step{e}{$y \in \mathbf{W}$}
\end{proof}
\step{5}{$x \in \mathbf{W}$}
\begin{proof}
	\step{a}{\pflet{$f : 0^+ \rightarrow \mathbf{V}$ be the function $f = \{(0,x)\}$}}
	\step{b}{$f(0) = x$ and $f(0) = x$ and vacuously $\forall m \in 0. f(m) \in f(m^+)$} 
	\step{c}{$E(0,x,x)$}
\end{proof}
\step{6}{For any set $w'$, if $w'$ is a transitive set that contains $x$ then $\mathbf{W} \subseteq w'$}
\begin{proof}
	\step{a}{\pflet{$w'$ be a transitive set that contains $x$}}
	\step{b}{$\forall n \in \mathbb{N}. \forall y. E(n,y,x) \Rightarrow y \in w'$}
	\begin{proof}
		\step{i}{$\forall y. E(0,y,x) \Rightarrow y \in w'$}
		\begin{proof}
			\step{a}{\pflet{$y$ be a set.}}
			\step{b}{\assume{$E(0,y,x)$}}
			\step{c}{\pick\ a function $f : 0^+ \rightarrow \mathbf{V}$ such that $f(0) = y$ and $f(0) = x$ and $\forall m \in 0. f(m) \in f(m^+)$}
			\step{d}{$x = y$}
			\begin{proof}
				\pf\ $x = f(0) = y$
			\end{proof}
			\step{e}{$y \in w'$}
			\begin{proof}
				\pf\ \stepref{a}
			\end{proof}
		\end{proof}
		\step{ii}{For all $n \in \mathbb{N}$, if $\forall y. E(n,y,x) \Rightarrow y \in w'$, then $\forall y. E(n^+,y,x) \Rightarrow y \in w'$}
		\begin{proof}
			\step{a}{\pflet{$n \in \mathbb{N}$}}
			\step{b}{\assume{$\forall y. E(n,y,x) \Rightarrow y \in w'$}}
			\step{c}{\pflet{$y$ be a set.}}
			\step{d}{\assume{$E(n^+,y,x)$}}
			\step{e}{\pick\ $f : n^{++} \rightarrow \mathbf{V}$ such that $f(0) = y$ and $f(n^+) = x$ and $\forall m \in n^+. f(m) \in f(m^+)$}
			\step{f}{\pflet{$g : n^+ \rightarrow \mathbf{V}$ be the function $g(m) = f(m^+)$}}
			\step{g}{$g(0) = f(0^+)$ and $g(n) = x$ and $\forall m \in n. g(m) \in g(m^+)$}
			\step{h}{$E(n,f(0^+),x)$}
			\step{ii}{$f(0^+) \in w'$}
			\begin{proof}
				\pf\ \stepref{e}
			\end{proof}
			\step{j}{$y \in f(0^+)$}
			\begin{proof}
				\pf\ $y = f(0) \in f(0^+)$ by \stepref{e}.
			\end{proof}
			\step{k}{$y \in w'$}
			\begin{proof}
				\pf\ From \stepref{ii} and \stepref{j} since $w'$ is transitive (\stepref{a}).
			\end{proof}
		\end{proof}
	\end{proof}
	\step{c}{$\mathbf{W} \subseteq w'$}
\end{proof}
\qed
\end{proof}

\begin{prop}[S without Foundation]
The transitive closure of an ordinal $\alpha$ is $\alpha^+$.
\end{prop}

\begin{proof}
\pf
\step{1}{$\alpha^+$ is a transitive set that contains $\alpha$.}
\step{2}{For any transitive set $x$, if $\alpha \in x$ then $\alpha^+ \subseteq x$.}
\qed
\end{proof}

\section{The Well-Ordering Theorem and Zorn's Lemma}

\begin{thm}[Hartogs (SF without Foundation)]
For any set $A$, there exists an ordinal not dominated by $A$.
\end{thm}

\begin{proof}
\pf
\step{1}{\pflet{$\alpha$ be the class of all ordinals $\beta$ such that $\beta \preccurlyeq A$} \prove{$\alpha$ is a set.}}
\step{2}{\pflet{$W = \{ (B, R) \mid B \subseteq A, R \text{ is a well ordering on } B \}$}}
\step{3}{$\alpha$ is the class of the ordinals of the elements of $W$.}
\begin{proof}
	\step{a}{For all $(B,R) \in W$, the ordinal of $(B,R)$ is in $\alpha$.}
	\begin{proof}
		\step{i}{\pflet{$(B,R) \in W$}}
		\step{ii}{\pflet{$\beta$ be the ordinal of $(B,R)$}}
		\step{iii}{\pflet{$E : B \cong \beta$ be the canonical isomorphism.}}
		\step{iv}{\pflet{$i : B \hookrightarrow A$ be the inclusion}}
		\step{v}{$i \circ E^{-1}$ is an injection $\beta \rightarrow A$}
		\step{vi}{$\beta \in \alpha$}
	\end{proof}
	\step{b}{For all $\beta \in \alpha$, there exists $(B,R) \in W$ such that $\beta$ is the ordinal number of $(B,R)$.}
	\begin{proof}
		\step{i}{\pflet{$\beta \in \alpha$}}
		\step{ii}{\pick\ an injection $f : \beta \rightarrow A$}
		\step{iii}{Define $\leq$ on $\ran f$ by $f(x) \leq f(y)$ iff $x \leq y$}
		\step{iv}{$(\ran f, \leq) \in W$}
		\step{v}{$\beta$ is the ordinal number of $(\ran f, \leq)$}
	\end{proof}
\end{proof}
\step{4}{$\alpha$ is a set.}
\begin{proof}
	\pf\ By an Axiom of Replacement.
\end{proof}
\step{5}{$\alpha$ is an ordinal.}
\begin{proof}
	\pf\ It is a transitive set of ordinals.
\end{proof}
\step{6}{$\alpha \not\preccurlyeq A$}
\begin{proof}
	\pf\ Since $\alpha \notin \alpha$.
\end{proof}
\qed
\end{proof}

\begin{thm}[Numeration Theorem (SFC without Foundation)]
Every set is equinumerous with some ordinal.
\end{thm}

\begin{proof}
\pf
\step{1}{\pflet{$A$ be any set.}}
\step{2}{\pick\ an ordinal $\alpha$ not dominated by $A$.}
\step{3}{\pick\ a choice function $G$ for $A$.}
\step{4}{\pick\ $e \notin A$}
\step{5}{\pflet{$F : \alpha \rightarrow A \cup \{e\}$ by transfinite recursion:
\[ F(\gamma) = \begin{cases}
G(A - F(\{ \delta \mid \delta < \gamma \}) & \text{if } A - F(\{ \delta \mid \delta < \gamma \}) \neq \emptyset \\
e & \text{if } A - F(\{\delta \mid \delta < \gamma \}) = \emptyset
\end{cases} \]}}
\step{6}{$e \in \ran F$}
\begin{proof}
	\step{a}{\assume{for a contradiction $e \notin \ran F$}}
	\step{b}{$F$ is an injection $\alpha \rightarrow A$.}
	\begin{proof}
		\step{i}{\pflet{$\beta, \gamma \in \alpha$ with $\beta \neq \gamma$} \prove{$F(\beta) \neq F(\gamma)$}}
		\step{ii}{\assume{w.l.o.g. $\beta < \gamma$}}
		\step{iii}{$F(\gamma) \in A - F(\{\delta \mid \delta < \gamma\})$}
		\step{iv}{$F(\gamma) \notin F(\{\delta \mid \delta < \gamma \})$}
		\step{v}{$F(\gamma) \neq F(\beta)$}
	\end{proof}
	\qedstep
	\begin{proof}
		\pf\ This contradicts \stepref{2}.
	\end{proof}
\end{proof}
\step{7}{\pflet{$\delta$ be least such that $F(\delta) = e$}}
\step{8}{$F \restriction \delta : \delta \approx A$}
\end{proof}

\begin{thm}[Well-Ordering Theorem (SFC without Foundation)]
Any set can be well ordered.
\end{thm}

\begin{proof}
\pf
\step{1}{\pick\ an ordinal $\delta$ and a bijection $F : A \approx \delta$}
\step{9}{Define $\leq$ on $A$ by $F(x) \leq F(y)$ iff $x \leq y$ for $x,y \in \delta$}
\step{10}{$\leq$ is a well ordering on $A$.}
\qed
\end{proof}

\begin{thm}[Zorn's Lemma (SFC without Foundation)]
Let $\mathcal{A}$ be a set such that, for every chain $\mathcal{B} \subseteq \mathcal{A}$, we have $\bigcup \mathcal{B} \in \mathcal{A}$. Then $\mathcal{A}$ has a maximal element.
\end{thm}

\begin{proof}
\pf
\step{1}{\pick\ a well ordering $<$ on $\mathcal{A}$.}
\step{2}{\pflet{$F : \mathcal{A} \rightarrow 2$ be the function defined by transfinite recursion by:
\[ F(A) = \begin{cases}
1 & \text{if $A$ includes every set $B < A$ for which $F(B) = 1$} \\
0 & \text{otherwise}
\end{cases} \]}}
\step{3}{\pflet{$\mathcal{C} = \{ A \in \mathcal{A} \mid F(A) = 1 \}$} \prove{$\bigcup \mathcal{C}$ is a maximal element of $\mathcal{A}$}}
\step{4}{For all $A \in \mathcal{A}$, we have $A \in \mathcal{C}$ iff $\forall B < A. B \in \mathcal{C} \Rightarrow B \subseteq A$}
\step{5}{$\mathcal{C}$ is a chain.}
\begin{proof}
	\step{a}{\pflet{$A,A' \in \mathcal{C}$}}
	\step{b}{\assume{w.l.o.g. $A \leq A'$}}
	\step{c}{$A \subseteq A'$}
	\begin{proof}
		\pf\ By \stepref{4}
	\end{proof}
\end{proof}
\step{6}{$\bigcup \mathcal{C} \in \mathcal{A}$}
\step{7}{$\bigcup \mathcal{C}$ is maximal in $\mathcal{A}$.}
\begin{proof}
	\step{a}{\pflet{$A \in \mathcal{A}$ and $\bigcup \mathcal{C} \subseteq A$}}
	\step{b}{$A \in \mathcal{C}$}
	\begin{proof}
		\pf\ By \stepref{4} since $\forall B \in \mathcal{C}. B \subseteq A$.
	\end{proof}
	\step{c}{$A \subseteq \bigcup \mathcal{C}$}
	\step{d}{$A = \bigcup \mathcal{C}$}
\end{proof}
\qed
\end{proof}

\begin{prop}[Teichm\"{u}ller-Tukey Lemma (SFC without Foundation)]
Let $\mathcal{A}$ be a nonempty set such that, for every $B$, we have $B \in \mathcal{A}$ if and only if every finite subset of $B$ is a member of $\mathcal{A}$. Then $\mathcal{A}$ has a maximal element.
\end{prop}

\begin{proof}
\pf
\step{1}{For every chain $\mathcal{B} \subseteq \mathcal{A}$, we have $\bigcup \mathcal{B} \in \mathcal{A}$}
\begin{proof}
	\step{a}{\pflet{$\mathcal{B} \subseteq \mathcal{A}$ be a chain.}}
	\step{b}{Every finite subset of $\bigcup \mathcal{B}$ is a member of $\mathcal{A}$.}
	\begin{proof}
		\step{i}{\pflet{$C$ be a finite subset of $\bigcup \mathcal{B}$.}}
		\step{ii}{\pick\ $B \in \mathcal{B}$ such that $C \subseteq B$.}
		\step{iii}{$B \in \mathcal{A}$}
		\step{iv}{Every finite subset of $B$ is in $\mathcal{A}$.}
		\step{v}{$C \in \mathcal{A}$}
	\end{proof}
	\step{c}{$\bigcup \mathcal{B} \in \mathcal{A}$.}
\end{proof}
\qedstep
\begin{proof}
	\pf\ Zorn's lemma.
\end{proof}
\qed
\end{proof}

\begin{thms}[SF without Foundation]
\label{thms:wellorderclass}
For any class $\mathbf{A}$, there exists a class $\mathbf{F}$ such that the following is a theorem:

If $\mathbf{A}$ is a proper class of ordinals, then $\mathbf{F} : \mathbf{On} \rightarrow \mathbf{A}$ is an order isomorphism.
\end{thms}

\begin{proof}
\pf
\step{1}{Define $\mathbf{F} : \mathbf{On} \rightarrow \mathbf{A}$ by transfinite recursion as follows: $\mathbf{F}(\alpha)$ is the least element of $\mathbf{A}$ that is different from $\mathbf{F}(\beta)$ for all $\beta < \alpha$.}
\step{2}{For all $\alpha, \beta \in \mathbf{On}$, if $\alpha < \beta$ then $\mathbf{F}(\alpha) < \mathbf{F}(\beta)$}
\begin{proof}
	\pf\ We have $\mathbf{F}(\alpha) \neq \mathbf{F}(\beta)$ by the definition of $\mathbf{F}(\beta)$, and $\mathbf{F}(\beta) \nless \mathbf{F}(\alpha)$ by the leastness of $\mathbf{F}(\alpha)$.
\end{proof}
\step{3}{$\mathbf{F}$ is surjective.}
\begin{proof}
	\step{a}{\pflet{$\alpha \in \mathbf{A}$}}
	\step{b}{\assume{as transfinite induction hypothesis $\forall \beta \in \mathbf{A}$, if $\beta < \alpha$ then there exists $\gamma$ such that $\beta = \mathbf{F}(\gamma)$.}}
	\step{c}{\pflet{$\gamma = \{ \delta \in \mathbf{On} \mid \mathbf{F}(\delta) < \alpha \}$}}
	\step{d}{$\gamma$ is a set.}
	\begin{proof}
		\pf\ Axiom of Replacement applied to $\alpha$.
	\end{proof}
	\step{e}{$\gamma$ is a transitive set.}
	\begin{proof}
		\pf\ If $\mathbf{F}(\delta) < \alpha$ and $\epsilon < \delta$ then $\mathbf{F}(\epsilon) < \alpha$ by \stepref{2}.
	\end{proof}
	\step{f}{$\gamma$ is an ordinal.}
	\begin{proof}
		\pf\ Corollary \ref{cor:transords}.
	\end{proof}
	\step{g}{$\mathbf{F}(\gamma) = \alpha$}
	\begin{proof}
		\step{i}{$\mathbf{F}(\gamma)$ is the least element of $\mathbf{A}$ different from $\mathbf{F}(\delta)$ for all $\delta < \gamma$}
		\step{ii}{$\mathbf{F}(\gamma)$ is the least element of $\mathbf{A}$ different from $x$ for all $x \in \mathbf{A}$ with $x < \alpha$}
		\step{iii}{$\mathbf{F}(\gamma) = \alpha$}
	\end{proof}
\end{proof}
\qed
\end{proof}

\section{Ordinal Operations}

\begin{df}[Ordinal Operation]
An \emph{ordinal operation} is a function $\mathbf{On} \rightarrow \mathbf{On}$.
\end{df}

\begin{df}[Continuous]
An ordinal operation $\mathbf{T} : \mathbf{On} \rightarrow \mathbf{On}$ is \emph{continuous} iff, for every limit ordinal $\lambda$, we have $\mathbf{T}(\lambda) = \bigcup_{\alpha < \lambda} \mathbf{T}(\alpha)$.
\end{df}

\begin{df}[Normal]
An ordinal operation is \emph{normal} iff it is continuous and strictly monotone.
\end{df}

\begin{props}[S without Foundation]
\label{prop:normal}
For any class $\mathbf{T}$, the following is a theorem.

If $\mathbf{T}$ is a continuous ordinal operation and $\forall \gamma. \mathbf{T}(\gamma) < \mathbf{T}(\gamma^+)$, then $\mathbf{T}$ is normal.
\end{props}

\begin{proof}
\pf
\step{1}{\pflet{$P[\beta]$ be the property $\forall \gamma < \beta. \mathbf{T}(\gamma) < \mathbf{T}(\beta)$}}
\step{2}{$P[0]$}
\begin{proof}
	\pf\ Vacuous.
\end{proof}
\step{3}{For any ordinal $\gamma$, if $P[\gamma]$ then $P[\gamma^+]$}
\begin{proof}
	\step{a}{\assume{$P[\gamma]$}}
	\step{b}{\pflet{$\delta < \gamma^+$}}
	\step{c}{\case{$\delta < \gamma$}}
	\begin{proof}
		\pf\ Then $\mathbf{T}(\delta) < \mathbf{T}(\gamma) < \mathbf{T}(\gamma^+)$.
	\end{proof}
	\step{d}{\case{$\delta = \gamma$}}
	\begin{proof}
		\pf\ Then $\mathbf{T}(\delta) = \mathbf{T}(\gamma) < \mathbf{T}(\gamma^+)$.
	\end{proof}
\end{proof}
\step{4}{For any limit ordinal $\lambda$, if $\forall \gamma < \lambda. P[\gamma]$ then $P[\lambda]$.}
\begin{proof}
	\step{a}{\assume{$\forall \gamma < \lambda. P[\gamma]$}}
	\step{b}{\pflet{$\delta < \lambda$}}
	\step{c}{$\mathbf{T}(\delta) < \mathbf{T}(\lambda)$}
	\begin{proof}
		\pf
		\begin{align*}
			\mathbf{T}(\delta) & < \mathbf{T}(\delta^+) \\
			& \leq \bigcup_{\epsilon < \lambda} \mathbf{T}(\epsilon) \\
			& = \mathbf{T}(\lambda)
		\end{align*}
	\end{proof}
\end{proof}
\qed
\end{proof}

\begin{props}[S without Foundation]
\label{prop:gammaltTgamma}
For any class $\mathbf{T}$, the following is a theorem:

Assume $\mathbf{T}$ is a normal ordinal operation. For every ordinal $\alpha$, we have $\alpha \leq \mathbf{T}(\alpha)$.
\end{props}

\begin{proof}
\pf
\step{i}{\pflet{$\gamma$ be an ordinal.}}
\step{ii}{\assume{as induction hypothesis $\forall \delta < \gamma. \mathbf{T}(\delta) \geq \delta$}}
\step{iii}{For all $\delta < \gamma$ we have $\delta < \mathbf{T}(\gamma)$}
\begin{proof}
	\pf\ $\mathbf{T}$ is strictly monotone.
\end{proof}
\step{iv}{$\gamma \leq \mathbf{T}(\gamma)$}
\qed
\end{proof}

\begin{props}[S without Foundation]
\label{prop:greatestordinal}
For any class $\mathbf{T}$, the following is a theorem:

Assume $\mathbf{T}$ is a normal ordinal operation. For any ordinal $\beta \geq \mathbf{T}(0)$, there exists a greatest ordinal $\gamma$ such that $\mathbf{T}(\gamma) \leq \beta$.
\end{props}

\begin{proof}
\pf
\step{1}{There exists $\gamma$ such that $\mathbf{T}(\gamma) > \beta$}
\begin{proof}
	\step{a}{For all $\gamma$ we have $\mathbf{T}(\gamma) \geq \gamma$}
	\begin{proof}
		\pf\ Proposition \ref{prop:gammaltTgamma}.
	\end{proof}
	\step{b}{$\mathbf{T}(\beta^+) > \beta$}
\end{proof}
\step{2}{\pflet{$\delta$ be least such that $\mathbf{T}(\delta) > \beta$}}
\step{3}{$\delta$ is a successor ordinal.}
\begin{proof}
	\step{a}{$\delta \neq 0$}
	\begin{proof}
		\pf\ Since $\mathbf{T}(0) \leq \beta$.
	\end{proof}
	\step{b}{$\delta$ is not a limit ordinal.}
	\begin{proof}
		\step{i}{\assume{for a contradiction $\delta$ is a limit ordinal.}}
		\step{ii}{$\beta < \bigcup_{\epsilon < \delta} \mathbf{T}(\epsilon)$}
		\begin{proof}
			\pf\ $\mathbf{T}$ is continuous.
		\end{proof}
		\step{iii}{There exists $\epsilon < \delta$ such that $\beta < \mathbf{T}(\epsilon)$}
		\qedstep
		\begin{proof}
			\pf\ This contradicts the minimality of $\delta$.
		\end{proof}
	\end{proof}
\end{proof}
\step{4}{\pflet{$\delta = \gamma^+$}}
\step{5}{$\gamma$ is greatest such that $\mathbf{T}(\gamma) \leq \beta$}
\qed
\end{proof}

\begin{thms}[S without Foundation]
\label{thm:normalsup}
For any class $\mathbf{T}$, the following is a theorem:

Assume that $\mathbf{T}$ is a normal ordinal operation. For any nonempty set of ordinals $S$, we have
\[ \mathbf{T}(\sup S) = \sup_{\alpha \in S} \mathbf{T}(\alpha) \enspace . \]
\end{thms}

\begin{proof}
\pf
\step{1}{$\forall \alpha \in S. \mathbf{T}(\alpha) \leq \mathbf{T}(\sup S)$}
\begin{proof}
	\pf\ Since $\mathbf{T}$ is monotone.
\end{proof}
\step{2}{For any ordinal $\beta$, if $\forall \alpha \in S. \mathbf{T}(\alpha) \leq \beta$, then $\mathbf{T}(\sup S) \leq \beta$}
\begin{proof}
	\step{a}{\pflet{$\beta$ be an ordinal.}}
	\step{b}{\pflet{$\gamma = \sup S$}}
	\step{c}{\assume{$\forall \alpha \in S. \mathbf{T}(\alpha) \leq \beta$}}
	\step{d}{\case{$\gamma$ is 0 or a successor ordinal}}
	\begin{proof}
		\pf\ Then we must have $\gamma \in S$ so $\mathbf{T}(\gamma) \leq \beta$ from \stepref{c}.
	\end{proof}
	\step{f}{\case{$\gamma$ is a limit ordinal}}
	\begin{proof}
		\step{i}{$\mathbf{T}(\gamma) = \sup_{\alpha < \gamma} \mathbf{T}(\alpha)$}
		\begin{proof}
			\pf\ $\mathbf{T}$ is continuous.
		\end{proof}
		\step{ii}{\assume{for a contradiction $\beta < \mathbf{T}(\gamma)$}}
		\step{iii}{\pick\ $\alpha < \gamma$ such that $\beta < \mathbf{T}(\alpha)$}
		\begin{proof}
			\pf\ \stepref{i}, \stepref{ii}
		\end{proof}
		\step{iv}{\pick\ $\alpha' \in S$ such that $\alpha < \alpha'$}
		\begin{proof}
			\pf\ \stepref{b}, \stepref{iii}
		\end{proof}
		\step{v}{$\beta < \mathbf{T}(\alpha') \leq \beta$}
		\begin{proof}
			\pf\ $\mathbf{T}$ is strictly monotone, \stepref{iii}, \stepref{iv}, \stepref{c}.
		\end{proof}
		\qedstep
		\begin{proof}
			\pf\ This is a contradiction.
		\end{proof}
	\end{proof}
\end{proof}
\qed
\end{proof}

\begin{prop}[SF without Foundation]
\label{prop:enumerationnormal}
For any classes $\mathbf{A}$ and $\mathbf{T}$, the following is a theorem:

Assume $\mathbf{A}$ is a proper class of ordinals such that, for every set $S \subseteq \mathbf{A}$, we have $\bigcup S \in \mathbf{A}$. Assume $\mathbf{T}$ is the unique order isomorphism $\mathbf{On} \cong \mathbf{A}$. Then $\mathbf{T}$ is normal.
\end{prop}

\begin{proof}
\pf
\step{1}{$\mathbf{T}$ is strictly monotone.}
\begin{proof}
	\pf\ Since it is an order isomorphism.
\end{proof}
\step{2}{$\mathbf{T}$ is continuous.}
\begin{proof}
	\step{a}{\pflet{$\lambda$ be a limit ordinal.}}
	\step{b}{$\mathbf{T}'(\lambda)$ is the least member of $\mathbf{A}$ that is greater than $\mathbf{T}'(\alpha)$ for all $\alpha < \lambda$}
	\step{c}{$\mathbf{T}'(\lambda) = \sup_{\alpha < \lambda} \mathbf{T}'(\alpha)$}
\end{proof}
\qed
\end{proof}

\begin{props}[S without Foundation]
For any class $\mathbf{T}$, the following is a theorem:

If $\mathbf{T}$ is a normal ordinal operation, then for any limit ordinal $\lambda$, we have $\mathbf{T}(\lambda)$ is a limit ordinal.
\end{props}

\begin{proof}
\pf
\step{1}{$\mathbf{T}(\lambda) \neq 0$}
\begin{proof}
	\pf\ Since $0 \leq \mathbf{T}(0) < \mathbf{T}(\lambda)$.
\end{proof}
\step{2}{$\mathbf{T}(\lambda)$ is not a successor ordinal.}
\begin{proof}
	\step{a}{\assume{for a contradiction $\mathbf{T}(\lambda) = \alpha^+$}}
	\step{b}{$\alpha < \mathbf{T}(\lambda) = \sup_{\beta < \lambda} \mathbf{T}(\beta)$}
	\step{c}{\pick\ $\beta < \lambda$ such that $\alpha < \mathbf{T}(\beta)$}
	\step{d}{$\alpha^+ \leq \mathbf{T}(\beta) < \mathbf{T}(\lambda)$}
	\qedstep
	\begin{proof}
		\pf\ This is a contradiction.
	\end{proof}
\end{proof}
\qed
\end{proof}

\section{Ordinal Arithmetic}

\subsection{Addition}

\begin{df}
Let $A$ and $B$ be disjoint well-ordered sets. The \emph{concatenation} of $A$ and $B$ is the set $A \cup B$ under the relation:
\begin{itemize}
\item if $a,a' \in A$ then $a \leq a'$ iff $a \leq a'$ in $A$
\item if $b, b' \in B$ then $b \leq b'$ iff $b \leq b'$ in $B$
\item if $a \in A$ and $b \in B$ then $a \leq b$ and $b \not\leq a$.
\end{itemize}
\end{df}

\begin{prop}[S without Foundation]
If $A$ and $B$ are disjoint well-ordered sets, then their concatenation is well-ordered.
\end{prop}

\begin{proof}
\pf
\step{1}{$\leq$ is reflexive.}
\begin{proof}
	\pf\ For all $a \in A$ we have $a \leq a$, and for all $b \in B$ we have $b \leq b$.
\end{proof}
\step{2}{$\leq$ is antisymmetric.}
\begin{proof}
	\step{a}{\assume{$x \leq y \leq x$}}
	\step{b}{\case{$x,y \in A$}}
	\begin{proof}
		\pf\ Then $x = y$ since the order on $A$ is antisymmetric.
	\end{proof}
	\step{c}{\case{$x \in A$ and $y \in B$}}
	\begin{proof}
		\pf\ This is impossible as it would imply $y \not\leq x$.
	\end{proof}
	\step{d}{\case{$x \in B$ and $y \in A$}}
	\begin{proof}
		\pf\ This is impossible as it would imply $x \not\leq y$.
	\end{proof}
	\step{e}{\case{$x,y \in B$}}
	\begin{proof}
		\pf\ Then $x = y$ since the order on $B$ is antisymmetric.
	\end{proof}
\end{proof}
\step{3}{$\leq$ is transitive.}
\begin{proof}
	\step{a}{\assume{$x \leq y \leq z$}}
	\step{b}{\case{$x, z \in A$}}
	\begin{proof}
		\pf\ In this case $y \in A$ since $y \leq z$, and so $x \leq z$ since the order on $A$ is transitive.
	\end{proof}
	\step{c}{\case{$x \in A$ and $z \in B$}}
	\begin{proof}
		\pf\ Then $x \leq z$ immediately.
	\end{proof}
	\step{d}{\case{$x \in B$ and $z \in A$}}
	\begin{proof}
		\pf\ This is impossible because we have $y \notin A$ since $x \leq y$ and $y \notin B$ since $y \leq z$.
	\end{proof}
	\step{e}{\case{$x,z \in B$}}
	\begin{proof}
		\pf\ In this case $y \in B$ since $x \leq y$, and so $x \leq z$ since the order on $B$ is transitive.
	\end{proof}
\end{proof}
\step{4}{$\leq$ is total.}
\begin{proof}
	\step{a}{\pflet{$x,y \in A \cup B$}}
	\step{b}{\case{$x,y \in A$}}
	\begin{proof}
		\pf\ Then $x \leq y$ or $y \leq x$ because the order on $A$ is total.
	\end{proof}
	\step{c}{\case{$x \in A$ and $y \in B$}}
	\begin{proof}
		\pf\ Then $x \leq y$.
	\end{proof}
	\step{d}{\case{$x \in B$ and $y \in A$}}
	\begin{proof}
		\pf\ Then $y \leq x$.
	\end{proof}
	\step{e}{\case{$x,y \in B$}}
	\begin{proof}
		\pf\ Then $x \leq y$ or $y \leq x$ because the order on $B$ is total.
	\end{proof}
\end{proof}
\step{5}{Every nonempty subset of $A \cup B$ has a least element.}
\begin{proof}
	\step{a}{\pflet{$S$ be a nonempty subset of $A \cup B$}}
	\step{b}{\case{$S \cap A = \emptyset$}}
	\begin{proof}
		\pf\ Then $S \subseteq B$ and so $S$ has a least element.
	\end{proof}
	\step{c}{\case{$S \cap A \neq \emptyset$}}
	\begin{proof}
		\pf\ The least element of $S \cap A$ is the least element of $S$.
	\end{proof}
\end{proof}
\qed
\end{proof}

\begin{df}[Ordinal Addition (SF without Foundation)]
Let $\alpha$ and $\beta$ be ordinal numbers. Then $\alpha + \beta$ is the ordinal number of the concatenation of $A$ and $B$, where $A$ is any well ordered set with ordinal $\alpha$ and $B$ is any well ordered set with ordinal $\beta$.
\end{df}

\begin{thm}[Associative Law for Addition (SF without Foundation)]
For any ordinals $\rho$, $\sigma$ and $\tau$, we have
\[ \rho + (\sigma + \tau) = (\rho + \sigma) + \tau \enspace . \]
\end{thm}

\begin{proof}
\pf\ Given disjoint well ordered sets $A$, $B$ and $C$, the concatenation of $A$ with (the concatenation of $B$ and $C$) is the same as the concatenation of (the concatenation of $A$ and $B$) and $C$. \qed
\end{proof}

\begin{thm}[SF without Foundation]
For any ordinal $\rho$ we have
\[ \rho + 0 = 0 + \rho = \rho \enspace . \]
\end{thm}

\begin{proof}
\pf\ For any well ordered set $A$, the concatenation of $A$ with $\emptyset$ is $A$, and the concatenation of $\emptyset$ with $A$ is $A$. \qed
\end{proof}

\begin{thm}[SF without Foundation]
For any ordinal $\alpha$ we have $\alpha + 1 = \alpha^+$.
\end{thm}

\begin{proof}
\pf\ Since $\alpha^+$ is the concatenation of $\alpha$ and $\{ \alpha \}$. \qed
\end{proof}

\begin{thm}[SF without Foundation]
For any ordinal $\alpha$, the operation that maps $\beta$ to $\alpha + \beta$ is normal.
\end{thm}

\begin{proof}
\pf
\step{1}{For any limit ordinal $\lambda$, we have $\alpha + \lambda = \sup_{\beta < \lambda} (\alpha + \beta)$.}
\begin{proof}
	\step{a}{\pflet{$\lambda$ be a limit ordinal.}}
	\step{b}{$(\{0\} \times \alpha) \cup (\{1\} \times \lambda) = \bigcup_{\beta \in \lambda} ((\{0\} \times \alpha) \cup (\{1\} \times \beta))$, where the order on the right hand side is as in Lemma \ref{lm:union_of_end_extensions}.}
	\begin{proof}
		\pf
		\begin{align*}
			(\{0\} \times \alpha) \cup (\{1\} \times \lambda) & = (\{0\} \times \alpha) \cup (\{1\} \times \bigcup_{\beta < \lambda} \beta) \\
			& = (\{0\} \times \alpha) \cup \bigcup_{\beta < \lambda} (\{1\} \times \beta) \\
			& = \bigcup_{\beta < \lambda} ((\{0\} \times \alpha) \cup (\{1\} \times \beta))
		\end{align*}
	\end{proof}
\end{proof}
\step{2}{For any ordinal $\beta$ we have $\alpha + \beta < \alpha + \beta^+$}
\begin{proof}
	\pf\ Since $\alpha + \beta^+ = \alpha + \beta + 1 = (\alpha + \beta)^+$.
\end{proof}
\qed
\end{proof}

\begin{cor}[SF without Foundation]
\label{cor:pluslt}
For any ordinals $\alpha$, $\beta$, $\gamma$, we have $\beta < \gamma$ if and only if $\alpha + \beta < \alpha + \gamma$.
\end{cor}

\begin{cor}[Left Cancellation for Addition (SF without Foundation)]
For any ordinals $\alpha$, $\beta$ and $\gamma$, if $\alpha + \beta = \alpha + \gamma$ then $\beta = \gamma$.
\end{cor}

\begin{thm}[SF without Foundation]
For any ordinals $\alpha$, $\beta$, $\gamma$, if $\beta \leq \gamma$ then $\beta + \alpha \leq \gamma + \alpha$.
\end{thm}

\begin{proof}
\pf\ Transfinite induction on $\alpha$. \qed
\end{proof}

\begin{thm}[Subtraction Theorem (SF without Foundation)]
Let $\alpha$ and $\beta$ be ordinals with $\alpha \leq \beta$. Then there exists a unique ordinal $\delta$ such that $\alpha + \delta = \beta$.
\end{thm}

\begin{proof}
\pf
\step{1}{For all ordinals $\alpha$ and $\beta$ with $\alpha \leq \beta$, there exists $\delta$ such that $\alpha + \delta = \beta$}
\begin{proof}
	\step{a}{\pflet{$\alpha$ and $\beta$ be ordinals with $\alpha \leq \beta$}}
	\step{b}{\pflet{$\delta$ be the greatest ordinal such that $\alpha + \delta \leq \beta$}}
	\begin{proof}
		\pf\ Proposition \ref{prop:greatestordinal}.
	\end{proof}
	\step{c}{$\alpha + \delta = \beta$}
	\begin{proof}
		\pf\ If $\alpha + \delta < \beta$ then $\alpha + \delta + 1 \leq \beta$ contradicting the greatestness of $\delta$.
	\end{proof}
\end{proof}
\qedstep
\begin{proof}
	\pf\ Uniqueness follows from the Left Cancellation Law.
\end{proof}
\qed
\end{proof}

\begin{df}[Additively Indecomposable]
An infinite ordinal $\alpha$ is \emph{additively indecomposable} iff, whenever $\beta + \gamma = \alpha$, then $\beta = \alpha$ or $\gamma = \alpha$.
\end{df}

\begin{prop}[SF without Foundation]
For an infinite ordinal $\alpha$, we have $\alpha$ is additively indecomposable if and only if, for all $\gamma < \alpha$, we have $\gamma + \alpha = \alpha$.
\end{prop}

\begin{proof}
\pf
\step{1}{If $\alpha$ is additively indecomposable then, for all $\gamma < \alpha$, we have $\gamma + \alpha = \alpha$.}
\begin{proof}
	\step{a}{\assume{$\alpha$ is additively indecomposable.}}
	\step{b}{\pflet{$\gamma < \alpha$}}
	\step{c}{\pflet{$\delta$ be the unique ordinal such that $\gamma + \delta = \alpha$}}
	\step{d}{$\delta = \alpha$}
	\step{e}{$\gamma + \alpha = \alpha$}
\end{proof}
\step{2}{If, for all $\gamma < \alpha$, we have $\gamma + \alpha = \alpha$, then $\alpha$ is additively indecomposable.}
\begin{proof}
	\step{a}{\assume{For all $\gamma < \alpha$, we have $\gamma + \alpha = \alpha$.}}
	\step{b}{\pflet{$\beta + \gamma = \alpha$}}
	\step{c}{\assume{$\beta < \alpha$}}
	\step{d}{$\beta + \alpha = \alpha = \beta + \gamma$}
	\step{e}{$\alpha = \gamma$}
\end{proof}
\qed
\end{proof}

\subsection{Multiplication}

\begin{df}[Ordinal Multiplication (SF without Foundation)]
Let $\alpha$ and $\beta$ be ordinal numbers. Then $\alpha \beta$ is the ordinal number of $A \times B$ under the lexicographic order, where $A$ is any well ordered set with ordinal $\alpha$ and $B$ is any well ordered set with ordinal $\beta$.
\end{df}

This is well defined by Proposition \ref{prop:lexicogwellorder}.

\begin{thm}[Associative Law (SF without Foundation)]
For any ordinals $\rho$, $\sigma$ and $\tau$, we have
\[ \rho (\sigma \tau) = (\rho \sigma) \tau \enspace . \]
\end{thm}

\begin{proof}
\pf\ Let $A$, $B$ and $C$ be well ordered sets with ordinals $\rho$, $\sigma$ and $\tau$. Then both $\rho (\sigma \tau)$ and $(\rho \sigma) \tau$ are the ordinal of $A \times B \times C$ under
\[ (a,b,c) \leq (a',b',c') \Leftrightarrow a \leq a' \vee (a = a' \wedge b \leq b') \vee (a = a' \wedge b = b' \wedge c \leq c') \enspace . \qquad \qed \]
\end{proof}

\begin{thm}[Left Distributive Law (SF without Foundation)]
For any ordinals $\rho$, $\sigma$ and $\tau$, we have
\[ \rho (\sigma + \tau) = \rho \sigma + \rho \tau \]
\end{thm}

\begin{proof}
\pf\ Let $A$, $B$ and $C$ be well ordered sets with ordinals $\rho$, $\sigma$ and $\tau$ and with $B \cap C = \emptyset$. Then both $\rho (\sigma + \tau)$ and $\rho \sigma + \rho \tau$ are the ordinal of $A \times (B \cup C)$ under the lexicographic ordering. \qed
\end{proof}

\begin{thm}[SF without Foundation]
For any ordinal $\rho$ we have $\rho 0 = 0 \rho = 0$.
\end{thm}

\begin{proof}
\pf\ For any well ordered set $A$ we have $A \times \emptyset = \emptyset \times A = \emptyset$. \qed
\end{proof}

\begin{thm}[SF without Foundation]
For any ordinal $\rho$ we have $\rho 1 = 1 \rho = \rho$.
\end{thm}

\begin{proof}
\pf\ Easy. \qed
\end{proof}

\begin{thm}[SF without Foundation]
For any ordinals $\rho$ and $\sigma$, if $\rho \sigma = 0$ then $\rho = 0$ or $\sigma = 0$.
\end{thm}

\begin{proof}
\pf\ If $A \times B = \emptyset$ then $A = \emptyset$ or $B = \emptyset$. \qed
\end{proof}

\begin{thm}[SF without Foundation]
\label{thm:multnormal}
For any non-zero ordinal $\alpha$, the operation that maps $\beta$ to $\alpha \beta$ is normal.
\end{thm}

\begin{proof}
\pf
\step{1}{For any limit ordinal $\lambda$, we have $\alpha \lambda = \bigcup_{\beta < \lambda} \alpha \beta$}
\begin{proof}
	\step{a}{\pflet{$\lambda$ be a limit ordinal}}
	\step{b}{$\alpha \times \lambda = \bigcup_{\beta < \lambda} (\alpha \times \beta)$ as well-ordered sets}
\end{proof}
\step{2}{For any ordinal $\beta$ we have $\alpha \beta < \alpha \beta^+$}
\begin{proof}
	\pf\ $\alpha \beta^+ = \alpha \beta + \alpha > \alpha \beta$
\end{proof}
\qed
\end{proof}

\begin{cor}[SF without Foundation]
For any ordinals $\alpha$, $\beta$, $\gamma$, if $\alpha \neq 0$ then $\beta < \gamma$ if and only if $\alpha \beta < \alpha \gamma$.
\end{cor}

\begin{cor}[Left Cancellation for Multiplication (SF without Foundation)]
For any ordinals $\alpha$, $\beta$, $\gamma$, if $\alpha \neq 0$ and $\alpha \beta = \alpha \gamma$ then $\beta = \gamma$.
\end{cor}

\begin{thm}[SF without Foundation]
For any ordinals $\alpha$, $\beta$ and $\gamma$, if $\beta \leq \gamma$ then $\beta \alpha \leq \gamma \alpha$.
\end{thm}

\begin{proof}
\pf\ Transfinite induction on $\alpha$. \qed
\end{proof}

\begin{thm}[Division Theorem (SF without Foundation)]
Let $\alpha$ and $\delta$ be ordinal numbers with $\delta \neq 0$. Then there exist unique ordinals $\beta$ and $\gamma$ with $\gamma < \delta$ and
\[ \alpha = \delta \beta + \gamma \enspace . \]
\end{thm}

\begin{proof}
\pf
\step{1}{For any ordinal numbers $\alpha$ and $\delta$ with $\delta \neq 0$, there exist ordinals $\beta$ and $\gamma$ such that $\gamma < \delta$ and $\alpha = \delta \beta + \gamma$}
\begin{proof}
	\step{a}{\pflet{$\alpha$ and $\delta$ be ordinals with $\delta \neq 0$}}
	\step{b}{\pflet{$\beta$ be the greatest ordinal such that $\delta \beta \leq \alpha$}}
	\begin{proof}
		\pf\ Proposition \ref{prop:greatestordinal}.
	\end{proof}
	\step{c}{There exists an ordinal $\gamma$ such that $\alpha = \delta \beta + \gamma$}
	\begin{proof}
		\pf\ Subtraction Theorem
	\end{proof}
\end{proof}
\step{2}{For any ordinals $\delta$, $\beta$, $\beta'$, $\gamma$, $\gamma'$, if $\delta \beta + \gamma = \delta \beta' + \gamma'$ and $\delta \neq 0$ and $\gamma, \gamma' < \delta$ then $\beta = \beta'$ and $\gamma = \gamma'$}
\begin{proof}
	\step{a}{\pflet{$\delta$, $\beta$, $\beta'$, $\gamma$, $\gamma'$ be ordinals.}}
	\step{b}{\assume{$\delta \neq 0$ and $\delta \beta + \gamma = \delta \beta' + \gamma'$}}
	\step{c}{$\beta = \beta'$}
	\begin{proof}
		\step{i}{$\beta \nless \beta'$}
		\begin{proof}
			\pf\ If $\beta < \beta'$ then
			\begin{align*}
				\delta \beta' + \gamma' & \geq \delta \beta' \\
				& \geq \delta (\beta + 1) \\
				& = \delta \beta + \delta \\
				& > \delta \beta + \gamma
			\end{align*}
		\end{proof}
		\step{ii}{$\beta' \nless \beta$}
		\begin{proof}
			\pf\ Similar.
		\end{proof}
	\end{proof}
	\step{d}{$\gamma = \gamma'$}
	\begin{proof}
		\pf\ By Cancellation.
	\end{proof}
\end{proof}
\qed
\end{proof}

\begin{df}[Multiplicatively Indecomposable]
An infinite ordinal $\alpha$ is \emph{multiplicatively indecomposable} iff, whenever $\beta \gamma \geq \alpha$, then either $\beta \geq \alpha$ or $\gamma \geq \alpha$.
\end{df}

\begin{prop}[SF without Foundation]
Let $\alpha$ be an infinite ordinal. Then $\alpha$ is multiplicatively indecomposable iff, for all $0 < \gamma < \alpha$, we have $\gamma \alpha = \alpha$.
\end{prop}

\begin{proof}
\pf
\step{1}{If $\alpha$ is multiplicatively indecomposable then, for all $0 < \gamma < \alpha$, we have $\gamma \alpha = \alpha$}
\begin{proof}
	\step{a}{\assume{$\alpha$ is multiplicatively indecomposable.}}
	\step{b}{\pflet{$0 < \gamma < \alpha$}}
	\step{c}{\pflet{$\alpha = \gamma \delta + \epsilon$ where $\epsilon < \gamma$}}
	\begin{proof}
		\pf\ Division Theorem.
	\end{proof}
	\step{d}{$\gamma \delta \leq \alpha < \gamma (\delta + 1)$}
	\step{f}{$\delta \leq \alpha \leq \delta + 1$}
	\step{g}{$\alpha \neq \delta + 1$}
	\begin{proof}
		\pf\ A successor ordinal cannot be multiplicatively indecomposable because we have $\delta + 1 < \delta 2$.
	\end{proof}
	\step{h}{$\alpha = \delta$}
	\step{i}{$\epsilon = 0$}
	\step{j}{$\alpha = \gamma \alpha$}
\end{proof}
\step{2}{If, for all $0 < \gamma < \alpha$, we have $\gamma \alpha = \alpha$, then $\alpha$ is multiplicatively indecomposable.}
\begin{proof}
	\step{a}{\assume{For all $0 < \gamma < \alpha$ we have $\gamma \alpha = \alpha$}}
	\step{b}{\pflet{$\gamma, \delta < \alpha$}}
	\step{c}{$\gamma \delta < \alpha$}
	\begin{proof}
		\pf\ If $\gamma \neq 0$ then $\gamma \delta < \gamma \alpha = \alpha$. If $\gamma = 0$ then $\gamma \delta = 0 < \alpha$.
	\end{proof}
\end{proof}
\qed
\end{proof}

\subsection{Exponentiation}

\begin{df}[SF without Foundation]
Given ordinals $\alpha$ and $\beta$, define the ordinal $\alpha^\beta$ as follows:
\begin{align*}
0^\alpha & := 0 & (\alpha > 0) \\
\alpha^0 & := 1 \\
\alpha^{\beta^+} & := \alpha^\beta \alpha & (\alpha > 0) \\
\alpha^\lambda & := \sup_{\beta < \lambda} \alpha^\beta & (\alpha > 0, \lambda \text{ a limit ordinal})
\end{align*}
\end{df}

\begin{thm}[SF without Foundation]
Let $\alpha$ be an ordinal $\geq 2$. The operation that maps $\beta$ to $\alpha^\beta$ is normal.
\end{thm}

\begin{proof}
\pf
\step{1}{For $\lambda$ a limit ordinal we have $\alpha^\lambda = \sup_{\beta < \lambda} \alpha^\beta$}
\begin{proof}
	\pf\ By definition.
\end{proof}
\step{2}{For any ordinal $\beta$ we have $\alpha^\beta < \alpha^{\beta^+}$}
\begin{proof}
	\pf\ We have $\alpha^{\beta^+} = \alpha^\beta \alpha > \alpha^\beta$ by Theorem \ref{thm:multnormal} since $\alpha > 1$ and $\alpha^\beta \neq 0$.
\end{proof}
\qed
\end{proof}

\begin{cor}[SF without Foundation]
For any ordinals $\alpha$, $\beta$, $\gamma$, if $\alpha \geq 2$ then $\beta < \gamma$ if and only if $\alpha^\beta < \alpha^\gamma$.
\end{cor}

\begin{cor}[Cancellation for Exponentiation (SF without Foundation)]
For any ordinals $\alpha$, $\beta$, $\gamma$, if $\alpha \geq 2$ and $\alpha^\beta = \alpha^\gamma$ then $\beta = \gamma$.
\end{cor}

\begin{thm}[SF without Foundation]
For any ordinals $\alpha$, $\beta$ and $\gamma$, if $\beta \leq \gamma$ then $\beta^\alpha \leq \gamma^\alpha$.
\end{thm}

\begin{proof}
\pf\ Transfinite induction on $\alpha$.
\end{proof}

\begin{thm}[Logarithm Theorem (SF without Foundation)]
Let $\alpha$ and $\beta$ be ordinal numbers with $\alpha \neq 0$ and $\beta > 1$. Then there exist unique ordinals $\gamma$, $\delta$ and $\rho$ such that
\[ \alpha = \beta^\gamma \delta + \rho, \qquad 0 \neq \delta < \beta, \qquad \rho < \beta^\gamma \enspace . \]
\end{thm}

\begin{proof}
\pf
\step{1}{For any ordinals $\alpha$ and $\beta$ with $\alpha \neq 0$ and $\beta > 1$, there exist ordinals $\gamma$, $\delta$, $\rho$ such that
\[ \alpha = \beta^\gamma \delta + \rho, \qquad 0 \neq \delta < \beta, \qquad \rho < \beta^\gamma \enspace . \]}
\begin{proof}
	\step{a}{\pflet{$\alpha$ and $\beta$ be ordinals with $\alpha \neq 0$ and $\beta > 1$.}}
	\step{b}{\pflet{$\gamma$ be the greatest ordinal such that $\beta^\gamma \leq \alpha$.}}
	\begin{proof}
		\pf\ Proposition \ref{prop:greatestordinal}.
	\end{proof}
	\step{c}{\pflet{$\delta$ and $\rho$ be the unique ordinals with $\rho < \beta^\gamma$ such that $\alpha = \beta^\gamma \delta + \rho$.}}
	\begin{proof}
		\pf\ By the Division Theorem.
	\end{proof}
	\step{d}{$\delta \neq 0$}
	\begin{proof}
		\pf\ If $\delta = 0$ then $\alpha = \beta^\gamma 0 + \rho = \rho < \beta^\gamma \leq \alpha$ which is a contradiction.
	\end{proof}
	\step{e}{$\delta < \beta$}
	\begin{proof}
		\pf\ If $\beta \leq \delta$ then $\alpha \geq \beta^\gamma \delta \geq \beta^\gamma \beta = \beta^{\gamma + 1}$, contradicting the greatestness of $\gamma$.
	\end{proof}
\end{proof}
\step{2}{If $\beta^\gamma \delta + \rho = \beta^{\gamma'} \delta' + \rho'$ with $\beta > 1$, $0 \neq \delta < \beta$, $0 \neq \delta' < \beta$, $\rho < \beta^\gamma$ and $\rho' < \beta^{\gamma'}$, then $\gamma = \gamma'$, $\delta = \delta'$ and $\rho = \rho'$.}
\begin{proof}
	\step{a}{\pflet{$\alpha = \beta^\gamma \delta + \rho = \beta^{\gamma'} \delta' + \rho'$}}
	\step{b}{$\beta^\gamma \leq \alpha < \beta^{\gamma + 1}$}
	\step{c}{$\beta^{\gamma'} \leq \alpha < \beta^{\gamma' + 1}$}
	\step{d}{$\beta^\gamma < \beta^{\gamma' + 1}$ and $\beta^{\gamma'} < \beta^{\gamma + 1}$}
	\step{e}{$\gamma < \gamma' + 1$ and $\gamma' < \gamma + 1$}
	\step{f}{$\gamma = \gamma'$}
	\step{g}{$\delta = \delta'$ and $\rho = \rho'$}
	\begin{proof}
		\pf\ By the Division Theorem.
	\end{proof}
\end{proof}
\qed
\end{proof}

\begin{thm}[SF without Foundation]
For any ordinal numbers $\alpha$, $\beta$, $\gamma$, we have
\[ \alpha^{\beta + \gamma} = \alpha^\beta \alpha^\gamma \enspace . \]
\end{thm}

\begin{proof}
\pf
\step{1}{\pflet{$P[\gamma]$ be the property: for any ordinals $\alpha$ and $\beta$ we have $\alpha^{\beta + \gamma} = \alpha^\beta \alpha^\gamma$}}
\step{2}{$P[0]$}
\begin{proof}
	\pf
	\begin{align*}
		\alpha^{\beta + 0} & = \alpha^\beta \\
		& = \alpha^\beta 1 \\
		& = \alpha^\beta \alpha^0
	\end{align*}
\end{proof}
\step{3}{For all $\gamma$, if $P[\gamma]$ then $P[\gamma + 1]$}
\begin{proof}
	\pf
	\begin{align*}
		\alpha^{\beta + \gamma + 1} & = \alpha^{\beta + \gamma} \alpha \\
		& = \alpha^\beta \alpha^\gamma \alpha & (\text{induction hypothesis}) \\
		& = \alpha^\beta \alpha^{\gamma + 1}
	\end{align*}
\end{proof}
\step{4}{For any limit ordinal $\lambda$, if $\forall \gamma < \lambda. P[\gamma]$ then $P[\lambda]$.}
\begin{proof}
	\step{a}{\pflet{$\lambda$ be a limit ordinal}}
	\step{b}{\assume{$\forall \gamma < \lambda. P[\gamma]$}}
	\step{c}{\pflet{$\alpha$ and $\beta$ be any ordinals.}}
	\step{d}{\case{$\alpha = 0$}}
	\begin{proof}
		\pf\ We have $\alpha^{\beta + \lambda} = \alpha^\beta \alpha^\lambda = 0$.
	\end{proof}
	\step{e}{\case{$\alpha = 1$}}
	\begin{proof}
		\pf\ We have $\alpha^{\beta + \lambda} = \alpha^\beta \alpha^\lambda = 1$.
	\end{proof}
	\step{f}{\case{$\alpha > 1$}}
	\begin{proof}
		\pf
		\begin{align*}
			\alpha^{\beta + \lambda} & = \alpha^{\sup_{\gamma < \lambda} (\beta + \gamma)} \\
			& = \sup_{\gamma < \lambda} \alpha^{\beta + \gamma} & (\text{Theorem \ref{thm:normalsup}}) \\
			& = \sup_{\gamma < \lambda} \alpha^\beta \alpha^\gamma & (\text{\stepref{b}}) \\
			& = \alpha^\beta \sup_{\gamma < \lambda} \alpha^\gamma & (\text{Theorem \ref{thm:normalsup}}) \\
			& = \alpha^\beta \alpha^\lambda
		\end{align*}
	\end{proof}
\end{proof}
\qed
\end{proof}

\begin{thm}[SF without Foundation]
For any ordinal numbers $\alpha$, $\beta$ and $\gamma$, we have
\[ (\alpha^\beta)^\gamma = \alpha^{\beta \gamma} \enspace . \]
\end{thm}

\begin{proof}
\pf
\step{1}{\pflet{$P[\gamma]$ be the property: For any ordinals $\alpha$ and $\beta$, we have $(\alpha^\beta)^\gamma = \alpha^{\beta \gamma}$}}
\step{2}{$P[0]$}
\begin{proof}
	\pf
	\begin{align*}
		(\alpha^\beta)^0 & = 1 \\
		& = \alpha^{\beta 0}
	\end{align*}
\end{proof}
\step{3}{$\forall \gamma \in \mathbf{On}. P[\gamma] \Rightarrow P[\gamma + 1]$}
\begin{proof}
	\pf
	\begin{align*}
		(\alpha^\beta)^{\gamma + 1} & = (\alpha^\beta)^\gamma \alpha^\beta \\
		& = \alpha^{\beta \gamma} \alpha^\beta \\
		& = \alpha^{\beta \gamma + \beta} \\
		& = \alpha^{\beta (\gamma + 1)}
	\end{align*}
\end{proof}
\step{4}{For any limit ordinal $\lambda$, if $\forall \gamma < \lambda. P[\gamma]$ then $P[\lambda]$.}
\begin{proof}
	\step{a}{\pflet{$\lambda$ be a limit ordinal.}}
	\step{b}{\assume{$\forall \gamma < \lambda. P[\gamma]$}}
	\step{c}{\pflet{$\alpha$ and $\beta$ be any ordinals.}}
	\step{d}{\case{$\alpha = 0$ and $\beta = 0$}}
	\begin{proof}
		\pf
		\begin{align*}
			(0^\beta)^\lambda & = 1^\lambda \\
			& = 1 \\
			& = 0^0 \\
			& = 0^{0 \lambda}
		\end{align*}
	\end{proof}
	\step{dd}{\case{$\alpha = 0$ and $\beta \neq 0$}}
	\begin{proof}
		\pf\ $(0^\beta)^\lambda = 0^{\beta \lambda} = 0$.
	\end{proof}
	\step{e}{\case{$\alpha = 1$}}	
	\begin{proof}
		\pf\ $(1^\beta)^\lambda = 1^{\beta \lambda} = 1$
	\end{proof}
	\step{f}{\case{$\alpha > 1$}}
	\begin{proof}
		\pf
		\begin{align*}
			(\alpha^\beta)^\lambda & = \sup_{\gamma < \lambda} (\alpha^\beta)^\gamma \\
			& = \sup_{\gamma < \lambda} \alpha^{\beta \gamma} \\
			& = \alpha^{\sup_{\gamma < \lambda} \beta \gamma} \\
			& = \alpha^{\beta \lambda}
		\end{align*}
	\end{proof}
\end{proof}
\qed
\end{proof}

\begin{prop}[ZF without Foundation]
For any ordinal $\delta > 1$ we have $\delta^\omega$ is additively indecomposable.
\end{prop}

\begin{proof}
\pf
\step{1}{\assume{$\beta + \gamma = \delta^\omega$}}
\step{2}{\assume{for a contradiction $\beta < \delta^\omega$ and $\gamma < \delta^\omega$}}
\step{3}{\pick\ $m$, $n$ such that $\beta < \delta^m$ and $\gamma < \delta^n$}
\step{4}{$\beta + \gamma < \delta^{\max(m,n) + 1}$}
\begin{proof}
	\pf
	\begin{align*}
		\beta + \gamma & < \delta^m + \delta^n \\
		& \leq \delta^{\max(m,n)} + \delta^{\max(m,n)} \\
		& = \delta^{\max(m,n)} 2 \\
		& \leq \delta^{\max(m,n)} \delta \\
		& = \delta^{\max(m,n) + 1}
	\end{align*}
\end{proof}
\qedstep
\begin{proof}
	\pf\ This contradicts \stepref{1}.
\end{proof}
\qed
\end{proof}

\begin{prop}[ZF without Foundation]
Let $\alpha$ be an infinite ordinal. Then $\alpha$ is additively indecomposable iff $\alpha = \omega^\delta$ for some $\delta$.
\end{prop}

\begin{proof}
\pf
\step{1}{If $\alpha$ is additively indecomposable then there exists $\delta$ such that $\alpha = \omega^\delta$.}
\begin{proof}
	\step{a}{\assume{$\alpha$ is additively indecomposable.}}
	\step{b}{\pflet{$\alpha = \omega^\gamma \delta + \rho$ where $0 \neq \delta < \omega$ and $\rho < \omega^\gamma$}}
	\step{c}{$\omega^\gamma \delta = \alpha$ \prove{$\delta = 1$}}
	\step{e}{$\omega^\gamma (\delta - 1) + \omega^\gamma = \alpha$}
	\step{f}{$\omega^\gamma = \alpha$}
\end{proof}
\step{2}{$\omega$ is additively indecomposable.}
\begin{proof}
	\pf\ If $m,n < \omega$ then $m+n < \omega$.
\end{proof}
\step{3}{For any ordinal $\delta$, if $\omega^\delta$ is additively indecomposable then so is $\omega^{\delta + 1}$}
\begin{proof}
	\step{a}{\pflet{$\gamma < \omega^{\delta + 1} = \omega^\delta \omega$}}
	\step{b}{\pick\ $n < \omega$ such that $\gamma < \omega^\delta n$}
	\step{c}{$\gamma + \omega^{\delta + 1} = \omega^{\delta + 1}$}
	\begin{proof}
		\pf
		\begin{align*}
			\gamma + \omega^{\delta + 1} & \leq \omega^\delta n + \omega^\delta \omega \\
			& = \omega^\delta (n + \omega) \\
			& = \omega^\delta \omega \\
			& = \omega^{\delta + 1}
		\end{align*}
	\end{proof}
\end{proof}
\step{4}{For any limit ordinal $\lambda$, if $\forall \delta < \lambda. \omega^\delta$ is additively indecomposable, then $\omega^\lambda$ is additively indecomposable.}
\begin{proof}
	\step{a}{\pflet{$beta, \gamma < \omega^\lambda$}}
	\step{b}{\pick\ $\delta, \epsilon < \lambda$ such that $\beta < \omega^\delta$ and $\gamma < \omega^\epsilon$}
	\step{c}{$\beta + \gamma < \omega^{\max(\delta, \epsilon) + 1}$}
	\begin{proof}
		\pf
		\begin{align*}
			\beta + \gamma & < \omega^\delta + \omega^\epsilon \\
			& \leq \omega^{\max(\delta, \epsilon)} + \omega^{\max(\delta, \epsilon)} \\
			& = \omega^{\max(\delta, \epsilon)} 2 \\
			& < \omega^{\max(\delta, \epsilon)} \omega \\
			& = \omega^{\max(\delta, \epsilon) + 1}
		\end{align*}
	\end{proof}
\end{proof}
\qed
\end{proof}

\begin{prop}[ZF without Foundation]
For any ordinal $\delta > 1$, we have $\delta^\omega$ is multiplicatively indecomposable.
\end{prop}

\begin{proof}
\pf
\step{1}{\pflet{$\beta, \gamma < \delta^\omega$}}
\step{2}{\pick\ $m, n < \omega$ such that $\beta < \delta^m$ and $\gamma < \delta^n$}
\step{3}{$\beta \gamma < \delta^{m+n} < \delta^\omega$}
\qed
\end{proof}

\begin{prop}[ZF without Foundation]
For any infinite ordinal $\alpha$, we have $\alpha$ is multiplicatively indecomposable iff there exists $\delta$ such that $\alpha = \omega^{\omega^\delta}$.
\end{prop}

\begin{proof}
\pf
\step{1}{If $\alpha$ is multiplicatively indecomposable then there exists $\delta$ such that $\alpha = \omega^{\omega^\delta}$.}
\begin{proof}
	\step{a}{\assume{$\alpha$ is multiplicatively indecomposable.}}
	\step{b}{\pflet{$\alpha = \omega^\beta \gamma + \delta$ where $0 < \gamma < \omega$ and $\delta < \omega^\beta$}}
	\step{c}{$\omega^\beta \leq \alpha < \omega^\beta (\gamma + 1)$}
	\step{d}{$\alpha = \omega^\beta$}
	\begin{proof}
		\pf\ By multiplicative indecomposability. We cannot have $\gamma + 1 \geq \alpha$ because $\gamma + 1$ is finite.
	\end{proof}
	\step{e}{\pflet{$\beta = \omega^\epsilon \phi + \lambda$ where $0 < \phi < \omega$ and $\lambda < \omega^\epsilon$}}
	\step{f}{$(\omega^{\omega^\epsilon})^\phi \leq \alpha = (\omega^{\omega^\epsilon})^\phi \omega^\lambda $}
	\step{g}{$\alpha = (\omega^{\omega^\epsilon})^\phi$}
	\step{h}{$\phi = 1$}
	\begin{proof}
		\step{i}{\assume{for a contradiction $\phi > 1$}}
		\step{ii}{$\alpha = (\omega^{\omega^\epsilon})^{\phi - 1} \omega^{\omega^\epsilon}$}
		\step{iii}{$\alpha \leq (\omega^{\omega^\epsilon})^{\phi - 1}$ or $\alpha \leq \omega^{\omega^\epsilon}$}
		\qedstep
		\begin{proof}
			\pf\ This contradicts \stepref{g} if $\phi > 1$.
		\end{proof}
	\end{proof}
	\step{i}{$\alpha = \omega^{\omega^\epsilon}$}
\end{proof}
\step{2}{$\omega$ is multiplicatively indecomposable.}
\begin{proof}
	\pf\ If $m,n < \omega$ then $mn < \omega$.
\end{proof}
\step{3}{For any ordinal $\delta$, if $\omega^{\omega^\delta}$ is multiplicatively indecomposable, then so is $\omega^{\omega^{\delta + 1}}$.}
\begin{proof}
	\step{a}{\pflet{$\delta$ be an ordinal.}}
	\step{b}{\assume{$\omega^{\omega^\delta}$ is multiplicatively indecomposable.}}
	\step{c}{\pflet{$\beta, \gamma < \omega^{\omega^{\delta + 1}} = (\omega^{\omega^\delta})^\omega$}}
	\step{d}{\pick\ $m, n < \omega$ such that $\beta < (\omega^{\omega^\delta})^m$ and $\gamma < (\omega^{\omega^\delta})^n$.}
	\step{e}{$\beta \gamma < (\omega^{\omega^\delta})^{m+n} < \omega^{\omega^{\delta + 1}}$}
\end{proof}
\step{4}{For any limit ordinal $\lambda$, if $\forall \delta < \lambda. \omega^{\omega^\delta}$ is multiplicatively indecomposable, then $\omega^{\omega^\lambda}$ is multiplicatively indecomposable.}
\begin{proof}
	\step{a}{\pflet{$\lambda$ be a limit ordinal.}}
	\step{b}{\assume{$\forall \delta < \lambda. \omega^{\omega^\delta}$ is multiplicatively indecomposable.}}
	\step{c}{\pflet{$\beta, \gamma < \omega^{\omega^\lambda}$}}
	\step{d}{\pick\ $\beta_1, \gamma_1 < \omega^\lambda$ such that $\beta < \omega^{\beta_1}$ and $\gamma < \omega^{\gamma_1}$}
	\step{e}{\pick\ $\beta_2, \gamma_2 < \lambda$ such that $\beta_1 < \omega^{\beta_2}$ and $\gamma_1 < \omega^{\gamma_2}$}
	\step{f}{$\beta \gamma <  \omega^{\omega^{\max(\beta_2, \gamma_2)}} < \omega^{\omega^\delta}$}
\end{proof}
\qed
\end{proof}

\section{Sequences}

\begin{df}[Sequence]
Given an ordinal $\alpha$ and class $\mathbf{A}$, an \emph{$\alpha$-sequence} in $\mathbf{A}$ is a function $a : \alpha \rightarrow \mathbf{A}$. We write $a_\beta$ for $a(\beta)$, and $(a_\beta)_{\beta < \alpha}$ for $a$.
\end{df}

\begin{df}[Strictly Increasing]
A sequence $(a_\beta)$ of ordinals is \emph{strictly increasing} iff, whenever $\beta < \gamma$, then $a_\beta < a_\gamma$.
\end{df}

\begin{df}[Subsequence]
Let $(a_\beta)_{\beta < \gamma}$ be a sequence in $\mathbf{A}$. A \emph{subsequence} of $(a_\beta)$ is a sequence of the form $(a_{\beta_\xi})_{\xi < \delta}$ where $(\beta_\xi)_{\xi < \delta}$ is a strictly increasing sequence in $\gamma$.
\end{df}

\begin{df}[Convergence]
Let $(a_\beta)_{\beta < \gamma}$ be a sequence of ordinals and $\lambda$ an ordinal. Then $(a_\beta)$ \emph{converges} to the \emph{limit} $\lambda$ iff $\lambda = \sup_{\beta < \gamma} a_\beta$.
\end{df}

\begin{lm}
\label{lm:convergentsubsequence}
Let $(a_\beta)_{\beta < \gamma}$ be a sequence of ordinals. Then there is a strictly increasing subsequence $(a_{\beta_\xi})_{\xi < \delta}$ such that $\sup_{\xi < \delta} a_{\beta_\xi} = \sup_{\beta < \gamma} a_\beta$.
\end{lm}

\begin{proof}
\pf\ Define $\beta_\xi$ by transfinite recursion as follows. $\beta_\xi$ is the least $\beta$ such that $a_\beta > a_{\beta_\zeta}$ for all $\zeta < \xi$ if there is such an $a_\beta$; if not, the sequence ends.
\qed
\end{proof}

\section{Strict Supremum}

\begin{df}[Strict Supremum]
For any set $S$ of ordinals, define the \emph{strict supremum} of $S$, $\ssup S$, to be the least ordinal greater than every member of $S$.
\end{df}

\chapter{Cardinal Numbers}

\section{Cardinal Numbers}

\begin{df}[Cardinal Number]
A \emph{cardinal number} or \emph{initial ordinal} is an ordinal $\kappa$ such that, for all $\alpha < \kappa$, we have $\alpha \not\approx \kappa$.

Let $\mathbf{Card}$ be the class of all cardinal numbers.
\end{df}

\begin{prop}[S without Foundation]
Every natural number is a cardinal number.
\end{prop}

\begin{prop}[Z without Foundation]
$\omega$ is a cardinal number.
\end{prop}

\begin{df}[Aleph]
For $x$ a set, define $\aleph(x) = \{ \alpha \in \mathbf{On} \mid \alpha \preccurlyeq x \}$.
\end{df}

\begin{thm}[SF without Foundation]
For every set $x$, we have $\aleph(x)$ is a cardinal number.
\end{thm}

\begin{proof}
\pf
\step{1}{\pflet{$x$ be a set.}}
\step{2}{$\aleph_x$ is a set.}
\begin{proof}
\step{2}{\pflet{$z = \{ r \in \mathcal{P} x^2 \mid \exists y \subseteq x. r \text{ is a well ordering on } y \}$}}
\step{3}{$\aleph_x = \{ \alpha \mid \exists r \in z. (\dom r \cup \ran r, r) \cong \alpha \}$}
\step{4}{$\aleph_x$ is a set.}
\begin{proof}
	\pf\ By an Axiom of Replacement using Theorem \ref{thm:unique_ordinal}.
\end{proof}
\end{proof}
\step{3}{$\aleph_x$ is an ordinal.}
\begin{proof}
	\step{a}{$\aleph_x$ is a transitive set.}
	\begin{proof}
		\pf\ If $\alpha < \beta \in \aleph_x$ then we have $\alpha \preccurlyeq \beta \preccurlyeq x$ and so $\alpha \preccurlyeq x$ and $\alpha \in \aleph_x$.
	\end{proof}
	\step{b}{$\aleph_x$ is a transitive set of ordinals.}
	\qedstep
	\begin{proof}
		\pf\ Corollary \ref{cor:transords}.
	\end{proof}
\end{proof}
\step{4}{For all $\alpha < \aleph_x$ we have $\alpha \not\approx \aleph_x$.}
\qed
\end{proof}

\begin{df}[Aleph (ZF without Foundation)]
Define the cardinal number $\aleph_\alpha$ for every ordinal $\alpha$ by transfinite recursion as follows:
\begin{align*}
\aleph_0 & := \omega \\
\aleph_{\alpha + 1} & := \aleph(\aleph_\alpha) \\
\aleph_\lambda & := \bigcup_{\beta < \lambda} \aleph_\beta & (\lambda \text{ a limit ordinal})
\end{align*}
\end{df}

\begin{thm}[ZF without Foundation]
For any initial ordinal $\alpha \geq \omega$, we have $\alpha \times \alpha \approx \alpha$.
\end{thm}

\begin{proof}
\pf
\step{1}{\pflet{$\alpha$ be an initial ordinal.}}
\step{2}{\assume{as induction hypothesis for every initial ordinal $\beta$ with $\omega \leq \beta < \alpha$ we have $\beta \times \beta \approx \beta$}}
\step{3}{\pflet{$r$ be the relation on $\alpha \times \alpha$ defined by $(\delta_1, \delta_2) r (\gamma_1, \gamma_2)$ iff $\max(\delta_1, \delta_2) < \max(\gamma_1, \gamma_2)$ or $(\max(\delta_1, \delta_2) = \max(\gamma_1, \gamma_2) \wedge \delta_1 < \gamma_1)$, or $(\max(\delta_1, \delta_2) = \max(\gamma_1, \gamma_2) \wedge \delta_1 = \gamma_1 \wedge \delta_2 < \gamma_2)$}}
\step{4}{$r$ is a well ordering of $\alpha \times \alpha$}
\begin{proof}
	\step{a}{$r$ is transitive.}
	\step{b}{Every nonempty subset of $\alpha \times \alpha$ has an $r$-least element.}
\end{proof}
\step{5}{\pflet{$g : (\alpha \times \alpha, r) \cong (\xi, \in)$ be the isomorphism with an ordinal $\xi$}}
\begin{proof}
	\pf\ Mostowski's Isomorphism Theorem.
\end{proof}
\step{6}{$\alpha \leq \xi$}
\step{7}{$\alpha = \xi$}
\begin{proof}
	\step{a}{\assume{for a contradiction $\alpha < \xi$}}
	\step{b}{\pick\ $\eta, \zeta < \alpha$ such that $\alpha = g(\eta, \zeta)$}
	\step{c}{\pflet{$\delta = \max(\eta, \zeta)$}}
	\step{d}{\pflet{$\beta$ be the initial ordinal equinumerous with $\delta$.}}
	\step{e}{$\omega \leq \beta$}
	\begin{proof}
		\pf\ If $\eta$ and $\zeta$ are natural numbers then $g(\eta, \zeta) = \alpha$ is a natural number.
	\end{proof}
	\step{f}{$\beta < \alpha$}
	\begin{proof}
		\pf\ $\beta \leq \delta < \alpha$
	\end{proof}
	\step{g}{\pflet{$y = \{ (\eta', \zeta') \mid (\eta', \zeta') r (\eta, \zeta) \}$}}
	\step{h}{The image of $y$ under $g$ is $\alpha$}
	\step{i}{$y \subseteq (\delta + 1) \times (\delta + 1)$}
	\step{j}{$y \preccurlyeq \beta \times \beta$}
	\step{k}{$\beta \approx \beta \times \beta$}
	\begin{proof}
		\pf\ \stepref{2}
	\end{proof}
	\step{l}{$\alpha \preccurlyeq \beta$}
	\qedstep
\end{proof}
\qed
\end{proof}

\begin{cor}[ZF without Foundation]
\label{cor:alpha_times_alpha_approx_alpha}
For any ordinal $\alpha \geq \omega$ we have $\alpha \times \alpha \approx \alpha$.
\end{cor}

\begin{proof}
\pf
\step{1}{\pflet{$\alpha$ be an ordinal.}}
\step{2}{\pflet{$\kappa$ be the initial ordinal such that $\alpha \approx \kappa$}}
\step{3}{$\alpha \approx \alpha \times \alpha$}
\begin{proof}
	\pf
	\begin{align*}
		\alpha & \approx \kappa \\
		& \approx \kappa \times \kappa \\
		& \approx \alpha \times \alpha
	\end{align*}
\end{proof}
\qed
\end{proof}

\begin{prop}[SF without Foundation]
\label{prop:alpha_plus_alpha_approx_alpha}
For any ordinal $\alpha \geq \omega$ we have $\alpha +_o \alpha \approx \alpha$, where $+_o$ denotes ordinal addition.
\end{prop}

\begin{proof}
\pf\ Since $\alpha +_o \alpha \approx \alpha \times 2 \preccurlyeq \alpha \times \alpha \approx \alpha$. \qed
\end{proof}

\begin{df}[Cardinality]
For any set $A$, the \emph{cardinality} or \emph{cardinal number} $|A|$ of $A$ is $\bigcap \{ \alpha \in \mathbf{On} \mid \alpha \approx A \}$.
\end{df}

\begin{prop}[SFC without Foundation]
For any set $A$, we have $|A|$ is an initial ordinal.
\end{prop}

\begin{prop}[S without Foundation]
For any initial ordinal $\alpha$, we have $|\alpha| = \alpha$.
\end{prop}

\begin{prop}[SFC without Foundation]
For any sets $A$ and $B$, we have $A \approx B$ iff $|A| = |B|$.
\end{prop}

\begin{proof}
\pf\ Easy. \qed
\end{proof}

\begin{df}[Addition]
Given cardinal numbers $\kappa$ and $\lambda$, we define $\kappa + \lambda$ to be $|A \cup B|$ where $A$ and $B$ are disjoint sets of cardinality $\kappa$ and $\lambda$ respectively.

We prove this is well-defined.
\end{df}

\begin{proof}
\pf
\step{1}{\assume{$A \approx A'$, $B \approx B'$, and $A \cap B = A' \cap B' = \emptyset$}}
\step{2}{\pick\ bijections $f : A \approx A'$ and $g : B \approx B'$}
\step{3}{The function $A \cup B \rightarrow A' \cup B'$ that maps $a \in A$ to $f(a)$ and $b \in B$ to $g(b)$ is a bijection.}
\qed
\end{proof}

\begin{prop}
For any cardinal number $\kappa$, we have $\kappa + 0 = \kappa$.
\end{prop}

\begin{proof}
\pf\ Let $A$ and $B$ be disjoint sets of cardinality $\kappa$ and 0. Then $B = \emptyset$ so $A \cup B = A$ and so $|A \cup B| = \kappa$. \qed
\end{proof}

\begin{thm}[Associative Law for Addition]
For any cardinal numbers $\kappa$, $\lambda$, $\mu$ we have $\kappa + (\lambda + \mu) = (\kappa + \lambda) + \mu$.
\end{thm}

\begin{proof}
\pf\ Since $A \cup (B \cup C) = (A \cup B) \cup C$. \qed
\end{proof}

\begin{prop}
For any cardinal numbers $\kappa$ and $\lambda$ we have $\kappa + \lambda = \lambda + \kappa$.
\end{prop}

\begin{proof}
\pf\ Since $A \cup B = B \cup A$. \qed
\end{proof}

\begin{prop}[ZF without Foundation]
For ordinals $\alpha, \beta \geq \omega$ we have $|\alpha| + |\beta| = |\max(\alpha, \beta)|$.
\end{prop}

\begin{proof}
\pf\ Since the cardinality of the concatenation of $\alpha$ and $\beta$ is $|\alpha +_o \beta| = |\alpha| + |\beta|$, and we have
\begin{align*}
|\alpha| + |\beta| & = |\alpha +_o \beta| \\
& \leq |\max(\alpha, \beta) +_o \max(\alpha, \beta)| \\
& = |\max(\alpha,\beta)| & (\text{Proposition \ref{prop:alpha_plus_alpha_approx_alpha}}) \qed
\end{align*}
\end{proof}

\begin{df}[Addition of a Family of Cardinals (SFC without Foundation)]
Let $\{\kappa_i\}_{i \in I}$ be a family of cardinal numbers. Choose pairwise disjoint sets $A_i$ for $i \in I$ with $|A_i| = \kappa_i$. Then the sum $\sum_{i \in I} \kappa_i$ is defined to be $\left| \bigcup_{i \in I} A_i \right|$.
\end{df}

\begin{df}[Multiplication]
For $\kappa$ and $\lambda$ cardinal numbers, we define $\kappa \lambda$ to be the cardinal number of $A \times B$, where $|A| = \kappa$ and $|B| = \lambda$.

We prove this is well-defined.
\end{df}

\begin{proof}
\pf\ If $f : A \approx A'$ and $g : B \approx B'$ then the function that maps $(a,b)$ to $(f(a),g(b))$ is a bijection $A \times B \approx A' \times B'$. \qed
\end{proof}

\begin{prop}
For any cardinal number $\kappa$ we have $\kappa \cdot 0 = 0$.
\end{prop}

\begin{proof}
\pf\ Since $A \times \emptyset = \emptyset$. \qed
\end{proof}

\begin{prop}
\label{prop:multone}
For any cardinal number $\kappa$ we have $\kappa \cdot 1 = \kappa$.
\end{prop}

\begin{proof}
\pf\ The function that maps $(a,e)$ to $a$ is a bijection $A \times \{e\} \approx A$. \qed
\end{proof}

\begin{thm}[Distributive Law]
For any cardinal numbers $\kappa$, $\lambda$ and $\mu$, we have $\kappa(\lambda + \mu) = \kappa \lambda + \kappa \mu$.
\end{thm}

\begin{proof}
\pf\ Since $A \times (B \cup C) = (A \times B) \cup (A \times C)$. \qed
\end{proof}

\begin{thm}[Associative Law for Multiplication]
For any cardinal numbers $\kappa$, $\lambda$ and $\mu$, we have $\kappa (\lambda \mu) = (\kappa \lambda) \mu$.
\end{thm}

\begin{proof}
\pf\ Since $A \times (B \times C) \approx (A \times B) \times C$. \qed
\end{proof}

\begin{thm}[Commutative Law for Multiplication]
For any cardinal numbers $\kappa$ and $\lambda$, we have $\kappa \lambda = \lambda \kappa$.
\end{thm}

\begin{proof}
\pf\ Since $A \times B \approx B \times A$. \qed
\end{proof}

\begin{thm}
For any cardinal numbers $\kappa$ and $\lambda$, if $\kappa \lambda = 0$ then $\kappa = 0$ or $\lambda = 0$.
\end{thm}

\begin{proof}
\pf\ if $A \times B = \emptyset$ then $A = \emptyset$ or $B = \emptyset$. \qed
\end{proof}

\begin{prop}[SF without Foundation]
For any ordinals $\alpha, \beta \geq \omega$, we have $|\alpha||\beta| = |\max(\alpha, \beta)|$.
\end{prop}

\begin{proof}
\pf
\begin{align*}
|\alpha||\beta| & = |\alpha \times \beta| \\
& \leq |\max(\alpha,\beta) \times \max(\alpha, \beta)| \\
& = |\max(\alpha, \beta)| & (\text{Corollary \ref{cor:alpha_times_alpha_approx_alpha}}) \qed
\end{align*}
\end{proof}

\begin{df}[Multiplication (SFC without Foundation)]
Let $\{ \kappa_i \}_{i \in I}$ be a family of cardinal numbers. Choose a set $A_i$ with $|A_i| = \kappa_i$ for all $i$. The \emph{product} $\prod_{i \in I} \kappa_i$ is defined to be the cardinality of the Cartesian product of $\{A_i\}_{i \in I}$.
\end{df}

\begin{df}[Exponentiation]
Given cardinal numbers $\kappa$ and $\lambda$, we define $\kappa^\lambda$ to be $|A^B|$, where $|A| = \kappa$ and $|B| = \lambda$.

We prove this is well-defined.
\end{df}

\begin{proof}
\pf
If $f : A \approx A'$ and $g : B \approx B'$, then the function that maps $h : B \rightarrow A$ to $f \circ h \circ g^{-1}$ is a bijection $A^B \approx A'^{B'}$. \qed
\end{proof}

\begin{prop}
For any cardinal numbers $\kappa$, $\lambda$ and $\mu$,
\[ \kappa^{\lambda + \mu} = (\kappa^{\lambda})^\mu \]
\end{prop}

\begin{proof}
\pf\ The function that maps $f : A \times B \rightarrow C$ to $\lambda a \in A. \lambda b \in B. f(a,b)$ is a bijection $A^{B \times C} \approx (A^B)^C$. \qed
\end{proof}

\begin{prop}
For any cardinal numbers $\kappa$, $\lambda$ and $\mu$,
\[ (\kappa \lambda)^\mu = \kappa^\mu \lambda^\mu \enspace . \]
\end{prop}

\begin{proof}
\pf\ The function $f : A^C \times B^C \rightarrow (A \times B)^C$ with $f(g,h)(c) = (g(c),h(c))$ is a bijection. \qed
\end{proof}

\begin{prop}
For any cardinal numbers $\kappa$, $\lambda$ and $\mu$, we have
\[ \kappa^{\lambda + \mu} = \kappa^\lambda \kappa^\mu \enspace .\]
\end{prop}

\begin{proof}
\pf\ If $B \cap C = \emptyset$, then $f : A^B \times A^C \rightarrow A^{B \cup C}$ given by $f(g,h)(b) = g(b)$ and $f(g,h)(c) = h(c)$ is a bijection. \qed
\end{proof}

\begin{prop}
For any cardinal number $\kappa$, we have $\kappa^0 = 1$.
\end{prop}

\begin{proof}
\pf\ For any set $A$, we have $A^\emptyset = \{ \emptyset \}$. \qed
\end{proof}

\begin{prop}
For any cardinal number $\kappa$, we have $\kappa^1 = \kappa$.
\end{prop}

\begin{proof}
\pf\ For any sets $A$ and $B$, if $B = \{b\}$ then the function $f : A \rightarrow A^B$ with $f(a)(b) = a$ is a bijection. \qed
\end{proof}

\begin{prop}
For any non-zero cardinal number $\kappa$ we have $0^\kappa = 0$.
\end{prop}

\begin{proof}
\pf\ If $A$ is nonempty then there is no function $A \rightarrow \emptyset$. \qed
\end{proof}

\begin{prop}
For any set $A$ we have $|\mathcal{P} A| = 2^{|A|}$.
\end{prop}

\begin{proof}
\pf\ The function $f : \mathcal{P} A \rightarrow 2^A$ where $f(X)(a) = 0$ if $a \notin X$ and $f(X)(a) = 1$ if $a \in X$. \qed
\end{proof}

\begin{thm}[K\"{o}nig]
Let $I$ be a set. Let $\{A_i\}_{i \in I}$ and $\{B_i\}_{i \in I}$ be families of sets. Assume that $\forall i \in I. |A_i| < |B_i|$. Then $\left| \bigcup_{i \in I} A_i \right| < \left| \prod_{i \in I} B_i \right|$.
\end{thm}

\begin{proof}
\pf
\step{1}{For all $i \in I$, choose an injection $f_i : A_i \rightarrowtail B_i$}
\step{2}{For all $i \in I$, choose $b_i \in B_i - f_i(A_i)$}
\step{3}{$\left| \bigcup_{i \in I} A_i \right| \leq \left| \prod_{i \in I} B_i \right|$}
\begin{proof}
	\step{a}{Define $g : \bigcup_{i \in I} A_i \rightarrow \prod_{i \in I} B_i$ by
	\[ g(i,a)(j) = \begin{cases}
	f_i(a) & \text{if } i = j \\
	b_j & \text{otherwise}
	\end{cases} \]}
	\step{b}{$g$ is injective.}
\end{proof}
\step{2}{$\left| \bigcup_{i \in I} A_i \right| \neq \left| \prod_{i \in I} B_i \right|$}
\begin{proof}
	\step{a}{\pflet{$h : \bigcup_{i \in I} A_i \rightarrow \prod_{i \in I} B_i$} \prove{$h$ is not surjective.}}
	\step{b}{For $i \in I$, \pick\ $c_i \in B_i - \{ h(i,a)(i) \mid i \in I \}$}
	\step{c}{$c \in \prod_{i \in I} B_i$}
	\step{d}{$c \notin \ran h$}
\end{proof}
\qed
\end{proof}

\begin{cor}
For any cardinal number $\kappa$ we have $\kappa < 2^\kappa$.
\end{cor}

\section{Ordering on Cardinal Numbers}

\begin{df}
Given cardinal numbers $\kappa$ and $\lambda$, we have $\kappa \leq \lambda$ iff $A \preccurlyeq B$, where $|A| = \kappa$ and $|B| = \lambda$.
\end{df}

\begin{proof}
\pf
\step{1}{\pflet{$|A| = \kappa$ and $|B| = \lambda$}}
\step{2}{\pick\ bijections $f : A \approx \kappa$ and $g : B \approx \lambda$}
\step{3}{If $\kappa \leq \lambda$ then $A \preccurlyeq B$}
\begin{proof}
	\pf\ Let $i : \kappa \hookrightarrow \lambda$ be the inclusion. Then $g^{-1} \circ i \circ f$ is an injection $A \rightarrow B$.
\end{proof}
\step{4}{If $A \preccurlyeq B$ then $\kappa \leq \lambda$}
\begin{proof}
	\step{a}{\assume{$A \preccurlyeq B$}}
	\step{b}{\pick\ an injection $h : A \rightarrowtail B$}
	\step{c}{$g(h(A)) \subseteq B$ is well-ordered by $\in$}
	\step{d}{\pflet{$\gamma$ be the ordinal number of $(g(h(A)), \in)$}}
	\step{e}{$\gamma \leq \lambda$}
	\begin{proof}
		\pf\ Proposition \ref{prop:subsetleq}.
	\end{proof}
	\step{f}{$\kappa \leq \gamma$}
	\begin{proof}
		\pf\ By the leastness of $\kappa$, since $A$ is equinumerous with $\gamma$.
	\end{proof}
	\step{g}{$\kappa \leq \lambda$}
\end{proof}
\qed
\end{proof}

\begin{cor}
There is no largest cardinal number.
\end{cor}

\begin{prop}
For any cardinal numbers $\kappa$, $\lambda$, $\mu$, if $\kappa \leq \lambda$ then $\kappa + \mu \leq \lambda + \mu$.
\end{prop}

\begin{proof}
\pf\ If $f : A \rightarrow B$ is injective, and $A \cap C = B \cap C = \emptyset$, then the function $A \cup C \rightarrow B \cup C$ that maps $a$ to $f(a)$ and maps $c$ to $c$ is an injection. \qed
\end{proof}

\begin{prop}
For any cardinal numbers $\kappa$, $\lambda$, $\mu$, if $\kappa \leq \lambda$ then $\kappa \mu \leq \lambda \mu$.
\end{prop}

\begin{proof}
\pf\ If $f : A \rightarrow B$ is injective, then the function $A \times C \rightarrow B \times C$ that maps $(a,c)$ to $(f(a),c)$ is injective. \qed
\end{proof}

\begin{prop}
For any cardinal numbers $\kappa$, $\lambda$, $\mu$, if $\kappa \leq \lambda$ then $\kappa^\mu \leq \lambda^\mu$.
\end{prop}

\begin{proof}
\pf\ Given an injection $f : A \rightarrow B$, the function that maps $A^C \rightarrow B^C$ that maps $g$ to $f \circ g$ is an injection. \qed
\end{proof}

\begin{prop}
For any cardinal numbers $\kappa$, $\lambda$, $\mu$, if $\kappa \leq \lambda$ and $\mu$ and $\kappa$ are not both 0, then $\mu^\kappa \leq \mu^\lambda$.
\end{prop}

\begin{proof}
\pf
\step{1}{\pflet{$A$, $B$ and $C$ be sets with $A$ and $C$ not both empty.}}
\step{2}{\pflet{$f : A \rightarrow B$ be an injection.} \prove{$C^A \preccurlyeq C^B$}}
\step{3}{\case{$C = \emptyset$}}
\begin{proof}
	\pf\ Then $A \neq \emptyset$ so $C^A = \emptyset \preccurlyeq C^B$.
\end{proof}
\step{4}{\case{$C \neq \emptyset$}}
\begin{proof}
	\step{a}{\pick\ $c \in C$}
	\step{b}{\pflet{$g : C^A \rightarrow C^B$ be the function $g(h)(f(a)) = h(a)$, $g(h)(b) = c$ if $b \notin f(A)$}}
	\step{c}{$g$ is an injection.}
\end{proof}
\qed
\end{proof}

\begin{prop}
\label{prop:unioncard}
Let $\mathcal{A}$ be a set such that $\forall X \in \mathcal{A}. |X| \leq \kappa$. Then
\[ \left| \bigcup \mathcal{A} \right| \leq |\mathcal{A}| \kappa \enspace . \]
\end{prop}

\begin{proof}
\pf
\step{a}{For $X \in \mathcal{A}$, choose a surjection $f_X : \kappa \rightarrow X$.}
\step{b}{Define $g : \mathcal{A} \times \kappa \rightarrow \bigcup \mathcal{A}$ by $g(X,\alpha) = f_X(\alpha)$}
\step{c}{$g$ is surjective.}
\qed
\end{proof}

\begin{lm}
\label{lm:supcard}
The union of a set of cardinal numbers is a cardinal number.
\end{lm}

\begin{proof}
\pf
\step{1}{\pflet{$A$ be a set of cardinal numbers.} 
\prove{$\bigcup A$ is the smallest ordinal equinumerous with $\bigcup A$}}
\step{2}{\pflet{$\alpha < \bigcup A$} \prove{$\alpha \not\approx \bigcup A$}}
\step{3}{\pick\ $\kappa \in A$ such that $\alpha < \kappa$}
\step{4}{$\alpha \prec \kappa$}
\step{5}{$\alpha \prec \bigcup A$}
\qed
\end{proof}

\section{Adjoints}

\begin{df}[Adjoint (SF)]
Let $\mathfrak{m}$ and $\mathfrak{n}$ be cardinal numbers. Then $\mathfrak{n}$ is an \emph{adjoint} of $\mathfrak{m}$ iff $\mathfrak{m} < \mathfrak{n}$, for every cardinal number $\mathfrak{p}$, if $\mathfrak{m} \leq \mathfrak{p} \leq \mathfrak{n}$ then $\mathfrak{m} = \mathfrak{p}$ or $\mathfrak{n} = \mathfrak{p}$.
\end{df}

\begin{prop}[SF]
Let $\mathfrak{n}$ be a cardinal number.
If $\omega \not\leq \mathfrak{n}$, then $\mathfrak{n+1}$ is an adjoint of $\mathfrak{n}$.
\end{prop}

\begin{proof}
\pf
\step{1}{$\mathfrak{n} < \mathfrak{n}+1$}
\begin{proof}
	\step{a}{\assume{$\mathfrak{n} = \mathfrak{n} + 1$} \prove{$\omega \leq \mathfrak{n}$}}
	\step{b}{\pick\ a set $A$ such that $\card A = \mathfrak{n}$ and an object $e \notin A$}
	\step{c}{\pick\ a bijection $f : A \cup \{e\} \approx A$}
	\step{d}{Define $g : \mathbb{N} \rightarrowtail A$ recursively by $g(0) = e$ and $g(n+1) = f(g(n))$ for all $n \in \mathbb{N}$.}
\end{proof}
\step{2}{For every cardinal number $\mathfrak{p}$, if $\mathfrak{n} \leq \mathfrak{p} \leq \mathfrak{n} + 1$, then $\mathfrak{n} = \mathfrak{p}$ or $\mathfrak{n} + 1 = \mathfrak{p}$.}
\begin{proof}
	\step{a}{\assume{$\mathfrak{n} \leq \mathfrak{p} \leq \mathfrak{n} + 1$}}
	\step{b}{\pick\ sets $A$ with $\card A = \mathfrak{n}$ and $B$ with $\card B = \mathfrak{p}$}
	\step{c}{\pick\ an object $e \notin A$}
	\step{d}{\pick\ injections $f : A \rightarrowtail B$ and $g : B \rightarrowtail A \cup \{e\}$}
	\step{e}{\assume{$\mathfrak{n} \neq \mathfrak{p}$} \prove{$\mathfrak{n} + 1 = \mathfrak{p}$}}
	\step{f}{\pick\ $b \in B - f(A)$}
	\step{g}{$A \cup \{e\} \preccurlyeq B$}
	\step{h}{$B \approx A \cup \{e\}$}
	\begin{proof}
		\pf\ Schr\"{o}der-Bernstein
	\end{proof}
\end{proof}
\qed
\end{proof}

\chapter{Natural Numbers}

\section{Inductive Sets}

\begin{prop}[Dependent Choice]
Let $A$ be a nonempty set and $R$ a relation on $A$ such that $\forall x \in A. \exists y \in A. (y,x) \in R$. Then there exists a function $f : \mathbb{N} \rightarrow A$ such that $\forall n \in \mathbb{N}. (f(n+1),f(n)) \in R$.
\end{prop}

\begin{proof}
\pf
\step{1}{\pick\ a choice function $F$ for $A$.}
\step{2}{\pick\ $a \in A$}
\step{3}{Define $f : \mathbb{N} \rightarrow A$ by $f(0) = a$ and $f(n+1) = F(\{ y \in A \mid (y,f(n)) \in R \})$.}
\qed
\end{proof}

\begin{thms}
For any classes $\mathbf{A}$ and $\mathbf{R}$, the following is a theorem:


Assume $\mathbf{R}$ is a relation on $\mathbf{A}$ and, for all $a \in \mathbf{A}$, the class $\{ x \in \mathbf{A} \mid x \mathbf{R} a \}$ is a set. Then $\mathbf{R}$ is well founded if and only if there does not exist a function $f : \mathbb{N} \rightarrow \mathbf{A}$ such that $\forall n \in \mathbb{N}. f(n+1) \mathbf{R} f(n)$.
\end{thms}

\begin{proof}
\pf
\step{1}{If there exists a function $f : \mathbb{N} \rightarrow \mathbf{A}$ such that $\forall n \in \mathbb{N}. f(n+1) \mathbf{R} f(n)$ then $\mathbf{R}$ is not well founded.}
\begin{proof}
	\pf\ $f(\mathbb{N})$ is a nonempty subset of $\mathbf{A}$ with no $\mathbf{R}$-minimal element.
\end{proof}
\step{2}{If $\mathbf{R}$ is not well founded then there exists a function $f : \mathbb{N} \rightarrow \mathbf{A}$ such that $\forall n \in \mathbb{N}. f(n+1) \mathbf{R} f(n)$.}
\begin{proof}
	\step{a}{\assume{$\mathbf{R}$ is not well founded.}}
	\step{b}{\pick\ a nonempty subset $B \subseteq \mathbf{A}$ that has no $\mathbf{R}$-minimal element.}
	\step{c}{$\forall x \in B. \exists y \in B. y \mathbf{R} x$}
	\step{d}{Choose a function $g : B \rightarrow B$ such that $\forall x \in B. g(x) \mathbf{R} x$}
	\step{e}{\pick\ $b \in B$}
	\step{f}{Define $f : \mathbb{N} \rightarrow \mathbf{A}$ recursively by $f(0) = b$ and $\forall n \in \mathbb{N}. f(n+1) = g(f(n))$}
	\step{g}{$\forall n \in \mathbb{N}. f(n+1) \mathbf{R} f(n)$}
\end{proof}
\qed
\end{proof}

\section{Cardinality}


\begin{cor}
No finite set is equinumerous to a proper subset of itself.
\end{cor}

\begin{cor}
Every natural number is a cardinal number.
\end{cor}

\begin{proof}
\pf\ For any natural number $n$, we have that $n$ is the least ordinal such that $n \approx n$. \qed
\end{proof}

\begin{cor}
$\mathbb{N}$ is a cardinal number.
\end{cor}

\begin{cor}
$\mathbb{N}$ is infinite.
\end{cor}

\begin{proof}
\pf\ The function that maps $n$ to $n+1$ is a bijection between $\mathbb{N}$ and $\mathbb{N} - \{0\}$. \qed
\end{proof}

\begin{cor}
If $C$ is a proper subset of a natural number $n$, then there exists $m < n$ such that $C \approx m$.
\end{cor}

\begin{proof}
\pf\ By Proposition \ref{prop:subsetleq}. \qed
\end{proof}

\begin{cor}
Any subset of a finite set is finite.
\end{cor}

\begin{prop}
For any natural numbers $m$ and $n$ we have $m+n$ (cardinal addition) is a natural number.
\end{prop}

\begin{proof}
\pf\ Induction on $n$. \qed
\end{proof}

\begin{cor}
The union of two finite sets is finite.
\end{cor}

\begin{cor}
\label{cor:finiteunion}
The union of a finite set of finite sets is finite.
\end{cor}

\begin{proof}
\pf\ By induction on the number of elements. \qed
\end{proof}

\begin{prop}
For natural numbers $m$ and $n$, the cardinal sum $m + n$ is equal to the ordinal sum $m + n$.
\end{prop}

\begin{proof}
\pf\ Induction on $n$. \qed
\end{proof}

\begin{prop}
For any natural numbers $m$ and $n$, we have $mn$ (cardinal multiplication) is a natural number.
\end{prop}

\begin{cor}
If $A$ and $B$ are finite sets then $A \times B$ is finite.
\end{cor}

\begin{prop}
For natural numbers $m$ and $n$, the cardinal product $mn$ is equal to the ordinal product $mn$.
\end{prop}

\begin{proof}
\pf\ Induction on $n$. \qed
\end{proof}

\begin{prop}
For any natural numbers $m$ and $n$ we have $m^n$  (cardinal exponentiation) is a natural number.
\end{prop}

\begin{proof}
\pf\ Induction on $n$.
\end{proof}

\begin{cor}
If $A$ and $B$ are finite sets then $A^B$ are finite.
\end{cor}

\begin{prop}
For natural numbers $m$ and $n$, the cardinal exponentiation $m^n$ and the ordinal exponentiation $m^n$ agree.
\end{prop}

\begin{proof}
\pf\ Induction on $n$. \qed
\end{proof}

\begin{prop}
\label{prop:N^2approxN}
$\mathbb{N}^2 \approx \mathbb{N}$
\end{prop}

\begin{proof}
\pf\ The function $J : \mathbb{N}^2 \rightarrow \mathbb{N}$ defined by $J(m,n) = ((m+n)^2 + 3m + n)/2$ is a bijection. \qed
\end{proof}

\begin{prop}
\label{prop:aleph0least}
For any infinite cardinal $\kappa$ we have $\aleph_0 \leq \kappa$.
\end{prop}

\begin{proof}
\pf
\step{1}{\pflet{$A$ be an infinite set.} \prove{$\mathbb{N} \preccurlyeq A$}}
\step{2}{\pick\ a choice function $F$ for $A$.}
\step{3}{Define $h : \mathbb{N} \rightarrow \{ X \in \mathcal{P} A \mid X \text{ is finite} \}$ by
\begin{align*}
h(0) & = \emptyset \\
h(n+1) & = h(n) \cup \{ F(A - \{ h(m) \mid m < n \}) \}
\end{align*}}
\step{4}{Define $g : \mathbb{N} \rightarrow A$ by $g(n) = F(A - \{ h(m) \mid m < n \})$}
\step{5}{$g$ is injective.}
\begin{proof}
	\pf\ If $m < n$ then $g(m) \neq g(n)$.
\end{proof}
\qed
\end{proof}

\begin{thms}[K\"{o}nig's Lemma]
For any classes $\mathbf{A}$ and $\mathbf{R}$, the following is a theorem:

Assume $\mathbf{R}$ is a well founded relation on $\mathbf{A}$ such that, for all $y \in \mathbf{A}$, the class $\{ x \in \mathbf{A} \mid x \mathbf{R} y \}$ is a finite set. Let $\mathbf{R}^t$ be the transitive closure of $\mathbf{R}$. Then,for all $y \in \mathbf{A}$, the class $\{ x \in \mathbf{A} \mid x \mathbf{R}^t y \}$ is a finite set.
\end{thms}

\begin{proof}
\pf
\step{1}{\pflet{$y \in \mathbf{A}$}}
\step{2}{\assume{as transfinite induction hypothesis $\forall x \mathbf{R} y. \{ z \in \mathbf{A} \mid z \mathbf{R}^t x \}$ is a finite set.}}
\step{3}{$\{ x \mid x \mathbf{R}^t y \} = \bigcup_{x \mathbf{R} y} (\{x\} \cup \{z \mid z \mathbf{R}^t x \}$}
\step{4}{$\{ x \mid x \mathbf{R}^t y\}$ is finite.}
\begin{proof}
	\pf\ Corollary \ref{cor:finiteunion}.
\end{proof}
\qed
\end{proof}

\section{Countable Sets}

\begin{df}[Countable]
A set $A$ is \emph{countable} iff $|A| \leq \aleph_0$.
\end{df}

\begin{thm}
The union of a countable set of countable sets is countable.
\end{thm}

\begin{proof}
\pf\ Proposition \ref{prop:unioncard}. \qed
\end{proof}

\section{Arithmetic}

\begin{df}[Even]
A natural number $n$ is \emph{even} iff there exists $m \in \mathbb{N}$ such that $n = 2m$.
\end{df}

\begin{df}[Odd]
A natural number $n$ is \emph{odd} iff there exists $p \in \mathbb{N}$ such that $n = 2p+1$.
\end{df}

\begin{prop}
Every natural number is either even or odd.
\end{prop}

\begin{proof}
\pf
\step{1}{$0$ is even.}
\begin{proof}
	\pf\ $0 = 2 \times 0$.
\end{proof}
\step{2}{For any natural number $n$, if $n$ is either even or odd then $n^+$ is either even or odd.}
\begin{proof}
	\pf
	\step{a}{\pflet{$n \in \mathbb{N}$}}
	\step{b}{If $n$ is even then $n^+$ is odd.}
	\begin{proof}
		\pf\ If $n = 2p$ then $n^+ = 2p+1$.
	\end{proof}
	\step{c}{If $n$ is odd then $n^+$ is even.}
	\begin{proof}
		\pf\ If $n = 2p+1$ then $n^+ = 2(p+1)$.
	\end{proof}
\end{proof}
\qed
\end{proof}

\begin{prop}
No natural number is both even and odd.
\end{prop}

\begin{proof}
\pf
\step{1}{0 is not odd.}
\begin{proof}
	\pf\ For any $p$ we have $2p+1 = (2p)^+ \neq 0$.
\end{proof}
\step{2}{For any natural number $n$, if $n$ is not both even and odd, then $n^+$ is not both even and odd.}
\begin{proof}
	\step{a}{\pflet{$n$ be a natural number.}}
	\step{b}{If $n^+$ is even then $n$ is odd.}
	\begin{proof}
		\step{i}{\assume{$n^+$ is even.}}
		\step{ii}{\pick\ $p$ such that $n^+ = 2p$}
		\step{iii}{$p \neq 0$}
		\begin{proof}
			\pf\ Since $n^+ \neq 0$.
		\end{proof}
		\step{iv}{\pick\ $q$ such that $p = q^+$}
		\begin{proof}
			\pf\ Theorem \ref{thm:zeroorsucc}.
		\end{proof}
		\step{vi}{$n^+ = 2q + 2$}
		\begin{proof}
			\pf\ \stepref{ii}, \stepref{iv}.
		\end{proof}
		\step{vii}{$n = 2q+1$}
		\begin{proof}
			\pf\ Proposition \ref{prop:Peano2}, \stepref{vi}
		\end{proof}
		\step{viii}{$n$ is odd.}
	\end{proof}
	\step{c}{If $n^+$ is odd then $n$ is even.}
	\begin{proof}
		\step{i}{\assume{$n^+$ is odd.}}
		\step{ii}{\pick\ $p$ such that $n^+ = 2p+1$}
		\step{iii}{$n = 2p$}
		\begin{proof}
			\pf\ Proposition \ref{prop:Peano2}, \stepref{ii}
		\end{proof}
		\step{iv}{$n$ is even.}
	\end{proof}
\end{proof}
\qed
\end{proof}

\begin{prop}
\label{prop:intltlemma}
Let $m$, $n$, $p$, $q$ be natural numbres. Assume $m + n = p + q$. Then $m < p$ if and only if $q < n$.
\end{prop}

\begin{proof}
\pf
\step{1}{If $m < p$ then $q < n$.}
\begin{proof}
	\pf\ If $m < p$ and $n \leq q$ then $m+n < p + q$.
\end{proof}
\step{2}{If $q < n$ then $m < p$.}
\begin{proof}
	\pf\ Similar.
\end{proof}
\qed
\end{proof}

\begin{prop}
\label{prop:intmultlemma}
Let $m$, $n$, $p$ and $q$ be natural numbers. Assume $n < m$ and $q < p$. Then
\[ mq + np < mp + nq \enspace . \]
\end{prop}

\begin{proof}
\pf
\step{1}{\pick\ positive natural numbers $a$ and $b$ such that $m = n + a$ and $p = q + b$.}
\step{2}{$mp + nq > mq + np$}
\begin{proof}
	\pf
	\begin{align*}
		mp + nq & = (n+a) (q + b) + nq \\
		& = 2nq + nb + aq + ab \\
		mq + np & = (n + a) q + n (q + b) \\
		& = 2nq + aq + nb \\
		\therefore mp + nq & = mq + np + ab \\
		& > mq + np
	\end{align*}
\end{proof}
\qed
\end{proof}

\section{Sequences}

\begin{df}[Sequence]
Let $A$ be a set. A \emph{finite sequence} in $A$ is a function $a : n \rightarrow A$ for some natural number $n$; we write it as $(a(0), a(1), \ldots, a(n-1))$. An \emph{(infinite) sequence} in $A$ is a function $\mathbb{N} \rightarrow A$.

We write $A^*$ for the set of all finite sequences in $A$.
\end{df}

\begin{prop}
If $A$ is countable then $A^*$ is countable.
\end{prop}

\begin{proof}
\pf\ For any $n$, the set $A^n$ is countable, and $A^*$ is equinumerous with $\bigcup_n A^n$. \qed
\end{proof}

\section{Transitive Closure of a Set}

\begin{prop}
For any set $A$, there exists a unique transitive set $C$ such that:
\begin{itemize}
\item $A \subseteq C$
\item For any transitive set $X$, if $A \subseteq X$ then $C \subseteq X$
\end{itemize}
\end{prop}

\begin{proof}
\pf
\step{1}{Define a function $F : \mathbb{N} \rightarrow \mathbf{V}$ by
\begin{align*}
F(0) & = A \\
F(n+1) & = A \cup \bigcup(F(0) \cup \cdots \cup F(n))
\end{align*}}
\step{2}{For all $n \in \mathbb{N}$ and $a \in F(n)$ we have $a \subseteq F(n+1)$}
\begin{proof}
	\pf\ $a \in F(0) \cup \cdots \cup F(n)$ so $a \subseteq \bigcup (F(0) \cup \cdots \cup F(n)) \subseteq F(n+1)$.
\end{proof}
\step{3}{\pflet{$C = \bigcup_{n \in \mathbb{N}} F(n)$}}
\step{4}{$C$ is transitive.}
\begin{proof}
	\step{a}{\pflet{$x \in y \in C$}}
	\step{b}{\pick\ $n \in \mathbb{N}$ such that $y \in F(n)$}
	\step{c}{$y \subseteq F(n+1)$}
	\begin{proof}
		\pf\ \stepref{2}
	\end{proof}
	\step{d}{$x \in F(n+1)$}
	\step{e}{$x \in C$}
\end{proof}
\step{5}{$A \subseteq C$}
\begin{proof}
	\pf\ Since $F(0) = A$.
\end{proof}
\step{6}{For any transitive set $X$, if $A \subseteq X$ then $C \subseteq X$}
\begin{proof}
	\step{a}{\pflet{$X$ be a transitive set}}
	\step{b}{\assume{$A \subseteq X$}}
	\step{c}{For all $n \in \mathbb{N}$ we have $F(n) \subseteq X$.}
	\begin{proof}
		\step{i}{$F(0) \subseteq X$}
		\begin{proof}
			\pf\ \stepref{b}
		\end{proof}
		\step{ii}{For all $n \in \mathbb{N}$, if $F(n) \subseteq X$, then $F(n+1) \subseteq X$.}
		\begin{proof}
			\step{A}{\pflet{$n \in \mathbb{N}$}}
			\step{B}{\assume{$\forall m < n. F(m) \subseteq X$}}
			\step{C}{$F(0) \cup \cdots \cup F(n) \subseteq X$}
			\step{D}{$\bigcup (F(0) \cup \cdots \cup F(n)) \subseteq X$}
			\begin{proof}
				\pf\ Since $X$ is transitive.
			\end{proof}
			\step{E}{$F(n+1) \subseteq X$}
		\end{proof}
	\end{proof}
	\step{d}{$C \subseteq X$}
\end{proof}
\step{7}{Let $D$ be a transitive set such that $A \subseteq D$ and, for any transitive set $X$, if $A \subseteq X$ then $D \subseteq X$. Then $D = C$.}
\begin{proof}
	\pf\ We have $C \subseteq D$ and $D \subseteq C$.
\end{proof}
\qed
\end{proof}

\begin{prop}[ZF]
\label{prop:inwellfounded}
For any class $\mathbf{A}$, the relation $\mathbf{E} = \{ (x,y) \in \mathbf{A}^2 \mid x \in y \}$ is well founded.
\end{prop}

\begin{proof}
\pf
\step{1}{Every nonempty set has an $\mathbf{E}$-minimal element.}
\begin{proof}
	\pf\ Axiom of Regularity.
\end{proof}
\step{2}{For every set $x$, there exists a set $u$ such that $x \subseteq u$ and, for all $w$, $y$, if $y \in u$ and $w \mathbf{E} y$ then $w \in u$.}
\begin{proof}
	\pf\ Take $u$ to be the transitive closure of $x$.
\end{proof}
\qed
\end{proof}

\section{The Veblen Fixed Point Theorem}

\begin{thms}[Veblen Fixed Point Theorem]
For any class $\mathbf{T}$, the following is a theorem:

Assume $\mathbf{T}$ is a normal ordinal operation. For every ordinal $\beta$, there exists $\gamma \geq \beta$ such that $\mathbf{T}(\gamma) = \gamma$.
\end{thms}

\begin{proof}
\pf
\step{1}{\pflet{$\beta$ be an ordinal.}}
\step{2}{\assume{w.l.o.g. $\beta < \mathbf{T}(\beta)$}}
\begin{proof}
	\pf\ We have $\beta \leq \mathbf{T}(\beta)$ by Proposition \ref{prop:gammaltTgamma}, and if $\beta = \mathbf{T}(\beta)$ we take $\gamma := \beta$.
\end{proof}
\step{3}{Define $f : \mathbb{N} \rightarrow \mathbf{On}$ by recursion thus:
\begin{align*}
f(0) & = \beta \\
f(n^+) & = \mathbf{T}(f(n))
\end{align*}}
\step{4}{\pflet{$\gamma = \sup_{n \in \mathbb{N}} f(n)$}}
\step{5}{$\beta \leq \gamma$}
\begin{proof}
	\pf\ Since $\beta = f(0)$.
\end{proof}
\step{5}{$\mathbf{T}(\gamma) = \gamma$}
\begin{proof}
	\step{a}{$\mathbf{T}(\gamma) \leq \gamma$}
	\begin{proof}
	\pf
	\begin{align*}
		\mathbf{T}(\gamma) & = \sup_{n \in \mathbb{N}} \mathbf{T}(f(n)) & (\text{Theorem \ref{thm:normalsup}}) \\
		& = \sup_{n \in \mathbb{N}} f(n^+) & (\text{\stepref{3}}) \\
		& \leq \sup_{n \in \mathbb{N}} f(n) \\
		& = \gamma
	\end{align*}
	\end{proof}
	\step{b}{$\gamma \leq \mathbf{T}(\gamma)$}
	\begin{proof}
		\pf Proposition \ref{prop:gammaltTgamma}.
	\end{proof}
\end{proof}
\qed
\end{proof}

\begin{df}[Derived Operation]
Let $\mathbf{T}$ be a normal ordinal operation. The \emph{derived} operation $\mathbf{T}' : \mathbf{On} \rightarrow \mathbf{V}$ is the unique order isomorphism between $\mathbf{On}$ and the fixed points of $\mathbf{T}$.
\end{df}

\begin{props}
For any class $\mathbf{T}$, the following is a theorem:

If $\mathbf{T}$ is a normal ordinal operation, then the derived operation is normal.
\end{props}

\begin{proof}
\pf
\step{1}{For any set $S$ of fixed points of $\mathbf{T}$, we have $\bigcup S$ is a fixed point of $\mathbf{T}$}
\begin{proof}
	\step{a}{\pflet{$S$ be a set of fixed points of $\mathbf{T}$.}}
	\step{b}{$\mathbf{T}(\sup S) = \sup S$}
	\begin{proof}
		\pf
		\begin{align*}
			\mathbf{T}(\sup S) & = \sup_{\alpha \in S} \mathbf{T}(\alpha) & (\text{Theorem \ref{thm:normalsup}}) \\
			& = \sup_{\alpha \in S} \alpha & (\text{\stepref{a}}) \\
			& = \sup S
		\end{align*}
	\end{proof}
\end{proof}
\qedstep
\begin{proof}
	\pf\ Proposition \ref{prop:enumerationnormal}.
\end{proof}
\qed
\end{proof}

\section{Cantor Normal Form}

\begin{thm}[ZF without Foundation]
For any ordinal $\alpha$, there exist a unique sequence of nonzero natural numbers $(n_1, \ldots, n_k)$ and sequence of ordinals $(\gamma_1, \ldots, \gamma_k)$ such that
\[ \gamma_k < \gamma_{k-1} < \cdots < \gamma_1 \]
and
\[ \alpha = \omega^{\gamma_1} n_1 + \omega^{\gamma_2} n_2 + \cdots + \omega^{\gamma_k} n_k \enspace . \]
\end{thm}

\begin{proof}
\pf
\step{1}{For any ordinal $\alpha$, there exist a sequence of nonzero natural numbers $(n_1, \ldots, n_k)$ and sequence of ordinals $(\gamma_1, \ldots, \gamma_k)$ such that
\[ \gamma_k < \gamma_{k-1} < \cdots < \gamma_1 \]
and
\[ \alpha = \omega^{\gamma_1} n_1 + \omega^{\gamma_2} n_2 + \cdots + \omega^{\gamma_k} n_k \enspace . \]}
\begin{proof}
	\step{a}{\pflet{$\alpha$ be an ordinal}}
	\step{b}{\assume{as an induction hypothesis that, for all $\beta < \alpha$, the theorem holds.}}
	\step{c}{\assume{w.l.o.g. $\alpha \neq 0$}}
	\step{c}{\pflet{$\gamma_1$, $n_1$, $\rho_1$ be the unique ordinals such that $0 \neq n_1 < \omega$, $\rho_1 < \omega^{\gamma_1}$, and $\alpha = \omega^{\gamma_1} n_1 + \rho_1$}}
	\step{d}{\pflet{$(\gamma_2, \ldots, \gamma_k)$ and $(n_2, \ldots, n_k)$ be sequences such that $\gamma_k < \gamma_{k-1} < \cdots < \gamma_2$ and $\rho_1 = \omega^{\gamma_2} n_2 + \cdots + \omega^{\gamma_k} n_k$}}
	\step{e}{$\gamma_2 < \gamma_1$}
	\begin{proof}
		\pf\ Since $\omega^{\gamma_2} \leq \rho_1 < \omega^{\gamma_1}$
	\end{proof}
\end{proof}
\step{2}{If
\[ \gamma_k < \gamma_{k-1} < \cdots < \gamma_1 \]
\[ \gamma'_k < \gamma'_{k-1} < \cdots < \gamma'_1 \]
and
\[ \omega^{\gamma_1} n_1 + \omega^{\gamma_2} n_2 + \cdots + \omega^{\gamma_k} n_k = \omega^{\gamma'_1} n'_1 + \omega^{\gamma'_2} n'_2 + \cdots + \omega^{\gamma'_k} n'_k \]
then $\gamma_i = \gamma'_i$ for all $i$ and $n_i = n'_i$ for all $i$}
\begin{proof}
	\pf\ Prove by induction on $i$ using the Logarithm Theorem.
\end{proof}
\qed
\end{proof}

\begin{df}[Cantor Normal Form]
For any ordinal $\alpha$, the \emph{Cantor normal form} of $\alpha$ is the expression $\alpha = \omega^{\gamma_1} n_1 + \cdots + \omega^{\gamma_k} n_k$ such that $n_1$, \ldots, $n_k$ are nonzero natural numbers and $\gamma_k < \gamma_{k-1} < \cdots < \gamma_1$.
\end{df}

\chapter{The Cumulative Hierarchy}

\begin{df}[Cumulative Hierarchy (ZF)]
Define the function $V : \mathbf{On} \rightarrow \mathbf{V}$ by transfinite recursion thus:
\[ V_\alpha = \bigcup_{\beta < \alpha} \mathcal{P} V_\beta \]
\end{df}

\begin{prop}[ZF]
For all $\alpha \in \mathbf{On}$, $V_\alpha$ is a transitive set.
\end{prop}

\begin{proof}
\pf
\step{1}{\pflet{$\alpha \in \mathbf{On}$}}
\step{2}{\assume{as transfinite induction hypothesis $\forall \beta < \alpha. V_\beta$ is a transitive set.}}
\step{3}{For all $\beta < \alpha$, $\mathcal{P} V_\beta$ is a transitive set.}
\begin{proof}
	\pf\ Proposition \ref{prop:powtransitive}.
\end{proof}
\step{4}{$V_\alpha$ is a transitive set.}
\begin{proof}
	\pf\ Proposition \ref{prop:uniontransitive}.
\end{proof}
\qed
\end{proof}

\begin{prop}[ZF]
\label{prop:Vmono}
For any ordinals $\alpha$ and $\beta$, if $\beta < \alpha$ then $V_\beta \subseteq V_\alpha$.
\end{prop}

\begin{proof}
\pf\ Since $V_\beta \in \mathcal{P} V_\beta \subseteq V_\alpha$ and $V_\alpha$ is a transitive set. \qed
\end{proof}

\begin{thm}[ZF]$ $
\begin{enumerate}
\item $V_0 = \emptyset$
\item $\forall \alpha \in \mathbf{On}. V_{\alpha^+} = \mathcal{P} V_\alpha$
\item For any limit ordinal $\lambda$, $V_\lambda = \bigcup_{\alpha < \lambda} V_\alpha$.
\end{enumerate}
\end{thm}

\begin{proof}
\pf
\step{1}{$V_0 = \emptyset$}
\begin{proof}
	\pf\ Immediate from definition.
\end{proof}
\step{2}{$\forall \alpha \in \mathbf{On}. V_{\alpha^+} = \mathcal{P} V_\alpha$}
\begin{proof}
	\pf
	\step{a}{\pflet{$\alpha \in \mathbf{On}$}}
	\step{b}{For all $\beta < \alpha$ we have $\mathcal{P} V_\beta \subseteq \mathcal{P} V_\alpha$}
	\begin{proof}
		\pf\ Propositions \ref{prop:powermono} and \ref{prop:Vmono}.
	\end{proof}
	\step{c}{$V_{\alpha^+} = \mathcal{P} V_\alpha$}
	\begin{proof}
	\begin{align*}
		V_{\alpha^+} & = \bigcup_{\beta < \alpha^+} \mathcal{P} V_\beta \\
		& = \bigcup_{\beta < \alpha} \mathcal{P} V_\beta \cup \mathcal{P} V_\alpha \\
		& \mathcal{P} V_\alpha & \qed
	\end{align*}
	\end{proof}
\end{proof}
\step{3}{For any limit ordinal $\lambda$, $V_\lambda = \bigcup_{\alpha < \lambda} V_\alpha$}
\begin{proof}
	\pf
	\step{a}{$V_\lambda \subseteq \bigcup_{\alpha < \lambda} V_\alpha$}
	\begin{proof}
	\pf
	\begin{align*}
		V_\lambda & = \bigcup_{\alpha < \lambda} \mathcal{P} V_\alpha \\
		& = \bigcup_{\alpha < \lambda} V_{\alpha^+} & (\text{\stepref{2}}) \\
		& \subseteq \bigcup_{\alpha < \lambda} V_\alpha
	\end{align*}
	\end{proof}
	\step{b}{$\bigcup_{\alpha < \lambda} V_\alpha \subseteq V_\lambda$}
	\begin{proof}
		\pf\ Proposition \ref{prop:Vmono}.
	\end{proof}
\end{proof}
\qed
\end{proof}

\begin{prop}[ZF]
For every set $A$, there exists an ordinal $\alpha$ such that $A \in V_\alpha$.
\end{prop}

\begin{proof}
\pf
\step{1}{Let us say a set $A$ is \emph{grounded} iff there exists an ordinal $\alpha$ such that $A \in V_\alpha$.}
\step{2}{For any set $A$, if every element of $A$ is grounded, then $A$ is grounded.}
\begin{proof}
	\step{a}{\pflet{$A$ be a set.}}
	\step{b}{$S = \{ \alpha \mid \exists a \in A. \alpha \text{ is the least ordinal such that } a \in V_\alpha \}$}
	\begin{proof}
		\pf\ $S$ is a set by an Axiom of Replacement.
	\end{proof}
	\step{c}{\pflet{$\beta = \sup S$}}
	\step{d}{$A \subseteq V_\beta$}
	\begin{proof}
		\step{i}{\pflet{$a \in A$}}
		\step{ii}{\pflet{$\alpha$ be the least ordinal such that $a \in V_\beta$}}
		\step{iii}{$\alpha \in S$}
		\step{iv}{$\alpha \leq \beta$}
		\step{v}{$a \in V_\beta$}
	\end{proof}
	\step{e}{$A \in V_{\beta^+}$}
\end{proof}
\step{1}{\assume{for a contradiction there exists an ungrounded set.}}
\step{3}{\pick\ a transitive set $B$ that has an ungrounded member.}
\begin{proof}
	\pf\ Pick a transitive set $c$, and take $B$ to be the transitive closure of $\{c\}$.
\end{proof}
\step{4}{\pflet{$A = \{ x \in B \mid x \text{ is ungrounded} \}$}}
\step{5}{\pick\ $m \in A$ such that $m \cap A = \emptyset$}
\begin{proof}
	\pf\ Axiom of Regularity.
\end{proof}
\step{6}{Every member of $m$ is grounded.}
\begin{proof}
	\step{a}{\assume{for a contradiction $x \in m$ is ungrounded.}}
	\step{b}{$x \in B$}
	\begin{proof}
		\pf\ Since $B$ is transitive (\stepref{3}).
	\end{proof}
	\step{c}{$x \in A$}
	\begin{proof}
		\pf\ \stepref{4}	
	\end{proof}
	\qedstep
	\begin{proof}
		\pf\ This contradicts \stepref{5}.
	\end{proof}
\end{proof}
\step{7}{$m$ is grounded.}
\begin{proof}
	\pf\ \stepref{2}
\end{proof}
\qedstep
\begin{proof}
	\pf\ This contradicts \stepref{5}.
\end{proof}
\qed
\end{proof}

\begin{df}[Rank (ZF)]
The \emph{rank} of a set $A$ is the least ordinal $\alpha$ such that $A \in V_{\alpha^+}$.
\end{df}

\begin{prop}[ZF]
\label{prop:allgrounded}
For any set $A$ we have
\[ \rank A = \bigcup_{a \in A} (\rank a)^+ \]
\end{prop}

\begin{proof}
\pf
\step{1}{\pflet{$\alpha = \bigcup_{a \in A} (\rank a)^+$}}
\step{2}{$A \subseteq V_\alpha$}
\begin{proof}
	\step{a}{\pflet{$a \in A$}}
	\step{b}{$a \in V_{(\rank a)^+}$}
	\step{c}{$a \in V_\alpha$}
\end{proof}
\step{3}{$A \in V_{\alpha^+}$}
\step{4}{If $A \subseteq V_\beta$ then $\alpha \leq \beta$}
\begin{proof}
	\step{a}{\assume{$A \subseteq V_\beta$}}
	\step{b}{For all $a \in A$ we have $(\rank a)^+ \leq \beta$}
	\begin{proof}
		\pf\ Since $a \in V_\beta$.
	\end{proof}
	\step{c}{$\alpha \leq \beta$}
\end{proof}
\qed
\end{proof}

\begin{cor}[ZF]
For any sets $a$ and $b$, if $a \in b$ then $\rank a < \rank b$.
\end{cor}

\begin{prop}[ZF]
For any ordinal number $\alpha$ we have $\rank \alpha = \alpha$.
\end{prop}

\begin{proof}
\pf
\step{1}{\pflet{$\alpha$ be an ordinal.}}
\step{2}{\assume{as transfinite induction hypothesis $\forall \beta < \alpha. \rank \beta = \beta$}}
\step{3}{$\rank \alpha = \bigcup_{\beta < \alpha} \beta^+$}
\begin{proof}
	\pf
	\begin{align*}
		\rank \alpha & = \bigcup_{\beta < \alpha} (\rank \beta)^+ \\
		& = \bigcup_{\beta < \alpha} \beta^+
	\end{align*}
\end{proof}
\step{4}{$\bigcup_{\beta < \alpha} \beta^+ \leq \alpha$}
\begin{proof}
	\pf\ Since for all $\beta < \alpha$ we have $\beta^+ \leq \alpha$.
\end{proof}
\step{5}{$\alpha \leq \bigcup_{\beta < \alpha} \beta^+$}
\begin{proof}
	\step{a}{\pflet{$\gamma = \bigcup_{\beta < \alpha} \beta^+$}}
	\step{b}{\assume{for a contradiction $\gamma < \alpha$}}
	\step{c}{$\gamma^+ \leq \bigcup_{\beta < \alpha} \beta^+ = \gamma$}
	\qedstep
	\begin{proof}
		\pf\ This is a contradiction.
	\end{proof}
\end{proof}
\qed
\end{proof}

\begin{df}[Hereditarily Finite (ZF)]
A set is \emph{hereditarily finite} iff it is in $V_\omega$.
\end{df}

\begin{prop}[ZF]
$V_\omega$ is the smallest set $X$ such that $\emptyset \in X$ and every finite subset of $X$ is an element of $X$.
\end{prop}

\section{Scott Cardinals}

\begin{df}[Scott Cardinal]
For any set $A$, the \emph{Scott cardinal} of $A$ is
\[ \card A := \{ y \mid y \approx x \wedge \forall z (z \approx x \Rightarrow \rank z \geq \rank y) \} \]
\end{df}

\begin{prop}[SF]
For any set $A$, we have $\card A$ is a set.
\end{prop}

\begin{prop}[SF]
For any sets $A$ and $B$, we have $\card A = \card B$ if and only if $A \approx B$.
\end{prop}

\chapter{Models of Set Theory}

\begin{df}[Relativization]
Let $\sigma$ be a sentence in the language of set theory and $\mathbf{M}$ a class. The \emph{relativization} of $\sigma$ to $\mathbf{M}$ is the sentence $\sigma^\mathbf{M}$ formed by replacing every quantifier $\forall x$ with $\forall x \in \mathbf{M}$, and $\exists x$ with $\exists x \in \mathbf{M}$.

We write '$\mathbf{M}$ is a model of $\sigma$' for the sentence $\sigma^\mathbf{M}$.
\end{df}

\begin{thms}
\label{thm:modelExtensionality}
For any class $\mathbf{M}$, the following is a theorem:

If $\mathbf{M}$ is a transitive class, then $\mathbf{M}$ is a model of the Axiom of Extensionality.
\end{thms}

\begin{proof}
\pf
\step{1}{\assume{$\mathbf{M}$ is a transitive class.} \prove{$\forall x,y \in \mathbf{M} (\forall z \in \mathbf{M} (z \in x \Leftrightarrow z \in y) \Rightarrow x = y)$}}
\step{2}{\pflet{$x,y \in \mathbf{M}$}}
\step{3}{\assume{$\forall z \in \mathbf{M} (z \in x \Leftrightarrow z \in y)$}}
\step{4}{$\forall z (z \in x \Leftrightarrow z \in y)$}
\begin{proof}
	\pf\ Since $z \in x \Rightarrow z \in \mathbf{M}$ and $z \in y \Rightarrow z \in \mathbf{M}$ by \stepref{1}.
\end{proof}
\step{5}{$x = y$}
\qed
\end{proof}

\begin{thm}
\label{thm:modelEmptySet}
If $\alpha$ is a non-zero ordinal then $V_\alpha$ is a model of the statement: The empty class is a set.
\end{thm}

\begin{proof}
\pf
\step{1}{\pflet{$\alpha \neq 0$} \prove{$\exists x \in V_\alpha. \forall y \in V_\alpha. y \notin x$}}
\step{2}{$\emptyset \in V_\alpha$}
\step{3}{$\forall y \in V_\alpha. y \notin \emptyset$}
\qed
\end{proof}

\begin{thm}
\label{thm:modelPairing}
For any limit ordinal $\lambda$, we have $V_\lambda$ is a model of the statement: for any sets $a$ and $b$, the class $\{ a,b \}$ is a set.
\end{thm}

\begin{proof}
\pf
\step{1}{\pflet{$\lambda$ be a limit ordinal.} \prove{$\forall a,b \in V_\lambda. \exists c \in V_\lambda. \forall x \in V_\lambda (x \in c \Leftrightarrow x = a \vee x = b)$}}
\step{2}{\pflet{$a,b \in V_\lambda$}}
\step{3}{\pick\ $\alpha, \beta < \lambda$ such that $a \in V_\alpha$ and $b \in V_\beta$}
\step{4}{\assume{w.l.o.g. $\alpha \leq \beta$}}
\step{5}{$a,b \in V_\beta$}
\step{6}{$\{a,b\} \in V_{\beta + 1}$}
\step{7}{$\{a,b\} \in V_\lambda$}
\step{8}{$\forall x \in V_\lambda (x \in \{a,b\} \Leftrightarrow x = a \vee x = b)$}
\qed
\end{proof}

\begin{thm}
\label{thm:modelUnion}
For any ordinal $\alpha$, we have $V_\alpha$ is a model of the Union Axiom.
\end{thm}

\begin{proof}
\pf
\step{1}{\pflet{$\alpha$ be an ordinal.} \prove{$\forall a \in V_\alpha. \exists b \in V_\alpha. \forall x \in V_\alpha (x \in b \Leftrightarrow \exists y \in V_\alpha (x \in y \wedge y \in a))$}}
\step{2}{\pflet{$a \in V_\alpha$}}
\step{3}{\pick\ $\beta < \alpha$ such that $a \subseteq V_\beta$}
\step{4}{$\bigcup a \subseteq V_\beta$}
\begin{proof}
	\pf\ $V_\beta$ is a transitive set.
\end{proof}
\step{5}{$\bigcup a \in V_\alpha$}
\step{6}{$\forall x \in V_\alpha (x \in \bigcup a \Leftrightarrow \exists y \in V_\alpha ( x \in y \wedge y \in a))$}
\begin{proof}
	\pf\ $V_\alpha$ is a transitive set.
\end{proof}
\qed 
\end{proof}

\begin{thm}
\label{thm:modelPowerSet}
For any limit ordinal $\lambda$, we have $V_\lambda$ is a model of the Power Set Axiom.
\end{thm}

\begin{proof}
\pf
\step{1}{\pflet{$\lambda$ be a limit ordinal.} \prove{$\forall a \in V_\lambda. \exists b \in V_\lambda. \forall x \in V_\lambda (x \in b \Leftrightarrow \forall y \in V_\lambda (y \in x \Rightarrow y \in a))$}}
\step{2}{\pflet{$a \in V_\lambda$}}
\step{3}{\pick\ $\alpha < \lambda$ such that $a \in V_\alpha$}
\step{4}{$\mathcal{P} a \in V_{\alpha + 1}$}
\step{5}{$\mathcal{P} a \in V_\lambda$}
\step{6}{$\forall x \in V_\lambda (x \in \mathcal{P} a \Leftrightarrow \forall y \in V_\lambda (y \in x \Rightarrow y \in a))$}
\qed
\end{proof}

\begin{thms}
\label{thm:modelComprehension}
For any property $P[x,y_1, \ldots, y_n]$, the following is a theorem:

For any ordinal $\alpha$, the set $V_\alpha$ is a model of the statement: for any sets $a_1$, \ldots, $a_n$, $B$, the class $\{ x \in B \mid P[x,a_1, \ldots, a_n] \}$ is a set.
\end{thms}

\begin{proof}
\pf
\step{1}{\pflet{$\alpha$ be an ordinal.}}
\step{2}{\pflet{$a_1, \ldots, a_n, B \in V_\alpha$}}
\step{3}{\pflet{$C = \{ x \in B \mid P[x,a_1, \ldots, a_n]^{V_\alpha} \}$}}
\step{4}{$C \in V_\alpha$}
\step{5}{$\forall x \in V_\alpha (x \in C \Leftrightarrow x \in B \wedge P[x, a_1, \ldots, a_n]^{V_\alpha})$}
\qed
\end{proof}

\begin{thm}
\label{thm:modelInfinity}
For any ordinal $\alpha > \omega$, we have: $V_\alpha$ is a model of the Axiom of Infinity.
\end{thm}

\begin{proof}
\pf
\step{1}{\pflet{$\alpha > \omega$}}
\step{2}{$\mathbb{N} \in V_\alpha$}
\step{3}{$\exists e \in V_\alpha (e \in \mathbb{N} \wedge \forall x \in V_\alpha. x \notin e)$}
\step{4}{$\forall x \in V_\alpha (x \in \mathbb{N} \Rightarrow \exists y \in V_\alpha \forall z \in V_\alpha (z \in y \Leftrightarrow z \in x \vee z = x))$}
\qed
\end{proof}

\begin{thm}
\label{thm:modelChoice}
For any ordinal $\alpha$, we have $V_\alpha$ is a model of the Axiom of Choice.
\end{thm}

\begin{proof}
\pf
\step{1}{\pflet{$\alpha$ be an ordinal.}}
\step{2}{\pflet{$A \in V_\alpha$}}
\step{3}{\assume{$\forall x \in V_\alpha (x \in A \Rightarrow \exists y \in V_\alpha. y \in A)$}}
\step{4}{\assume{$\forall x,y,z \in V_\alpha (x \in A \wedge y \in A \wedge z \in x \wedge z \in y \Rightarrow x = y)$}}
\step{5}{$A$ is a set of pairwise disjoint nonempty sets.}
\step{6}{\pick\ $c$ such that, for all $x \in A$, $x \cap c = \emptyset$}
\step{7}{$c \cap \bigcup A \in V_\alpha$}
\step{8}{$\forall x \in V_\alpha (x \in A \Rightarrow \exists y \in V_\alpha \forall z \in V_\alpha (z \in x \wedge z \in c \cap \bigcup A \Leftrightarrow z = y))$}
\qed
\end{proof}

\begin{thm}
\label{thm:modelRegularity}
For any ordinal $\alpha$, we have $V_\alpha$ is a model of the Axiom of Regularity.
\end{thm}

\begin{proof}
\pf
\step{1}{\pflet{$\alpha$ be an ordinal.}}
\step{2}{\pflet{$A \in V_\alpha$}}
\step{3}{\assume{$\exists x \in V_\alpha. x \in A$}}
\step{4}{\pick\ $m \in A$ of least rank.}
\step{5}{$m \in V_\alpha$}
\step{6}{$\neg \exists x \in V_\alpha (x \in m \wedge x \in A)$}
\qed
\end{proof}

\begin{thms}
For any axiom $\alpha$ of Zermelo set theory, the following is a theorem:

For any limit ordinal $\lambda > \omega$, we have $V_\lambda$ is a model of $\alpha$.
\end{thms}

\begin{proof}
\pf\ Theorems \ref{thm:modelExtensionality}, \ref{thm:modelEmptySet}, \ref{thm:modelPairing}, \ref{thm:modelUnion}, \ref{thm:modelPowerSet}, \ref{thm:modelComprehension}, \ref{thm:modelInfinity}, \ref{thm:modelChoice}, \ref{thm:modelRegularity}. \qed
\end{proof}

\begin{cors}
\label{cor:modelZermelo}
for any axiom $\alpha$ of Zermelo set theory, the following is a theorem:

$V_{\omega 2}$ is a model of $\alpha$.
\end{cors}

\begin{lm}
There exists a well-ordered structure in $V_{\omega 2}$ whose ordinal is not in $V_{\omega 2}$.
\end{lm}

\begin{proof}
\pf\ Take the well-ordered set $\mathbb{N} \times \{0,1\}$ whose ordinal is $\omega 2$. \qed
\end{proof}

\begin{cors}
There exists an instance $\alpha$ of the Axiom Schema of Replacement such that the following is a theorem:

$V_{\omega 2}$ is not a model of $\alpha$.
\end{cors}

\chapter{Infinite Cardinals}

\section{Arithmetic of Infinite Cardinals}

\begin{prop}
\label{prop:kappasquared}
For any infinite cardinal $\kappa$ we have $\kappa \kappa = \kappa$.
\end{prop}

\begin{proof}
\pf
\step{1}{\pick\ a set $B$ with $|B| = \kappa$}
\step{2}{\pflet{$\mathcal{H} = \{ f \mid f = \emptyset \vee \exists A \subseteq B. (A \text{ is infinite} \wedge f : A \times A \approx A \}$}}
\step{3}{For any chain $\mathcal{C} \subseteq \mathcal{H}$ we have $\bigcup \mathcal{C} \in \mathcal{H}$}
\begin{proof}
	\step{a}{\pflet{$\mathcal{C} \subseteq \mathcal{H}$ be a chain.}}
	\step{b}{\assume{w.l.o.g. $\mathcal{C}$ has a nonempty element.}}
	\step{c}{$\bigcup \mathcal{C}$ is a function.}
	\begin{proof}
		\step{i}{\assume{$(x,y),(x,z) \in \bigcup \mathcal{C}$}}
		\step{ii}{\pick\ $f,g \in \mathcal{C}$ such that $f(x) = y$ and $g(x) = z$}
		\step{iii}{\assume{w.l.o.g. $f \subseteq g$}}
		\step{iv}{$y = z$}
	\end{proof}
	\step{d}{$\bigcup \mathcal{C}$ is injective.}
	\begin{proof}
		\pf\ Similar.
	\end{proof}
	\step{e}{\pflet{$A = \ran \bigcup \mathcal{C}$}}
	\step{f}{$A$ is infinite.}
	\begin{proof}
		\step{i}{\pick\ a nonzero $f \in \mathcal{C}$}
		\step{ii}{\pflet{$A'$ be the infinite subset of $B$ such that $f : A'^2 \approx A'$}}
		\step{iii}{$A' \subseteq A$}
	\end{proof}
	\step{g}{$\dom \bigcup \mathcal{C} = A^2$}
	\begin{proof}
		\step{i}{\pflet{$x,y \in A$}}
		\step{ii}{\pick\ $f,g \in \mathcal{C}$ such that $x \in \ran f$ and $y \in \ran g$}
		\step{iii}{\assume{w.l.o.g. $f \subseteq g$}}
		\step{iv}{\pflet{$A'$ be the infinite subset of $B$ such that $g : A'^2 \approx A'$}}
		\step{v}{$x,y \in A'$}
		\step{vi}{$(x,y) \in \dom g$}
		\step{vii}{$(x,y) \in \dom \bigcup \mathcal{C}$}
	\end{proof}
	\step{h}{$\bigcup \mathcal{C} \in \mathcal{H}$}
\end{proof}
\step{4}{\pick\ a maximal $f_0 \in \mathcal{H}$}
\step{5}{$f_0 \neq \emptyset$}
\begin{proof}
	\step{a}{\pick\ a countably infinite subset $A$ of $B$.}
	\begin{proof}
		\pf\ Proposition \ref{prop:aleph0least}.
	\end{proof}
	\step{b}{\pick\ a bijection $f : A^2 \approx A$}
	\begin{proof}
		\pf\ Proposition \ref{prop:N^2approxN}.
	\end{proof}
	\step{c}{$\emptyset \subseteq f \in \mathcal{H}$}
	\step{d}{$\emptyset$ is not maximal in $\mathcal{H}$}
\end{proof}
\step{6}{\pflet{$A_0$ be the infinite subset of $B$ such that $f_0 : A_0^2 \approx A_0$}}
\step{7}{\pflet{$\lambda = |A_0|$}}
\step{8}{$\lambda$ is infinite.}
\step{9}{$\lambda^2 = \lambda$}
\step{10}{$\lambda = \kappa$}
\begin{proof}
	\step{a}{\assume{for a contradiction $\lambda < \kappa$}}
	\step{b}{$\lambda \leq |B - A_0|$}
	\step{c}{\pick\ a subset $D \subseteq B - A_0$ with $|D| = \lambda$}
	\step{d}{$(A_0 \cup D)^2 = A_0^2 \cup (A_0 \times D) \cup (D \times A_0) \cup D^2$}
	\step{e}{\pflet{$C = (A_0 \times D) \cup (D \times A_0) \cup D^2$}}
	\step{f}{$|C| = \lambda$}
	\begin{proof}
		\pf
		\begin{align*}
			|(A_0 \times D) \cup (D \times A_0) \cup D^2| & = \lambda^2 + \lambda^2 + \lambda^2 \\
			& = \lambda + \lambda + \lambda & (\text{\stepref{9}}) \\
			& = 3 \lambda \\
			& \leq \lambda \cdot \lambda \\
			& = \lambda & (\text{\stepref{9}})
		\end{align*}
	\end{proof}
	\step{g}{\pick\ a bijection $g : C \approx D$}
	\step{h}{$f_0 \cup g : (A_0 \cup D)^2 \approx A_0 \cup D$}
	\qedstep
	\begin{proof}
		\pf\ This contradicts the maximality of $f_0$.
	\end{proof}
\end{proof}
\qed
\end{proof}

\begin{thm}[Absorpution Law of Cardinal Arithmetic]
Let $\kappa$ and $\lambda$ be nonzero cardinal numbers such that at least one is infinite. Then
\[ \kappa + \lambda = \kappa \lambda = \max(\kappa, \lambda) \]
\end{thm}

\begin{proof}
\pf
\step{1}{\assume{w.l.o.g. $\lambda \leq \kappa$}}
\step{2}{$\kappa + \lambda = \kappa \lambda = \kappa$}
\begin{proof}
	\pf
	\begin{align*}
		\kappa & \leq \kappa + \lambda \\
		& \leq \kappa + \kappa \\
		& = 2 \kappa \\
		& \leq \kappa \lambda \\
		& \leq \kappa \kappa \\
		& = \kappa & (\text{Proposition \ref{prop:kappasquared}})
	\end{align*}
\end{proof}
\qed
\end{proof}

\section{Alephs}

\begin{prop}
The operation $\aleph$ is normal.
\end{prop}

\begin{proof}
\pf\ Proposition \ref{prop:enumerationnormal} and Lemma \ref{lm:supcard}. \qed
\end{proof}

\begin{df}[Continuum Hypothesis]
The \emph{continuum hypothesis} is the statement that $\aleph_1 = 2^{\aleph_0}$.
\end{df}

\begin{df}[Generalised Continuum Hypothesis]
The \emph{generalised continuum hypothesis} is the statement that, for all $\alpha$, $\aleph_{\alpha^+} = 2^{\aleph_\alpha}$.
\end{df}

\section{Beths}

\begin{df}[Beth]
Define the operation $\beth : \mathbf{On} \rightarrow \mathbf{Card}$ by transfinite recursion as follows:
\begin{align*}
\beth_0 & := \aleph_0 \\
\beth_{\alpha^+} & := 2^{\beth_\alpha} \\
\beth_\lambda & := \bigcup_{\alpha < \lambda} \beth_\alpha & (\lambda \text{ a limit ordinal})
\end{align*}
\end{df}

\begin{prop}
$\beth$ is a normal operation.
\end{prop}

\begin{proof}
\pf\ It is continuous by definition, and $\beth_\alpha < \beth_{\alpha^+}$ by Cantor's Theorem. \qed
\end{proof}

\begin{prop}
The continuum hypothesis is equivalent to the statement $\beth_1 = \aleph_1$.

The generalised continuum hypothesis is equivalent to the statement $\beth = \alpha$.
\end{prop}

\begin{proof}
\pf\ Immediate from definitions. \qed
\end{proof}

\begin{lm}
For any ordinal number $\alpha$, we have $|V_{\omega + \alpha}| = \beth_\alpha$.
\end{lm}

\begin{proof}
\pf
\step{1}{$|V_\omega| = \beth_0$}
\begin{proof}
	\pf\ Since $V_\omega$ is the union of $\aleph_0$ finite sets of increasing size.
\end{proof}
\step{2}{For any ordinal $\alpha$, if $|V_{\omega + \alpha}| = \beth_\alpha$ then $|V_{\omega + \alpha + 1}| = \beth_{\alpha + 1}$}
\begin{proof}
	\pf\ Since $V_{\omega + \alpha + 1} = \mathcal{P} V_{\omega + \alpha}$.
\end{proof}
\step{3}{For any limit ordinal $\lambda$, if $\forall \alpha < \lambda. |V_{\omega + \alpha}| = \beth_\alpha$ then $|V_{\omega + \lambda}| = \beth_\lambda$.}
\begin{proof}
	\pf
	\begin{align*}
		|V_{\omega + \lambda}| & = \left| \bigcup_{\alpha < \lambda} V_{\omega + \alpha} \right| \\
		& = \sup_{\alpha < \lambda} |V_{\omega + \alpha}| \\
		& = \sup_{\alpha < \lambda} \beth_\alpha \\
		& = \beth_\lambda
	\end{align*}
\end{proof}
\qed
\end{proof}

\section{Cofinality}

\begin{df}[Cofinal]
Let $\lambda$ be a limit ordinal and $S$ a set of ordinals smaller than $\lambda$. Then $S$ is \emph{cofinal} in $\lambda$ if and only if $\lambda = \sup S$.
\end{df}

\begin{df}[Cofinality]
For any ordinal $\alpha$, define the \emph{cofinality} of $\alpha$, $\cf \alpha$, as follows:
\begin{itemize}
\item $\cf 0 = 0$
\item For any ordinal $\alpha$, $\cf \alpha^+ = 1$
\item For any limit ordinal $\lambda$, $\cf \lambda$ is the smallest cardinal such that there exists a set $S$ of ordinals cofinal in $\lambda$ with $|S| = \cf \lambda$.
\end{itemize}
\end{df}

\begin{df}[Regular]
A cardinal $\kappa$ is \emph{regular} iff $\cf \kappa = \kappa$; otherwise it is \emph{singular}.
\end{df}

\begin{prop}
$\aleph_0$ is regular.
\end{prop}

\begin{proof}
\pf\ $\aleph_0$ is not the supremum of $< \aleph_0$ smaller ordinals, because a finite union of finite ordinals is finite. \qed
\end{proof}

\begin{prop}
For every ordinal $\alpha$, $\aleph_{\alpha + 1}$ is regular.
\end{prop}

\begin{proof}
\pf\ If $S$ is a set of ordinals with $|S| < \aleph_{\alpha + 1}$ and $\forall \beta \in S. \beta < \aleph_{\alpha + 1}$, then we have $|S| \leq \aleph_\alpha$ and $\forall \beta \in S. \beta \leq \aleph_\alpha$, hence
\begin{align*}
\left|\bigcup S\right| & \leq \aleph_{\alpha}^2 & (\text{Proposition \ref{prop:unioncard}}) \\
& = \aleph_{\alpha} & (\text{Proposition \ref{prop:kappasquared}}) \qed
\end{align*}
\end{proof}

\begin{props}
For any class $\mathbf{T}$, the following is a theorem.

Assume $\mathbf{T} : \mathbf{On} \rightarrow \mathbf{On}$ is a normal operation.
For any limit ordinal $\lambda$ we have $\cf \mathbf{T}(\lambda) = \cf \lambda$.
\end{props}

\begin{proof}
\pf
\step{1}{$\cf \mathbf{T}(\lambda) \leq \cf \lambda$}
\begin{proof}
	\step{a}{\pick\ a set $S$ of ordinals $< \lambda$ with $|S| = \cf \lambda$ and $\sup S = \lambda$}
	\step{b}{$\mathbf{T}(\lambda) = \sup_{\alpha \in S} \mathbf{T}(\alpha)$}
	\begin{proof}
		\pf\ Theorem \ref{thm:normalsup}.
	\end{proof}
\end{proof}
\step{2}{$\cf \lambda \leq \cf \mathbf{T}(\lambda)$}
\begin{proof}
	\step{a}{\pick\ a set $A$ of ordinals $< \mathbf{T}(\lambda)$ such that $|A| = \cf \mathbf{T}(\lambda)$ and $\sup A = \mathbf{T}(\lambda)$}
	\step{b}{\pflet{$B = \{ \gamma < \lambda \mid \exists \alpha \in A. |\alpha| = \mathbf{T}(\gamma) \}$}}
	\step{c}{$|B| \leq |A| = \cf \mathbf{T}(\lambda)$ \prove{$\sup B = \lambda$}}
	\step{d}{$\forall \alpha \in A. |\alpha| \leq \mathbf{T}(\sup B)$}
	\step{e}{$\forall \alpha \in A. \alpha < \mathbf{T}(\sup B + 1)$}
	\step{f}{$\aleph_\lambda = \sup A \leq \mathbf{T}(\sup B + 1)$}
	\step{g}{$\lambda \leq \sup B + 1$}
	\step{h}{$\lambda \leq \sup B$}
	\begin{proof}
		\pf\ $\lambda$ is a limit ordinal.
	\end{proof}
	\step{i}{$\sup B = \lambda$}
\end{proof}
\qed
\end{proof}

\begin{cor}
$\aleph_\omega$ is singular.
\end{cor}

\begin{proof}
\pf\ $\cf \aleph_\omega = \cf \aleph_0 = \aleph_0$. \qed
\end{proof}

\begin{cor}
The operation $\cf$ is not strictly monotone or continuous.
\end{cor}

\begin{proof}
\pf\ $\cf \aleph_\omega < \cf \aleph_1$ \qed
\end{proof}

\begin{df}[Weakly Inaccessible]
A cardinal is \emph{weakly inaccessible} iff it is $\aleph_\lambda$ for some limit ordinal $\lambda$ and regular.
\end{df}

\begin{lm}
Let $\lambda$ be a limit ordinal. Then there exists a strictly increasing $\cf \lambda$-sequence that converges to $\lambda$.
\end{lm}

\begin{proof}
\pf
\step{1}{\pick\ a set $S$ of ordinals $< \lambda$ with $|S| = \cf \lambda$ and $\sup S = \lambda$}
\step{2}{\pick\ a bijection $a : \cf \lambda \approx S$}
\step{3}{\pick\ a strictly increasing subsequence $(b_\delta)_{\delta < \beta}$ of $a$ that converges to $\lambda$.}
\begin{proof}
	\pf\ Lemma \ref{lm:convergentsubsequence}.
\end{proof}
\step{4}{$\beta = \cf \lambda$}
\begin{proof}
	\pf\ By minimiality of $\cf \lambda$.
\end{proof}
\qed
\end{proof}

\begin{cor}
Let $\lambda$ be a limit ordinal. Then $\cf \lambda$ is the least ordinal such that there exists a strictly increasing $\cf \lambda$-sequence that converges to $\lambda$.
\end{cor}

\begin{prop}
For any ordinal $\lambda$, $\cf \lambda$ is a regular cardinal.
\end{prop}

\begin{proof}
\pf
\step{1}{\pflet{$\lambda$ be an ordinal.}}
\step{2}{\assume{w.l.o.g. $\lambda$ is a limit ordinal.}}
\step{3}{\pick\ a strictly increasing sequence $(a_\alpha)_{\alpha < \cf \lambda}$ that converges to $\lambda$.}
\step{4}{\pflet{$S$ be a set of ordinals $< \cf \lambda$ such that $|S| = \cf \cf \lambda$ and $\sup S = \cf \lambda$.}}
\step{5}{\pflet{$a(S) = \{ a_\alpha \mid \alpha \in S \}$}}
\step{6}{$a(S)$ is cofinal in $\lambda$.}
\begin{proof}
	\step{a}{\pflet{$\beta < \lambda$}}
	\step{b}{\pick\ $\gamma < \cf \lambda$ such that $\beta < a_\gamma$}
	\step{c}{\pick\ $\delta \in S$ such that $\gamma < \delta$}
	\step{d}{$a_\delta \in a(S)$ and $\beta < a_\gamma < a_\delta$}
\end{proof}
\step{7}{$\cf \lambda \leq \cf \cf \lambda$}
\begin{proof}
	\pf\ Since $a(S)$ is a set of ordinals $< \lambda$ with $|a(S)| = \cf \cf \lambda$ and $\sup a(S) = \lambda$.
\end{proof}
\step{8}{$\cf \cf \lambda = \cf \lambda$}
\qed
\end{proof}

\begin{thm}
\label{thm:cfpartition}
Let $\lambda$ be an infinite cardinal. Then $\cf \lambda$ is the least cardinal such that $\lambda$ can be partitioned into $\cf \lambda$ sets, each of cardinality $< \lambda$.
\end{thm}

\begin{proof}
\pf
\step{1}{$\lambda$ can be partitioned into $\cf \lambda$ sets, each of cardinality $< \lambda$}
\begin{proof}
	\step{a}{\pick\ a strictly increasing sequence of ordinlas $(a_\alpha)_{\alpha < \cf \lambda}$ that converges to $\lambda$}
	\step{b}{$\{ \{ \beta \mid a_\alpha \leq \beta < a_{\alpha + 1} \} \mid \alpha < \cf \lambda \}$ is a partition of $\lambda$ into $\cf \lambda$ sets, each of cardinality $< \lambda$}
\end{proof}
\step{2}{If $\lambda$ can be partitioned into $\kappa$ sets, each of cardinality $< \lambda$, then $\cf \lambda \leq \kappa$.}
\begin{proof}
	\step{a}{\pflet{$\mathcal{A}$ be a partition of $\lambda$ into sets of cardinality $< \lambda$}}
	\step{b}{\pflet{$\kappa = |P|$}}
	\step{c}{\pick\ a bijection $A : \kappa \approx P$}
	\step{d}{$\lambda = \bigcup_{\xi < \kappa} A(\xi)$}
	\step{e}{For all $\xi < \kappa$ we have $|A(\xi)| < \lambda$}
	\step{f}{\pflet{$\mu = \sup_{\xi < \kappa} |A(\xi)|$}}
	\step{ff}{$\mu \leq \lambda$}
	\step{g}{For all $\xi < \kappa$ we have $|A(\xi)| \leq \mu$}
	\step{h}{$\lambda \leq \mu \kappa$}
	\begin{proof}
		\pf\ Proposition \ref{prop:unioncard}.
	\end{proof}
	\step{i}{\assume{w.l.o.g. $\kappa < \lambda$}}
	\begin{proof}
		\pf\ If $\lambda \leq \kappa$ then $\cf \lambda \leq \kappa$ since $\cf \lambda \leq \lambda$.
	\end{proof}
	\step{j}{$\lambda = \mu$}
	\begin{proof}
		\pf
		\begin{align*}
			\lambda & \leq \mu \kappa & (\text{\stepref{h}}) \\
			& \leq \lambda \lambda & (\text{\stepref{ff}, \stepref{i}}) \\
			& = \lambda & (\text{Proposition \ref{prop:kappasquared}})
		\end{align*}
	\end{proof}
	\step{k}{$\{ |A(\xi)| \mid \xi < \kappa \}$ is a set of $\leq \kappa$ ordinals all $< \lambda$ whose supremum is $\lambda$}
	\step{l}{$\cf \lambda \leq \kappa$}
\end{proof}
\qed
\end{proof}

\begin{thm}[K\"{o}nig]
For any infinite cardinal $\kappa$ we have $\kappa < \cf 2^\kappa$.
\end{thm}

\begin{proof}
\pf
\step{1}{\assume{for a contradiction $\cf 2^\kappa \leq \kappa$}}
\step{2}{\pflet{$S = 2^\kappa$}}
\step{3}{\pick\ a partition $\{A_\xi \mid \xi < \kappa \}$ of $S^\kappa$ with $\forall \xi < \kappa. |A_\xi| < 2^\kappa$.}
\begin{proof}
	\pf\ Theorem \ref{thm:cfpartition}.
\end{proof}
\step{4}{$\forall \xi < \kappa. \{ g(\xi) \mid g \in A_\xi \} \subsetneq S$}
\begin{proof}
	\pf\ We do not have equality because $|\{ g(\xi) \mid g \in A_\xi \}| \leq |A_\xi| < 2^\kappa$.
\end{proof}
\step{5}{For all $\xi < \kappa$, choose $s_\xi \in S - \{ g(\xi) \mid g \in A_\xi \}$}
\step{6}{$s \in S^\kappa$}
\step{7}{For all $\xi < \kappa$ we have $s \notin A_\xi$}
\begin{proof}
	\pf\ Since for all $\xi < \kappa$ and $g \in A_\xi$ we have $s_\xi(\xi) \neq g(\xi)$.
\end{proof}
\qedstep
\begin{proof}
	\pf\ This contradicts \stepref{3}.
\end{proof}
\qed
\end{proof}

\begin{cor}
\[ 2^{\aleph_0} \neq \aleph_\omega \]
\end{cor}

\begin{prop}
For any ordinal $\alpha$, we have $\cf \alpha$ is the least cardinal such that $\alpha$ is the strict supremum of $\cf \alpha$ smaller ordinals.
\end{prop}

\begin{proof}
\pf
\step{1}{\case{$\alpha = 0$}}
\begin{proof}
	\pf\ Since $0 = \ssup \emptyset$.
\end{proof}
\step{2}{\case{$\alpha = \beta^+$}}
\begin{proof}
	\pf\ Since $\beta^+ = \ssup \{ \beta \}$.
\end{proof}
\step{3}{\case{$\alpha$ is a limit ordinal.}}
\begin{proof}
	\step{a}{There exists a set $S$ of ordinals $< \alpha$ such that $|S| = \cf \alpha$ and $\alpha = \ssup S$.}
	\begin{proof}
		\step{a}{\pick\ a set $S$ of ordinals $< \alpha$ such that $|S| = \cf \alpha$ and $\sup S = \alpha$ \prove{$\alpha = \ssup S$}}
		\step{b}{$\forall \beta \in S. \beta < \alpha$}
		\step{c}{For any ordinal $\gamma$, if $\forall \beta \in S. \beta < \gamma$ then $\alpha \leq \gamma$}
	\end{proof}
	\step{c}{If $T$ is a set of ordinals $< \alpha$ such that $\alpha = \ssup T$, then $\cf \alpha \leq |T|$.}
	\begin{proof}
		\step{i}{\pflet{$T$ be a set of ordinals $< \alpha$ such that $\alpha = \ssup T$}}
		\step{ii}{$\alpha = \sup T$}
		\begin{proof}
			\step{one}{For all $\beta \in T$ we have $\beta \leq \alpha$}
			\step{two}{\pflet{$\mu$ be any upper bound for $T$} \prove{$\alpha \leq \mu$}}
			\step{three}{$\alpha \leq \mu + 1$}
			\begin{proof}
				\pf\ Since $\forall \beta \in T. \beta < \mu + 1$.
			\end{proof}
			\step{four}{$\alpha \neq \mu + 1$}
			\begin{proof}
				\pf\ Since $\alpha$ is a limit ordinal.
			\end{proof}
			\step{five}{$\alpha < \mu + 1$}
			\step{six}{$\alpha \leq \mu$}
		\end{proof}
		\step{iii}{$\cf \alpha \leq |T|$}
	\end{proof}
\end{proof}
\qed
\end{proof}

\section{Inaccessible Cardinals}

\begin{df}[Inaccessible Cardinal]
A cardinal number $\kappa$ is \emph{inaccessible} iff:
\begin{itemize}
\item $\kappa > \aleph_0$
\item $\forall \lambda < \kappa. 2^\lambda < \kappa$ (cardinal exponentiation)
\item $\kappa$ is regular.
\end{itemize}
\end{df}

\begin{proof}
Any inaccessible cardinal is weakly inaccessible.
\end{proof}

\begin{proof}
\pf
\step{1}{\pflet{$\kappa = \aleph_\lambda$ be weakly inaccessible.} \prove{$\lambda$ is a limit ordinal.}}
\step{2}{$\lambda \neq 0$}
\step{3}{\assume{for a contradiction $\lambda = \beta + 1$}}
\step{4}{$\aleph_\beta < \kappa$}
\step{5}{$2^{\aleph_\beta} < \kappa$}
\step{6}{$\aleph_{\beta + 1} < \kappa$}
\begin{proof}
	\pf\ Since $\aleph_{\beta + 1} \leq 2^{\aleph_\beta}$.
\end{proof}
\qedstep
\begin{proof}
\pf\ This contradicts \stepref{3}.
\end{proof}
\qed
\end{proof}

\begin{prop}
If the Generalized Continuum Hypothesis is true, then every weakly inaccessible cardinal is inaccessible.
\end{prop}

\begin{proof}
\pf
\step{1}{\assume{The Generalized Continuum Hypothesis.}}
\step{2}{\pflet{$\kappa = \aleph_\lambda$ be weakly inaccessible.}}
\step{3}{$\kappa > \aleph_0$}
\begin{proof}
	\pf\ $\lambda > 0$ because $\lambda$ is a limit ordinal.
\end{proof}
\step{4}{For all $\mu < \kappa$ we have $2^\mu < \kappa$}
\begin{proof}
	\step{a}{\pflet{$\mu < \kappa$}}
	\step{b}{\pflet{$\mu = \aleph_\alpha$}}
	\step{c}{$\alpha < \lambda$}
	\step{d}{$\alpha + 1 < \lambda$}
	\begin{proof}
		\pf\ $\lambda$ is a limit ordinal.
	\end{proof}
	\step{e}{$2^\mu < \kappa$}
	\begin{proof}
		\pf
		\begin{align*}
			2^\mu & = 2^{\aleph_\alpha} & (\text{\stepref{b}})  \\
			& = 2^{\beth_\alpha} & (\text{\stepref{1}}) \\
			& = \beth_{\alpha + 1} \\
			& = \aleph_{\alpha + 1} & (\text{\stepref{1}}) \\
			& < \aleph_\lambda & (\text{\stepref{d}}) \\
			& = \kappa & (\text{\stepref{2}})
		\end{align*}
	\end{proof}
\end{proof}
\step{5}{$\kappa$ is regular.}
\begin{proof}
	\pf\ \stepref{2}
\end{proof}
\qed
\end{proof}

%TODO: If ZFC is consistent then it cannot prove that weakly inaccessible cardinals exist

\begin{lm}
Let $\kappa$ be an inaccessible cardinal. For every ordinal $\alpha < \kappa$ we have $\beth_\alpha < \kappa$.
\end{lm}

\begin{proof}
\pf
\step{1}{$\beth_0 < \kappa$}
\begin{proof}
	\pf\ Since $\kappa > \aleph_0$.
\end{proof}
\step{2}{For any ordinal $\alpha$, if $\beth_\alpha < \kappa$ then $\beth_{\alpha + 1} < \kappa$.}
\begin{proof}
	\pf\ Since $\beth_{\alpha + 1} = 2^{\beth_\alpha} < \kappa$.
\end{proof}
\step{3}{For any limit ordinal $\lambda$, if $\forall \alpha < \lambda. \beth_\alpha < \kappa$ and $\lambda < \kappa$ then $\beth_\lambda < \kappa$.}
\begin{proof}
	\pf\ By regularity of $\kappa$, since $\beth_\lambda$ is the union of $|\lambda|$ cardinals all $< \kappa$.
\end{proof}
\qed
\end{proof}

\begin{lm}
Let $\kappa$ be an inaccessible cardinal. For all $A \in V_\kappa$ we have $|A| < \kappa$.
\end{lm}

\begin{proof}
\pf
\step{1}{\pflet{$A \in V_\kappa$}}
\step{2}{\pick\ $\alpha < \kappa$ such that $A \in V_\alpha$}
\step{3}{$A \subseteq V_\alpha$}
\step{4}{$|A| \leq |V_\alpha| \leq \beth_\alpha < \kappa$}
\qed
\end{proof}

\begin{thms}
For every axiom $\alpha$ of ZFC, the following is a theorem:

For any inaccessible cardinal $\kappa$, we have $V_\kappa$ is a model of $\alpha$.
\end{thms}

\begin{proof}
\pf\ For every axiom except the Replacement Axioms, we have Corollary \ref{cor:modelZermelo}.

For an Axiom of Replacement using the property $P[x,y,z_1, \ldots, z_n]$, we reason as follows:

\step{1}{\pflet{$\kappa$ be an inaccessible cardinal}
\prove{\[ \forall a_1, \ldots, a_n, B \in V_\kappa (\forall x \in B. \forall y,y' \in V_\kappa \]
\[ (P[x,y,a_1, \ldots, a_n]^{V_\kappa} \wedge P[x,y',a_1, \ldots, a_n]^{V_\kappa} \Rightarrow y = y') \Rightarrow \]
\[ \exists C \in V_\kappa \forall y \in V_\kappa (y \in C \Leftrightarrow \exists x \in B. P[x,y,a_1, \ldots, a_n]^{V_\kappa})) \]
}}
\step{2}{\pflet{$a_1, \ldots, a_n, B \in V_\kappa$}}
\step{3}{\assume{for all $x \in B$, there exists at most one $y \in V_\kappa$ such that $P[x,y,a_1, \ldots, a_n]^{V_\kappa}$.}}
\step{4}{\pflet{$F = \{(x,y) \in B \times V_\kappa \mid P[x,y,a_1, \ldots, a_n]^{V_\kappa} \}$}}
\step{5}{\pflet{$C = \ran F$} \prove{$C \in V_\kappa$}}
\step{6}{\pflet{$S = \{ \rank F(x) \mid x \in \dom F \}$}}
\step{7}{$|S| < \kappa$}
\begin{proof}
	\pf\ Since $|S| \leq |\dom F| \leq |B| < \kappa$.
\end{proof}
\step{8}{$\forall \alpha \in S. \alpha < \kappa$}
\begin{proof}
	\pf\ Since $F(x) \in V_\kappa$ for all $x \in \dom F$.
\end{proof}
\step{9}{$\sup S < \kappa$}
\begin{proof}
	\pf\ Since $\kappa$ is regular.
\end{proof}
\step{10}{$\rank C \leq \sup S + 1$}
\step{11}{$\rank C < \kappa$}
\step{12}{$C \in V_\kappa$}
\qed
\end{proof}
\chapter{Group Theory}

\section{Groups}

\begin{df}[Group]
A \emph{group} $G$ consists of a set $G$ and a function $\cdot : G^2 \rightarrow G$ such that:
\begin{enumerate}
\item $\cdot$ is associative
\item There exists $e \in G$ such that $\forall x \in G. xe = x$ and $\forall x \in G. \exists y \in G. xy = e$.
\end{enumerate}
\end{df}

\begin{prop}
The inverse of an element in a group is unique.
\end{prop}

\begin{proof}
\pf
\step{1}{\assume{$b$ and $b'$ are inverses of $a$.}}
\step{2}{$b = b'$}
\begin{proof}
	\pf
	\begin{align*}
		b & = be \\
		& = bab' \\
		& = eb' \\
		& = b'
	\end{align*}
\end{proof}
\qed
\end{proof}

\begin{df}
We write $x^{-1}$ for the inverse of $x$.
\end{df}

\begin{prop}
\label{prop:groupcancel}
In any group, if $ab = ac$ then $b = c$.
\end{prop}

\begin{proof}
\pf
\begin{align*}
b & = eb \\
& = a^{-1}ab \\
& = a^{-1}ac \\
& = ec \\
& = c & \qed
\end{align*}
\end{proof}

\begin{df}
Let $\mathrm{Grp}$ be the category of groups and group homomorphisms.
\end{df}

\begin{df}
We identify any group $G$ with the category $G$ with one object whose morphisms are the elements of $G$, with composition given by the multiplication in $G$.
\end{df}

\section{Abelian Groups}

\begin{df}[Abelian group]
An \emph{Abelian group} is a group whose multiplication is commutative.

We may say we are writing an Abelian group \emph{additively}, meaning we write $a + b$ for $ab$, $0$ for $e$ and $-a$ for $a^{-1}$. In this case we write $a-b$ for $ab^{-1}$.
\end{df}

\chapter{Ring Theory}

\section{Rings}

\begin{df}[Commutative Ring]
A \emph{commutative ring} consists of a set $R$ and two binary operations $+$, $\cdot$ on $R$ such that:
\begin{itemize}
\item
$D$ is an Abelian group under $+$. Let us write 0 for its identity element.
\item
$\cdot$ is commutative and associative, and distributes over $+$.
\item
$\cdot$ has an identity element 1 that is different from 0.
\end{itemize}
\end{df}

\begin{prop}
\label{prop:timeszero}
In any commutative ring, $0x = 0$.
\end{prop}

\begin{proof}
\pf
\begin{align*}
(0+0)x & = 0x \\
\therefore 0x + 0x & = 0x + 0 \\
\therefore 0x & = 0 & (\text{Proposition \ref{prop:groupcancel}}) \qed
\end{align*}
\end{proof}

\begin{prop}
In any commutative ring, $(-a)b = -(ab)$.
\end{prop}

\begin{proof}
\pf
\begin{align*}
ab + (-a)b & = (a + (-a))b \\
& = 0b \\
& = 0 & (\text{Proposition \ref{prop:timeszero}}) \qed
\end{align*}
\end{proof}

\section{Ordered Rings}

\begin{df}[Ordered Commutative Ring]
An \emph{ordered commutative ring} consists of a commutative ring $R$ with a linear order $<$ on $R$ such that:
\begin{itemize}
\item for all $x,y,z \in R$, we have $x < y$ if and only if $x + z < y + z$.
\item for all $x,y,z \in R$, if $0 < z$ then we have $x < y$ if and only if $xz < yz$.
\end{itemize}
\end{df}

\begin{prop}
In any ordered commutative ring, $0 < 1$.
\end{prop}

\begin{proof}
\pf\ If $1 < 0$ then we have $0 < -1$ and so $0 < (-1)(-1) = 1$, which is a contradiction. \qed
\end{proof}

\begin{prop}
The ordering on an ordered commutative ring is dense; that is, if $x < y$ then there exists $z$ such that $x < z < y$.
\end{prop}

\begin{proof}
\pf\ Take $z = (x + y) / 2$. \qed
\end{proof}

\section{Integral Domains}

\begin{df}[Integral Domain]
An \emph{integral domain} is a commutative ring such that, for all $a,b \in D$, if $ab = 0$ then $a = 0$ or $b = 0$.
\end{df}

\begin{prop}
\label{prop:multcancel}
In any integral domain, if $ab = ac$ and $a \neq 0$ then $b = c$.
\end{prop}

\begin{proof}
\pf\ We have $a(b-c) = 0$ and $a \neq 0$ so $b - c = 0$ hence $b = c$. \qed
\end{proof}

\begin{df}[Ordered Integral Domain]
An \emph{ordered integral domain} is an ordered commutative ring that is an integral domain.
\end{df}

\chapter{Field Theory}

\section{Fields}

\begin{df}[Field]
A \emph{field} $F$ is a commutative ring such that $0 \neq 1$ and, for all $x \in F$, if $x \neq 0$ then there exists $y \in F$ such that $xy = 1$.
\end{df}

\begin{prop}
Every field is an integral domain.
\end{prop}

\begin{proof}
\pf\ If $ab = 0$ and $a \neq 0$ then $b = a^{-1}ab = 0$. \qed
\end{proof}

\begin{prop}
In any field $F$, we have $F - \{0\}$ is an Abelian group under multiplication.
\end{prop}

\begin{proof}
\pf\ Immediate from the definition. \qed
\end{proof}

\begin{df}[Field of Fractions]
Let $D$ be an integral domain. The \emph{field of fractions} of $D$ is the quotient set $F = (D \times (D - \{0\})) / \sim$ where
\[ (a,b) \sim (c,d) \Leftrightarrow ad = bc \]
under
\begin{align*}
[(a,b)] + [(c,d)] & = [(ad+bc,bd)] \\
[(a,b)][(c,d)] & = [(ac,bd)]
\end{align*}

We prove this is a field.
\end{df}

\begin{proof}
\pf
\step{1}{$\sim$ is an equivalence relation on $D \times (D - \{0\})$.}
\begin{proof}
\pf
\step{1}{$\sim$ is reflexive.}
\begin{proof}
	\pf\ We always have $ab = ba$.
\end{proof}
\step{2}{$\sim$ is symmetric.}
\begin{proof}
	\pf\ If $ad = bc$ then $cb = da$.
\end{proof}
\step{3}{$\sim$ is transitive.}
\begin{proof}
	\step{a}{\assume{$(a,b) \sim (c,d) \sim (e,f)$}}
	\step{b}{$ad = bc$ and $cf = de$}
	\step{c}{$adf = bde$}
	\begin{proof}
		\pf\ $adf = bcf = bde$
	\end{proof}
	\step{d}{$af = be$}
	\begin{proof}
		\pf\ Proposition \ref{prop:multcancel}.
	\end{proof}
\end{proof}
\qed
\end{proof}
\step{2}{Addition is well-defined.}
\begin{proof}
\pf
\step{1}{If $b \neq 0$ and $d \neq 0$ then $bd \neq 0$.}
\begin{proof}
	\pf\ Since $D$ is an integral domain.
\end{proof}
\step{2}{If $ab' = a'b$ and $cd' = c'd$ then $(ad+bc)b'd' = (a'd' + b'c')bd$.}
\begin{proof}
	\pf
	\begin{align*}
		(ad+bc)b'd' & = ab'dd' + bb'cd' \\
		& = a'bdd' + bb'c'd \\
		& = (a'd' + b'c')bd
	\end{align*}
\end{proof}
\qed
\end{proof}
\step{3}{Multiplication is well-defined.}
\begin{proof}
\pf
\step{1}{If $b \neq 0$ and $d \neq 0$ then $bd \neq 0$.}
\begin{proof}
	\pf\ Since $D$ is an integral domain.
\end{proof}
\step{2}{If $[(a,b)] = [(a',b')]$ and $[(c,d)] = [(c',d')]$ then $[(ac,bd)] = [(a'c',b'd')]$.}
\begin{proof}
	\pf\ If $ab' = a'b$ and $cd' = c'd$ then $acb'd' = a'c'bd$.
\end{proof}
\qed
\end{proof}
\step{4}{Addition is commutative.}
\begin{proof}
\pf\ $[(a,b)] + [(c,d)] = [(ad+bc,bd)] = [(cb+da,db)] = [(c,d)] + [(a,b)]$ \qed
\end{proof}
\step{5}{Addition is associative.}
\begin{proof}
\pf
\begin{align*}
[(a,b)] + ([(c,d)] + [(e,f)]) & = [(a,b)] + [(cf+de,df)] \\
& = [(adf + bcf + bde,bdf)] \\
& = [(ad+bc,bd)] + [(e,f)] \\
& = ([(a,b)] + [(c,d)]) + [(e,f)] & \qed
\end{align*}
\end{proof}
\step{6}{For any $x \in F$ we have $x + [(0,1)] = x$}
\begin{proof}
\pf\ $[(a,b)] + [(0,1)] = [(a \cdot 1 + b \cdot 0, b \cdot 1)] = [(a,b)]$ \qed
\end{proof}
\step{7}{For any $x \in F$, there exists $y \in F$ such that $x + y = [(0,1)]$.}
\begin{proof}
\pf\ $[(a,b)] + [(-a,b)] = [(ab-ab,b^2)] = [(0,b^2)] = [(0,1)]$ \qed
\end{proof}
\step{8}{Multiplication is commutative.}
\begin{proof}
\pf\ $[(a,b)][(c,d)] = [(c,d)][(a,b)] = [(ac,bd)]$. \qed
\end{proof}
\step{9}{Multiplication is assocative.}
\begin{proof}
\pf\ $[(a,b)]([(c,d)][(e,f)]) = ([(a,b)][(c,d)])[(e,f)] = [(ace,bdf)]$. \qed
\end{proof}
\step{10}{For any $x \in F$ we have $x[(1,1)] = x$}
\begin{proof}
\pf\ $[(a,b)][(1,1)] = [(a,b)]$ \qed
\end{proof}
\step{11}{For any non-zero $x \in F$, there exists $y \in F$ such that $xy = [(1,1)]$.}
\begin{proof}
\pf
\step{a}{\pflet{$[(a,b)] \in \mathbb{Q}$}}
\step{b}{\assume{$[(a,b)] \neq [(0,1)]$}}
\step{c}{$a \neq 0$}
\step{d}{$[(a,b)][(b,a)] = [(1,1)]$}
\qed
\end{proof}
\qed
\end{proof}

\begin{df}
For any field $F$, let $N(F)$ be the intersection of all the subsets $S \subseteq F$ such that $1 \in S$ and $\forall x \in S. x + 1 \in S$.
\end{df}

\begin{df}[Characteristic Zero]
A field $F$ has \emph{characteristic 0} iff $0 \notin N(F)$.
\end{df}

\begin{prop}
\label{prop:NFbij}
In a field $F$ with characteristic 0, the function $n : \mathbb{N} \rightarrow N(F)$ defined by
\begin{align*}
n(0) & = 1 \\
n(x+1) & = n(x) + 1
\end{align*}
is a bijection.
\end{prop}

\begin{proof}
\pf
\step{1}{$n$ is injective.}
\begin{proof}
	\step{a}{\assume{for a contradiction $n(i) = n(j)$ with $i \neq j$}}
	\step{b}{\assume{w.l.o.g. $i < j$}}
	\step{c}{$n(j-i) = 0$}
	\qedstep
	\begin{proof}
		\pf\ This contradicts the fact that $F$ has characteristic 0.
	\end{proof}
\end{proof}
\step{2}{$n$ is surjective.}
\begin{proof}
	\pf\ Since $\ran n$ is a subset of $F$ that includes 1 and is closed under $+1$.
\end{proof}
\qed
\end{proof}

\begin{df}
In any field $F$, let
\[ I(F) = N(F) \cup \{0\} \cup \{-x \mid x \in N(F) \} \]
\end{df}

\begin{df}
In any field $F$, let
\[ Q(F) = \{ x/y \mid x,y \in I(F), y \neq 0 \} \]
\end{df}

\begin{prop}
$Q(F)$ is the smallest subfield of $F$.
\end{prop}

\begin{proof}
\pf\ $Q(F)$ is closed under $+$ and $\cdot$, and any subset of $F$ closed under $+$ and $\cdot$ that contains 0 and 1 must include $Q(F)$. \qed
\end{proof}

\begin{thm}
\label{thm:Qisounique}
Let $F$ and $G$ be fields of characteristic 0. Then there exists a unique field isomorphism between $Q(F)$ and $Q(G)$.
\end{thm}

\begin{proof}
\pf
\step{1}{\pflet{$\phi : N(F) \rightarrow N(G)$ be the unique function such that $\phi(1) = 1$ and $\forall x \in N(F). \phi(x+1) = \phi(x) + 1$.}}
\step{2}{$\phi$ is a bijection.}
\begin{proof}
	\pf\ Similar to Proposition \ref{prop:NFbij}.
\end{proof}
\step{3}{$\forall x,y \in N(F). \phi(x+y) = \phi(x) + \phi(y)$}
\begin{proof}
	\pf\ Induction on $y$.
\end{proof}
\step{4}{$\forall x,y \in N(F). \phi(xy) = \phi(x)\phi(y)$}
\begin{proof}
	\pf\ Induction on $y$.
\end{proof}
\step{5}{Extend $\phi$ to a bijection $I(F) \cong I(G)$ such that $\forall x,y \in I(F).\phi(x+y) = \phi(x) + \phi(y)$ and $\forall x,y \in I(F). \phi(xy) = \phi(x)\phi(y)$}
\begin{proof}
	\step{a}{Define $\phi(0) = 0$ and $\phi(-x) = -\phi(x)$ for $x \in N(F)$}
	\begin{proof}
		\step{i}{$0 \notin N(F)$}
		\step{ii}{For all $x \in N(F)$ we have $-x \notin N(F)$}
		\begin{proof}
			\pf\ Then we would have $x + -x = 0 \in N(F)$.
		\end{proof}
		\step{iii}{For all $x \in N(F)$ we have $-x \neq 0$}
	\end{proof}
	\step{b}{For all $x,y \in I(F)$ we have $\phi(x+y) = \phi(x) + \phi(y)$}
	\begin{proof}
		\pf\ Case analysis on $x$ and $y$.
	\end{proof}
	\step{c}{For all $x,y \in I(F)$ we have $\phi(xy) = \phi(x) \phi(y)$}
	\begin{proof}
		\pf\ Case analysis on $x$ and $y$.
	\end{proof}
\end{proof}
\step{6}{Extend $\phi$ to a bijection $Q(F) \cong Q(G)$ such that $\forall x,y \in Q(F).\phi(x+y) = \phi(x) + \phi(y)$ and $\forall x,y \in Q(F). \phi(xy) = \phi(x)\phi(y)$}
\begin{proof}
	\step{a}{Define $\phi(x/y) = \phi(x)/\phi(y)$}
\end{proof}
\step{7}{$\phi$ is unique.}
\begin{proof}
	\step{a}{\pflet{$\theta$ satisfy the theorem.}}
	\step{b}{For all $x \in N(F)$ we have $\theta(x) = \phi(x)$}
	\step{c}{For all $x \in I(F)$ we have $\theta(x) = \phi(x)$}
	\step{d}{For all $x \in Q(F)$ we have $\theta(x) = \phi(x)$}
\end{proof}
\qed
\end{proof}

\section{Ordered Fields}

\begin{df}[Ordered Field]
An \emph{ordered field} is an ordered commutative ring that is a field.
\end{df}

\begin{prop}
Every ordered field $F$ has characteristic 0.
\end{prop}

\begin{proof}
\pf\ We have $0 < n$ for all $n \in N(F)$. \qed
\end{proof}

\begin{prop}
Let $F$ be a field of characteristic 0. Then there exists a unique relation $<$ on $Q(F)$ that makes $Q(F)$ into an ordered field.
\end{prop}

\begin{proof}
\pf\ Easy. \qed
\end{proof}

\begin{cor}
\label{cor:isoQs}
Let $F$ and $G$ be ordered fields. Let $\phi$ be the unique field isomorphism between $Q(F)$ and $Q(G)$. Then $\phi$ is an ordered field isomorphism.
\end{cor}

\begin{df}[Archimedean]
An ordered field $F$ is \emph{Archimedean} iff
\[ \forall x \in F. \exists n \in N(F). n > x \enspace . \]
\end{df}

\begin{prop}
\label{prop:Archimedean1}
Let $F$ be an Archimedean ordered field. Let $x,y \in F$ with $x > 0$. Then there exists $n \in N(F)$ such that $nx > y$.
\end{prop}

\begin{proof}
\pf\ Pick $n > y/x$. \qed
\end{proof}

\begin{prop}
\label{prop:Qdense}
Let $F$ be an Archimedean ordered field. For all $x,y \in F$, if $x < y$, then there exists $r \in Q(F)$ such that $x < r < y$.
\end{prop}

\begin{proof}
\pf
\step{one}{\case{$x > 0$}}
\begin{proof}
\step{1}{\pick\ $n \in N(F)$ such that $n(y-x) > 1$}
\begin{proof}
	\pf\ Proposition \ref{prop:Archimedean1}.
\end{proof}
\step{2}{$ny > 1 + nx$}
\step{3}{\pflet{$m$ be the least element of $N(F)$ such that $m > nx$.}}
\step{4}{$m-1 \leq nx$}
\step{5}{$nx < m < ny$}
\step{6}{$x < m/n < y$}
\end{proof}
\step{two}{\case{$x \leq 0$}}
\begin{proof}
	\step{1}{\pick\ $k \in N(F)$ such that $k > -x$}
	\step{2}{$0 < x+k < y + k$}
	\step{3}{\pick\ $r \in Q(F)$ such that $x + k < r < y + k$}
	\begin{proof}
		\pf\ \stepref{one}
	\end{proof}
	\step{4}{$x < r-k < y$}
\end{proof}
\end{proof}

\begin{df}[Complete]
An ordered field $F$ is \emph{complete} iff every nonempty subset of $F$ bounded above has a least upper bound.
\end{df}

\begin{prop}
Every complete ordered field is Archimedean.
\end{prop}

\begin{proof}
\pf
\step{1}{\pflet{$F$ be a complete ordered field.}}
\step{2}{\pflet{$x \in F$}}
\step{3}{\assume{for a contradiction there is no member of $N(F)$ greater than $x$.}}
\step{4}{$x$ is an upper bound for $N(F)$.}
\step{5}{\pflet{$y = \sup N(F)$}}
\step{6}{\pick\ $n \in N(F)$ such that $y-1 < n$}
\step{7}{$y < n + 1$}
\qedstep
\begin{proof}
	\pf\ This is a contradiction.
\end{proof}
\qed
\end{proof}

\begin{prop}
Let $F$ be a complete ordered field and $a \in F$ be nonnegative. Then there exists $b \in F$ such that $b^2 = a$.
\end{prop}

\begin{proof}
\pf
\step{1}{\pflet{$B = \{ x \in F \mid 0 \leq x \leq 1 + a \}$}}
\step{2}{\pflet{$\phi : B \rightarrow B$ be the function
\[ \phi(x) = x + \frac{1}{2(1+a)}(a-x^2) \enspace . \]}}
\step{3}{$\phi$ is strictly monotone.}
\begin{proof}
	\step{a}{\pflet{$0 \leq x < y \leq 1 + a$}}
	\step{b}{$1 - \frac{x+y}{2(1+a)} > 0$}
	\step{c}{$\phi(y) - \phi(x) = (y-x)(1 - \frac{x+y}{2(1+a)}) > 0$}
	\step{d}{$\phi(x) < \phi(y)$}
\end{proof}
\step{4}{\pick\ $b \in B$ such that $\phi(b) = b$.}
\begin{proof}
	\pf\ Knaster Fixed-Point Theorem.
\end{proof}
\step{5}{$b^2 = a$}
\qed
\end{proof}

\begin{thm}[Uniqueness of the Complete Ordered Field]
If $F$ and $G$ are complete ordered fields, then there exists a unique bijection $\phi : F \cong G$ such that, for all $x,y \in F$,
\begin{align*}
\phi(x+y) & =\phi(x) + \phi(y) \\
\phi(xy) & = \phi(x) \phi(y)
\end{align*}
This bijection also satisfies: for all $x,y \in F$,
\[ x < y \Leftrightarrow \phi(x) < \phi(y) \enspace . \]
\end{thm}

\begin{proof}
\pf
\step{1}{\pick\ a bijection $\phi : Q(F) \cong Q(G)$ such that, for all $x,y \in Q(F)$,
\begin{align*}
\phi(x+y) & = \phi(x) + \phi(y) \\
\phi(xy) & = \phi(x)\phi(y) \\
x < y & \Leftrightarrow \phi(x) < \phi(y)
\end{align*}}
\begin{proof}
	\pf\ Corollary \ref{cor:isoQs}.
\end{proof}
\step{2}{$Q(F)$ intersects every interval in $F$.}
\begin{proof}
	\pf\ Proposition \ref{prop:Qdense}.
\end{proof}
\step{3}{$Q(G)$ intersects every interval in $G$.}
\begin{proof}
	\pf\ Proposition \ref{prop:Qdense}.
\end{proof}
\step{4}{\pick\ an order isomorphism $\psi : F \cong G$ that extends $\phi$.}
\begin{proof}
	\pf\ Theorem \ref{thm:extendorderiso}.
\end{proof}
\step{5}{$\forall x,y \in F. \psi(x+y) = \psi(x) + \psi(y)$}
\begin{proof}
	\step{a}{\pflet{$x,y \in F$}}
	\step{b}{$\psi(x) + \psi(y) \nless \psi(x+y)$}
	\begin{proof}
		\step{i}{\assume{for a contradiction $\psi(x) + \psi(y) < \psi(x+y)$}}
		\step{ii}{\pick\ $r' \in Q(G)$ such that $\psi(x) < r' < \psi(x+y) - \psi(y)$}
		\step{iii}{\pick\ $s' \in Q(G)$ such that $\psi(y) < s' < \psi(x+y) - r'$}
		\step{vi}{$r' + s' < \psi(x+y)$}
		\step{vii}{\pick\ $r,s \in Q(F)$ such that $\phi(r) = r'$ and $\phi(s) = s'$}
		\step{viii}{$\phi(r+s) = r' + s'$}
		\step{ix}{$\psi(x) < \psi(r)$}
		\step{x}{$\psi(y) < \psi(s)$}
		\step{xi}{$\psi(x+y) >\psi(r+s)$}
		\step{xii}{$x < r$}
		\step{xiii}{$y < s$}
		\step{xiv}{$x+y > r+s$}
		\qedstep
		\begin{proof}
			\pf\ This is a contradiction.
		\end{proof}		 
	\end{proof}
	\step{c}{$\psi(x+y) \nless \psi(x) + \psi(y)$}
	\begin{proof}
		\pf\ Similar.
	\end{proof}
\end{proof}
\step{6}{$\forall x,y \in F. \psi(xy) = \psi(x)\psi(y)$}
\begin{proof}
	\step{a}{\pflet{$x,y \in F$}}
	\step{b}{\case{$x$ and $y$ are positive.}}
	\begin{proof}
		\step{i}{$\psi(x) \psi(y) \nless \psi(xy)$}
		\begin{proof}
			\step{one}{\assume{for a contradiction $\psi(x) \psi(y) < \psi(xy)$}}
			\step{two}{\pick\ $r' \in Q(G)$ such that $\psi(x) < r' < \psi(xy) / \psi(y)$}
			\step{three}{\pick\ $s' \in Q(G)$ such that $\psi(y) < s' < \psi(xy) / r'$}
			\step{four}{$r's' < \psi(xy)$}
			\step{five}{\pick\ $r,s \in Q(F)$ such that $\phi(r) = r'$ and $\phi(s) = s'$}
			\step{six}{$\phi(rs) = r's'$}
			\step{seven}{$x < r$, $y < s$ and $rs < xy$}
			\qedstep
			\begin{proof}
				\pf\ This is a contradiction.
			\end{proof}
		\end{proof}
		\step{ii}{$\psi(xy) \nless \psi(x) \psi(y)$}
		\begin{proof}
			\pf\ Similar.
		\end{proof}
	\end{proof}
	\step{c}{\case{$x$ and $y$ are not both positive.}}
	\begin{proof}
		\pf\ Follows from \stepref{b} since $\psi(-x) = -\psi(x)$ by \stepref{5}.
	\end{proof}
\end{proof}
\step{7}{For any field isomorphism $\theta : F \cong G$, we have $\theta = \psi$.}
\begin{proof}
	\step{a}{$\theta \restriction Q(F) = \phi$}
	\begin{proof}
		\pf\ Theorem \ref{thm:Qisounique}.
	\end{proof}
	\step{b}{$\theta$ is strictly monotone.}
	\begin{proof}
		\step{i}{\pflet{$x,y \in F$ with $x < y$}}
		\step{ii}{$y-x > 0$}
		\step{iii}{\pick\ $z \in F$ such that $z^2 = y-x$}
		\step{iv}{$\theta(z)^2 = \theta(y) - \theta(x)$}
		\step{v}{$\theta(y) - \theta(x) > 0$}
		\step{vi}{$\theta(x) < \theta(y)$}
	\end{proof}
	\step{c}{$\theta = \psi$}
	\begin{proof}
		\pf\ By the uniqueness of $\psi$.
	\end{proof}
\end{proof}
\qed
\end{proof}

\chapter{Number Systems}

\section{The Integers}

\begin{df}
The set of \emph{integers} $\mathbb{Z}$ is the quotient set $\mathbb{N}^2 / \sim$, where $(m,n) \sim (p,q)$ iff $m + q = n + p$.

We prove$\sim$ is an equivalence relation on $\mathbb{N}^2$.
\end{df}

\begin{proof}
\pf
\step{1}{$\sim$ is reflexive.}
\begin{proof}
	\pf\ For all $m,n \in \mathbb{N}$ we have $m + n = n + m$.
\end{proof}
\step{2}{$\sim$ is symmetric.}
\begin{proof}
	\pf\ If $m + q = n + p$ then $p + n = q + m$.
\end{proof}
\step{3}{$\sim$ is transitive.}
\begin{proof}
	\step{a}{\assume{$(m,n) \sim (p,q) \sim (r,s)$}}
	\step{b}{$m + q = n + p$ and $p + s = q + r$}
	\step{c}{$m + q + s = n + q + r$}
	\step{d}{$m + s = n + r$}
	\begin{proof}
		\pf\ By cancellation.
	\end{proof}
\end{proof}
\qed
\end{proof}

\begin{df}[Addition]
Define \emph{addition} $+$ on $\mathbb{Z}$ by $[(m,n)] + [(p,q)] = [(m+p,n+q)]$.

We prove this is well-defined.
\end{df}

\begin{proof}
\pf\ If $m + n' = n + m'$ and $p + q' = q + p'$ then $m + p + n' + q' = n + q + m' + p'$. \qed
\end{proof}

\begin{prop}
\label{prop:plusZcomm}
Addition on $\mathbb{Z}$ is commutative.
\end{prop}

\begin{proof}
\pf\ $[(m,n)] + [(p,q)] = [(m+p,n+q)] = [(p+m,q+n)] = [(p,q)] + [(m,n)]$. \qed
\end{proof}

\begin{prop}
\label{prop:plusZassoc}
Addition on $\mathbb{Z}$ is associative.
\end{prop}

\begin{proof}
\pf\ $[(m,n)] + ([(p,q)] + [(r,s)]) = [(m+p+r,n+q+s)] = ([(m,n)] + [(p,q)]) + [(r,s)]$. \qed
\end{proof}

\begin{prop}
Given natural numbers $m$ and $n$, we have $[(m,0)] = [(n,0)]$ iff $m = n$.
\end{prop}

\begin{proof}
\pf\ Immediate from definitions. \qed
\end{proof}

\begin{df}
We identify any natural number $n$ with the integer $[(n,0)]$.
\end{df}

\begin{prop}
Addition on integers agrees with addition on natural numbers.
\end{prop}

\begin{proof}
\pf\ Since $[(m,0)] + [(n,0)] = [(m+n,0)]$. \qed
\end{proof}

\begin{prop}
\label{prop:plusZzero}
For all $a \in \mathbb{Z}$ we have $a + 0 = a$.
\end{prop}

\begin{proof}
\pf\ $[(m,n)] + [(0,0)] = [(m+0,n+0)] = [(m,n)]$. \qed
\end{proof}

\begin{prop}
\label{prop:plusZinv}
For all $a \in \mathbb{Z}$, there exists $b \in \mathbb{Z}$ such that $a + b = 0$.
\end{prop}

\begin{proof}
\pf\ $[(m,n)] + [(n,m)] = [(m+n,m+n)] = [(0,0)]$ \qed
\end{proof}

\begin{prop}
\label{prop:Zgroup}
The integers form an Abelian group under addition.
\end{prop}

\begin{proof}
\pf\ Proposition \ref{prop:plusZcomm}, \ref{prop:plusZassoc}, \ref{prop:plusZzero}, \ref{prop:plusZinv}. \qed
\end{proof}

\begin{df}
Define multiplication $\cdot$ on $\mathbb{Z}$ by: $[(m,n)][(p,q)] = [(mp+nq,mq+np)]$.

We prove this is well defined.
\end{df}

\begin{proof}
\pf
\step{1}{\assume{$m + n' = n + m'$ and $p + q' = q + p'$} \prove{$mp + nq + m'q' + n'p' = mq + np + m'p' + n'q'$}}
\step{2}{$mp + n'p = np + m'p$}
\step{3}{$nq + m'q = mq + n'q$}
\step{4}{$m'p + m'q' = m'q + m'p'$}
\step{5}{$n'q + n'p' = n'p + n'q'$}
\step{6}{$mp + n'p + nq + m'q + m'p + m'q' + n'q + n'p'
= np + m'p + mq + n'q + m'q + m'p' + n'p + n'q'$}
\step{7}{$mp + nq + m'q' + n'p' = mq + np + m'p' + n'q'$}
\begin{proof}
	\pf\ By cancellation.
\end{proof}
\qed
\end{proof}

\begin{prop}
Multiplication on integers agrees with multiplication on natural numbers.
\end{prop}

\begin{proof}
\pf\ Since $[(m,0)][(n,0)] = [(mn+0,m0+n0)] = [(mn,0)]$. \qed
\end{proof}

\begin{prop}
\label{prop:timesZcomm}
Multiplication on $\mathbb{Z}$ is commutative.
\end{prop}

\begin{proof}
\pf\ $[(m,n)][(p,q)] = [(mp+nq,mq+np)] = [(pm+qn,pn+qm)] = [(p,q)][(m,n)]$. \qed
\end{proof}

\begin{prop}
\label{prop:timesZassoc}
Multiplication on $\mathbb{Z}$ is associative.
\end{prop}

\begin{proof}
\pf
\begin{align*}
[(m,n)]([(p,q)][(r,s)]) & = [(m,n)][(pr+qs,ps+qr)] \\
& = [(mpr+mqs+nps+nqr,mps+mqr+npr+nqs)] \\
& = [(mp+nq,mq+np)][(r,s)] \\
& = ([(m,n)][(p,q)])[(r,s)] & \qed
\end{align*}
\end{proof}

\begin{prop}
\label{prop:timesZdist}
Multiplication distributes over addition.
\end{prop}

\begin{proof}
\pf
\begin{align*}
[(m,n)]([(p,q)]+[(r,s)]) & = [(m,n)][(p+r,q+s)] \\
& = [(mp+mr+nq+ns, np+nr+mq+ms)] \\
[(m,n)][(p,q)] + [(m,n)][(r,s)] & = [(mp+nq,mq+np)] + [(mr+ns,ms+nr)] \\
& = [(mp+nq+mr+ns,mq+np+ms+nr)] & \qed
\end{align*}
\end{proof}

\begin{prop}
\label{prop:timesZone}
For any integer $a$ we have $a1 = a$.
\end{prop}

\begin{proof}
\pf\ Since $[(m,n)][(1,0)] = [(m1+n0,m0+n1)] = [(m,n)]$. \qed
\end{proof}

\begin{prop}
\label{prop:Zno_zero_div}
For any integeres $a$ and $b$, if $ab = 0$ then $a = 0$ or $b = 0$.
\end{prop}

\begin{proof}
\pf
\step{1}{\assume{$[(m,n)][(p,q)] = [(0,0)]$}}
\step{2}{$mp+nq = mq+np$}
\step{3}{\assume{$[(m,n)] \neq [(0,0)]$}}
\step{4}{$m \neq n$ \prove{$p = q$}}
\step{5}{\case{$m < n$}}
\begin{proof}
	\step{a}{$p \nless q$}
	\begin{proof}
		\pf\ If $p < q$ then $mq+np < mp+nq$ by Proposition \ref{prop:intmultlemma}.
	\end{proof}
	\step{b}{$q \nless p$}
	\begin{proof}
		\pf\ If $q < p$ then $mp+nq < mq+np$ by Proposition \ref{prop:intmultlemma}.
	\end{proof}
	\step{c}{$p = q$}
	\begin{proof}
		\pf\ By trichotomy.
	\end{proof}
\end{proof}
\step{6}{\case{$n < m$}}
\begin{proof}
	\pf\ Similar.
\end{proof}
\qed
\end{proof}

\begin{prop}
The integers $\mathbb{Z}$ form an integral domain.
\end{prop}

\begin{proof}
\pf\ Propositions \ref{prop:timesZcomm}, \ref{prop:timesZassoc}, \ref{prop:timesZdist}, \ref{prop:timesZone}, \ref{prop:Zno_zero_div}, \ref{prop:Zgroup}. \qed
\end{proof}

\begin{df}
Define $<$ on $\mathbb{Z}$ by $[(m,n)] < [(p,q)]$ if and only if $m + q < n + p$.

We prove this is well-defined.
\end{df}

\begin{proof}
\pf
\step{1}{\assume{$m + n' = n + m'$ and $p + q' = q + p'$.} \prove{$m + q < n + p$ if and only if $m' + q' < n' + p'$}}
\step{2}{$m + q < n + p$ if and only if $m' + q' < n' + p'$}
\begin{proof}
	\pf
	\begin{align*}
		m + q < n + p & \Leftrightarrow m + n' + q < n + n' + p & (\text{Corollary \ref{cor:pluslt}}) \\
		& \Leftrightarrow m' + n + q < n + n' + p \\
		& \Leftrightarrow m' + q < n' + p & (\text{Corollary \ref{cor:pluslt}}) \\
		& \Leftrightarrow m' + q + p' < n' + p + p' & (\text{Corollary \ref{cor:pluslt}}) \\
		& \Leftrightarrow m' + q' + p < n' + p + p' \\
		& \Leftrightarrow m' + q' < n' + p' & (\text{Corollary \ref{cor:pluslt}}) & \qed \\
	\end{align*}
\end{proof}
\end{proof}

\begin{prop}
The ordering on the integers agrees with the ordering on the natural numbers.
\end{prop}

\begin{proof}
\pf\ We have $[(m,0)] < [(n,0)]$ iff $m < n$. \qed
\end{proof}

\begin{prop}
$<$ is a linear order on $\mathbb{Z}$.
\end{prop}

\begin{proof}
	\pf
	\step{1}{$<$ is irreflexive.}
	\begin{proof}
		\pf\ We never have $m + n < m + n$.
	\end{proof}
	\step{2}{$<$ is transitive.}
	\begin{proof}
		\step{a}{\assume{$[(m,n)] < [(p,q)] < [(r,s)]$}}
		\step{b}{$m + q < n + p$ and $p + s < q + r$}
		\step{c}{$m + q + s < n + q + r$}
		\begin{proof}
			\pf\ $m + q + s < n + p + s < n + q + r$
		\end{proof}
		\step{d}{$m + s < n + r$}
		\begin{proof}
			\pf\ Corollary \ref{cor:pluslt}.
		\end{proof}
	\end{proof}
	\step{3}{$<$ is total.}
	\begin{proof}
		\pf\ Given natural numbers $m$, $n$, $p$ and $q$, either $m +q < n + p$, or $m + q = n + p$, or $n + p < m + q$.
	\end{proof}
	\qed
\end{proof}

\begin{df}[Positive]
An integer $a$ is \emph{positive} iff $a > 0$.
\end{df}

\begin{thm}
For any integers $a$, $b$ and $c$, we have $a < b$ if and only if $a + c < b + c$.
\end{thm}

\begin{proof}
\pf
\step{1}{If $a < b$ then $a + c < b + c$.}
\begin{proof}
	\step{a}{\pflet{$a = [(m,n)]$, $b = [(p,q)]$ and $c = [(r,s)]$.}}
	\step{b}{\assume{$a < b$}}
	\step{c}{$m + q < n + p$}
	\step{d}{$m + r + q + s < n + r + p + s$}
	\step{e}{$[(m+r,n+s)] < [(p+r,q+s)]$}
	\step{f}{$a + c < b + c$}
\end{proof}
\step{2}{If $a + c < b + c$ then $a < b$.}
\begin{proof}
	\pf\ From \stepref{1} and Proposition \ref{prop:strictmonotoneinv}.
\end{proof}
\qed
\end{proof}

\begin{prop}
Let $a$, $b$ and $c$ be integers. If $0 < c$, then $a < b$ if and only if $ac < bc$.
\end{prop}

\begin{proof}
\pf
\step{1}{\pflet{$c = [(r,s)]$}}
\step{2}{\assume{$0 < c$}}
\step{3}{$s < r$}
\step{4}{For all integers $a$ and $b$, if $a < b$ then $ac < bc$}
\begin{proof}
	\step{1}{\pflet{$a = [(m,n)]$, $b = [(p,q)]$.}}
	\step{a}{\assume{$a < b$}}
	\step{b}{$m + q < n + p$}
	\step{w}{$(m+q)r + (p+n)s < (m+q)s + (p+n)r$}
	\begin{proof}
		\pf\ Proposition \ref{prop:intmultlemma}, \stepref{3}, \stepref{b}.
	\end{proof}
	\step{x}{$mr + ns + ps + qr < ms + nr + pr + qs$}
	\step{y}{$[(mr+ns,ms+nr)] < [(pr+qs,ps+qr)]$}
	\step{z}{$ac < bc$}
\end{proof}
\step{5}{For all integers $a$ and $b$, if $ac < bc$ then $a < b$}
\begin{proof}
	\pf\ From \stepref{4} and Proposition \ref{prop:strictmonotoneinv}.
\end{proof}
\qed
\end{proof}

\begin{prop}
\label{prop:Zarchimedean}
Let $a$ be a positive integer. For any integer $b$, there exists $k \in \mathbb{N}$ such that $b < ak$.
\end{prop}

\begin{proof}
\pf
\step{1}{\case{$b \leq 0$}}
\begin{proof}
	\pf\ Take $k = 1$.
\end{proof}
\step{2}{\case{$b > 0$}}
\begin{proof}
	\pf\ Take $k = b + 1$.
\end{proof}
\qed
\end{proof}

\section{The Rationals}

\begin{df}[Rational Numbers]
The set $\mathbb{Q}$ of \emph{rational numbers} is the field of fractions over the integers.
\end{df}

\begin{prop}
For any integers $a$ and $b$, we have $[(a,1)] = [(b,1)]$ iff $a = b$.
\end{prop}

\begin{proof}
\pf\ Immediate from definitions. \qed
\end{proof}

Henceforth we identify any integer $a$ with the rational number $[(a,1)]$.

\begin{prop}
Addition on the rationals agrees with addition on the integers.
\end{prop}

\begin{proof}
\pf\ $[(a,1)] + [(b,1)] = [(a \cdot 1 + b \cdot 1, 1 \cdot 1)] = [(a + b, 1)]$. \qed
\end{proof}

\begin{prop}
Multiplication on the rationals agrees with multiplication on the integers.
\end{prop}

\begin{proof}
\pf\ $[(a,1)][(b,1)] = [(ab,1)]$ \qed
\end{proof}

\begin{df}
Define the ordering $<$ on the rationals by: if $b$ and $d$ are positive, then $[(a,b)] < [(c,d)]$ iff $ad < bc$.

We prove this is well-defined.
\end{df}

\begin{proof}
\pf
\step{1}{For any rational $q$, there exist integers $a$, $b$ with $b$ positive such that $q = [(a,b)]$.}
\begin{proof}
	\pf\ Since $[(a,b)] = [(-a,-b)]$, and if $b \neq 0$ then one of $b$ and $-b$ is positive.
\end{proof}
\step{2}{If $b$, $b'$, $d$ and $d'$ are positive, $[(a,b)] = [(a',b')]$, and $[(c,d)] = [(c',d')]$, then $ad < bc$ iff $a'd' < b'c'$.}
\begin{proof}
	\pf
	\step{a}{If $ad < bc$ then $a'd' < b'c'$.}
	\begin{proof}
		\step{i}{\assume{$ad < bc$}}
		\step{ii}{$ab'd < bb'c$}
		\step{iii}{$a'bd < bb'c$}
		\step{iv}{$a'd < b'c$}
		\step{v}{$a'dd'< b'cd'$}
		\step{vi}{$a'dd' < b'c'd$}
		\step{vii}{$a'd' < b'c'$}
	\end{proof}
	\step{b}{If $a'd' < b'c'$ then $ad < bc$.}
	\begin{proof}
		\pf\ Similar.
	\end{proof}
\end{proof}
\qed
\end{proof}

\begin{prop}
The ordering on the rationals agrees with the ordering on the integers.
\end{prop}

\begin{proof}
\pf\ We have $[(a,1)] < [(b,1)]$ if and only if $a < b$. \qed
\end{proof}

\begin{prop}
The relation $<$ is a linear ordering on $\mathbb{Q}$.
\end{prop}

\begin{proof}
\pf
\step{1}{$<$ is irreflexive.}
\begin{proof}
	\pf\ We never have $ab < ab$.
\end{proof}
\step{2}{$<$ is transitive.}
\begin{proof}
	\step{a}{\assume{$[(a,b)] < [(c,d)] < [(e,f)]$ where $b$, $d$ and $f$ are positive.}}
	\step{b}{$ad < bc$ and $cf < de$}
	\step{c}{$adf < bde$}
	\begin{proof}
		\pf\ $adf < bcf < bde$
	\end{proof}
	\step{d}{$af < be$}
\end{proof}
\step{3}{$<$ is total.}
\begin{proof}
	\pf\ For any integers $a$, $b$, $c$, $d$, we have $ad < bc$ or $ad = bc$ or $bc < ad$.
\end{proof}
\qed
\end{proof}

\begin{prop}
For any rationals $r$, $s$ and $t$, we have $r < s$ if and only if $r + t < s + t$.
\end{prop}

\begin{proof}
\pf
\step{1}{\pflet{$a$, $b$, $c$, $d$, $e$, $f$ be integers with $b$, $d$ and $f$ positive.}}
\step{2}{$[(a,b)] + [(e,f)] < [(c,d)] + [(e,f)]$ if and only if $[(a,b)] < [(c,d)]$.}
\begin{proof}
	\pf
	\begin{align*}
		[(a,b)] + [(e,f)] < [(c,d)] + [(e,f)] & \Leftrightarrow [(af + be, bf)] < [(cf + de, df)] \\
		& \Leftrightarrow (af + be) df < (cf+de)bf \\
		& \Leftrightarrow afdf + bedf < cfbf + debf \\
		& \Leftrightarrow afdf < cfbf \\
		& \Leftrightarrow ad < bc \\
		& \Leftrightarrow [(a,b)] < [(c,d)]
	\end{align*}
\end{proof}
\qed
\end{proof}

\begin{cor}
For any rational $r$, we have $r < 0$ if and only if $0 < -r$.
\end{cor}

\begin{df}[Absolute Value]
For any rational $r$, the \emph{absolute value} of $r$ is defined by
\[ |r| := \begin{cases}
-r & \text{if } 0 < -r \\
r & \text{otherwise}
\end{cases} \]
\end{df}

\begin{prop}
For any rationals $r$, $s$ and $t$, if $t$ is positive then $r < s$ iff $rt < st$.
\end{prop}

\begin{proof}
\pf
\step{1}{\pflet{$r = [(a,b)]$, $s = [(c,d)]$ and $t = [(e,f)]$ where $b$, $d$ and $f$ are positive.}}
\step{2}{\assume{$0 < t$}}
\step{3}{$e > 0$}
\step{4}{$rt < st$ iff $r < s$}
\begin{proof}
	\pf
	\begin{align*}
		rt < st & \Leftrightarrow [(ae,bf)] < [(ce,df)]  \\
		& \Leftrightarrow aedf < cebf \\
		& \Leftrightarrow ad < bc \\
		& \Leftrightarrow r < s
	\end{align*}
\end{proof}
\qed
\end{proof}

\begin{cor}
The rationals form an ordered field.
\end{cor}

\begin{prop}
\label{prop:Qarchimedean}
Let $p$ be a positive rational. For any rational number $r$, there exists $k \in \mathbb{N}$ such that $r < pk$.
\end{prop}

\begin{proof}
\pf
\step{1}{\pflet{$p = a/b$ and $r = c/d$ where $a$, $b$ and $d$ are positive.}}
\step{2}{\pick\ $k \in \mathbb{N}$ such that $bc < adk$}
\begin{proof}
	\pf\ Proposition \ref{prop:Zarchimedean}.
\end{proof}
\step{3}{$r < pk$}
\qed
\end{proof}

\begin{prop}
$\mathbb{Q} \approx \mathbb{N}$
\end{prop}

\begin{proof}
\pf\ Arrange the rationals in order $0/1$, $1/1$, $1/2$, $0/2$, $-1/2$, $-1/1$, $-2/1$, $-2/2$, $-2/3$, $-1/3$, $0/3$, $1/3$, $2/3$, etc. then remove all duplicates. \qed
\end{proof}
\section{The Real Numbers}

\begin{df}[Cauchy Sequence]
A \emph{Cauchy sequence} is a sequence $(q_n)$ of rationals such that, for every positive rational $\epsilon$, there exists $k \in \mathbb{N}$ such that $\forall m,n > k. |q_m - q_n| < \epsilon$.
\end{df}

%TODO: Cauchy sequences modulo $\sim$ form the reals.

\begin{df}[Dedekind Cut]
A \emph{Dedekind cut} is a set $x \subseteq \mathbb{Q}$ such that:
\begin{enumerate}
\item $\emptyset \neq x \neq \mathbb{Q}$
\item $x$ is closed downwards.
\item $x$ has no greatest member.
\end{enumerate}

The set $\mathbb{R}$ of \emph{real numbers} is the set of Dedekind cuts.
\end{df}

\begin{prop}
For any rational $q$, we have $\{ r \in \mathbb{Q} \mid r < q \} \in \mathbb{R}$.
\end{prop}

\begin{proof}
\pf
\step{1}{\pflet{$q \in \mathbb{Q}$}}
\step{2}{\pflet{$q \downarrow = \{ r \mid r < q \}$}}
\step{3}{$q \downarrow \neq \emptyset$}
\begin{proof}
	\pf\ We have $q - 1 \in q \downarrow$.
\end{proof}
\step{4}{$q \downarrow \neq \mathbb{Q}$}
\begin{proof}
	\pf\ Since $q \notin q \downarrow$.
\end{proof}
\step{5}{$q \downarrow$ is closed downwards.}
\begin{proof}
	\pf\ Trivial.
\end{proof}
\step{6}{$q \downarrow$ has no greatest element.}
\begin{proof}
	\pf\ For all $r \in q \downarrow$ we have $r < (q+r)/2 \in q \downarrow$.
\end{proof}
\qed
\end{proof}

\begin{prop}
\label{prop:QtoRwd}
For rationals $q$ and $r$, we have $q = r$ if and only if $\{ s \in \mathbb{Q} \mid s < q \} = \{ s \in \mathbb{Q} \mid s < r \}$.
\end{prop}

\begin{proof}
\pf
\step{1}{\pflet{$q \downarrow = \{ s \in \mathbb{Q} \mid s < q \}$}}
\step{2}{\pflet{$r \downarrow = \{ s \in \mathbb{Q} \mid s < r \}$}}
\step{3}{If $q = r$ then $q \downarrow = r \downarrow$}
\begin{proof}
	\pf\ Trivial.
\end{proof}
\step{4}{If $q < r$ then $q \downarrow \neq r \downarrow$}
\begin{proof}
	\pf\ We have $q \in r \downarrow$ and $q \notin q \downarrow$.
\end{proof}
\step{5}{If $r < q$ then $q \downarrow \neq r \downarrow$}
\begin{proof}
	\pf\ We have $r \in q \downarrow$ and $q \notin q \downarrow$.
\end{proof}
\qed
\end{proof}

Henceforth we identify a rational $q$ with the real number $\{ r \in \mathbb{Q} \mid r < q \}$.

\begin{df}
Define the ordering $<$ on $\mathbb{R}$ by: $x < y$ iff $x \subsetneq y$.
\end{df}

\begin{prop}
The ordering on the reals agrees with the ordering on the rationals.
\end{prop}

\begin{proof}
\pf
\step{1}{\pflet{$q, r \in \mathbb{Q}$}}
\step{2}{\pflet{$q \downarrow = \{ s \in \mathbb{Q} \mid s < q \}$.}}
\step{3}{\pflet{$r \downarrow = \{ s \in \mathbb{Q} \mid s < r \}$.} \prove{$q < r$ iff $q \downarrow \subsetneq r \downarrow$}}
\step{4}{If $q < r$ then $q \downarrow \subsetneq r \downarrow$}
\begin{proof}
	\step{a}{\assume{$q < r$}}
	\step{b}{$q \downarrow \subseteq r \downarrow$}
	\begin{proof}
		\pf\ If $s < q$ then $s < r$.
	\end{proof}
	\step{c}{$q \downarrow \neq r \downarrow$}
	\begin{proof}
		\pf\ Proposition \ref{prop:QtoRwd}.
	\end{proof}
\end{proof}
\step{5}{If $q \downarrow \subsetneq r \downarrow$ then $q < r$}
\begin{proof}
	\step{a}{\assume{$q \downarrow \subsetneq r \downarrow$}}
	\step{b}{\pick\ $s \in r \downarrow$ such that $s \notin q \downarrow$}
	\step{c}{$q \leq s < r$}
\end{proof}
\qed
\end{proof}

\begin{prop}
The ordering $<$ is a linear ordering on $\mathbb{R}$.
\end{prop}

\begin{proof}
\pf
\step{1}{$<$ is irreflexive.}
\begin{proof}
	\pf\ No set is a proper subset of itself.
\end{proof}
\step{2}{$<$ is transitive.}
\begin{proof}
	\pf\ Since the relationship $\subsetneq$ is transitive on the class of all sets.
\end{proof}
\step{3}{$<$ is total.}
\begin{proof}
	\step{a}{\pflet{$x$, $y$ be Dedekind cuts.}}
	\step{b}{\assume{$x \nsubseteq y$} \prove{$y \subsetneq x$}}
	\step{c}{\pick\ $q \in x$ such that $q \notin y$}
	\step{d}{\pflet{$r \in y$} \prove{$r \in x$}}
	\step{e}{$q \not\leq r$}
	\begin{proof}
		\pf\ Since $y$ is closed downwards.
	\end{proof}
	\step{f}{$r < q$}
	\step{g}{$r \in x$}
	\begin{proof}
		\pf\ Since $x$ is closed downwards.
	\end{proof}
\end{proof}
\qed
\end{proof}

\begin{prop}
Any bounded nonempty subset of $\mathbb{R}$ has a least upper bound.
\end{prop}

\begin{proof}
\pf
\step{1}{\pflet{$A$ be a bounded nonempty subset of $\mathbb{R}$.}}
\step{2}{$\bigcup A$ is a Dedekind cut.}
\begin{proof}
	\step{a}{$\bigcup A \neq \emptyset$}
	\begin{proof}
		\step{i}{\pick $x \in A$}
		\step{ii}{\pick\ $q \in x$}
		\step{iii}{$q \in \bigcup A$}
	\end{proof}
	\step{b}{$\bigcup A \neq \mathbb{Q}$}
	\begin{proof}
		\step{i}{\pick\ an upper bound $u$ for $A$}
		\step{ii}{\pick\ $q \notin u$ \prove{$q \notin \bigcup A$}}
		\step{iii}{\assume{for a contradiction $q \in \bigcup A$}}
		\step{iv}{\pick\ $x \in A$ such that $q \in x$}
		\step{v}{$x \leq u$}
		\step{vi}{$q \in u$}
		\qedstep
		\begin{proof}
			\pf\ This is a contradiction.
		\end{proof}
	\end{proof}
	\step{c}{$\bigcup A$ is closed downwards.}
	\begin{proof}
		\step{i}{\pflet{$q \in \bigcup A$ and $r < q$}}
		\step{ii}{\pick\ $x \in A$ such that $q \in x$}
		\step{iii}{$r \in x$}
		\step{iv}{$r \in \bigcup A$}
	\end{proof}
	\step{d}{$\bigcup A$ has no greatest element.}
	\begin{proof}
		\step{i}{\pflet{$q \in \bigcup A$}}
		\step{ii}{\pick\ $x \in A$ such that $q \in x$}
		\step{iii}{\pick\ $r \in x$ such that $q < r$}
		\step{iv}{$r \in \bigcup A$}
	\end{proof}
\end{proof}
\step{3}{$\bigcup A$ is an upper bound for $A$.}
\begin{proof}
	\pf\ For all $x \in A$ we have $x \subseteq \bigcup A$.
\end{proof}
\step{4}{For any upper bound $u$ for $\bigcup A$ we have $\bigcup A \leq u$.}
\begin{proof}
	\pf\ If $\forall x \in A. x \subseteq u$ we have $\bigcup A \subseteq u$.
\end{proof}
\qed
\end{proof}

\begin{df}[Addition]
Define \emph{addition} $+$ on the reals by
\[ x + y := \{ q + r \mid q \in x, r \in y \} \enspace . \]

We prove this is well-defined.
\end{df}

\begin{proof}
\pf
\step{1}{\pflet{$x,y \in \mathbb{R}$} \prove{$X+y$ is a Dedekind cut.}}
\step{2}{$x + y \neq \emptyset$}
\begin{proof}
	\pf\ Pick $q \in x$ and $r \in y$; then $q+r \in x+y$.
\end{proof}
\step{3}{$x + y \neq \mathbb{Q}$}
\begin{proof}
	\step{a}{\pick\ $q \notin x$ and $r \notin y$ \prove{$q+r \notin x+y$}}
	\step{b}{\assume{for a contradiction $q + r \in x + y$}}
	\step{c}{\pick\ $q' \in x$ and $r' \in y$ such that $q + r = q' + r'$}
	\step{d}{$q' < q$ and $r' < r$}
	\step{e}{$q' + r' < q + r$}
	\qedstep
	\begin{proof}
		\pf\ This is a contradiction.
	\end{proof}
\end{proof}
\step{4}{$x + y$ is closed downwards.}
\begin{proof}
	\step{a}{\pflet{$q \in x$ and $r \in y$}}
	\step{b}{\pflet{$s < q + r$} \prove{$s \in x + y$}}
	\step{c}{$s - r < q$}
	\step{d}{$s - r \in x$}
	\step{e}{$s = (s - r) + r \in x + y$}
\end{proof}
\step{5}{$x + y$ has no greatest element.}
\begin{proof}
	\step{a}{\pflet{$q \in x$ and $r \in y$} \prove{There exists $s \in x + y$ such that $q + r < s$}}
	\step{b}{\pick\ $q' \in x$ and $r' \in y$ such that $q < q'$ and $r < r'$}
	\step{c}{$q + r < q' + r' \in x + y$}
\end{proof}
\qed
\end{proof}

\begin{prop}
Addition on the reals agrees with addition on the rationals.
\end{prop}

\begin{proof}
\pf
\step{1}{\pflet{$q, r \in \mathbb{Q}$}}
\step{2}{$q \downarrow + r \downarrow \subseteq (q+r)\downarrow$}
\begin{proof}
	\pf\ If $s_1 < q$ and $s_2 < r$ then $s_1 + s_2 < q + r$.
\end{proof}
\step{3}{$(q+r) \downarrow \subseteq q \downarrow + r \downarrow$}
\begin{proof}
	\step{a}{\pflet{$s < q + r$}}
	\step{b}{$s - r < q$}
	\step{c}{\pick\ $t$ such that $s - r < t < q$}
	\step{d}{$s - t < r$}
	\step{e}{$s = t + (s - t) \in q\downarrow + r\downarrow$}
\end{proof}
\qed
\end{proof}

\begin{prop}
Addition is associative.
\end{prop}

\begin{proof}
\pf
\begin{align*}
x + (y + z) & = \{ q + r \mid q \in x, r \in y + z \} \\
& = \{ q + s_1 + s_2 \mid q \in x, s_1 \in y, s_2 \in z \} \\
& = \{ r + s_2 \mid r \in x + y, s_2 \in z \} \\
& = (x + y) + z & \qed
\end{align*}
\end{proof}

\begin{prop}
Addition is commutative.
\end{prop}

\begin{proof}
\pf
\begin{align*}
x + y & = \{ q + r \mid q \in x, r \in y \} \\
& = \{ r + q \mid r \in y, q \in x \} \\
& = y + x & \qed
\end{align*}
\end{proof}

\begin{prop}
For any $x \in \mathbb{R}$ we have $x + 0 = x$.
\end{prop}

\begin{proof}
\pf
\step{1}{$x + 0 \subseteq x$}
\begin{proof}
	\pf\ If $q \in x$ and $r < 0$ then $q + r < q$ so $q + r \in x$.
\end{proof}
\step{2}{$x \subseteq x + 0$}
\begin{proof}
	\step{a}{\pflet{$q \in x$}}
	\step{b}{\pick\ $r \in x$ such that $q < r$.}
	\begin{proof}
		\pf\ $x$ has no greatest element.
	\end{proof}
	\step{c}{$q - r < 0$}
	\step{d}{$q = r + (q-r) \in x + 0$}
\end{proof}
\qed
\end{proof}

\begin{df}
For $x \in \mathbb{R}$, define $-x := \{ q \in \mathbb{Q} \mid \exists r > q. -r \notin x \}$.
\end{df}

\begin{prop}
For all $x \in \mathbb{R}$ we have $-x \in \mathbb{R}$.
\end{prop}

\begin{proof}
\pf
\step{1}{\pflet{$x \in \mathbb{R}$}}
\step{2}{$-x \neq \emptyset$}
\begin{proof}
	\step{a}{\pick\ $s \notin x$}
	\step{b}{$-s-1 \in -x$}
\end{proof}
\step{3}{$-x \neq \mathbb{Q}$}
\begin{proof}
	\step{a}{\pick\ $s \in x$ \prove{$-s \notin -x$}}
	\step{b}{\assume{for a contradiction $-s \in -x$}}
	\step{c}{\pick\ $r > -s$ such that $-r \notin x$}
	\step{d}{$-r < s$}
	\qedstep
	\begin{proof}
		\pf\ This contradicts the fact that $x$ is closed downwards.
	\end{proof}
\end{proof}
\step{4}{$-x$ is closed downwards.}
\begin{proof}
	\pf\ Immediate from definition.
\end{proof}
\step{5}{$-x$ has no greatest element.}
\begin{proof}
	\step{a}{\pflet{$q \in -x$}}
	\step{b}{\pick\ $r > q$ such that $-r \notin x$}
	\step{c}{\pick\ $s$ such that $q < s < r$}
	\step{d}{$s \in -x$}
\end{proof}
\qed
\end{proof}

\begin{lm}
\label{lm:Rinv}
Let $p$ be a positive rational number. For any real number $x$, there exists a rational $q \in x$ such that $p + q \notin x$.
\end{lm}

\begin{proof}
\pf
\step{1}{\pick\ $q_0 \in x$}
\step{2}{There exists $k \in \mathbb{N}$ such that $q_0 + kp \notin x$}
\begin{proof}
	\step{a}{\pick\ $q_1 \notin x$}
	\step{b}{\pick\ $k \in \mathbb{N}$ such that $q_1 - q_0 < p k$}
	\begin{proof}
		\pf\ Proposition \ref{prop:Qarchimedean}.
	\end{proof}
	\step{c}{$q_1 < q_0 + kp$}
	\step{d}{$q_0 + kp \notin x$}
\end{proof}
\step{3}{\pflet{$k$ be the least natural number such that $q_0 + kp \notin x$}}
\step{4}{$k \neq 0$}
\begin{proof}
	\pf\ \stepref{1}
\end{proof}
\step{5}{\pflet{$q = q_0 + (k-1)p$}}
\step{6}{$q \in x$ and $q + p \notin x$.}
\qed
\end{proof}

\begin{prop}
For every real $x$ we have $x + (-x) = 0$.
\end{prop}

\begin{proof}
\pf
\step{1}{\pflet{$x$ be a real number.}}
\step{2}{$x + (-x) \subseteq 0$}
\begin{proof}
	\step{a}{\pflet{$q_1 \in x$ and $q_2 \in -x$}}
	\step{b}{\pick\ $r > q_2$ such that $-r \notin x$}
	\step{c}{$q_1 < -r$}
	\step{d}{$r < - q_1$}
	\step{e}{$q_2 < - q_1$}
	\step{f}{$q_1 + q_2 < 0$}
\end{proof}
\step{3}{$0 \subseteq x + (-x)$}
\begin{proof}
	\step{a}{\pflet{$p < 0$}}
	\step{b}{$0 < -p$}
	\step{c}{\pick\ $q \in x$ such that $q-p/2 \notin x$}
	\begin{proof}
		\pf\ Lemma \ref{lm:Rinv}.
	\end{proof}
	\step{d}{\pflet{$s = p/2 - q$}}
	\step{e}{$-s \notin x$}
	\step{f}{$p-q < s$}
	\step{g}{$p-q \in -x$}
	\step{h}{$p \in x + (-x)$}
\end{proof}
\qed
\end{proof}

\begin{cor}
The reals form an Abelian group under addition.
\end{cor}

\begin{prop}
For any reals $x$, $y$ and $z$, we have $x < y$ if and only if $x + z < y + z$.
\end{prop}

\begin{proof}
\pf
\step{1}{$\forall x,y,z \in \mathbb{R}. x \leq y \Rightarrow x + z \leq y + z$}
\begin{proof}
	\step{a}{\pflet{$x,y,z \in \mathbb{R}$}}
	\step{b}{\assume{$x \leq y$}}
	\step{c}{For all $q \in x$ and $r \in z$ we have $q + r \in y + z$}
\end{proof}
\step{2}{$\forall x,y,z \in \mathbb{R}. x + z = y + z \Leftrightarrow x = y$}
\begin{proof}
	\pf\ Proposition \ref{prop:groupcancel}.
\end{proof}
\step{3}{$\forall x,y,z \in \mathbb{R}. x < y \Rightarrow x + z < y + z$}
\qedstep
\begin{proof}
	\pf\ Proposition \ref{prop:strictmonotoneinv}.
\end{proof}
\qed
\end{proof}

\begin{df}[Absolute Value]
The \emph{absolute value} of a real number $x$ is defined to be
\[ |x| = \begin{cases}
x & \text{if } 0 \leq x \\
-x & \text{if } x < 0
\end{cases} \]
\end{df}

\begin{df}[Multiplication]
Define \emph{multiplication} $\cdot$ on $\mathbb{R}$ as follows:
\begin{itemize}
\item If $x$ and $y$ are non-negative then
\[ xy = 0 \cup \{ rs \mid 0 \leq r \in x \wedge 0 \leq s \in y \} \enspace . \]
\item If $x$ and $y$ are both negative then
\[ xy = (-x)(-y) \enspace . \]
\item If one of $x$ and $y$ is negative and one is non-negative then
\[ xy = -(|x||y|) \enspace .\]
\end{itemize}
We prove this is well-defined.
\end{df}

\begin{proof}
\pf
\step{1}{\pflet{$x$ and $y$ be non-negative reals.} \prove{$xy$ is real.}}
\step{2}{$xy \neq \emptyset$}
\begin{proof}
	\pf\ Since $-1 \in xy$.
\end{proof}
\step{3}{$xy \neq \mathbb{Q}$}
\begin{proof}
	\step{a}{\pick\ $r \notin x$ and $s \notin y$ \prove{$rs \notin xy$}}
	\step{b}{$0 \leq r$ and $0 \leq s$}
	\begin{proof}
		\pf\ Since $0 \subseteq x$ and $0 \subseteq y$.
	\end{proof}
	\step{c}{\assume{for a contradiction $rs \in xy$}}
	\step{d}{\pick\ $r'$ and $s'$ such that $0 \leq r' \in x$, $0 \leq s' \in y$ and $rs = r's'$}
	\step{e}{$r' < r$}
	\step{f}{$s' < s$}
	\step{g}{$r's' < rs$}
	\qedstep
	\begin{proof}
		\pf\ This is a contradiction.
	\end{proof}
\end{proof}
\step{4}{$xy$ is closed downwards.}
\begin{proof}
	\step{a}{\pflet{$q \in xy$ and $r < q$}}
	\step{b}{\case{$q \in 0$}}
	\begin{proof}
		\pf\ Then $r < q < 0$ so $r \in xy$
	\end{proof}
	\step{c}{\case{$q = s_1 s_2$ where $0 \leq s_1 \in x$ and $0 \leq s_2 \in y$}}
	\begin{proof}
		\step{i}{\assume{w.l.o.g. $0 \leq r$}}
		\step{ii}{$0 < s_1$ and $0 < s_2$}
		\step{iii}{$r / s_2 < s_1$}
		\step{iv}{$r / s_2 \in x$}
		\step{v}{$r = (r/s_2)s_2 \in xy$}
	\end{proof}
\end{proof}
\step{5}{$xy$ has no greatest element.}
\begin{proof}
	\step{a}{\pflet{$q \in xy$}}
	\step{b}{\case{$q \in 0$}}
	\begin{proof}
		\pf\ $q < q/2 \in 0$
	\end{proof}
	\step{c}{\case{$q = rs$ where $0 \leq r \in x$ and $0 \leq s \in y$}}
	\begin{proof}
		\step{i}{\pick\ $r'$ and $s'$ with $r < r' \in x$ and $s < s' \in y$}
		\step{ii}{$q < r's' \in xy$}
	\end{proof}
\end{proof}
\qed
\end{proof}

\begin{prop}
Multiplication is commutative.
\end{prop}

\begin{proof}
\pf\ Immediate from definition. \qed
\end{proof}

\begin{prop}
Multiplication is associative.
\end{prop}

\begin{proof}
\pf
\step{1}{For non-negative reals $x$, $y$ and $z$, we have $x(yz) = (xy)z$}
\begin{proof}
	\pf\ It computes to $0 \cup \{qrs \mid 0 \leq q \in x, 0 \leq r \in y, 0 \leq s \in z\}$.
\end{proof}
\step{2}{For all reals $x$, $y$ and $z$, we have $x(yz) = (xy)z$}
\begin{proof}
	\pf\ It is equal to $|x||y||z|$ if an even number of them are negative, and $-(|x||y||z|)$ otherwise.
\end{proof}
\qed
\end{proof}

\begin{prop}
Multiplication distributes over addition.
\end{prop}

\begin{proof}
\pf
\step{1}{For all non-negative reals $x$, $y$ and $z$, we have $x(y+z) = xy + xz$}
\begin{proof}
	\step{a}{\pflet{$x$, $y$ and $z$ be non-negative reals.}}
	\step{b}{$x(y+z) \subseteq xy+xz$}
	\begin{proof}
		\step{i}{\pflet{$q \in x(y+z)$}}
		\step{ii}{\case{$q < 0$}}
		\begin{proof}
			\pf\ Then we have $q/2 \in xy$ and $q/2 \in xz$ so $q \in xy+xz$.
		\end{proof}
		\step{iii}{\case{$q = rs$ where $0 \leq r \in x$ and $0 \leq s \in y+z$}}
		\begin{proof}
			\step{one}{\pick\ $s_1 \in y$ and $s_2 \in z$ such that $s = s_1 + s_2$}
			\step{two}{$rs_1 \in xy$}
			\begin{proof}
				\pf\ If $s_1 < 0$ then $rs_1 < 0$ so $rs_1 \in xy$. If $0 \leq s_1$ then we also have $rs_1 \in xy$.
			\end{proof}
			\step{three}{$rs_2 \in xz$}
			\begin{proof}
				\pf\ Similar.
			\end{proof}
			\step{four}{$q \in xy+xz$}
			\begin{proof}
				\pf\ Since $q = rs_1 + rs_2$.
			\end{proof}
		\end{proof}
	\end{proof}
	\step{c}{$xy+xz \subseteq x(y+z)$}
	\begin{proof}
		\step{i}{\pflet{$q \in xy$ and $r \in xz$.} \prove{$q+r \in x(y+z)$}}
		\step{ii}{\case{$q < 0$ and $r < 0$}}
		\begin{proof}
			\pf\ Then $q + r < 0$ so $q + r \in x(y+z)$.
		\end{proof}
		\step{iii}{\case{$q < 0$ and $r = r_1 r_2$ where $0 \leq r_1 \in x$ and $0 \leq r_2 \in z$}}
		\begin{proof}
			\step{one}{$q + r < r$}
			\step{two}{$q + r \in xz$}
			\step{three}{\assume{w.l.o.g. $0 \leq q + r$}}
			\begin{proof}
				\pf\ Otherwise $q + r \in x(y+z)$ immediately.
			\end{proof}
			\step{four}{\pick\ $s_1$, $s_2$ with $0 \leq s_1 \in x$, $0 \leq s_2 \in y$ and $q + r = s_1 s_2$}
			\step{five}{$s_2 \in y + z$}
			\begin{proof}
				\pf\ Since $0 \in z$ so $s_2 = s_2 + 0 \in y + z$.
			\end{proof}
			\step{six}{$q + r \in x(y+z)$}
		\end{proof}
		\step{iv}{\case{$q = q_1 q_2$ where $0 \leq q_1 \in x$ and $0 \leq q_2 \in y$ and $r < 0$}}
		\begin{proof}
			\pf\ Similar.
		\end{proof}
		\step{v}{\case{$q = q_1 q_2$ where $0 \leq q_1 \in x$ and $0 \leq q_2 \in y$ and $r = r_1 r_2$ where $0 \leq r_1 \in x$ and $0 \leq r_2 \in z$}}
		\begin{proof}
			\step{one}{\assume{w.l.o.g. $q_1 \leq r_1$}}
			\step{two}{$q + r \leq r_1(q_2 + r_2) \in x(y+z)$}
		\end{proof}
	\end{proof}
\end{proof}
\step{2}{For any negative real $x$ and non-negative reals $y$ and $z$, we have $x(y+z) = xy+xz$}
\begin{proof}
	\pf
	\begin{align*}
	x(y+z) = -(-x)(y+z)
	& = -((-x)y + (-x)z) & (\text{\stepref{1}}) \\
	& = -((-x)y) -((-x)z) \\
	& = xy + xz
	\end{align*}
\end{proof}
\step{3}{For any non-negative real $x$ and reals $y$ and $z$ with one negative and one non-negative, we have $x(y+z) = xy+xz$}
\begin{proof}
	\step{a}{\assume{w.l.o.g. $y$ is negative and $z$ is non-negative.}}
	\step{b}{\case{$0 \leq y + z$}}
	\begin{proof}
		\pf
		\begin{align*}
		xy + xz & = xy + x(-y + y + z) \\
		& = -(x(-y)) + x(-y + y + z) \\
		& = -(x(-y)) + x(-y) + x(y+z) & (\text{\stepref{1}}) \\
		& = x(y+z)
		\end{align*}
	\end{proof}
	\step{c}{\case{$y + z < 0$}}
	\begin{proof}
		\step{one}{$-y-z > 0$}
		\step{two}{$-y = z -y-z$}
		\step{three}{$xy + xz = x(y+z)$}
		\begin{proof}
			\pf
			\begin{align*}
				xy + xz & = -(x(-y)) + xz \\
				& = -(x(z -y-z)) + xz \\
				& = -(xz + x(-y-z)) + xz & (\text{\stepref{1}}) \\
				& = -xy -x(-y-z) + xz \\
				& = -x(-y-z) \\
				& = x(y+z)
			\end{align*}
		\end{proof}
	\end{proof}
\end{proof}
\step{4}{For any non-negative real $x$ and negative reals $y$ and $z$, we have $x(y+z) = xy+xz$}
\begin{proof}
	\pf
	\begin{align*}
		x(y+z) & = -x(-y-z) \\
		& = -(x(-y) + x(-z)) & (\text{\stepref{1}}) \\
		& = -x(-y) -x(-z) \\
		& = xy + xz
	\end{align*}
\end{proof}
\step{5}{For any negative real $x$ and reals $y$ and $z$ with one negative and one non-negative, we have $x(y+z) = xy+xz$}
\begin{proof}
	\step{a}{\assume{w.l.o.g. $y$ is negative and $z$ is non-negative.}}
	\step{b}{\case{$0 \leq y + z$}}
	\begin{proof}
		\pf
		\begin{align*}
			x(y+z) & = -((-x)(y+z)) \\
			& = -((-x)y + (-x)z) & (\text{\stepref{3}}) \\
			& = -((-x)y) -((-x)z) \\
			& = (-x)(-y) - ((-x)z) \\
			& = xy+xz
		\end{align*}
	\end{proof}
	\step{c}{\case{$y + z < 0$}}
	\begin{proof}
		\pf
		\begin{align*}
			x(y+z) & = (-x)(-y-z) \\
			& = (-x)(-y) + (-x)(-z) & (\text{\stepref{3}}) \\
			& = xy + xz
		\end{align*}
	\end{proof}
\end{proof}
\step{6}{For any negative reals $x$, $y$ and $z$, we have $x(y+z) = xy+xz$}
\begin{proof}
	\pf
	\begin{align*}
		x(y+z) & = (-x)(-y-z) \\
		& = (-x)(-y) + (-x)(-z) & (\text{\stepref{1}}) \\
		& = xy + xz
	\end{align*}
\end{proof}
\qed
\end{proof}

\begin{prop}
For any real $x$ we have $x1 = x$.
\end{prop}

\begin{proof}
\pf
\step{1}{\case{$0 \leq x$}}
\begin{proof}
	\step{a}{$x1 \subseteq x$}
	\begin{proof}
		\step{one}{\pflet{$q \in x1$}}
		\step{two}{\case{$q < 0$}}
		\begin{proof}
			\pf\ Then $q \in x$ because $0 \leq x$.
		\end{proof}
		\step{three}{$q = rs$ where $0 \leq r \in x$ and $0 \leq s < 1$}
		\begin{proof}
			\pf\ Then $q < r$ so $q \in x$.
		\end{proof}
	\end{proof}
	\step{b}{$x \subseteq x1$}
	\begin{proof}
		\step{one}{\pflet{$q \in x$}}
		\step{two}{\assume{w.l.o.g. $0 \leq q$}}
		\step{two}{\pick\ $r$ such that $q < r \in x$}
		\step{three}{$0 \leq q/r < 1$}
		\step{four}{$q = r(q/r) \in x1$}
	\end{proof}
\end{proof}
\step{2}{\case{$x < 0$}}
\begin{proof}
	\pf
	\begin{align*}
		x1 & = -((-x)1) \\
		& = -(-x) & (\text{\stepref{1}}) \\
		& = x
	\end{align*}
\end{proof}
\qed
\end{proof}

\begin{lm}
\label{lm:Rmultinv}
Let $x \in \mathbb{R}$ and $c$ be a positive rational. Then there exists $a \in x$ and a non-least rational upper bound $b$ for $x$ such that $b-a = c$.
\end{lm}

\begin{proof}
\pf
\step{1}{\pick\ $a_1 \in x$ such that if $x$ has a rational supremum $s$ then $a_1 > s - c$}
\step{2}{There exists a natural number $n$ such that $a_1 + nc$ is an upper bound for $x$.}
\begin{proof}
\step{2}{\pick\ a non-least upper bound $b_1$ for $x$.}
\step{3}{\pick\ a natural number $n$ such that $nc > b_1 - a_1$}
\begin{proof}
	\pf\ Proposition \ref{prop:Qarchimedean}.
\end{proof}
\step{4}{$a_1 + nc > b_1$}
\step{5}{$a_1 + nc$ is an upper bound for $x$.}
\end{proof}
\step{6}{\pflet{$k$ be the least natural number such that $a_1 + kc$ is an upper bound for $x$.}}
\step{7}{$a_1 + (k-1)c \in x$}
\step{8}{$a_1 + kc$ is not the supremum of $x$.}
\begin{proof}
	\step{a}{\assume{for a contradiction $a_1 + kc$ is the supremum of $x$.}}
	\step{b}{$a_1 > a_1 + (k-1)c$}
	\begin{proof}
		\pf\ \stepref{1}
	\end{proof}
	\qedstep
	\begin{proof}
		\pf\ This is a contradiction.
	\end{proof}
\end{proof}
\step{9}{\pflet{$a = a_1 + (k-1)c$}}
\step{10}{\pflet{$b = a_1 + kc$}}
\step{11}{$b - a = c$}
\qed
\end{proof}

\begin{prop}
For any non-zero real $x$, there exists a real $y$ such that $xy = 1$.
\end{prop}

\begin{proof}
\pf
\step{1}{\case{$0 < x$}}
\begin{proof}
	\step{a}{\pflet{$y = \{ q \in \mathbb{Q} \mid q \leq 0 \} \cup \{ u^{-1} \mid u \text{ is an upper bound for } x \text{ but not the supremum of } x \}$}}
	\step{b}{$y$ is a real number.}
	\begin{proof}
		\step{i}{$y \neq \emptyset$}
		\begin{proof}
			\pf\ Since $0 \in y$.
		\end{proof}
		\step{ii}{$y \neq \mathbb{Q}$}
		\begin{proof}
			\step{one}{\pick\ $q \in x$ such that $0 < q$}
			\step{two}{$q^{-1} \notin y$}
		\end{proof}
		\step{iii}{$y$ is closed downwards.}
		\begin{proof}
			\step{one}{\pflet{$q \in y$ and $r < q$} \prove{$r \in y$}}
			\step{two}{\assume{w.l.o.g. $0 < r$}}
			\step{three}{$q^{-1}$ is a non-least upper bound for $x$.}
			\step{four}{$q^{-1} < r^{-1}$}
			\step{five}{$r^{-1}$ is a non-least upper bound for $x$.}
			\step{six}{$r \in y$}
		\end{proof}
		\step{iv}{$y$ has no greatest element.}
		\begin{proof}
			\step{one}{\pflet{$q \in y$} \prove{There exists $r \in y$ such that $q < r$}}
			\step{two}{\case{$q \leq 0$}}
			\begin{proof}
				\step{A}{\pick\ a non-least upper bound $u$ for $x$.}
				\step{B}{$q < u^{-1} \in x$}
			\end{proof}
			\step{three}{\case{$q = u^{-1}$ where $u$ is a non-least upper bound for $x$.}}
			\begin{proof}
				\step{A}{\pick\ a non-least upper bound $v$ with $v < u$}
				\step{B}{$u^{-1} < v^{-1} \in y$}
			\end{proof}
		\end{proof}
	\end{proof}
	\step{c}{$0 < y$}
	\step{d}{$xy \subseteq 1$}
	\begin{proof}
		\step{i}{\pflet{$q \in xy$}}
		\step{ii}{\assume{w.l.o.g. $0 < q$}}
		\step{iii}{\pick\ $0 < r \in x$ and $0 < s \in y$ such that $q = rs$}
		\step{iv}{$s^{-1}$ is a non-least upper bound for $x$}
		\step{v}{$r < s^{-1}$}
		\step{vi}{$rs < 1$}
	\end{proof}
	\step{e}{$1 \subseteq xy$}
	\begin{proof}
		\step{i}{\pflet{$q < 1$} \prove{$q \in xy$}}
		\step{ii}{\assume{w.l.o.g. $0 < q$}}
		\step{iii}{\pick\ $a_1$ with $0 < a_1 \in x$}
		\step{iv}{$(1-q)a_1 > 0$}
		\step{v}{\pick\ $a \in x$ and a non-least upper bound $w$ of $x$ such that $w - a = (1-q)a_1$}
		\begin{proof}
			\pf\ Lemma \ref{lm:Rmultinv}.
		\end{proof}
		\step{vi}{$w - a < (1-q)w$}
		\step{vii}{$qw < a$}
		\step{viii}{$w < a / q$}
		\step{ix}{$a/q$ is a non-least upper bound for $x$}
		\step{xxx}{$q/a \in y$}
		\step{x}{$q \in xy$}
	\end{proof}
\end{proof}
\step{2}{\case{$x < 0$}}
\begin{proof}
	\step{a}{\pick\ $y$ such that $(-x)y = 1$}
	\begin{proof}
		\pf\ \stepref{1}
	\end{proof}
	\step{b}{$x(-y) = 1$}
\end{proof}
\qed
\end{proof}

\begin{prop}
For real numbers $x$, $y$ and $z$, if $0 < z$ then $x < y$ if and only if $xz < yz$.
\end{prop}

\begin{proof}
\pf
\step{1}{For any real numbers $x$, $y$ and $z$, if $0 < z$ and $x < y$ then $xz < yz$}
\begin{proof}
	\step{a}{\pflet{$x$, $y$ and $z$ be real numbers.}}
	\step{b}{\assume{$0 < z$ and $x < y$.}}
	\step{c}{$y = x + (y-x)$}
	\step{d}{$y-x > 0$}
	\step{e}{$(y-x)z > 0$}
	\step{f}{$yz > xz$}
	\begin{proof}
		\pf
		\begin{align*}
			yz & = (x+(y-x))z \\
			& = xz + (y-x)z \\
			& > xz
		\end{align*}
	\end{proof}
\end{proof}
\step{2}{For any real numbers $x$, $y$ and $z$, if $0 < z$ and $xz < yz$ then $x < y$}
\begin{proof}
	\pf\ Proposition \ref{prop:strictmonotoneinv}.
\end{proof}
\qed
\end{proof}

\begin{cor}
The real numbers form a complete ordered field.
\end{cor}

\begin{prop}
\[ (0,1) \approx \mathbb{R} \]
\end{prop}

\begin{proof}
\pf\ The function $f(x) = (2x-1)/(x-x^2)$ is a bijection between $(0,1)$ and $\mathbb{R}$. \qed
\end{proof}

%TODO Define expansions in base n
\begin{prop}
\[ |\mathbb{R}| = 2^{\aleph_0} \]
\end{prop}

\begin{proof}
\pf
\step{1}{$(0,1) \preccurlyeq 2^\mathbb{N}$}
\begin{proof}
	\pf\ The function $H$ where $H(x)(n)$ is the $n$th binary digit of the binary expansion of $x$ is an injection.
\end{proof}
\step{2}{$2^\mathbb{N} \preccurlyeq \mathbb{R}$}
\begin{proof}
	\pf\ Map $f$ to the real number in $[0, 1/9]$ whose $n+1$st decimal digit is $f(n)$.
\end{proof}
\qed
\end{proof}

%TODO Define algebraic numbers
\begin{prop}
The set of algebraic numbers is countable.
\end{prop}

\begin{proof}
\pf\ There are countably many integer polynomials, each with finitely many roots. \qed
\end{proof}

\begin{cor}
There are uncountably many transcendental numbers.
\end{cor}

%TODO Define circles in the plane
\begin{prop}
Let $A$ be a set of disks in the plane, no two of which intersect. Then $A$ is countable.
\end{prop}

\begin{proof}
\pf\ Every circle includes a point with rational coordinates. Define $f : \{ q \in \mathbb{Q}^2 \mid \exists C \in A. q \in C \} \rightarrow A$ by $f(q) = C$ iff $q \in C$. Then $f$ is surjective. \qed
\end{proof}

\begin{prop}
There exists an uncountable set of circles in the plane that do not intersect.
\end{prop}

\begin{proof}
\pf\ The set of all circles with origin O is uncountable. \qed
\end{proof}

\chapter{Real Analysis}

\begin{thm}[Weierstrass]
Let $a,b \in \mathbb{R}$ with $a < b$. Let $f : [a,b] \rightarrow \mathbb{R}$ be continuous. For every $\epsilon > 0$, there exists a polynomial $p$ such that $\forall x \in [a,b]. |f(x) - p(x)| < \epsilon$.
\end{thm}

%TODO Prove this

\begin{thm}[Bolzano-Weierstrass]
Every bounded sequence in $\mathbb{R}^n$ has a convergent subsequence.
\end{thm}
%TODO Prove this
%TODO Complex version

\section{Step Functions}

\begin{df}[Step Function]
A \emph{step function} on $\mathbb{R}$ is a function $f : \mathbb{R} \rightarrow \mathbb{R}$, where $[a_1,b_1)$, \ldots, $[a_n,b_n)$ are disjoint half-open intervals, such that $f$ is constant on each $[a_i,b_i)$, and zero outside them.
\end{df}

\begin{df}[Basic Representation]
Let $f$ be a step function. The \emph{basic representation} of $f$ is defined as follows.

Let $a_0$, $a_1$, \ldots, $a_n$ be the points of discontinuity of $f$. For $k = 1, \ldots, n$, let $\alpha_k = f(a_{k-1})$ and $g_k$ be the characteristic function of $[a_{k-1},a_k)$. Then the basic representation of $f$ is
\[ f = \alpha_1 g_1 + \cdots + \alpha_n g_n \enspace . \]
\end{df}

\begin{prop}
If $f$ and $g$ are step functions then $\lambda x.f(x) + g(x)$ is a step function.
\end{prop}

\begin{prop}
If $c \in \mathbb{R}$ and $f$ is a step function then $\lambda x. cf(x)$ is a step function.
\end{prop}

\begin{prop}
If $f$ is a step function then $\lambda x. |f(x)|$ is a step function.
\end{prop}

\begin{prop}
If $f$ and $g$ are step functions then $\lambda x. \min(f(x),g(x))$ and $\lambda x.\max(f(x),g(x))$ are step functions.
\end{prop}

\begin{prop}
If $f$ is a step function and $c \in \mathbb{R}$ then $\lambda x. f(x-c)$ is a step function.
\end{prop}

\begin{df}[Support]
Given $f : \mathbb{R} \rightarrow \mathbb{R}$, the \emph{support} of $f$ is
\[ \supp f := \{ x \in \mathbb{R} \mid f(x) \neq 0 \} \enspace . \]
\end{df}

\begin{df}[Integral of a Step Function]
Given a step function $f$, define $\int f = \int f(x) dx \in \mathbb{R}$ as follows.

Let $f(x) = \lambda_1 f_1(x) + \cdots + \lambda_n f_n(x)$, where $f_i$ is the characteristic function of $[a_i,b_i)$. Then
\[ \int f = \lambda_1 (b_1 - a_1) + \cdots + \lambda_n (b_n - a_n) \enspace . \]
We prove this is well defined.
\end{df}

\begin{proof}
\pf
\step{1}{\pflet{$f = \lambda_1 f_1(x) + \cdots + \lambda_m f_m(x) = \mu_1 g_1(x) + \cdots + \mu_n g_n(x)$, where $f_i$ is the characteristic function of $[a_i,b_i)$ and $g_i$ is the characteristic function of $[c_i,d_i)$, with $a_1 < b_1 \leq a_2 < b_2 \leq \cdots \leq a_m < b_m$ and $c_1 < d_1 \leq c_2 < d_2 \leq \cdots \leq c_n < d_n$.}}
\step{2}{\assume{w.l.o.g. none of the $\lambda_i$ or $\mu_i$ is zero.}}
\step{3}{\assume{w.l.o.g. we never have $\lambda_i = \lambda_{i+1}$ and $b_i = a_{i+1}$, and we never have $\mu_i = \mu_{i+1}$ and $d_i = c_{i+1}$.}}
\step{4}{We have $m = n$ and for all $i$, $a_i = b_i$ and $c_i = d_i$ and $\lambda_i = \mu_i$.}
\begin{proof}
	\step{a}{\assume{w.l.o.g. $m \leq n$}}
	\step{b}{\assume{as induction hypothesis $a_i = b_i$, $c_i = d_i$, $\lambda_i = \mu_i$ for $i = 1, \ldots, k$ with $k < m$.}}
	\step{c}{$a_{k+1} = b_{k+1}$}
	\begin{proof}
		\pf\ It is $\inf \{ x \in (a_k, + \infty) : f(x) \neq 0 \}$ (or $\inf \{x \in \mathbb{R} : f(x) \neq 0 \}$ if $k = 0$).
	\end{proof}
	\step{d}{$\lambda_{k+1} = \mu_{k+1}$}
	\begin{proof}
		\pf\ It is $f(a_{k+1})$.
	\end{proof}
	\step{e}{$c_{k+1} = d_{k+1}$}
	\begin{proof}
		\pf\ It is $\sup \{ x \in (a_{k+1}, + \infty) : f(x) = \lambda_{k+1} \}$.
	\end{proof}
	\step{f}{$m = n$}
	\begin{proof}
		\pf\ For all $x > b_m$ we have $f(x) = 0$.
	\end{proof}
\end{proof}
\qed
\end{proof}

\begin{prop}
If $f$ and $g$ are step functions then $\int (f(x) + g(x)) dx = \int f + \int g$.
\end{prop}

\begin{prop}
If $f$ is a step function then $\int cf(x)dx = c \int f$.
\end{prop}

\begin{prop}
If $f$ and $g$ are step functions and $\forall x. f(x) \leq g(x)$ then $\int f \leq \int g$.
\end{prop}

\begin{proof}
\pf We have $g(x) - f(x) \geq 0$ for all $x$ and so $\int (g(x) - f(x))dx \geq 0$. \qed
\end{proof}

\begin{prop}
If $f$ is a step function then $\left| \int f \right| \leq \int |f(x)|dx$.
\end{prop}

\begin{prop}
If $f$ is a step function and $c \in \mathbb{R}$ then $\int f(x-c)dx = \int f$.
\end{prop}

\begin{lm}
Let $f$ be a step function with $\supp f \subseteq [a_1, b_1) \cup \cdots \cup [a_n, b_n)$. Let $M$ be a constant. If $\forall x. |f(x)| < M$ then
\[ \int |f(x)| dx \leq M \sum_{k=1}^n (b_k - a_k) \enspace . \]
\end{lm}

\begin{lm}
\label{lm:sum_of_interval_lengths}
Let $\{ [a_i, b_i) \mid i \in \mathbb{N} \}$ be a partition of $[a,b)$. Then
\[ \sum_{i=0}^\infty (b_i - a_i) = b - a \enspace . \]
\end{lm}

\begin{proof}
\pf
\step{1}{For all $c \in (a,b]$ we have $\{ [a_i,b_i) \cap [a,c) \mid i \in \mathbb{N} \}$ is a partition of $[a,c)$.}
\step{2}{For $c \in (a,b]$ and $n \in \mathbb{N}$, \pflet{$b_{cn} := min(b_n,c)$}}
\step{3}{\pflet{$S = \left\{ c \in (a,b] \mid \sum_{a_n < b_{cn}} (b_{cn} - a_n) = c - a \right\}$}}
\step{4}{$S \neq \emptyset$}
\begin{proof}
	\step{a}{\pick\ $n$ such that $a_n = a$}
	\step{b}{$b_n \in S$}
\end{proof}
\step{5}{\pflet{$s := \sup S$}}
\step{6}{$s \in S$}
\begin{proof}
	\step{a}{\pick\ an increasing sequence $(s_n)$ in $S$ that converges to $S$.}
	\step{b}{For all $n$ we have $s_n - a \leq \sum_{a_m < b_{sm}} (b_{sm} - a_m) \leq s - a$.}
	\begin{proof}
		\pf
		\begin{align*}
			s_n - a & = \sum_{a_m < b_{s_nm}} (b_{s_nm} - a_m) & (s_n \in S) \\
			& \leq \sum_{a_m < b_{sm}} (b_{sm} - a_m) & (s_n \leq s) \\
			& \leq s - a & (\text{\stepref{1}})
		\end{align*}
	\end{proof}
	\step{c}{$\sum_{a_m < b_{sm}}(b_{sm} - a_m) = s - a$}
	\begin{proof}
		\pf\ Sandwich Theorem
	\end{proof}
\end{proof}
\step{7}{$s = b$}
\begin{proof}
	\step{a}{\assume{for a contradiction $s < b$}}
	\step{b}{\pick\ $k$ such that $s \in [a_k,b_k)$}
	\step{c}{$b_k \in S$}
	\begin{proof}
		\pf\ Since $\sum_{a_m < b_{sm}} (b_{sm} - a_m) = \sum_{a_m < b_{b_km}} (b_{b_km} - a_m)$.
	\end{proof}
	\step{d}{$b_k \leq s$}
	\qedstep
	\begin{proof}
		\pf\ This is a contradiction.
	\end{proof}
\end{proof}
\step{8}{$b \in S$}
\step{9}{$\sum_{a_n < b_n} (b_n - a_n) = b - a$}
\qed
\end{proof}

\begin{thm}
Let $(f_n)$ be a decreasing sequence of non-negative step functions such that, for all $x \in \mathbb{R}$, we have $f_n(x) \rightarrow 0$ as $n \rightarrow \infty$. Then $\int f_n \rightarrow 0$ as $n \rightarrow \infty$.
\end{thm}

\begin{proof}
\pf
\step{1}{$\left( \int f_n \right)$ is decreasing and bounded below by 0.}
\step{2}{\pflet{$\epsilon := \lim_{n \rightarrow 0} \int f_n$}}
\step{3}{\assume{for a contradiction $\epsilon > 0$.}}
\step{4}{\pick\ $a,b \in \mathbb{R}$ such that $\supp f_0 \subseteq [a,b)$}
\step{5}{\pflet{$\alpha := \epsilon / 2(b-a)$}}
\step{6}{For $n \in \mathbb{N}$, \pflet{
\[ A_n := \{ x \in [a,b) : f_n(x) < \alpha \} \enspace . \]}}
\step{7}{\pflet{$B_0 := A_0$}}
\step{8}{For $n$ a positive integer, \pflet{$B_n = A_n - A_{n-1}$.}}
\step{9}{For all $n$ we have $A_n \subseteq A_{n+1}$.}
\step{10}{For $m \neq n$ we have $B_m \cap B_n = \emptyset$.}
\step{11}{$\bigcup_{n=0}^\infty A_n = [a,b)$}
\begin{proof}
	\pf\ For all $x \in [a,b)$, there exists $N$ such that $f_N(x) < \alpha$ because $f_n(x) \rightarrow 0$ as $n \rightarrow \infty$.
\end{proof}
\step{12}{$\bigcup_{n=0}^\infty B_n = [a,b)$}
\step{13}{For $n \in \mathbb{N}$, \pflet{$B_n = [a_{n1},b_{n1}) \cup \cdots \cup [a_{nk_n},b_{nk_n})$.}}
\step{14}{$\sum_{n=0}^\infty 	\sum_{k=1}^{k_n} (b_{nk} - a_{nk}) = b - a$}
\begin{proof}
	\pf\ Lemma \ref{lm:sum_of_interval_lengths}.
\end{proof}
\step{15}{\pflet{
\[ \delta = \frac{\epsilon}{2 \max_x |f_0(x)|} \enspace . \]}}
\step{15}{\pick\ $n_0$ such that
\[ \sum_{n=n_0}^\infty \sum_{k=1}^{k_n} (b_{nk} - a_{nk}) < \delta \enspace . \]}
\step{16}{$A_{n_0} = B_0 \cup \cdots \cup B_{n_0}$}
\step{17}{\pflet{$g : \mathbb{R} \rightarrow \mathbb{R}$,
\[ g(x) = \begin{cases}
f_{n_0}(x) & \text{if } x \in A_{n_0} \\
0 & \text{if } x \notin A_{n_0}
\end{cases} \]}}
\step{18}{\pflet{$h : \mathbb{R} \rightarrow \mathbb{R}$,
\[ h(x) = \begin{cases}
0 & \text{if } x \in A_{n_0} \\
f_{n_0}(x) & \text{if } x \notin A_{n_0}
\end{cases} \]}}
\step{19}{$\forall x \in B. f_{n_0}(x) < \alpha$}
\step{20}{$\forall x \in \mathbb{R}. g(x) < \alpha$}
\step{21}{\[ \int g < \frac{\epsilon}{2} \]}
\step{22}{\[ \int h < \frac{\epsilon}{2} \]}
\begin{proof}
	\pf
	\begin{align*}
		\int h & < \delta \max_x |f_{n_0}(x)| \\
		& \leq \delta \max_x |f_0(x)| \\
		& = \epsilon / 2
	\end{align*}
\end{proof}
\step{23}{\[ \int f_{n_0} < \epsilon \]}
\begin{proof}
	\pf\ Since $\forall x. f_{n_0}(x) = g(x) + h(x)$.
\end{proof}
\step{24}{\[ \lim_{n \rightarrow \infty} \int f_n < \epsilon \]}
\qedstep
\begin{proof}
	\pf\ This contradicts \stepref{2}.
\end{proof}
\qed
\end{proof}

\begin{cor}
Let $(f_n)$ be an increasing sequence of step functions. If $\lim_{n \rightarrow \infty} f_n(x) \geq 0$ for all $x$, then $\lim_{n \rightarrow \infty} \int f_n \geq 0$.
\end{cor}

\begin{proof}
\pf
\step{1}{For $n \in \mathbb{N}$, \pflet{$g_n : \mathbb{R} \rightarrow \mathbb{R}$ be the function $g_n(x) = \max(0, -f_n(x))$.}}
\step{2}{$(g_n)$ is a decreasing sequence of step functions and $g_n(x) \rightarrow 0$ as $n \rightarrow \infty$ for all $x$.}
\step{3}{$\int g_n \rightarrow 0$ as $n \rightarrow \infty$}
\step{4}{For all $n \in \mathbb{N}$ we have $f_n(x) = \max(0,f_n(x)) - \max(0,-f_n(x))$.}
\step{5}{\[ \int f_n = \int \max(0, f_n(x)) dx - \int \max(0, -f_n(x)) dx \]}
\step{6}{\[ \lim_{n \rightarrow \infty} \int f_n = \lim_{n \rightarrow \infty} \int \max(0, f_n(x)) dx \]}
\step{7}{$\lim_{n \rightarrow \infty} \int f_n \geq 0$}
\qed
\end{proof}

\section{Lebesgue Integration}

\begin{df}[Lebesgue Integration]
Let $f : \mathbb{R} \rightarrow \mathbb{R}$. Then $f$ is \emph{Lebesgue integrable} iff there exists a sequence $(f_n)$ of step functions such that:
\begin{itemize}
\item
\[ \sum_{n=0}^\infty \int |f_n(x)|dx < \infty \]
\item
For all $x \in \mathbb{R}$, if $\sum_{n=0}^\infty |f_n(x)| < \infty$, then $f(x) = \sum_{n=0}^\infty f_n(x)$.
\end{itemize}
We write $f \simeq \sum_{n=0}^\infty f_n$ for these two conditions.

The \emph{integral} of $f$ is then
\[ \int f := \sum_{n=0}^\infty \int f_n \enspace . \]
\end{df}

\chapter{Complex Analysis}

\begin{thm}[H\"{o}lder's Inequality]
Let $p$ and $q$ be real numbers with $p > 1$, $q > 1$ and $1/p + 1/q = 1$. If $(x_n) \in l^p$ and $(y_n) \in l^q$ then
\[ \sum_{n=0}^\infty |x_n y_n| \leq \left( \sum_{n=0}^\infty  |x_n|^p \right)^{1/p} \left( \sum_{n=0}^\infty |y_n|^q \right)^{1/q} \]
\end{thm}

\begin{proof}
\pf
\step{1}{\pflet{$p$ and $q$ be real numbers with $p > 1$ and $q > 1$}}
\step{2}{\assume{$1/p + 1/q = 1$}}
\step{3}{\pflet{$(x_n) \in l^p$}}
\step{4}{\pflet{$(y_n) \in l^q$}}
\step{5}{\assume{w.l.o.g. $x_0 \neq 0$ and $y_0 \neq 0$}}
\step{6}{For all $x \in [0,1]$, we have
\[ x^{1/p} \leq \frac{1}{p} x + \frac{1}{q} \enspace . \]}
\begin{proof}
	\pf
	\step{a}{\pflet{$f : [0,1] \rightarrow \mathbb{R}$ be the function
	\[ f(x) = \frac{1}{p} x + \frac{1}{q} - x^{1/p} \enspace . \]}}
	\step{b}{$f'(x) = \frac{1}{p} - \frac{1}{p} x^{-1/q}$ for $x \in (0,1]$}
	\step{c}{$f'(x) < 0$ for $x \in (0,1]$}
	\step{d}{$f(1) = 1/p + 1/q - 1 = 0$}
	\step{e}{$f(x) \geq 0$ for all $x \in [0,1]$}
\end{proof}
\step{7}{For all non-negative reals $a$ and $b$, we have
\[ ab \leq \frac{a^p}{p} + \frac{b^q}{q} \enspace . \]}
\begin{proof}
	\step{a}{\pflet{$a$ and $b$ be non-negative reals.}}
	\step{b}{\case{$a^p \leq b^q$}}
	\begin{proof}
		\step{i}{$0 \leq a^p / b^q \leq 1$}
		\step{ii}{\[ ab^{-q/p} \leq \frac{1}{p} \frac{a^p}{b^q} + \frac{1}{q} \]}
		\begin{proof}
			\pf\ Taking $x = a^p / b^q$ in \stepref{6}.
		\end{proof}
		\step{iii}{\[ ab^{1-q} \leq \frac{1}{p} \frac{a^p}{b^q} + \frac{1}{q} \]}
		\begin{proof}
			\pf\ $-q/p = 1 - q$ from \stepref{2}.
		\end{proof}
		\step{iv}{\[ ab \leq \frac{a^p}{p} + \frac{b^q}{q} \]}
	\end{proof}
	\step{c}{\case{$b^q \leq a^p$}}
	\begin{proof}
		\pf\ Similar.
	\end{proof}
\end{proof}
\step{8}{For $j = 1, \ldots, n$, we have
\[ \frac{|x_j|}{\left( \sum_{k=0}^n |x_k|^p \right)^{1/p}}
\frac{|y_j|}{\left( \sum_{k=0}^n |y_k|^q \right)^{1/q}}
\leq
\frac{1}{p} \frac{|x_j|^p}{\sum_{k=0}^n |x_k|^p} + \frac{1}{q} \frac{|y_j|^q}{\sum_{k=0}^n |y_k|^q} \]}
\begin{proof}
	\pf\ From \stepref{7} with
	\[ a = \frac{|x_j|}{\left( \sum_{k=0}^n |x_k|^p \right)^{1/p}} \text{ and } b = \frac{|y_j|}{\left( \sum_{k=0}^n |y_k|^q \right)^{1/q}} \enspace . \]
\end{proof}
\step{9}{
\[ \frac{\sum_{j=0}^n |x_j||y_j|}{\left( \sum_{k=0}^n |x_k|^p \right)^{1/p} \left( \sum_{k=0}^n |y_k|^q \right)^{1/q}}
\leq 1 \]}
\begin{proof}
	\pf
	\begin{align*}
	\frac{\sum_{j=0}^n |x_j||y_j|}{\left( \sum_{k=0}^n |x_k|^p \right)^{1/p} \left( \sum_{k=0}^n |y_k|^q \right)^{1/q}}
	& \leq \frac{1}{p} + \frac{1}{q} & (\text{Taking the sum $j = 0$ to $n$ in \stepref{8}}) \\
	& = 1 & (\text{\stepref{2}})
	\end{align*}
\end{proof}
\qedstep
\begin{proof}
	\pf\ Taking the limit $n \rightarrow \infty$ in \stepref{9}.
\end{proof}
\qed
\end{proof}

\begin{thm}[Minkowski's Inequality]
Let $p$ be a real number, $p \geq 1$. Let $(x_n), (y_n) \in l^p$. Then
\[ \left( \sum_{n=1}^\infty |x_n + y_n|^p \right)^{1/p} \leq \left( \sum_{n=1}^\infty |x_n|^p \right)^{1/p} + \left( \sum_{n=1}^\infty |y_n|^p \right)^{1/p} \]
\end{thm}
%TODO Prove this

\begin{proof}
\pf
\step{1}{\pflet{$p$ be a real number with $p \geq 1$}}
\step{2}{\assume{w.l.o.g. $p > 1$}}
\begin{proof}
	\pf\ The case $p=1$ is just the Triangle Inequality.
\end{proof}
\step{3}{\pflet{$q$ be the real such that $1/p + 1/q = 1$}}
\step{4}{\begin{align*}
\sum_{n=0}^\infty |x_n + y_n|^p & \leq \left( \sum_{n=0}^\infty |x_n|^p \right)^{1/p} \left( \sum_{n=0}^\infty |x_n + y_n|^{q(p-1)} \right)^{1/q} \\ & + \left( \sum_{n=0}^\infty |y_n|^p \right)^{1/p} \left( \sum_{n=0}^\infty |x_n + y_n|^{q(p-1)} \right)^{1/q} \end{align*}}
\begin{proof}
	\pf
	\begin{align*}
		\sum_{n=0}^\infty |x_n + y_n|^p
		& = \sum_{n=0}^\infty |x_n + y_n| |x_n + y_n|^{p-1} \\
		& \leq \sum_{n=0}^\infty |x_n| |x_n + y_n|^{p-1} + 
		\sum_{n=0}^\infty |y_n| |x_n + y_n|^{p-1} & (\text{Triangle Inequality}) \\
		& \leq \left( \sum_{n=0}^\infty |x_n|^p \right)^{1/p} \left( \sum_{n=0}^\infty |x_n + y_n|^{q(p-1)} \right)^{1/q} \\
		& + \left( \sum_{n=0}^\infty |y_n|^p \right)^{1/p} \left( \sum_{n=0}^\infty |x_n + y_n|^{q(p-1)} \right)^{1/q} & (\text{H\"{o}lder's Inequality})
	\end{align*}
	%TODO Fallacy: We do not know that $((x_n + y_n)^{p-1}) \in l^q$
\end{proof}
\step{5}{\[ \sum_{n=0}^\infty |x_n + y_n|^p \leq \left\{ \left( \sum_{n=0}^\infty |x_n|^p \right)^{1/p} + \left( \sum_{n=0}^\infty |y_n|^p \right)^{1/p} \right\} \left( \sum_{n=0}^\infty |x_n + y_n|^{p} \right)^{1/q}\]}
\qedstep
\qed
\end{proof}

\chapter{Topology}

\section{Topological Spaces}

\begin{df}[Topology]
Let $X$ be a set. A \emph{topology} on $X$ is a set $\mathcal{T} \subseteq \mathcal{P} X$, whose elements are called \emph{open sets}, such that:
\begin{itemize}
\item $X \in \mathcal{T}$
\item $\forall \mathcal{U} \subseteq \mathcal{T}. \bigcup \mathcal{U} \in \mathcal{T}$
\item $\forall U,V \in \mathcal{T}. U \cap V \in \mathcal{T}$
\end{itemize}
A \emph{topological space} is a pair $(X, \mathcal{T})$ such that $X$ is a set and $\mathcal{T}$ is a topology on $X$. We refer to the elements of $X$ as \emph{points}.

An \emph{open neighbourhood} of a point $x$ is an open set $U$ such that $x \in U$. We write $\mathcal{T}_x$ for the set of all open neighbourhoods of $x$.
\end{df}

\begin{df}[Closed Set]
In a topological space $X$, a set $C$ is \emph{closed} iff $X - C$ is open.
\end{df}

\begin{df}[Discrete Topology]
The \emph{discrete topology} on a set $X$ is $\mathcal{P} X$.
\end{df}

\begin{df}[Indiscrete Topology]
The \emph{indiscrete topology} or \emph{trivial topology} on a set $X$ is $\{ \emptyset, X \}$.
\end{df}

\begin{df}[Finer, Coarser]
Let $\mathcal{T}$ and $\mathcal{T}'$ be topologies on the same set $X$. Then $\mathcal{T}$ is \emph{finer}, \emph{larger} or \emph{stronger} than $\mathcal{T}'$, and $\mathcal{T}'$ is \emph{coarser}, \emph{smaller} or \emph{weaker} than $\mathcal{T}$, iff $\mathcal{T}' \subseteq \mathcal{T}$.
\end{df}

\begin{df}[Basis]
Let $X$ be a set. A \emph{basis} for a topology on $X$ is a set $\mathcal{B} \subseteq \mathcal{P} X$, whose elements we call \emph{basic open neighbourhoods}, such that:
\begin{itemize}
\item $\bigcup \mathcal{B} = X$
\item $\forall A,B \in \mathcal{B}. \forall x \in A \cap B. \exists C \in \mathcal{B}. x \in C \subseteq A \cap B$.
\end{itemize}
The topology \emph{generated} by $\mathcal{B}$ is the coarsest topology that includes $\mathcal{B}$.
\end{df}

\begin{prop}
The topology generated by $\mathcal{B}$ is $\{ U \in \mathcal{P} X \mid \forall x \in U. \exists B \in \mathcal{B}. x \in B \subseteq U \}$
\end{prop}

\section{Continuous Functions}

\begin{df}[Continuous]
Let $X$ and $Y$ be topological spaces. Let $f : X \rightarrow Y$. Then $f$ is \emph{continuous} iff, for every open set $U$ in $Y$, the set $f^{-1}(U)$ is open in $X$.
\end{df}

\begin{prop}
For any topological space $X$, the identity function $\id{X} : X \rightarrow X$ is continuous.
\end{prop}

\begin{prop}
If $f : X \rightarrow Y$ and $g : Y \rightarrow Z$ are continuous then $g \circ f : X \rightarrow Z$ is continuous.
\end{prop}

\begin{df}
Let $\mathrm{Top}$ be the category of topological spaces and continuous functions.
\end{df}

\begin{prop}
Let $X$ and $Y$ be topological spaces and $f : X \rightarrow Y$. Then $f$ is continuous if and only if, for every closed set $C$ in $Y$, we have $f^{-1}(C)$ is closed in $X$.
\end{prop}

\begin{df}[Continuous at a Point]
Let $X$ and $Y$ be topological spaces. Let $f : X \rightarrow Y$. Let $x \in X$. Then $f$ is \emph{continuous at $x$} iff, for every open neighbourhood $V$ of $f(x)$, we have $f^{-1}(V)$ is open.
\end{df}

\begin{df}
The category of \emph{pointed} topological spaces, $\mathrm{Top}_*$, is the category with:
\begin{itemize}
\item objects all pairs $(A,a)$ where $A$ is a topological space and $a \in A$;
\item morphisms $f : (A,a) \rightarrow (B,b)$ all continuous functions $f : A \rightarrow B$ such that $f(a) = b$.
\end{itemize}
\end{df}

\begin{df}[Homeomorphism]
A \emph{homeomorphism} is an isomorphism in $\mathrm{Top}$. Two isomorphic topological spaces are called \emph{homeomorphic}.
\end{df}

\begin{df}[Topological Property]
A property of topological spaces is a \emph{topological} property iff it is preserved by homeomorphism.
\end{df}

\begin{prop}
Cardinality of a topological space is a topological property.
\end{prop}

\section{Convergence}

\begin{df}[Convergence]
Let $X$ be a topological space. Let $(x_n)$ be a sequence in $X$ and $l \in X$. Then $(x_n)$ \emph{converges} to $l$, $x_n \rightarrow l$ as $n \rightarrow \infty$, if and only if, for every open neighbourhood $U$ of $l$, there exists $N$ such that $\forall n \geq N. x_n \in U$.
\end{df}

%TODO Check hypotheses
\begin{thm}
Let $X$ and $Y$ be topological spaces. Let $Z$ be a closed subspace of $X$ and $f : Z \rightarrow Y$ a continuous function. Then the graph of $f$, $G = \{ (x,f(x)) \mid x \in Z \}$, is closed in $X \times Y$.
\end{thm}

\begin{proof}
\pf
\step{1}{\pflet{$((x_n, f(x_n)))$ be a sequence in $G$.}}
\step{2}{\pflet{$(x_n, T(x_n)) \rightarrow (x,y)$ as $n \rightarrow \infty$}}
\step{3}{$x \in W$}
\begin{proof}
	\pf\ Since $x_n \rightarrow x$ and $W$ is closed.
\end{proof}
\step{4}{$y = f(x)$}
\begin{proof}
	\pf
	\begin{align*}
		y & = \lim_{n \rightarrow \infty} f(x_n) \\
		& = f \left( \lim_{n \rightarrow \infty} x_n \right) \\
		& = f(x)
	\end{align*}
\end{proof}
\qed
\end{proof}

\section{Homotopy}

%TODO Define homotopy
\begin{df}
Let $\mathrm{hTop}$ be the category whose objects are topological spaces, and whose morphisms are homotopy classes of continuous functions.
\end{df}

\begin{df}
A \emph{homotopy equivalence} is an isomorphism in $\mathrm{hTop}$. Isomorphic topological spaces are called \emph{homotopic}.
\end{df}

\section{Metric Spaces}

\begin{df}[Metric]
Let $X$ be a set. A \emph{metric} on a set $X$ is a function $d : X^2 \rightarrow \mathbb{R}$ such that:
\begin{itemize}
\item $\forall x,y \in X. d(x,y) \geq 0$
\item $\forall x,y \in X. d(x,y) = d(y,x)$
\item \emph{Triangle Inequality} $\forall x,y,z \in X. d(x,y) + d(y,z) \geq d(x,z)$
\item $\forall x,y \in X. d(x,y) = 0$ iff $x = y$.
\end{itemize}
A \emph{metric space} is a pair $(X,d)$ such that $d$ is a metric on $X$.
\end{df}

\begin{df}[Open Ball]
In a metric space $X$, let $c \in X$ and $r > 0$. The \emph{open ball} with \emph{centre} $c$ and \emph{radius} $r$ is
\[ B(c,r) := \{ x \in X \mid d(x,c) < r \} \enspace . \]
\end{df}

\begin{prop}
In a metric space, the set of open balls forms a basis for a topology.
\end{prop}

\begin{df}[Metric Topology]
Given a metric space $X$, the \emph{metric topology} on $X$ is the topology generated by the basis of open balls.

A topological space $(X, \mathcal{T})$ is \emph{metrizable} iff there exists a metric $d$ on $X$ such that $\mathcal{T}$ is the metric topology induced by $d$.
\end{df}

We identify a metric space with this topological space.

\begin{prop}
If $d$ is a metric on $X$ and $Y \subseteq X$ then $d \restriction Y^2$ is a metric on $Y$.
\end{prop}

We write just $Y$ for the metric space $(Y, d \restriction Y^2)$.

\begin{prop}
Let $X$ and $Y$ be metric spaces. Let $f : X \rightarrow Y$. Then $f$ is continuous if and only if, for every sequence $(x_n)$ in $X$ and $l \in X$, if $x_n \rightarrow l$ as $n \rightarrow \infty$ then $f(x_n) \rightarrow f(l)$ as $n \rightarrow \infty$.
\end{prop}

\begin{proof}
\pf
\step{1}{If $f$ is continuous then, for every sequence $(x_n)$ in $X$ and $l \in X$, if $x_n \rightarrow l$ as $n \rightarrow \infty$, then $f(x_n) \rightarrow f(l)$ as $n \rightarrow \infty$.}
\begin{proof}
	\step{a}{\assume{$f$ is continuous.}}
	\step{b}{\pflet{$(x_n)$ be a sequence in $X$.}}
	\step{c}{\pflet{$l \in X$}}
	\step{d}{\assume{$x_n \rightarrow l$ as $n \rightarrow \infty$.}}
	\step{e}{\pflet{$V$ be an open neighbourhood of $f(l)$}}
	\step{f}{$f^{-1}(V)$ is an open neighbourhood of $l$.}
	\step{g}{\pick\ $N$ such that $\forall n \geq N. x_n \in f^{-1}(V)$}
	\step{h}{$\forall n \geq N. f(x_n) \in V$}
\end{proof}
\step{2}{If, for every sequence $(x_n)$ in $X$ and $l \in X$, if $x_n \rightarrow l$ as $n \rightarrow \infty$, then $f(x_n) \rightarrow f(l)$ as $n \rightarrow \infty$, then $f$ is continuous.}
\begin{proof}
	\step{b}{\assume{$f$ is not continuous.}}
	\step{c}{\pick\ an open set $V$ in $Y$ such that $f^{-1}(V)$ is not open in $X$}
	\step{d}{\pick\ $l \in f^{-1}(V)$ such that, for all $\epsilon > 0$, $B(l,\epsilon) \nsubseteq f^{-1}(V)$.}
	\step{e}{For $n \in \mathbb{N}$, \pick\ $x_n \in B(l,1/(n+1))$ such that $x_n \notin f^{-1}(V)$.}
	\step{f}{$x_n \rightarrow l$ as $n \rightarrow \infty$.}
	\step{g}{$f(x_n) \not\rightarrow f(l)$ as $n \rightarrow \infty$}
\end{proof}
\qed
\end{proof}

\begin{prop}
Completeness is not a topological property.
\end{prop}

\begin{proof}
\pf\ We have $(-1,1) \cong \mathbb{R}$, but $\mathbb{R}$ is complete and $(-1,1)$ is not. \qed
\end{proof}

\chapter{Ring Theory}

%Define these
\begin{df}
Given a ring $R$, let $R-\mathrm{Mod}$ be the category of modules over $R$ and $R$-linear maps.
\end{df}

\chapter{Linear Algebra}

\section{Vector Spaces}

\begin{df}[Vector Space]
Let $K$ be a field. A \emph{vector space} over $K$ consists of:
\begin{itemize}
\item a set $V$, whose elements are called \emph{vectors};
\item an operation $+ : V^2 \rightarrow V$, \emph{addition};
\item an operation $\cdot : K \times V \rightarrow V$, \emph{scalar multiplication}
\end{itemize}
such that:
\begin{itemize}
\item $V$ is an Abelian group under $+$
\item $\forall \alpha, \beta \in K. \forall x \in V. \alpha (\beta x) = (\alpha \beta) x$
\item $\forall \alpha, \beta \in K. \forall x \in V. (\alpha + \beta) x = \alpha x + \beta x$
\item $\forall \alpha \in K. \forall x,y \in V. \alpha (x + y) = \alpha x + \alpha y$
\item $\forall x \in V. 1x = x$
\end{itemize}

We call the elements of $K$ \emph{scalars}. A \emph{real vector space} is a vector space over $\mathbb{R}$, and a \emph{complex vector space} is a vector space over $\mathbb{C}$.
\end{df}

\begin{prop}
Let $K$ be a field. Let $V$ be a vector space over $K$. For any $\lambda \in K$ we have $\lambda 0 = 0$.
\end{prop}

\begin{proof}
\pf\
\begin{align*}
\lambda 0 & = \lambda (0 + 0) \\
& = \lambda 0 + \lambda 0 \\
\therefore 0 & = \lambda 0 & \qed
\end{align*}
\end{proof}

\begin{prop}
Let $K$ be a field. Let $V$ be a vector space over $K$. Let $\lambda \in K$ and $x \in V$. If $\lambda x = 0$ then either $\lambda = 0$ or $x = 0$.
\end{prop}

\begin{proof}
\pf\ If $\lambda \neq 0$ then $x = 1x = \lambda^{-1} \lambda x = \lambda^{-1} 0 = 0$. \qed
\end{proof}

\begin{prop}
\label{prop:zerotimes}
Let $K$ be a field. Let $V$ be a vector space over $K$. For any $x \in V$ we have $0x = 0$.
\end{prop}

\begin{proof}
\pf
\begin{align*}
0x & = (0 + 0) x \\
& = 0x + 0x \\
\therefore 0 & = 0x & \qed
\end{align*}
\end{proof}

\begin{prop}
Let $K$ be a field. Let $V$ be a vector space over $K$. For any $x \in V$, we have $(-1)x = -x$.
\end{prop}

\begin{proof}
\pf
\begin{align*}
x + (-1)x & = 1x + (-1) x \\
& = (1 + (-1))x \\
& = 0 x \\
& = 0 \\
\therefore (-1)x & = -x & \qed
\end{align*}
\end{proof}

\begin{prop}
Let $K$ be a field. Then $K$ is a vector space over $K$ under addition and multiplication in $K$.
\end{prop}

\begin{proof}
\pf\ Easy. \qed
\end{proof}

\begin{prop}
$\mathbb{C}$ is a vector space over $\mathbb{R}$.
\end{prop}

\begin{proof}
\pf\ Easy. \qed
\end{proof}

\begin{prop}
Let $K$ be a field. Let $\{ V_i \}_{i \in I}$ be a family of vector spaces over $K$. Then $\prod_{i \in I} V_i$ is a vector space under
\begin{align*}
(f + g)(i) & = f(i) + g(i) & (f,g \in \prod_{i \in I} V_i, x \in X) \\
(\lambda f)(x) & = \lambda f(x) & (\lambda \in K, f \in \prod_{i \in I} V_i, x \in X)
\end{align*}
\end{prop}

\begin{proof}
\pf\ Easy. \qed
\end{proof}

\section{Subspaces}

\begin{df}[Vector Subspace]
Let $K$ be a field. Let $V$ be a vector space over $K$. A \emph{vector subspace} of $V$ is a subset $U \subseteq V$ such that, for all $\alpha, \beta \in K$ and $x,y \in U$, we have $\alpha x + \beta y \in U$.

It is a \emph{proper} subspace iff $U \neq V$.
\end{df}

\begin{prop}
If $U$ is a subspace of $V$ then $U$ is a vector space under the restrictions of $+$ and $\cdot$ to $U$.
\end{prop}

\begin{proof}
\pf\ Easy. \qed
\end{proof}

\begin{prop}
$V$ is a subspace of $V$.
\end{prop}

\begin{proof}
\pf\ Easy. \qed
\end{proof}

\begin{prop}
If $U$ is a subspace of $V$ and $V$ is a subspace of $W$ then $U$ is a subspace of $W$.
\end{prop}

\begin{proof}
\pf\ Easy. \qed
\end{proof}

\begin{df}
Let $\Omega$ be a topological space. Then $\mathcal{C}(\Omega)$ is the complex vector space of all continuous functions from $\Omega$ to $\mathbb{C}$. This is a subspace of $\mathbb{C}^\Omega$.
\end{df}

%TODO Define partial derivative
\begin{df}
Let $n, k \in \mathbb{N}$.
Let $\Omega$ be an open subset of $\mathbb{R}^n$. Then $\mathcal{C}^k(\Omega)$ is the complex vector space of all functions $\Omega \rightarrow \mathbb{C}$ that have all continuous partial derivatives of order $k$. This is a subspace of $\mathcal{C}(\Omega)$. If $l > k$ then $\mathcal{C}^l(\Omega)$ is a subpase of $\mathcal{C}^k(\Omega)$.
\end{df}

\begin{df}
Let $n \in \mathbb{N}$. Let $\Omega$ be an open subset of $\mathbb{R}^n$. Then $\mathcal{C}^\infty(\Omega)$ is the complex vector space of all infinitely differentiable functions $\Omega \rightarrow \mathbb{C}$. This is a subspace of $\mathcal{C}^k(\Omega)$ for all $k$.
\end{df}

\begin{df}
Let $n \in \mathbb{N}$. Let $\Omega$ be an open subset of $\mathbb{R}^n$. Then $\mathcal{P}(\Omega)$ is the complex vector space of all complex polynomials of $n$ variables, considered as functions $\Omega \rightarrow \mathbb{C}$. This is a subspace of $\mathcal{C}^\infty(\Omega)$.
\end{df}

\begin{prop}
The space of all convergent sequences in $\mathbb{C}$ is a subspace of the space of all bounded sequences in $\mathbb{C}$, which is a subspace of $\mathbb{C}^\mathbb{N}$.
\end{prop}

\begin{proof}
\pf\ Easy. \qed
\end{proof}

%TODO Define sum of a series
\begin{df}
Let $p$ be a real number, $p \geq 1$. Let $l^p$ be the set of all complex sequences $(z_n)$ such that $\sum_{n=1}^\infty |z_n|^p < \infty$.
\end{df}

\begin{prop}
For $p$ a real number $\geq 1$, we have that $l^p$ is a subspace of $\mathbb{C}^\mathbb{N}$.
\end{prop}

\begin{proof}
\pf
\step{1}{For all $(x_n),(y_n) \in l^p$, we have $(x_n + y_n) \in l^p$.}
\begin{proof}
	\pf\ From Minkowski's Inequality.
\end{proof}
\step{2}{For all $\lambda \in \mathbb{C}$ and $(x_n) \in l^p$ we have $(\lambda x_n) \in l^p$}
\begin{proof}
	\pf
	\[ \sum_{n=1}^\infty |\lambda x_n|^p = |\lambda|^p \sum_{n=1}^\infty |x_n|^p < \infty \]
\end{proof}
\qed
\end{proof}

\begin{df}[Linear Combination]
Let $K$ be a field. Let $V$ be a vector space over $K$. Let $x, x_1, \ldots, x_n \in V$. Then $x$ is a \emph{linear combination} of $x_1$, \ldots, $x_n$ iff there exist $\alpha_1, \ldots, \alpha_n \in K$ such that
\[ x = \alpha_1 x_1 + \cdots + \alpha_n x_n \enspace . \]
\end{df}

\begin{df}[Linearly Independent]
A finite set of vectors $\{x_1, \ldots, x_n\}$ is \emph{linearly independent} iff, whenever $\alpha_1 x_1 + \cdots + \alpha_n x_n = 0$, then $\alpha_1 = \cdots = \alpha_n = 0$.

A set of vectors is \emph{linearly independent} iff every finite subset is linearly independent; otherwise, it is \emph{linearly dependent}.
\end{df}

\begin{df}[Span]
Let $\mathcal{A}$ be a set of vectors. The \emph{span} of $\mathcal{A}$, $\spn \mathcal{A}$, is the set of all linear combinations of elements of $\mathcal{A}$.
\end{df}

\begin{prop}
$\spn \mathcal{A}$ is the smallest subspace of $V$ that includes $\mathcal{A}$.
\end{prop}

\begin{proof}
\pf\ Easy. \qed
\end{proof}

\begin{df}[Basis]
A \emph{basis} for $V$ is a linearly independent set of vectors $\mathcal{B}$ such that $\spn \mathcal{B} = V$.
\end{df}

\begin{df}[Finite Dimensional]
A vector space is \emph{finite dimensional} iff it has a finite basis; otherwise it is \emph{infinite dimensional}.
\end{df}

\begin{prop}
In a finite dimensional vector space, any two bases have the same number of elements.
\end{prop}

%TODO Prove this

\begin{df}[Dimension]
The \emph{dimension} of a finite dimensional vector space $V$, $\dim V$, is the number of elements in any basis.
\end{df}

\begin{prop}
\[ \dim K^n = n \]
\end{prop}

\begin{proof}
\pf\ The standard basis is the set of vectors with one coordinate 1 and all others 0. \qed
\end{proof}

\begin{prop}
The dimension of $\mathbb{C}^n$ as a real vector space is $2n$.
\end{prop}

\begin{prop}
The set of all step functions is a subspace of $\mathbb{R}^\mathbb{R}$.
\end{prop}

\section{Linear Transformations}

\begin{df}[Linear Transformation]
Let $K$ be a field.
Let $U$ and $V$ be vector spaces over $K$. Let $T : U \rightarrow V$. Then $T$ is a \emph{linear transformation} iff
\[ \forall \alpha, \beta \in K. \forall x,y \in U. T(\alpha x + \beta y) = \alpha T(x) + \beta T(y) \enspace . \]

Let $\mathrm{Vect}_K$ be the category of vector spaces over $K$ and linear transformations.
\end{df}

\begin{prop}
If $T : U \rightarrow V$ is a linear transformation then $T(U)$ is a subspace of $V$.
\end{prop}

\begin{prop}
If $T : U \rightarrow V$ is a linear transformation then the graph of $T$, $\{ (x, T(x)) \mid x \in U \}$, is a subspace of $U \times V$.
\end{prop}

\begin{df}[Null Space]
Let $U$ and $V$ be vector spaces over $K$ and $T : U \rightarrow V$. The \emph{null space} of $T$ is
\[ \mathcal{N}(T) := \{ x \in U \mid T(x) = 0 \} \enspace . \]
\end{df}

\begin{prop}
If $T : U \rightarrow V$ is a linear transformation then $\mathcal{N}(T)$ is a subspace of $U$.
\end{prop}

\begin{prop}
Let $U$ and $V$ be vector spaces over $K$. The set of all linear transformations $U \rightarrow V$ is a vector space over $K$ under
\begin{align*}
(S + T)(u) & = S(u) + T(u) \\
(\lambda S)(u) & = \lambda S(u)
\end{align*}
\end{prop}

\section{Normed Spaces}

\begin{df}[Norm]
Let $K$ be either $\mathbb{R}$ or $\mathbb{C}$.
A \emph{norm} on a vector space $V$ over $K$ is a function $\| \ \| : V \rightarrow \mathbb{R}$ such that:
\begin{enumerate}
\item $\forall x \in V. \| x \| = 0 \Rightarrow x = 0$
\item $\forall \lambda \in K. \forall x \in V. \| \lambda x \| = |\lambda| \| x \|$
\item \emph{Triangle Inequality} $\forall x,y \in V. \| x + y \| \leq \| x \| + \| y \|$
\end{enumerate}
\end{df}

\begin{prop}
Let $K$ be either $\mathbb{R}$ or $\mathbb{C}$. Let $V$ be a normed space over $K$. Define $d : V^2 \rightarrow \mathbb{R}$ by $d(x,y) = \| x-y \|$. Then $d$ is a metric on $V$.
\end{prop}

\begin{proof}
\pf\ Easy. \qed
\end{proof}

We identify any normed space $V$ with this metric space.

\begin{prop}
\label{prop:normzero}
Let $K$ be either $\mathbb{R}$ or $\mathbb{C}$. Let $V$ be a vector space over $K$. Let $\|\ \|$ be a norm on $V$. Then
\[ \| 0 \| = 0 \enspace . \]
\end{prop}

\begin{proof}
\pf
\begin{align*}
\| 0 \| & = \| 0 \cdot 0 \| & (\text{Proposition \ref{prop:zerotimes}}) \\
& = |0| \| 0 \| & (\text{Axiom 2 for a norm}) \\
& = 0 & \qed
\end{align*}
\end{proof}

\begin{prop}
Let $K$ be either $\mathbb{R}$ or $\mathbb{C}$. Let $V$ be a vector space over $K$. Let $\|\ \|$ be a norm on $V$. Let $x \in V$. Then
\[ \| x \| \geq 0 \enspace . \]
\end{prop}

\begin{proof}
\pf
\begin{align*}
0 & = \| 0 \| &  (\text{Proposition \ref{prop:normzero}}) \\
& = \| x - x \| \\
& \leq \| x \| + \| - x \| & (\text{Triangle Inequality}) \\
& = \| x \| + \| x \| & (\text{Axiom 2 for a norm}) \\
& = 2 \| x\| & \qed
\end{align*}
\end{proof}

\begin{prop}
\label{prop:distancebetweennorms}
Let $K$ be either $\mathbb{R}$ or $\mathbb{C}$. Let $V$ be a vector space over $K$. Let $\|\ \|$ be a norm on $V$. Let $x,y \in V$. Then
\[ | \| x \| - \| y \| | \leq \| x - y \| \enspace . \]
\end{prop}

\begin{proof}
\pf
\step{1}{$\| x \| - \| y \| \leq \| x - y \|$}
\begin{proof}
	\pf\ $\|x\| \leq \| x - y \| + \|y \|$ by the Triangle Inequality.
\end{proof}
\step{2}{$\| y \| - \| x \| \leq \| x - y \|$}
\begin{proof}
	\pf
	\begin{align*}
		\| x \| + \| x - y \| & = \| x \| + \| y - x\| & (\text{Axiom 2 for a norm}) \\
		& \leq \| y \| & (\text{Triangle Inequality})
	\end{align*}
\end{proof}
\qed
\end{proof}

\begin{cor}
Let $V$ be a normed space. Then $\|\ \| : V \rightarrow \mathbb{R}$ is continuous.
\end{cor}

\begin{df}[Euclidean Norm]
The \emph{Euclidean norm} on $\mathbb{C}^n$ is defined by
\[ \| (z_1, \ldots, z_n) \| = \sqrt{|z_1|^2 + \cdots + |z_n|^2} \]
\end{df}
%TODO Prove this is a norm

\begin{prop}
Define $\|\ \| : \mathbb{C}^n \rightarrow \mathbb{R}$ by
\[ \| (z_1, \ldots, z_n) \| = |z_1| + \cdots + |z_n| \]
Then this defines a norm on $\mathbb{C}^n$.
\end{prop}

\begin{proof}
\pf\ Easy. \qed
\end{proof}

\begin{prop}
Define $\|\ \| : \mathbb{C}^n \rightarrow \mathbb{R}$ by
\[ \| (z_1, \ldots, z_n) \| = \max(|z_1|, \ldots, |z_n|) \]
Then this defines a norm on $\mathbb{C}^n$.
\end{prop}

\begin{proof}
\pf\ Easy. \qed
\end{proof}

\begin{prop}
Let $\Omega$ be a closed bounded subset of $\mathbb{R}^n$. Define $\|\ \| : \mathcal{C}(\Omega) \rightarrow \mathbb{R}$ by $\| f \| = \max_{x \in \Omega} |f(x)|$. Then $\|\ \|$ defines a norm on $\mathcal{C}(\Omega)$.
\end{prop}

\begin{proof}
\pf\ Easy. \qed
\end{proof}

\begin{prop}
Let $p$ be a real number, $p \geq 1$. Define $\|\ \| : l^p \rightarrow \mathbb{R}$ by
\[ \| (z_n) \| = \left( \sum_{n=0}^\infty |z_n|^p \right)^{1/p} \enspace . \]
Then this defines a norm on $l^p$.
\end{prop}

\begin{proof}
\pf\ Easy. The triangle inequality is Minkowski's Inequality. \qed
\end{proof}

\begin{df}[Normed Space]
Let $K$ be either $\mathbb{R}$ or $\mathbb{C}$. A \emph{normed space} over $K$ consists of a vector space $V$ over $K$ and a norm on $V$.
\end{df}

We shall write simply:
\begin{itemize}
\item $K^n$ for the normed space $K^n$ under the Euclidean norm
\item $l^p$ for the normed space $l^p$ under the norm $\| (z_n) \| = \left( \sum_{n=0}^\infty |z_n|^p \right)^{1/p}$.
\end{itemize}

\begin{prop}
Let $K$ be either $\mathbb{R}$ or $\mathbb{C}$.
Let $V$ be a normed space over $K$. If $x_n \rightarrow l$ as $n \rightarrow \infty$ in $V$ and $\lambda_n \rightarrow \lambda$ as $n \rightarrow \infty$ in $K$, then $\lambda_n x_n \rightarrow \lambda l$ as $n \rightarrow \infty$.
\end{prop}

\begin{proof}
\pf
\step{1}{\pflet{$\epsilon > 0$}}
\step{11}{\pflet{$K = |\lambda| + \epsilon / 2\|l\|$}}
\step{2}{\pick\ $N$ such that, for all $n \geq N$, we have
$|\lambda_n - \lambda| < \epsilon / 2 \|l\|$ and $\| x_n - l \| < \epsilon / (2K)$}
\step{3}{For all $n \geq N$ we have $|\lambda_n| < K$}
\step{3}{$\| \lambda_n x_n - \lambda l\| < \epsilon$}
\begin{proof}
	\pf
	\begin{align*}
		\| \lambda_n x_n - \lambda l \| & \leq \| \lambda_n x_n - \lambda_n l \| + \| \lambda_n l - \lambda l \| \\
		& = |\lambda_n| \| x_n - l \| + | \lambda_n - \lambda | \|l\| \\
		& < K \frac{\epsilon}{2K} + \frac{\epsilon}{2 \|l\|} \|l\| \\
		& = \epsilon
	\end{align*}
\end{proof}
\qed
\end{proof}

\begin{prop}
In a normed space, if $x_n \rightarrow l$ and $y_n \rightarrow m$ then $x_n + y_n \rightarrow l + m$
\end{prop}

\begin{proof}
\pf
\begin{align*}
\| (x_n + y_n) - (l + m) \| & \leq \| x_n - l \| + \| y_n - m \| \\
& \rightarrow 0 & \qed
\end{align*}
\end{proof}

\begin{df}[Uniform Convergence]
Let $\Omega$ be a closed bounded set in $\mathbb{R}^n$. Let $(f_n)$ be a sequence in $\mathcal{C}(\Omega)$ and $f \in \mathcal{C}(\Omega)$. Then $(f_n)$ \emph{converges uniformly} to $f$ if and only if, for every $\epsilon > 0$, there exists $N$ such that $\forall x \in \Omega. \forall n \geq N. |f_n(x) - f(x)| < \epsilon$.
\end{df}

\begin{prop}
$(f_n)$ converges uniformly to $f$ iff $(f_n)$ converges to $f$ under the uniform convergence norm.
\end{prop}

\begin{proof}
\pf\ Easy. \qed
\end{proof}

\begin{prop}
There is no norm on $\mathcal{C}([0,1])$ that induces pointwise convergence.
\end{prop}

\begin{proof}
\pf
\step{1}{\pflet{$\|\ \|$ be any norm on $\mathcal{C}([0,1])$}}
\step{2}{For $n \in \mathbb{N}$, define $g_n \in \mathcal{C}([0,1])$ by
\[ g_n(t) = \begin{cases}
2^n t & \text{if } 0 \leq t \leq 2^{-n} \\
2 - 2^n t & \text{if } 2^{-n} \leq t \leq 2^{1-n} \\
0 & \text{otherwise}
\end{cases} \]}
\step{3}{For all $n \in \mathbb{N}$ we have $\|g_n\| \neq 0$}
\step{4}{For $n \in \mathbb{N}$, \pflet{$f_n = g_n / \| g_n \|$}}
\step{5}{For all $n \in \mathbb{N}$, $\| f_n \| = 1$}
\step{55}{$f_n$ does not converge to 0}
\step{6}{$f_n \rightarrow 0$ as $n \rightarrow \infty$ pointwise.}
\qed
\end{proof}

\begin{df}[Equivalent Norms]
Let $K$ be either $\mathbb{R}$ or $\mathbb{C}$. Let $V$ be a vector spaces over $K$. Then two norms $\|\ \|_1$ and $\|\ \|_2$ are \emph{equivalent} if and only if, for any sequence $(x_n)$ in $V$ and $l \in V$, we have $x_n \rightarrow l$ under $\|\ \|_1$ if and only if $x_n \rightarrow l$ under $\|\ \|_2$.
\end{df}

\begin{prop}
Let $K$ be either $\mathbb{R}$ or $\mathbb{C}$. Let $V$ be a vector spaces over $K$. Let $\|\ \|_1$ and $\|\ \|_2$ be norms on $V$. Then $\|\ \|_1$ and $\|\ \|_2$ are equivalent if and only if there exist positive reals $\alpha$ and $\beta$ such that, for all $x \in V$,
\begin{equation}
\label{eq:eqnorm}
 \alpha \| x \|_1 \leq \| x \|_2 \leq \beta \| x \|_1
 \end{equation}
\end{prop}

\begin{proof}
\pf
\step{1}{If $\|\ \|_1$ and $\|\ \|_2$ are equivalent then (\ref{eq:eqnorm}) holds.}
\begin{proof}
	\step{a}{\assume{$\|\ \|_1$ and $\|\ \|_2$ are equivalent.}}
	\step{b}{There exists $\alpha > 0$ such that, for all $x \in V$, we have $\alpha \|x\|_1 \leq \| x \|_2$}
	\begin{proof}
		\step{i}{\assume{for a contradiction $\forall \alpha > 0. \exists x \in V. \alpha \|x\|_1 > \|x\|_2$}}
		\step{ii}{For $n \in \mathbb{Z}^+$, choose $x_n \in V$ such that $1/n \|x_n\|_1 > \|x_n\|_2$}
		\step{iii}{For $n \in \mathbb{Z}^+$, \pflet{
		\[ y_n = \frac{1}{\sqrt{n}} \frac{x_n}{\|x_n\|_2} \]}}
		\step{iv}{$\|y_n\|_2 \rightarrow 0$ as $n \rightarrow \infty$}
		\step{v}{For all $n \in \mathbb{Z}^+$, $\|y_n\|_1 > \sqrt{n}$}
		\step{vi}{$\|y_n\| \not\rightarrow 0$ as $n \rightarrow \infty$}
	\end{proof}
	\step{c}{There exists $\beta > 0$ such that, for all $x \in V$, we have $\|x\|_2 \leq \beta \| x \|_1$}
	\begin{proof}
		\pf\ Similar.
	\end{proof}
\end{proof}
\step{2}{If (\ref{eq:eqnorm}) holds then $\|\ \|_1$ and $\|\ \|_2$ are equivalent.}
\begin{proof}
	\step{a}{\assume{(\ref{eq:eqnorm}) holds.}}
	\step{b}{\pflet{$(x_n)$ be a sequence in $V$ and $l \in V$}}
	\step{c}{If $x_n \rightarrow l$ under $\|\ \|_1$ then $x_n \rightarrow l$ under $\|\ \|_2$.}
	\begin{proof}
		\step{i}{\assume{$x_n \rightarrow l$ und $\|\ \|_1$.}}
		\step{ii}{\pflet{$\epsilon > 0$}}
		\step{iii}{\pick\ $N$ such that $\forall n \geq N. \| x_n - l \| < \epsilon / \beta$}
		\step{iv}{\pflet{$n \geq N$}}
		\step{v}{$\|x_n - l\|_2 < \epsilon$}
		\begin{proof}
			\pf
			\begin{align*}
				\|x_n - l\|_2 & \leq \beta \| x_n - l\|_1 \\
				& < \epsilon
			\end{align*}
		\end{proof}
	\end{proof}
	\step{d}{If $x_n \rightarrow l$ under $\|\ \|_2$ then $x_n \rightarrow l$ under $\|\ \|_1$.}
	\begin{proof}
		\pf\ Similar.
	\end{proof}
\end{proof}
\qed
\end{proof}

\begin{prop}
\label{prop:norm0lm}
Let $K$ be either $\mathbb{R}$ or $\mathbb{C}$. Let $V$ be a normed space over $K$. If $x_1, \ldots, x_n \in V$ are linearly independent, then there exists $c > 0$ such that, for all $\alpha_1, \ldots, \alpha_n \in K$,
\[ \| \alpha_1 x_1 + \cdots + \alpha_n x_n \| \geq c (|\alpha_1| + \cdots + |\alpha_n|) \enspace . \]
\end{prop}

\begin{proof}
\pf
\step{2}{\pflet{$B = \{(\beta_1, \ldots, \beta_n) \in K^n \mid |\beta_1| + \cdots + |\beta_n| = 1\}$}}
\step{1}{\pflet{$f : B \rightarrow \mathbb{R}$ be the function
\[ f(\beta_1, \ldots, \beta_n) = \| \beta_1 x_1 + \cdots + \beta_n x_n \| \enspace . \]}}
\step{3}{\pflet{$c$ be the minimum value in $f(B)$}}
\begin{proof}
	\pf\ $f$ is continuous and $B$ is compact.
\end{proof}
\step{4}{$c > 0$}
\begin{proof}
	\pf\ We never have $f(\beta_1, \ldots, \beta_n) = 0$ by linear independence.
\end{proof}
\step{5}{\pflet{$\alpha_1, \ldots, \alpha_n \in K$}}
\step{6}{\assume{w.l.o.g. $\alpha_1$, \ldots, $\alpha_n$ are not all zero.}}
\step{7}{For $i = 1, \ldots, n$, \pflet{$\beta_i = \alpha_i / (|\alpha_1| + \cdots + |\alpha_n|)$}}
\step{8}{$(\beta_1, \ldots, \beta_n) \in B$}
\step{9}{$f(\beta_1, \ldots, \beta_n) \geq c$}
\step{10}{$\| \alpha_1 x_1 + \cdots + \alpha_n x_n\| \geq c (|\alpha_1| + \cdots + |\alpha_n|)$}
\qed
\end{proof}

\begin{thm}
Let $K$ be either $\mathbb{R}$ or $\mathbb{C}$. Let $V$ be a finite dimensional vector space over $K$. Then any two norms on $V$ are equivalent.
\end{thm}

\begin{proof}
\pf
\step{1}{\pick\ a basis $\{ e_1, \ldots, e_n \}$ for $V$}
\step{2}{\pflet{$\|\ \|_0 : V \rightarrow \mathbb{R}$ be the function
\[ \| \alpha_1 e_1 + \cdots + \alpha_n e_n \|_0 = |\alpha_1| + \cdots + |\alpha_n| \enspace . \]}}
\step{3}{$\|\ \|_0$ is a norm.}
\begin{proof}
	\step{a}{$\forall x \in V. \| x \|_0 = 0 \Rightarrow x = 0$}
	\begin{proof}
		\pf\ If $|\alpha_1| + \cdots | \alpha_n| = 0$ then $\alpha_1 = \cdots = \alpha_n = 0$.
	\end{proof}
	\step{b}{$\forall \lambda \in K. \forall x \in V. \| \lambda x \| = |\lambda| \|x\|$}
	\begin{proof}
		\pf
		\begin{align*}
			\| \lambda (\alpha_1 e_1 + \cdots + \alpha_n e_n) \|_0 & = \| \lambda \alpha_1 e_1 + \cdots + \lambda \alpha_n e_n \|_0 \\
			& = |\lambda \alpha_1| + \cdots + |\lambda \alpha_n| \\
			& = |\lambda| (|\alpha_1| + \cdots + |\alpha_n|) \\
			& = |\lambda| \| \alpha_1 e_1 + \cdots + \alpha_n e_n \|
		\end{align*}
	\end{proof}
	\step{c}{The triangle inequality holds.}
	\begin{proof}
		\pf
		\begin{align*}
			\| (\alpha_1 e_1 + \cdots + \alpha_n e_n) + (\beta_1 e_1 + \cdots + \beta_n e_n) \|
			& = \| (\alpha_1 + \beta_1) e_1 + \cdots + (\alpha_n + \beta_n) e_n \| \\
			& = |\alpha_1 + \beta_1| + \cdots + |\alpha_n + \beta_n| \\
			& \leq (|\alpha_1| + \cdots + |\alpha_n|) + (|\beta_1| + \cdots + |\beta_n|) \\
			& = \| \alpha_1 e_1 + \cdots + \alpha_n e_n\|_0 + \| \beta_1 e_1 + \cdots + \beta_n e_n \|_0
		\end{align*}
	\end{proof}
\end{proof}
\step{4}{\pflet{$\|\ \|$ be any norm on $V$.} \prove{$\|\ \|$ is equivalent to $\|\ \|_0$.}}
\step{5}{For all $\alpha_1, \ldots, \alpha_n \in K$,
\[ \| \alpha_1 e_1 + \cdots + \alpha_n e_n \| \leq \max(\| e_1 \|, \ldots, \| e_n \|) (|\alpha_1| + \cdots | \alpha_n|) \]}
\begin{proof}
	\step{a}{\pflet{$\alpha_1, \ldots, \alpha_n \in K$}}
	\step{b}{$\| \alpha_1 e_1 + \cdots + \alpha_n e_n \| \leq \max(\| e_1 \|, \ldots, \| e_n \|) (|\alpha_1| + \cdots | \alpha_n|)$}
	\begin{proof}
		\pf
		\begin{align*}
			\| \alpha_1 e_1 + \cdots + \alpha_n e_n \|
			& \leq |\alpha_1| \| e_1 \| + \cdots + |\alpha_n| \|e_n\| \\
			& \leq (|\alpha_1| + \cdots + |\alpha_n|) \max(\|e_1\|, \ldots, \|e_n\|) & \qed
		\end{align*}
	\end{proof}
\end{proof}
\step{6}{\pflet{$\beta = \max(\|e_1\|, \ldots, \|e_n\|)$}}
\step{7}{For all $x \in V$,
\[ \| x \| \leq \beta \| x \|_0 \enspace . \]}
\step{8}{There exists $\alpha > 0$ such that, for all $x \in V$,
\[ \alpha \| x \|_0 \leq \| x \| \enspace . \]}
\begin{proof}
	\pf\ Proposition \ref{prop:norm0lm}.
\end{proof}
\qed
\end{proof}

\begin{df}[Closed Ball]
Let $K$ be either $\mathbb{R}$ or $\mathbb{C}$. Let $V$ be a vector space over $K$. Let $x \in V$ and let $r$ be a positive real number. The \emph{closed ball} with \emph{centre} $x$ and \emph{radius} $r$ is
\[ \overline{B(x,r)} := \{ y \in V \mid \| x - y \| \leq r \} \enspace . \]
\end{df}

\begin{df}[Sphere]
Let $K$ be either $\mathbb{R}$ or $\mathbb{C}$. Let $V$ be a vector space over $K$. Let $x \in V$ and let $r$ be a positive real number. The \emph{sphere} with \emph{centre} $x$ and \emph{radius} $r$ is
\[ S(x,r) := \{ y \in V \mid \| x - y \| = r \} \enspace . \]
\end{df}

\begin{prop}
\label{prop:closedball}
Every closed ball is closed.
\end{prop}

\begin{prop}
Every sphere is closed.
\end{prop}

\begin{prop}
The union of two closed sets is closed.
\end{prop}

\begin{prop}
The intersection of a nonempty set of closed sets is closed.
\end{prop}

\begin{prop}
In a normed space $V$, both $\emptyset$ and $V$ are closed.
\end{prop}

\begin{prop}
\label{prop:closedconverge}
Let $V$ be a normed space and $C \subseteq V$. Then $C$ is closed iff, for every sequence $(x_n)$ in $C$ and $l \in V$, if $x_n \rightarrow l$ then $l \in C$.
\end{prop}

\begin{proof}
\pf
\step{1}{If $C$ is closed then, for every sequence $(x_n)$ in $C$ and $l \in V$, if $x_n \rightarrow l$ then $l \in C$.}
\begin{proof}
	\step{a}{\assume{$C$ is closed.}}
	\step{b}{\pflet{$(x_n)$ be a sequence in $C$.}}
	\step{c}{\pflet{$l \in V$}}
	\step{d}{\assume{$x_n \rightarrow l$ as $n \rightarrow \infty$}}
	\step{e}{\assume{for a contradiction $l \notin C$}}
	\step{f}{\pick\ $\epsilon > 0$ such that $B(l,\epsilon) \subseteq V - C$}
	\step{g}{\pick\ $N$ such that $\forall n \geq N. \| x_n - l \| < \epsilon$}
	\step{h}{$x_N \in C$}
	\step{i}{$\| x_N - l \| < \epsilon$}
	\qedstep
	\begin{proof}
		\pf\ This is a contradiction.
	\end{proof}
\end{proof}
\step{2}{If, for every sequence $(x_n)$ in $C$ and $l \in V$, if $x_n \rightarrow l$ then $l \in C$, then $C$ is closed.}
\begin{proof}
	\step{a}{\assume{for every sequence $(x_n)$ in $C$ and $l \in V$, if $x_n \rightarrow l$ then $l \in C$.}}
	\step{b}{\pflet{$x \in V - C$}}
	\step{c}{\assume{for a contradiction there is no $\epsilon > 0$ such that $B(x,\epsilon) \subseteq V - C$}}
	\step{d}{For $n \in \mathbb{Z}^+$, \pick\ $x_n \in B(x,\epsilon) \cap C$}
	\step{e}{$x_n \rightarrow x$ as $n \rightarrow \infty$}
	\qedstep
	\begin{proof}
		\pf\ This is a contradiction.
	\end{proof}
\end{proof}
\qed
\end{proof}

\begin{df}[Closure]
Let $V$ be a normed space and $A \subseteq V$. The \emph{closure} of $A$, $\cl A$, is the intersection of all the closed sets that include $A$.
\end{df}

\begin{thm}
\label{thm:closureaslimits}
Let $V$ be a normed space and $S \subseteq V$. Then the closure of $S$ is the set of all limits of convergent sequences in $S$.
\end{thm}

\begin{proof}
\pf
\step{1}{For all $l \in \cl S$, there exists a sequence $(x_n)$ that converges to $l$.}
\begin{proof}
	\step{a}{\pflet{$l \in \cl S$}}
	\step{b}{For $n \in \mathbb{N}$, \pick\ $x_n \in B(l,1/(n+1)) \cap S$}
	\begin{proof}
		\step{i}{\assume{for a contradiction $B(l,1/(n+1))$ does not intersect $S$.}}
		\step{ii}{$V - B(l,1/(n+1))$ is a closed set that includes $S$.}
		\step{iii}{$\cl S \subseteq V - B(l,1/(n+1))$}
		\step{iv}{$l \notin B(l,1/(n+1))$}
		\qedstep
		\begin{proof}
			\pf\ This is a contradiction.
		\end{proof}
	\end{proof}
	\step{c}{$x_n \rightarrow l$}
\end{proof}
\step{2}{For every sequence $(x_n)$ in $S$, if $x_n \rightarrow l$ then $l \in \cl S$.}
\begin{proof}
	\pf\ Proposition \ref{prop:closedconverge}.
\end{proof}
\qed
\end{proof}

\begin{df}[Dense]
Let $V$ be a normed space and $S \subseteq V$. Then $S$ is \emph{dense} iff $\cl S = V$.
\end{df}

\begin{prop}
In $\mathcal{C}([a,b])$, the set of polynomials is dense.
\end{prop}

\begin{proof}
\pf\ By the Weierstrass Theorem. \qed
\end{proof}

\begin{prop}
For any real $p \geq 1$, the set of all sequences with only finitely many non-zero terms is dense in $l^p$.
\end{prop}

\begin{proof}
\pf
\step{1}{\pflet{$p \geq 1$}}
\step{2}{\pflet{$(z_n) \in l^p$}}
\step{3}{\pflet{$\epsilon > 0$}}
\step{4}{\pick\ $N$ such that $\forall n \geq N. |z_n| < \epsilon / 2$}
\step{5}{\pflet{$(y_n)$ be the sequence with $y_n = z_n$ for $n < N$, and $y_n = 0$ for $n \geq N$}}
\step{6}{$\| (z_n) - (y_n) \| \leq \epsilon / 2$}
\step{7}{$\| (z_n) - (y_n) \| < \epsilon$}
\qed
\end{proof}

\begin{thm}
Let $V$ be a normed space. Let $S \subseteq V$. Then the following are equivalent:
\begin{enumerate}
\item $S$ is dense.
\item For all $x \in V$, there exists a sequence $(x_n)$ in $S$ such that $x_n \rightarrow x$.
\item Every nonempty open subset of $V$ intersects $S$.
\end{enumerate}
\end{thm}

\begin{proof}
\pf
\step{1}{$1 \Leftrightarrow 2$}
\begin{proof}
	\pf\ Theorem \ref{thm:closureaslimits}.
\end{proof}
\step{2}{$1 \Leftrightarrow 3$}
\begin{proof}
	\pf
	\begin{align*}
	S \text{ is dense} & \Leftrightarrow \text{the only closed set that includes $S$ is $V$} \\
	& \Leftrightarrow \text{the only open set that does not intersect $S$ is empty}
	\end{align*}
\end{proof}
\qed
\end{proof}

\begin{df}[Compact]
Let $V$ be a normed space and $S \subseteq V$. Then $S$ is \emph{compact} if and only if every sequence in $S$ has a subsequence that converges to a limit in $S$.
\end{df}

\begin{prop}
\label{prop:compactclosed}
Every compact set is closed.
\end{prop}

\begin{proof}
\pf
\step{0}{\pflet{$V$ be a normed space.}}
\step{1}{\pflet{$C \subseteq V$ be compact.}}
\step{2}{\pflet{$(x_n)$ be a sequence in $C$ that converges to $l \in V$.}}
\step{3}{\pick\ a subsequence $(y_n)$ of $(x_n)$ that converges to $m \in C$.}
\step{4}{$l = m$}
\step{5}{$l \in C$}
\qedstep
\begin{proof}
	\pf\ Proposition \ref{prop:closedconverge}.
\end{proof}
\qed
\end{proof}

\begin{df}[Bounded]
Let $V$ be a normed space and $S \subseteq V$. Then $S$ is \emph{bounded} iff there exists $r > 0$ such that $S \subseteq B(0,r)$.
\end{df}

\begin{prop}
In $\mathbb{R}^n$ and $\mathbb{C}^n$, the compact sets are the closed bounded sets.
\end{prop}

\begin{proof}
\pf
\step{1}{\pflet{$C \subseteq K^n$}}
\step{2}{If $C$ is compact then $C$ is closed.}
\begin{proof}
	\pf\ Proposition \ref{prop:compactclosed}.
\end{proof}
\step{3}{If $C$ is compact then $C$ is bounded.}
\begin{proof}
	\step{a}{\assume{$C$ is compact.}}
	\step{b}{\assume{for a contradiction $C$ is not bounded.}}
	\step{c}{For $n \in \mathbb{N}$, \pick\ $x_n \in C$ with $\| x_n \| > n + 1$.}
	\step{d}{\pick\ a convergent subsequence $(x_{n_r})$ that converges to $l \in C$}
	\step{e}{$\| x_{n_r} \| \rightarrow \| l \|$}
	\step{f}{$\| x_{n_r} \| \rightarrow + \infty$}
	\qedstep
	\begin{proof}
		\pf\ This is a contradiction.
	\end{proof}
\end{proof}
\step{4}{If $C$ is closed and bounded then $C$ is compact.}
\begin{proof}
	\pf\ By the Bolzano-Weierstrass Theorem.
\end{proof}
\qed
\end{proof}

\begin{prop}
Let $V$ be a normed space and $S \subseteq V$. Then $S$ is bounded if and only if, for every sequence $(x_n)$ in $S$ and every sequence $(\lambda_n)$ in $K$, if $\lambda_n \rightarrow 0$ then $\| \lambda_n x_n \| \rightarrow 0$.
\end{prop}

\begin{proof}
\pf
\step{1}{If $S$ is bounded then, for every sequence $(x_n)$ in $S$ and every sequence $(\lambda_n)$ in $K$, if $\lambda_n \rightarrow 0$ then $\| \lambda_n x_n \| \rightarrow 0$.}
\begin{proof}
	\step{a}{\assume{$S$ is bounded.}}
	\step{aa}{\pick\ $r > 0$ such that $S \subseteq B(0,r)$.}
	\step{b}{\pflet{$(x_n)$ be a sequence in $S$.}}
	\step{c}{\pflet{$(\lambda_n)$ be a sequence in $K$.}}
	\step{d}{\assume{$\lambda_n \rightarrow 0$}}
	\step{e}{\pflet{$\epsilon > 0$}}
	\step{f}{\pick\ $N$ such that $\forall n \geq N. |\lambda_n| < \epsilon / r$}
	\step{g}{$\forall n \geq N. \| \lambda_n x_n \| < \epsilon$}
\end{proof}
\step{2}{If $S$ is unbounded then there exists a sequence $(x_n)$ in $S$ and $(\lambda_n)$ in $K$ such that $\lambda_n \rightarrow 0$ and $\| \lambda_n x_n \| \not\rightarrow 0$.}
\begin{proof}
	\step{a}{$S$ is unbounded.}
	\step{b}{For $n \in \mathbb{N}$, \pick\ $x_n \in S$ such that $\| x_n \| > n$.}
	\step{c}{For $n \in \mathbb{N}$, \pflet{$\lambda_n = 1/n$ if $n > 0$, $1$ if $n = 0$}}
	\step{d}{$\lambda_n \rightarrow 0$}
	\step{e}{$\| \lambda_n x_n \| > 1$ for all $n > 1$}
\end{proof}
\qed
\end{proof}

\begin{prop}
In $\mathcal{C}([0,1])$, the unit ball $\overline{B(0,1)}$ is closed and bounded but not compact.
\end{prop}

\begin{proof}
\pf
\step{1}{$\overline{B(0,1)}$ is closed.}
\begin{proof}
	\pf\ Proposition \ref{prop:closedball}.
\end{proof}
\step{2}{$\overline{B(0,1)}$ is bounded.}
\begin{proof}
	\pf\ $\overline{B(0,1)} \subseteq B(0,2)$.
\end{proof}
\step{3}{$\overline{B(0,1)}$ is not compact.}
\begin{proof}
	\step{a}{For $n \in \mathbb{N}$, \pflet{$x_n : [0,1] \rightarrow \mathbb{R}$ be the function $x_n(t) = t^n$.}}
	\step{b}{For $n \in \mathbb{N}$, we have $x_n \in \overline{B(0,1)}$.}
	\step{c}{No subsequence of $(x_n)$ converges.}
\end{proof}
\qed
\end{proof}

\begin{thm}[Riesz's Lemma]
Let $X$ be a closed proper subspace of a normed space $V$. For every $\epsilon \in (0,1)$, there exists $x_\epsilon \in V$ such that $\| x_\epsilon \| = 1$ and $\forall x \in X. \| x_\epsilon - x \| \geq \epsilon$.
\end{thm}

\begin{proof}
\pf
\step{1}{\pick\ $z \in E - X$}
\step{2}{\pflet{$d = \inf_{x \in X} \| z - x \|$}}
\step{3}{$d > 0$}
\begin{proof}
	\step{a}{\pick- $\epsilon > 0$ such that $B(z,\epsilon) \subseteq E - X$}
	\step{b}{$d \geq \epsilon$}
\end{proof}
\step{4}{For all $\epsilon \in (0,1)$, choose $y_\epsilon \in X$ such that
\[ d \leq \| z - y_\epsilon \| < d / \epsilon \enspace . \]}
\step{5}{For $\epsilon \in (0,1)$, \pflet{
\[ x_\epsilon = \frac{z - y_\epsilon}{\|z - y_\epsilon\|} \enspace . \]
}}
\step{6}{For all $x \in X$ we have $\| x_\epsilon - x \| > \epsilon$}
\begin{proof}
	\pf
	\begin{align*}
		\| x_\epsilon - x \| & =
		\| \frac{z - y_\epsilon}{\|z - y_\epsilon\|} - x \| \\
		& = \frac{1}{\| z - y_\epsilon \|} \| z - y_\epsilon - \| z - y_\epsilon \| x \| & (y_\epsilon + \| z - y_\epsilon \| x \in X) \\
		& \geq \frac{1}{\| z - y_\epsilon \|} d \\
		& > \epsilon
	\end{align*}
\end{proof}
\qed
\end{proof}

\begin{thm}
Let $V$ be a normed space. Then $V$ is finite dimensional if and only if $\overline{B(0,1)}$ is compact.
\end{thm}

\begin{proof}
\pf
\step{1}{If $V$ is finite dimensional then $\overline{B(0,1)}$ is compact.}
\begin{proof}
	\step{a}{\assume{$V$ is finite dimensional.}}
	\step{b}{\pick\ a basis $\{e_1, \ldots, e_n\}$.}
	\step{c}{\assume{w.l.o.g. $\forall \alpha_1, \ldots, \alpha_n \in K. \| \alpha_1 e_1 + \cdots + \alpha_n e_n \| = | \alpha_1 | + \cdots + | \alpha_n |$}}
	\step{d}{\pflet{$(x_m)$ be a sequence in $\overline{B(0,1)}$}}
	\step{e}{For $m \in \mathbb{N}$, \pflet{$x_m = \alpha_{m1} e_1 + \cdots + \alpha_{mn} e_n$}}
	\step{f}{For $m \in \mathbb{N}$ and $i = 1, \ldots, n$, we have $|\alpha_{mi}| \leq 1$}
	\step{g}{For $i = 1, \ldots, n$, \pick\ a convergent subsequence $(\alpha_{m_ri})$ of $(\alpha_{mi})$ in $\mathbb{C}$ that converges to $l_i$}
	\begin{proof}
		\pf\ Since $\overline{B(0,1)}$ is compact in $K$.
	\end{proof}
	\step{h}{$x_m$ converges to $l_1 e_1 + \cdots + l_n e_n$}
	\begin{proof}
		\pf
		\begin{align*}
			\| x_m - l_1 e_1 - \cdots - l_n e_n \| & = \| (\alpha_{m1} - l_1) e_1 + \cdots + (\alpha_{mn} - l_n) e_n \| \\
			& = | \alpha_{m1} - l_1 | + \cdots + |\alpha_{mn} - l_n| \\
			& \rightarrow 0 & \text{as } m \rightarrow \infty
		\end{align*}
	\end{proof}
\end{proof}
\step{2}{If $V$ is infinite dimensional then $\overline{B(0,1)}$ is not compact.}
\begin{proof}
	\step{a}{\assume{$V$ is infinite dimensional.}}
	\step{b}{Choose a sequence $(x_n)$ such that $\|x_n\| = 1$ and $\| x_m - x_n \| \geq 1/2$ for all $m \neq n$.}
	\begin{proof}
		\step{i}{\assume{as induction hypothesis there exists a sequence $(x_0, x_1, \ldots, x_n)$ such that $\| x_i \| = 1$ and $\| x_i - x_j \| \geq 1/2$ for $i \neq j$}}
		\step{ii}{\pick\ $x_{n+1}$ such that $\| x_{n+1} \| = 1$ and $\| x_{n_1} - x \| \geq 1/2$ for $x \in \{ x_1, \ldots, x_n \}$.}
	\end{proof}
	\step{c}{\assume{for a contradiction $(x_{n_r})$ is a subsequence that converges to $l$}}
	\step{d}{For all $r$ we have $1/2 \leq \| x_{n_r} - l \| + \| x_{n_{r+1}} - l \|$}
	\begin{proof}
		\pf
		\begin{align*}
			1/2 & \leq \| x_{n_r} - x_{n_{r+1}} \| & (\text{\stepref{b}}) \\
			& \leq \| x_{n_r} - l \| + \| x_{n_{r+1}} - l \| & (\text{Triangle Inequality})
		\end{align*}
	\end{proof}
	\qedstep
	\begin{proof}
		\pf\ This is a contradiction.
	\end{proof}
\end{proof}
\qed
\end{proof}

%TODO Generalize
\begin{prop}
\label{prop:continuous_at_one_point}
Let $U$ and $V$ be normed spaces. Let $f : U \rightarrow V$. If $f$ is continuous at one point, then it is continuous.
\end{prop}

\begin{proof}
\pf
\step{1}{\assume{$f$ is continuous at $x_0 \in U$.}}
\step{2}{\pflet{$(x_n)$ be a sequence in $U$ that converges to $l \in U$.}}
\step{3}{$x_n - l + x_0 \rightarrow x_0$ as $n \rightarrow \infty$}
\step{4}{$f(x_n - l + x_0) \rightarrow f(x_0)$ as $n \rightarrow \infty$}
\step{5}{$f(x_n) - f(l) + f(x_0) \rightarrow f(x_0)$ as $n \rightarrow \infty$}
\step{6}{$f(x_n) \rightarrow f(l)$ as $n \rightarrow \infty$.}
\qed
\end{proof}

\begin{df}[Bounded]
Let $U$ and $V$ be normed spaces and $L : U \rightarrow V$ be a linear transformation. Then $L$ is \emph{bounded} iff there exists $\alpha > 0$ such that $\forall x \in U. \| L(x) \| \leq \alpha \| x \|$.
\end{df}

\begin{thm}
Let $U$ and $V$ be normed spaces. Let $L : U \rightarrow V$ be a linear transformation. Then $L$ is continuous if and only if it is bounded.
\end{thm}

\begin{proof}
\pf
\step{1}{If $L$ is continuous then $L$ is bounded.}
\begin{proof}
	\step{a}{\assume{$L$ is not bounded.}}
	\step{b}{For $n \in \mathbb{N}$, choose $x_n \in U$ such that $\| L(x_n) \| > (n+1) \| x_n \|$}
	\step{c}{For $n \in \mathbb{N}$, \pflet{$y_n = x_n / (n+1) \| x_n \|$}}
	\step{d}{$y_n \rightarrow 0$ as $n \rightarrow \infty$}
	\step{e}{For $n \in \mathbb{N}$, we have $\| L(y_n) \| > 1$}
	\step{f}{$L(y_n) \not\rightarrow 0 = L(0)$ as $n \rightarrow \infty$}
	\step{g}{$L$ is not continuous.}
\end{proof}
\step{2}{If $L$ is bounded then $L$ is continuous.}
\begin{proof}
	\step{a}{\pflet{$\alpha > 0$ be such that $\forall x \in U. \| L(x) \| \leq \alpha \| x \|$} \prove{$L$ is continuous at 0.}}
	\step{b}{\pflet{$(x_n)$ be a sequence in $U$ that converges to 0.}}
	\step{c}{$L(x_n) \rightarrow 0$ as $n \rightarrow \infty$}
	\qedstep
	\begin{proof}
		\pf\ Proposition \ref{prop:continuous_at_one_point}.
	\end{proof}
	\qed
\end{proof}
\qed
\end{proof}

\begin{cor}
If $U$ and $V$ are finite dimensional normed spaces, then every linear transformation $U \rightarrow V$ is continuous.
\end{cor}

\begin{df}
For $U$ and $V$ normed spaces, let $\mathcal{B}(U,V)$ be the space of all bounded linear transformations $U \rightarrow V$. This is a subspace of the space of all linear transformations $U \rightarrow V$.

Define the \emph{uniform convergence norm} on $\mathcal{B}(U,V)$ by
\[ \| L \| = \sup \{ \| L(x) \| \mid x \in U, \| x \| = 1 \} \enspace . \]

We prove this is a norm.
\end{df}

\begin{proof}
\pf
\step{1}{$\forall L \in \mathcal{B}(U,V). \| L \| = 0 \Rightarrow L = 0$}
\begin{proof}
	\step{a}{\pflet{$L \in \mathcal{B}(U,V)$}}
	\step{b}{\assume{$\|L\\ = 0$}}
	\step{c}{For all $x \in U$, if $\|x\| = 1$ then $\|L(x)\| = 0$}
	\step{d}{\pflet{$x \in U$} \prove{$L(x) = 0$}}
	\step{e}{\assume{w.l.o.g. $x \neq 0$}}
	\step{f}{$\| L(x / \| x \|) \| = 0$}
	\begin{proof}
		\pf\ \stepref{c}
	\end{proof}
	\step{g}{$L(x / \|x\|) = 0$}
	\step{h}{$L(x) / \|x\| = 0$}
	\step{i}{$L(x) = 0$}
\end{proof}
\step{2}{$\forall \lambda \in K. \forall L \in \mathcal{B}(U,V). \| \lambda L \| = |\lambda| \| L \|$}
\begin{proof}
	\step{a}{\pflet{$\lambda \in K$}}
	\step{b}{\pflet{$L \in \mathcal{B}(U,V)$}}
	\step{c}{$\| \lambda L \| = |\lambda| \|L\|$}
	\begin{proof}
		\pf
		\begin{align*}
			\| \lambda L \| & = \sup_{\|x\| = 1} \| \lambda L(x) \| \\
			& = \sup_{\|x\| = 1} (|\lambda| \|L(x)\|) \\
			& = |\lambda| \sup_{\|x\| = 1} \|L(x)\| \\
			& = |\lambda| \|L\|
		\end{align*}
	\end{proof}
\end{proof}
\step{3}{The triangle inequality holds.}
\begin{proof}
	\step{a}{\pflet{$L,M \in \mathcal{B}(U,V)$}}
	\step{b}{For all $x \in U$, if $\|x\| = 1$ then
	\[ \| L(x) + M(x) \| \leq \| L(x) \| + \| M(x) \| \enspace . \]}
	\step{c}{For all $x \in U$, if $\|x\| = 1$ then
	\[ \| L(x) + M(x) \| \leq \| L \| + \| M \| \enspace . \]}
	\step{d}{$\| L + M \| \leq \| L \| + \| M \|$}
\end{proof}
\qed
\end{proof}

\begin{prop}
Let $U$ and $V$ be normed spaces and $L \in \mathcal{B}(U,V)$. Then $\|L\|$ is the least number such that $\forall x \in U. \| L(x) \| \leq \| L \| \| x \|$.
\end{prop}

\begin{thm}
Let $U$ and $V$ be normed spaces. Let $T : U \rightarrow V$ be a continuous linear transformation. Then the null space $\mathcal{N}(T)$ is closed in $U$.
\end{thm}

\begin{proof}
\pf\ If $(x_n)$ is a sequence in $\mathcal{N}(T)$ and $x_n \rightarrow l$ then $T(l) = \lim_{n \rightarrow \infty} T(x_n) = 0$ so $l \in \mathcal{N}(T)$. \qed
\end{proof}

\begin{thm}[Diagonal Theorem]
Let $V$ be a normed space. Let $(x_{ij})_{i,j \in \mathbb{N}}$ be an infinite matrix in $V$. If:
\begin{enumerate}
\item
$\forall j \in \mathbb{N}. x_{ij} \rightarrow 0$ as $i \rightarrow \infty$;
\item Every strictly increasing sequence of natural numbers $(p_i)$ has a subsequence $(q_i)$ such that
\[ \sum_{j=0}^\infty x_{q_iq_j} \rightarrow 0 \text{ as } i \rightarrow \infty \]
\end{enumerate}
then $x_{ii} \rightarrow 0$ as $i \rightarrow \infty$.
\end{thm}

\begin{proof}
\pf
\step{1}{\pflet{$V$ be a normed space.}}
\step{2}{\pflet{$(x_{ij})_{i,j \in \mathbb{N}}$ be an infinite matrix in $V$.}}
\step{3}{\assume{$\forall j \in \mathbb{N}. x_{ij} \rightarrow 0$ as $i \rightarrow \infty$}}
\step{4}{\assume{Every strictly increasing sequence of natural numbers $(p_i)$ has a subsequence $(q_i)$ such that
\[ \sum_{j=0}^\infty x_{q_iq_j} \rightarrow 0 \text{ as } i \rightarrow \infty \]}}
\step{5}{\assume{for a contradiction $x_{ii} \not\rightarrow 0$ as $i \rightarrow \infty$}}
\step{6}{\pick\ a strictly increasing sequence $(p_i)$ and $\epsilon > 0$ such that $\forall i \in \mathbb{N}. \| x_{p_ip_i} \| \geq \epsilon$}
\begin{proof}
	\step{a}{\pick\ $\epsilon > 0$ such that, for all $N$, there exists $n \geq N$ such that $\| x_{ii} \| \geq \epsilon$}
	\step{b}{Choose a sequence $(p_i)$ such that, for all $i$, we have $p_{i+1} \geq p_i + 1$ and $\| x_{p_ip_i} \| \geq \epsilon$}
\end{proof}
\step{7}{\pick\ a subsequence $(q_i)$ of $(p_i)$ such that $\sum_{j=0}^\infty x_{q_iq_j} \rightarrow 0$ as $i \rightarrow \infty$.}
\begin{proof}
	\pf\ \stepref{4}
\end{proof}
\step{8}{For all $i$ we have $x_{q_iq_j} \rightarrow 0$ as $j \rightarrow \infty$.}
\begin{proof}
	\pf\ Since $\sum_{j=0}^\infty x_{q_iq_j} < \infty$.
\end{proof}
\step{9}{For all $j$ we have $x_{q_iq_j} \rightarrow 0$ as $i \rightarrow \infty$.}
\begin{proof}
	\pf\ \stepref{3}
\end{proof}
\step{10}{\pick\ a subsequence $(r_i)$ of $(q_i)$ such that, for all $i$, $j$ with $i \neq j$, we have
\[ \| x_{r_ir_j} \| < \epsilon / 2^{j+2} \enspace . \]}
\begin{proof}
	\step{a}{\pflet{$r_1 = q_1$}}
	\step{b}{\assume{as induction hypothesis we have defined $(r_1, \ldots, r_n)$ such that:
	\begin{itemize}
	\item $r_1 < r_2 < \cdots < r_n$
	\item Whenever $q_i > r_j$ then $\| x_{q_ir_j} \| < \epsilon / 2^{j+2}$
	\item Whenever $i < j$ then $\| x_{r_ir_j} \| < \epsilon / 2^{j+2}$
\end{itemize}}}
	\step{c}{\pflet{$r_{n+1}$ be the first element of $(q_i)$ such that $r_{n+1} > r_n$, $\| x_{q_ir_{n+1}} \| < \epsilon / 2^{n+2}$ whenever $q_i \geq r_{n+1}$, and $\| x_{r_jr_{n+1}} \| < \epsilon / 2^{n+3}$ for $j = 1, \ldots, n$.}}
	\begin{proof}
		\pf\ One exists by \stepref{8}.
	\end{proof}
	\step{d}{$r_1 < \cdots < r_n < r_{n+1}$}
	\step{e}{Whenever $q_i > r_j$ then $\| x_{q_ir_j} \| < \epsilon / 2^{j+2}$}
	\step{f}{Whenever $i < j$ then $\| x_{r_ir_j} \| < \epsilon / 2^{j+2}$}
\end{proof}
\step{11}{\pick\ a subsequence $(s_i)$ of $(r_i)$ such that $\sum_{j=0}^\infty x_{s_is_j} \rightarrow 0$ as $i \rightarrow \infty$}
\begin{proof}
	\pf\ \stepref{4}
\end{proof}
\step{12}{$\forall i \in \mathbb{N}. \left\| \sum_{j=0}^\infty x_{s_is_j} \right\| > \epsilon /2$}
\begin{proof}
	\pf
	\begin{align*}
		\left\| \sum_{j=0}^\infty x_{s_is_j} \right\|
		& = \left\| x_{s_is_i} + \sum_{j \neq i} x_{s_is_j} \right\| \\
		& \geq \left| \| x_{s_is_i} \| - \left\| \sum_{j \neq i} x_{s_is_j} \right\| \right| & (\text{Proposition \ref{prop:distancebetweennorms}}) \\
		& = \| x_{s_is_i} \| - \left\| \sum_{j \neq i} x_{s_is_j} \right\| & \left( \| x_{s_is_i} \| > \epsilon, \left\| \sum_j x_{s_is_j} \right\| \leq \epsilon /2\right) \\
		& \geq  \| x_{s_is_i} \| - \sum_{j \neq i} \| x_{s_is_j} \| & (\text{Triangle Inequality, }) \\
		& > \epsilon - \sum_{j \neq i} \epsilon / 2^{j+2} & (\text{\stepref{6}, \stepref{10}}) \\
		& > \epsilon / 2
	\end{align*}
\end{proof}
\qedstep
\begin{proof}
	\pf\ This contradicts \stepref{11}.
\end{proof}
\qed
\end{proof}

\subsection{Functionals}

\begin{df}[Functional]
Let $K$ be either $\mathbb{R}$ or $\mathbb{C}$. Let $V$ be a normed space over $K$. A \emph{functional} over $V$ is a bounded linear transformation $V \rightarrow K$. The \emph{dual space} of $V$ is
\[ V' := \mathcal{B}(V,K) \enspace . \]
\end{df}

\subsection{Contraction Mappings}

\begin{df}[Contraction]
Let $V$ be a normed space and $A \subseteq V$. Let $f : A \rightarrow V$. Then $f$ is a \emph{contraction} iff there exists a real number $\alpha$ with $0 < \alpha < 1$ such that, for all $x,y \in A$, we have
\[ \| f(x) - f(y) \| \leq \alpha \| x - y \| \enspace . \]
\end{df}

\begin{prop}
Every contraction mapping is continuous.
\end{prop}

\section{Banach Spaces}

\begin{df}[Cauchy Sequence]
A sequence $(x_n)$ in a normed space is a \emph{Cauchy sequence} iff, for every $\epsilon > 0$, there exists $N$ such that $\forall m,n \geq N. \| x_m - x_n \| < \epsilon$.
\end{df}

\begin{thm}
\label{thm:Cauchy}
Let $V$ be a normed space. Let $(x_n)$ be a sequence in $V$. Then the following are equivalent.
\begin{enumerate}
\item $(x_n)$ is Cauchy.
\item For every pair of strictly increasing sequences of natural numbers $(p_n)$ and $(q_n)$, we have $\| x_{p_n} - x_{q_n} \| \rightarrow 0$ as $n \rightarrow \infty$.
\item For every strictly increasing sequence of natural numbers $(p_n)$, we have $\| x_{p_{n+1}} - x_{p_n} \| \rightarrow 0$ as $n \rightarrow \infty$.
\end{enumerate}
\end{thm}

\begin{proof}
\pf
\step{1}{$1 \Rightarrow 2$}
\begin{proof}
	\step{a}{\assume{$(x_n)$ is Cauchy.}}
	\step{b}{\pflet{$(p_n)$ and $(q_n)$ be a pair of increasing sequences of natural numbers.}}
	\step{c}{\pflet{$\epsilon > 0$}}
	\step{d}{\pick\ $N$ such that $\forall m,n \geq N. \| x_m - x_n \| < \epsilon$}
	\step{e}{$\forall n \geq N. p_n, q_n \geq N$}
	\step{f}{$\forall n \geq N. \| x_{p_n} - x_{q_n} \| < \epsilon$}
\end{proof}
\step{2}{$2 \Rightarrow 3$}
\begin{proof}
	\pf\ Trivial.
\end{proof}
\step{3}{$2 \Rightarrow 1$}
\begin{proof}
	\step{a}{\assume{$(x_n)$ is not Cauchy.}}
	\step{b}{\pick\ $\epsilon > 0$ such that, for all $N$, there exist $m, n \geq N$ such that $\| x_m - x_n \| \geq \epsilon$}
	\step{c}{\pick\ strictly increasing sequences of natural numbers $(p_n)$ and $(q_n)$ such that, for all $n$, $\| x_{p_n} - x_{q_n} \| \geq \epsilon$}
	\begin{proof}
		\step{i}{\assume{as induction hypothesis we have chosen $(p_1, \ldots, p_n)$ and $(q_1, \ldots, q_n)$ strictly increasing such that $\forall i. \| x_{p_i} - x_{q_i} \| \geq \epsilon$}}
		\step{ii}{\pick\ $p_{n+1}, q_{n+1} \geq \max(p_n,q_n)$ such that $\| x_{p_{n+1}} - x_{q_{n+1}} \| \geq \epsilon$}
	\end{proof}
	\step{d}{2 is false.}
\end{proof}
\step{4}{$3 \Rightarrow 2$}
\begin{proof}
	\step{a}{\assume{$(p_n)$ and $(q_n)$ are strictly increasonig sequences such that $\| x_{p_n} - x_{q_n} \| \not\rightarrow 0$ as $n \rightarrow \infty$.}}
	\step{b}{\pick\ $\epsilon > 0$ such that, for all $N$, there exists $n \geq N$ such that $\| x_{p_n} - x_{q_n} \| \geq \epsilon$}
	\step{c}{Choose a strictly increasing sequence $(r_n)$ such that, for all $n$, we have $\| x_{r_{2n}} - x_{r_{2n+1}} \| \geq \epsilon$}
	\begin{proof}
		\step{i}{\assume{as induction hypothesis we have chosen $(r_0, r_1, \ldots, r_{2n+1})$ such that, for $i = 0, 1, \ldots, n$, we have $\| x_{r_{2i}} - x_{r_{2i+1}} \| \geq \epsilon$}}
		\step{ii}{\pick\ $i,j \geq r_{2n+1}$ such that $\| x_i - x_j \| \geq \epsilon$}
		\step{iii}{Set $r_{2n+2} = \min(i,j)$ and $r_{2n+3} = \max(i,j)$}
	\end{proof}
	\step{d}{3 is false.}
\end{proof}
\qed
\end{proof}

\begin{prop}
Every convergent sequence is a Cauchy sequence.
\end{prop}

\begin{proof}
\pf
\step{1}{\pflet{$(x_n)$ be a convergent sequence in a normed space $V$ with limit $l$.}}
\step{2}{\pflet{$\epsilon > 0$}}
\step{3}{\pick\ $N$ such that $\forall n \geq N. \| x_n - l \| < \epsilon / 2$}
\step{4}{$\forall m,n \geq N. \| x_m - x_n \| < \epsilon$}
\qed
\end{proof}

\begin{prop}
Let $\mathcal{P}([0,1])$ be the space of polynomials on $[0,1]$ under the norm of uniform convergence. For $n \in \mathbb{N}$, let $P_n = 1 + x + x^2 / 2! + \cdots + x^n / n!$. Then $(P_n)$ is Cauchy but does not converge, since $e^x$ is not a polynomial.
\end{prop}

\begin{proof}
\pf\ Easy. \qed
\end{proof}

\begin{prop}
If $(x_n)$ is a Cauchy sequence in a normed space $V$, then $(\| x_n \|)$ converges in $\mathbb{R}$.
\end{prop}

\begin{proof}
\pf
\step{1}{$(\|x_n\|)$ is Cauchy.}
\begin{proof}
	\step{a}{\pflet{$\epsilon > 0$}}
	\step{b}{\pick\ $N$ such that $\forall m,n \geq N. \| x_m - x_n \| < \epsilon$}
	\step{c}{$\forall m,n \geq N. | \| x_m \| - \| x_n \| | < \epsilon$}
	\begin{proof}
		\pf\ Proposition \ref{prop:distancebetweennorms}.
	\end{proof}
\end{proof}
\qedstep
\begin{proof}
	\pf\ Since $\mathbb{R}$ is complete.
\end{proof}
\qed
\end{proof}

\begin{cor}
Every Cauchy sequence is bounded.
\end{cor}

\begin{df}[Banach space]
A normed space is \emph{complete} or a \emph{Banach space} iff every Cauchy sequence converges.
\end{df}

\begin{prop}
For all $p \geq 1$, the space $l^p$ in complete.
\end{prop}

\begin{proof}
\pf
\step{1}{\pflet{$(a_n)$ be a Cauchy sequence in $l^p$.}}
\step{2}{For $n \in \mathbb{N}$, \pflet{$a_n = (\alpha_{n0}, \alpha_{n1}, \ldots)$}}
\step{3}{For all $\epsilon > 0$, there exists $N$ such that $\forall m,n \geq N$
\[ \sum_{k=0}^\infty | \alpha_{mk} - \alpha_{nk} |^p < \epsilon \enspace . \]}
\begin{proof}
	\pf\ \stepref{1}
\end{proof}
\step{4}{For all $k \in \mathbb{N}$ and $\epsilon > 0$, there exists $N$ such that $\forall m,n \geq N$
\[ | \alpha_{mk} - \alpha_{nk} | < \epsilon \enspace . \]}
\begin{proof}
	\pf\ \stepref{3}
\end{proof}
\step{5}{For all $k \in \mathbb{N}$, the sequence $(\alpha_{nk})_n$ converges in $\mathbb{C}$.}
\begin{proof}
	\pf\ $\mathbb{C}$ is complete.
\end{proof}
\step{6}{For $k \in \mathbb{N}$, \pflet{
\[ \alpha_k = \lim_{n \rightarrow \infty} \alpha_{nk} \enspace . \]}}
\step{7}{\pflet{$a = (\alpha_k)_k$}}
\step{77}{For all $\epsilon > 0$, there exists $N$ such that $\forall n \geq N$
\[ \sum_{k=0}^\infty | \alpha_k - \alpha_{nk} |^p < \epsilon \enspace . \]}
\begin{proof}
	\pf\ Take the limit $m \rightarrow \infty$ in \stepref{4}.
\end{proof}
\step{8}{$a \in l^p$}
\begin{proof}
	\step{a}{\pick\ $N$ such that $\forall n \geq N. \sum_{k=0}^\infty |\alpha_k - \alpha_{nk} |^p < 1$}
	\step{b}{$a - a_N \in l^p$}
	\qedstep
	\begin{proof}
		\pf\ Since $l^p$ is closed under $+$.
	\end{proof}
\end{proof}
\step{9}{$a_n \rightarrow a$ as $n \rightarrow \infty$.}
\begin{proof}
	\pf\ Immediate from \stepref{77}.
\end{proof}
\qed
\end{proof}

\begin{prop}
For any real number $a$, $b$ with $a < b$, the space $\mathcal{C}([a,b])$ is complete.
\end{prop}

\begin{proof}
\pf
\step{1}{\pflet{$(f_n)$ be a Cauchy sequence in $\mathcal{C}([a,b])$}}
\step{2}{For all $\epsilon > 0$, there exists $N$ such that, for all $m,n \geq N$ and $x \in [a,b]$,
\[ |f_m(x) - f_n(x)| < \epsilon \enspace . \]}
\step{3}{For all $x \in [a,b]$, $(f_n(x))_n$ is Cauchy.}
\step{4}{\pflet{$f : [a,b] \rightarrow \mathbb{C}$ be the function
\[ f(x) = \lim_{n \rightarrow \infty} f_n(x) \enspace . \]}}
\step{5}{For all $\epsilon > 0$, there exists $N$ such that, for all $n \geq N$ and $x \in [a,b]$,
\[ |f_n(x) - f(x)| < \epsilon \enspace . \]}
\begin{proof}
	\pf\ Take the limit $m \rightarrow \infty$ in \stepref{2}.
\end{proof}
\step{6}{$f$ is continuous.}
\begin{proof}
	\step{a}{\pflet{$x_0 \in [a,b]$}}
	\step{b}{\pflet{$\epsilon > 0$}}
	\step{c}{\pick\ $N$ such that, for all $n \geq N$ and $y \in [a,b]$, we have
	\[ |f_n(y) - f(y)| < \epsilon/3 \enspace . \]}
	\begin{proof}
		\pf\ \stepref{5}
	\end{proof}
	\step{d}{\pick\ $\delta > 0$ such that, for all $y \in [a,b]$ with $|x_0 - y| < \delta$, we have $|f_N(x_0) - f_N(y)| < \epsilon / 3$}
	\step{e}{For all $y \in [a,b]$, if $|x_0 - y| < \delta$ then $|f(x_0) - f(y)| < \epsilon$}
	\begin{proof}
		\pf
		\begin{align*}
			& |f(x_0) - f(y)| \\
			\leq & |f(x_0) - f_N(x_0)| + |f_N(x_0) - f_N(y)| + |f_N(y) - f(y)| & (\text{Triangle Inequality}) \\
			< & \epsilon / 3 + \epsilon / 3 + \epsilon / 3 & (\text{\stepref{c}, \stepref{d}}) \\
			= & \epsilon
		\end{align*}
	\end{proof}
\end{proof}
\step{7}{$f_n \rightarrow f$ as $n \rightarrow \infty$.}
\begin{proof}
	\pf\ Immediate from \stepref{5}.
\end{proof}
\qed
\end{proof}

\begin{df}[Convergent Series]
Let $(x_n)$ be a sequence in a normed space $V$. We say the series $\sum_{n=0}^\infty x_n$ is \emph{convergent} iff the sequence $(\sum_{n=0}^N x_n)_N$ is convergent, and then we write $\sum_{n=0}^\infty x_n = l$ for $\lim_{N \rightarrow \infty} \sum_{n=0}^N x_n = l$.
\end{df}

\begin{df}[Absolutely Convergent Series]
Let $(x_n)$ be a sequence in a normed space $V$. We say the series $\sum_{n=0}^\infty x_n$ is \emph{absolutely convergent} iff the series $\sum_{n=0}^\infty \| x_n \|$ converges in $\mathbb{R}$.
\end{df}

\begin{thm}
\label{thm:completeabsconv}
A normed space is complete if and only if every absolutely convergent series is convergent.
\end{thm}

\begin{proof}
\pf
\step{1}{\pflet{$V$ be a normed space.}}
\step{2}{If $V$ is complete then every absolutely convergent series is convergent.}
\begin{proof}
	\step{a}{\assume{$V$ is complete.}}
	\step{b}{\pflet{$(x_n)$ be a sequence in $V$.}}
	\step{c}{\assume{$\sum_{n=0}^\infty \| x_n \|$ converges.} \prove{$(\sum_{n=0}^N x_n)_N$ is Cauchy.}}
	\step{d}{\pflet{$\epsilon > 0$}}
	\step{e}{\pick\ $N$ such that $\sum_{n=N}^\infty \| x_n \| < \epsilon$}
	\step{f}{\pflet{$m > n \geq N$}}
	\step{g}{$\| \sum_{k=0}^m x_k - \sum_{k=0}^n x_k \| < \epsilon$}
	\begin{proof}
		\pf
		\begin{align*}
			\| \sum_{k=0}^m x_k - \sum_{k=0}^n x_k \| & = \| \sum_{k=n+1}^m x_k \| \\
			& \leq \sum_{k=n+1}^m \| x_k \| \\
			& \leq \sum_{k=n+1}^\infty \| x_k \| \\
			& < \epsilon
		\end{align*}
	\end{proof}
\end{proof}
\step{3}{If every absolutely convergent series is convergent then $V$ is complete.}
\begin{proof}
	\step{a}{\assume{Every absolutely convergent series is convergent.}}
	\step{b}{\pflet{$(x_n)$ be a Cauchy sequence in $V$.}}
	\step{c}{Choose an increasing sequence of natural numbers $(p_k)$ such that, for all $m, n \geq p_k$, we have
	\[ \| x_m - x_n \| < 2^{-k} \enspace . \]}
	\step{d}{$\sum_{k=0}^\infty \| x_{p_{k+1}} - x_{p_k} \|$ is absolutely convergent.}
	\step{e}{$\sum_{k=0}^\infty \| x_{p_{k+1}} - x_{p_k} \|$ is convergent.}
	\step{f}{$(x_{p_k})$ converges.}
	\step{g}{\pflet{$l = \lim_{k \rightarrow \infty} x_{p_k}$}}
	\step{h}{$x_n \rightarrow l$ as $n \rightarrow \infty$.}
	\begin{proof}
		\pf
		\begin{align*}
			\| x_n - l \| & \leq \| x_n - x_{p_n} \| + \| x_{p_n} - l \| \\
			& \rightarrow 0 & \text{as } n \rightarrow \infty & (\text{Theorem \ref{thm:Cauchy}})
		\end{align*}
	\end{proof}
\end{proof}
\qed
\end{proof}

\begin{prop}
A closed subspace of a Banach space is a Banach space.
\end{prop}

\begin{proof}
\pf
\step{1}{\pflet{$V$ be a Banach space.}}
\step{2}{\pflet{$U$ be a closed subspace of $V$.}}
\step{3}{\pflet{$(x_n)$ be a Cauchy sequence in $U$.}}
\step{4}{$(x_n)$ is a Cauchy sequence in $V$.}
\step{5}{\pflet{$l$ be the limit of $(x_n)$ in $V$.}}
\step{6}{$l \in U$}
\begin{proof}
	\pf\ Proposition \ref{prop:closedconverge}.
\end{proof}
\qed
\end{proof}

\begin{df}[Completion]
Let $V$ be a normed space. A \emph{completion} of $V$ consists of a Banach space $W$ and a function $\phi : V \rightarrow W$ such that:
\begin{itemize}
\item $\phi$ is injective.
\item $\phi$ is a linear transformation.
\item $\phi$ preserves the norm.
\item $\phi(V)$ is dense in $W$.
\end{itemize}
\end{df}

\begin{df}[Equivalent Cauchy Sequences]
Let $V$ be a normed space. Let $(x_n)$ and $(y_n)$ be Cauchy sequences in $V$. Then $(x_n)$ and $(y_n)$ are \emph{equivalent}, $(x_n) \sim (y_n)$, iff $\| x_n - y_n \| \rightarrow 0$ as $n \rightarrow \infty$.
\end{df}

\begin{prop}
Equivalence is an equivalence relation on the set of Cauchy sequences.
\end{prop}

\begin{prop}
If $(x_n) \sim (y_n)$ then $\lim_{n \rightarrow \infty} \| x_n \| = \lim_{n \rightarrow \infty} \| y_n \|$.
\end{prop}

\begin{thm}
Let $V$ be a normed space. Let $W$ be the quotient set of all Cauchy sequences modulo $\sim$. Define $+$, $\cdot$ and $\|\ \|$ on $W$ by
\begin{align*}
[(x_n)] + [(y_n)] & = [(x_n + y_n)] \\
\lambda [(x_n)] & = [(\lambda x_n)] \\
\| [(x_n)] \| & = \lim_{n \rightarrow \infty} \| x_n \|
\end{align*}
Define $\phi : V \rightarrow W$ by $\phi(x)$ is the constant sequence $(x)$. Then $\phi : V \rightarrow W$ is the completion of $V$.
\end{thm}

\begin{proof}
\pf
\step{1}{$+$, $\cdot$ and $\|\ \|$ are well defined.}
\step{2}{$W$ is a normed space.}
\step{7}{$\phi(V)$ is dense in $W$.}
\begin{proof}
	\pf\ For any $[(x_n)] \in W$ we have $[(x_n)] = \lim_{n \rightarrow \infty} \phi(x_n)$.
\end{proof}
\step{3}{$W$ is complete.}
\begin{proof}
	\step{a}{\pflet{$(X_n)$ be a Cauchy sequence in $W$.}}
	\step{b}{For all $n \in \mathbb{N}$, \pick\ $x_n \in V$ such that $\| \phi(x_n) - X_n \| < 1/n$}
	\begin{proof}
		\pf\ \stepref{7}
	\end{proof}
	\step{c}{For all $m$, $n$ we have $\| x_n - x_m \| \leq \| X_n - X_m \| + 1/n + 1/m$}
	\begin{proof}
		\pf
		\begin{align*}
			\| x_n - x_m \| & = \| \phi(x_n) - \phi(x_m) \| \\
			& \leq \| \phi(x_n) - X_n \| + \| X_n - X_m \| + \| \phi(x_m) - X_m \| \\
			& \leq \| X_n - X_m \| + 1/n + 1/m
		\end{align*}
	\end{proof}
	\step{d}{$(x_n)$ is a Cauchy sequence in $V$.}
	\step{e}{\pflet{$X = [(x_n)]$} \prove{$X_n \rightarrow X$ as $n \rightarrow \infty$}}
	\step{f}{$\| X_n - X \| \rightarrow 0$ as $n \rightarrow \infty$}
	\begin{proof}
		\pf
		\begin{align*}
			\| X_n - X \| & \leq \| X_n - \phi(x_n) \| + \| \phi(x_n) - X \| \\
			& < \| \phi(x_n) - X \| + 1/n \\
			& \rightarrow 0
		\end{align*}
	\end{proof}
\end{proof}
\step{4}{$\phi$ is injective.}
\step{5}{$\phi$ is a linear transformation.}
\step{6}{$\phi$ preserves the norm.}
\qed
\end{proof}

\begin{thm}
Let $U$ be a normed space and $V$ a Banach space. Then $\mathcal{B}(U,V)$ is a Banach space.
\end{thm}

\begin{proof}
\pf
\step{1}{\pflet{$(L_n)$ be a Cauchy sequence in $\mathcal{B}(U,V)$}}
\step{2}{For all $x \in U$, we have $(L_n(x))$ is a Cauchy sequence in $V$.}
\begin{proof}
	\step{x}{\pflet{$x \in U$}}
	\step{xx}{\assume{w.l.o.g. $x \neq 0$}}
	\step{a}{\pflet{$\epsilon > 0$}}
	\step{b}{\pick\ $N$ such that $\forall m,n \geq N. \| L_m - L_n \| < \epsilon / \| x \|$}
	\step{c}{$\forall m,n \geq N. \| L_m(x) - L_n(x) \| < \epsilon$}
	\begin{proof}
		\pf\ $\| L_m(x) - L_n(x) \| \leq \| L_m - L_n \| \| x \| < \epsilon$
	\end{proof}
\end{proof}
\step{3}{Define $L : U \rightarrow V$ by $L(x) = \lim_{n \rightarrow \infty} L_n(x)$}
\step{4}{$L \in \mathcal{B}(U,V)$}
\begin{proof}
	\step{a}{$L$ is linear.}
	\begin{proof}
		\step{i}{\pflet{$\lambda, \mu \in K$ and $x,y \in U$}}
		\step{ii}{$L(\lambda x + \mu y) = \lambda L(x) + \mu L(y)$}
		\begin{proof}
			\pf
			\begin{align*}
				L(\lambda x + \mu y) & = \lim_{n \rightarrow \infty} L_n(\lambda x + \mu y) \\
				& = \lim_{n \rightarrow \infty} (\lambda L_n(x) + \mu L_n(y)) \\
				& = \lambda L(x) + \mu L(y)
			\end{align*}
		\end{proof}
	\end{proof}
	\step{b}{$L$ is bounded.}
	\begin{proof}
		\step{i}{\pick\ $N$ such that $\forall m,n \geq N. \| L_m - L_n \| < 1$ \prove{$\forall x \in U. \| L(x) \| \leq (\| L_N \| + 1) \| x \|$}}
		\step{x}{$\forall n \geq N. \| L_n \| \leq \| L_N \| + 1$}
		\begin{proof}
			\pf\ Since $| \| L_n \| - \| L_N \| | \leq \| L_n - L_N \| < 1$.
		\end{proof}
		\step{ii}{\pflet{$x \in U$}}
		\step{iii}{$\| L(x) \| \leq (\| L_N \| + 1) \| x \|$}
		\begin{proof}
			\pf
			\begin{align*}
				\| L(x) \| & = \left\| \lim_{n \rightarrow \infty} L_n(x) \right\| \\
				& = \lim_{n \rightarrow \infty} \| L_n(x) \| \\
				& \leq \lim_{n \rightarrow \infty} \| L_n \| \| x \| \\
				& \leq (\| L_N \| + 1) \| x \|
			\end{align*}
		\end{proof}
	\end{proof}
\end{proof}
\step{5}{$L_n \rightarrow L$ as $n \rightarrow \infty$}
\begin{proof}
	\step{a}{\pflet{$\epsilon > 0$}}
	\step{b}{\pick\ $N$ such that $\forall m,n \geq N. \| L_m - L_n \| < \epsilon / 4$}
	\step{c}{\pflet{$n \geq N$}}
	\step{d}{For all $x \in U$ we have $\| L_n(x) - L(x) \| < (\epsilon / 2) \| x \|$}
	\begin{proof}
		\step{i}{\pflet{$x \in U$}}
		\step{ii}{For all $m \geq N$ we have $\| L_n(x) - L_m(x) \| < (\epsilon / 4) \| x \|$}
		\step{iii}{$\| L_n(x) - L(x) \| \leq (\epsilon / 4) \| x \|$}
		\begin{proof}
			\pf\ Taking the limit as $m \rightarrow \infty$.
		\end{proof}
	\end{proof}
	\step{e}{$\| L_n - L \| \leq \epsilon / 2$}
	\step{f}{$\| L_n - L \| < \epsilon$}
\end{proof}
\qed
\end{proof}

\begin{cor}
The dual space of a normed space is a Banach space.
\end{cor}

\begin{thm}
Let $U$ be a normed space and $V$ a Banach space. Let $W$ be a subspace of $U$. Let $L : W \rightarrow V$ be a bounded linear transformation. Then $L$ has a unique extension to a bounded linear transformation $\overline{W} \rightarrow V$.
\end{thm}

\begin{proof}
\pf
\step{1}{Define $L' : \overline{W} \rightarrow V$ as follows. Given $x \in \overline{W}$, pick a sequence $(x_n)$ in $W$ that converges to $x$. Then $L'(x) = \lim_{n \rightarrow \infty} L(x_n)$}
\begin{proof}
	\step{a}{For all $x \in \overline{W}$, there exists a sequence $(x_n)$ in $W$ that converges to $x$.}
	\begin{proof}
		\pf\ Theorem \ref{thm:closureaslimits}.
	\end{proof}
	\step{b}{For any sequence $(x_n)$ in $W$ that converges in $\overline{W}$, we have $(L(x_n))$ converges in $V$.}
	\begin{proof}
		\step{i}{\pflet{$(x_n)$ be a sequence in $W$ that converges in $\overline{W}$}}
		\step{ii}{$(x_n)$ is Cauchy.}
		\step{iii}{$(L(x_n))$ is Cauchy.}
		\begin{proof}
			\pf\ For any strictly increasing sequence of natural numbers $(p_n)$, we have $\| L(x_{p_{n+1}}) - L(x_{p_n}) \| \leq \| L \| \| x_{p_{n+1}} - x_{p_n} \| \rightarrow 0$ as $n \rightarrow \infty$.
		\end{proof}
		\qedstep
		\begin{proof}
			\pf\ $W$ is a Banach space.
		\end{proof}
	\end{proof}
	\step{c}{If $(x_n)$ and $(y_n)$ are sequences in $W$ that converge to the same point in $\overline{W}$, then $\lim_{n \rightarrow \infty} L(x_n) = \lim_{n \rightarrow \infty} L(y_n)$}
	\begin{proof}
		\pf\ Since $\| L(x_n) - L(y_n) \| \leq \| L \| \| x_n - y_n \| \rightarrow 0$.
	\end{proof}
\end{proof}
\step{2}{$L'$ extends $L$}
\begin{proof}
	\pf\ For $x \in W$ we have the constant sequence $(x)$ converges to $x$, and the constant sequence $(L(x))$ converges to $L(x)$, so $L'(x) = L(x)$.
\end{proof}
\step{3}{$L'$ is a linear transformation.}
\begin{proof}
	\step{a}{\pflet{$\lambda, \mu \in K$ and $x,y \in \overline{W}$}}
	\step{b}{\pick\ sequences $(x_n)$, $(y_n)$ in $W$ that converge to $x$ and $y$ respectively.}
	\step{c}{$\lambda x_n + \mu y_n \rightarrow \lambda x + \mu y$}
	\qedstep
	\begin{proof}
		\pf
		\begin{align*}
			L'(\lambda x + \mu y) & = \lim_{n \rightarrow \infty} L(\lambda x_n + \mu y_n) \\
			& = \lim_{n \rightarrow \infty} (\lambda L(x_n) + \mu L(y_n)) \\
			& = \lambda L'(x) + \mu L'(y)
		\end{align*}
	\end{proof}
\end{proof}
\step{4}{$L'$ is bounded.}
\begin{proof}
	\step{a}{\pflet{$x \in \overline{W}$}}
	\step{b}{\pick\ a sequence $(x_n)$ in $W$ that converges to $x$.}
	\step{c}{$\| L'(x) \| \leq \| L \| \| x \|$}
	\begin{proof}
		\pf
		\begin{align*}
			\| L'(x) \| & = \| \lim_{n \rightarrow \infty} L(x_n) \| \\
			& = \lim_{n \rightarrow \infty} \| L(x_n) \| \\
			& \leq \|L\| \lim_{n \rightarrow \infty} \| x_n \| \\
			& = \| L \| \| x \|
		\end{align*}
	\end{proof}
\end{proof}
\step{5}{If $L'' : \overline{W} \rightarrow V$ is a bounded linear transformation that extends $L$, then $L'' = L'$.}
\begin{proof}
	\step{a}{\pflet{$x \in \overline{W}$}}
	\step{b}{\pick\ a sequence $(x_n)$ in $W$ that converges to $x$.}
	\step{c}{$L''(x) = L'(x)$}
	\begin{proof}
		\pf
		\begin{align*}
			L''(x) & = \lim_{n \rightarrow \infty} L''(x_n) \\
			& = \lim_{n \rightarrow \infty} L(x_n) \\
			& = L'(x_n)
		\end{align*}
	\end{proof}
\end{proof}
\qed
\end{proof}

\begin{thm}[Banach-Steinhaus]
Let $X$ be a Banach space and $Y$ a normed space. Let $\mathcal{T}$ be a set of bounded linear transformations from $X$ into $Y$. Assume that, for all $x \in X$, there exists $M_x > 0$ such that, for all $T \in \mathcal{T}$, we have $\| T(x) \| \leq M_x$. Then there exists $M > 0$ such that, for all $T \in \mathcal{T}$, we have $\| T \| \leq M$.
\end{thm}

\begin{proof}
\pf
\step{1}{\pflet{$X$ be a Banach space.}}
\step{2}{\pflet{$Y$ be a normed space.}}
\step{3}{\pflet{$\mathcal{T}$ be a set of bounded linear transformations.}}
\step{4}{\assume{For all $x \in X$, there exists $M_x > 0$ such that, for all $T \in \mathcal{T}$, we have $\| T(x) \| \leq M_x$.}}
\step{5}{\assume{for a contradiction there is no $M > 0$ such that, for all $T \in \mathcal{T}$, we have $\| T \| \leq M$}}
\step{6}{For every positive integer $n$, choose $T_n \in \mathcal{T}$ such that $\| T_n \| > n2^n$.}
\step{7}{For every positive integer $n$, choose $x_n \in U$ such that $\| x_n \| = 1$ and $\| T_n(x_n) \| > n2^n$.}
\step{8}{For every positive integer $n$ we have
\[ \left\| \frac{1}{n} T_n \left( \frac{x_n}{2^n} \right) \right\| > 1 \enspace . \]}
\step{9}{For positive integers $i$ and $j$, \pflet{\[ y_{ij} = \frac{1}{i} T_i(\frac{x_j}{2^j}) \]}}
\step{10}{\pflet{$z = \sum_{j=1}^\infty \frac{x_j}{2^j}$}}
\begin{proof}
	\step{a}{$\sum_{j=1}^\infty \frac{x_j}{2^j}$ is absolutely convergent.}
	\begin{proof}
		\pf
		\begin{align*}
			\sum_{j=1}^\infty \left\| \frac{x_j}{2^j} \right\| & = \sum_{j=1}^\infty \frac{1}{2^j} & (\text{\stepref{7}}) \\
			& = 1
		\end{align*}
	\end{proof}
	\qedstep
	\begin{proof}
		\pf\ Theorem \ref{thm:completeabsconv}.
	\end{proof}
\end{proof}
\step{11}{\pick\ $C > 0$ such that, for all $i$, we have $\left\| \sum_{j=1}^\infty y_{ij} \right\| \leq C / i$.}
\begin{proof}
	\step{a}{\pick\ $C > 0$ such that, for all $T \in \mathcal{T}$, we have $\| T(z) \| \leq C$.}
	\begin{proof}
		\pf\ \stepref{4}.
	\end{proof}
	\step{b}{For all $i$ we have $\left\| \sum_{j=1}^\infty y_{ij} \right\| \leq C / i$}
	\begin{proof}
		\pf
		\begin{align*}
			\left\| \sum_{j=1}^\infty y_{ij} \right\|
			& = \left\| \sum_{j=1}^\infty \frac{1}{i} T_i \left( \frac{x_j}{2^j} \right) \right\| & (\text{\stepref{9}}) \\
			& = \frac{1}{i} \left\| T_i \left( \sum_{j=1}^\infty \frac{x_j}{2^j} \right) \right\| & (T_i \text{ continuous by \stepref{3}}) \\
			& = \frac{1}{i} \| T_i(z) \| & (\text{\stepref{10}}) \\
			& \leq \frac{C}{i} & (\text{\stepref{a}})
		\end{align*}
	\end{proof}
\end{proof}
\step{12}{$\sum_{j=1}^\infty y_{ij} \rightarrow 0$ as $i \rightarrow \infty$}
\step{13}{For any increasing sequence of positive integers $(q_i)$, we have $\sum_{j=0}^\infty y_{q_iq_j} \rightarrow 0$ as $i \rightarrow \infty$}
\begin{proof}
	\pf\ Similar.
\end{proof}
\step{14}{For all $j$ we have $y_{ij} \rightarrow 0$ as $i \rightarrow \infty$}
\begin{proof}
	\pf\ From \stepref{9} and the fact that $T_i$ is continuous.
\end{proof}
\step{15}{$y_{ii} \rightarrow 0$ as $i \rightarrow \infty$.}
\begin{proof}
	\pf\ By the Diagonal Theorem.
\end{proof}
\qedstep
\begin{proof}
	\pf\ This contradicts \stepref{8}.
\end{proof}
\qed
\end{proof}

\begin{thm}[Banach Fixed Point Theorem]
Let $V$ be a Banach space. Let $F \subseteq V$ be closed and nonempty. Let $f : F \rightarrow F$ be a contraction. Then there exists a unique $z \in F$ such that $f(z) = z$.
\end{thm}

\begin{proof}
\pf
\step{0}{\pick\ $\alpha$ such that $0 < \alpha < 1$ and $\forall x,y \in V. \| f(x) - f(y) \| \leq \alpha \| x - y \|$.}
\step{1}{\pick\ $x_0 \in F$}
\step{2}{Extend to the sequence $(x_n)$ in $F$ by defining $x_{n+1} := f(x_n)$.}
\step{3}{$(x_n)$ is Cauchy.}
\begin{proof}
	\step{a}{$\forall n \in \mathbb{N}. \| x_{n+1} - x_n \| \leq \alpha^n \| x_1 - x_0 \|$}
	\step{b}{\pflet{$\epsilon > 0$}}
	\step{c}{\pick\ $N$ such that $\frac{\|x_1 - x_0\|}{1 - \alpha} \alpha^N \leq \epsilon$}
	\step{d}{$\forall m,n \geq N. \| x_n - x_m \| < \epsilon$}
	\begin{proof}
		\step{i}{\pflet{$m,n \geq N$}}
		\step{ii}{\assume{w.l.o.g. $m < n$}}
		\step{iii}{$\| x_n - x_m \| < \epsilon$}
		\begin{proof}
			\pf		
			\begin{align*}
				\| x_n - x_m \|
				& \leq \| x_n - x_{n-1} \| + \| x_{n-1} - x_{n-2} \| + \cdots + \| x_{m+1} - x_m \| \\
				& \leq (\alpha^{n-1} + \alpha^{n-2} + \cdots + \alpha^m) \| x_1 - x_0 \| \\
				& < \frac{\|x_1 - x_0\|}{1 - \alpha} \alpha^m \\
				& \leq \frac{\|x_1 - x_0\|}{1 - \alpha} \alpha^N \\
				& \leq \epsilon
			\end{align*}
		\end{proof}
	\end{proof}
\end{proof}
\step{4}{\pflet{$z = \lim_{n \rightarrow \infty} x_n$}}
\step{5}{$z$ is unique such that $f(z) = z$.}
\begin{proof}
	\step{a}{$f(z) = z$}
	\begin{proof}
		\pf
		\begin{align*}
			f(z) & = f \left( \lim_{n \rightarrow \infty} x_n \right) \\
			& = \lim_{n \rightarrow \infty} f(x_n) \\
			& = \lim_{n \rightarrow \infty} x_{n+1} \\
			& = z
		\end{align*}
	\end{proof}
	\step{b}{If $f(w) = w$ then $w = z$.}
	\begin{proof}
		\step{i}{$\| w - z \| \leq \alpha \| w - z \|$}
		\begin{proof}
			\pf\ $\| w - z \| = \| f(w) - f(z) \| \leq \alpha \| w - z \|$
		\end{proof}
		\step{ii}{$\| w - z \| = 0$}
		\begin{proof}
			\pf\ Otherwise $\| w - z \| < \| w - z \|$.
		\end{proof}
		\step{iii}{$w = z$}
	\end{proof}
\end{proof}
\qed
\end{proof}

\section{Inner Product Spaces}

\begin{df}[Inner Product Space]
An \emph{inner product} on a complex vector space $V$ is a function $\langle \ ,\ \rangle : V^2 \rightarrow \mathbb{C}$ such that:
\begin{itemize}
\item $\forall x,y \in V. \langle x,y \rangle = \overline{ \langle y,x \rangle}$
\item $\forall \lambda \in \mathbb{C}. \forall x,y \in V. \langle \lambda x,y \rangle = \lambda \langle x,y \rangle$
\item $\forall x,y,z \in V. \langle x + y, z \rangle = \langle x,z \rangle + \langle y,z \rangle$
\item $\forall x \in V. \langle x,x \rangle = 0 \Rightarrow x = 0$
\end{itemize}
An \emph{inner product space} or \emph{pre-Hilbert space} is a complex vector space with an inner product.
\end{df}

\begin{prop}
$\mathbb{C}^n$ is an inner product space under
\[ \langle (x_1, \ldots, x_n) , (y_1 ,\ldots, y_n) \rangle = x_1 \overline{y_1} + \cdots + x_n \overline{y_n} \enspace . \]
\end{prop}

\begin{prop}
$l^2$ is an inner product under
\[ \langle (x_n),(y_n) \rangle = \sum_{n=0}^\infty x_n \overline{y_n} \enspace . \]
\end{prop}

\begin{proof}
\pf
\step{1}{For all $(x_n),(y_n) \in l^2$ we have $\sum_n x_n \overline{y_n} < \infty$}
\begin{proof}
	\step{a}{\pflet{$(x_n),(y_n) \in l^2$}}
	\step{b}{$\sum_{n=1}^N |x_n \overline{y_n}| \leq \left( \sum_{n=1}^\infty |x_n|^2 \right)^{1/2} \left( \sum_{n=1}^\infty |y_n|^2 \right)^{1/2}$}
	\begin{proof}
		\pf
		\begin{align*}
			\sum_{n=1}^N |x_n \overline{y_n}|
			& = \sum_{n=1}^N |x_n| |y_n| \\
			& \leq \left( \sum_{n=1}^N |x_n|^2 \right)^{1/2} \left( \sum_{n=1}^N |y_n|^2 \right)^{1/2} & (\text{Cauchy-Schwarz}) \\
			&\leq \left( \sum_{n=1}^\infty |x_n|^2 \right)^{1/2} \left( \sum_{n=1}^\infty |y_n|^2 \right)^{1/2}
		\end{align*}
	\end{proof}
	\step{c}{$\sum_{n=1}^N x_n \overline{y_n}$ is absolutely convergent.}
\end{proof}
\step{2}{$\forall (x_n),(y_n) \in l^2. \langle (x_n),(y_n) \rangle = \overline{\langle (y_n),(x_n) \rangle}$}
\step{3}{$\forall \lambda \in \mathbb{C}. \forall (x_n),(y_n) \in l^2. \langle \lambda (x_n), (y_n) \rangle = \lambda \langle (x_n),(y_n) \rangle$}
\step{4}{$\forall (x_n),(y_n),(z_n) \in l^2. \langle (x_n) + (y_n), (z_n) \rangle = \langle (x_n), (z_n) \rangle + \langle (y_n), (z_n) \rangle$}
\step{5}{$\forall (x_n) \in l^2. \langle (x_n), (x_n) \rangle = 0 \Rightarrow (x_n) = 0$}
\qed
\end{proof}

\begin{prop}
The space $C[0,1]$ of all continuous functions $[0,1] \rightarrow \mathbb{C}$ is an inner product space under
\[ \langle f,g \rangle = \int_0^1 f(t) \overline{g(t)} dt \enspace . \]
\end{prop}

\begin{prop}
The space $\mathbb{C}^{mn}$ of all $m \times n$ complex matrices is an inner product space under
\[ \langle A,B \rangle = \tr (B^* A) \]
where $B^*$ is the conjugate transpose of $B$.
\end{prop}

\begin{thm}
Let $V$ be an inner product space. Let $x,y,z \in V$ and $\lambda \in \mathbb{C}$. Then:
\begin{enumerate}
\item $\langle x, y + z \rangle = \langle x,y \rangle + \langle x,z \rangle$
\item $\langle x, \lambda y \rangle = \overline{\lambda} \langle x,y \rangle$
\item $\langle x,0 \rangle = \langle 0,x \rangle = 0$
\item If $\forall w \in V. \langle x,w \rangle = \langle y,w \rangle$ then $x = y$
\item $\langle x,x \rangle$ is a non-negative real.
\end{enumerate}
\end{thm}

\begin{proof}
\pf\ For part 4, take $w = x - y$. \qed
\end{proof}

\begin{thm}[Cauchy-Schwarz Inequality]
Let $V$ be an inner product space. Let $x,y \in V$. Then
\[ |\langle x,y \rangle| \leq \langle x,x \rangle^{1/2} \langle y,y \rangle^{1/2} \enspace . \]
Equality holds if and only if $x$ and $y$ are linearly dependent.
\end{thm}

\begin{proof}
\pf
\step{1}{If $x$ and $y$ are linearly dependent then $|\langle x,y \rangle| = \langle x,x \rangle^{1/2} \langle y,y \rangle^{1/2}$.}
\begin{proof}
	\pf\ If $y = \lambda x$ then both sides are equal to $|\lambda| \langle x,x \rangle$.
\end{proof}
\step{2}{If $x$ and $y$ are linearly independent then $|\langle x,y \rangle| < \langle x,x \rangle^{1/2} \langle y,y \rangle^{1/2}$.}
\begin{proof}
	\step{a}{For any $\lambda \in \mathbb{C}$ with $x + \lambda y \neq 0$ we have
	\[ \langle x,x \rangle + 2 \Re (\overline{\lambda} \langle x,y \rangle) + |\lambda|^2 \langle y,y \rangle > 0 \enspace . \]}
	\begin{proof}
		\pf
		\begin{align*}
			0 & < \langle x + \lambda y, x + \lambda y \rangle \\
			& = \langle x,x \rangle + \overline{\lambda} \langle x,y \rangle + \lambda \langle y,x \rangle + |\lambda|^2 \langle y,y \rangle \\
			& = \langle x,x \rangle + 2 \Re (\overline{\lambda} \langle x,y \rangle) + |\lambda|^2 \langle y,y \rangle
		\end{align*}
	\end{proof}
	\step{b}{\pflet{$u = |\langle x,y \rangle|/\langle x,y \rangle$ or $u = 1$ if $\langle x,y \rangle = 0$}}
	\step{c}{For any $t \in \mathbb{R}$,
	\[ \langle x,x \rangle + 2 |\langle x,y \rangle| t + \langle y,y \rangle t^2 > 0 \]}
	\begin{proof}
		\pf\ Take $\lambda = tu$ in \stepref{a}
	\end{proof}
	\step{d}{\[ 4 |\langle x,y \rangle|^2 - 4 \langle x,x \rangle \langle y,y \rangle < 0 \]}
	\begin{proof}
		\pf\ The quadratic \stepref{c} must have negative discriminant.
	\end{proof}
	\step{e}{\[ |\langle x,y \rangle| < \langle x,x \rangle^{1/2} \langle y,y \rangle^{1/2} \]}
\end{proof}
\qed
\end{proof}

\begin{thm}
Every inner product space is a normed space under $\| x \| = \langle x,x \rangle^{1/2}$.
\end{thm}

\begin{proof}
\pf
\step{1}{If $\| x \| = 0$ then $x = 0$}
\step{2}{$\| \lambda x \| = |\lambda| \| x \| $}
\step{3}{$\| x + y \| \leq \| x \| + \| y \|$}
\begin{proof}
	\pf
	\begin{align*}
		\| x + y \|^2 & = \langle x + y, x + y \rangle \\		
		& = \langle x,x \rangle + \langle x,y \rangle + \langle y,x \rangle + \langle y,y \rangle \\
		& = \| x \|^2 + 2 \Re \langle x,y \rangle + \| y \|^2 \\
		& \leq \| x \|^2 + 2 |\langle x,y \rangle| + \| y \|^2 \\
		& < \| x \|^2 + 2 \|x\| \|y\| + \|y\|^2 & (\text{Cauchy-Schwarz}) \\
		& = (\|x\| + \|y\|)^2
	\end{align*}
\end{proof}
\qed
\end{proof}

\begin{thm}[Parallelogram Law]
Let $V$ be an inner product space. Let $x,y \in V$. Then
\[ \| x + y \|^2 + \| x - y \|^2 = 2 \| x \|^2 + 2 \| y \|^2 \enspace . \]
\end{thm}

\begin{proof}
	\pf
	\begin{align*}
		\| x + y\|^2 & = \| x\|^2 + \langle x,y \rangle + \langle y,x \rangle + \| y\|^2 \\
		\| x - y \|^2 & = \| x\|^2 - \langle x,y \rangle - \langle y,x \rangle + \| y \|^2
	\end{align*}
\end{proof}

\begin{thm}[Polarization Identity]
Let $V$ be an inner product space. Let $x,y \in V$. Then
\[ 4 \langle x,y \rangle = \| x + y \|^2 - \| x - y \|^2 + i \| x + i y \|^2 - i \| x - iy \|^2 \enspace . \]
\end{thm}

\begin{proof}
	\pf\ Straightforward calculation. \qed
\end{proof}

\end{document}