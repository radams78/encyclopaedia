\chapter{Rngs}


\begin{df}[Ring]
A \emph{rng} consists of a set $R$ and binary operations $+, \cdot : R^2 \rightarrow R$ such that:
\begin{itemize}
\item $(R,+)$ is an Abelian group
\item $\cdot$ is associative.
\item The \emph{distributive properties} hold: for all $r,s,t \in R$ we have
\[ (r+s)t = rt + st, \qquad r(s+t) = rs + rt \enspace .\]
\end{itemize}
\end{df}

\begin{ex}
\begin{itemize}
\item The \emph{zero rng} is $\{0\}$.
\item $\mathbb{Z}$, $\mathbb{Q}$, $\mathbb{R}$ and $\mathbb{C}$ are rngs.
\item
$2 \mathbb{Z}$ is a rng.
\item
Given a rng $R$ and natural number $n$, then the set $\gl{n}{R}$ of all $n \times n$ matrices with entries in $R$ is a rng under matrix addition and matrix multiplication.
\item
For any set $S$, the power set $\mathcal{P} S$ is a rng under $A + B = (A \cup B) - (A \cap B)$ and $AB = A \cap B$.
\item
Given a rng $R$ and a set $S$, then $R^S$ is a rng under $(f + g)(s) = f(s) + g(s)$ and $(fg)(s) = f(s)g(s)$ for all $f,g \in R^S$ and $s \in S$.
\item The set $\sl{n}{\mathbb{R}} = \{ M \in \gl{n}{\mathbb{R}} : \tr M = 0 \}$ is a rng.
\item The set $\sl{n}{\mathbb{C}} = \{ M \in \gl{n}{\mathbb{C}} : \tr M = 0 \}$ is a rng.
\item $\mathbb{Z} / n \mathbb{Z}$ is a rng.
\item The ring $\mathbb{H}$ of \emph{quaternions} is $\mathbb{R}^4$ under the following operations, where we write $(a,b,c,d)$ as $a + bi + cj + dk$:
\begin{align*}
(a + bi + cj + dk) + (a' + b'i + c'j + d'k) & = (a + a') + (b + b')i\\
&  + (c + c')j + (d + d')k \\
(a + bi + cj + dk)(a' + b'i + c'j + d'k) & = 
(aa' - bb' - cc' - dd') \\
& + (ab' + ba' + cd' - dc') i \\
& + (ac' - bd' + ca' + db')j \\
& + (ad' + bc' - cb' + da') k
\end{align*}
\item For any Abelian group $G$, the set $\End{\Ab}{G}$ is a ring under pointwise addition and composition.
\end{itemize}
\end{ex}

\begin{prop}
In any rng $R$ we have
\[ \forall x \in R. x0 = 0x = 0 \enspace . \]
\end{prop}

\begin{proof}
\pf
\begin{align*}
x0 & = x(0+0) \\
& = x0+x0
\end{align*}
and so $x0 = 0$ by Cancellation. Similarly $0x = 0$. \qed
\end{proof}

\begin{df}[Zero Divisor]
Let $R$ be a rng and $a \in R$.

Then $a$ is a \emph{left-zero-divisor} iff there exists $b \in R - \{0\}$ such that $ab = 0$.

The element $a$ is a \emph{right-zero-divisor} iff there exists $b \in R - \{0\}$ such that $ba = 0$.
\end{df}

\begin{ex}
0 is a left- and right-zero-divisor in every non-zero rng.

The zero rng is the only ring with no zero-divisors.
\end{ex}

\begin{prop}
Let $R$ be a rng and $a \in R$. Then $a$ is not a left-zero-divisor if and only if left multiplication by $a$ is an injective function $R \rightarrow R$.
\end{prop}

\begin{proof}
\pf
\step{1}{If $a$ is not a left-zero-divisor then left multiplication by $a$ is injective.}
\begin{proof}
	\step{a}{\assume{$a$ is not a left-zero-divisor.}}
	\step{b}{\pflet{$ab = ac$}}
	\step{c}{$a(b-c) = 0$}
	\step{d}{$b - c = 0$}
	\step{e}{$b = c$}
\end{proof}
\step{2}{If $a$ is a left-zero-divisor then left multiplication by $a$ is not injective.}
\begin{proof}
	\step{a}{\pick\ $b \neq 0$ such that $ab = 0$.}
	\step{b}{$ab = a0$ but $b \neq 0$}
\end{proof}
\qed
\end{proof}

\section{Commutative Rngs}

\begin{df}[Commutative]
A rng $R$ is \emph{commutative} iff $\forall x,y \in R. xy = yx$.
\end{df}

\begin{ex}
\begin{itemize}
\item The zero rng is commutative.
\item $\mathbb{Z}$, $\mathbb{Q}$, $\mathbb{R}$ and $\mathbb{C}$ are commutative.
\item
$2 \mathbb{Z}$ is commutative.
\item $\gl{2}{\mathbb{R}}$ is not commutative.
\item For any set $S$, the rng $\mathcal{P} S$ is commutative.
\item If $R$ is commutative then $R^S$ is commutative.
\end{itemize}
\end{ex}

\section{Rng Homomorphisms}

\begin{df}
Let $R$ and $S$ be rngs. A \emph{rng homomorphism} $\phi : R \rightarrow S$ is a function such that, for all $x,y \in R$, we have
\begin{align*}
\phi(x + y) & = \phi(x) + \phi(y) \\
\phi(xy) & = \phi(x) \phi(y)
\end{align*}
Let $\mathbf{Rng}$ be the category of rngs and rng homomorphisms.
\end{df}

\section{Quaternions}

\begin{df}[Norm]
The \emph{norm} of a quaternion is defined by
\[ N(a + bi + cj + dk) = a^2 + b^2 + c^2 + d^2 \enspace . \]
\end{df}

\chapter{Rings}

\begin{df}[Ring]
A \emph{ring} $R$ is a rng such that there exists $1 \in R$, the \emph{multiplicative identity}, such that
\[ \forall x \in R. x1 = 1x = x \enspace . \]
\end{df}

\begin{ex}
\begin{itemize}
\item The zero rng is a ring with $1 = 0$.
\item $\mathbb{Z}$, $\mathbb{Q}$, $\mathbb{R}$ and $\mathbb{C}$ are rngs.
\item $2 \mathbb{Z}$ is not a ring.
\item If $R$ is a ring then $\gl{n}{R}$ is a ring.
\item For any set $S$, the rng $\mathcal{P} S$ is a ring with $1 = S$.
\item If $R$ is a ring then $R^S$ is a ring.
\item $\sl{n}{\mathbb{R}}$ is not a ring for $n > 0$.
\item $\sl{n}{\mathbb{C}}$ is not a ring for $n > 0$.
\item $\so{n}{\mathbb{R}} = \{ M \in \sl{n}{\mathbb{R}} : M + M^T = 0 \}$ is not a ring.
\item $\mathbb{Z} / n \mathbb{Z}$ is a ring.
\end{itemize}
\end{ex}

\begin{prop}
In any ring $R$, if $0 = 1$ then $R$ is the zero ring.
\end{prop}

\begin{proof}
\pf\ For any $x \in R$ we have $x = 1x = 0x = 0$. \qed
\end{proof}

\begin{prop}
In any ring we have $(-1)x = -x$.
\end{prop}

\begin{proof}
\pf\ Since
\begin{align*}
x + (-1)x & = 1x + (-1) x \\
& = (1 + (-1))x \\
& = 0x \\
& = 0 & \qed
\end{align*}
\end{proof}

\section{Units}

\begin{df}[Left-Unit, Right-Unit]
Let $R$ be a ring and $a \in R$. Then $a$ is a \emph{left-unit} iff there exists $b \in R$ such that $ab = 1$. The element $a$ is a \emph{right-unit} iff there exists $b \in R$ such that $ba = 1$.

An element is a \emph{unit} iff it is a left-unit and a right-unit.
\end{df}

\begin{prop}
Let $R$ be a ring and $a \in R$. Then $a$ is a left-unit iff left multiplication by $a$ is a surjective function $R \rightarrow R$.
\end{prop}

\begin{proof}
\pf
\step{1}{If $a$ is a left-unit then left multiplication by $a$ is surjective.}
\begin{proof}
	\step{a}{\pick\ $b \in R$ such that $ab = 1$.}
	\step{b}{For all $c \in R$ we have $c = a(bc)$.}
\end{proof}
\step{2}{If left multiplication by $a$ is surjective then $a$ is a left-unit.}
\begin{proof}
	\pf\ Immediate.
\end{proof}
\qed
\end{proof}

\begin{prop}
Let $R$ be a ring and $a \in R$. Then $a$ is a right-unit iff right multiplication by $a$ is a surjective function $R \rightarrow R$.
\end{prop}

\begin{proof}
\pf\ Similar. \qed
\end{proof}

\begin{prop}
\label{prop:no-left-unit-is-a-right-zero-divisor}
No left-unit is a right-zero-divisor.
\end{prop}

\begin{proof}
\pf
\step{1}{\assume{for a contradiction $ab = 1$ and $ca = 0$ where $c \neq 0$.}}
\step{2}{$c = 0$}
\begin{proof}
	\pf
	\begin{align*}
		0 & = 0b \\
		& = cab \\
		& = c1 \\
		& = c
	\end{align*}
\end{proof}
\qedstep
\begin{proof}
	\pf\ This is a contradiction.
\end{proof}
\qed
\end{proof}

\begin{prop}
\label{prop:no-right-unit-is-a-left-zero-divisor}
No right-unit is a left-zero-divisor.
\end{prop}

\begin{proof}
\pf\ Similar. \qed
\end{proof}

\begin{prop}
The inverse of a unit is unique.
\end{prop}

\begin{proof}
\pf\ If $ba = 1$ and $ac = 1$ then $b = bac = c$. \qed
\end{proof}

\begin{prop}
The units of a ring form a group under multiplication.
\end{prop}

\begin{proof}
\pf
\step{1}{If $a$ and $b$ are units then $ab$ is a unit.}
\begin{proof}
	\pf\ We have $\inv{b} \inv{a} a b = 1$ and $ab \inv{b} \inv{a} = 1$.
\end{proof}
\step{2}{1 is a unit.}
\begin{proof}
	\pf\ Since $1 \cdot 1 = 1$.
\end{proof}
\step{3}{If $a$ is a unit then its inverse is a unit.}
\begin{proof}
	\pf\ Immediate from definitions.
\end{proof}
\qed
\end{proof}

\begin{df}[Group of Units]
For any ring $R$, we write $R^*$ for the group of the units of $R$ under multiplication.
\end{df}

\begin{ex}
The quaternionic group is a subgroup of $\mathbb{H}^*$.
\end{ex}

\begin{ex}
The norm is a group homomorphism $\mathbb{H}^* \rightarrow \mathbb{R}^+$ where $\mathbb{R}^+$ is the group of positive real numbers under multiplication with kernel isomorphic to $\mathrm{SU}_2(\mathbb{C})$. The isomorphism maps a quaternion $a + bi + cj + dk$ to $\left( \begin{array}{cc}
a + bi & c + di \\ -c + di & a - bi
\end{array} \right)$.
\end{ex}

\begin{thm}[Fermat's Little Theorem]
Let $p$ be a prime number and $a$ any integer. Then $a^p \equiv a (\mod p)$.
\end{thm}

\begin{proof}
\pf\ If $p \mid a$ then $a^p \equiv a \equiv 0 (\mod p)$. Otherwise, we have $a^{p-1} \equiv 1 (\mod p)$ by applying Lagrange's Theorem to $(\mathbb{Z} / p \mathbb{Z})^*$. \qed
\end{proof}

\begin{ex}
It is not true that, if $n \mid |G|$, then $G$ has a subgroup of order $n$. The group $A_4$ has order 12 but no subgroup of order 6.
\end{ex}

\begin{prop}
    If $p$ is prime then $(\mathbb{Z} / p \mathbb{Z})^*$ is cyclic.
\end{prop}

\begin{proof}
    \pf
    \step{1}{\pflet{$g$ be an element of maximal order in $(\mathbb{Z} / p \mathbb{Z})^*$.}}
    \step{2}{For all $h \in (\mathbb{Z} / p \mathbb{Z})^*$ we have $h^{|g|} = 1$.}
    \begin{proof}
        \pf\ Proposition \ref{prop:maximal-finite-order}.
    \end{proof}
    \step{3}{There are at most $|g|$ elements $x$ such that $x^{|g|} = 1$ in $\mathbb{Z} / p \mathbb{Z}$} %TODO
    \step{4}{$p-1 \leq |g|$}
    \step{5}{$|g| = p-1$}
    \step{6}{$g$ generates $(\mathbb{Z} / p \mathbb{Z})^*$.}
    \qed
\end{proof}

\begin{ex}
    $(\mathbb{Z} / 12 \mathbb{Z})^*$ is not cyclic. Its elements are 1, 5, 7 and 11 with orders 1, 2, 2 and 2.
\end{ex}

\begin{thm}[Wilson's Theorem]
    A positive integer $p$ is prime if and only if $(p-1)! \equiv 1 (\mod p)$.
\end{thm}

\begin{proof}
    \step{1}{If $p$ is prime then $(p-1)! \equiv 1 (\mod p)$.}
    \begin{proof}
        \step{a}{\assume{$p$ is prime.}}
        \step{b}{$(p-1)!$ is the product of all the elements of $(\mathbb{Z} / p \mathbb{Z})^*$}
        \step{c}{The only element of $(\mathbb{Z} / p \mathbb{Z})^*$ with order 2 is $-1$.}
        \step{d}{$(p-1)! \equiv -1 (\mod p)$}
        \begin{proof}
            \pf\ Proposition \ref{prop:product-of-all-elements}.
        \end{proof}
    \end{proof}
    \step{2}{If $(p-1)! \equiv -1 (\mod p)$ then $p$ is prime.}
    \begin{proof}
        \step{a}{\assume($(p-1)! \equiv -1 (\mod p)$)}
        \step{b}{\pflet{$d$ be a proper divisor of $p$.} \prove{$d = 1$}}
        \step{c}{$d \mid (p-1)!$}
        \step{d}{$d \mid 1$}
        \begin{proof}
            \pf\ Since $d \mid p \mid (p-1)! + 1$.
        \end{proof}
        \step{e}{$d = 1$}
    \end{proof}
    \qed
\end{proof}

\begin{prop}
    If $p$ and $q$ are distinct odd primes then $(\mathbb{Z} / pq\mathbb{Z})^*$ is not cyclic.
\end{prop}

\begin{proof}
    \pf
    \step{1}{$|(\mathbb{Z} / pq \mathbb{Z})^*| = (p-1)(q-1)$}
    \step{2}{\pflet{$g \in (\mathbb{Z} / p q \mathbb{Z})^*$} \prove{$g$ does not have order $(p-1)(q-1)$}}
    \step{3}{$g^{(p-1)(q-1)/2} \equiv 1 (\mod p)$}
    \step{4}{$g^{(p-1)(q-1)/2} \equiv 1 (\mod q)$}
    \step{5}{$pq \mid g^{(p-1)(q-1)/2} - 1$}
    \step{6}{$g^{(p-1)(q-1)/2} \equiv 1 (\mod pq)$}
    \step{7}{$|g| \mid (p-1)(q-1)/2$}
    \qed
\end{proof}

\begin{prop}
    For any prime $p$, we have $\Aut{\Grp}{C_p} \cong C_{p-1}$.
\end{prop}

\begin{proof}
    \pf
    \step{1}{\pflet{$\phi : \Aut{\Grp}{C_p} \rightarrow (\mathbb{Z} / p \mathbb{Z})^*$ be the function $\phi(\alpha) = \alpha(1)$.}}
    \begin{proof}
        \pf\ $\alpha(1)$ has order $p$ in $C_p$ and so is coprime with $p$.
    \end{proof}
    \step{2}{$\phi$ is a homomorphism.}
    \begin{proof}
        \pf\ $\phi(\alpha \circ \beta) = \alpha(\beta(1)) = \alpha(\beta(1) 1) = \beta(1) \alpha(1) = \phi(\alpha) \phi(\beta)$
    \end{proof}
    \step{3}{$\phi$ is injective.}
    \begin{proof}
        \pf\ If $\phi(\alpha) = \phi(\beta)$ then for any $n$ we have $\alpha(n) = n \alpha(1) = n \phi(\alpha) = n \phi(\beta) = n \beta(1) = \beta(n)$.
    \end{proof}
    \step{4}{$\phi$ is surjective.}
    \begin{proof}
        \pf\ For any $r \in (\mathbb{Z} / p \mathbb{Z})^*$ we have $r = \phi(\alpha)$ where $\alpha(n) = nr \mod p$.
    \end{proof}
    \step{5}{$(\mathbb{Z} / p \mathbb{Z})^* \cong C_{p-1}$}
    \qed
\end{proof}

\section{Euler's $\phi$-function}

\begin{prop}
    For $n$ a positive integer, we have $(\mathbb{Z} / n \mathbb{Z})^* = \{ m \in \mathbb{Z} / n \mathbb{Z} : \gcd(m,n) = 1 \}$.
\end{prop}

\begin{proof}
\pf
\begin{align*}
m \in (\mathbb{Z} / n \mathbb{Z})^* & \Leftrightarrow \exists a. am \equiv 1 (\mod n) \\
& \Leftrightarrow \exists a,b. am + bn = 1 \\
& \Leftrightarrow \gcd(m,n) = 1 & \qed
\end{align*}
\end{proof}

\begin{df}[Euler's Totient Function]
    For $n$ a positive integer, let $\phi(n) = |(\mathbb{Z} / n \mathbb{Z})^*|$.
\end{df}

\begin{prop}
    If $n$ is an odd positive integer then $\phi(2n) = \phi(n)$.
\end{prop}

\begin{proof}
    \pf
    \step{1}{\pflet{$n$ be an odd positive integer.}}
    \step{2}{For any integer $m$, if $\gcd(m,n) = 1$ then $\gcd(2m+n,2n) = 1$}
    \begin{proof}
        \pf\ For $p$ a prime, if $p \mid 2m+n$ and $p \mid 2n$ then $p \neq 2$ (since $2m+n$ is odd) so $p \mid n$ and hence $p \mid m$, which is a contradiction.
    \end{proof}
    \step{3}{For any integer $r$, if $\gcd(r,2n) = 1$ then $\gcd(\frac{r+n}{2},n) = 1$}
    \begin{proof}
        \pf\ If $p \mid n$ and $p \mid \frac{r + n}{2}$ then $p \mid r + n$ so $p \mid r$ which is a contradiction.
    \end{proof}
    \step{4}{The function that maps $m$ to $2m+n$ is a bijection between $(\mathbb{Z} / n\mathbb{Z})^*$
        and $(\mathbb{Z} / 2n\mathbb{Z})^*$.}
    \qed
\end{proof}

\begin{thm}
For any positive integer $n$ we have
\[ \sum_{m > 0, m \mid n} \phi(m) = n \enspace . \]
\end{thm}

\begin{proof}
\pf
\step{1}{Define $\chi : \{ 0, 1, \ldots, n-1 \} \rightarrow \{ (m,d) : m > 0, m \mid n, d \text{ generates } \langle n/m \rangle \}$ by: $\chi(x) = (\gcd(x,n), x)$.}
\step{2}{$\chi$ is injective.}
\step{3}{$\chi$ is surjective.}
\begin{proof}
\pf\ Given $(m,d)$ such that $d$ generates $\langle n/m \rangle$ we have $\chi(d) = (m,d)$.
\end{proof}
\step{4}{$n = \sum_{m > 0, m \mid n} \phi(m)$}
\begin{proof}
\pf\ Since $\langle n/m \rangle \cong C_m$ and so has $\phi(m)$ generators.
\end{proof}
\qed
\end{proof}

\begin{prop}
For any positive integers $a$ and $n$, we have $n \mid \phi(a^n - 1)$.
\end{prop}

\begin{proof}
\pf\ Since the order of $a$ is $n$ in $(\mathbb{Z} / (a^n - 1)\mathbb{Z})^*$. \qed
\end{proof}

\begin{thm}[Euler's Theorem]
For any coprime integers $a$ and $n$ we have $a^{\phi(n)} \equiv a (\mod n)$.
\end{thm}

\begin{proof}
\pf\ Immediate from Lagrange's Theorem. \qed
\end{proof}

\begin{prop}
    \[ |\Aut{\Grp}{C_n}| = \phi(n) \]
\end{prop}

\begin{proof}
    \pf\ An automorphism $\alpha$ is determined by $\alpha(1)$ which is any element of order $n$, and $g$ has order $n$ iff $\gcd(g,n) = 1$. \qed
\end{proof}

\begin{ex}
    \[ \Aut{\Grp}{\mathbb{Z}} \cong C_2 \]
\end{ex}

\begin{proof}
    \pf\ The only automorphisms are the identity and multiplication by -1. \qed
\end{proof}

\section{Nilpotent Elements}

\begin{df}[Nilpotent]
Let $R$ be a ring and $a \in R$. Then $a$ is \emph{nilpotent} iff there exists $n$ such that $a^n = 0$.
\end{df}

\begin{prop}
Let $R$ be a ring and $a,b \in R$. If $a$ and $b$ are nilpotent and $ab = ba$ then $a + b$ is nilpotent.
\end{prop}

\begin{proof}
\pf
\step{1}{\pick\ $m$ and $n$ such that $a^m = b^n = 0$.}
\step{2}{$(a+b)^{m+n} = 0$}
\begin{proof}
	\pf\ Since $(a+b)^{m+n} = \sum_k \left( \begin{array}{c} m + n \\ k \end{array} \right) a^k b^{m+n-k}$ and every term in this sum is 0 since, for every $k$, either $k \geq m$ or $m+n-k \geq n$.
\end{proof}
\qed
\end{proof}

\begin{prop}
$m$ is nilpotent in $\mathbb{Z} / n \mathbb{Z}$ if and only if $m$ is divisible by all the prime factors of $n$.
\end{prop}

\begin{proof}
\pf
\step{1}{If $m$ is nilpotent then $m$ is divisible by all the prime factors of $n$.}
\begin{proof}
	\step{a}{\assume{$m^a \equiv 0 (\mod n)$}}
	\step{b}{For every prime $p$, if $p \mid n$ then $p \mid m^a$.}
	\step{c}{For every prime $p$, if $p \mid n$ then $p \mid m$.}
\end{proof}
\step{2}{If $m$ is divisible by all the prime factors of $n$ then $m$ is nilpotent in $\mathbb{Z} / n \mathbb{Z}$.}
\begin{proof}
	\step{a}{\assume{$m$ is divisible by all the prime factors of $n$.}}
	\step{b}{\pflet{$a$ be the largest number such that $p^a \mid n$ for some prime $p$.}}
	\step{c}{For every prime $p$ that divides $n$ we have $p^a \mid m^a$}
	\step{d}{$n \mid m^a$}
	\step{e}{$m^a \equiv 0 (\mod n)$}
	\step{f}{$m$ is nilpotent in $\mathbb{Z} / n \mathbb{Z}$.}
\end{proof}
\qed
\end{proof}

\chapter{Ring Homomorphisms}

\begin{df}[Ring Homomorphism]
Let $R$ and $S$ be rings. A \emph{ring homomorphism} $\phi : R \rightarrow S$ is a rng homomorphism such that $\phi(1) = 1$.
\end{df}

\begin{prop}
The zero-ring is terminal in $\mathbf{Ring}$.
\end{prop}

\begin{proof}
\pf\ Easy. \qed
\end{proof}

\begin{prop}
The ring $\mathbb{Z}$ is initial in $\mathbf{Ring}$.
\end{prop}

\begin{proof}
\pf\ Easy. \qed
\end{proof}

\begin{prop}
Let $R$ and $S$ be rings and $\phi : R \rightarrow S$ be a rng homomorphism. If $\phi$ is surjective, then $\phi$ is a ring homomorphism.
\end{prop}

\begin{proof}
\pf
\step{1}{\pick\ $a \in R$ such that $\phi(a) = 1$}
\step{3}{$\phi(1) = 1$}
\begin{proof}
\pf
\begin{align*}
\phi(1) & = \phi(1) \phi(a) \\
& = \phi(1a) \\
& = \phi(a) \\
& = 1 & \qed
\end{align*}
\end{proof}
\end{proof}

\begin{ex}
For any set $S$ we have $\mathcal{P} S \cong (\mathbb{Z} / 2 \mathbb{Z})^S$ in $\mathbf{Ring}$ with the isomorphism
\begin{align*}
\phi & : \mathcal{P} S \cong (\mathbb{Z} / 2 \mathbb{Z})^S \\
\phi(A)(s) & = \begin{cases}
1 & \text{if } s \in A \\
0 & \text{if } s \notin A
\end{cases}
\end{align*}
\end{ex}

\begin{ex}
The function $\mathbb{H} \rightarrow \mathfrak{gl}_4(\mathbb{R})$ that maps $a + bi + cj + dk$ to
\[ \left( \begin{array}{cccc}
a & b & c & d \\
-b & a & -d & c \\
-c & d & a & -b \\
-d & -c & b & a
\end{array} \right) \]
is a monomorphism in $\mathbf{Ring}$, as is the function $\mathbb{H} \rightarrow \mathfrak{sl}_2(\mathbb{C})$ that maps $a + bi + cj + dk$ to
\[ \left( \begin{array}{cc}
a + bi & c + di \\
-c + di & a - bi
\end{array} \right) \enspace . \]
\end{ex}

\begin{prop}
Ring homomorphisms preserve units.
\end{prop}

\begin{proof}
\pf\ If $uv = 1$ then $\phi(u) \phi(v) = 1$. \qed
\end{proof}

\begin{prop}
Let $\phi : R \rightarrow S$ be a ring homomorphism. Then the following are equivalent.
\begin{enumerate}
\item $\phi$ is a monomorphism.
\item $\ker \phi = \{0\}$
\item $\phi$ is injective.
\end{enumerate}
\end{prop}

\begin{proof}
\pf
\step{1}{$1 \Rightarrow 2$}
\begin{proof}
	\step{a}{\assume{$\phi$ is a monomorphism.}}
	\step{b}{\pflet{$r \in \ker \phi$}}
	\step{c}{\pflet{$\mathrm{ev}_r : \mathbb{Z}[x] \rightarrow R$ be the unique ring homomorphism such that $\mathrm{ev}_r(x) = r$.}}
	\step{d}{\pflet{$\mathrm{ev}_0 : \mathbb{Z}[x] \rightarrow R$ be the unique ring homomorphism such that $\mathrm{ev}_0(x) = 0$.}}
	\step{e}{$\phi \circ \mathrm{ev}_r = \phi \circ \mathrm{ev}_0$}
	\step{f}{$\mathrm{ev}_r = \mathrm{ev}_0$}
	\step{g}{$r = 0$}
\end{proof}
\step{2}{$2 \Rightarrow 3$}
\begin{proof}
	\pf\ Proposition \ref{prop:ker-zero}.
\end{proof}
\step{3}{$3 \Rightarrow 1$}
\begin{proof}
	\pf\ Easy.
\end{proof}
\qed
\end{proof}

\begin{ex}
It is not true that every epimorphism in $\Ring$ is surjective. The inclusion $\mathbb{Z} \hookrightarrow \mathbb{Q}$ is an epimorphism but not surjective.

The same example shows that a ring homomorphism may be a monomorphism and an epimorphism but not be an isomorphism.
\end{ex}

\begin{ex}
\[ \End{\Ab}{\mathbb{Z}} \cong \mathbb{Z} \]
The isomorphism maps any group endomorphism $\phi : \mathbb{Z} \rightarrow \mathbb{Z}$ to $\phi(1)$.
\end{ex}

\begin{ex}
The group of units of $\End{\Ab}{G}$ is $\Aut{\Ab}{G}$.
\end{ex}

\begin{ex}
Let $R$ be a ring. Then the function $\lambda : R \rightarrow \End{\Ab}{R}$ defined by
\[ \lambda(a)(b) = ab \]
is a ring monomorphism.
\end{ex}
 
\begin{proof}
\pf\ Easy. \qed
\end{proof}
 
\section{Products}

\begin{prop}
Let $R$ and $S$ be rings. Then $R \times S$ is a ring under componentwise addition and multiplication, and this ring is the product of $R$ and $S$ in $\Ring$.
\end{prop}

\begin{proof}
\pf\ Easy. \qed
\end{proof}


\chapter{Subrings}

\begin{df}[Subring]
Let $S$ be a ring. A \emph{subring} of $S$ is a ring $R$ such that $R$ is a subset of $S$ and the inclusion $R \hookrightarrow S$ is a ring homomorphism.
\end{df}

\begin{prop}
Let $R$ and $S$ be rings. Then $R$ is a subring of $S$ if and only if $R$ is a subset of $S$, the unit 1 of $S$ is an element of $R$, and the operations of $R$ are the restrictions of the operations of $S$ to $R$.
\end{prop}

\begin{proof}
\pf\ Easy. \qed
\end{proof}

\begin{cor}
The zero ring is not a subring of any non-zero ring.
\end{cor}

\begin{prop}
Let $\phi : R \rightarrow S$ be a ring homomorphism. Then $\phi(R)$ is a subring of $S$.
\end{prop}

\begin{proof}
\pf\ Easy. \qed
\end{proof}

\section{Centralizer}

\begin{df}[Centralizer]
Let $R$ be a ring and $a \in R$. The \emph{centralizer} of $a$ is $\{ r \in R : ar = ra \}$.
\end{df}

\begin{prop}
The centralizer of $a$ is a subring of $R$.
\end{prop}

\begin{proof}
\pf\ Easy. \qed
\end{proof}

\section{Center}

\begin{df}[Center]
The \emph{center} of a ring $R$ is $\{ x \in R : \forall y \in R. xy = yx \}$.
\end{df}

\begin{prop}
The center of a ring is a subring.
\end{prop}

\begin{proof}
\pf\ Easy. \qed
\end{proof}

\begin{prop}
Let $R$ be a ring. The center of $\End{\Ab}{R}$ is isomorphic to the center of $R$.
\end{prop}

\begin{proof}
\pf
\step{1}{\pflet{$\lambda : R \rightarrow \End{\Ab}{R}$ be left multiplication.}}
\step{2}{$\lambda$ maps $Z(R)$ to $Z(\End{\Ab}{R})$.}
\begin{proof}
	\step{a}{\pflet{$a \in Z(R)$}}
	\step{b}{\pflet{$\phi \in \End{\Ab}{R}$} \prove{$\lambda(a) \circ \phi = \phi \circ \lambda(a)$}}
	\step{c}{\pflet{$x \in R$}}
	\step{d}{$a + \phi(x) = \phi(a + x)$}
\end{proof}
\step{3}{$\lambda(Z(R)) = Z(\End{\Ab}{R})$}
\begin{proof}
	\step{a}{\pflet{$\phi \in Z(\End{\Ab}{R})$}}
	\step{b}{For all $r \in R$, \pflet{$\mu_r \in \End{\Ab}{R}$ be right multiplication by $r$.}}
	\step{c}{For all $r \in R$ we have $\phi \circ \mu_r = \mu_r \circ \phi$.}
	\step{d}{For all $r,x \in R$ we have $\phi(xr) = \phi(x)r$}
	\step{e}{For all $r \in R$ we have $\phi(r) = \phi(1)r$}
	\step{f}{$\phi = \lambda(\phi(1))$}
\end{proof}
\qed
\end{proof}

\begin{cor}
If $R$ is a commutative ring then $R$ is isomorphic to the center of $\End{\Ab}{R}$.
\end{cor}

\begin{ex}
For $n$ a positive integer we have $\mathbb{Z} / n \mathbb{Z} \cong \End{\Ab}{\mathbb{Z} / n \mathbb{Z}}$.

Since, for any $\phi \in \End{\Ab}{\mathbb{Z} / n \mathbb{Z}}$ we have $\phi(m) = m \phi(1)$ and so the whole of $\End{\Ab}{\mathbb{Z} / n \mathbb{Z}}$ is the image of $\lambda$.
\end{ex}


\chapter{Monoid Rings}

\begin{df}[Monoid Ring]
Let $R$ be a ring and $M$ a monoid. Define $R[M]$ to be the ring whose elements are the families $\{ a_m \}_{m \in M}$ such that $a_m = 0$ for all but finitely many $m \in M$, written
\[ \sum_{m \in M} a_m m \enspace , \]
under
\begin{align*}
\sum_m a_m m + \sum_m b_m m & = \sum_m (a_m + b_m) m \\
\left( \sum_m a_m m \right) \left( \sum_m b_m m \right) & = \sum_{m \in M} \sum_{m_1 m_2 = m} a_{m_1} b_{m_2} m
\end{align*}
\end{df}

\begin{ex}
Ring homomorphisms do not necessarily preserve zero-divisors. The canonical homomorphism $\pi : \mathbb{Z} \rightarrow \mathbb{Z} / 6 \mathbb{Z}$ maps the non-zero-divisor 2 to a zero-divisor.
\end{ex}

\section{Polynomials}

\begin{df}[Polynomial]
Let $R$ be a ring. The ring of \emph{polynomials} $R[x]$ is $R[\mathbb{N}]$. We write 
\[ \sum_n a_n x^n \text{ for } \sum_n a_n n \enspace . \]

Concretely, a \emph{polynomial} in $R$ is a sequence $(a_n)$ in $R$ such that there exists $N$ such that $\forall n \geq N. a_n = 0$. We write the polynomial as
\[ \sum_{n=0}^{N-1} a_n x^n = a_0 + a_1 x + a_2 x^2 + \cdots + a_{N-1} x^{N-1} \enspace . \]
We write $R[x]$ for the set of all polynomials in $R$.

Define addition and multiplication on $R[x]$ by
\begin{align*}
\sum_n a_n x^n + \sum_n b_n x^n & = \sum_n (a_n + b_n) x^n \\
\left( \sum_n a_n x^n \right) \left( \sum_n b_n x^n \right) & = \sum_n \sum_{i+j=n} a_i b_j x^n
\end{align*}

A \emph{constant} is a polynomial of the form $a + 0x + 0x^2 + \cdots$ for some $a \in R$.

We write $R[x_1, \ldots, x_n]$ for $R[x_1][x_2] \cdots [x_n]$.
\end{df}

\begin{prop}
For any ring $R$, the set of polynomials $R[x]$ is a ring.
\end{prop}

\begin{proof}
\pf\ Easy. \qed
\end{proof}

\begin{df}[Degree]
The \emph{degree} of a polynomial $\sum_n a_n x^n$ is the largest integer $d$ such that $a_d \neq 0$. We take the degree of the zero polynomial to be $- \infty$.
\end{df}

\begin{prop}
Let $R$ be a ring and $f,g \in R[x]$ be nonzero polynomials. Then
\[ \deg(f + g) \leq \max(\deg f, \deg g) \enspace . \]
\end{prop}

\begin{proof}
\pf\ If $a_n + b_n \neq 0$ then $a_n \neq 0$ or $b_n \neq 0$. \qed
\end{proof}

\begin{prop}
The function $i : n \rightarrow \mathbb{Z}[x_1, \ldots, x_n]$ that maps $k$ to $x_k$ is initial in the category with:
\begin{itemize}
\item objects all pairs $j : n \rightarrow R$ where $R$ is a commutative ring and $j$ a function
\item morphisms $\phi : (j_1, R_1) \rightarrow (j_2,R_2)$ are ring homomorphisms $\phi : R_1 \rightarrow R_2$ such that $\phi \circ j_1 = j_2$.
\end{itemize}
\end{prop}

\begin{proof}
\pf\ The unique morphism $(i, \mathbb{Z}[x_1, \ldots, x_n]) \rightarrow (j, R)$ maps a polynomial $p$ to $p(j(0), j(1), \ldots, j(n-1))$. \qed
\end{proof}

\begin{prop}
Let $\alpha : R \rightarrow S$ be a ring homomorphism. Let $s \in S$ commute with $\alpha(r)$ for all $r \in R$. Then there exists a unique ring homomorphism $\overline{\alpha} : R[x] \rightarrow S$ such that $\overline{\alpha}(x) = s$ and the following diagram commutes:
\[ \begin{tikzcd}
R[x] \arrow[r,"\overline{\alpha}"] & S \\
R \arrow[u] \arrow[ur,"\alpha"]
\end{tikzcd} \]
\end{prop}

\begin{proof}
\pf\ The map $\overline{\alpha}$ is given by
\[ \overline{\alpha}(a_0 + a_1 x + a_2 x^2 + \cdots + a_n x^n) = \alpha(a_0) + \alpha(a_1) s + \alpha(a_2) s^2 + \cdots + \alpha(a_n) s^n \enspace . \]
\qed
\end{proof}

\begin{df}
Let $R$ be a commutative ring.
Given a polynomial $p \in R[x]$, the \emph{polynomial function} $p : R \rightarrow R$ is the function given by: $p(r) = \alpha_r(p)$, where $\alpha_r : R[x] \rightarrow R$ is the unique ring homomorphism such that the following diagram commutes.
\[ \begin{tikzcd}
R[x] \arrow[r,"\alpha_r"] & R \\
1 \arrow[u,"x"] \arrow[ur,"r"]
\end{tikzcd} \]
\end{df}

\begin{prop}
$\mathbb{Z}[x,y]$ is the coproduct of $\mathbb{Z}[x]$ and $\mathbb{Z}[y]$ in the category of commutative rings.
\end{prop}

\begin{proof}
\pf\ Given ring homomorphisms $f : \mathbb{Z}[x] \rightarrow R$ and $g : \mathbb{Z}[y] \rightarrow R$, the required morphism $\mathbb{Z}[x,y] \rightarrow R$ maps $p(x,y)$ to $p(f(x),g(y))$. \qed
\end{proof}

\begin{ex}
$\mathbb{Z}[x,y]$ is not the coproduct of $\mathbb{Z}[x]$ and $\mathbb{Z}[y]$ in $\Ring$. Given $f : \mathbb{Z}[x] \rightarrow R$ and $g : \mathbb{Z}[y] \rightarrow R$ with $f(x) \neq g(y)$, the mediating morphism $\mathbb{Z}[x,y] \rightarrow R$ cannot exist since it must map $xy$ to both $f(x)g(y)$ and $g(y)f(x)$. \qed
\end{ex}

\begin{df}
A polynomial is \emph{monic} iff its last non-zero coefficient is 1.
\end{df}

\begin{prop}
A monic polynomial is not a left- or right-zero-divisor.
\end{prop}

\begin{proof}
\pf\ Easy. \qed
\end{proof}

\begin{prop}
Let $R$ be a ring. Let $f,g \in R[x]$ with $f$ monic. Then there exist unique polynomials $q,r \in R[x]$ with $\deg r < \deg f$ such that
\[ g = qf + r \enspace . \]
\end{prop}

\begin{proof}
\pf
\step{0}{\pflet{$d = \deg f$}}
\step{1}{For all $a \in R$ and $n > d$, there exists $h \in R[x]$ with $\deg h < n$ such that 
\[ ax^n = ax^{n-d} f + h \enspace . \]}
\begin{proof}
\pf\ Take $h = ax^n - ax^{n-d}f$.
\end{proof}
\step{1a}{For all $a \in R$ and $n > d$, there exists $q,h \in R[x]$ with $\deg h \leq d$ such that 
\[ ax^n = q f + h \enspace . \]}
\begin{proof}
\pf\ Repeating \stepref{1} by induction.
\end{proof}
\step{2}{\pflet{$g = \sum_{i=0}^n a_i x^i$}}
\step{3}{For $i > d$, \pick\ $q_ih_i \in R[x]$ with $\deg h < \deg f$ such that $a_i x^i = q_i f + h_i$}
\step{4}{$g = \left( \sum_{i=d+1}^n q_i \right) f + \sum_{i=d+1}^n h_i$}
\step{5}{$q$ and $r$ are unique.}
\begin{proof}
\pf\ If $q_1 f + r_1 = q_2 f + r_2$ then $r_1 - r_2 = (q_2 - q_1)f$ and so $r_1 - r_2 = (q_2 - q_1) f = 0$ since $\deg (r_1 - r_2) < \deg f$.
\end{proof}
\qed
\end{proof}

\section{Laurent Polynomials}

\begin{df}[Laurent Polynomial]
Let $R$ be a ring. The ring of \emph{Laurent polynomials} is the group ring $R[\mathbb{Z}]$. We write $\sum_{n \in \mathbb{Z}} a_n x^{n}$ for $\sum_n a_n n$.
\end{df}

\section{Power Series}

\begin{df}[Power Series]
Let $R$ be a ring. A \emph{power series} in $R$ is a sequence $(a_n)$ in $R$. We write the power series as
\[ \sum_{n=0}^{\infty} a_n x^n = a_0 + a_1 x + a_2 x^2 + \cdots \enspace . \]
We write $R[[x]]$ for the set of all power series in $R$.

Define addition and multiplication on $R[[x]]$ by
\begin{align*}
\sum_n a_n x^n + \sum_n b_n x^n & = \sum_n (a_n + b_n) x^n \\
\left( \sum_n a_n x^n \right) \left( \sum_n b_n x^n \right) & = \sum_n \sum_{i+j=n} a_i b_j x^n
\end{align*}
\end{df}

\begin{prop}
For any ring $R$, the set of power series $R[[x]]$ is a ring.
\end{prop}

\begin{proof}
\pf\ Easy. \qed
\end{proof}

\begin{prop}
A power series $\sum_n a_n x^n$ is a unit in $R[[x]]$ if and only if $a_0$ is a unit in $R$.
\end{prop}

\begin{proof}
\pf
\step{1}{If $\sum_n a_n x^n$ is a unit then $a_0$ is a unit.}
\begin{proof}
	\step{a}{\pflet{$\sum_n b_n x^n$ be the inverse of $\sum_n a_n x^n$.}}
	\step{b}{$a_0 b_0 = b_0 a_0 = 1$}
\end{proof}
\step{2}{If $a_0$ is a unit then $\sum_n a_n x^n$ is a unit.}
\begin{proof}
	\pf\ Define the sequence $(b_n)$ in $R$ by
	\[ b_n = - \inv{a_0} \sum_{i=1}^n a_i b_{n-i} \]
	Then $\sum_n b_n x^n$ is the inverse of $\sum_n a_n x^n$.
\end{proof}
\qed
\end{proof}

\chapter{Ideals}

\begin{df}[Left-Ideal]
Let $R$ be a ring.

A subgroup $I$ of $R$ is a \emph{left-ideal} iff, for all $r \in R$, we have $rI \subseteq I$.

A subgroup $I$ of $R$ is a \emph{right-ideal} iff, for all $r \in R$, we have $Ir \subseteq I$.

A subgroup $I$ of $R$ is a \emph{(two-sided) ideal} iff it is a left-ideal and a right-ideal.
\end{df}

\begin{ex}
Let $R$ be a ring and $a \in R$. Then $Ra$ is a left-ideal and $aR$ is a right-ideal.

In particular, $\{0\}$ is always a two-sided ideal.
\end{ex}

\begin{ex}
Let $S$ be a set and $T \subseteq S$. Then $\{ X \in \mathcal{P} S : X \subseteq T \}$ is an ideal in $\mathcal{P} S$.
\end{ex}

\begin{prop}
Let $S$ be a finite set. Then every ideal in $\mathcal{P} S$ is of the form $\{ X \in \mathcal{P} S : X \subseteq T \}$ for some $T \subseteq S$.
\end{prop}

\begin{proof}
\pf
\step{1}{\pflet{$I$ be an ideal in $\mathcal{P} S$.}}
\step{2}{\pflet{$T = \bigcup I$}}
\step{3}{For all $i \in T$ we have $\{i\} \in I$.}
\begin{proof}
	\step{a}{\pflet{$i \in T$}}
	\step{b}{\pick\ $X \in I$ such that $i \in X$}
	\step{c}{$\{i\} = \{i \} \cap X \in I$}
\end{proof}
\step{4}{For all $X \subseteq T$ we have $X \in I$.}
\begin{proof}
	\pf\ If $X = \{x_1, \ldots, x_n\}$ then $X = \{x_1\} + \cdots + \{x_n\} \in I$.
\end{proof}
\qed
\end{proof}

\begin{ex}
If $S$ is an infinite set, then there is always an ideal in $\mathcal{P} S$ that is not of the form $\{X \in \mathcal{P} S : X \subseteq T \}$ for some $T \subseteq S$, namely the set of all finite subsets of $S$.
\end{ex}

\begin{prop}
Let $\phi : R \twoheadrightarrow S$ be a surjective ring homomorphism. Let $J$ be an ideal in $R$. Then $\phi(J)$ is an ideal in $S$.
\end{prop}

\begin{proof}
\pf
\step{1}{\pflet{$j \in J$ and $s \in S$} \prove{$s \phi(j), \phi(j)s \in \phi(J)$}}
\step{2}{\pick\ $r \in R$ such that $\phi(r) = s$}
\step{3}{$rj, jr \in J$}
\step{4}{$s\phi(j), \phi(j)s \in \phi(J)$}
\qed
\end{proof}

\begin{ex}
We cannot remove the hypothesis that $\phi$ is surjective.

Let $i : \mathbb{Z} \hookrightarrow \mathbb{Q}$ be the inclusion. Then $i(2 \mathbb{Z}) = 2 \mathbb{Z}$ is not an ideal in $\mathbb{Q}$.
\end{ex}

\begin{prop}
Let $\phi : R \rightarrow S$ be a ring homomorphism and $I$ a (left-, right-)ideal in $S$. Then $\inv{\phi}{I}$ is a (left-, right-)ideal in $R$.
\end{prop}

\begin{proof}
\pf\ Easy. \qed
\end{proof}

\begin{cor}
Let $\phi : R \rightarrow S$ be a ring homomorphism. Then $\ker \phi$ is an ideal in $R$.
\end{cor}

\begin{df}[Quotient Ring]
Let $I$ be an ideal in $R$. The \emph{quotient ring} $R / I$ is the quotient group $R/I$ under
\[ (a + I)(b + I) = ab + I \enspace . \]

This is well-defined as, if $a + I = a' + I$ and $b + I = b' + I$ then
\begin{align*}
a-a' & \in I \\
b - b' & \in I \\
\therefore ab - a'b & \in I \\
a'b - a'b' & \in I \\
\therefore ab - a'b' & \in I
\end{align*}
\end{df}

\begin{prop}
Let $I$ be an ideal in $R$. Then the canonical group homomorphism $\pi : R \rightarrow R / I$ is a ring homomorphism.
\end{prop}

\begin{proof}
\pf\ By construction. \qed
\end{proof}

\begin{prop}
Let $I$ be an ideal in a ring $R$. For every ring homomorphism $\phi : R \rightarrow S$ such that $I \subseteq \ker \phi$, there exists a unique ring homomorphism $\overline{\phi} : R / I \rightarrow S$ such that the following diagram commutes.

\begin{center}
\begin{tikzcd}
R \arrow[rr,"\phi"] \arrow[dr, two heads, "\pi"] & & S \\
& R / I \arrow[ur,"\overline{\phi}"]
\end{tikzcd}
\end{center}
\end{prop}

\begin{proof}
\pf\ Easy. \qed
\end{proof}

\begin{cor}
Every ring homomorphism $\phi : R \rightarrow S$ decomposes as follows.

\begin{center}
\begin{tikzcd}
R \arrow[r, two heads] \arrow[rrr, bend left=25, "\phi"] & R / \ker \phi \arrow[r,"\cong"] & \im \phi \arrow[r, hook] & S
\end{tikzcd}
\end{center}
\end{cor}

\begin{cor}[First Isomorphism Theorem]
Let $\phi : R \twoheadrightarrow S$ be a surjective ring homomorphism. Then
\[ S \cong R / \ker \phi \enspace . \]
\end{cor}

\begin{thm}[Third Isomorphism Theorem]
Let $I$ and $J$ be ideals in $R$ with $I \subseteq J$. Then $J/I$ is an ideal in $R/I$, and
\[ \frac{R/I}{J/I} \cong R/J \]
\end{thm}

\begin{proof}
\pf\ Since the function $R/I \rightarrow R/J$ that maps $r + I$ to $r + J$ is a surjective ring homomorphism with kernel $J/I$. \qed
\end{proof}

\begin{cor}
Let $\phi : R \twoheadrightarrow S$ be a surjective ring homomorphism. Let $J$ be an ideal in $R$. Then
\[ \frac{S}{\phi(J)} \cong \frac{R}{\ker S + J} \]
\end{cor}

\begin{prop}
Let $R$ be a ring and $J$ an ideal in $\gl{n}{R}$. Let $A \in \gl{n}{R}$. Then $A \in J$ if and only if the matrices obtained by placing any entry of $A$ in any position and zeros elsewhere all belong to $J$.
\end{prop}

\begin{proof}
\pf\ Each such matrix can be obtained by pre- and post-multiplying $A$ by matrices which have a single 1 and 0s elsewhere. Conversely, $A$ is a sum of such matrices. \qed
\end{proof}

\begin{cor}
\label{cor:ideals-in-glnR}
Let $R$ be a ring. Let $J$ be an ideal in $\gl{n}{R}$. Let $I$ be the set of all entries of elements of $J$. Then $I$ is an ideal in $R$, and $J$ is the set of all matrices whose entries are in $I$.
\end{cor}

\begin{prop}
Let $R$ be a ring. Let $\{I_\alpha\}_{\alpha \in A}$ be a family of ideals in $R$. Let
\[ \sum_{\alpha \in A} I_\alpha = \{ \sum_{\alpha \in A} r_\alpha : \forall \alpha. r_\alpha \in I_\alpha, r_\alpha = 0 \text{ for all but finitely many } \alpha \in A \}\enspace . \]
Then $\sum_{\alpha \in A} I_\alpha$ is an ideal, and is the smallest ideal that includes every $I_\alpha$.
\end{prop}

\begin{proof}
\pf\ Easy. \qed
\end{proof}

\begin{prop}
The intersection of a set of ideals is an ideal.
\end{prop}

\begin{proof}
\pf\ Easy. \qed
\end{proof}

\section{Characteristic}

\begin{df}[Characteristic]
The \emph{characteristic} of a ring $R$ is the nonnegative integer $n$ such that $n \mathbb{Z}$ is the kernel of the unique ring homomorphism $\mathbb{Z} \rightarrow R$.
\end{df}

\begin{prop}
Let $R$ be a ring.
If the unit 1 has finite order in $R$, then its order is the characteristic of $R$; otherwise, the characteristic of $R$ is 0.
\end{prop}

\begin{proof}
\pf\ Easy. \qed
\end{proof}

\begin{ex}
The zero ring is the only ring with characteristic 1.
\end{ex}

\section{Nilradical}

\begin{df}[Nilradical]
Let $R$ be a commutative ring. The \emph{nilradical} of $R$ is the set of all nilpotent elements.
\end{df}

\begin{prop}
Let $R$ be a commutative ring. The nilradical of $R$ is an ideal in $R$.
\end{prop}

\begin{proof}
\pf\ If $a^n = 0$ then for any $b$ we have $(ba)^n = 0$. \qed
\end{proof}

\begin{ex}
We cannot remove the assumption that $R$ is commutative. In $\gl{2}{\mathbb{R}}$ we have that $\left( \begin{array}{cc} 0 & 1 \\ 0 & 0 \end{array} \right)$ is nilpotent but $\left( \begin{array}{cc} 0 & 1 \\ 0 & 0 \end{array} \right) \left( \begin{array}{cc} 0 & 0 \\ 1 & 1 \end{array} \right) = \left( \begin{array}{cc} 1 & 1 \\ 0 & 0 \end{array} \right)$ is not.
\end{ex}

\section{Principal Ideals}

\begin{df}[Principal Ideal]
Let $R$ be a commutative ring and $a \in R$. The \emph{principal ideal} generated by $a$ is $(a) = Ra = aR$.
\end{df}

\begin{ex}
$\{0\} = (0)$ and $R = \{1\}$ are principal ideals.
\end{ex}

\begin{df}
Let $R$ be a commutative ring and $\{a_\alpha\}_{\alpha \in A}$ be a family of elements of $R$. The \emph{ideal generated by the elements $a_\alpha$} is
\[ (a_\alpha)_{\alpha \in A} := \sum_{\alpha \in A} (a_\alpha) \enspace . \]

An ideal is \emph{finitely generated} iff it is generated by a finite family of elements.
\end{df}

\begin{df}
Let $R$ be a commutative ring and $I$, $J$ be ideals in $R$. Then $IJ$ is the ideal generated by $\{ ij \}_{i \in I, j \in J}$.
\end{df}

\begin{prop}
\[ IJ \subseteq I \cap J \]
\end{prop}

\begin{proof}
\pf\ Easy. \qed
\end{proof}

\begin{prop}
Let $R$ be a commutative ring. Let $I$ and $J$ be ideals in $R$. If $I + J = R$ then $IJ = I \cap J$.
\end{prop}

\begin{proof}
\pf
\step{1}{\pflet{$r \in I \cap J$}}
\step{2}{\pick\ $i \in I$ and $j \in J$ such that $i + j = 1$.}
\step{3}{$ri, rj \in IJ$}
\step{4}{$r = ri + rj \in IJ$}
\qed
\end{proof}

\begin{prop}
Let $R$ be a commutative ring. Let $f \in R[x]$ be a monic polynomial of degree $d$. Then the function
\[ \phi : R[x] \rightarrow R^{\oplus d} \]
that sends a polynomial $g$ to the remainder of the division of $g$ by $f$ induces an isomorphism of Abelian groups
\[ \frac{R[x]}{(f(x))} \cong R^{\oplus d} \enspace . \]
\end{prop}

\begin{proof}
\pf\ It is clearly a group homomorphism; it is surjective since it maps any polynomial of degree $< d$ to itself, and its kernel is $(f(x))$ since these are the polynomials with remainder 0. \qed
\end{proof}

\begin{cor}
Let $R$ be a commutative ring and $a \in R$. Then we have
\[ \frac{R[x]}{(x-a)} \cong R \]
\end{cor}

\begin{proof}
\pf
\step{1}{\pflet{$\phi : R[x] \rightarrow R$ be evaluation at $a$.}}
\step{2}{$\phi(g)$ is the remainder when dividing $g$ by $x-a$.}
\begin{proof}
\pf\ If $g = (x-a)q + r$ then $g(a) = (a-a)q(a) + r = r$.
\end{proof}
\step{3}{$\phi$ induces a group isomorphism $R[x]/(x-a) \cong R$}
\begin{proof}
\pf\ By the theorem.
\end{proof}
\step{4}{This isomorphism is a ring isomorphism.}
\begin{proof}
\pf\ Since evaluation at $a$ is a ring homomorphism.
\end{proof}
\qed
\end{proof}

\begin{ex}
We have
\[ \frac{\mathbb{R}[x]}{(x^2 + 1)} \cong \mathbb{C} \]
as rings.
\end{ex}

\section{Maximal Ideals}

\begin{df}[Maximal Ideal]
Let $R$ be a ring and $I$ an ideal in $R$. Then $I$ is a \emph{maximal ideal} iff $I \neq R$ and, whenever $J$ is an ideal with $I \subseteq J$, then either $I = J$ or $J = R$.
\end{df}

\chapter{Integral Domains}

\begin{df}[Integral Domain]
An \emph{integral domain} is a non-trivial commutative ring with no nonzero zero-divisors.
\end{df}

\begin{ex}
$\mathbb{Z}$, $\mathbb{Q}$, $\mathbb{R}$ and $\mathbb{C}$ are integral domains.
\end{ex}

\begin{prop}
$\mathbb{Z} / n \mathbb{Z}$ is an integral domain if and only if $n$ is prime.
\end{prop}

\begin{proof}
\pf
\begin{align*}
n \text{ is prime} & \Leftrightarrow \forall a,b \in \mathbb{Z} (n \mid a b \Rightarrow n \mid a \vee n \mid b) \\
& \Leftrightarrow \forall a,b \in \mathbb{Z} / n \mathbb{Z} (ab \cong 0 (\mod n) \Rightarrow a \cong 0 (\mod n) \vee b \cong 0 (\mod n)) \\
& \Leftrightarrow \mathbb{Z} / n \mathbb{Z} \text{ is an integral domain} & \qed
\end{align*}
\end{proof}

\begin{prop}
In an integral domain, if $x^2 = 1$ then $x = \pm 1$.
\end{prop}

\begin{proof}
\pf\ We have $x^2 - 1 = (x-1)(x+1) = 0$ so $x-1 = 0$ or $x+1 = 0$. \qed
\end{proof}

\begin{prop}
Let $R$ be an integral domain and $f,g \in R[x]$. Then
\[ \deg(fg) = \deg f + \deg g \]
\end{prop}

\begin{proof}
\pf
\step{1}{\pflet{$f = \sum_n a_n x^n$ and $g = \sum_n b_n x^n$.}}
\step{2}{\pflet{$d = \deg f$ and $e = \deg g$.}}
\step{3}{The $d + e$th term of $fg$ is
\[ a_d b_e x^{d + e} \]
which is non-zero.}
\step{4}{For $n > d + e$ the $n$th term of $fg$ is 0.}
\qed
\end{proof}

\begin{cor}
Let $R$ be a ring. Then $R[x]$ is an integral domain if and only if $R$ is an integral domain.
\end{cor}

\begin{prop}
Let $R$ be a ring. Then $R[[x]]$ is an integral domain if and only if $R$ is an integral domain.
\end{prop}

\begin{proof}
\pf
\step{1}{If $R[[x]]$ is an integral domain then $R$ is an integral domain.}
\begin{proof}
	\pf\ Easy.
\end{proof}
\step{2}{If $R$ is an integral domain then $R[[x]]$ is an integral domain.}
\begin{proof}
	\step{a}{\assume{$R$ is an integral domain.}}
	\step{b}{\pflet{$\left( \sum_n a_n x^n \right) \left( \sum_n b_n x^n \right) = 0$}}
	\step{c}{$a_0 b_0 = 0$}
	\step{d}{$a_0 = 0$ or $b_0 = 0$}
	\step{e}{\assume{w.l.o.g. $b_0 \neq 0$} \prove{For all $n$ we have $a_n = 0$}}
	\step{f}{\assume{as induction hypothesis $a_0 = a_1 = \cdots = a_{n-1} = 0$}}
	\step{g}{$\sum_{i=0}^n a_i b_{n-i} = 0$}
	\step{h}{$a_n b_0 = 0$}
	\step{i}{$a_n = 0$}
\end{proof}
\qed
\end{proof}

\begin{prop}
Let $R$ be a ring and $S$ an integral domain. Every rng homomorphism $\phi : R \rightarrow S$ is a ring homomorphism.
\end{prop}

\begin{proof}
\pf
\begin{align*}
\phi(1) & = \phi(1 \cdot 1) \\
& = \phi(1) \phi(1)
\end{align*}
and so $\phi(1) = 1$ by Cancellation. \qed
\end{proof}

\begin{prop}
The characteristic of an integral domain is either 0 or a prime number.
\end{prop}

\begin{proof}
\pf
\step{0}{\pflet{$D$ be an integral domain.}}
\step{1}{\pflet{$n$ be the characteristic of $D$}}
\step{2}{\assume{$n \neq 0$}}
\step{3}{\assume{$n = ab$}}
\step{4}{$ab = 0$ in $D$}
\step{5}{$a = 0$ or $b = 0$ in $D$}
\step{6}{$n \mid a$ or $n \mid b$}
\step{7}{One of $a$, $b$ is 1 and the other is $n$.}
\qed
\end{proof}

\section{Prime Ideals}

\begin{df}[Prime Ideal]
Let $I$ be an ideal in a commutative ring $R$. Then $I$ is a \emph{prime ideal} iff $R/I$ is an integral domain.
\end{df}

\begin{ex}
Let $R$ be a commutative ring and $a \in R$. Then $(x-a)$ is a prime ideal in $R$ iff $R$ is an integral domain.
\end{ex}

\begin{prop}
Let $R$ be a commutative ring and $I$ a proper ideal in $R$. Then $I$ is prime iff, whenever $ab \in I$, then $a \in I$ or $b \in I$.
\end{prop}

\begin{proof}
\pf\ The condition is the same as saying that, if $(a+I)(b+I) = I$, then $a+I = I$ or $b+I = I$. \qed
\end{proof}

\begin{df}[Spectrum]
The \emph{spectrum} of a commutative ring $R$, $\Spec R$, is the set of prime ideals.
\end{df}

\begin{prop}
Let $\phi : R \rightarrow S$ be a ring homomorphism. If $I$ is a prime ideal in $S$ then $\inv{\phi}(I)$ is a prime ideal in $R$.
\end{prop}

\begin{proof}
\pf
If $ab \in \inv{\phi}(I)$ then $\phi(a)\phi(b) \in I$ so either $\phi(a) \in I$ or $\phi(b) \in I$, i.e. either $a \in \inv{\phi}(I)$ or $b \in \inv{\phi}(I)$. \qed
\end{proof}

\begin{prop}
Let $R$ be a commutative ring. Suppose there exists a prime ideal $P$ in $R$ such that the only zero-divisor in $P$ is 0. Then $R$ is an integral domain.
\end{prop}

\begin{proof}
\pf
\step{1}{\assume{$ab = 0$ in $R$}}
\step{2}{$ab \in P$}
\step{3}{$a \in P$ or $b \in P$}
\step{4}{$a = 0$ or $b = 0$}
\qed
\end{proof}

\begin{prop}
Let $R$ be a commutative ring. The nilradical of $R$ is included in every prime ideal of $R$.
\end{prop}

\begin{proof}
\pf\ Let $P$ be a prime ideal. If $a^n = 0$ then $a^n \in P$ hence $a \in P$. \qed
\end{proof}

%TODO The nilradical is the intersection of all the prime ideals

\begin{df}[Krull Dimension]
The \emph{(Krull) dimension} of a commutative ring $R$ is the length of the longest chain of prime ideals in $R$.
\end{df}

\begin{ex}
$\mathbb{Z}[x]$ has Krull dimension 2. %TODO
\end{ex}

\chapter{Unique Factorization Domains}

%TODO Define this

\begin{ex}
$\mathbb{Z}$ is a UFD.
\end{ex}

\chapter{Noetherian Rings}

\begin{df}[Noetherian Ring]
A commutative ring is \emph{Noetherian} iff every ideal is finitely generated.
\end{df}

\begin{prop}
The homomorphic image of a Noetherian ring is Noetherian.
\end{prop}

\begin{proof}
\pf
\step{1}{\pflet{$R$ be a Noetherian ring, $S$ be a commutative ring, and $\phi : R \rightarrow S$ a surjective ring homomorphism.}}
\step{2}{\pflet{$I$ be an ideal in $S$.}}
\step{3}{\pflet{$\inv{\phi}(I) = (a_1, \ldots, a_n)$}}
\step{4}{$I = (\phi(a_1), \ldots, \phi(a_n))$}
\qed
\end{proof}

\chapter{Principal Ideal Domains}

\begin{df}[Principal Ideal Domain]
A commutative ring is a \emph{principal ideal domain} (\emph{PID}) iff every ideal is principal.
\end{df}

\begin{ex}
$\mathbb{Z}$ is a PID by Proposition \ref{prop:subgroups-of-Z}.
\end{ex}

\begin{ex}
$\mathbb{Z}[x]$ is not a PID. The ideal $(2,x)$ is not principal.
\end{ex}

\begin{prop}
Every PID is Noetherian.
\end{prop}

\begin{proof}
\pf\ Trivial. \qed
\end{proof}

\begin{prop}
Every nonzero prime ideal in a PID is maximal.
\end{prop}

\begin{proof}
\pf
\step{1}{\pflet{$R$ be a PID.}}
\step{2}{\pflet{$I$ be a nonzero prime ideal in $R$.}}
\step{3}{\pick\ $a \in R$ such that $I = (a)$.}
\step{4}{\pflet{$J$ be an ideal such that $I \subseteq J$}}
\step{5}{\pick\ $b \in R$ such that $J = (b)$.}
\step{6}{\pick\ $t \in R$ such that $a = bt$.}
\step{7}{$b \in I$ or $t \in I$}
\step{8}{\case{$b \in I$}}
\begin{proof}
	\pf\ Then $J \subseteq I$ so $I = J$.
\end{proof}
\step{9}{\case{$t \in I$}}
\begin{proof}
	\step{a}{\pick\ $s \in R$ such that $t = as$.}
	\step{b}{$a = ast$}
	\step{c}{$st = 1$}
	\begin{proof}
		\pf\ Since $R$ is an integral domain.
	\end{proof}
	\step{d}{$1 \in I$}
	\step{e}{$I = R$}
\end{proof}
\qed
\end{proof}

\begin{cor}
Any PID has Krull dimension 1.
\end{cor}

\chapter{Euclidean Domains}

%TODO Define this

\begin{ex}
$\mathbb{Z}$ is a Euclidean domain.
\end{ex}

\chapter{Division Rings}

\begin{df}[Division Ring]
A \emph{division ring} is a ring in which every nonzero element is a two-sided unit.
\end{df}

\begin{ex}
The quaternions form a division ring, with the inverse of a non-zero element $a + bi + cj + dk$ being
\[ \frac{1}{a^2 + b^2 + c^2 + d^2} (a - bi - cj - dk) \enspace . \]
\end{ex}

\begin{ex}
For any ring $R$, the ring of polynomials $R[x]$ is not a division ring, since $x$ has no inverse.
\end{ex}

\begin{prop}
Every centralizer in a division ring is a division ring.
\end{prop}

\begin{proof}
\pf\ If $ar = ra$ then $r\inv{a} = \inv{a}r$. \qed
\end{proof}

\begin{prop}
\label{prop:division-ring-ideals}
A non-trivial ring $R$ is a division ring if and only if its only left-ideals and right-ideals are $\{0\}$ and $R$.
\end{prop}

\begin{proof}
\pf
\step{1}{If $R$ is a division ring then the only left-ideals and right-ideals are $\{0\}$ and $R$.}
\begin{proof}
	\step{a}{\assume{$R$ is a division ring.}}
	\step{b}{The only left-ideals are $\{0\}$ and $R$.}
	\begin{proof}
		\step{i}{\pflet{$I$ be a left-ideal that is not $\{0\}$.} \prove{$I = R$}}
		\step{ii}{\pick\ $a \in I - \{0\}$}
		\step{iii}{\pick\ a left inverse $b$ for $a$}
		\step{iv}{$1 \in I$}
		\begin{proof}
			\pf\ Since $1 = ba$.
		\end{proof}
		\step{v}{$I = R$}
		\begin{proof}
			\pf\ For any $r \in R$ we have $r = r1 \in I$.
		\end{proof}
	\end{proof}
	\step{c}{The only right-ideals are $\{0\}$ and $R$.}
	\begin{proof}
		\pf\ Similar.
	\end{proof}
\end{proof}
\step{2}{If the only left-ideals and right-ideals are $\{0\}$ and $R$ then $R$ is a division ring.}
\qed
\end{proof}

\begin{prop}
Let $K$ be a division ring and $R$ a non-trivial ring. Every ring homomorphism $K \rightarrow R$ is injective.
\end{prop}

\begin{proof}
\pf
\step{1}{\pflet{$\phi : K \rightarrow R$ be a ring homomorphism.} \prove{$\ker \phi = \{0\}$}}
\step{2}{\pflet{$x \in \ker \phi$}}
\step{3}{\assume{for a contradiction $x \neq 0$.}}
\step{4}{$\phi(x \inv{x}) = 1$}
\step{5}{$0 = 1$}
\qedstep
\begin{proof}
\pf\ This contradicts the assumption that $R$ is non-trivial.
\end{proof}
\qed
\end{proof}

\chapter{Simple Rings}

\begin{df}[Simple Ring]
A non-trivial ring is $R$ \emph{simple} iff its only two-sided ideals are $\{0\}$ and $R$.
\end{df}

\begin{ex}
For any simple ring $R$ we have $\gl{n}{R}$ is simple, by Corollary \ref{cor:ideals-in-glnR}.
\end{ex}

\begin{prop}
\label{prop:maximal-iff-quotient-simple}
Let $R$ be a ring and $I$ an ideal in $R$. Then $I$ is maximal iff $R/I$ is simple.
\end{prop}

\begin{proof}
\pf
\begin{align*}
R / I \text{ is simple}
& \Leftrightarrow \text{the only ideals in } R / I \text{ are } \{I\} \text{ and } R / I \\
& \Leftrightarrow \text{the only ideals in } R \text{ that include } I \text{ are } I \text{ and } R \\
& \Leftrightarrow I \text{ is maximal}
\end{align*}
\qed
\end{proof}

\chapter{Reduced Rings}

\begin{df}[Reduced Ring]
A ring is \emph{reduced} iff it has no non-zero nilpotent elements.
\end{df}

\begin{prop}
Let $R$ be a commutative ring. Let $N$ be its nilradical. Then $R/N$ is reduced.
\end{prop}

\begin{proof}
\pf
\step{1}{\pflet{$r + N$ be nilpotent.}}
\step{2}{\pick\ $n$ such that $(r+N)^n = N$}
\step{3}{$r^n \in N$}
\step{4}{\pick\ $k$ such that $(r^n)^k = 0$}
\step{5}{$r^{nk} = 0$}
\step{6}{$r \in N$}
\step{7}{$r + N = N$}
\qed
\end{proof}

\begin{prop}
Let $R$ be a commutative ring. Let $I$ and $J$ be ideals in $R$. If $R/IJ$ is reduced then $IJ = I \cap J$.
\end{prop}

\begin{proof}
\pf
\step{1}{\pflet{$r \in I \cap J$} \prove{$r \in IJ$}}
\step{2}{$r^2 \in IJ$}
\step{3}{$(r + IJ)^2 = IJ$}
\step{4}{$r+IJ = IJ$}
\begin{proof}
	\pf\ Since $R/IJ$ is reduced.
\end{proof}
\step{5}{$r \in IJ$}
\qed
\end{proof}

\chapter{Boolean Rings}

\begin{df}[Boolean]
A ring is \emph{Boolean} iff $a^2 = a$ for every element $a$.
\end{df}

\begin{ex}
For any set $S$, the ring $\mathcal{P} S$ is Boolean.
\end{ex}

\begin{prop}
\label{prop:Boolean-char-2}
Every non-trivial Boolean ring has characteristic 2.
\end{prop}

\begin{proof}
\pf\ We have $4 = 2$ and so $2 = 0$. \qed
\end{proof}

\begin{prop}
Every Boolean ring is commutative.
\end{prop}

\begin{proof}
\pf
\begin{align*}
(a+b)^2 & = a+b \\
\therefore a^2 + ab + ba + b^2 & = a+b \\
\therefore a + ab + ba + b & = a + b \\
\therefore ab + ba & = 0 \\
\therefore ab & = -ba \\
& = ba & (\text{Proposition \ref{prop:Boolean-char-2}})
\end{align*}
\end{proof}

\begin{ex}
The only Boolean integral domain is $\mathbb{Z} / 2 \mathbb{Z}$. For, if $D$ is a Boolean integral domain and $x \in D$, we have $x^2 = x$, so $x^2 - x = x(x-1) = 0$ and so $x = 0$ or $x = 1$, i.e. $D = \{0,1\}$.
\end{ex}

\begin{prop}
Every Boolean ring has Krull dimension 0.
\end{prop}

\begin{proof}
\pf
\step{1}{\pflet{$R$ be a Boolean ring.}}
\step{2}{\pflet{$I$ be a prime ideal in $R$.} \prove{$I$ is maximal.}}
\step{3}{\pflet{$J$ be an ideal with $I \subsetneq J$}}
\step{4}{\pick\ $a \in J$ with $a \notin I$}
\step{5}{$a^2 - a = 0 \in I$}
\step{6}{$a(a-1) \in I$}
\step{7}{$a-1 \in I$}
\step{8}{$a-1 \in J$}
\step{9}{$1 \in J$}
\step{10}{$J = R$}
\qed
\end{proof}

\chapter{Modules}

\begin{df}[Left Module]
Let $R$ be a ring and $M$ an Abelian group. A \emph{left-action} of $R$ on $M$ is a ring homomorphism
\[ R \rightarrow \End{\Ab}{M} \enspace . \]
A \emph{left $R$-module} consists of an Abelian group $M$ and a left-action of $R$ on $M$.
\end{df}

\begin{prop}
Let $R$ be a ring and $M$ an Abelian group. Let $\cdot : R \times M \rightarrow M$. Then $\cdot$ defines a left-action of $R$ on $M$ if and only if, for all $r,s \in R$ and $m,n \in M$:
\begin{itemize}
\item $r(m+n) = rm + rn$
\item $(r+s)m = rm + sm$
\item $(rs)m = r(sm)$
\item $1m = m$
\end{itemize}
\end{prop}

\begin{proof}
\pf\ Immediate from definitions. \qed
\end{proof}

\begin{prop}
In any $R$-module $M$ we have $0m= 0$ for all $m \in M$.
\end{prop}

\begin{proof}
\pf\ Since $0m = (0+0)m = 0m+0m$ and so $0m = 0$ by cancellation in $M$. \qed
\end{proof}

\begin{prop}
In any $R$-module $M$ we have $(-1)m = -m$ for all $m \in M$.
\end{prop}

\begin{proof}
\pf\ Since $m + (-1)m = 1m + (-1)m = (1+(-1))m = 0m = 0$. \qed
\end{proof}

\begin{prop}
Every Abelian group is a $\mathbb{Z}$-module in exactly one way.
\end{prop}

\begin{proof}
\pf\ Since $\mathbb{Z}$ is initial in $\Ring$. \qed
\end{proof}

\begin{df}[Right Module]
Let $R$ be a ring. A \emph{right $R$-module} consists of an Abelian group $M$ and a function $\cdot : M \times R \rightarrow M$ such that, for all $r,s \in R$ and $m,n \in M$:
\begin{itemize}
\item $(m+n)r = mr + nr$
\item $m(r+s) = mr + ms$
\item $m(rs) = (mr)s$
\item $m1 = m$
\end{itemize}
\end{df}

\section{Homomorphisms}

\begin{df}[Homomorphism of Left-Modules]
Let $R$ be a ring. Let $M$ and $N$ be left-$R$-modules. A \emph{homomorphism of left-$R$-modules} $\phi : M \rightarrow N$ is a group homomorphism such that, for all $r \in R$ and $m \in M$, we have $\phi(rm) = r\phi(m)$.

Let $\Mod{R}$ be the category of left-$R$-modules and left-$R$-module homomorphisms.
\end{df}

\begin{ex}
\[ \Mod{\mathbb{Z}} \cong \Ab \]
\end{ex}

\begin{ex}
The trivial group $0$ is the zero object in $\Mod{R}$.
\end{ex}

\begin{prop}
Every bijective $R$-module homomorphism is an isomorphism.
\end{prop}

\begin{proof}
\pf\ Easy. \qed
\end{proof}

\begin{prop}
Let $R$ be a ring. Let $M$ be an $R$-module. Then
\[ M \cong \Mod{R}[R,M] \]
as $R$-modules.
\end{prop}

\begin{proof}
\pf\ The isomorphism maps $m$ to the function $\lambda r.rm$. Its inverse maps an $R$-module homomorphism $\alpha$ to $\alpha(1)$. \qed
\end{proof}

\begin{prop}
Let $R$ be a commutative ring. Let $M$ be an $R$-module. Then there is a bijection between the set of $R[x]$-module structures on $M$ that extend the given $R$-module structure and $\End{\Mod{R}}{M}$.
\end{prop}

\begin{proof}
\pf
\step{1}{\pflet{$\alpha : R \rightarrow \End{\Ab}{M}$ be the given $R$-module structure on $M$.}}
\step{2}{An $R[x]$-module structure on $M$ that extends $\alpha$ is a ring homomorphism $\beta : R[x] \rightarrow \End{\Ab}{M}$ such that $\beta \circ i = \alpha$, where $i$ is the inclusion $R \rightarrow R[x]$.}
\step{3}{There is a bijection between the $R[x]$-module structures on $M$ that extend $\alpha$ and the elements $s \in \End{\Ab}{M}$ that commute with $\alpha(r)$ for all $r \in R$.}
\begin{proof}
	\pf\ By the universal property for polynomials.
\end{proof}
\step{3}{There is a bijection between the $R[x]$-module structures on $M$ that extend $\alpha$ and the $R$-module homomorphisms $(M,\alpha) \rightarrow (M,\alpha)$.}
\qed
\end{proof}

\begin{prop}
Let $R$ be a commutative ring. Let $M$ and $N$ be $R$-modules. Then $\Mod{R}[M,N]$ is an $R$-module under
\begin{align*}
(\phi + \psi)(m) & = \phi(m) + \psi(m) \\
(r \phi)(m) & = r \phi(m)
\end{align*}
\end{prop}

\begin{proof}
\pf\ Easy. \qed
\end{proof}

\begin{prop}
Let $R$ be an integral domain. Let $I$ be a nonzero principal ideal of $R$. Then $I \cong R$ in $\Mod{R}$.
\end{prop}

\begin{proof}
\pf
\step{1}{\pick\ $a \in R$ such that $I = (a)$.}
\step{2}{\pflet{$\phi : R \rightarrow I$ be the map $\phi(r) = ra$.}}
\step{3}{$\phi$ is an $R$-module homomorphism.}
\begin{proof}
	\pf\ Since $(r+s)a = ra + sa$ and $(rs)a = r(sa)$.
\end{proof}
\step{4}{$\phi$ is surjective.}
\step{5}{$\phi$ is injective.}
\begin{proof}
	\pf\ If $ra = sa$ then $(r-s)a = 0$ so $r - s = 0$ and $r = s$.
\end{proof}
\step{6}{$\phi : R \cong I$}
\qed
\end{proof}

\section{Submodules}

\begin{df}[Submodule]
Let $M$ be a left-$R$-module and $N \subseteq M$. Then $N$ is a \emph{submodule} of $M$ iff $N$ is a subgroup of $M$ and $\forall r \in R. \forall n \in N. rn \in N$.
\end{df}

\begin{prop}
Let $R$ be a ring and $I \subseteq R$. Then $I$ is a left-ideal in $R$ iff $I$ is a submodule of $R$ as an $R$-module.
\end{prop}

\begin{proof}
\pf\ Immediate from definitions. \qed
\end{proof}

\begin{prop}
Let $R$ be a ring. Let $M$ and $N$ be left-$R$-modules and $\phi : M \rightarrow N$ an $R$-module homomorphism. Then $\ker \phi$ is a submodule of $M$ and $\im \phi$ is a submodule of $N$.
\end{prop}

\begin{proof}
\pf\ Easy. \qed
\end{proof}

\begin{prop}
Let $R$ be a commutative ring. Let $M$ be a left-$R$-module. Let $r \in R$. Then $rM = \{ rm : m \in M \}$ is a submodule of $M$.
\end{prop}

\begin{proof}
\pf\ Easy. \qed
\end{proof}

\begin{prop}
Let $R$ be a ring. Let $M$ be a left-$R$-module. Let $I$ be a left-ideal in $R$. Then $IM = \{ rm : r \in I, m \in M \}$ is a submodule of $M$.
\end{prop}

\begin{proof}
\pf
\step{1}{$IM$ is a subgroup of $M$.}
\begin{proof}
	\step{a}{\pflet{$r,s \in I$ and $m,n \in M$.} \prove{$rm + sn \in IM$}}
	\step{b}{$rm+sn = r(m-n) + (s-r)n$}
\end{proof}
\step{2}{For all $r \in R$ and $x \in IM$ we have $rx \in IM$.}
\qed
\end{proof}

\section{Quotient Modules}

\begin{df}[Quotient Module]
Let $R$ be a ring. Let $M$ be a left-$R$-module. Let $N$ be a submodule of $M$. Then the \emph{quotient module} $M/N$ is the quotient group $M/N$ under
\[ r(m+N) = rm+N \enspace . \]
\end{df}

\begin{prop}
Let $R$ be a ring. Let $M$ and $P$ be left-$R$-modules. Let $N$ be a submodule of $M$. Let $\phi : M \rightarrow P$ be an $R$-module homomorphism. If $N \subseteq \ker \phi$, then there exists a unique $R$-module homomorphism $\overline{\phi} : M / N \rightarrow P$ such that the following diagram commutes.
\[ \begin{tikzcd}
M \arrow[rr,"\phi"] \arrow[dr] & & P \\
& M/N \arrow[ur,"\overline{\phi}"]
\end{tikzcd} \]
\end{prop}

\begin{proof}
\pf\ Easy. \qed
\end{proof}

\begin{thm}
Every $R$-module homomorphism $\phi : M \rightarrow M'$ may be decomposed as:
\[ \begin{tikzcd}
M \arrow[r] & M / \ker \phi \arrow[r,"\cong"] & \im \phi \arrow[r] & N 
\end{tikzcd} \]
\end{thm}

\begin{proof}
\pf\ Easy. \qed
\end{proof}

\begin{cor}[First Isomorphism Theorem]
Let $\phi : M \rightarrow M'$ be a surjective $R$-module homomorphism. Then
\[ M' \cong \frac{M}{\ker \phi} \enspace . \]
\end{cor}

\begin{prop}[Second Isomorphism Theorem]
Let $R$ be a ring. Let $M$ be a left-$R$-module. Let $N$ and $P$ be submodules of $M$. Then $N+P$ is a submodule of $M$, $N \cap P$ is a submodule of $P$, and
\[ \frac{N+P}{N} \cong \frac{P}{N \cap P} \]
\end{prop}

\begin{proof}
\pf\ The function that maps $P$ to $p + N$ is a surjective homomorphism $P \rightarrow (N+P)/N$ with kernel $N \cap P$. \qed
\end{proof}

\begin{prop}[Third Isomorphism Theorem]
Let $R$ be a ring. Let $M$ be a left-$R$-module. Let $N$ be a submodule of $M$ and $P$ a submodule of $N$. Then $N/P$ is a submodule of $M/P$ and
\[ \frac{M/P}{N/P} \cong \frac{M}{N} \]
\end{prop}

\begin{proof}
\pf\ The canonical map $M \rightarrow M/N$ induces a surjective homomorphism $M/P \rightarrow M/N$ which has kernel $N/P$. \qed
\end{proof}

\begin{prop}
Let $R$ be a ring. Let $M$ be a left-$R$-module.
The sum and intersection of a family of submodules of $M$ are submodules of $M$.
\end{prop}

\begin{proof}
\pf\ Easy. \qed
\end{proof}

\section{Products}

\begin{prop}
$\Mod{R}$ has products.
\end{prop}

\begin{proof}
\pf\ Given a family $\{ M_\alpha \}_{\alpha \in A}$ of left-$R$-modules, we make $\prod_{\alpha \in A} M_\alpha$ into a left-$R$-module by
\begin{align*}
(f + g)(\alpha) & = f(\alpha) + g(\alpha) \\
(rf)(\alpha) & = rf(\alpha) & \qed
\end{align*}
\end{proof}

\section{Coproducts}

\begin{prop}
$\Mod{R}$ has coproducts.
\end{prop}

\begin{proof}
\pf\ Given a family $\{M_\alpha \}_{\alpha \in A}$ of left-$R$-modules, take $\bigoplus_{\alpha \in A} M_\alpha$ to be $\{ f \in \prod_{\alpha \in A} M_\alpha : f(\alpha) = 0 \text{ for all but finitely many } \alpha \in A \}$. \qed
\end{proof}

\section{Direct Sum}

\begin{df}[Direct Sum]
Let $R$ be a ring. Let $M$ and $N$ be left-$R$-modules. Then the direct sum $M \oplus N$ is an $R$-module under
\[ r(m,n) = (rm,rn) \enspace . \]
\end{df}

\begin{prop}
$M \oplus N$ is the biproduct of $M$ and $N$ in $\Mod{R}$.
\end{prop}

\begin{proof}
\pf\ Easy. \qed
\end{proof}

\begin{ex}
Infinite products and coproducts are in general different. We have $\mathbb{Z}^{\mathbb{N}} \not\cong \mathbb{Z}^{\oplus \mathbb{N}}$ since $\mathbb{Z}^{\mathbb{N}}$ is uncountable but $\mathbb{Z}^{\oplus \mathbb{N}}$ is countable.
\end{ex}

\section{Kernels and Cokernels}

\begin{prop}
Let $R$ be a ring.
Let $\phi : M \rightarrow N$ be a left-$R$-module homomorphism. Then $\ker \phi \hookrightarrow M$ is terminal in the category of left-$R$-module homomorphisms $\alpha : P \rightarrow M$ such that $\phi \circ \alpha = 0$.
\end{prop}

\begin{proof}
\pf\ Easy. \qed
\end{proof}

\begin{prop}
Let $R$ be a ring.
Let $\phi : M \rightarrow N$ be a left-$R$-module homomorphism. Then $N \twoheadrightarrow \coker \phi$ is initial in the category of left-$R$-module homomorphisms $\alpha : N \rightarrow P$ such that $\alpha \circ \phi = 0$.
\end{prop}

\begin{proof}
\pf\ Easy. \qed
\end{proof}

\begin{prop}
Let $R$ be a ring. Let $\phi : M \rightarrow N$ be a left-$R$-module homomorphism. Then the following are equivalent.
\begin{itemize}
\item $\phi$ is a monomorphism.
\item $\ker \phi$ is trivial.
\item $\phi$ is injective.
\end{itemize}
\end{prop}

\begin{proof}
\pf\ Easy. \qed
\end{proof}

\begin{prop}
Let $R$ be a ring. Let $\phi : M \rightarrow N$ be a left-$R$-module homomorphism. Then the following are equivalent.
\begin{itemize}
\item $\phi$ is an epimorphism.
\item $\coker \phi$ is trivial.
\item $\phi$ is surjective.
\end{itemize}
\end{prop}

\begin{proof}
\pf\ Easy. \qed
\end{proof}

\begin{prop}
Every monomorphism in $\Mod{R}$ is the kernel of some homomorphism.
\end{prop}

\begin{proof}
\pf\ If $\phi : M \rightarrow N$ is a monomorphism then it is the kernel of $N \twoheadrightarrow N / \im \phi$. \qed
\end{proof}

\begin{prop}
Every epimorphism in $\Mod{R}$ is the cokernel of some homomorphism.
\end{prop}

\begin{proof}
\pf\ If $\phi : M \rightarrow N$ is epi then it is the cokernel of $\ker \phi \hookrightarrow M$. \qed
\end{proof}

\begin{ex}
Monomorphisms do not split in $\Mod{R}$. Multiplication by 2 is a monomorphism $\mathbb{Z} \rightarrow \mathbb{Z}$ but has no left inverse.
\end{ex}

\begin{ex}
Epimorphisms do not split in $\Mod{R}$. The canonical map $\mathbb{Z} \rightarrow \mathbb{Z} / 2 \mathbb{Z}$ is an epimorphism without a right inverse.
\end{ex}

\section{Free Modules}

\begin{prop}
Let $R$ be a ring and $A$ a set. Then there exists a left-$R$-module $F^R(A)$ and function $j : A \rightarrow F^R(A)$ such that, for any left-$R$-module $M$ and function $f : A \rightarrow M$, there exists a unique left-$R$-module homomorphism $\overline{f} : F^R(A) \rightarrow M$ such that the following diagram commutes.
\[ \begin{tikzcd}
F^R(A) \arrow[r,"\overline{f}"] & M \\
A \arrow[u,"j"] \arrow[ur,"f"]
\end{tikzcd} \]
\end{prop}

\begin{proof}
\pf
\step{1}{\pflet{$R^{\oplus A} = \{ \alpha : A \rightarrow R : \alpha(a) = 0 \text{ for all but finitely many } a \in A \}$ under the operations
\begin{align*}
(\alpha + \beta)(a) & = \alpha(a) + \beta(a) \\
(r \alpha)(a) & = r\alpha(a)
\end{align*}}}
\step{2}{$R^{\oplus A}$ is a left-$R$-module.}
\step{3}{\pflet{$j : A \rightarrow R^{\oplus A}$ be the function
\[ j(a)(a') = \begin{cases}
1 & \text{if } a = a' \\
0 & \text{if } a \neq a'
\end{cases} \]}}
\step{4}{\pflet{$M$ be any left-$R$-module.}}
\step{5}{\pflet{$f : A \rightarrow M$ be a function.}}
\step{6}{\pflet{$\overline{f} : R^{\oplus A} \rightarrow M$ be the function
\[ \overline{f}(\alpha) = \sum_{a \in A, \alpha(a) \neq 0} \alpha(a) f(a) \]}}
\step{7}{$\overline{f}$ is a left-$R$-module homomorphism.}
\step{8}{$\overline{f} \circ j = f$}
\step{9}{$\overline{f}$ is unique.}
\end{proof}

\begin{df}
We call $j : A \rightarrow F^R(A)$ the \emph{free} left-$R$-module over $A$.
\end{df}

\begin{prop}
$j$ is injective.
\end{prop}

\begin{proof}
\pf\ By the proof of the previous proposition. \qed
\end{proof}

\begin{prop}
Let $R$ be a ring. Let $F$ be a non-zero free left-$R$-module. Let $\phi : M \rightarrow N$ be a left-$R$-module homomorphism. Then $\phi$ is onto if and only if, for every left-$R$-module homomorphism $\alpha : F \rightarrow N$, there exists a left-$R$-module homomorphism $\beta : F \rightarrow M$ such that the diagram below commutes.
\[ \begin{tikzcd}
M \arrow[r,"\phi"] & N \\
F \arrow[u,"\beta"] \arrow[ur,"\alpha"]
\end{tikzcd} \]
\end{prop}

\begin{proof}
\pf
\step{0}{\pflet{$F$ be the free left-$R$-module over $A$ with injection $j : A \rightarrow F$.}}
\step{1}{If $\phi$ is onto then, for every homomorphism $\alpha : F \rightarrow N$, there exists a homomorphism $\beta : F \rightarrow M$ such that $\phi \circ \beta = \alpha$.}
\begin{proof}
	\step{a}{\assume{$\phi$ is onto.}}
	\step{b}{\pflet{$\alpha : F \rightarrow N$ be a homomorphism.}}
	\step{c}{For $a \in A$, \pick\ $f(a) \in M$ such that $\phi(f(a)) = \alpha(j(a))$}
	\step{d}{\pflet{$\beta : F \rightarrow M$ be the unique homomorphism such that $\beta \circ j = f$}}
	\step{e}{$\phi \circ \beta = \alpha$}
	\begin{proof}
		\pf\ Each is the unique homomorphism such that $\alpha \circ j = \phi \circ f$.
	\end{proof}
	\qed
	\[ \begin{tikzcd}
& 	M \arrow[r,"\phi"] & N \\
A \arrow[ur,"f"] \arrow[r,"j"] & F \arrow[u,"\beta"] \arrow[ur,"\alpha"]
\end{tikzcd} \]	
\end{proof}
\step{2}{If, for every homomorphism $\alpha : F \rightarrow N$, there exists a homomorphism $\beta : F \rightarrow M$ such that $\phi \circ \beta = \alpha$, then $\phi$ is onto.}
\begin{proof}
	\step{a}{\assume{For every homomorphism $\alpha : F \rightarrow N$ there exists a homomorphism $\beta : F \rightarrow M$ such that $\phi \circ \alpha = \beta$.}}
	\step{b}{\pflet{$n \in N$}}
	\step{c}{\pflet{$\alpha : F \rightarrow N$ be the unique homomorphism such that, for all $a \in A$, we have $\alpha(j(a)) = n$}}
	\step{d}{\pick\ a homomorphism $\beta : F \rightarrow M$ such that $\phi \circ \beta = \alpha$}
	\step{e}{\pick\ $a \in A$}
	\step{f}{$\phi(\beta(j(a))) = n$}
\end{proof}
\qed
\end{proof}

\section{Generators}

\begin{df}[Submodule Generated by a Set]
Let $R$ be a ring. Let $M$ be a left-$R$-module. Let $A$ be a subset of $M$. Let $\phi_A : F^R(A) \rightarrow M$ be the unique left-$R$-module homomorphism such that the following diagram commutes.
\[ \begin{tikzcd}
F^R(A) \arrow[r,"\phi_A"] & M \\
A \arrow[u] \arrow[ur,hook]
\end{tikzcd} \]
The submodule of $M$ \emph{generated} by $A$, denoted $\langle A \rangle$, is defined to be $\im \phi_A$.
\end{df}

\begin{df}[Finitely Generated]
Let $R$ be a ring. Let $M$ be a left-$R$-module. Then $M$ is \emph{finitely generated} iff there exists a finite set $A \subseteq M$ such that $M = \langle A \rangle$.
\end{df}

\begin{ex}
A submodule of a finitely generated module is not necessarily finitely generated.

Let $R = \mathbb{Z}[x_1, x_2, \ldots]$. Then $R$ is finitely generated as an $R$-module, but $(x_1, x_2, \ldots)$ is not.
\end{ex}

\begin{prop}
The homomorphic image of a finitely generated module is finitely generated.
\end{prop}

\begin{proof}
\pf\ Easy.  \qed
\end{proof}

\begin{prop}
Let $R$ be a ring. Let $M$ be a left-$R$-module. Let $N$ be a submodule of $M$. If $N$ and $M/N$ are finitely generated then $M$ is finitely generated.
\end{prop}

\begin{proof}
\pf
\step{1}{\pick\ $a_1$, \ldots, $a_n$ that generate $N$.}
\step{2}{\pick\ $b_1$, \ldots, $b_m$ such that $b_1 + N$, \ldots, $b_m + N$ generate $M/N$. \prove{$a_1$, \ldots, $a_n$, $b_1$, \ldots, $b_m$ generate $M$.}}
\step{3}{\pflet{$m \in M$}}
\step{4}{\pick\ $r_1, \ldots, r_m \in R$ such that $m + N = r_1 b_1 + \cdots + r_m b_m + N$}
\step{5}{$m - r_1b_1 - \cdots - r_m b_m \in N$}
\step{6}{\pick\ $s_1, \ldots, s_n \in R$ such that $m - r_1 b_1 - \cdots - r_m b_m = s_1 a_1 + \cdots + s_n a_n$}
\step{7}{$m = r_1 b_1 + \cdots + r_m b_m + s_1 a_1 + \cdots + s_n a_n$}
\qed
\end{proof}

\section{Projections}

\begin{df}[Projection]
Let $R$ be a ring. Let $M$ be a left-$R$-module. Let $p : M \rightarrow M$ be a left-$R$-module homomorphism. Then $p$ is a \emph{projection} iff $p^2 = p$.
\end{df}

\begin{prop}
Let $R$ be a ring. Let $M$ be a left-$R$-module. Let $p : M \rightarrow M$ be a projection. Then
\[ M \cong \ker p \oplus \im p \enspace . \]
\end{prop}

\begin{proof}
\pf
\step{1}{\pflet{$\phi : M \rightarrow \ker p \oplus \im p$ be the map $\phi(m) = (m - p(m), p(m))$}}
\step{2}{$\phi$ is a left-$R$-module homomorphism.}
\step{3}{$\phi$ is injective.}
\step{4}{$\phi$ is surjective.}
\qed
\end{proof}

\section{Pullbacks}

\begin{prop}
$\Mod{R}$ has pullbacks.
\end{prop}

\begin{proof}
\pf
\step{1}{\pflet{$\mu : M \rightarrow Z$, $\nu : N \rightarrow Z$ be left-$R$-module homomorphisms.}}
\step{2}{\pflet{$M \times_Z N = \{ (m,n) \in M \times N : \mu(m) = \nu(n) \}$ under
\begin{align*}
(m,n) + (m',n') & = (m+m',n+n') \\
r(m,n) & = (rm,rn)
\end{align*}}}
\step{3}{$M \times_Z N$ is the pullback of $M$ and $N$.}
\qed
\end{proof}

\section{Pushouts}

\begin{prop}
$\Mod{R}$ has pushouts.
\end{prop}

\begin{proof}
\pf
\step{1}{\pflet{$\mu : A \rightarrow M$ and $\nu : A \rightarrow N$ be left-$R$-module homomorphisms.}}
\end{proof}

\chapter{Cyclic Modules}

\begin{df}[Cyclic Module]
Let $R$ be a ring. Let $M$ be a left-$R$-module. Then $M$ is \emph{cyclic} iff there exists $m \in M$ such that $M = \langle m \rangle$.
\end{df}

\begin{prop}
Let $R$ be a ring. Let $M$ be a left-$R$-module. Then $M$ is cyclic if and only if there exists a left-ideal $I$ in $R$ such that $M \cong R / I$.
\end{prop}

\begin{proof}
\pf
\step{1}{If $M$ is cyclic then there exists a left-ideal $I$ in $R$ such that $M \cong R / I$.}
\begin{proof}
	\step{a}{\assume{$M$ is cyclic.}}
	\step{b}{\pick\ $m \in M$ such that $M = \langle m \rangle$}
	\step{c}{\pflet{$\phi : R \rightarrow M$ be the left-$R$-module homomorphism $\phi(r) = rm$.}}
	\step{d}{$\phi$ is surjective.}
	\step{e}{$M \cong R / \ker \phi$}
\end{proof}
\step{2}{For every left-ideal $I$ in $R$, we have that $R/I$ is cyclic.}
\begin{proof}
	\pf\ $R/I$ is generated by $1+I$.
\end{proof}
\qed
\end{proof}

\begin{prop}
A quotient of a cyclic module is cyclic.
\end{prop}

\begin{proof}
\pf\ If $M$ is generated by $m$ then $M/N$ is generated by $m+N$. \qed
\end{proof}

\begin{prop}
Let $R$ be a ring.
For any left-ideal $I$ in $R$ and any left-$R$-module $N$, we have
\[ \Mod{R}[R/I,N] \cong \{ n \in N : \forall a \in I. an = 0 \} \enspace . \]
\end{prop}

\begin{proof}
\pf
\step{1}{\pflet{$\Phi : \Mod{R}[R/I,N] \rightarrow \{n \in N : \forall a \in I. an = 0\}$ be the function $\Phi(\alpha) = \alpha(1 + I)$}}
\begin{proof}
	\pf\ For all $a \in I$ we have $a \alpha(1 + I) = \alpha(a +I) = \alpha(I) = 0$.
\end{proof}
\step{2}{$\Phi$ is injective.}
\begin{proof}
	\pf\ If $\alpha(1+I) = \beta(1+I)$ then $\alpha(r+I) = r\alpha(1+I) = r\beta(1+I) = \beta(r+I)$ for all $r \in R$, hence $\alpha = \beta$.
\end{proof}
\step{3}{$\Phi$ is surjective.}
\begin{proof}
	\pf\ Given $n \in N$ such that $\forall a \in I. an = 0$, define $\alpha : R/I \rightarrow N$ by $\alpha(r+I) = rn$.
\end{proof}
\step{4}{If $R$ is commutative then $\Phi$ is an $R$-module homomorphism.}
\qed
\end{proof}

\begin{cor}
For all $a,b \in \mathbb{Z}$ we have $\Ab[\mathbb{Z} / a \mathbb{Z}, \mathbb{Z} / b \mathbb{Z}] \cong \mathbb{Z} / \gcd(a,b) \mathbb{Z}$.
\end{cor}

\begin{proof}
\pf
\begin{align*}
\Ab[\mathbb{Z} / a \mathbb{Z}, \mathbb{Z} / b \mathbb{Z}]
& \cong \Mod{\mathbb{Z}}[\mathbb{Z} / a \mathbb{Z}, \mathbb{Z} / b \mathbb{Z}] \\
& \cong \{ n \in \mathbb{Z} / b \mathbb{Z} : \forall x \in a \mathbb{Z}. xn \cong 0 (\mod b) \} \\
& \cong \{ n \in \mathbb{Z} / b \mathbb{Z} : \forall x \in \mathbb{Z}. b \mid xan \} \\
& = \{ n \in \mathbb{Z} / b \mathbb{Z} : b \mid an \}
\end{align*}
\end{proof}
\chapter{Simple Modules}

\begin{df}[Simple Module]
Let $R$ be a ring.
An $R$-module $M$ is \emph{simple} or \emph{irreducible} iff its only submodules are $\{0\}$ and $M$.
\end{df}

\begin{prop}[Schur's Lemma]
Let $R$ be a ring. Let $M$ and $N$ be simple $R$-modules. Let $\phi : M \rightarrow N$ be an $R$-module homomorphism. Then either $\phi = 0$ or $\phi$ is an isomorphism.
\end{prop}

\begin{proof}
\pf
\step{1}{\assume{$\phi \neq 0$}}
\step{3}{$\ker \phi = 0$}
\begin{proof}
	\pf\ Since $\ker \phi$ is a submodule of $M$ that is not $M$.
\end{proof}
\step{4}{$\im \phi = N$}
\begin{proof}
	\pf\ Since $\im \phi$ is a submodule of $N$ that is not $\{0\}$.
\end{proof}
\qed
\end{proof}

\begin{prop}
Every simple module is cyclic.
\end{prop}

\begin{proof}
\pf
\step{1}{\pflet{$M$ be a simple module.}}
\step{2}{\assume{w.l.o.g. $M \neq \{0\}$}}
\begin{proof}
	\pf\ $\{0\} = \langle 0 \rangle$ is cyclic.
\end{proof}
\step{3}{\pick\ $m \in M$ with $m \ne 0$}
\step{4}{$\langle m \rangle = M$}
\begin{proof}
	\pf\ Since $\langle m \rangle$ is a submodule of $M$ that is not $\{0\}$.
\end{proof}
\qed
\end{proof}

\chapter{Noetherian Modules}

\begin{df}[Noetherian Module]
Let $R$ be a ring. A left-$R$-module is \emph{Noetherian} iff every submodule is finitely generated.
\end{df}

\begin{prop}
Let $R$ be a ring. Let $M$ be a left-$R$-module and $N$ a submodule of $M$. Then $M$ is Noetherian if and only if $N$ and $M/N$ are Noetherian.
\end{prop}

\begin{proof}
\pf
\step{1}{If $M$ is Noetherian then $N$ is Noetherian.}
\begin{proof}
	\pf\ Every submodule of $N$ is a submodule of $M$, hence finitely generated.
\end{proof}
\step{2}{If $M$ is Noetherian then $M/N$ is Noetherian.}
\begin{proof}
	\step{a}{\assume{$M$ is Noetherian.}}
	\step{b}{\pflet{$\pi : M \twoheadrightarrow M / N$ be the canonical epimorphism.}}
	\step{c}{\pflet{$P$ be a submodule of $M/N$.}}
	\step{d}{\pick\ $a_1, \ldots, a_n \in M$ that generate $\inv{\pi}(P)$.}
	\step{e}{$a_1 + N$, \ldots, $a_n + N$ generate $P$.}
\end{proof}
\step{3}{If $N$ and $M/N$ are Noetherian then $M$ is Noetherian.}
\begin{proof}
	\step{a}{\assume{$N$ and $M/N$ are Noetherian.}}
	\step{b}{\pflet{$P$ be a submodule of $M$.}}
	\step{c}{\pick\ $a_1, \ldots, a_m \in P$ such that $a_1 + N$, \ldots, $a_m + N$ generate $\pi(P)$.}
	\step{d}{\pick\ $b_1, \ldots, b_n \in M$ that generated $P \cap N$. \prove{$a_1, \ldots, a_m, b_1, \ldots, b_n$ generate $P$.}}
	\step{e}{\pflet{$p \in P$}}
	\step{f}{\pick\ $r_1, \ldots, r_m \in R$ such that $p + N = r_1 a_1 + \cdots + r_m a_m + N$}
	\step{g}{$p - r_1 a_1 - \cdots - r_m a_m \in P \cap N$}
	\step{h}{\pick\ $s_1, \ldots, s_n \in R$ such that $p - r_1 a_1 - \cdots - r_m a_m = s_1 b_1 + \cdots + s_n b_n$}
	\step{i}{$p = r_1 a_1 + \cdots + r_m a_m + s_1 b_1 + \cdots + s_n b_n$}
\end{proof}
\qed
\end{proof}

\begin{cor}
If $R$ is a Noetherian ring then $R^{\oplus n}$ is a Noetherian left-$R$-module.
\end{cor}

\begin{proof}
\pf\ The proof is by induction on $n$. The case $n = 1$ is immediate.

The induction step holds since $R^{\oplus (n+1)} / R^{\oplus n} \cong R$. \qed
\end{proof}

\begin{cor}
If $R$ is a Noetherian ring and $M$ is a finitely generated left-$R$-module then $M$ is Noetherian.
\end{cor}

\begin{proof}
\pf\ There is a surjective homomorphism $R^{\oplus n} \twoheadrightarrow M$ for some $n$, so $M$ is a quotient of $R^{\oplus n}$. \qed
\end{proof}

\chapter{Algebras}

\begin{df}[Algebra]
Let $R$ be a commutative ring. An \emph{$R$-algebra} consists of a ring $S$ and a ring homomorphism $\alpha : R \rightarrow S$ such that $\alpha(R)$ is included in the center of $S$. We write $rs$ for $\alpha(r)s$.
\end{df}

\begin{prop}
Let $R$ be a commutative ring and $S$ a ring. Let $\cdot : R \times S \rightarrow S$. Then there exists $\alpha : R \rightarrow S$ that makes $S$ into an $R$-algebra such that
\[ rs = \alpha(r)s \qquad (r \in R, s \in S) \]
iff $S$ is an $R$-module under $\cdot$ and, for all $r_1,r_2 \in R$ and $s_1,s_2 \in S$,
\[ (r_1 s_1)(r_2 s_2) = (r_1 r_2)(s_1 s_2) \enspace . \]
\end{prop}

\begin{proof}
\pf\ Immediate from definitions. \qed
\end{proof}

\begin{ex}
Let $R$ be a commutative ring. Then $R$ is an $R$-algebra under multiplication.
\end{ex}

\begin{ex}
Let $R$ be a commutative ring and $I$ an ideal in $R$. Then $R/I$ is an $R$-algebra.
\end{ex}

\begin{ex}
Let $R$ be a commutative ring and $M$ an $R$-module. Then $\End{\Mod{R}}{M}$ is an $R$-algebra under composition.
\end{ex}

\begin{ex}
Let $R$ be a commutative ring. Then $\gl{n}{R}$ is an $R$-algebra under matrix multiplication.
\end{ex}

\begin{df}[Algebra Homomorphism]
Let $R$ be a commutative ring. Let $S$ and $T$ be $R$-algebras. An \emph{$R$-algebra homomorphism} $\phi : S \rightarrow T$ is a ring homomorphism such that, for all $r \in R$ and $s \in S$, we have $\phi(rs) = r\phi(s)$.

Let $\Alg{R}$ be the category of $R$-algebras and $R$-algebra homomorphisms.
\end{df}

\begin{ex}
\[ \Alg{\mathbb{Z}} \cong \Ring \]
\end{ex}

\begin{ex}
Let $R$ be a commutative ring. Then $R[x_1, \ldots, x_n]$, and any quotient ring of $R[x_1, \ldots, x_n]$, is a commutative $R$-algebra.
\end{ex}

\begin{ex}
$R$ is the initial object in $\Alg{R}$.
\end{ex}

\section{Rees Algebra}

\begin{df}[Rees Algebra]
Let $R$ be a commutative ring. Let $I$ be an ideal in $R$. The \emph{Rees algebra} is the direct sum
\[ \mathrm{Rees}_R(I) = \bigoplus_{j \geq 0} I^j \]
under the multiplication 
\begin{align*}
(r_0, r_1, r_2, r_3, \ldots)(s_0, s_1, s_2, \ldots) & = (r_0s_0, r_1s_0 + r_0s_1, r_0s_2 + r_1s_1 + r_2s_0, \ldots) \\
r(r_0, r_1, r_2, \ldots) & = (rr_0, rr_1, rr_2, \ldots)
\end{align*}
\end{df}

\begin{prop}
Let $R$ be a commutative ring. Let $a \in R$ be a non-zero-divisor. Then $R[x]$ is the Rees algebra of $(a)$.
\end{prop}

\begin{proof}
\pf
\step{1}{\pflet{$\phi : R[x] \rightarrow \mathrm{Rees}_R((a))$ be the function $\phi(r_0 + r_1 x + r_2 x^2 + \cdots) = (r_0, r_1 a, r_2 a^2, \ldots)$.}}
\step{2}{$\phi$ is an $R$-algebra homomorphism.}
\step{3}{$\phi$ is injective.}
\begin{proof}
	\step{a}{\pflet{$\phi(r_0 + r_1 x + r_2 x^2 + \cdots) = \phi(s_0 + s_1 x + s_2 x^2 + \cdots)$}}
	\step{b}{For all $n$ we have $r_n a^n = s_n a^n$}
	\step{c}{$(r_n - s_n)a^n = 0$}
	\step{d}{$r_n - s_n = 0$}
	\begin{proof}
		\pf\ Since $a$ is not a zero-divisor.
	\end{proof}
	\step{e}{$r_n = s_n$}
\end{proof}
\step{4}{$\phi$ is surjective.}
\qed
\end{proof}

\begin{prop}
Let $R$ be a commutative ring. Let $a \in R$ be a non-zero-divisor. Let $I$ be an ideal of $R$. Then $\mathrm{Rees}_R(I) \cong \mathrm{Rees}_R(aI)$.
\end{prop}

\begin{proof}
\pf
\step{1}{\pflet{$\phi : \mathrm{Rees}_R(I) \rightarrow \mathrm{Rees}_R(aI)$ be the function $\phi(r_0, r_1, r_2, \ldots) = (r_0, ar_1, a^2r_2, \ldots)$.}}
\step{2}{$\phi$ is an $R$-algebra homomorphism.}
\step{3}{$\phi$ is injective.}
\step{4}{$\phi$ is surjective.}
\qed
\end{proof}

\section{Free Algebras}

\begin{prop}
Let $R$ be a ring. Then $R[x_1, \ldots, x_n]$ is the free commutative $R$-algebra on $\{ 1, \ldots, n \}$.
\end{prop}

\begin{proof}
\pf\ Easy. \qed
\end{proof}

\begin{prop}
Let $R$ be a ring and $A$ a set. Let $A^*$ be the free monoid on $A$. Then the monoid ring $R[A^*]$ is the free $R$-algebra on $A$.
\end{prop}

\begin{proof}
\pf\ Easy. \qed
\end{proof}

\begin{prop}
Let $R$ be a commutative ring and $S$ a commutative $R$-algebra. Then $S$ is finitely generated as an $R$-algebra if and only if $S$ is finitely generated as a commutative $R$-algebra.
\end{prop}

\begin{proof}
\pf\ Since a subalgebra of a commutative subalgebra is commutative, so the smallest algebra that contains $\{a_1, \ldots, a_n\}$ is the smallest commutative subalgebra that contains $\{a_1, \ldots, a_n\}$. \qed
\end{proof}

\chapter{Algebras of Finite Type}

\begin{df}[Algebra of Finite Type]
Let $R$ be a ring. Let $S$ be an $R$-algebra. Then $R$ is of \emph{finite type} iff $S$ is a finitely generated $R$-algebra.
\end{df}

\begin{prop}
Let $R$ be a Noetherian ring. Let $S$ be a finite-type $R$-algebra. Then $S$ is a Noetherian ring.
\end{prop}

%TODO

\chapter{Finite Algebras}

\begin{df}[Finite Algebra]
Let $R$ be a ring. Let $S$ be an $R$-algebra. Then $S$ is a \emph{finite} $R$-algebra iff it is a finitely generated left-$R$-module.
\end{df}

\begin{prop}
Let $R$ be a ring. Every finite $R$-algebra is of finite type.
\end{prop}

\begin{proof}
\pf\ If $S$ is generated by $a_1$, \ldots, $a_n$ as an $R$-module, then it is generated by $a_1$, \ldots, $a_n$ as an $R$-algebra. \qed
\end{proof}

\begin{ex}
The converse does not hold. $R[x]$ is of finite type but is not finite.
\end{ex}

\chapter{Division Algebras}

\begin{df}[Division Algebra]
Let $R$ be a commutative ring.
A \emph{division $R$-algebra} is an $R$-algebra that is a division ring.
\end{df}

\begin{ex}
Let $R$ be a commutative ring. Let $M$ be a simple $R$-algebra. Then $\End{\Mod{R}}{M}$ is a division algebra. For if $\phi \circ \psi = 0$ then $\phi$ and $\psi$ cannot both be isomorphisms, hence $\phi = 0$ or $\psi = 0$ by Schur's Lemma.
\end{ex}

\chapter{Chain Complexes}

\begin{df}[Chain Complex]
Let $R$ be a ring. A \emph{chain complex of left-$R$-modules} $M_\bullet = (M_\bullet, d_\bullet)$ consists of a family of left-$R$-modules $\{ M_i \}_{i \in \mathbb{Z}}$ and a family of left-$R$-module homomorphisms $\{ d_i : M_i \rightarrow M_{i-1} \}_{i \in \mathbb{Z}}$ such that, for all $i$,
\[ d_i \circ d_{i+1} = 0 \enspace . \]

We call each $d_i$ a \emph{differential} and the family $\{d_i\}_i$ the \emph{boundary} of the chain complex.
\end{df}

\begin{df}[Exact]
A chain complex $M_\bullet$ is \emph{exact} at $M_i$ iff $\im d_{i+1} = \ker d_i$.

It is \emph{exact} or an \emph{exact sequence} iff it is exact at $M_i$ for all $i$.
\end{df}

\begin{prop}
A complex
\[ \cdots \rightarrow 0 \rightarrow L \stackrel{\alpha}{\rightarrow} M \rightarrow \cdots \]
is exact at $L$ iff $\alpha$ is a monomorphism.
\end{prop}

\begin{proof}
\pf\ Since both are equivalent to $\ker \alpha = 0$. \qed
\end{proof}

\begin{prop}
A complex
\[ \cdots \rightarrow M \stackrel{\beta}{\rightarrow} N \rightarrow 0 \rightarrow \cdots \]
is exact at $N$ iff $\beta$ is a epimorphism.
\end{prop}

\begin{proof}
\pf\ Since both are equivalent to $\im \beta = N$. \qed
\end{proof}

\begin{df}[Short Exact Sequence]
A \emph{short exact sequence} is an exact complex of the form
\[ 0 \rightarrow L \stackrel{\alpha}{\rightarrow} M \stackrel{\beta}{\rightarrow} N \rightarrow 0 \enspace . \]
\end{df}

\begin{prop}[Four-Lemma]
If
\[ \begin{tikzcd}
A_1 \arrow[r,"f_1"] \arrow[d,"\alpha"] & B_1 \arrow[r,"g_1"] \arrow[d,"\beta"] & C_1 \arrow[r,"h_1"] \arrow[d,"\gamma"] & D_1 \arrow[d,"\delta"] \\
A_2 \arrow[r,"f_2"] & B_2 \arrow[r,"g_2"] & C_2 \arrow[r,"h_2"] & D_2
\end{tikzcd} \]
is a commutative diagram of left-$R$-modules with exact rows, $\alpha$ is an epimorphism, and $\beta$ and $\delta$ are monomorphisms, then $\gamma$ is an monomorphism.
\end{prop}

\begin{proof}
\pf
	\step{a}{\pflet{$x,y \in C_1$}}
	\step{b}{\assume{$\gamma(x) = \gamma(y)$}}
	\step{c}{$\delta(h_1(x)) = \delta(h_1(y))$}
	\step{d}{$h_1(x) = h_1(y)$}
	\begin{proof}
		\pf\ $\delta$ is injective.
	\end{proof}
	\step{e}{$x-y \in \ker h_1$}
	\step{f}{$x-y \in \im g_1$}
	\step{g}{\pick\ $b \in B_1$ such that $g_1(b) = x - y$.}
	\step{h}{$g_2(\beta(b)) = 0$}
	\begin{proof}
		\pf\ $g_2(\beta(b)) = \gamma(g_1(b)) = \gamma(x-y) = 0$
	\end{proof}
	\step{i}{$\beta(b) \in \ker g_2$}
	\step{j}{$\beta(b) \in \im f_2$}
	\step{k}{\pick\ $a' \in A_2$ such that $f_2(a') = \beta(b)$}
	\step{l}{\pick\ $a \in A_1$ such that $\alpha(a) = a'$}
	\begin{proof}
		\pf\ $\alpha$ is surjective.
	\end{proof}
	\step{m}{$\beta(f_1(a)) = \beta(b)$}
	\step{n}{$f_1(a) = b$}
	\begin{proof}
		\pf\ $\beta$ is injective.
	\end{proof}
	\step{o}{$0 = g_1(b)$}
	\begin{proof}
		\pf\ Since $g_1(b) = g_1(f_1(a)) = 0$.
	\end{proof}
	\step{p}{$x = y$}
	\begin{proof}
		\pf\ \stepref{g}
	\end{proof}
\qed
\end{proof}

\begin{prop}[Four-Lemma]
If
\[ \begin{tikzcd}
A_1 \arrow[r,"f_1"] \arrow[d,"\beta"] & B_1 \arrow[r,"g_1"] \arrow[d,"\gamma"] & C_1 \arrow[r,"h_1"] \arrow[d,"\delta"] & D_1 \arrow[d,"\epsilon"] \\
A_2 \arrow[r,"f_2"] & B_2 \arrow[r,"g_2"] & C_2 \arrow[r,"h_2"] & D_2
\end{tikzcd} \]
is a commutative diagram of left-$R$-modules with exact rows, $\beta$ and $\delta$ are epimorphisms, and $\epsilon$ is a monomorphism, then $\gamma$ is an epimorphism.
\end{prop}

\begin{proof}
\pf
\step{1}{\pflet{$b_2 \in B_2$}}
\step{2}{\pick\ $c_1 \in C_1$ such that $\delta(c_1) = g_2(b_2)$}
\begin{proof}
	\pf\ $\delta$ is surjective.
\end{proof}
\step{3}{$\epsilon(h_1(c_1)) = 0$}
\step{4}{$h_1(c_1) = 0$}
\begin{proof}
	\pf\ $\epsilon$ is injective.
\end{proof}
\step{5}{$c_1 \in \ker h_1$}
\step{6}{$c_1 \in \im g_1$}
\step{7}{\pick\ $b_1 \in B_1$ such that $g_1(b_1) = c_1$}
\step{8}{$g_2(\gamma(b_1)) = g_2(b_2)$}
\step{9}{$\gamma(b_1) - b_2 \in \ker g_2$}
\step{10}{$\gamma(b_1) - b_2 \in \im f_2$}
\step{11}{\pick\ $a_2 \in A_2$ such that $f_2(a_2) = \gamma(b_1) - b_2$.}
\step{12}{\pick\ $a_1 \in A_1$ such that $\beta(a_1) = a_2$.}
\begin{proof}
	\pf\ $\beta$ is surjective.
\end{proof}
\step{13}{$\gamma(f_1(a_1)) = \gamma(b_1) - b_2$}
\step{14}{$b_2 = \gamma(b_1 - f_1(a_1))$}
\qed
\end{proof}

\begin{thm}[Snake Lemma]
Suppose we have $R$-modules and homomorphisms
\[ \begin{tikzcd}
0 \arrow[r] & L_1 \arrow[r,"\alpha_1"] \arrow[d,"\lambda"] & M_1 \arrow[r,"\beta_1"] \arrow[d,"\mu"] & N_1 \arrow[r] \arrow[d,"\nu"] & 0 \\
0 \arrow[r] & L_0 \arrow[r,"\alpha_0"] & M_0 \arrow[r,"\beta_0"] & N_0 \arrow[r] & 0
\end{tikzcd} \]
such that the diagram commutes and the two rows are short exact sequences. Then there exists a homomorphism $\delta : \ker \nu \rightarrow \coker \lambda$ such that the following is an exact sequence.
\[ 0 \rightarrow \ker \lambda \stackrel{\alpha_1}{\rightarrow} \ker \mu \stackrel{\beta_1}{\rightarrow} \ker \nu \stackrel{\delta}{\rightarrow} \coker \lambda \stackrel{\alpha_0}{\rightarrow} \coker \mu \stackrel{\beta_0}{\rightarrow} \coker \nu \rightarrow 0 \enspace . \]
\end{thm}

\begin{proof}
\pf
\step{1}{Define $\delta : \ker \nu \rightarrow \coker \lambda$.}
\begin{proof}
	\step{a}{\pflet{$a \in \ker \nu$}}
	\step{b}{\pick\ $c \in M_1$ such that $\beta_1(c) = a$.}
	\begin{proof}
		\pf\ Since $\beta_1$ is surjective.
	\end{proof}
	\step{c}{\pflet{$d = \mu(c)$}}
	\step{d}{$d \in \ker \beta_0 = \im \alpha_0$}
	\begin{proof}
		\pf\ Since $\beta_0(d) = \beta_0(\mu(c)) = \nu(a) = 0$.
	\end{proof}
	\step{e}{\pflet{$e \in L_0$ be the element such that $\alpha_0(e) = d$.}}
	\step{f}{\pflet{$\delta(a) = e + \im \lambda$}}
\end{proof}
\step{2}{$\delta$ is a left-$R$-module homomorphism.}
\begin{proof}
	\step{a}{For $a,a' \in \ker \nu$ we have $\delta(a + a') = \delta(a) + \delta(a')$.}
	\begin{proof}
		\step{i}{\pflet{$a,a' \in \ker \nu$}}
		\step{ii}{\pflet{$c,c',c'' \in M_1$ and $e,e',e'' \in L_0$ be the elements such that
		\begin{align*}
		\beta_1(c) & = a \\
		\beta_1(c') & = a' \\
		\beta_1(c'') & = a + a' \\
		\alpha_0(e) & = \mu(c) \\
		\alpha_0(e') & = \mu(c') \\
		\alpha_0(e'') & = \mu(c'') \\
		\delta(a) & = e + \im \lambda \\
		\delta(a') & = e' + \im \lambda \\
		\delta(a + a') & = e'' + \im \lambda
		\end{align*}}}
		\step{iii}{$c'' - c - c' \in \ker \beta_1 = \im \alpha_1$}
		\step{iv}{\pick\ $g \in L_1$ such that $\alpha_1(g) = c'' - c -c'$.}
		\step{v}{$\alpha_0(\lambda(g)) = \alpha_0(e'' - e - e')$}
		\step{vi}{$\lambda(g) = e'' - e - e'$}
		\step{vii}{$e'' - e - e' \in \im \lambda$}
		\step{viii}{$e'' + \im \lambda = e + e' + \im \lambda$}
		\step{ix}{$\delta(a + a') = \delta(a) + \delta(a')$}
	\end{proof}
	\step{b}{For $r \in R$ and $a \in \ker \nu$ we have $\delta(ra) = r \delta(a)$.}
	\begin{proof}
		\step{i}{\pflet{$r \in R$ and $a \in \ker \nu$}}
		\step{ii}{\pflet{$c,c'\in M_1$ and $e,e' \in L_0$ be the elements such that
		\begin{align*}
		\beta_1(c) & = a \\
		\beta_1(c') & = ra \\
		\alpha_0(e) & = \mu(c) \\
		\alpha_0(e') & = \mu(c') \\
		\delta(a) & = e + \im \lambda \\
		\delta(ra) & = e' + \im \lambda
		\end{align*}}}
		\step{iii}{$rc - c' \in \ker \beta_1 = \im \alpha_1$}
		\step{iv}{\pick\ $g \in L_1$ such that $\alpha_1(g) = rc - c'$.}
		\step{v}{$\alpha_0(\lambda(g)) = \alpha_0(re - e')$}
		\step{vi}{$\lambda(g) = re - e'$}
		\step{viii}{$re - e' \in \im \lambda$}
		\step{ix}{$re + \im \lambda = e' + \im \lambda$}
		\step{x}{$r\delta(a) = \delta(ra)$}
	\end{proof}
\end{proof}
\step{3}{The sequence is exact at $\ker \lambda$.}
\begin{proof}
	\pf\ Since $\alpha_1$ is injective.
\end{proof}
\step{4}{The sequence is exact at $\ker \mu$.}
\begin{proof}
	\pf\ Since $\im \alpha_1 = \ker \beta_1$.
\end{proof}
\step{5}{The sequence is exact at $\ker \nu$, i.e. $\\beta_1(\ker \mu) = \ker \delta$.}
\begin{proof}
	\step{i}{\pflet{$a \in \ker \nu$}}
	\step{ii}{\pflet{$c \in M_1$ and $e \in L_0$ be the elements such that $\beta_1(c) = a$, $\alpha_0(e) = \mu(c)$, and $\delta(a) = e + \im \lambda$.}}
	\step{iii}{If $\delta(a) = \im \lambda$ then $a \in \beta_1(\ker \mu)$}
	\begin{proof}
		\step{one}{\assume{$\delta(a) = \im \lambda$}}
		\step{two}{$e \in \im \lambda$}
		\step{three}{\pick\ $g \in L_1$ such that $\lambda(g) = e$}
		\step{four}{$\mu(\alpha_1(g)) = \mu(c)$}
		\step{five}{$c - \alpha_1(g) \in \ker \mu$}
		\step{six}{$a = \beta_1(c - \alpha_1(g))$}
	\end{proof}
	\step{iv}{If $a \in \beta_1(\ker \mu)$ then $\delta(a) = \im \lambda$}
	\begin{proof}
		\step{one}{\assume{$c' \in \ker \mu$ and $a = \beta_1(c')$}}
		\step{two}{$c - c' \in \ker \beta_1 = \im \alpha_1$}
		\step{three}{\pick\ $g \in L_1$ such that $\alpha_1(g) = c - c'$}
		\step{four}{$\alpha_0(\lambda(g)) = \mu(c) - \mu(c') = \alpha_0(e) - 0 = \alpha_0(e)$}
		\step{five}{$\lambda(g) = e$}
		\step{six}{$e \in \im \lambda$}
		\step{seven}{$\delta(a) = \im \lambda$}
	\end{proof}
\end{proof}
\step{6}{THe sequence is exact at $\coker \lambda$.}
\begin{proof}
	\step{a}{\pflet{$e \in L_0$} \prove{$e + \im \lambda \in \im \delta$ iff $\alpha_0(e) \in \im \mu$.}}
	\step{b}{For all $a \in \ker \nu$, if $\delta(a) = e + \im \lambda$ then $\alpha_0(e) \in \im \mu$}
	\begin{proof}
		\pf\ From \stepref{1} and the fact that $\alpha_0$ is injective hence $e$ is unique given $a$.
	\end{proof}
	\step{c}{For all $e \in L_0$, if $\alpha_0(e) \in \im \mu$ then $e + \im \lambda \in \im \delta$.}
	\begin{proof}
		\step{i}{\pflet{$e \in L_0$}}
		\step{ii}{\assume{$\alpha_0(e) \in \im \mu$}}
		\step{iii}{\pick\ $c \in M_1$ such that $\mu(c) = \alpha_0(e)$. \prove{$e + \im \lambda = \delta(\beta_1(c))$}}
		\step{iv}{\pick\ $c' \in M_1$ and $e' \in L_0$ such that $\beta_1(c') = \beta_1(c)$, $\alpha_0(e') = \mu(c')$ and $\delta(\beta_1(c)) = e' + \im \lambda$}
		\step{v}{$c - c' \in \ker \beta_1 = \im \alpha_1$}
		\step{vi}{\pick\ $g \in L_1$ such that $\alpha_1(g) = c - c'$.}
		\step{vii}{$\alpha_0(\lambda(g)) = \alpha_0(e- e')$}
		\step{viii}{$\lambda(g) = e - e'$}
		\step{ix}{$e + \im \lambda = e' + \im \lambda = \delta(\beta_1(c))$}
	\end{proof}
\end{proof}
\step{7}{The sequence is exact at $\coker \mu$.}
\begin{proof}
	\pf\ Since $\im \alpha_0 = \ker \beta_0$.
\end{proof}
\step{8}{The sequence is exact at $\coker \nu$.}
\begin{proof}
	\pf\ Since $\beta_0$ is surjective.
\end{proof}
\qed
\end{proof}

\begin{cor}
Suppose we have $R$-modules and homomorphisms
\[ \begin{tikzcd}
0 \arrow[r] & L_1 \arrow[r,"\alpha_1"] \arrow[d,"\lambda"] & M_1 \arrow[r,"\beta_1"] \arrow[d,"\mu"] & N_1 \arrow[r] \arrow[d,"\nu"] & 0 \\
0 \arrow[r] & L_0 \arrow[r,"\alpha_0"] & M_0 \arrow[r,"\beta_0"] & N_0 \arrow[r] & 0
\end{tikzcd} \]
such that the diagram commutes and the two rows are short exact sequences. Suppose $\mu$ is surjective and $\nu$ is injective. Then $\lambda$ is surjective and $\nu$ is an isomorphism.
\end{cor}

\begin{proof}
\pf\ We have $\ker \nu = \coker \mu = 0$ and so $0 \stackrel{\delta}{\rightarrow} \coker \lambda \stackrel{\alpha_0}{\rightarrow} 0$ is an exact sequence, hence $\coker \lambda = 0$ and so $\lambda$ is surjective.

Since $\coker \mu = 0$ we have $0 \rightarrow \coker \nu \rightarrow 0$ is an exact sequence and so $\coker \nu = 0$, hence $\nu$ is surjective, hence $\nu$ is an isomorphism. \qed
\end{proof}

\begin{prop}[Short Five-Lemma]
Suppose we have $R$-modules and homomorphisms
\[ \begin{tikzcd}
0 \arrow[r] & L_1 \arrow[r,"\alpha_1"] \arrow[d,"\lambda"] & M_1 \arrow[r,"\beta_1"] \arrow[d,"\mu"] & N_1 \arrow[r] \arrow[d,"\nu"] & 0 \\
0 \arrow[r] & L_0 \arrow[r,"\alpha_0"] & M_0 \arrow[r,"\beta_0"] & N_0 \arrow[r] & 0
\end{tikzcd} \]
such that the diagram commutes and the two rows are short exact sequences. If $\lambda$ and $\nu$ are isomorphisms then $\mu$ is an isomorphism.
\end{prop}

\begin{proof}
\pf
\step{1}{There exists a homomorphism $\delta : 0 \rightarrow L_0$ such that the following is an exact sequence.
\[ 0 \rightarrow 0 \rightarrow \ker \mu \rightarrow 0 \stackrel{\delta}{\rightarrow} L_0 \stackrel{\alpha_0}{\rightarrow} \coker \mu \stackrel{\beta_0}{\rightarrow} N_0 \rightarrow 0 \enspace . \]}
\begin{proof}
	\pf\ Snake Lemma
\end{proof}
\step{2}{$\ker \mu = 0$}
\step{3}{$\coker \mu = M_0$}
\qed
\end{proof}

\begin{prop}
If $L \stackrel{\alpha}{\rightarrow} M \stackrel{\beta}{\rightarrow} N$ is an exact sequence and $L$ and $N$ are Noetherian then $M$ is Noetherian.
\end{prop}

\begin{proof}
\pf
\step{1}{\pflet{$P$ be a submodule of $M$.}}
\step{2}{\pick\ $a_1$, \ldots, $a_m$ generate $\inv{\alpha}(P)$.}
\step{3}{\pick\ $c_1$, \ldots, $c_n$ that generate $\beta(P)$.}
\step{4}{For $i=1, \ldots, n$, \pick\ $b_i$ such that $\beta(b_i) = c_i$. \prove{$\alpha(a_1)$, \ldots, $\alpha(a_m)$, $b_1$, \ldots, $b_n$ generate $P$.}}
\step{5}{\pflet{$p \in P$}}
\step{6}{\pick\ $r_1, \ldots, r_n \in R$ such that $r_1 c_1 + \cdots + r_n c_n = \beta(p)$}
\step{7}{$r_1 b_1 + \cdots + r_n b_n - p \in \ker \beta = \im \alpha$}
\step{8}{\pick\ $s_1, \ldots, s_m \in R$ such that $\alpha(s_1 a_1 + \cdots + s_m a_m) = r_1 b_1 + \cdots + r_n b_n - p$.}
\step{9}{$p = s_1 \alpha(a_1) + \cdots + s_m \alpha(a_m) + r_1 b_1 + \cdots + r_n b_n$}
\qed
\end{proof}

\begin{prop}
Let $R$ be a ring. Let 
\[ 0 \rightarrow M \stackrel{\alpha}{\rightarrow} N \stackrel{\beta}{\rightarrow} P \rightarrow 0 \]
be a short exact sequence of left-$R$-modules. Let $L$ be an $R$-module. Then the following is an exact sequence:
\[ 0 \rightarrow \Mod{R}[P,L] \stackrel{\Mod{R}[\beta,\id{L}]}{\longrightarrow} \Mod{R}[N,L] \stackrel{\Mod{R}[\alpha,\id{L}]}{\longrightarrow} \Mod{R}[M,L] \enspace . \]
\end{prop}

\begin{proof}
\pf
\step{1}{$\Mod{R}[\beta,\id{L}]$ is injective.}
\begin{proof}
	\pf\ Since $\beta$ is epi.
\end{proof}
\step{2}{$\im \Mod{R}[\beta,\id{L}] = \ker \Mod{R}[\alpha,\id{L}]$}
\begin{proof}
	\step{a}{$\im \Mod{R}[\beta, \id{L}] \subseteq \ker \Mod{R}[\alpha, \id{L}]$}
	\begin{proof}
		\pf\ For any $\gamma \in \Mod{R}[P,L]$ we have $\gamma \circ \beta \circ \alpha = 0$ because $\beta \circ \alpha = 0$.
	\end{proof}
	\step{b}{$\ker \Mod{R}[\alpha, \id{L}] \subseteq \im \Mod{R}[\beta, \id{L}]$}
	\begin{proof}
		\step{i}{\pflet{$\gamma \in \ker \Mod{R}[\alpha, \id{L}]$}}
		\step{ii}{$\gamma \circ \alpha = 0$}
		\step{iii}{\pick\ $\delta : P \rightarrow L$ by: for all $p \in P$, we have $\delta(p) = \gamma(n)$ where $n \in N$ is an element such that $\beta(n) = p$. \prove{$\delta \circ \beta = \gamma$}}
		\step{iv}{\pflet{$n \in N$} \prove{$\delta(\beta(n)) = \gamma(n)$}}
		\step{v}{\pick\ $n' \in N$ such that $\delta(\beta(n)) = \gamma(n')$ and $\beta(n') = \beta(n)$}
		\step{vi}{$n - n' \in \ker \beta = \im \alpha$}
		\step{vii}{\pick\ $m \in M$ such that $\alpha(m) = n - n'$}
		\step{viii}{$0 = \gamma(\alpha(m)) = \gamma(n) - \gamma(n')$}
		\step{ix}{$\gamma(n) = \gamma(n') = \delta(\beta(n))$}
	\end{proof}
\end{proof}
\qed
\end{proof}

\begin{thm}[Nine-Lemma]
Let the following be a commuting diagram of left-$R$-modules.
\[ \begin{tikzcd}
& 0 \arrow[d] & 0 \arrow[d] & 0 \arrow[d] \\
0 \arrow[r] & L_2 \arrow[r,"f_2"] \arrow[d,"\alpha_1"] & M_2 \arrow[r,"g_2"] \arrow[d,"\beta_1"] & N_2 \arrow[r] \arrow[d,"\gamma_1"] & 0 \\
0 \arrow[r] & L_1 \arrow[r,"f_1"] \arrow[d,"\alpha_0"] & M_1 \arrow[r,"g_1"] \arrow[d,"\beta_0"] & N_1 \arrow[r] \arrow[d,"\gamma_0"] & 0 \\
0 \arrow[r] & L_0 \arrow[r,"f_0"] \arrow[d] & M_0 \arrow[r,"g_0"] \arrow[d] & N_0 \arrow[r] \arrow[d] & 0 \\
& 0 & 0 & 0
\end{tikzcd} \]
If the rows are exact and the two rightmost columns are exact then the left column is exact.
\end{thm}

\begin{proof}
\pf
\step{1}{$(L_2,f_2)$ is the kernel of $g_2$, $(L_1,f_1)$ is the kernel of $g_1$ and $(L_0,f_0)$ is the kernel of $g_0$.}
\step{2}{0 is the cokernel of $g_2$, $g_1$ and $g_0$.}
\step{1}{\pick\ a homomomorphism $\delta : L_0 \rightarrow 0$ such that the following is an exact sequence:
\[ L_2 \stackrel{\beta_1 \restriction L_2}{\rightarrow} L_1 \stackrel{\beta_0 \restriction L_1}{\rightarrow} L_0 \stackrel{\delta}{\rightarrow} 0 \rightarrow 0 \rightarrow 0 \]}
\begin{proof}
	\pf\ Snake Lemma.
\end{proof}
\step{2}{$\beta_1 \restriction L_2 = \alpha_1$}
\step{3}{$\beta_0 \restriction L_1 = \alpha_0$}
\step{4}{The following is an exact sequence:
\[ 0 \rightarrow L_2 \stackrel{\alpha_1}{\rightarrow} L_1 \stackrel{\alpha_0}{\rightarrow} L_0 \rightarrow 0 \]}
\qed
\end{proof}

\begin{thm}
Let the following be a commuting diagram of left-$R$-modules.
\[ \begin{tikzcd}
& \vdots \arrow[d] & \vdots \arrow[d] & \vdots \arrow[d] \\
0 \arrow[r] & L_{i+1} \arrow[r] \arrow[d,"\alpha_{i+1}"] & M_{i+1} \arrow[r] \arrow[d] & N_{i+1} \arrow[r] \arrow[d] & 0 \\
0 \arrow[r] & L_i \arrow[r] \arrow[d,"\alpha_i"] & M_i \arrow[r] \arrow[d] & N_i \arrow[r] \arrow[d] & 0 \\
0 \arrow[r] & L_{i-1} \arrow[r,"f_{i-1}"] \arrow[d] & M_{i-1} \arrow[r] \arrow[d] & N_{i-1} \arrow[r] \arrow[d] & 0 \\
& \vdots & \vdots & \vdots
\end{tikzcd} \]
Assume the central column is a complex and every row is an exact complex. Then the left and right columns are complexes. Further, if any two of the columns are exact, then so is the third.
\end{thm}

\begin{proof}
\pf
\step{1}{The left column is a complex.}
\begin{proof}
\step{1}{\pflet{$x \in L_{i+1}$}}
\step{2}{$f_{i-1}(\alpha_i(\alpha_{i+1}(x))) = 0$}
\step{3}{$\alpha_i(\alpha_{i+1}(x)) = 0$}
\begin{proof}
	\pf\ $f_{i-1}$ is injective.
\end{proof}
\end{proof}
\step{2}{The right column is a complex.}
\begin{proof}
\step{1}{\pflet{$x \in N_{i+1}$}}
\step{2}{\pick\ $y \in N_{i+1}$ such that $g_{i+1}(y) = x$}
\step{3}{$\gamma_i(\gamma_{i+1}(x)) = 0$}
\begin{proof}
	\pf
	\begin{align*}
		\gamma_i(\gamma_{i+1}(x)) & = \gamma_i(\gamma_{i+1}(g_{i+1}(y))) \\
		& = g_{i-1}(\beta_i(\beta_{i+1}(y))) \\
		& = g_{i-1}(0) \\
		& = 0
	\end{align*}
\end{proof}
\end{proof}
\step{3}{If the left and center columns are exact then the right column is exact.}
\begin{proof}
	\step{a}{\pflet{$n_i \in \ker \gamma_{i-1}$} \prove{$n_i \in \im \gamma_i$}}
	\step{b}{\pick\ $m_i \in M_i$ such that $g_i(m_i) = n_i$}
	\step{c}{$g_{i-1}(\beta_i(m_i)) = 0$}
	\step{d}{$\beta_i(m_i) \in \ker g_{i-1} = \im f_{i-1}$}
	\step{e}{\pick\ $l_{i-1} \in L_{i-1}$ such that $f_{i-1}(l_{i-1}) = \beta_i(m_i)$}
	\step{f}{$\beta_{i-1}(f_{i-1}(l_{i-1})) = 0$}
	\step{g}{$f_{i-2}(\alpha_{i-1}(l_{i-1})) = 0$}
	\step{h}{$\alpha_{i-1}(l_{i-1}) = 0$}
	\step{i}{$l_{i-1} \in \ker \alpha_{i-1} = \im \alpha_i$}
	\step{j}{\pick\ $l_i \in L_i$ such that $\alpha_i(l_i) = l_{i-1}$}
	\step{k}{$\beta_i(f_i(l_i)) = \beta_i(m_i)$}
	\step{l}{$f_i(l_i) - m_i \in \ker \beta_i = \im \beta_{i+1}$}
	\step{m}{\pick\ $m_{i+1} \in M_{i+1}$ such that $\beta_{i+1}(m_{i+1}) = f_i(l_i) - m_i$}
	\step{n}{$\gamma_{i+1}(-g_{i+1}(m_{i+1})) = n_i$}
\end{proof}
\step{4}{If the left and right columns are exact then the center column is exact.}
\begin{proof}
	\step{a}{\pflet{$x \in \ker \beta_i$} \prove{$x \in \im \beta_{i+1}$}}
	\step{b}{$g_{i-1}(\beta_i(x)) = 0$}
	\step{c}{$\gamma_i(g_i(x)) = 0$}
	\step{d}{$g_i(x) \in \ker \gamma_i = \im \gamma_{i+1}$}
	\step{e}{\pick\ $n_{i+1} \in N_{i+1}$ such that $\gamma_{i+1}(n_{i+1}) = g_i(x)$}
	\step{f}{\pick\ $m_{i+1} \in M_{i+1}$ such that $g_{i+1}(m_{i+1}) = n_{i+1}$}
	\step{g}{$g_i(\beta_{i+1}(m_{i+1})) = g_i(x)$}
	\step{h}{$\beta_{i+1}(m_{i+1}) - x \in \ker g_i = \im f_i$}
	\step{i}{\pick\ $l_i \in L_i$ such that $f_i(l_i) = \beta_{i+1}(m_{i+1}) - x$}
	\step{j}{$\beta_i(f_i(l_i)) = 0$}
	\step{k}{$f_{i-1}(\alpha_i(l_i)) = 0$}
	\step{l}{$\alpha_i(l_i) = 0$}
	\step{m}{$l_i \in \ker \alpha_i = \im \alpha_{i+1}$}
	\step{n}{\pick\ $l_{i+1} \in L_{i+1}$ such that $\alpha_{i+1}(l_{i+1}) = l_i$}
	\step{o}{$\beta_{i+1}(f_{i+1}(l_{i+1})) = \beta_{i+1}(m_{i+1}) - x$}
	\step{p}{$x = \beta_{i+1}(m_{i+1} - f_{i+1}(l_{i+1}))$}
	\end{proof}
\step{5}{If the center and right columns are exact then the left column is exact.}
\begin{proof}
	\step{a}{\pflet{$l_i \in \ker \alpha_i$} \prove{$l_i \in \im \alpha_{i+1}$}}
	\step{b}{$\beta_i(f_i(l_i)) = 0$}
	\step{c}{$f_i(l_i) \in \ker \beta_i = \im \beta_{i+1}$}
	\step{d}{\pick\ $m_{i+1} \in M_{i+1}$ such that $\beta_{i+1}(m_{i+1}) = f_i(l_i)$}
	\step{e}{$\gamma_{i+1}(g_{i+1}(m_{i+1})) = 0$}
	\step{f}{$g_{i+1}(m_{i+1}) \in \ker \gamma_{i+1} = \im \gamma_{i+2}$}
	\step{g}{\pick\ $n_{i+2} \in N_{i+2}$ such that $\gamma_{i+2}(n_{i+2}) = g_{i+1}(m_{i+1})$}
	\step{h}{\pick\ $m_{i+2} \in M_{i+2}$ such that $g_{i+2}(m_{i+2}) = n_{i+2}$}
	\step{i}{$g_{i+1}(\beta_{i+2}(n_{i+2})) = g_{i+1}(m_{i+1})$}
	\step{j}{$\beta_{i+2}(n_{i+2}) - m_{i+1} \in \ker g_{i+1} = \im f_{i+1}$}
	\step{k}{\pick\ $l_{i+1} \in L_{i+1}$ such that $f_{i+1}(l_{i+1}) = \beta_{i+2}(n_{i+2}) - m_{i+1}$}
	\step{l}{$f_i(\alpha_{i+1}(l_{i+1})) = - f_i(l_i)$}
	\step{m}{$l_i = \alpha_{i+1}(-l_{i+1})$}
\end{proof}
\qed
\end{proof}

\begin{cor}[Nine-Lemma]
Let the following be a commuting diagram of left-$R$-modules.
\[ \begin{tikzcd}
& 0 \arrow[d] & 0 \arrow[d] & 0 \arrow[d] \\
0 \arrow[r] & L_2 \arrow[r,"f_2"] \arrow[d,"\alpha_1"] & M_2 \arrow[r,"g_2"] \arrow[d,"\beta_1"] & N_2 \arrow[r] \arrow[d,"\gamma_1"] & 0 \\
0 \arrow[r] & L_1 \arrow[r,"f_1"] \arrow[d,"\alpha_0"] & M_1 \arrow[r,"g_1"] \arrow[d,"\beta_0"] & N_1 \arrow[r] \arrow[d,"\gamma_0"] & 0 \\
0 \arrow[r] & L_0 \arrow[r,"f_0"] \arrow[d] & M_0 \arrow[r,"g_0"] \arrow[d] & N_0 \arrow[r] \arrow[d] & 0 \\
& 0 & 0 & 0
\end{tikzcd} \]
If the rows are exact and the two leftmost columns are exact then the right column is exact.
\end{cor}

\begin{prop}
Let the following be a commuting diagram of left-$R$-modules.
\[ \begin{tikzcd}
& 0 \arrow[d] & 0 \arrow[d] & 0 \arrow[d] \\
0 \arrow[r] & L_2 \arrow[r,"f_2"] \arrow[d,"\alpha_1"] & M_2 \arrow[r,"g_2"] \arrow[d,"\beta_1"] & N_2 \arrow[r] \arrow[d,"\gamma_1"] & 0 \\
0 \arrow[r] & L_1 \arrow[r,"f_1"] \arrow[d,"\alpha_0"] & M_1 \arrow[r,"g_1"] \arrow[d,"\beta_0"] & N_1 \arrow[r] \arrow[d,"\gamma_0"] & 0 \\
0 \arrow[r] & L_0 \arrow[r,"f_0"] \arrow[d] & M_0 \arrow[r,"g_0"] \arrow[d] & N_0 \arrow[r] \arrow[d] & 0 \\
& 0 & 0 & 0
\end{tikzcd} \]
If the rows are exact and the left and right columns are exact then $\beta_1$ is monic.
\end{prop}

\begin{proof}
\pf\ By the Snake Lemma, the following is an exact sequence
\[ 0 \rightarrow \ker \alpha_1 \rightarrow \ker \beta_1 \rightarrow \ker \gamma_1 \]
But $\ker \alpha_1 = \ker \gamma_1 = 0$ so $\ker \beta_1 = 0$. \qed
\end{proof}

\begin{prop}
Let the following be a commuting diagram of left-$R$-modules.
\[ \begin{tikzcd}
& 0 \arrow[d] & 0 \arrow[d] & 0 \arrow[d] \\
0 \arrow[r] & L_2 \arrow[r,"f_2"] \arrow[d,"\alpha_1"] & M_2 \arrow[r,"g_2"] \arrow[d,"\beta_1"] & N_2 \arrow[r] \arrow[d,"\gamma_1"] & 0 \\
0 \arrow[r] & L_1 \arrow[r,"f_1"] \arrow[d,"\alpha_0"] & M_1 \arrow[r,"g_1"] \arrow[d,"\beta_0"] & N_1 \arrow[r] \arrow[d,"\gamma_0"] & 0 \\
0 \arrow[r] & L_0 \arrow[r,"f_0"] \arrow[d] & M_0 \arrow[r,"g_0"] \arrow[d] & N_0 \arrow[r] \arrow[d] & 0 \\
& 0 & 0 & 0
\end{tikzcd} \]
If the rows are exact and the left and right columns are exact then $\beta_0$ is epi.
\end{prop}

\begin{proof}
\pf\ Similar. \qed
\end{proof}

\begin{prop}
Let the following be a commuting diagram of left-$R$-modules.
\[ \begin{tikzcd}
& 0 \arrow[d] & 0 \arrow[d] & 0 \arrow[d] \\
0 \arrow[r] & L_2 \arrow[r,"f_2"] \arrow[d,"\alpha_1"] & M_2 \arrow[r,"g_2"] \arrow[d,"\beta_1"] & N_2 \arrow[r] \arrow[d,"\gamma_1"] & 0 \\
0 \arrow[r] & L_1 \arrow[r,"f_1"] \arrow[d,"\alpha_0"] & M_1 \arrow[r,"g_1"] \arrow[d,"\beta_0"] & N_1 \arrow[r] \arrow[d,"\gamma_0"] & 0 \\
0 \arrow[r] & L_0 \arrow[r,"f_0"] \arrow[d] & M_0 \arrow[r,"g_0"] \arrow[d] & N_0 \arrow[r] \arrow[d] & 0 \\
& 0 & 0 & 0
\end{tikzcd} \]
If the rows are exact, the left and right columns are exact, and the central column is a complex, then the central column is exact.
\end{prop}

\begin{proof}
\pf
\step{1}{\pflet{$x \in \ker \beta_0$} \prove{$x \in \im \beta_1$}}
\step{2}{$\gamma_0(g_1(x)) = 0$}
\step{3}{$g_1(x) \in \ker \gamma_0 = \im \gamma_1$}
\step{4}{\pick\ $n_2 \in N_2$ such that $\gamma_1(n_2) = g_1(x)$}
\step{5}{\pick\ $m_2 \in M_2$ such that $g_2(m_2) = n_2$}
\step{6}{$g_1(\beta_1(m_2)) = g_1(x)$}
\step{7}{$\beta_1(m_2) - x \in \ker g_1 = \im f_1$}
\step{8}{\pick\ $l_1 \in L_1$ such that $f_1(l) = \beta_1(m_2) - x$.}
\step{9}{$f_0(\alpha_0(l_1)) = 0$}
\step{10}{$\alpha_0(l_1) = 0$}
\step{11}{$l_1 \in \ker \alpha_0 = \im \alpha_1$}
\step{12}{\pick\ $l_2 \in L_2$ such that $\alpha_1(l_2) = l_1$.}
\step{13}{$\beta_1(f_2(l_2)) = \beta_1(m_2) - x$}
\step{14}{$x = \beta_1(m_2 - f_2(l_2))$}
\qed
\end{proof}

\begin{ex}
We cannot remove the hypothesis that the central column is a complex. Consider the situation
\[ \begin{tikzcd}
& 0 \arrow[d] & 0 \arrow[d] & 0 \arrow[d] \\
0 \arrow[r] & 0 \arrow[r] \arrow[d] & \mathbb{Z} \arrow[r,equals] \arrow[d,"\Delta"] & \mathbb{Z} \arrow[r] \arrow[d,equals] & 0 \\
0 \arrow[r] & \mathbb{Z} \arrow[r,"\kappa_1"] \arrow[d,equals] & \mathbb{Z} \oplus \mathbb{Z} \arrow[r,"\pi_2"] \arrow[d, "\pi_1"] & \mathbb{Z} \arrow[r] \arrow[d] & 0 \\
0 \arrow[r] & \mathbb{Z} \arrow[r,equals] \arrow[d] & \mathbb{Z} \arrow[r] \arrow[d] & 0 \arrow[r] \arrow[d] & 0 \\
& 0 & 0 & 0
\end{tikzcd} \]
This diagram commutes, the rows are exact, the left and right columns are exact, but the central column is not a complex and $\im \Delta \neq \ker \pi_1$.
\end{ex}

\section{Split Exact Sequences}

\begin{df}[Split Sequence]
Let $0 \rightarrow M_1 \stackrel{\alpha}{\rightarrow} N \stackrel{\beta}{\rightarrow} M_2 \rightarrow 0$ be a short exact sequence. Then this sequence \emph{splits} iff there exists an isomorphism
\[ \phi : N \cong M_1 \oplus M_2 \]
such that $\phi \circ \alpha = \kappa_1 : M_1 \rightarrow M_1 \oplus M_2$ and $\beta \circ \inv{\phi} = \pi_2 : M_1 \oplus M_2 \rightarrow M_2$.
\end{df}

\begin{prop}
Let $\phi : M \rightarrow N$ be a left-$R$-module homomorphism. Then $\phi$ has a left-inverse if and only if the sequence
\[ 0 \rightarrow M \stackrel{\phi}{\rightarrow} N \rightarrow \coker \phi \rightarrow 0 \]
splits.
\end{prop}

\begin{proof}
\pf
\step{1}{If $\phi$ has a left-inverse then the sequence splits.}
\begin{proof}
	\step{a}{\assume{$\phi$ has a left-inverse $\psi : N \rightarrow M$.}}
	\step{b}{Define $i : N \rightarrow M \oplus \coker \phi$ by $i(n) = (\psi(n), n + \im \phi)$.}
	\step{c}{Define $\inv{i} : M \oplus \coker \phi$ by $\inv{i}(m,x + \im \phi) = \phi(m) + x - \phi(\psi(x))$.}
	\step{d}{$i \circ \inv{i} = \id{M \oplus \coker \phi}$}
	\begin{proof}
		\pf
		\begin{align*}
			\psi(\phi(m) + x - \phi(\psi(x))) & = m + \psi(x) - \psi(x) \\
			& = m
		\end{align*}
	\end{proof}
	\step{e}{$\inv{i} \circ i = \id{N}$}
	\begin{proof}
		\pf
		\begin{align*}
			\inv{i}(\psi(n), n + \im \phi)
			& = \phi(\psi(n)) + n - \phi(\psi(n)) \\
			& = n
		\end{align*}
	\end{proof}
	\step{f}{$i \circ \phi = \kappa_1 : M \rightarrow M \oplus \coker \phi$}
	\begin{proof}
		\pf
		\begin{align*}
			i(\phi(m)) & = (\psi(\phi(m)), \phi(m) + \im \phi) \\
			& = (m, \im \phi)
		\end{align*}
	\end{proof}
	\step{g}{$\pi \circ \inv{i} = \pi_2 : M \oplus \coker \phi \rightarrow \coker \phi$}
	\begin{proof}
		\pf
		\begin{align*}
			\inv{i}(\psi(n), n + \im \phi) + \im \phi
			& = \phi(\psi(n)) + n - \phi(\psi(n)) + \im \phi \\
			& = n + \im \phi
		\end{align*}
	\end{proof}
\end{proof}
\step{2}{If the sequence splits then $\phi$ has a left-inverse.}
\begin{proof}
	\pf\ Since $\kappa_1 : M \rightarrow M \oplus \coker \phi$ has left inverse $\pi_1$.
\end{proof}
\qed
\end{proof}

\begin{prop}
Let $\phi : M \rightarrow N$ be a left-$R$-module homomorphism. Then $\phi$ has a right-inverse if and only if the sequence
\[ 0 \rightarrow \ker \phi \rightarrow M \stackrel{\phi}{\rightarrow} N \rightarrow 0 \]
splits.
\end{prop}

\begin{proof}
\pf
\step{1}{If $\phi$ has a right-inverse then the sequence splits.}
\begin{proof}
	\step{a}{\pflet{$\psi : N \rightarrow M$ be a right inverse to $\phi$.}}
	\step{b}{\pflet{$i : M \rightarrow \ker \phi \oplus N$ be the function $i(m) = (m - \psi(\phi(m)),\phi(m))$.}}
	\begin{proof}
		\pf\ $m - \psi(\phi(m)) \in \ker \phi$ since $\phi(m - \psi(\phi(m))) = \phi(m) - \phi(m) = 0$.
	\end{proof}
	\step{c}{\pflet{$\inv{i} : \ker \phi \oplus N \rightarrow M$ be the function $\inv{i}(x,n) = x + \psi(n)$.}}
	\step{d}{$i \circ \inv{i} = \id{\ker \phi \oplus N}$}
	\begin{proof}
		\pf
		\begin{align*}
			i(\inv{i}(x,n)) & = i(x + \psi(n)) \\
			& = (x + \psi(n) - \psi(\phi(x)) - \psi(\phi(\psi(n))), \phi(x) + \phi(\psi(n))) \\
			& = (x + \psi(n) - \psi(n), n) \\
			& = (x, n)
		\end{align*}
	\end{proof}
	\step{e}{$\inv{i} \circ i = \id{M}$}
	\begin{proof}
		\pf
		\begin{align*}
			\inv{i}(i(m)) & = m - \psi(\phi(m)) + \psi(\phi(m)) \\
			& = m
		\end{align*}
	\end{proof}
	\step{f}{$i \circ \iota = \kappa_1$}
	\begin{proof}
		\pf\ For $m \in \ker \phi$ we have $i(m) = (m - \psi(\phi(m)), \phi(m)) = (m,0)$.
	\end{proof}
	\step{g}{$\phi \circ \inv{i} = \pi_2$}
	\begin{proof}
		\pf
		\begin{align*}
			\phi(\inv{i}(x,n)) & = \phi(x) + \phi(\psi(n)) \\
			& = 0 + n \\
			& = n
		\end{align*}
	\end{proof}
\end{proof}
\step{2}{If the sequence splits then $\phi$ has a right-inverse.}
\begin{proof}
\pf\ Since $\kappa_2 : N \rightarrow M \oplus N$ is a right-inverse to $\pi_2$.
\end{proof}
\qed
\end{proof}

\begin{prop}
Let
\[ 0 \rightarrow M \stackrel{\alpha}{\rightarrow} N \stackrel{\beta}{\rightarrow} F \rightarrow 0 \]
be a short exact sequence where $F$ is free. Then the sequence splits.
\end{prop}

\begin{proof}
\pf
\step{1}{\pflet{$F = R^{\oplus A}$}}
\step{2}{\pick\ $\gamma : F \rightarrow N$ such that $\id{F} = \beta \circ \gamma$}
\step{3}{\pflet{$i : M \oplus F \rightarrow N$ be the homomorphism $i(m,f) = \alpha(m) + \gamma(f)$}}
\step{4}{$i$ is injective.}
\begin{proof}
	\step{a}{\assume{$i(m,f) = i(m',f')$}}
	\step{b}{$\alpha(m) + \gamma(f) = \alpha(m') + \gamma(f')$}
	\step{c}{$\alpha(m-m') = \gamma(f-f')$}
	\step{d}{$f-f' = 0$}
	\begin{proof}
		\pf\ Applying $\beta$ to both sides of \stepref{c}.
	\end{proof}
	\step{e}{$f = f'$}
	\step{f}{$\alpha(m-m') = 0$}
	\step{g}{$m = m'$}
	\begin{proof}
		\pf\ Since $\alpha$ is injective. 
	\end{proof}
\end{proof}
\step{5}{$i$ is surjective.}
\begin{proof}
	\step{a}{\pflet{$n \in N$}}
	\step{b}{$n - \gamma(\beta(n)) \in \ker \beta = \im \alpha$}
	\step{c}{\pick\ $m \in M$ such that $\alpha(m) = n - \gamma(\beta(n))$}
	\step{d}{$n = i(m, \beta(n))$}
\end{proof}
\step{6}{$\alpha = i \circ \kappa_1$}
\step{7}{$\beta \circ i = \pi_2$}
\qed
\end{proof}

\chapter{Homology}

\begin{df}[Homology]
Let $(M_\bullet, d_\bullet)$ be a chain complex.
The \emph{$i$th homology} of the complex is the $R$-module
\[ H_i(M_\bullet) := \frac{\ker d_i}{\im d_{i+1}} \enspace . \]
\end{df}

\begin{prop}
Consider the complex
\[ 0 \rightarrow M_1 \stackrel{\phi}{\rightarrow} M_0 \rightarrow 0 \enspace . \]
The 1st homology is $\ker \phi$, and the 0th homology is $\coker \phi$.
\end{prop}

\part{Field Theory}

\chapter{Fields}

\begin{df}[Field]
A \emph{field} is a non-trivial commutative division ring.
\end{df}

\begin{ex}
$\mathbb{Q}$, $\mathbb{R}$ and $\mathbb{C}$ are fields.
\end{ex}

\begin{prop}
Every field is an integral domain.
\end{prop}

\begin{proof}
\pf\ By Propositions \ref{prop:no-left-unit-is-a-right-zero-divisor} and \ref{prop:no-right-unit-is-a-left-zero-divisor}. \qed
\end{proof}

\begin{ex}
The converse does not hold: $\mathbb{Z}$ is an integral domain but not a field.
\end{ex}

\begin{prop}
Every finite integral domain is a field.
\end{prop}

\begin{proof}
\pf\ In a finite integral domain, multiplication by any non-zero element is injective, hence surjective. \qed
\end{proof}

\begin{cor}
For any positive integer $n$, the following are equivalent:
\begin{itemize}
\item $n$ is prime.
\item $\mathbb{Z} / n \mathbb{Z}$ is an integral domain.
\item $\mathbb{Z} / n \mathbb{Z}$ is a field.
\end{itemize}
\end{cor}

\begin{thm}[Wedderburn's Little Theorem]
Every finite division ring is a field.
\end{thm}

%TODO

\begin{prop}
Every subring of a field is an integral domain.
\end{prop}

\begin{proof}
\pf\ Easy. \qed
\end{proof}

\begin{prop}
The center of a division ring is a field.
\end{prop}

\begin{proof}
\pf
\step{1}{\pflet{$R$ be a division ring.}}
\step{2}{\pflet{$Z$ be the center of $R$.}}
\step{3}{$Z$ is non-trivial.}
\begin{proof}
\pf\ Since $1 \in Z$.
\end{proof}
\step{4}{$Z$ is commutative.}
\step{5}{$Z$ is a division ring.}
\begin{proof}
	\step{a}{\pflet{$a \in Z$}}
	\step{b}{$\inv{a} \in Z$}
	\begin{proof}
		\step{i}{\pflet{$x \in R$}}
		\step{ii}{$ax = xa$}
		\step{iii}{$x \inv{a} = \inv{a} x$}
	\end{proof}
\end{proof}
\qed
\end{proof}

\begin{df}
For any prime $p$ and positive integer $r$, define a multiplication on $(\mathbb{Z} / p \mathbb{Z})^r$ that makes this group into a field by:
%TODO
\end{df}

\begin{prop}
A commutative ring is a field if and only if it is simple.
\end{prop}

\begin{proof}
\pf\ Proposition \ref{prop:division-ring-ideals}. \qed
\end{proof}

\begin{cor}
Every field has Krull dimension 0.
\end{cor}

\begin{prop}
Let $K$ be a field. Then $K[x]$ is a PID, and every non-zero ideal in $K[x]$ is generated by a unique monic polynomial.
\end{prop}

\begin{proof}
\pf
\step{1}{\pflet{$I$ be a non-zero ideal in $K[x]$}}
\step{2}{\pick\ a monic polynomial $f \in K[x]$ of minimal degree. \prove{$I = (f)$}}
\step{3}{\pflet{$g \in I$}}
\step{4}{\pick\ polynomials $q$, $r$ with $\deg r < \deg f$ such that $g = qf + r$}
\step{5}{$r \in I$}
\step{6}{$r = 0$}
\step{7}{$g \in (f)$}
\qed
\end{proof}

\begin{prop}
Let $R$ be a commutative ring and $I$ an ideal in $R$. Then $I$ is maximal iff $R/I$ is a field.
\end{prop}

\begin{proof}
\pf\ From Proposition \ref{prop:maximal-iff-quotient-simple}. \qed
\end{proof}

\begin{ex}
\label{ex:x-minus-a-maximal}
Let $R$ be a commutative ring and $a \in R$. Then $(x-a)$ is a maximal ideal in $R[x]$ iff $R$ is a field, since $R[x]/(x-a) \cong R$.
\end{ex}

\begin{ex}
The ideal $(2,x)$ is a maximal ideal in $\mathbb{Z}[x]$, since $\mathbb{Z}[x] / (2,x) \cong \mathbb{Z} / 2 \mathbb{Z}$.
\end{ex}

\begin{prop}
Every maximal ideal in a commutative ring is a prime ideal.
\end{prop}

\begin{proof}
\pf\ Since every field is an integral domain. \qed
\end{proof}

\begin{prop}
Let $R$ be a commutative ring and $I$ an ideal in $R$. If $I$ is a prime ideal and $R/I$ is finite then $I$ is a maximal ideal.
\end{prop}

\begin{proof}
\pf\ Since every finite integral domain is a field. \qed
\end{proof}

\begin{prop}
Let $R$ be a commutative ring and $I$ a proper ideal in $R$. Then $I$ is maximal iff, whenever $J$ is an ideal and $I \subseteq J$, then $I = J$ or $J = R$.
\end{prop}


\begin{ex}
The inverse image of a maximal ideal under a homomorphism is not necessarily maximal.

Let $i : \mathbb{Z}[x] \rightarrow \mathbb{Q}[x]$ be the inclusion. Then $(x)$ is maximal in $\mathbb{Q}[x]$ but its inverse image $(x)$ is not maximal in $\mathbb{Z}[x]$.
\end{ex}

\begin{df}[Maximal Spectrum]
Let $R$ be a commutative ring. The \emph{maximal spectrum} of $R$ is the set of all maximal ideals in $R$.
\end{df}

\begin{prop}
Let $K$ be a field. The Krull dimension of $K[x_1, \ldots, x_n]$ is $n$.
\end{prop}

%TODO

\begin{thm}[Hilbert's Nullstellensatz]
Let $K$ be a field and $L$ a subfield of $K$. If $K$ is an $L$-algebra of finite type, then $K$ is a finite $L$-algebra.
\end{thm}

%TODO

\begin{prop}
Let $K$ be a subfield of $L$. Then $L$ is a $K$-algebra under multiplication.
\end{prop}

\begin{proof}
\pf\ Easy. \qed
\end{proof}

\chapter{Algebraically Closed Fields}

\begin{df}[Algebraically Closed]
A field $K$ is \emph{algebraically closed} iff, for every $f \in K[x]$ that is not constant, there exists $r \in K$ such that $f(r) = 0$.
\end{df}

\begin{thm}
$\mathbb{C}$ is algebraically closed.
\end{thm}

%TODO

\begin{prop}
Let $K$ be an algebraically closed field. Let $I$ be an ideal in $K[x]$. Then $I$ is maximal if and only if $I = (x-c)$ for some $c \in K$.
\end{prop}

\begin{proof}
\pf
\step{1}{If $I$ is maximal then there exists $c \in K$ such that $I = (x-c)$.}
\begin{proof}
	\step{a}{\assume{$I$ is maximal.}}
	\step{b}{\pick\ $f$ monic of minimal degree such that $f \in I$.}
	\step{c}{$f$ is not constant.}
	\begin{proof}
		\pf\ Otherwise $f = 1$ and $I = K[x]$.
	\end{proof}
	\step{d}{\pick\ $c \in K$ such that $f(c) = 0$}
	\step{e}{$x-c \mid f$}
	\step{f}{$I \subseteq (x-c)$}
	\step{g}{$I = (x-c)$}
\end{proof}
\step{2}{For all $c \in K$ we have $(x-c)$ is maximal.}
\begin{proof}
	\pf\ Example \ref{ex:x-minus-a-maximal}.
\end{proof}
\qed
\end{proof}

\part{Linear Algebra}

\chapter{Vector Spaces}

\begin{df}[Vector Space]
Let $K$ be a field. A \emph{$K$-vector space} is a $K$-module. A \emph{linear map} is a homomorphism of $K$-modules. We write $K-\mathbf{Vect}$ for $\Mod{K}$.
\end{df}

\begin{df}
    Let $\mathrm{GL}_n(\mathbb{R})$ be the group of invertible $n \times n$ real matrices.
    
    $\mathrm{GL}_n(\mathbb{R})$ acts on $\mathbb{R}^n$ by matrix multiplication.
\end{df}

\begin{df}
    Let $\mathrm{GL}_n(\mathbb{C})$ be the group of invertible $n \times n$ complex matrices.
    
    $\mathrm{GL}_n(\mathbb{C})$ acts on $\mathbb{C}^n$ by matrix multiplication.
\end{df}

\begin{df}
Let $\mathrm{SL}_n(\mathbb{R}) = \{ M \in \mathrm{GL}_n(\mathbb{R}) : \det M = 1 \}$.
\end{df}

\begin{prop}
$\mathrm{SL}_n(\mathbb{R})$ is a normal subgroup of $\mathrm{GL}_n(\mathbb{R})$.
\end{prop}

\begin{proof}
\pf\ If $\det M = 1$ then $\det(AMA^{-1}) = (\det A) (\det M) (\det A)^{-1} = 1$. \qed
\end{proof}

\begin{prop}
\[ \mathrm{GL}_n(\mathbb{R}) / \mathrm{SL}_n(\mathbb{R}) \cong \mathbb{R}^* \]
\end{prop}

\begin{proof}
\end{proof}

\begin{df}
Let $\mathrm{SL}_n(\mathbb{C}) = \{ M \in \mathrm{GL}_n(\mathbb{C}) : \det M = 1 \}$.
\end{df}

\begin{df}
Let $\mathrm{O}_n(\mathbb{R}) = \{ M \in \mathrm{GL}_n(\mathbb{R}) : M M^\mathrm{T} = M^\mathrm{T} M = I_n \}$.
\end{df}

\begin{prop}
The action of $\mathrm{O}_n(\mathbb{R})$ on $\mathbb{R}^n$ preserves lengths and angles.
\end{prop}

%TODO

\begin{df}
Let $\mathrm{SO}_n(\mathbb{R}) = \{ M \in \mathrm{O}_n(\mathbb{R}) : \det M = 1 \}$.
\end{df}

\begin{df}
Let $\mathrm{U}_n(\mathbb{C}) = \{ M \in \mathrm{GL}_n(\mathbb{C}) : M M^\dagger = M^\dagger M = I_n \}$.
\end{df}

\begin{df}
Let $\mathrm{SU}_n(\mathbb{C}) = \{ M \in U_n(\mathbb{C}) : \det M = 1 \}$.
\end{df}

\begin{prop}
Every matrix in $\mathrm{SU}_2(\mathbb{C})$ can be written in the form
\[ \left( \begin{array}{cc}
a + b i & c + d i \\
-c + di & a - bi
\end{array} \right) \]
for some $a,b,c,d \in \mathbb{R}$ with $a^2 + b^2 + c^2 + d^2 = 1$.
\end{prop}

\begin{proof}
\pf
\step{1}{\pflet{$M = \left( \begin{array}{cc}
\alpha & \beta \\
\gamma & \delta
\end{array} \right) \in \mathrm{SU}_2(\mathbb{C})$}}
\step{2}{$\inv{M} = M^\dagger$}
\step{3}{$\left( \begin{array}{cc}
\delta & - \beta \\
- \gamma & \alpha
\end{array} \right) = \left( \begin{array}{cc}
\overline{\alpha} & \overline{\gamma} \\
\overline{\beta} & \overline{\delta}
\end{array} \right)$}
\step{4}{\pflet{$\alpha = a + bi$ and $\beta = c+di$.}}
\step{5}{$\delta = \overline{\alpha} = a-bi$}
\step{6}{$\gamma = - \overline{\beta} = -c+di$}
\step{7}{$\det M = a^2 + b^2 + c^2 + d^2 = 1$}
\qed
\end{proof}

\begin{cor}
$\mathrm{SU}_2(\mathbb{C})$ is simply connected.
\end{cor}

%TODO

\begin{cor}
\[ \mathrm{SO}_3(\mathbb{R}) \cong \mathrm{SU}_2(\mathbb{C}) / \{ I, -I \} \]
\end{cor}

\begin{proof}
\pf\ The function that maps $\left( \begin{array}{cc}
a + b i & c + d i \\
-c + di & a - bi
\end{array} \right)$ to $\left( \begin{array}{ccc}
a^2 + b^2 - c^2 - d^2 & 2(bc - ad) & 2(ac + bd) \\
2(ad + bc) & a^2 - b^2 + c^2 - d^2 & 2(cd - ab) \\
2(bd - ac) & 2(ab + cd) & a^2 - b^2 - c^2 + d^2
\end{array} \right)$ is a surjective homomorphism with kernel $\{ I, -I \}$. \qed
\end{proof}

\begin{cor}
The fundamental group of $\mathrm{SO}_3(\mathbb{R})$ is $C_2$.
\end{cor}

%TODO
